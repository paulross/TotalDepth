%% Generated by Sphinx.
\def\sphinxdocclass{report}
\documentclass[letterpaper,10pt,english]{sphinxmanual}
\ifdefined\pdfpxdimen
   \let\sphinxpxdimen\pdfpxdimen\else\newdimen\sphinxpxdimen
\fi \sphinxpxdimen=.75bp\relax

\usepackage[utf8]{inputenc}
\ifdefined\DeclareUnicodeCharacter
 \ifdefined\DeclareUnicodeCharacterAsOptional
  \DeclareUnicodeCharacter{"00A0}{\nobreakspace}
  \DeclareUnicodeCharacter{"2500}{\sphinxunichar{2500}}
  \DeclareUnicodeCharacter{"2502}{\sphinxunichar{2502}}
  \DeclareUnicodeCharacter{"2514}{\sphinxunichar{2514}}
  \DeclareUnicodeCharacter{"251C}{\sphinxunichar{251C}}
  \DeclareUnicodeCharacter{"2572}{\textbackslash}
 \else
  \DeclareUnicodeCharacter{00A0}{\nobreakspace}
  \DeclareUnicodeCharacter{2500}{\sphinxunichar{2500}}
  \DeclareUnicodeCharacter{2502}{\sphinxunichar{2502}}
  \DeclareUnicodeCharacter{2514}{\sphinxunichar{2514}}
  \DeclareUnicodeCharacter{251C}{\sphinxunichar{251C}}
  \DeclareUnicodeCharacter{2572}{\textbackslash}
 \fi
\fi
\usepackage{cmap}
\usepackage[T1]{fontenc}
\usepackage{amsmath,amssymb,amstext}
\usepackage{babel}
\usepackage{times}
\usepackage[Bjarne]{fncychap}
\usepackage[dontkeepoldnames]{sphinx}

\usepackage{geometry}

% Include hyperref last.
\usepackage{hyperref}
% Fix anchor placement for figures with captions.
\usepackage{hypcap}% it must be loaded after hyperref.
% Set up styles of URL: it should be placed after hyperref.
\urlstyle{same}

\addto\captionsenglish{\renewcommand{\figurename}{Fig.}}
\addto\captionsenglish{\renewcommand{\tablename}{Table}}
\addto\captionsenglish{\renewcommand{\literalblockname}{Listing}}

\addto\captionsenglish{\renewcommand{\literalblockcontinuedname}{continued from previous page}}
\addto\captionsenglish{\renewcommand{\literalblockcontinuesname}{continues on next page}}

\addto\extrasenglish{\def\pageautorefname{page}}

\setcounter{tocdepth}{1}



\title{TotalDepth Documentation}
\date{Sep 26, 2017}
\release{0.2.0}
\author{Paul Ross}
\newcommand{\sphinxlogo}{\vbox{}}
\renewcommand{\releasename}{Release}
\makeindex

\begin{document}

\maketitle
\sphinxtableofcontents
\phantomsection\label{\detokenize{index::doc}}


TotalDepth is an Open Source, cross platform, software collection that processes petrophysical data from the oil field such as wireline logs, seismic data and so on.

If you are new here then have a look at some {\hyperref[\detokenize{examples/examples:totaldepth-examples}]{\sphinxcrossref{\DUrole{std,std-ref}{TotalDepth Example Outputs}}}}. For more detail see {\hyperref[\detokenize{intro:totaldepth-intro}]{\sphinxcrossref{\DUrole{std,std-ref}{Introduction to TotalDepth}}}}.

The \sphinxhref{https://github.com/paulross/TotalDepth}{TotalDepth project} is currently at \sphinxstylestrong{Alpha} version 0.2.0, release 0.2.0. For the licence see {\hyperref[\detokenize{licence:licence-text}]{\sphinxcrossref{\DUrole{std,std-ref}{TotalDepth Licence}}}}.


\chapter{Contents}
\label{\detokenize{index:contents}}\label{\detokenize{index:totaldepth}}

\section{Introduction to TotalDepth}
\label{\detokenize{intro:totaldepth-intro}}\label{\detokenize{intro::doc}}\label{\detokenize{intro:introduction-to-totaldepth}}
TotalDepth is an Open Source, cross platform, software collection that can process petrophysical data from the oil field such as wireline logs, seismic data and so on.

Conventional, proprietary, software for petrophysical data tends to be expensive to licence, restrictive, slow to develop for and tied to expensive hardware. TotalDepth changes all of that.

TotalDepth is open and cross-platform, and produces results straight to the bowser. TotalDepth supports such technologies such as HTML5, AJAX, Software as a Service (SaaS) and Cloud Computing.


\subsection{Software Technology}
\label{\detokenize{intro:software-technology}}
For rapid development and performance TotalDepth is written in \sphinxhref{http://www.python.org}{Python} with performance critical code written in C/C++ or \sphinxhref{http://www.cython.org}{Cython}.

TotalDepth was developed using \sphinxstyleabbreviation{TDD} (Test Driven Development) in an agile fashion using \sphinxstyleemphasis{nano-iterations}.


\section{TotalDepth Example Outputs}
\label{\detokenize{examples/examples:totaldepth-example-outputs}}\label{\detokenize{examples/examples::doc}}\label{\detokenize{examples/examples:totaldepth-examples}}
This shows some examples of the kind of thing that TotalDepth can do.


\subsection{Wireline Plots}
\label{\detokenize{examples/examples:wireline-plots}}\label{\detokenize{examples/examples:totaldepth-examples-plots}}
TotalDepth produced these time honoured plots from LIS and LAS wireline logs in SVG format that can be viewed in most browsers %
\begin{footnote}[1]\sphinxAtStartFootnote
There is good SVG support among current browsers such as \sphinxhref{http://www.opera.com}{Opera},  \sphinxhref{https://www.google.com/chrome/}{Chrome} and  \sphinxhref{https://www.apple.com/safari/}{Safari}. You can find a comparison of browser support for SVG at \sphinxhref{https://en.wikipedia.org/wiki/Comparison\_of\_layout\_engines\_\%28Scalable\_Vector\_Graphics\%29}{Wikipedia}.
%
\end{footnote}.


\subsubsection{Plotting from LIS}
\label{\detokenize{examples/examples:plotting-from-lis}}
Some examples of plots generated from LIS79 files:
\begin{itemize}
\item {} 
A \sphinxhref{../\_static/plots\_LIS/index.html}{collection} of 9 LIS files where TotalDepth used their internal plot specifications to generate 22 separate plots.

\item {} 
A \sphinxhref{../\_static/plots/HDT\_Example.svg}{High Resolution Dipmeter} plot on a scale of 1:25 with an API header. Fast channel FC0 (red) overlaid on FC1.

\end{itemize}


\subsubsection{Plotting From LAS}
\label{\detokenize{examples/examples:plotting-from-las}}
This shows off some examples of plots generated from the Canadian Well Logging Society’s LAS formated files %
\begin{footnote}[2]\sphinxAtStartFootnote
Thanks to the \sphinxhref{http://www.kgs.ku.edu/Magellan/Logs/index.html}{University of Kansas} {[}kgs.ku.edu{]} for the original data. For these examples that data has been edited or truncated or both.
%
\end{footnote}.


\paragraph{Single LAS Plot examples}
\label{\detokenize{examples/examples:single-las-plot-examples}}
This shows plots of a single LAS file that has 200 feet of 15 curves. TotalDepth can plot this with, linear and log scales and with an API header:
\begin{itemize}
\item {} 
Plotted with the \sphinxhref{../\_static/plots/Resistivity\_3Track\_Logrithmic.xml\_47\_LAS.svg}{Resistivity 3Track Logrithmic} format.

\item {} 
The same data plotted with the \sphinxhref{../\_static/plots/Triple\_Combo\_46\_LAS.svg}{Triple Combo} format.

\end{itemize}

The original LAS file is \sphinxhref{../\_static/plots/1001178923.las.txt}{here}.


\paragraph{A Collection of LAS Plots}
\label{\detokenize{examples/examples:a-collection-of-las-plots}}
Here is a \sphinxhref{../\_static/plots\_LAS/index.html}{directory of six LAS files} that was used to make 31 individual plots complete with an index that summarises them. For each LAS file the plotting program automatically choose from 29 plot formats the formats that produce useful plots %
\begin{footnote}[3]\sphinxAtStartFootnote
A \sphinxstyleemphasis{useful} plot format is one that can handle at least \sphinxstyleemphasis{n} curves where \sphinxstyleemphasis{n} is a number that is specified by the user. If the user specifies 4 then there will be at least 4 curves on each plot.
%
\end{footnote}.


\paragraph{Making LAS Plots}
\label{\detokenize{examples/examples:making-las-plots}}
The \sphinxcode{PlotLogs.py} command line tool was used with the command:

\begin{sphinxVerbatim}[commandchars=\\\{\}]
\PYGZdl{} python3 PlotLogs.py \PYGZhy{}A \PYGZhy{}j4 \PYGZhy{}r \PYGZhy{}X 4 Data/ Plot/
\end{sphinxVerbatim}

This searched for LAS files in directory \sphinxcode{Data/} with the plots being written in directory \sphinxcode{Plot/}.

The following options have been set:
\begin{itemize}
\item {} 
API headers on the top of each plot: \sphinxcode{-A}

\item {} 
Multiprocessing on with 4 simultaneous jobs: \sphinxcode{-j4}

\item {} 
Recursive search of input directory: \sphinxcode{-r}

\item {} 
Uses any available plot specifications from LgFormat XML files which result in 4 curves or more being plotted: \sphinxcode{-X 4}

\end{itemize}

This took around six seconds to compute. More detail on the \sphinxcode{PlotLogs.py} is here: {\hyperref[\detokenize{cmd_line_tools/TD_cmd_line_tools:totaldepth-cmdline-plotlogs}]{\sphinxcrossref{\DUrole{std,std-ref}{tdplotlogs}}}}


\subsection{LIS Log HTML Summaries}
\label{\detokenize{examples/examples:lis-log-html-summaries}}
The program \sphinxcode{LisToHtml.py} takes LIS file(s) and generates an \sphinxhref{../\_static/LISExampleHTML/index.html}{HTML summary} for each one.


\subsubsection{How This HTML was Made}
\label{\detokenize{examples/examples:how-this-html-was-made}}
The \sphinxcode{LisToHtml.py} command line tool was used with the command:

\begin{sphinxVerbatim}[commandchars=\\\{\}]
\PYGZdl{} python3 LisToHtml.py \PYGZhy{}k \PYGZhy{}j4 \PYGZhy{}r Data/ HTML/
\end{sphinxVerbatim}

This searched for LAS files in directory \sphinxcode{Data/} with the files being written to directory \sphinxcode{HTML/}.

The following options have been set:
\begin{itemize}
\item {} 
Keep going as far as possible: \sphinxcode{-k}

\item {} 
Multiprocessing on with 4 simultaneous jobs: \sphinxcode{-j4}

\item {} 
Recursive search of input directory: \sphinxcode{-r}

\end{itemize}

More detail on the \sphinxcode{LisToHtml.py} is here: {\hyperref[\detokenize{cmd_line_tools/LIS_cmd_line_tools:totaldepth-liscmdline-listohtml}]{\sphinxcrossref{\DUrole{std,std-ref}{tdlistohtml}}}}


\section{Installation}
\label{\detokenize{installation::doc}}\label{\detokenize{installation:installation}}

\subsection{Stable release}
\label{\detokenize{installation:stable-release}}
To install TotalDepth, run this command in your terminal:

\begin{sphinxVerbatim}[commandchars=\\\{\}]
\PYG{g+gp}{\PYGZdl{}} pip install TotalDepth
\end{sphinxVerbatim}

This is the preferred method to install TotalDepth, as it will always install the most recent stable release.

If you don’t have \sphinxhref{https://pip.pypa.io}{pip} installed, this \sphinxhref{http://docs.python-guide.org/en/latest/starting/installation/}{Python installation guide} can guide
you through the process.


\subsection{From sources}
\label{\detokenize{installation:python-installation-guide}}\label{\detokenize{installation:from-sources}}
First make a virtual environment in your \sphinxcode{\sphinxstyleemphasis{\textless{}PYTHONVENVS\textgreater{}}}, say \sphinxcode{\sphinxstyleemphasis{\textasciitilde{}/pyvenvs}}:

\begin{sphinxVerbatim}[commandchars=\\\{\}]
\PYG{g+gp}{\PYGZdl{}} python3 \PYGZhy{}m venv \PYGZlt{}PYTHONVENVS\PYGZgt{}/TotalDepth
\PYG{g+gp}{\PYGZdl{}} . \PYGZlt{}PYTHONVENVS\PYGZgt{}/TotalDepth/bin/activate
\PYG{g+gp}{(TotalDepth) \PYGZdl{}}
\end{sphinxVerbatim}

Install the dependencies, \sphinxcode{numpy} and \sphinxcode{Cython}:

\begin{sphinxVerbatim}[commandchars=\\\{\}]
\PYG{g+gp}{(TotalDepth) \PYGZdl{}} pip install numpy
\PYG{g+gp}{(TotalDepth) \PYGZdl{}} pip install Cython
\end{sphinxVerbatim}

The sources for TotalDepth can be downloaded from the \sphinxhref{https://github.com/paulross/TotalDepth}{Github repo}.

You can either clone the public repository:

\begin{sphinxVerbatim}[commandchars=\\\{\}]
\PYG{g+gp}{(TotalDepth) \PYGZdl{}} git clone git://github.com/paulross/TotalDepth.git
\end{sphinxVerbatim}

Or download the \sphinxhref{https://github.com/paulross/TotalDepth/tarball/master}{tarball}:

\begin{sphinxVerbatim}[commandchars=\\\{\}]
\PYG{g+gp}{(TotalDepth) \PYGZdl{}} curl  \PYGZhy{}OL https://github.com/paulross/TotalDepth/tarball/master
\end{sphinxVerbatim}

Once you have a copy of the source, you can install it with:

\begin{sphinxVerbatim}[commandchars=\\\{\}]
\PYG{g+go}{    (TotalDepth) \PYGZdl{} cd TotalDepth}
\PYG{g+gp}{(TotalDepth) \PYGZdl{}} python setup.py install
\end{sphinxVerbatim}

Install the test dependencies and run TotalDepth’s tests:

\begin{sphinxVerbatim}[commandchars=\\\{\}]
\PYG{g+gp}{(TotalDepth) \PYGZdl{}} pip install pytest
\PYG{g+gp}{(TotalDepth) \PYGZdl{}} pip install pytest\PYGZhy{}runner
\PYG{g+gp}{(TotalDepth) \PYGZdl{}} python setup.py \PYG{n+nb}{test}
\end{sphinxVerbatim}


\subsection{Developing with TotalDepth}
\label{\detokenize{installation:developing-with-totaldepth}}
If you are developing with TotalDepth you need test coverage and documentation tools.


\subsubsection{Test Coverage}
\label{\detokenize{installation:test-coverage}}
Install \sphinxcode{pytest-cov}:

\begin{sphinxVerbatim}[commandchars=\\\{\}]
\PYG{g+gp}{(TotalDepth) \PYGZdl{}} pip install pytest\PYGZhy{}cov
\end{sphinxVerbatim}

The most meaningful invocation that elimates the top level tools is:

\begin{sphinxVerbatim}[commandchars=\\\{\}]
\PYG{g+gp}{(TotalDepth) \PYGZdl{}} pytest \PYGZhy{}\PYGZhy{}cov\PYG{o}{=}TotalDepth.LAS.core \PYGZhy{}\PYGZhy{}cov\PYG{o}{=}TotalDepth.LIS.core \PYGZhy{}\PYGZhy{}cov\PYG{o}{=}TotalDepth.RP66.core \PYGZhy{}\PYGZhy{}cov\PYG{o}{=}TotalDepth.util \PYGZhy{}\PYGZhy{}cov\PYGZhy{}report html tests/
\end{sphinxVerbatim}


\subsubsection{Documentation}
\label{\detokenize{installation:documentation}}
If you want to build the documentation you need to:

\begin{sphinxVerbatim}[commandchars=\\\{\}]
\PYG{g+gp}{(TotalDepth) \PYGZdl{}} pip install Sphinx
\PYG{g+gp}{(TotalDepth) \PYGZdl{}} \PYG{n+nb}{cd} docs
\PYG{g+gp}{(TotalDepth) \PYGZdl{}} make html
\end{sphinxVerbatim}


\subsection{System Testing}
\label{\detokenize{installation:system-testing}}
See {\hyperref[\detokenize{testing/test_plot::doc}]{\sphinxcrossref{\DUrole{doc}{Testing the Plot Package}}}} for comprehensive testing of your installation to see if LIS/LAS files can be written, read and plotted. This pretty much executes all TotalDepth code.


\subsection{Unit Testing}
\label{\detokenize{installation:unit-testing}}
See {\hyperref[\detokenize{testing/unit_tests::doc}]{\sphinxcrossref{\DUrole{doc}{Unit Tests}}}} for more information about testing and unit tests.


\section{Petrophysical File Formats that TotalDepth can Process}
\label{\detokenize{file_formats:petrophysical-file-formats-that-totaldepth-can-process}}\label{\detokenize{file_formats::doc}}\label{\detokenize{file_formats:tarball}}
TotalDepth is currently at \sphinxstylestrong{Alpha}, release version 0.1.0.


\begin{savenotes}\sphinxattablestart
\centering
\begin{tabulary}{\linewidth}[t]{|T|T|T|T|}
\hline
\sphinxstylethead{\sphinxstyletheadfamily 
Format
\unskip}\relax &\sphinxstylethead{\sphinxstyletheadfamily 
Support
\unskip}\relax &\sphinxstylethead{\sphinxstyletheadfamily 
Since
\unskip}\relax &\sphinxstylethead{\sphinxstyletheadfamily 
Notes.
\unskip}\relax \\
\hline
LIS79
&
Full
&
0.1.0
&\\
\hline
LIS-A
&
Partial
&
0.1.0
&
No direct XNAM support.
\\
\hline
LAS version 1.2
&
Full
&
0.1.0
&\\
\hline
LAS version 2.0
&
Full
&
0.1.0
&\\
\hline
LAS version 3.0
&
None
&&\\
\hline
WellLogML
&
None
&&
Absence of good test cases.
The format seems to have made little impact on the industry.
\\
\hline
RP66v1 “DLIS”
&
Proto.
&&
Absence of good test cases.
\\
\hline
RP66v2 “DLIS”
&
Proto.
&&
Absence of good test cases.
\\
\hline
SEG-Y etc.
&
None
&&
Use cases?
\\
\hline
Positioning
&
None
&&
Fairly simple, use cases?
\\
\hline
\end{tabulary}
\par
\sphinxattableend\end{savenotes}


\section{Command Line Tools}
\label{\detokenize{cmd_line_tools/index_cmd_line_tools::doc}}\label{\detokenize{cmd_line_tools/index_cmd_line_tools:command-line-tools}}
TotalDepth provides a number of tools run from the command line that can analyse and visualise petrophysical data.
All the command line tools start with \sphinxcode{td}.


\subsection{TotalDepth Command Line Tools}
\label{\detokenize{cmd_line_tools/TD_cmd_line_tools:totaldepth-command-line-tools}}\label{\detokenize{cmd_line_tools/TD_cmd_line_tools::doc}}
This describes the command line tools that are available for processing any TotalDepth file.

The tools are located in \sphinxcode{TotalDepth/}


\subsubsection{Plotting Well Logs}
\label{\detokenize{cmd_line_tools/TD_cmd_line_tools:plotting-well-logs}}
These command line tools plot wireline data.


\paragraph{tdplotlogs}
\label{\detokenize{cmd_line_tools/TD_cmd_line_tools:tdplotlogs}}\label{\detokenize{cmd_line_tools/TD_cmd_line_tools:totaldepth-cmdline-plotlogs}}
Produces SVG plots from LIS and LAS files.


\subparagraph{Usage}
\label{\detokenize{cmd_line_tools/TD_cmd_line_tools:usage}}
Usage:

\begin{sphinxVerbatim}[commandchars=\\\{\}]
\PYG{n}{usage}\PYG{p}{:} \PYG{n}{tdplotlogs} \PYG{p}{[}\PYG{o}{\PYGZhy{}}\PYG{n}{h}\PYG{p}{]} \PYG{p}{[}\PYG{o}{\PYGZhy{}}\PYG{o}{\PYGZhy{}}\PYG{n}{version}\PYG{p}{]} \PYG{p}{[}\PYG{o}{\PYGZhy{}}\PYG{n}{j} \PYG{n}{JOBS}\PYG{p}{]} \PYG{p}{[}\PYG{o}{\PYGZhy{}}\PYG{n}{k}\PYG{p}{]} \PYG{p}{[}\PYG{o}{\PYGZhy{}}\PYG{n}{l} \PYG{n}{LOGLEVEL}\PYG{p}{]} \PYG{p}{[}\PYG{o}{\PYGZhy{}}\PYG{n}{g}\PYG{p}{]} \PYG{p}{[}\PYG{o}{\PYGZhy{}}\PYG{n}{r}\PYG{p}{]}
                   \PYG{p}{[}\PYG{o}{\PYGZhy{}}\PYG{n}{A}\PYG{p}{]} \PYG{p}{[}\PYG{o}{\PYGZhy{}}\PYG{n}{x} \PYG{n}{LGFORMAT}\PYG{p}{]} \PYG{p}{[}\PYG{o}{\PYGZhy{}}\PYG{n}{X} \PYG{n}{LGFORMAT\PYGZus{}MIN}\PYG{p}{]} \PYG{p}{[}\PYG{o}{\PYGZhy{}}\PYG{n}{s} \PYG{n}{SCALE}\PYG{p}{]}
                   \PYG{o+ow}{in} \PYG{n}{out}
\end{sphinxVerbatim}


\subparagraph{Arguments}
\label{\detokenize{cmd_line_tools/TD_cmd_line_tools:arguments}}
These are required arguments unless \sphinxcode{-h} or \sphinxcode{-{-}version} options are specified (in which case no processing is done):
\begin{enumerate}
\item {} 
The path to the input LAS or LIS file or directory thereof.

\item {} 
The path to the output SVG file or directory, any directories will be created as necessary.

\end{enumerate}


\subparagraph{Options}
\label{\detokenize{cmd_line_tools/TD_cmd_line_tools:options}}

\begin{savenotes}\sphinxattablestart
\centering
\begin{tabulary}{\linewidth}[t]{|T|T|}
\hline
\sphinxstylethead{\sphinxstyletheadfamily 
Option
\unskip}\relax &\sphinxstylethead{\sphinxstyletheadfamily 
Description
\unskip}\relax \\
\hline
\sphinxcode{-{-}version}
&
Show program’s version number and exit
\\
\hline
\sphinxcode{-h, -{-}help}
&
Show this help message and exit.
\\
\hline
\sphinxcode{-j JOBS, -{-}jobs=JOBS}
&
Max processes when multiprocessing. Zero uses number of native CPUs {[}8{]}.
-1 disables multiprocessing. {[}default: -1{]}
\\
\hline
\sphinxcode{-k, -{-}keep-going}
&
Keep going as far as sensible. {[}default: False{]}
\\
\hline
\sphinxcode{-l LOGLEVEL, -{-}loglevel=LOGLEVEL}
&
Log Level (debug=10, info=20, warning=30, error=40, critical=50) {[}default: 20{]}
\\
\hline
\sphinxcode{-g, -{-}glob}
&
File pattern match. {[}default none{]}
\\
\hline
\sphinxcode{-r, -{-}recursive}
&
Process input recursively. {[}default: False{]}
\\
\hline
\sphinxcode{-A, -{-}API}
&
Put an API header on each plot. {[}default: False{]}
\\
\hline
\sphinxcode{-x LGFORMAT, -{-}xml LGFORMAT}
&
Use XML LgFormat UniqueId to use for plotting (additive). Use -x? to see what
LgFormats (UniqueID+Description) are available.
Use -x?? to see what curves each format can plot.
See also -X.
This is additive so can used multiple times to get multiple plots from the
same data.
\\
\hline
\sphinxcode{-X LGFORMAT\_MIN,}
\sphinxcode{-{-}XML LGFORMAT\_MIN}
&
Use all available LgFormat XML plots that use LGFORMAT\_MIN or more outputs.
If -x option present limited by those LgFormats {[}default: 0{]}
\\
\hline
\sphinxcode{-s SCALE, -{-}scale SCALE}
&
Scale of X axis to use (an integer). This overrides the scale(s)
specified in the LgFormat file or FILM table. {[}default: 0{]}.
\\
\hline
\end{tabulary}
\par
\sphinxattableend\end{savenotes}


\subparagraph{Examples}
\label{\detokenize{cmd_line_tools/TD_cmd_line_tools:examples}}

\subparagraph{LgFormat XML}
\label{\detokenize{cmd_line_tools/TD_cmd_line_tools:lgformat-xml}}
Using \sphinxcode{-x?} to see what formats are available:

\begin{sphinxVerbatim}[commandchars=\\\{\}]
\PYGZdl{} python3 tdplotlogs \PYGZhy{}x? spam eggs
\end{sphinxVerbatim}

The output is something like:

\begin{sphinxVerbatim}[commandchars=\\\{\}]
Cmd: tdplotlogs \PYGZhy{}x? spam eggs
XML LgFormats available: [29]
UniqueId                              Description
\PYGZhy{}\PYGZhy{}\PYGZhy{}\PYGZhy{}\PYGZhy{}\PYGZhy{}\PYGZhy{}\PYGZhy{}\PYGZhy{}\PYGZhy{}\PYGZhy{}\PYGZhy{}\PYGZhy{}\PYGZhy{}\PYGZhy{}\PYGZhy{}\PYGZhy{}\PYGZhy{}\PYGZhy{}\PYGZhy{}\PYGZhy{}\PYGZhy{}\PYGZhy{}\PYGZhy{}\PYGZhy{}\PYGZhy{}\PYGZhy{}\PYGZhy{}\PYGZhy{}\PYGZhy{}\PYGZhy{}\PYGZhy{}\PYGZhy{}\PYGZhy{}\PYGZhy{}   \PYGZhy{}\PYGZhy{}\PYGZhy{}\PYGZhy{}\PYGZhy{}\PYGZhy{}\PYGZhy{}\PYGZhy{}\PYGZhy{}\PYGZhy{}\PYGZhy{}\PYGZhy{}\PYGZhy{}\PYGZhy{}\PYGZhy{}\PYGZhy{}\PYGZhy{}\PYGZhy{}\PYGZhy{}\PYGZhy{}\PYGZhy{}\PYGZhy{}\PYGZhy{}\PYGZhy{}\PYGZhy{}\PYGZhy{}\PYGZhy{}\PYGZhy{}\PYGZhy{}\PYGZhy{}\PYGZhy{}\PYGZhy{}
ADN\PYGZus{}Image\PYGZus{}Format                    : ADN Image Log
Azimuthal\PYGZus{}Density\PYGZus{}3Track.xml        : Azimuthal Density 3Track
Azimuthal\PYGZus{}Resistivity\PYGZus{}3Track.xml    : Azimuthal Resistivity 3Track
Blank\PYGZus{}3Track\PYGZus{}Depth                  : Blank 3Track
Blank\PYGZus{}3Track\PYGZus{}Time.xml               : Blank 3Track Time
FMI\PYGZus{}IMAGE\PYGZus{}ALIGNED                   : FMI Image Aligned
FMI\PYGZus{}IMAGE\PYGZus{}PROCESSED                 : FMI Image Processed
Formation\PYGZus{}Test                      : Formation Test Time
HDT                                 : High Definition Dipmeter
Micro\PYGZus{}Resistivity\PYGZus{}3Track.xml        : Micro Resistivity 3 Track Format
Natural\PYGZus{}GR\PYGZus{}Spectrometry\PYGZus{}3Track.xml  : Natural GR Spectrometry 3Track
OBMI\PYGZus{}IMAGE\PYGZus{}EQUAL                    : OBMI Image Equalized
Porosity\PYGZus{}GR\PYGZus{}3Track                  : Standard Porosity Curves
Pulsed\PYGZus{}Neutron\PYGZus{}3Track.xml           : Pulsed Neutron 3Track
Pulsed\PYGZus{}Neutron\PYGZus{}Time.xml             : Pulsed Neutron Time
RAB\PYGZus{}Image\PYGZus{}Format\PYGZus{}Deep               : Resistivity At the Bit Image
RAB\PYGZus{}Image\PYGZus{}Format\PYGZus{}Medium             : Resistivity At the Bit Image
RAB\PYGZus{}Image\PYGZus{}Format\PYGZus{}Shallow            : Resistivity At the Bit Image
RAB\PYGZus{}Std\PYGZus{}Format                      : Resistivity At the Bit
Resistivity\PYGZus{}3Track\PYGZus{}Correlation.xml  : Resistivity Linear Correlation Format
Resistivity\PYGZus{}3Track\PYGZus{}Logrithmic.xml   : Logrithmic Resistivity 3Track
Resistivity\PYGZus{}Investigation\PYGZus{}Image.xml : AIT Radial Investigation Image
Sonic\PYGZus{}3Track.xml                    : Sonic DT Porosity 3 Track
Sonic\PYGZus{}PWF4                          : SONIC Packed Waveform 4
Sonic\PYGZus{}SPR1\PYGZus{}VDL                      : SONIC Receiver Array Lower Dipole VDL
Sonic\PYGZus{}SPR2\PYGZus{}VDL                      : SONIC Receiver Array Upper Dipole VDL
Sonic\PYGZus{}SPR3\PYGZus{}VDL                      : SONIC Receiver Array Stonely VDL
Sonic\PYGZus{}SPR4\PYGZus{}VDL                      : SONIC Receiver Array P and S VDL
Triple\PYGZus{}Combo                        : Resistivity Density Neutron GR 3Track Format
\end{sphinxVerbatim}

The first column is the UniqueID to be used in identifying plots for the \sphinxcode{-x} option.

Using \sphinxcode{-x??} to see what formats and what curves would be plotted by each plot specification:

\begin{sphinxVerbatim}[commandchars=\\\{\}]
\PYGZdl{} python3 tdplotlogs \PYGZhy{}x?? a b
\end{sphinxVerbatim}

The output is something like:

\begin{sphinxVerbatim}[commandchars=\\\{\}]
Cmd: tdplotlogs \PYGZhy{}x?? a b
XML LgFormats available: [29]
UniqueId                              Description
\PYGZhy{}\PYGZhy{}\PYGZhy{}\PYGZhy{}\PYGZhy{}\PYGZhy{}\PYGZhy{}\PYGZhy{}\PYGZhy{}\PYGZhy{}\PYGZhy{}\PYGZhy{}\PYGZhy{}\PYGZhy{}\PYGZhy{}\PYGZhy{}\PYGZhy{}\PYGZhy{}\PYGZhy{}\PYGZhy{}\PYGZhy{}\PYGZhy{}\PYGZhy{}\PYGZhy{}\PYGZhy{}\PYGZhy{}\PYGZhy{}\PYGZhy{}\PYGZhy{}\PYGZhy{}\PYGZhy{}\PYGZhy{}\PYGZhy{}\PYGZhy{}\PYGZhy{}   \PYGZhy{}\PYGZhy{}\PYGZhy{}\PYGZhy{}\PYGZhy{}\PYGZhy{}\PYGZhy{}\PYGZhy{}\PYGZhy{}\PYGZhy{}\PYGZhy{}\PYGZhy{}\PYGZhy{}\PYGZhy{}\PYGZhy{}\PYGZhy{}\PYGZhy{}\PYGZhy{}\PYGZhy{}\PYGZhy{}\PYGZhy{}\PYGZhy{}\PYGZhy{}\PYGZhy{}\PYGZhy{}\PYGZhy{}\PYGZhy{}\PYGZhy{}\PYGZhy{}\PYGZhy{}\PYGZhy{}\PYGZhy{}
ADN\PYGZus{}Image\PYGZus{}Format                    : ADN Image Log
    DRHB, GR  , GR\PYGZus{}RAB, ROBB, ROP5, TNPH
Azimuthal\PYGZus{}Density\PYGZus{}3Track.xml        : Azimuthal Density 3Track
    BS  , DCAL, DRHB, DRHL, DRHO, DRHR, DRHU, DTAB, HDIA, PEB , PEF , PEL
    PER , PEU , RHOB, ROBB, ROBL, ROBR, ROBU, ROP5, RPM , SCN2, SOAB, SOAL
    SOAR, SOAU, SONB, SOXB, VDIA
Azimuthal\PYGZus{}Resistivity\PYGZus{}3Track.xml    : Azimuthal Resistivity 3Track
    AAI , BS  , C1  , C2  , CALI, GR  , GRDN\PYGZus{}RAB, GRLT\PYGZus{}RAB, GRRT\PYGZus{}RAB, GRUP\PYGZus{}RAB, PCAL, RDBD
    RDBL, RDBR, RDBU, RLA0, RLA1, RLA2, RLA3, RLA4, RLA5, RMBD, RMBL, RMBR
    RMBU, ROP5, RPM , RSBD, RSBL, RSBR, RSBU, SP  , TENS
Blank\PYGZus{}3Track\PYGZus{}Depth                  : Blank 3Track
Blank\PYGZus{}3Track\PYGZus{}Time.xml               : Blank 3Track Time
FMI\PYGZus{}IMAGE\PYGZus{}ALIGNED                   : FMI Image Aligned
    C1  , C2  , GR  , HAZIM, P1AZ, SP  , TENS
FMI\PYGZus{}IMAGE\PYGZus{}PROCESSED                 : FMI Image Processed
    C1  , C2  , GR  , HAZIM, P1AZ, SP  , TENS
Formation\PYGZus{}Test                      : Formation Test Time
    B1TR, BFR1, BQP1, BQP1, BQP1, BQP1, BSG1, POHP
HDT                                 : High Definition Dipmeter
    C1  , C2  , DEVI, FC0 , FC1 , FC2 , FC3 , FC4 , GR  , HAZI, P1AZ, RB
Micro\PYGZus{}Resistivity\PYGZus{}3Track.xml        : Micro Resistivity 3 Track Format
    BMIN, BMNO, BS  , CALI, GR  , HCAL, HMIN, HMNO, MINV, MLL , MNOR, MSFL
    PROX, RXO , SP  , TENS
Natural\PYGZus{}GR\PYGZus{}Spectrometry\PYGZus{}3Track.xml  : Natural GR Spectrometry 3Track
    CGR , PCAL, POTA, ROP5, SGR , SIGM, TENS, THOR, URAN
OBMI\PYGZus{}IMAGE\PYGZus{}EQUAL                    : OBMI Image Equalized
    C1  , C1\PYGZus{}OBMT, C2  , C2\PYGZus{}OBMT, GR  , HAZIM, OBRA3, OBRB3, OBRC3, OBRD3, P1AZ, P1NO\PYGZus{}OBMT
    TENS
Porosity\PYGZus{}GR\PYGZus{}3Track                  : Standard Porosity Curves
    APDC, APLC, APSC, BS  , C1  , C2  , CALI, CALI\PYGZus{}CDN, CMFF, CMRP, DPHB, DPHI
    DPHZ, DPOR\PYGZus{}CDN, DRHO, ENPH, GR  , HCAL, NPHI, NPOR, PCAL, RHOB, RHOZ, ROP5
    SNP , SP  , SPHI, TENS, TNPB, TNPH, TNPH\PYGZus{}CDN, TPHI
Pulsed\PYGZus{}Neutron\PYGZus{}3Track.xml           : Pulsed Neutron 3Track
    FBAC, GR  , INFD, SIGM, TAU , TCAF, TENS, TPHI, TSCF, TSCN
Pulsed\PYGZus{}Neutron\PYGZus{}Time.xml             : Pulsed Neutron Time
    FBAC\PYGZus{}SL, GR\PYGZus{}SL, INFD\PYGZus{}SL, SIGM\PYGZus{}SL, TAU\PYGZus{}SL, TCAF\PYGZus{}SL, TENS\PYGZus{}SL, TPHI\PYGZus{}SL, TSCF\PYGZus{}SL, TSCN\PYGZus{}SL
RAB\PYGZus{}Image\PYGZus{}Format\PYGZus{}Deep               : Resistivity At the Bit Image
    GR\PYGZus{}RAB, RES\PYGZus{}BD, RES\PYGZus{}BM, RES\PYGZus{}BS, RES\PYGZus{}RING, ROP5
RAB\PYGZus{}Image\PYGZus{}Format\PYGZus{}Medium             : Resistivity At the Bit Image
    GR\PYGZus{}RAB, RES\PYGZus{}BD, RES\PYGZus{}BM, RES\PYGZus{}BS, RES\PYGZus{}RING, ROP5
RAB\PYGZus{}Image\PYGZus{}Format\PYGZus{}Shallow            : Resistivity At the Bit Image
    GR\PYGZus{}RAB, RES\PYGZus{}BD, RES\PYGZus{}BM, RES\PYGZus{}BS, RES\PYGZus{}RING, ROP5
RAB\PYGZus{}Std\PYGZus{}Format                      : Resistivity At the Bit
    AAI , BDAV, BDM3, BMAV, BMM2, BSAV, BSM1, BTAB, CALI, DEVI, GR\PYGZus{}RAB, HAZI
    OBIT, RBIT, RING, ROP5, RPM , RTAB
Resistivity\PYGZus{}3Track\PYGZus{}Correlation.xml  : Resistivity Linear Correlation Format
    AHT20, AHT60, AHT90, ATR , BS  , CALI, CATR, CILD, CLLD, GR  , HCAL, ILD
    ILM , LLD , LLS , MSFL, PCAL, PSR , RLA0, ROP5, RT  , RXO , SFL , SP
    TENS
Resistivity\PYGZus{}3Track\PYGZus{}Logrithmic.xml   : Logrithmic Resistivity 3Track
    A22H, A34H, AHF10, AHF20, AHF30, AHF60, AHF90, AHO10, AHO20, AHO30, AHO60, AHO90
    AHT10, AHT20, AHT30, AHT60, AHT90, ATR , BS  , CALI, GR  , HCAL, ILD , ILM
    LLD , LLM , MSFL, P16H\PYGZus{}RT, P28H\PYGZus{}RT, P34H\PYGZus{}RT, PCAL, PSR , RLA0, RLA1, RLA2, RLA3
    RLA4, RLA5, ROP5, RXO , SFL , SP  , TENS
Resistivity\PYGZus{}Investigation\PYGZus{}Image.xml : AIT Radial Investigation Image
    AHT10, AHT20, AHT30, AHT60, AHT90, BS  , GR  , HCAL, SP
Sonic\PYGZus{}3Track.xml                    : Sonic DT Porosity 3 Track
    BS  , CALI, DT  , DT0S, DT1R, DT2 , DT2R, DT4S, DTBC, DTCO, DTCU, DTL
    DTLF, DTLN, DTR2, DTR5, DTRA, DTRS, DTSH, DTSM, DTST, DTTA, GR  , HCAL
    PCAL, ROP5, SP  , SPHI, TENS
Sonic\PYGZus{}PWF4                          : SONIC Packed Waveform 4
    CALI, DT1 , DT2 , DTCO, DTSM, DTST, GR  , HCAL, TENS
Sonic\PYGZus{}SPR1\PYGZus{}VDL                      : SONIC Receiver Array Lower Dipole VDL
    CALI, DT1 , DT1 , DT2 , DTCO, DTSM, DTST, GR  , HCAL, TENS
Sonic\PYGZus{}SPR2\PYGZus{}VDL                      : SONIC Receiver Array Upper Dipole VDL
    CALI, DT1 , DT2 , DT2 , DTCO, DTSM, DTST, GR  , HCAL, TENS
Sonic\PYGZus{}SPR3\PYGZus{}VDL                      : SONIC Receiver Array Stonely VDL
    CALI, DT1 , DT2 , DT3R, DTCO, DTSM, DTST, GR  , HCAL, TENS
Sonic\PYGZus{}SPR4\PYGZus{}VDL                      : SONIC Receiver Array P and S VDL
    CALI, DT1 , DT2 , DTCO, DTRP, DTRS, DTSM, DTST, GR  , HCAL, TENS
Triple\PYGZus{}Combo                        : Resistivity Density Neutron GR 3Track Format
    AHT10, AHT20, AHT30, AHT60, AHT90, APDC, APLC, APSC, ATR , BS  , C1  , C2
    CALI, CMFF, CMRP, DPHB, DPHI, DPHZ, DPOR\PYGZus{}CDN, DSOZ, ENPH, GR  , HCAL, HMIN
    HMNO, ILD , ILM , LLD , LLM , MSFL, NPHI, NPOR, PCAL, PEFZ, PSR , RLA0
    RLA1, RLA2, RLA3, RLA4, RLA5, ROP5, RSOZ, RXO , RXOZ, SFL , SNP , SP
    SPHI, TENS, TNPB, TNPH, TNPH\PYGZus{}CDN, TPHI
\end{sphinxVerbatim}


\subparagraph{Plotting Logs}
\label{\detokenize{cmd_line_tools/TD_cmd_line_tools:plotting-logs}}
Here is an example of plotting LIS and LAS files in directory \sphinxcode{in/} with the plots in directory \sphinxcode{out/}. The following options have been invoked:
\begin{itemize}
\item {} 
API headers on the top of each plot: \sphinxcode{-A}

\item {} 
Multiprocessing on with 4 simultaneous jobs: \sphinxcode{-j4}

\item {} 
Recursive search of input directory: \sphinxcode{-r}

\item {} 
Uses any available plot specifications from LgFormat XML files which result in 4 curves or more being plotted: \sphinxcode{-X 4}

\end{itemize}

The command line is:

\begin{sphinxVerbatim}[commandchars=\\\{\}]
\PYGZdl{} python3 tdplotlogs \PYGZhy{}A \PYGZhy{}j4 \PYGZhy{}r \PYGZhy{}X 4 in/ out/
\end{sphinxVerbatim}

First tdplotlogs echos the command:

\begin{sphinxVerbatim}[commandchars=\\\{\}]
\PYG{n}{Cmd}\PYG{p}{:} \PYG{n}{tdplotlogs} \PYG{o}{\PYGZhy{}}\PYG{n}{A} \PYG{o}{\PYGZhy{}}\PYG{n}{j4} \PYG{o}{\PYGZhy{}}\PYG{n}{r} \PYG{o}{\PYGZhy{}}\PYG{n}{X} \PYG{l+m+mi}{4} \PYG{o+ow}{in}\PYG{o}{/} \PYG{n}{out}\PYG{o}{/}
\end{sphinxVerbatim}

When complete tdplotlogs writes out a summary, first the number of files read (output is wrapped here with ‘' for clarity):

\begin{sphinxVerbatim}[commandchars=\\\{\}]
\PYG{n}{plotLogInfo} \PYG{n}{PlotLogInfo} \PYG{o}{\PYGZlt{}}\PYG{n}{\PYGZus{}\PYGZus{}main\PYGZus{}\PYGZus{}}\PYG{o}{.}\PYG{n}{PlotLogInfo} \PYG{n+nb}{object} \PYG{n}{at} \PYG{l+m+mh}{0x101e0da90}\PYG{o}{\PYGZgt{}} \PYGZbs{}
        \PYG{n}{Files}\PYG{o}{=}\PYG{l+m+mi}{23} \PYGZbs{}
        \PYG{n}{Bytes}\PYG{o}{=}\PYG{l+m+mi}{10648531} \PYGZbs{}
        \PYG{n}{LogPasses}\PYG{o}{=}\PYG{l+m+mi}{23} \PYGZbs{}
        \PYG{n}{Plots}\PYG{o}{=}\PYG{l+m+mi}{8} \PYGZbs{}
        \PYG{n}{Curve} \PYG{n}{points}\PYG{o}{=}\PYG{l+m+mi}{229991}
\end{sphinxVerbatim}

Then as summary of each plot in detail (output is wrapped here with ‘' for clarity):

\begin{sphinxVerbatim}[commandchars=\\\{\}]
\PYG{p}{(}\PYG{l+s+s1}{\PYGZsq{}}\PYG{l+s+s1}{in/1003578128.las}\PYG{l+s+s1}{\PYGZsq{}}\PYG{p}{,} \PYGZbs{}
        \PYG{l+m+mi}{0}\PYG{p}{,} \PYGZbs{}
        \PYG{l+s+s1}{\PYGZsq{}}\PYG{l+s+s1}{Natural\PYGZus{}GR\PYGZus{}Spectrometry\PYGZus{}3Track.xml}\PYG{l+s+s1}{\PYGZsq{}}\PYG{p}{,} \PYGZbs{}
        \PYG{n}{IndexTableValue}\PYG{p}{(} \PYGZbs{}
                \PYG{n}{scale}\PYG{o}{=}\PYG{l+m+mi}{100}\PYG{p}{,} \PYGZbs{}
                \PYG{n}{evFirst}\PYG{o}{=}\PYG{l+s+s1}{\PYGZsq{}}\PYG{l+s+s1}{800.5 (FEET)}\PYG{l+s+s1}{\PYGZsq{}}\PYG{p}{,} \PYGZbs{}
                \PYG{n}{evLast}\PYG{o}{=}\PYG{l+s+s1}{\PYGZsq{}}\PYG{l+s+s1}{3019.5 (FEET)}\PYG{l+s+s1}{\PYGZsq{}}\PYG{p}{,} \PYGZbs{}
                \PYG{n}{evInterval}\PYG{o}{=}\PYG{l+s+s1}{\PYGZsq{}}\PYG{l+s+s1}{2219.0 (FEET)}\PYG{l+s+s1}{\PYGZsq{}}\PYG{p}{,} \PYGZbs{}
                \PYG{n}{curves}\PYG{o}{=}\PYG{l+s+s1}{\PYGZsq{}}\PYG{l+s+s1}{CGR\PYGZus{}2, POTA, SGR\PYGZus{}1, TENS\PYGZus{}16, THOR, URAN}\PYG{l+s+s1}{\PYGZsq{}}\PYG{p}{,} \PYGZbs{}
                \PYG{n}{numPoints}\PYG{o}{=}\PYG{l+m+mi}{26213}\PYG{p}{,} \PYGZbs{}
                \PYG{n}{outPath}\PYG{o}{=}\PYG{l+s+s1}{\PYGZsq{}}\PYG{l+s+s1}{out//1003578128.las\PYGZus{}0000\PYGZus{}Natural\PYGZus{}GR\PYGZus{}Spectrometry\PYGZus{}3Track.xml.svg}\PYG{l+s+s1}{\PYGZsq{}} \PYGZbs{}
        \PYG{p}{)}
\PYG{p}{)}
\PYG{p}{(}\PYG{l+s+s1}{\PYGZsq{}}\PYG{l+s+s1}{in/1003578128.las}\PYG{l+s+s1}{\PYGZsq{}}\PYG{p}{,} \PYGZbs{}
        \PYG{l+m+mi}{0}\PYG{p}{,} \PYGZbs{}
        \PYG{l+s+s1}{\PYGZsq{}}\PYG{l+s+s1}{Porosity\PYGZus{}GR\PYGZus{}3Track}\PYG{l+s+s1}{\PYGZsq{}}\PYG{p}{,} \PYGZbs{}
        \PYG{n}{IndexTableValue}\PYG{p}{(} \PYGZbs{}
                \PYG{n}{scale}\PYG{o}{=}\PYG{l+m+mi}{100}\PYG{p}{,} \PYGZbs{}
                \PYG{n}{evFirst}\PYG{o}{=}\PYG{l+s+s1}{\PYGZsq{}}\PYG{l+s+s1}{800.5 (FEET)}\PYG{l+s+s1}{\PYGZsq{}}\PYG{p}{,} \PYGZbs{}
                \PYG{n}{evLast}\PYG{o}{=}\PYG{l+s+s1}{\PYGZsq{}}\PYG{l+s+s1}{3019.5 (FEET)}\PYG{l+s+s1}{\PYGZsq{}}\PYG{p}{,} \PYGZbs{}
                \PYG{n}{evInterval}\PYG{o}{=}\PYG{l+s+s1}{\PYGZsq{}}\PYG{l+s+s1}{2219.0 (FEET)}\PYG{l+s+s1}{\PYGZsq{}}\PYG{p}{,} \PYGZbs{}
                \PYG{n}{curves}\PYG{o}{=}\PYG{l+s+s1}{\PYGZsq{}}\PYG{l+s+s1}{Cali, DRHO, DensityPorosity, GammaRay, NeutronPorosity, OLDESTNeutronPorosity, OLDNeutronPorosity, RHOB, SP, SonicPorosity, Tension}\PYG{l+s+s1}{\PYGZsq{}}\PYG{p}{,} \PYGZbs{}
                \PYG{n}{numPoints}\PYG{o}{=}\PYG{l+m+mi}{46170}\PYG{p}{,} \PYGZbs{}
                \PYG{n}{outPath}\PYG{o}{=}\PYG{l+s+s1}{\PYGZsq{}}\PYG{l+s+s1}{out//1003578128.las\PYGZus{}0000\PYGZus{}Porosity\PYGZus{}GR\PYGZus{}3Track.svg}\PYG{l+s+s1}{\PYGZsq{}} \PYGZbs{}
        \PYG{p}{)}
\PYG{p}{)}

\PYG{o}{.}\PYG{o}{.}\PYG{o}{.} \PYG{l+m+mi}{8}\PYG{o}{\PYGZlt{}}\PYG{o}{\PYGZhy{}}\PYG{o}{\PYGZhy{}}\PYG{o}{\PYGZhy{}}\PYG{o}{\PYGZhy{}}\PYG{o}{\PYGZhy{}}\PYG{o}{\PYGZhy{}}\PYG{o}{\PYGZhy{}}\PYG{o}{\PYGZhy{}}\PYG{o}{\PYGZhy{}}\PYG{o}{\PYGZhy{}}\PYG{o}{\PYGZhy{}}\PYG{o}{\PYGZhy{}}\PYG{o}{\PYGZhy{}} \PYG{n}{Snip} \PYG{o}{\PYGZhy{}}\PYG{o}{\PYGZhy{}}\PYG{o}{\PYGZhy{}}\PYG{o}{\PYGZhy{}}\PYG{o}{\PYGZhy{}}\PYG{o}{\PYGZhy{}}\PYG{o}{\PYGZhy{}}\PYG{o}{\PYGZhy{}}\PYG{o}{\PYGZhy{}}\PYG{o}{\PYGZhy{}}\PYG{o}{\PYGZhy{}}\PYG{o}{\PYGZhy{}}\PYG{o}{\PYGZhy{}}\PYG{o}{\PYGZgt{}}\PYG{l+m+mi}{8}

\PYG{p}{(}\PYG{l+s+s1}{\PYGZsq{}}\PYG{l+s+s1}{in/1006346987.las}\PYG{l+s+s1}{\PYGZsq{}}\PYG{p}{,} \PYGZbs{}
        \PYG{l+m+mi}{0}\PYG{p}{,} \PYG{l+s+s1}{\PYGZsq{}}\PYG{l+s+s1}{Sonic\PYGZus{}3Track.xml}\PYG{l+s+s1}{\PYGZsq{}}\PYG{p}{,} \PYGZbs{}
        \PYG{n}{IndexTableValue}\PYG{p}{(}
                \PYG{n}{scale}\PYG{o}{=}\PYG{l+m+mi}{100}\PYG{p}{,} \PYGZbs{}
                \PYG{n}{evFirst}\PYG{o}{=}\PYG{l+s+s1}{\PYGZsq{}}\PYG{l+s+s1}{4597.5 (FEET)}\PYG{l+s+s1}{\PYGZsq{}}\PYG{p}{,} \PYGZbs{}
                \PYG{n}{evLast}\PYG{o}{=}\PYG{l+s+s1}{\PYGZsq{}}\PYG{l+s+s1}{5799.5 (FEET)}\PYG{l+s+s1}{\PYGZsq{}}\PYG{p}{,} \PYGZbs{}
                \PYG{n}{evInterval}\PYG{o}{=}\PYG{l+s+s1}{\PYGZsq{}}\PYG{l+s+s1}{1202.0 (FEET)}\PYG{l+s+s1}{\PYGZsq{}}\PYG{p}{,} \PYGZbs{}
                \PYG{n}{curves}\PYG{o}{=}\PYG{l+s+s1}{\PYGZsq{}}\PYG{l+s+s1}{Caliper, DT, DTL\PYGZus{}DDBHC, GammaRay, SonicPorosity, TENSION}\PYG{l+s+s1}{\PYGZsq{}}\PYG{p}{,} \PYGZbs{}
                \PYG{n}{numPoints}\PYG{o}{=}\PYG{l+m+mi}{14430}\PYG{p}{,} \PYGZbs{}
                \PYG{n}{outPath}\PYG{o}{=}\PYG{l+s+s1}{\PYGZsq{}}\PYG{l+s+s1}{out//1006346987.las\PYGZus{}0000\PYGZus{}Sonic\PYGZus{}3Track.xml.svg}\PYG{l+s+s1}{\PYGZsq{}} \PYGZbs{}
        \PYG{p}{)}
\PYG{p}{)}
\end{sphinxVerbatim}

The fields in each tuple are:
\begin{itemize}
\item {} 
Input file name.

\item {} 
LogPass number in the file. For example “Repeat Section” might be 0 and “Main Log” 1.

\item {} 
LgFormat used for the plot (several plots my be generated from one LogPass).

\item {} \begin{description}
\item[{An IndexTableValue object (used to generate the index.html file) that has the following fields:}] \leavevmode\begin{itemize}
\item {} 
Plot scale as an integer.

\item {} 
First reading and units as an Engineering Value.

\item {} 
Last reading and units as an Engineering Value.

\item {} 
Log interval and units as an Engineering Value.

\item {} 
List of curve names plotted.

\item {} 
Total number of data points plotted.

\item {} 
The ouput file.

\end{itemize}

\end{description}

\end{itemize}

Finally the total number of curve feet plotted and the time it took:

\begin{sphinxVerbatim}[commandchars=\\\{\}]
Interval*curves: EngVal: 121020.000 (FEET)
  CPU time =    0.043 (S)
Exec. time =   25.119 (S)
Bye, bye!
\end{sphinxVerbatim}

In this case (under Unix) the “CPU Time” is the cumulative amount of CPU time used. As we are using multiprocessing it is the CPU time of the parent process which is very small since it just invokes child processes. The Exec. time is the wall clock time between starting and finishing tdplotlogs.

In the output directory will be an index.html file that has a table with the fields that duplicate those on the command line output. It looks like this:

\noindent\sphinxincludegraphics{{PlotLogsIndex}.png}

The links in the last column are to the SVG plots. Her is a screen shot of one:

\noindent\sphinxincludegraphics{{PlotLogsPlotExample}.png}


\subparagraph{Sample Plots}
\label{\detokenize{cmd_line_tools/TD_cmd_line_tools:sample-plots}}
Here is an actual plot from a \sphinxhref{../copy\_to\_html/plots/Triple\_Combo\_46\_LAS.svg}{LAS file} and there are many more examples here: {\hyperref[\detokenize{examples/examples:totaldepth-examples-plots}]{\sphinxcrossref{\DUrole{std,std-ref}{Wireline Plots}}}}.


\subsection{LIS Command Line Tools}
\label{\detokenize{cmd_line_tools/LIS_cmd_line_tools:lis-command-line-tools}}\label{\detokenize{cmd_line_tools/LIS_cmd_line_tools::doc}}
This describes the command line tools that are available for processing LIS files.


\begin{savenotes}\sphinxattablestart
\centering
\begin{tabulary}{\linewidth}[t]{|T|T|}
\hline
\sphinxstylethead{\sphinxstyletheadfamily 
Tool Name
\unskip}\relax &\sphinxstylethead{\sphinxstyletheadfamily 
Description
\unskip}\relax \\
\hline
\sphinxcode{tdlisdetif}
&
Removes TIF markers from a LIS file.
\\
\hline
\sphinxcode{tdlisdumpframeset}
&
Writes out the frame values as a CSV file.
\\
\hline
\sphinxcode{tdlisindex}
&
Indexes a LIS file.
\\
\hline
\sphinxcode{tdlisplotlogpasses}
&
Plots the log data as SVG pages.
\\
\hline
\sphinxcode{tdlisscanlogidata}
&
Scans the logical data.
\\
\hline
\sphinxcode{tdlisscanlogirecord}
&
Scans all Logical records.
\\
\hline
\sphinxcode{tdlisscanphysrec}
&
Scans all the Physical Records.
\\
\hline
\sphinxcode{tdlistablehistogram}
&
Analyses the contents of table Logical Records.
\\
\hline
\sphinxcode{tdlistohtml}
&
Generates a HTML page about the LIS file.
\\
\hline
\sphinxcode{tdplotlogs}
&
Plots logs as SVG files.
\\
\hline
\end{tabulary}
\par
\sphinxattableend\end{savenotes}


\subsubsection{Scanning LIS Files}
\label{\detokenize{cmd_line_tools/LIS_cmd_line_tools:scanning-lis-files}}
These command line tools scan LIS files without changing them.


\paragraph{\sphinxstyleliteralintitle{tdlisscanphysrec}}
\label{\detokenize{cmd_line_tools/LIS_cmd_line_tools:tdlisscanphysrec}}
Scans a LIS79 file and reports the Physical Record structure.


\subparagraph{Arguments}
\label{\detokenize{cmd_line_tools/LIS_cmd_line_tools:arguments}}
One argument that will be treated as a path to a LIS file.


\subparagraph{Options}
\label{\detokenize{cmd_line_tools/LIS_cmd_line_tools:options}}

\begin{savenotes}\sphinxattablestart
\centering
\begin{tabulary}{\linewidth}[t]{|T|T|}
\hline
\sphinxstylethead{\sphinxstyletheadfamily 
Option
\unskip}\relax &\sphinxstylethead{\sphinxstyletheadfamily 
Description
\unskip}\relax \\
\hline
\sphinxcode{-{-}version}
&
Show program’s version number and exit
\\
\hline
\sphinxcode{-h, -{-}help}
&
Show this help message and exit.
\\
\hline
\sphinxcode{-k, -{-}keep-going}
&
Keep going as far as sensible. {[}default: False{]}
\\
\hline
\sphinxcode{-l LOGLEVEL, -{-}loglevel=LOGLEVEL}
&
Log Level (debug=10, info=20, warning=30, error=40, critical=50) {[}default: 20{]}
\\
\hline
\end{tabulary}
\par
\sphinxattableend\end{savenotes}


\subparagraph{Examples}
\label{\detokenize{cmd_line_tools/LIS_cmd_line_tools:examples}}
Example of scanning a non-TIF encoded file:

\begin{sphinxVerbatim}[commandchars=\\\{\}]
\PYGZdl{} {}`{}`tdlisscanphysrec{}`{}` LIS.lis
Cmd: ScanPhysRec.py LIS.lis
PR:     tell()  Length    Attr  LD\PYGZus{}len  RecNum  FilNum  ChkSum   LR Attr [Total LD]
\PYGZhy{}\PYGZhy{}\PYGZhy{}\PYGZhy{}\PYGZhy{}\PYGZhy{}\PYGZhy{}\PYGZhy{}\PYGZhy{}\PYGZhy{}\PYGZhy{}\PYGZhy{}\PYGZhy{}\PYGZhy{}\PYGZhy{}\PYGZhy{}\PYGZhy{}\PYGZhy{}\PYGZhy{}\PYGZhy{}\PYGZhy{}\PYGZhy{}\PYGZhy{}\PYGZhy{}\PYGZhy{}\PYGZhy{}\PYGZhy{}\PYGZhy{}\PYGZhy{}\PYGZhy{}\PYGZhy{}\PYGZhy{}\PYGZhy{}\PYGZhy{}\PYGZhy{}\PYGZhy{}\PYGZhy{}\PYGZhy{} start \PYGZhy{}\PYGZhy{}\PYGZhy{}\PYGZhy{}\PYGZhy{}\PYGZhy{}\PYGZhy{}\PYGZhy{}\PYGZhy{}\PYGZhy{}\PYGZhy{}\PYGZhy{}\PYGZhy{}\PYGZhy{}\PYGZhy{}\PYGZhy{}\PYGZhy{}\PYGZhy{}\PYGZhy{}\PYGZhy{}\PYGZhy{}\PYGZhy{}\PYGZhy{}\PYGZhy{}\PYGZhy{}\PYGZhy{}\PYGZhy{}\PYGZhy{}\PYGZhy{}\PYGZhy{}\PYGZhy{}\PYGZhy{}\PYGZhy{}\PYGZhy{}\PYGZhy{}\PYGZhy{}\PYGZhy{}\PYGZhy{}
PR: 0x       0      62  0x   0      58  \PYGZhy{}\PYGZhy{}\PYGZhy{}\PYGZhy{}\PYGZhy{}\PYGZhy{}  \PYGZhy{}\PYGZhy{}\PYGZhy{}\PYGZhy{}\PYGZhy{}\PYGZhy{}  \PYGZhy{}\PYGZhy{}\PYGZhy{}\PYGZhy{}\PYGZhy{}\PYGZhy{} 0x80 0x00 [      58]
PR: 0x      3e    1024  0x   1    1020  \PYGZhy{}\PYGZhy{}\PYGZhy{}\PYGZhy{}\PYGZhy{}\PYGZhy{}  \PYGZhy{}\PYGZhy{}\PYGZhy{}\PYGZhy{}\PYGZhy{}\PYGZhy{}  \PYGZhy{}\PYGZhy{}\PYGZhy{}\PYGZhy{}\PYGZhy{}\PYGZhy{} 0x22 0x00
PR: 0x     43e    1024  0x   3    1020  \PYGZhy{}\PYGZhy{}\PYGZhy{}\PYGZhy{}\PYGZhy{}\PYGZhy{}  \PYGZhy{}\PYGZhy{}\PYGZhy{}\PYGZhy{}\PYGZhy{}\PYGZhy{}  \PYGZhy{}\PYGZhy{}\PYGZhy{}\PYGZhy{}\PYGZhy{}\PYGZhy{} + \PYGZhy{}\PYGZhy{}   \PYGZhy{}\PYGZhy{}
PR: 0x     83e    1024  0x   3    1020  \PYGZhy{}\PYGZhy{}\PYGZhy{}\PYGZhy{}\PYGZhy{}\PYGZhy{}  \PYGZhy{}\PYGZhy{}\PYGZhy{}\PYGZhy{}\PYGZhy{}\PYGZhy{}  \PYGZhy{}\PYGZhy{}\PYGZhy{}\PYGZhy{}\PYGZhy{}\PYGZhy{} + \PYGZhy{}\PYGZhy{}   \PYGZhy{}\PYGZhy{}
PR: 0x     c3e    1024  0x   3    1020  \PYGZhy{}\PYGZhy{}\PYGZhy{}\PYGZhy{}\PYGZhy{}\PYGZhy{}  \PYGZhy{}\PYGZhy{}\PYGZhy{}\PYGZhy{}\PYGZhy{}\PYGZhy{}  \PYGZhy{}\PYGZhy{}\PYGZhy{}\PYGZhy{}\PYGZhy{}\PYGZhy{} + \PYGZhy{}\PYGZhy{}   \PYGZhy{}\PYGZhy{}
PR: 0x    103e    1024  0x   3    1020  \PYGZhy{}\PYGZhy{}\PYGZhy{}\PYGZhy{}\PYGZhy{}\PYGZhy{}  \PYGZhy{}\PYGZhy{}\PYGZhy{}\PYGZhy{}\PYGZhy{}\PYGZhy{}  \PYGZhy{}\PYGZhy{}\PYGZhy{}\PYGZhy{}\PYGZhy{}\PYGZhy{} + \PYGZhy{}\PYGZhy{}   \PYGZhy{}\PYGZhy{}
PR: 0x    143e    1024  0x   3    1020  \PYGZhy{}\PYGZhy{}\PYGZhy{}\PYGZhy{}\PYGZhy{}\PYGZhy{}  \PYGZhy{}\PYGZhy{}\PYGZhy{}\PYGZhy{}\PYGZhy{}\PYGZhy{}  \PYGZhy{}\PYGZhy{}\PYGZhy{}\PYGZhy{}\PYGZhy{}\PYGZhy{} + \PYGZhy{}\PYGZhy{}   \PYGZhy{}\PYGZhy{}
PR: 0x    183e    1024  0x   3    1020  \PYGZhy{}\PYGZhy{}\PYGZhy{}\PYGZhy{}\PYGZhy{}\PYGZhy{}  \PYGZhy{}\PYGZhy{}\PYGZhy{}\PYGZhy{}\PYGZhy{}\PYGZhy{}  \PYGZhy{}\PYGZhy{}\PYGZhy{}\PYGZhy{}\PYGZhy{}\PYGZhy{} + \PYGZhy{}\PYGZhy{}   \PYGZhy{}\PYGZhy{}
PR: 0x    1c3e    1024  0x   3    1020  \PYGZhy{}\PYGZhy{}\PYGZhy{}\PYGZhy{}\PYGZhy{}\PYGZhy{}  \PYGZhy{}\PYGZhy{}\PYGZhy{}\PYGZhy{}\PYGZhy{}\PYGZhy{}  \PYGZhy{}\PYGZhy{}\PYGZhy{}\PYGZhy{}\PYGZhy{}\PYGZhy{} + \PYGZhy{}\PYGZhy{}   \PYGZhy{}\PYGZhy{}
PR: 0x    203e      34  0x   2      30  \PYGZhy{}\PYGZhy{}\PYGZhy{}\PYGZhy{}\PYGZhy{}\PYGZhy{}  \PYGZhy{}\PYGZhy{}\PYGZhy{}\PYGZhy{}\PYGZhy{}\PYGZhy{}  \PYGZhy{}\PYGZhy{}\PYGZhy{}\PYGZhy{}\PYGZhy{}\PYGZhy{} + \PYGZhy{}\PYGZhy{}   \PYGZhy{}\PYGZhy{} [    8190]
PR: 0x    2060     304  0x   0     300  \PYGZhy{}\PYGZhy{}\PYGZhy{}\PYGZhy{}\PYGZhy{}\PYGZhy{}  \PYGZhy{}\PYGZhy{}\PYGZhy{}\PYGZhy{}\PYGZhy{}\PYGZhy{}  \PYGZhy{}\PYGZhy{}\PYGZhy{}\PYGZhy{}\PYGZhy{}\PYGZhy{} 0x40 0x00 [     300]
PR: 0x    2190    1014  0x   0    1010  \PYGZhy{}\PYGZhy{}\PYGZhy{}\PYGZhy{}\PYGZhy{}\PYGZhy{}  \PYGZhy{}\PYGZhy{}\PYGZhy{}\PYGZhy{}\PYGZhy{}\PYGZhy{}  \PYGZhy{}\PYGZhy{}\PYGZhy{}\PYGZhy{}\PYGZhy{}\PYGZhy{} 0x00 0x00 [    1010]
PR: 0x    2586    1014  0x   0    1010  \PYGZhy{}\PYGZhy{}\PYGZhy{}\PYGZhy{}\PYGZhy{}\PYGZhy{}  \PYGZhy{}\PYGZhy{}\PYGZhy{}\PYGZhy{}\PYGZhy{}\PYGZhy{}  \PYGZhy{}\PYGZhy{}\PYGZhy{}\PYGZhy{}\PYGZhy{}\PYGZhy{} 0x00 0x00 [    1010]
PR: 0x    297c    1014  0x   0    1010  \PYGZhy{}\PYGZhy{}\PYGZhy{}\PYGZhy{}\PYGZhy{}\PYGZhy{}  \PYGZhy{}\PYGZhy{}\PYGZhy{}\PYGZhy{}\PYGZhy{}\PYGZhy{}  \PYGZhy{}\PYGZhy{}\PYGZhy{}\PYGZhy{}\PYGZhy{}\PYGZhy{} 0x00 0x00 [    1010]
PR: 0x    2d72    1014  0x   0    1010  \PYGZhy{}\PYGZhy{}\PYGZhy{}\PYGZhy{}\PYGZhy{}\PYGZhy{}  \PYGZhy{}\PYGZhy{}\PYGZhy{}\PYGZhy{}\PYGZhy{}\PYGZhy{}  \PYGZhy{}\PYGZhy{}\PYGZhy{}\PYGZhy{}\PYGZhy{}\PYGZhy{} 0x00 0x00 [    1010]
PR: 0x    3168    1014  0x   0    1010  \PYGZhy{}\PYGZhy{}\PYGZhy{}\PYGZhy{}\PYGZhy{}\PYGZhy{}  \PYGZhy{}\PYGZhy{}\PYGZhy{}\PYGZhy{}\PYGZhy{}\PYGZhy{}  \PYGZhy{}\PYGZhy{}\PYGZhy{}\PYGZhy{}\PYGZhy{}\PYGZhy{} 0x00 0x00 [    1010]
PR: 0x    355e    1014  0x   0    1010  \PYGZhy{}\PYGZhy{}\PYGZhy{}\PYGZhy{}\PYGZhy{}\PYGZhy{}  \PYGZhy{}\PYGZhy{}\PYGZhy{}\PYGZhy{}\PYGZhy{}\PYGZhy{}  \PYGZhy{}\PYGZhy{}\PYGZhy{}\PYGZhy{}\PYGZhy{}\PYGZhy{} 0x00 0x00 [    1010]
PR: 0x    3954    1014  0x   0    1010  \PYGZhy{}\PYGZhy{}\PYGZhy{}\PYGZhy{}\PYGZhy{}\PYGZhy{}  \PYGZhy{}\PYGZhy{}\PYGZhy{}\PYGZhy{}\PYGZhy{}\PYGZhy{}  \PYGZhy{}\PYGZhy{}\PYGZhy{}\PYGZhy{}\PYGZhy{}\PYGZhy{} 0x00 0x00 [    1010]
PR: 0x    3d4a    1014  0x   0    1010  \PYGZhy{}\PYGZhy{}\PYGZhy{}\PYGZhy{}\PYGZhy{}\PYGZhy{}  \PYGZhy{}\PYGZhy{}\PYGZhy{}\PYGZhy{}\PYGZhy{}\PYGZhy{}  \PYGZhy{}\PYGZhy{}\PYGZhy{}\PYGZhy{}\PYGZhy{}\PYGZhy{} 0x00 0x00 [    1010]
PR: 0x    4140     942  0x   0     938  \PYGZhy{}\PYGZhy{}\PYGZhy{}\PYGZhy{}\PYGZhy{}\PYGZhy{}  \PYGZhy{}\PYGZhy{}\PYGZhy{}\PYGZhy{}\PYGZhy{}\PYGZhy{}  \PYGZhy{}\PYGZhy{}\PYGZhy{}\PYGZhy{}\PYGZhy{}\PYGZhy{} 0x00 0x00 [     938]
PR: 0x    44ee      62  0x   0      58  \PYGZhy{}\PYGZhy{}\PYGZhy{}\PYGZhy{}\PYGZhy{}\PYGZhy{}  \PYGZhy{}\PYGZhy{}\PYGZhy{}\PYGZhy{}\PYGZhy{}\PYGZhy{}  \PYGZhy{}\PYGZhy{}\PYGZhy{}\PYGZhy{}\PYGZhy{}\PYGZhy{} 0x81 0x00 [      58]
PR: EOF
\PYGZhy{}\PYGZhy{}\PYGZhy{}\PYGZhy{}\PYGZhy{}\PYGZhy{}\PYGZhy{}\PYGZhy{}\PYGZhy{}\PYGZhy{}\PYGZhy{}\PYGZhy{}\PYGZhy{}\PYGZhy{}\PYGZhy{}\PYGZhy{}\PYGZhy{}\PYGZhy{}\PYGZhy{}\PYGZhy{}\PYGZhy{}\PYGZhy{}\PYGZhy{}\PYGZhy{}\PYGZhy{}\PYGZhy{}\PYGZhy{}\PYGZhy{}\PYGZhy{}\PYGZhy{}\PYGZhy{}\PYGZhy{}\PYGZhy{}\PYGZhy{}\PYGZhy{}\PYGZhy{}\PYGZhy{}\PYGZhy{}\PYGZhy{} EOF \PYGZhy{}\PYGZhy{}\PYGZhy{}\PYGZhy{}\PYGZhy{}\PYGZhy{}\PYGZhy{}\PYGZhy{}\PYGZhy{}\PYGZhy{}\PYGZhy{}\PYGZhy{}\PYGZhy{}\PYGZhy{}\PYGZhy{}\PYGZhy{}\PYGZhy{}\PYGZhy{}\PYGZhy{}\PYGZhy{}\PYGZhy{}\PYGZhy{}\PYGZhy{}\PYGZhy{}\PYGZhy{}\PYGZhy{}\PYGZhy{}\PYGZhy{}\PYGZhy{}\PYGZhy{}\PYGZhy{}\PYGZhy{}\PYGZhy{}\PYGZhy{}\PYGZhy{}\PYGZhy{}\PYGZhy{}\PYGZhy{}\PYGZhy{}
PR Count: 21
Histogram of Physical Record lengths:
Bytes
   34 [1] \textbar{} +++++++++++
   62 [2] \textbar{} ++++++++++++++++++++++
  304 [1] \textbar{} +++++++++++
  942 [1] \textbar{} +++++++++++
 1014 [8] \textbar{} ++++++++++++++++++++++++++++++++++++++++++++++++++++++++++++++++++++++++++++++++++++++++
 1024 [8] \textbar{} ++++++++++++++++++++++++++++++++++++++++++++++++++++++++++++++++++++++++++++++++++++++++
CPU time =    0.001 (S)
Bye, bye!
\end{sphinxVerbatim}

First \sphinxcode{tdlisscanphysrec} echo’s the command line then it scans the Physical Records an writes out a table that has the following columns:


\begin{savenotes}\sphinxattablestart
\centering
\begin{tabulary}{\linewidth}[t]{|T|T|}
\hline
\sphinxstylethead{\sphinxstyletheadfamily 
Heading
\unskip}\relax &\sphinxstylethead{\sphinxstyletheadfamily 
Description
\unskip}\relax \\
\hline
\sphinxcode{tell()}
&
The file position of the start of the Physical Record as a hex integer.
\\
\hline
\sphinxcode{Length}
&
The length of the Physical Record as a decimal integer.
\\
\hline
\sphinxcode{Attr}
&
The Physical Record Header attributes as a hex integer.
\\
\hline
\sphinxcode{LD\_len}
&
The length of the logical data payload contained in this Physical Record.
\\
\hline
\sphinxcode{RecNum}
&
A record number from the Physical Record trailer if present, otherwise: \sphinxcode{-{-}-{-}-{-}}
\\
\hline
\sphinxcode{FilNum}
&
A file number from the Physical Record trailer if present, otherwise: \sphinxcode{-{-}-{-}-{-}}
\\
\hline
\sphinxcode{ChkSum}
&
A checksum from the Physical Record trailer if present, otherwise: \sphinxcode{-{-}-{-}-{-}}
\\
\hline
\sphinxcode{LR}
&
Logical Record type from the Logical Record Header as a hex integer.
\\
\hline
\sphinxcode{Attr}
&
Logical Record attributes from the Logical Record Header as a hex integer. This is (almost?) always 0x00
\\
\hline
\sphinxcode{{[}Total LD{]}}
&
The total length of the logical data in the Logical Record if a terminator Physical Record, otherwise blank.
\\
\hline
\end{tabulary}
\par
\sphinxattableend\end{savenotes}

This is followed by an ASCII histogram of the lengths of all Physical Records with the following columns:
\begin{enumerate}
\item {} 
The size in bytes.

\item {} 
The frequency count.

\item {} 
A series of \sphinxcode{+} that is proportionate to the frequency count.

\end{enumerate}

If TIF markers are detected then the output adds TIF columns thus:

\begin{sphinxVerbatim}[commandchars=\\\{\}]
TIF     ?  :        Type        Back        Next  PR:     tell()  Length    Attr  LD\PYGZus{}len  RecNum  FilNum  ChkSum   LR Attr [Total LD]
\PYGZhy{}\PYGZhy{}\PYGZhy{}\PYGZhy{}\PYGZhy{}\PYGZhy{}\PYGZhy{}\PYGZhy{}\PYGZhy{}\PYGZhy{}\PYGZhy{}\PYGZhy{}\PYGZhy{}\PYGZhy{}\PYGZhy{}\PYGZhy{}\PYGZhy{}\PYGZhy{}\PYGZhy{}\PYGZhy{}\PYGZhy{}\PYGZhy{}\PYGZhy{}\PYGZhy{}\PYGZhy{}\PYGZhy{}\PYGZhy{}\PYGZhy{}\PYGZhy{}\PYGZhy{}\PYGZhy{}\PYGZhy{}\PYGZhy{}\PYGZhy{}\PYGZhy{}\PYGZhy{}\PYGZhy{}\PYGZhy{}\PYGZhy{}\PYGZhy{}\PYGZhy{}\PYGZhy{}\PYGZhy{}\PYGZhy{}\PYGZhy{}\PYGZhy{}\PYGZhy{}\PYGZhy{}\PYGZhy{}\PYGZhy{}\PYGZhy{}\PYGZhy{}\PYGZhy{}\PYGZhy{}\PYGZhy{}\PYGZhy{}\PYGZhy{}\PYGZhy{}\PYGZhy{}\PYGZhy{}\PYGZhy{}\PYGZhy{}\PYGZhy{} start \PYGZhy{}\PYGZhy{}\PYGZhy{}\PYGZhy{}\PYGZhy{}\PYGZhy{}\PYGZhy{}\PYGZhy{}\PYGZhy{}\PYGZhy{}\PYGZhy{}\PYGZhy{}\PYGZhy{}\PYGZhy{}\PYGZhy{}\PYGZhy{}\PYGZhy{}\PYGZhy{}\PYGZhy{}\PYGZhy{}\PYGZhy{}\PYGZhy{}\PYGZhy{}\PYGZhy{}\PYGZhy{}\PYGZhy{}\PYGZhy{}\PYGZhy{}\PYGZhy{}\PYGZhy{}\PYGZhy{}\PYGZhy{}\PYGZhy{}\PYGZhy{}\PYGZhy{}\PYGZhy{}\PYGZhy{}\PYGZhy{}\PYGZhy{}\PYGZhy{}\PYGZhy{}\PYGZhy{}\PYGZhy{}\PYGZhy{}\PYGZhy{}\PYGZhy{}\PYGZhy{}\PYGZhy{}\PYGZhy{}\PYGZhy{}\PYGZhy{}\PYGZhy{}\PYGZhy{}\PYGZhy{}\PYGZhy{}\PYGZhy{}\PYGZhy{}\PYGZhy{}\PYGZhy{}\PYGZhy{}\PYGZhy{}\PYGZhy{}\PYGZhy{}
TIF  True \PYGZgt{}:  0x       0  0x       0  0x      4a  PR: 0x       0      62  0x   0      58  \PYGZhy{}\PYGZhy{}\PYGZhy{}\PYGZhy{}\PYGZhy{}\PYGZhy{}  \PYGZhy{}\PYGZhy{}\PYGZhy{}\PYGZhy{}\PYGZhy{}\PYGZhy{}  \PYGZhy{}\PYGZhy{}\PYGZhy{}\PYGZhy{}\PYGZhy{}\PYGZhy{} 0x80 0x00 [      58]
TIF  True \PYGZgt{}:  0x       0  0x       0  0x     456  PR: 0x      4a    1024  0x   1    1020  \PYGZhy{}\PYGZhy{}\PYGZhy{}\PYGZhy{}\PYGZhy{}\PYGZhy{}  \PYGZhy{}\PYGZhy{}\PYGZhy{}\PYGZhy{}\PYGZhy{}\PYGZhy{}  \PYGZhy{}\PYGZhy{}\PYGZhy{}\PYGZhy{}\PYGZhy{}\PYGZhy{} 0x22 0x00
TIF  True \PYGZgt{}:  0x       0  0x      4a  0x     862  PR: 0x     456    1024  0x   3    1020  \PYGZhy{}\PYGZhy{}\PYGZhy{}\PYGZhy{}\PYGZhy{}\PYGZhy{}  \PYGZhy{}\PYGZhy{}\PYGZhy{}\PYGZhy{}\PYGZhy{}\PYGZhy{}  \PYGZhy{}\PYGZhy{}\PYGZhy{}\PYGZhy{}\PYGZhy{}\PYGZhy{} + \PYGZhy{}\PYGZhy{}   \PYGZhy{}\PYGZhy{}
TIF  True \PYGZgt{}:  0x       0  0x     456  0x     c6e  PR: 0x     862    1024  0x   3    1020  \PYGZhy{}\PYGZhy{}\PYGZhy{}\PYGZhy{}\PYGZhy{}\PYGZhy{}  \PYGZhy{}\PYGZhy{}\PYGZhy{}\PYGZhy{}\PYGZhy{}\PYGZhy{}  \PYGZhy{}\PYGZhy{}\PYGZhy{}\PYGZhy{}\PYGZhy{}\PYGZhy{} + \PYGZhy{}\PYGZhy{}   \PYGZhy{}\PYGZhy{}
TIF  True \PYGZgt{}:  0x       0  0x     862  0x    107a  PR: 0x     c6e    1024  0x   3    1020  \PYGZhy{}\PYGZhy{}\PYGZhy{}\PYGZhy{}\PYGZhy{}\PYGZhy{}  \PYGZhy{}\PYGZhy{}\PYGZhy{}\PYGZhy{}\PYGZhy{}\PYGZhy{}  \PYGZhy{}\PYGZhy{}\PYGZhy{}\PYGZhy{}\PYGZhy{}\PYGZhy{} + \PYGZhy{}\PYGZhy{}   \PYGZhy{}\PYGZhy{}
TIF  True \PYGZgt{}:  0x       0  0x     c6e  0x    1486  PR: 0x    107a    1024  0x   3    1020  \PYGZhy{}\PYGZhy{}\PYGZhy{}\PYGZhy{}\PYGZhy{}\PYGZhy{}  \PYGZhy{}\PYGZhy{}\PYGZhy{}\PYGZhy{}\PYGZhy{}\PYGZhy{}  \PYGZhy{}\PYGZhy{}\PYGZhy{}\PYGZhy{}\PYGZhy{}\PYGZhy{} + \PYGZhy{}\PYGZhy{}   \PYGZhy{}\PYGZhy{}
TIF  True \PYGZgt{}:  0x       0  0x    107a  0x    1892  PR: 0x    1486    1024  0x   3    1020  \PYGZhy{}\PYGZhy{}\PYGZhy{}\PYGZhy{}\PYGZhy{}\PYGZhy{}  \PYGZhy{}\PYGZhy{}\PYGZhy{}\PYGZhy{}\PYGZhy{}\PYGZhy{}  \PYGZhy{}\PYGZhy{}\PYGZhy{}\PYGZhy{}\PYGZhy{}\PYGZhy{} + \PYGZhy{}\PYGZhy{}   \PYGZhy{}\PYGZhy{}
TIF  True \PYGZgt{}:  0x       0  0x    1486  0x    1c9e  PR: 0x    1892    1024  0x   3    1020  \PYGZhy{}\PYGZhy{}\PYGZhy{}\PYGZhy{}\PYGZhy{}\PYGZhy{}  \PYGZhy{}\PYGZhy{}\PYGZhy{}\PYGZhy{}\PYGZhy{}\PYGZhy{}  \PYGZhy{}\PYGZhy{}\PYGZhy{}\PYGZhy{}\PYGZhy{}\PYGZhy{} + \PYGZhy{}\PYGZhy{}   \PYGZhy{}\PYGZhy{}
TIF  True \PYGZgt{}:  0x       0  0x    1892  0x    20aa  PR: 0x    1c9e    1024  0x   3    1020  \PYGZhy{}\PYGZhy{}\PYGZhy{}\PYGZhy{}\PYGZhy{}\PYGZhy{}  \PYGZhy{}\PYGZhy{}\PYGZhy{}\PYGZhy{}\PYGZhy{}\PYGZhy{}  \PYGZhy{}\PYGZhy{}\PYGZhy{}\PYGZhy{}\PYGZhy{}\PYGZhy{} + \PYGZhy{}\PYGZhy{}   \PYGZhy{}\PYGZhy{}
TIF  True \PYGZgt{}:  0x       0  0x    1c9e  0x    20ec  PR: 0x    20aa      54  0x   2      50  \PYGZhy{}\PYGZhy{}\PYGZhy{}\PYGZhy{}\PYGZhy{}\PYGZhy{}  \PYGZhy{}\PYGZhy{}\PYGZhy{}\PYGZhy{}\PYGZhy{}\PYGZhy{}  \PYGZhy{}\PYGZhy{}\PYGZhy{}\PYGZhy{}\PYGZhy{}\PYGZhy{} + \PYGZhy{}\PYGZhy{}   \PYGZhy{}\PYGZhy{} [    8210]
\end{sphinxVerbatim}

The additional columns are:


\begin{savenotes}\sphinxattablestart
\centering
\begin{tabulary}{\linewidth}[t]{|T|T|}
\hline
\sphinxstylethead{\sphinxstyletheadfamily 
Heading
\unskip}\relax &\sphinxstylethead{\sphinxstyletheadfamily 
Description
\unskip}\relax \\
\hline
\sphinxcode{?}
&
?
\\
\hline
\sphinxcode{Type}
&
TIF marker type, 0 for in-file record, 1 for EOF.
\\
\hline
\sphinxcode{Back}
&
The file position of the precious TIF marker as a hex integer.
\\
\hline
\sphinxcode{Next}
&
The file position of the next TIF marker as a hex integer.
\\
\hline
\end{tabulary}
\par
\sphinxattableend\end{savenotes}


\paragraph{\sphinxstyleliteralintitle{tdlisscanlogirecord}}
\label{\detokenize{cmd_line_tools/LIS_cmd_line_tools:tdlisscanlogirecord}}
Scans a LIS79 file and reports the Logical Record structure.


\subparagraph{Arguments}
\label{\detokenize{cmd_line_tools/LIS_cmd_line_tools:id1}}
One argument that will be treated as a path to a LIS file.


\subparagraph{Options}
\label{\detokenize{cmd_line_tools/LIS_cmd_line_tools:id2}}

\begin{savenotes}\sphinxattablestart
\centering
\begin{tabulary}{\linewidth}[t]{|T|T|}
\hline
\sphinxstylethead{\sphinxstyletheadfamily 
Option
\unskip}\relax &\sphinxstylethead{\sphinxstyletheadfamily 
Description
\unskip}\relax \\
\hline
\sphinxcode{-{-}version}
&
Show program’s version number and exit
\\
\hline
\sphinxcode{-h, -{-}help}
&
Show this help message and exit.
\\
\hline
\sphinxcode{-k, -{-}keep-going}
&
Keep going as far as sensible. {[}default: False{]}
\\
\hline
\sphinxcode{-l LOGLEVEL, -{-}loglevel=LOGLEVEL}
&
Log Level (debug=10, info=20, warning=30, error=40, critical=50) {[}default: 20{]}
\\
\hline
\sphinxcode{-v, -{-}verbose}
&
Verbose output, this outputs a representation of table data and DFSRs.
\\
\hline
\end{tabulary}
\par
\sphinxattableend\end{savenotes}


\subparagraph{Examples}
\label{\detokenize{cmd_line_tools/LIS_cmd_line_tools:id3}}
Example of scanning a LIS file:

\begin{sphinxVerbatim}[commandchars=\\\{\}]
\PYGZdl{} tdlisscanlogirecord RW.lis
Cmd: ScanLogiRec.py RW.lis
0x00000000 \PYGZlt{}TotalDepth.LIS.core.LogiRec.LrFileHeadRead object at 0x1007981d0\PYGZgt{}: \PYGZdq{}File header\PYGZdq{}
2012\PYGZhy{}02\PYGZhy{}08 17:43:45,078 WARNING  LrTableRead(): Discarding duplicate row b\PYGZsq{}BS7 \PYGZsq{} in table b\PYGZsq{}CONS\PYGZsq{}
2012\PYGZhy{}02\PYGZhy{}08 17:43:45,087 WARNING  LrTableRead.\PYGZus{}\PYGZus{}init\PYGZus{}\PYGZus{}(): Tell: 0x4a LD index: 0x32 Error: FileRead.unpack(): Bytes: b\PYGZsq{}\PYGZbs{}x00\PYGZsq{} not enough for struct that needs: 12 bytes.
0x0000004a \PYGZlt{}TotalDepth.LIS.core.LogiRec.LrTableRead object at 0x100798210\PYGZgt{}: \PYGZdq{}Well site data\PYGZdq{}
0x000020ec \PYGZlt{}TotalDepth.LIS.core.LogiRec.LrDFSRRead object at 0x1007981d0\PYGZgt{}: \PYGZdq{}Data format specification record\PYGZdq{}
0x0006141c \PYGZlt{}TotalDepth.LIS.core.LogiRec.LrFileTailRead object at 0x10058e7d0\PYGZgt{}: \PYGZdq{}File trailer\PYGZdq{}
0x00061466 \PYGZlt{}TotalDepth.LIS.core.LogiRec.LrFileHeadRead object at 0x10058e850\PYGZgt{}: \PYGZdq{}File header\PYGZdq{}
2012\PYGZhy{}02\PYGZhy{}08 17:43:45,103 WARNING  LrTableRead(): Discarding duplicate row b\PYGZsq{}BS7 \PYGZsq{} in table b\PYGZsq{}CONS\PYGZsq{}
0x000614b0 \PYGZlt{}TotalDepth.LIS.core.LogiRec.LrTableRead object at 0x10058e7d0\PYGZgt{}: \PYGZdq{}Well site data\PYGZdq{}
0x0006353e \PYGZlt{}TotalDepth.LIS.core.LogiRec.LrDFSRRead object at 0x10058e850\PYGZgt{}: \PYGZdq{}Data format specification record\PYGZdq{}
0x00065a44 \PYGZlt{}TotalDepth.LIS.core.LogiRec.LrFileTailRead object at 0x10058e850\PYGZgt{}: \PYGZdq{}File trailer\PYGZdq{}
0x00065a8e \PYGZlt{}TotalDepth.LIS.core.LogiRec.LrFileHeadRead object at 0x10058e7d0\PYGZgt{}: \PYGZdq{}File header\PYGZdq{}
2012\PYGZhy{}02\PYGZhy{}08 17:43:45,116 WARNING  LrTableRead(): Discarding duplicate row b\PYGZsq{}BS7 \PYGZsq{} in table b\PYGZsq{}CONS\PYGZsq{}
2012\PYGZhy{}02\PYGZhy{}08 17:43:45,124 WARNING  LrTableRead.\PYGZus{}\PYGZus{}init\PYGZus{}\PYGZus{}(): Tell: 0x65ad8 LD index: 0x32 Error: FileRead.unpack(): Bytes: b\PYGZsq{}\PYGZbs{}x00\PYGZsq{} not enough for struct that needs: 12 bytes.
0x00065ad8 \PYGZlt{}TotalDepth.LIS.core.LogiRec.LrTableRead object at 0x10058e850\PYGZgt{}: \PYGZdq{}Well site data\PYGZdq{}
0x00067b7a \PYGZlt{}TotalDepth.LIS.core.LogiRec.LrDFSRRead object at 0x10058e7d0\PYGZgt{}: \PYGZdq{}Data format specification record\PYGZdq{}
0x000d2c44 \PYGZlt{}TotalDepth.LIS.core.LogiRec.LrFileTailRead object at 0x10058e7d0\PYGZgt{}: \PYGZdq{}File trailer\PYGZdq{}
CPU time =    0.064 (S)
Bye, bye!
\end{sphinxVerbatim}


\paragraph{\sphinxstyleliteralintitle{tdlisscanlogidata}}
\label{\detokenize{cmd_line_tools/LIS_cmd_line_tools:tdlisscanlogidata}}
Scans a LIS79 file and reports the Logical Record structure.


\subparagraph{Arguments}
\label{\detokenize{cmd_line_tools/LIS_cmd_line_tools:id4}}
One argument that will be treated as a path to a LIS file.


\subparagraph{Options}
\label{\detokenize{cmd_line_tools/LIS_cmd_line_tools:id5}}

\begin{savenotes}\sphinxattablestart
\centering
\begin{tabulary}{\linewidth}[t]{|T|T|}
\hline
\sphinxstylethead{\sphinxstyletheadfamily 
Option
\unskip}\relax &\sphinxstylethead{\sphinxstyletheadfamily 
Description
\unskip}\relax \\
\hline
\sphinxcode{-{-}version}
&
Show program’s version number and exit
\\
\hline
\sphinxcode{-h, -{-}help}
&
Show this help message and exit.
\\
\hline
\sphinxcode{-k, -{-}keep-going}
&
Keep going as far as sensible. {[}default: False{]}
\\
\hline
\sphinxcode{-d DUMP, -{-}dump=DUMP}
&
Dump complete data at these integer positions (ws
separated, hex/dec). {[}default: {]}
\\
\hline
\sphinxcode{-l LOGLEVEL, -{-}loglevel=LOGLEVEL}
&
Log Level (debug=10, info=20, warning=30, error=40, critical=50) {[}default: 20{]}
\\
\hline
\sphinxcode{-v, -{-}verbose}
&
Verbose output, this outputs a representation of table data and DFSRs.
\\
\hline
\end{tabulary}
\par
\sphinxattableend\end{savenotes}


\subparagraph{Examples}
\label{\detokenize{cmd_line_tools/LIS_cmd_line_tools:id6}}
Example of scanning a LIS file:

\begin{sphinxVerbatim}[commandchars=\\\{\}]
\PYGZdl{} {}`{}`tdlisscanlogidata{}`{}` LIS.lis
Cmd: ScanLogiData.py LIS.lis
Offset        Length  Type  Logical Data
0x00000000        58   128  b\PYGZsq{}\PYGZbs{}x80\PYGZbs{}x00RUN1R .S01\PYGZbs{}x00\PYGZbs{}x00DAT2TF            \PYGZsq{}...
0x0000003E      8190    34  b\PYGZsq{}\PYGZdq{}\PYGZbs{}x00IA\PYGZbs{}x04\PYGZbs{}x00TYPE    CONS\PYGZbs{}x00A\PYGZbs{}x04\PYGZbs{}x00MNEM    HI\PYGZsq{}...
0x00002060       300    64  b\PYGZsq{}@\PYGZbs{}x00\PYGZbs{}x01\PYGZbs{}x02O\PYGZbs{}x00\PYGZbs{}x00\PYGZbs{}x02\PYGZbs{}x02O\PYGZbs{}x00\PYGZbs{}x00\PYGZbs{}x03\PYGZbs{}x04I\PYGZbs{}x00\PYGZbs{}x00\PYGZbs{}x00\PYGZbs{}x18\PYGZbs{}x04\PYGZbs{}x02O\PYGZbs{}x00\PYGZbs{}x01\PYGZbs{}x08\PYGZbs{}x04D?N\PYGZbs{}x07\PYGZus{}\PYGZbs{}t\PYGZsq{}...
0x00002190      1010     0  b\PYGZdq{}\PYGZbs{}x00\PYGZbs{}x00F@\PYGZsq{}\PYGZbs{}xde\PYGZbs{}xbe76\PYGZbs{}xfb@\PYGZbs{}xd6\PYGZbs{}x1a\PYGZbs{}xc0@\PYGZbs{}xd0\PYGZbs{}xdc\PYGZbs{}xc7D\PYGZbs{}xe0\PYGZbs{}xa6P\PYGZbs{}xba\PYGZbs{}x83\PYGZbs{}x18\PYGZbs{}x00F@\PYGZam{}\PYGZbs{}xa6\PYGZbs{}xbe\PYGZhy{}\PYGZdq{}...
0x00002586      1010     0  b\PYGZsq{}\PYGZbs{}x00\PYGZbs{}x00E\PYGZbs{}xff\PYGZbs{}xe9S\PYGZbs{}xbe:\PYGZbs{}x1f\PYGZbs{}x82@\PYGZbs{}xfe\PYGZpc{}\PYGZbs{}xc9@\PYGZbs{}xf7\PYGZbs{}xd5\PYGZbs{}xb7EA\PYGZbs{}x90\PYGZbs{}xda\PYGZbs{}xba\PYGZbs{}x83\PYGZbs{}x18\PYGZbs{}x00E\PYGZbs{}xff\PYGZbs{}xe6\PYGZbs{}xe3\PYGZbs{}xbe\PYGZbs{}x8a\PYGZsq{}...
0x0000297C      1010     0  b\PYGZsq{}\PYGZbs{}x00\PYGZbs{}x00E\PYGZbs{}xff\PYGZbs{}x82\PYGZbs{}xea\PYGZbs{}xbe\PYGZhy{}\PYGZbs{}xe1\PYGZbs{}xa8@\PYGZbs{}xd83\PYGZbs{}xb6@\PYGZbs{}xf0\PYGZbs{}x0f\PYGZbs{}x0fET\PYGZbs{}x149D\PYGZbs{}xc8\PYGZbs{}x08\PYGZbs{}xc5E\PYGZbs{}xff\PYGZbs{}x80y\PYGZbs{}xbe\PYGZhy{}\PYGZsq{}...
0x00002D72      1010     0  b\PYGZsq{}\PYGZbs{}x00\PYGZbs{}x00E\PYGZbs{}xff\PYGZbs{}x1c\PYGZbs{}x80\PYGZbs{}xbd\PYGZbs{}xba\PYGZbs{}x7f\PYGZbs{}x19@\PYGZbs{}xc4\PYGZbs{}xbf\PYGZbs{}xe8@y\PYGZbs{}x0b\PYGZbs{}xb3E\PYGZbs{}xc0\PYGZbs{}x08\PYGZbs{}x03D\PYGZbs{}xd5\PYGZbs{}xednE\PYGZbs{}xff\PYGZbs{}x1a\PYGZbs{}x10\PYGZbs{}xbd\PYGZbs{}xb4\PYGZsq{}...
0x00003168      1010     0  b\PYGZsq{}\PYGZbs{}x00\PYGZbs{}x00E\PYGZbs{}xfe\PYGZbs{}xb6\PYGZbs{}x16\PYGZbs{}xbe\PYGZbs{}x12\PYGZbs{}xde\PYGZbs{}xf0@\PYGZbs{}xcbl\PYGZbs{}xe7@zF\PYGZbs{}xc2Ew\PYGZbs{}xba/D\PYGZbs{}xd0\PYGZbs{}xca\PYGZbs{}xd9E\PYGZbs{}xfe\PYGZbs{}xb3\PYGZbs{}xa6\PYGZbs{}xbe\PYGZbs{}x17\PYGZsq{}...
0x0000355E      1010     0  b\PYGZsq{}\PYGZbs{}x00\PYGZbs{}x00E\PYGZbs{}xfeO\PYGZbs{}xac\PYGZbs{}xbe40\PYGZbs{}x85@\PYGZbs{}xcc4\PYGZbs{}x8d@of\PYGZbs{}xd9E\PYGZbs{}xc1F\PYGZbs{}xd8D\PYGZbs{}xd4\PYGZbs{}xaa+E\PYGZbs{}xfeM\PYGZlt{}\PYGZbs{}xbe6\PYGZsq{}...
0x00003954      1010     0  b\PYGZdq{}\PYGZbs{}x00\PYGZbs{}x00E\PYGZbs{}xfd\PYGZbs{}xe9C\PYGZbs{}xbd\PYGZbs{}xb2\PYGZbs{}x19\PYGZbs{}xf0\PYGZbs{}xba\PYGZbs{}x83\PYGZbs{}x18\PYGZbs{}x00AK\PYGZsq{}\PYGZpc{}D\PYGZbs{}xed\PYGZbs{}x9d\PYGZbs{}xdbD\PYGZbs{}xd0\PYGZbs{}x17RE\PYGZbs{}xfd\PYGZbs{}xe6\PYGZbs{}xd3\PYGZbs{}xbd\PYGZbs{}xab\PYGZdq{}...
0x00003D4A      1010     0  b\PYGZsq{}\PYGZbs{}x00\PYGZbs{}x00E\PYGZbs{}xfd\PYGZbs{}x82\PYGZbs{}xd9\PYGZbs{}xba\PYGZbs{}x83\PYGZbs{}x18\PYGZbs{}x00\PYGZbs{}xba\PYGZbs{}x83\PYGZbs{}x18\PYGZbs{}x00\PYGZbs{}xba\PYGZbs{}x83\PYGZbs{}x18\PYGZbs{}x00\PYGZbs{}xba\PYGZbs{}x83\PYGZbs{}x18\PYGZbs{}x00D\PYGZbs{}xd0\PYGZbs{}xad\PYGZbs{}xf3E\PYGZbs{}xfd\PYGZbs{}x80i\PYGZbs{}xba\PYGZbs{}x83\PYGZsq{}...
0x00004140       938     0  b\PYGZsq{}\PYGZbs{}x00\PYGZbs{}x00E\PYGZbs{}xfd\PYGZbs{}x1co\PYGZbs{}xba\PYGZbs{}x83\PYGZbs{}x18\PYGZbs{}x00\PYGZbs{}xba\PYGZbs{}x83\PYGZbs{}x18\PYGZbs{}x00\PYGZbs{}xba\PYGZbs{}x83\PYGZbs{}x18\PYGZbs{}x00\PYGZbs{}xba\PYGZbs{}x83\PYGZbs{}x18\PYGZbs{}x00D\PYGZbs{}xd8\PYGZbs{}x8c\PYGZbs{}xb5E\PYGZbs{}xfd\PYGZbs{}x19\PYGZbs{}xff\PYGZbs{}xba\PYGZbs{}x83\PYGZsq{}...
0x000044EE        58     0  b\PYGZsq{}\PYGZbs{}x81\PYGZbs{}x00RUN1R .S01\PYGZbs{}x00\PYGZbs{}x00DAT2TF            \PYGZsq{}...
Histogram of Logical Data lengths:
Bytes
   58 [1] \textbar{} +++++++++++
  300 [1] \textbar{} +++++++++++
  938 [1] \textbar{} +++++++++++
 1010 [8] \textbar{} ++++++++++++++++++++++++++++++++++++++++++++++++++++++++++++++++++++++++++++++++++++++++
 8190 [1] \textbar{} +++++++++++
Histogram of Logical Record types:
  0 [9] \textbar{} ++++++++++++++++++++++++++++++++++++++++++++++++++++++++++++++++++++++++++++++++++++++++++
 34 [1] \textbar{} ++++++++++
 64 [1] \textbar{} ++++++++++
128 [1] \textbar{} ++++++++++
CPU time =    0.001 (S)
Bye, bye!
\end{sphinxVerbatim}

First \sphinxcode{tdlisscanlogidata} echo’s the command line then it scans the file an writes out a table that has the following columns:


\begin{savenotes}\sphinxattablestart
\centering
\begin{tabulary}{\linewidth}[t]{|T|T|}
\hline
\sphinxstylethead{\sphinxstyletheadfamily 
Heading
\unskip}\relax &\sphinxstylethead{\sphinxstyletheadfamily 
Description
\unskip}\relax \\
\hline
\sphinxcode{Offset}
&
The file position of the start of the Physical Record as a hex integer.
\\
\hline
\sphinxcode{Length}
&
The length of the Logical Record as a decimal integer.
\\
\hline
\sphinxcode{Type}
&
The Logical Record type as a decimal integer.
\\
\hline
\sphinxcode{Logical Data}
&
The logical data payload. Only the first 32 bytes are shown. \sphinxcode{...} is shown if the payload is longer than 32 bytes. If the verbose or dump options are given then all bytes are shown.
\\
\hline
\end{tabulary}
\par
\sphinxattableend\end{savenotes}

This is followed by an ASCII histogram of the lengths of all logical data with the following columns:
\begin{enumerate}
\item {} 
The size in bytes.

\item {} 
The frequency count.

\item {} 
A series of \sphinxcode{+} that is proportionate to the frequency count.

\end{enumerate}

This is followed by an ASCII histogram of the lengths of all Logical Record types with the following columns:
\begin{enumerate}
\item {} 
The size in bytes.

\item {} 
The frequency count.

\item {} 
A series of \sphinxcode{+} that is proportionate to the frequency count.

\end{enumerate}

Using the -d option expands the output when the file position value matches. So given the above then adding \sphinxcode{-d 0x44EE} changes this:

\begin{sphinxVerbatim}[commandchars=\\\{\}]
\PYG{o}{.}\PYG{o}{.}\PYG{o}{.}
\PYG{l+m+mh}{0x000044EE}        \PYG{l+m+mi}{58}     \PYG{l+m+mi}{0}  \PYG{l+s+sa}{b}\PYG{l+s+s1}{\PYGZsq{}}\PYG{l+s+se}{\PYGZbs{}x81}\PYG{l+s+se}{\PYGZbs{}x00}\PYG{l+s+s1}{RUN1R .S01}\PYG{l+s+se}{\PYGZbs{}x00}\PYG{l+s+se}{\PYGZbs{}x00}\PYG{l+s+s1}{DAT2TF            }\PYG{l+s+s1}{\PYGZsq{}}\PYG{o}{.}\PYG{o}{.}\PYG{o}{.}
\PYG{o}{.}\PYG{o}{.}\PYG{o}{.}
\end{sphinxVerbatim}

To this:

\begin{sphinxVerbatim}[commandchars=\\\{\}]
\PYG{o}{.}\PYG{o}{.}\PYG{o}{.}
\PYG{l+m+mh}{0x000044EE}        \PYG{l+m+mi}{58}     \PYG{l+m+mi}{0}  \PYG{l+s+sa}{b}\PYG{l+s+s1}{\PYGZsq{}}\PYG{l+s+se}{\PYGZbs{}x81}\PYG{l+s+se}{\PYGZbs{}x00}\PYG{l+s+s1}{RUN1R .S01}\PYG{l+s+se}{\PYGZbs{}x00}\PYG{l+s+se}{\PYGZbs{}x00}\PYG{l+s+s1}{DAT2TF                }\PYG{l+s+se}{\PYGZbs{}x00}\PYG{l+s+s1}{ 1024}\PYG{l+s+se}{\PYGZbs{}x00}\PYG{l+s+se}{\PYGZbs{}x00}\PYG{l+s+se}{\PYGZbs{}x00}\PYG{l+s+se}{\PYGZbs{}x00}\PYG{l+s+se}{\PYGZbs{}x00}\PYG{l+s+se}{\PYGZbs{}x00}\PYG{l+s+se}{\PYGZbs{}x00}\PYG{l+s+se}{\PYGZbs{}x00}\PYG{l+s+se}{\PYGZbs{}x00}\PYG{l+s+se}{\PYGZbs{}x00}\PYG{l+s+se}{\PYGZbs{}x00}\PYG{l+s+se}{\PYGZbs{}x00}\PYG{l+s+se}{\PYGZbs{}x00}\PYG{l+s+se}{\PYGZbs{}x00}\PYG{l+s+se}{\PYGZbs{}x00}\PYG{l+s+se}{\PYGZbs{}x00}\PYG{l+s+s1}{\PYGZsq{}}
\PYG{o}{.}\PYG{o}{.}\PYG{o}{.}
\end{sphinxVerbatim}


\subsubsection{Extracting Data from LIS}
\label{\detokenize{cmd_line_tools/LIS_cmd_line_tools:extracting-data-from-lis}}

\paragraph{\sphinxstyleliteralintitle{tdlisdumpframeset}}
\label{\detokenize{cmd_line_tools/LIS_cmd_line_tools:tdlisdumpframeset}}
Reads a LIS file and writes out tab separated values of each frame.


\subparagraph{Arguments}
\label{\detokenize{cmd_line_tools/LIS_cmd_line_tools:id7}}\begin{enumerate}
\item {} 
The path to the LIS file.

\end{enumerate}


\subparagraph{Options}
\label{\detokenize{cmd_line_tools/LIS_cmd_line_tools:id8}}

\begin{savenotes}\sphinxattablestart
\centering
\begin{tabulary}{\linewidth}[t]{|T|T|}
\hline
\sphinxstylethead{\sphinxstyletheadfamily 
Option
\unskip}\relax &\sphinxstylethead{\sphinxstyletheadfamily 
Description
\unskip}\relax \\
\hline
\sphinxcode{-{-}version}
&
Show program’s version number and exit
\\
\hline
\sphinxcode{-h, -{-}help}
&
Show this help message and exit.
\\
\hline
\sphinxcode{-k, -{-}keep-going}
&
Keep going as far as sensible. {[}default: False{]}
\\
\hline
\sphinxcode{-l LOGLEVEL, -{-}loglevel=LOGLEVEL}
&
Log Level (debug=10, info=20, warning=30, error=40, critical=50) {[}default: 20{]}
\\
\hline
\sphinxcode{-s, -{-}summary}
&
Display summary only {[}default: False{]}.
\\
\hline
\end{tabulary}
\par
\sphinxattableend\end{savenotes}


\subparagraph{Examples}
\label{\detokenize{cmd_line_tools/LIS_cmd_line_tools:id9}}
\begin{sphinxVerbatim}[commandchars=\\\{\}]
\PYGZdl{} {}`{}`tdlisdumpframeset{}`{}` LIS.lis
Cmd: DumpFrameSet.py LIS.lis
2012\PYGZhy{}02\PYGZhy{}09 08:41:38,372 INFO     Index.indexFile(): LIS.lis
\PYGZlt{}TotalDepth.LIS.core.LogPass.LogPass object at 0x101a0c510\PYGZgt{}
b\PYGZsq{}DEPT\PYGZsq{} [b\PYGZsq{}M   \PYGZsq{}]   b\PYGZsq{}SP  \PYGZsq{} [b\PYGZsq{}MV  \PYGZsq{}]   b\PYGZsq{}SN  \PYGZsq{} [b\PYGZsq{}OHMM\PYGZsq{}]   b\PYGZsq{}ILD \PYGZsq{} [b\PYGZsq{}OHMM\PYGZsq{}]   b\PYGZsq{}CILD\PYGZsq{} [b\PYGZsq{}MMHO\PYGZsq{}]   b\PYGZsq{}DT  \PYGZsq{} [b\PYGZsq{}US/M\PYGZsq{}]
2052.98 \PYGZhy{}4.54908    1.34538 1.26348 386.599 \PYGZhy{}999.25
2052.83 \PYGZhy{}5.1372     1.36062 1.29521 500.511 \PYGZhy{}999.25
2052.68 \PYGZhy{}6.66747    1.38543 1.45786 595.623 \PYGZhy{}999.25
2052.53 \PYGZhy{}6.69616    1.43226 1.61085 592.447 \PYGZhy{}999.25
2052.37 \PYGZhy{}4.93782    1.51647 1.6622  590.846 \PYGZhy{}999.25
2052.22 \PYGZhy{}4.38823    1.66883 1.70584 586.092 \PYGZhy{}999.25
2052.07 \PYGZhy{}4.70347    1.8102  1.70607 577.873 \PYGZhy{}999.25
...
1996.44 \PYGZhy{}999.25     \PYGZhy{}999.25 \PYGZhy{}999.25 \PYGZhy{}999.25 \PYGZhy{}999.25
1996.29 \PYGZhy{}999.25     \PYGZhy{}999.25 \PYGZhy{}999.25 \PYGZhy{}999.25 \PYGZhy{}999.25
1996.14 \PYGZhy{}999.25     \PYGZhy{}999.25 \PYGZhy{}999.25 \PYGZhy{}999.25 \PYGZhy{}999.25
1995.99 \PYGZhy{}999.25     \PYGZhy{}999.25 \PYGZhy{}999.25 \PYGZhy{}999.25 \PYGZhy{}999.25

Sc Name          Count      Min     Mean      Max Std Dev.       \PYGZhy{}\PYGZhy{}       ==       ++     Bias    Drift Activity
DEPT [M   ]        375    2e+03 2.02e+03 2.05e+03     16.5      374        0        0        1   \PYGZhy{}0.152 0.000144
SP   [MV  ]        262    \PYGZhy{}13.7    \PYGZhy{}5.67   \PYGZhy{}0.769     2.66      124        0      137  \PYGZhy{}0.0498   0.0144    0.678
SN   [OHMM]        252    0.866     1.36     1.98    0.277      123        0      128  \PYGZhy{}0.0199 \PYGZhy{}0.000719   0.0425
ILD  [OHMM]        253    0.361     1.31     2.35    0.412       95        0      157   \PYGZhy{}0.246  0.00429    0.134
CILD [MMHO]        253      387      787 1.75e+03      236      130        0      122   0.0317    0.205    0.101
DT   [US/M]        292      133      320      460     42.5      139        0      152  \PYGZhy{}0.0447   \PYGZhy{}0.451    0.106
CPU time =    0.047 (S)
Bye, bye!
\end{sphinxVerbatim}

The summary table at the end has the following columns:


\begin{savenotes}\sphinxattablestart
\centering
\begin{tabulary}{\linewidth}[t]{|T|T|}
\hline
\sphinxstylethead{\sphinxstyletheadfamily 
Heading
\unskip}\relax &\sphinxstylethead{\sphinxstyletheadfamily 
Description
\unskip}\relax \\
\hline
\sphinxcode{Sc Name}
&
The sub-channel name and units of measure.
\\
\hline
\sphinxcode{Count}
&
The number of non-null values.
\\
\hline
\sphinxcode{Min}
&
Minimum value.
\\
\hline
\sphinxcode{Mean}
&
Arithmetic mean of values.
\\
\hline
\sphinxcode{Max}
&
Maximum value.
\\
\hline
\sphinxcode{Std Dev.}
&
Standard deviation of values.
\\
\hline
\sphinxcode{-{-}}
&
Number of values that are a decrease over the previous value.
\\
\hline
\sphinxcode{==}
&
Number of values that are equal to the previous value.
\\
\hline
\sphinxcode{++}
&
Number of values that are an increase over the previous value.
\\
\hline
\sphinxcode{Bias}
&
(\sphinxcode{-{-}} - \sphinxcode{++}) / total
\\
\hline
\sphinxcode{Drift}
&
(last value - first value) / number of values
\\
\hline
\sphinxcode{Activity}
&
The RMS exponent change.
\\
\hline
\end{tabulary}
\par
\sphinxattableend\end{savenotes}


\paragraph{\sphinxstyleliteralintitle{tdlistohtml}}
\label{\detokenize{cmd_line_tools/LIS_cmd_line_tools:tdlistohtml}}\label{\detokenize{cmd_line_tools/LIS_cmd_line_tools:totaldepth-liscmdline-listohtml}}
Generates HTML from input LIS file or directory to an output destination.


\subparagraph{Arguments}
\label{\detokenize{cmd_line_tools/LIS_cmd_line_tools:id10}}\begin{enumerate}
\item {} 
The path to the input LIS file or directory.

\item {} 
The path to the output file or directory, any directories will be created as necessary.

\end{enumerate}


\subparagraph{Options}
\label{\detokenize{cmd_line_tools/LIS_cmd_line_tools:id11}}

\begin{savenotes}\sphinxattablestart
\centering
\begin{tabulary}{\linewidth}[t]{|T|T|}
\hline
\sphinxstylethead{\sphinxstyletheadfamily 
Option
\unskip}\relax &\sphinxstylethead{\sphinxstyletheadfamily 
Description
\unskip}\relax \\
\hline
\sphinxcode{-{-}version}
&
Show program’s version number and exit
\\
\hline
\sphinxcode{-h, -{-}help}
&
Show this help message and exit.
\\
\hline
\sphinxcode{-g, -{-}glob}
&
File pattern match. {[}default none{]}
\\
\hline
\sphinxcode{-j JOBS, -{-}jobs=JOBS}
&
Max processes when multiprocessing. Zero uses number of native CPUs {[}8{]}.
-1 disables multiprocessing. {[}default: -1{]}
\\
\hline
\sphinxcode{-k, -{-}keep-going}
&
Keep going as far as sensible. {[}default: False{]}
\\
\hline
\sphinxcode{-l LOGLEVEL, -{-}loglevel=LOGLEVEL}
&
Log Level (debug=10, info=20, warning=30, error=40, critical=50) {[}default: 20{]}
\\
\hline
\sphinxcode{-v, -{-}verbose}
&
Verbose output, this outputs a representation of table data and DFSRs.
\\
\hline
\sphinxcode{-r, -{-}recursive}
&
Process input recursively. {[}default: False{]}
\\
\hline
\end{tabulary}
\par
\sphinxattableend\end{savenotes}


\subparagraph{Examples}
\label{\detokenize{cmd_line_tools/LIS_cmd_line_tools:id12}}
Command to process a directory of LIS:

\begin{sphinxVerbatim}[commandchars=\\\{\}]
\PYGZdl{} {}`{}`tdlistohtml{}`{}` Simple LIS\PYGZus{}plot/Simple\PYGZus{}00
\end{sphinxVerbatim}

Output:

\begin{sphinxVerbatim}[commandchars=\\\{\}]
Cmd: {}`{}`tdlistohtml{}`{}` Simple LIS\PYGZus{}plot/Simple\PYGZus{}00
plotLogInfo:
FileInfo: \PYGZdq{}Simple/LIS.lis\PYGZdq{} \PYGZhy{}\PYGZgt{} \PYGZdq{}LIS\PYGZus{}plot/Simple\PYGZus{}00/LIS.lis.html\PYGZdq{} 17 (kb) LR count=4 t=0.070
FileInfo: \PYGZdq{}Simple/RW.lis\PYGZdq{} \PYGZhy{}\PYGZgt{} \PYGZdq{}LIS\PYGZus{}plot/Simple\PYGZus{}00/RW.lis.html\PYGZdq{} 843 (kb) LR count=12 t=3.206
FileInfo: \PYGZdq{}Simple/RW\PYGZus{}No\PYGZus{}TIF.lis\PYGZdq{} \PYGZhy{}\PYGZgt{} \PYGZdq{}LIS\PYGZus{}plot/Simple\PYGZus{}00/RW\PYGZus{}No\PYGZus{}TIF.lis.html\PYGZdq{} 833 (kb) LR count=12 t=3.200
  CPU time =    6.568 (S)
Exec. time =    6.568 (S)
Bye, bye!
\end{sphinxVerbatim}

For each file the output lists:
\begin{itemize}
\item {} 
Input file.

\item {} 
Output HTML file.

\item {} 
File size.

\item {} 
Count of Logical Records.

\item {} 
Execution time.

\end{itemize}

In the output directory there will be an index.html file, for example:

\noindent\sphinxincludegraphics{{LisToHtml_index}.png}

The columns are:
\begin{itemize}
\item {} 
The name of the LIS file.

\item {} 
The size of the LIS file.

\item {} 
Count of Logical Records.

\item {} 
Execution time.

\item {} 
Processing rate.

\end{itemize}

In the linked HTML file is a summary of the content of the LIS file.

The Log Pass merits several entries, the first summarises the frame shape and the shape of each channel, for example:

\noindent\sphinxincludegraphics{{LisToHtml_LogPass_00}.png}

Then there is a couple of tables, the first summarises the X axis and the second summarises each channel (min, max mean etc.), for example:

\noindent\sphinxincludegraphics{{LisToHtml_LogPass_01}.png}


\paragraph{\sphinxstyleliteralintitle{tdlistablehistogram}}
\label{\detokenize{cmd_line_tools/LIS_cmd_line_tools:tdlistablehistogram}}
Provides a count of elements in LIS tables.


\subparagraph{Arguments}
\label{\detokenize{cmd_line_tools/LIS_cmd_line_tools:id13}}\begin{enumerate}
\item {} 
A path to a LIS file or directory of LIS files.

\end{enumerate}


\subparagraph{Options}
\label{\detokenize{cmd_line_tools/LIS_cmd_line_tools:id14}}

\begin{savenotes}\sphinxattablestart
\centering
\begin{tabulary}{\linewidth}[t]{|T|T|}
\hline
\sphinxstylethead{\sphinxstyletheadfamily 
Option
\unskip}\relax &\sphinxstylethead{\sphinxstyletheadfamily 
Description
\unskip}\relax \\
\hline
\sphinxcode{-{-}version}
&
Show program’s version number and exit
\\
\hline
\sphinxcode{-h, -{-}help}
&
Show this help message and exit.
\\
\hline
\sphinxcode{-k, -{-}keep-going}
&
Keep going as far as sensible. {[}default: False{]}
\\
\hline
\sphinxcode{-r, -{-}recursive}
&
Process input recursively. {[}default: False{]}
\\
\hline
\sphinxcode{-s, -{-}structure}
&
Display table structure (row/col range). {[}default: False{]}
\\
\hline
\sphinxcode{-{-}type=LRTYPE}
&
Logical record table type e.g. 34. {[}default: 34{]}
\\
\hline
\sphinxcode{-{-}name=NAME}
&
Logical record table name e.g. PRES. {[}default: {]}
\\
\hline
\sphinxcode{-{-}row=ROW}
&
Logical record table row e.g. “GR  “. {[}default: {]}
\\
\hline
\sphinxcode{-{-}col=COL}
&
Logical record table column e.g. “LEDG”. {[}default: {]}
\\
\hline
\sphinxcode{-l LOGLEVEL, -{-}loglevel=LOGLEVEL}
&
Log Level (debug=10, info=20, warning=30, error=40, critical=50) {[}default: 20{]}
\\
\hline
\end{tabulary}
\par
\sphinxattableend\end{savenotes}


\subparagraph{Examples}
\label{\detokenize{cmd_line_tools/LIS_cmd_line_tools:id15}}
Count of all entries regardless of the table/row/column that they appear in:

\begin{sphinxVerbatim}[commandchars=\\\{\}]
\PYGZdl{} {}`{}`tdlistablehistogram{}`{}` \PYGZhy{}l 40 Simple/
Cmd: TableHistogram.py \PYGZhy{}l 40 Simple/
======================== Count of all table entries =======================
\PYGZob{}\PYGZdq{}(34, b\PYGZsq{}    \PYGZsq{})\PYGZdq{}: 1414,
 \PYGZdq{}(34, b\PYGZsq{}0.445\PYGZsq{})\PYGZdq{}: 5,
 \PYGZdq{}(34, b\PYGZsq{}0.621\PYGZsq{})\PYGZdq{}: 5,
 \PYGZdq{}(34, b\PYGZsq{}013529700231\PYGZsq{})\PYGZdq{}: 7,
 \PYGZdq{}(34, b\PYGZsq{}1\PYGZsq{})\PYGZdq{}: 5,
 \PYGZdq{}(34, b\PYGZsq{}1.22\PYGZsq{})\PYGZdq{}: 5,
 \PYGZdq{}(34, b\PYGZsq{}1.70\PYGZsq{})\PYGZdq{}: 2,
 \PYGZdq{}(34, b\PYGZsq{}116\PYGZsq{})\PYGZdq{}: 2,
 \PYGZdq{}(34, b\PYGZsq{}12.25\PYGZsq{})\PYGZdq{}: 2,
 \PYGZdq{}(34, b\PYGZsq{}15\PYGZsq{})\PYGZdq{}: 5,
 \PYGZdq{}(34, b\PYGZsq{}15\PYGZhy{}4\PYGZhy{}76\PYGZsq{})\PYGZdq{}: 5,
 \PYGZdq{}(34, b\PYGZsq{}17\PYGZsq{})\PYGZdq{}: 5,
 \PYGZdq{}(34, b\PYGZsq{}17.5\PYGZsq{})\PYGZdq{}: 5,
 \PYGZdq{}(34, b\PYGZsq{}19\PYGZsq{})\PYGZdq{}: 5,
 \PYGZdq{}(34, b\PYGZsq{}1976\PYGZsq{})\PYGZdq{}: 7,
 \PYGZdq{}(34, b\PYGZsq{}2\PYGZsq{})\PYGZdq{}: 2,
 \PYGZdq{}(34, b\PYGZsq{}2055.0\PYGZsq{})\PYGZdq{}: 2,
 \PYGZdq{}(34, b\PYGZsq{}2071.2\PYGZsq{})\PYGZdq{}: 4,
 \PYGZdq{}(34, b\PYGZsq{}25\PYGZsq{})\PYGZdq{}: 2,
 \PYGZdq{}(34, b\PYGZsq{}25\PYGZhy{}6\PYGZhy{}76\PYGZsq{})\PYGZdq{}: 2,
 \PYGZdq{}(34, b\PYGZsq{}257.0\PYGZsq{})\PYGZdq{}: 7,
...
 \PYGZdq{}(34, b\PYGZsq{}WN  \PYGZsq{})\PYGZdq{}: 7,
 \PYGZdq{}(34, b\PYGZsq{}YEAR\PYGZsq{})\PYGZdq{}: 7,
 \PYGZsq{}(34,)\PYGZsq{}: 443\PYGZcb{}
====================== Count of all table entries END =====================
CPU time =    0.205 (S)
Bye, bye!
\end{sphinxVerbatim}

The result is a dictionary that has the key as a pair \sphinxcode{(lr\_type, cell\_value)} and the value as a count of the number of occurrences.

If the \sphinxcode{-s} option is used then an additional summary is provided:

\begin{sphinxVerbatim}[commandchars=\\\{\}]
\PYG{o}{==}\PYG{o}{==}\PYG{o}{==}\PYG{o}{==}\PYG{o}{==}\PYG{o}{==}\PYG{o}{==}\PYG{o}{==}\PYG{o}{==}\PYG{o}{==}\PYG{o}{==}\PYG{o}{==}\PYG{o}{==}\PYG{o}{==}\PYG{o}{==}\PYG{o}{=} \PYG{n}{Row} \PYG{n}{entries} \PYG{o}{==}\PYG{o}{==}\PYG{o}{==}\PYG{o}{==}\PYG{o}{==}\PYG{o}{==}\PYG{o}{==}\PYG{o}{==}\PYG{o}{==}\PYG{o}{==}\PYG{o}{==}\PYG{o}{==}\PYG{o}{==}\PYG{o}{==}\PYG{o}{==}\PYG{o}{=}
\PYG{p}{\PYGZob{}}\PYG{p}{(}\PYG{l+m+mi}{34}\PYG{p}{,} \PYG{l+s+sa}{b}\PYG{l+s+s1}{\PYGZsq{}}\PYG{l+s+s1}{CONS}\PYG{l+s+s1}{\PYGZsq{}}\PYG{p}{,} \PYG{l+s+sa}{b}\PYG{l+s+s1}{\PYGZsq{}}\PYG{l+s+s1}{APIN}\PYG{l+s+s1}{\PYGZsq{}}\PYG{p}{)}\PYG{p}{:} \PYG{l+m+mi}{7}\PYG{p}{,}
 \PYG{p}{(}\PYG{l+m+mi}{34}\PYG{p}{,} \PYG{l+s+sa}{b}\PYG{l+s+s1}{\PYGZsq{}}\PYG{l+s+s1}{CONS}\PYG{l+s+s1}{\PYGZsq{}}\PYG{p}{,} \PYG{l+s+sa}{b}\PYG{l+s+s1}{\PYGZsq{}}\PYG{l+s+s1}{BLI }\PYG{l+s+s1}{\PYGZsq{}}\PYG{p}{)}\PYG{p}{:} \PYG{l+m+mi}{7}\PYG{p}{,}
 \PYG{p}{(}\PYG{l+m+mi}{34}\PYG{p}{,} \PYG{l+s+sa}{b}\PYG{l+s+s1}{\PYGZsq{}}\PYG{l+s+s1}{CONS}\PYG{l+s+s1}{\PYGZsq{}}\PYG{p}{,} \PYG{l+s+sa}{b}\PYG{l+s+s1}{\PYGZsq{}}\PYG{l+s+s1}{BS1 }\PYG{l+s+s1}{\PYGZsq{}}\PYG{p}{)}\PYG{p}{:} \PYG{l+m+mi}{7}\PYG{p}{,}
 \PYG{p}{(}\PYG{l+m+mi}{34}\PYG{p}{,} \PYG{l+s+sa}{b}\PYG{l+s+s1}{\PYGZsq{}}\PYG{l+s+s1}{CONS}\PYG{l+s+s1}{\PYGZsq{}}\PYG{p}{,} \PYG{l+s+sa}{b}\PYG{l+s+s1}{\PYGZsq{}}\PYG{l+s+s1}{BS2 }\PYG{l+s+s1}{\PYGZsq{}}\PYG{p}{)}\PYG{p}{:} \PYG{l+m+mi}{7}\PYG{p}{,}
 \PYG{p}{(}\PYG{l+m+mi}{34}\PYG{p}{,} \PYG{l+s+sa}{b}\PYG{l+s+s1}{\PYGZsq{}}\PYG{l+s+s1}{CONS}\PYG{l+s+s1}{\PYGZsq{}}\PYG{p}{,} \PYG{l+s+sa}{b}\PYG{l+s+s1}{\PYGZsq{}}\PYG{l+s+s1}{BS3 }\PYG{l+s+s1}{\PYGZsq{}}\PYG{p}{)}\PYG{p}{:} \PYG{l+m+mi}{7}\PYG{p}{,}
\PYG{o}{.}\PYG{o}{.}\PYG{o}{.}
 \PYG{p}{(}\PYG{l+m+mi}{34}\PYG{p}{,} \PYG{l+s+sa}{b}\PYG{l+s+s1}{\PYGZsq{}}\PYG{l+s+s1}{CONS}\PYG{l+s+s1}{\PYGZsq{}}\PYG{p}{,} \PYG{l+s+sa}{b}\PYG{l+s+s1}{\PYGZsq{}}\PYG{l+s+s1}{WN  }\PYG{l+s+s1}{\PYGZsq{}}\PYG{p}{)}\PYG{p}{:} \PYG{l+m+mi}{7}\PYG{p}{,}
 \PYG{p}{(}\PYG{l+m+mi}{34}\PYG{p}{,} \PYG{l+s+sa}{b}\PYG{l+s+s1}{\PYGZsq{}}\PYG{l+s+s1}{CONS}\PYG{l+s+s1}{\PYGZsq{}}\PYG{p}{,} \PYG{l+s+sa}{b}\PYG{l+s+s1}{\PYGZsq{}}\PYG{l+s+s1}{YEAR}\PYG{l+s+s1}{\PYGZsq{}}\PYG{p}{)}\PYG{p}{:} \PYG{l+m+mi}{7}\PYG{p}{\PYGZcb{}}
\PYG{o}{==}\PYG{o}{==}\PYG{o}{==}\PYG{o}{==}\PYG{o}{==}\PYG{o}{==}\PYG{o}{==}\PYG{o}{==}\PYG{o}{==}\PYG{o}{==}\PYG{o}{==}\PYG{o}{==}\PYG{o}{==}\PYG{o}{==}\PYG{o}{=} \PYG{n}{Row} \PYG{n}{entries} \PYG{n}{END} \PYG{o}{==}\PYG{o}{==}\PYG{o}{==}\PYG{o}{==}\PYG{o}{==}\PYG{o}{==}\PYG{o}{==}\PYG{o}{==}\PYG{o}{==}\PYG{o}{==}\PYG{o}{==}\PYG{o}{==}\PYG{o}{==}\PYG{o}{==}\PYG{o}{=}
\PYG{o}{==}\PYG{o}{==}\PYG{o}{==}\PYG{o}{==}\PYG{o}{==}\PYG{o}{==}\PYG{o}{==}\PYG{o}{==}\PYG{o}{==}\PYG{o}{==}\PYG{o}{==}\PYG{o}{==}\PYG{o}{==}\PYG{o}{==}\PYG{o}{==} \PYG{n}{Column} \PYG{n}{entries} \PYG{o}{==}\PYG{o}{==}\PYG{o}{==}\PYG{o}{==}\PYG{o}{==}\PYG{o}{==}\PYG{o}{==}\PYG{o}{==}\PYG{o}{==}\PYG{o}{==}\PYG{o}{==}\PYG{o}{==}\PYG{o}{==}\PYG{o}{==}\PYG{o}{=}
\PYG{p}{\PYGZob{}}\PYG{p}{(}\PYG{l+m+mi}{34}\PYG{p}{,} \PYG{l+s+sa}{b}\PYG{l+s+s1}{\PYGZsq{}}\PYG{l+s+s1}{CONS}\PYG{l+s+s1}{\PYGZsq{}}\PYG{p}{,} \PYG{l+s+sa}{b}\PYG{l+s+s1}{\PYGZsq{}}\PYG{l+s+s1}{ALLO}\PYG{l+s+s1}{\PYGZsq{}}\PYG{p}{)}\PYG{p}{:} \PYG{l+m+mi}{707}\PYG{p}{,}
 \PYG{p}{(}\PYG{l+m+mi}{34}\PYG{p}{,} \PYG{l+s+sa}{b}\PYG{l+s+s1}{\PYGZsq{}}\PYG{l+s+s1}{CONS}\PYG{l+s+s1}{\PYGZsq{}}\PYG{p}{,} \PYG{l+s+sa}{b}\PYG{l+s+s1}{\PYGZsq{}}\PYG{l+s+s1}{MNEM}\PYG{l+s+s1}{\PYGZsq{}}\PYG{p}{)}\PYG{p}{:} \PYG{l+m+mi}{707}\PYG{p}{,}
 \PYG{p}{(}\PYG{l+m+mi}{34}\PYG{p}{,} \PYG{l+s+sa}{b}\PYG{l+s+s1}{\PYGZsq{}}\PYG{l+s+s1}{CONS}\PYG{l+s+s1}{\PYGZsq{}}\PYG{p}{,} \PYG{l+s+sa}{b}\PYG{l+s+s1}{\PYGZsq{}}\PYG{l+s+s1}{PUNI}\PYG{l+s+s1}{\PYGZsq{}}\PYG{p}{)}\PYG{p}{:} \PYG{l+m+mi}{707}\PYG{p}{,}
 \PYG{p}{(}\PYG{l+m+mi}{34}\PYG{p}{,} \PYG{l+s+sa}{b}\PYG{l+s+s1}{\PYGZsq{}}\PYG{l+s+s1}{CONS}\PYG{l+s+s1}{\PYGZsq{}}\PYG{p}{,} \PYG{l+s+sa}{b}\PYG{l+s+s1}{\PYGZsq{}}\PYG{l+s+s1}{TUNI}\PYG{l+s+s1}{\PYGZsq{}}\PYG{p}{)}\PYG{p}{:} \PYG{l+m+mi}{707}\PYG{p}{,}
 \PYG{p}{(}\PYG{l+m+mi}{34}\PYG{p}{,} \PYG{l+s+sa}{b}\PYG{l+s+s1}{\PYGZsq{}}\PYG{l+s+s1}{CONS}\PYG{l+s+s1}{\PYGZsq{}}\PYG{p}{,} \PYG{l+s+sa}{b}\PYG{l+s+s1}{\PYGZsq{}}\PYG{l+s+s1}{VALU}\PYG{l+s+s1}{\PYGZsq{}}\PYG{p}{)}\PYG{p}{:} \PYG{l+m+mi}{707}\PYG{p}{\PYGZcb{}}
\PYG{o}{==}\PYG{o}{==}\PYG{o}{==}\PYG{o}{==}\PYG{o}{==}\PYG{o}{==}\PYG{o}{==}\PYG{o}{==}\PYG{o}{==}\PYG{o}{==}\PYG{o}{==}\PYG{o}{==}\PYG{o}{==}\PYG{o}{==} \PYG{n}{Column} \PYG{n}{entries} \PYG{n}{END} \PYG{o}{==}\PYG{o}{==}\PYG{o}{==}\PYG{o}{==}\PYG{o}{==}\PYG{o}{==}\PYG{o}{==}\PYG{o}{==}\PYG{o}{==}\PYG{o}{==}\PYG{o}{==}\PYG{o}{==}\PYG{o}{==}\PYG{o}{=}
\end{sphinxVerbatim}

This are dictionaries that have the key as a tripple \sphinxcode{(lr\_type, table\_name, row\_name)} and \sphinxcode{(lr\_type, table\_name, column\_name)}
respectively and the value as a count of the number of occurrences.

Filtering by Logical Record type, table name, row name and column name (note quoting of spaces):

\begin{sphinxVerbatim}[commandchars=\\\{\}]
\PYGZdl{} {}`{}`tdlistablehistogram{}`{}` \PYGZhy{}l 40 \PYGZhy{}\PYGZhy{}type=34 \PYGZhy{}\PYGZhy{}name=CONS \PYGZhy{}\PYGZhy{}row=\PYGZdq{}WN  \PYGZdq{} \PYGZhy{}\PYGZhy{}col=VALU Simple/
Cmd: TableHistogram.py \PYGZhy{}l 40 \PYGZhy{}\PYGZhy{}type=34 \PYGZhy{}\PYGZhy{}name=CONS \PYGZhy{}\PYGZhy{}row=WN   \PYGZhy{}\PYGZhy{}col=VALU Simple/
======================== Count of all table entries =======================
\PYGZob{}\PYGZdq{}(34, b\PYGZsq{}CONS\PYGZsq{}, b\PYGZsq{}WN  \PYGZsq{}, b\PYGZsq{}VALU\PYGZsq{}, b\PYGZsq{}B897 \PYGZhy{} 14\PYGZsq{})\PYGZdq{}: 1,
 \PYGZdq{}(34, b\PYGZsq{}CONS\PYGZsq{}, b\PYGZsq{}WN  \PYGZsq{}, b\PYGZsq{}VALU\PYGZsq{}, b\PYGZsq{}DIEKSAND 111A\PYGZsq{})\PYGZdq{}: 3,
 \PYGZdq{}(34, b\PYGZsq{}CONS\PYGZsq{}, b\PYGZsq{}WN  \PYGZsq{}, b\PYGZsq{}VALU\PYGZsq{}, b\PYGZsq{}VOELKERSEN AZ4\PYGZsq{})\PYGZdq{}: 3\PYGZcb{}
====================== Count of all table entries END =====================
CPU time =    0.174 (S)
Bye, bye!
\end{sphinxVerbatim}

The result is a dictionary that has the key as a quadruple \sphinxcode{(lr\_type, table\_name, row\_name, column\_name, cell\_value)} and
the value as a count of the number of occurrences.


\subsubsection{LIS Developer Tools}
\label{\detokenize{cmd_line_tools/LIS_cmd_line_tools:lis-developer-tools}}

\paragraph{\sphinxstyleliteralintitle{tdlisdetif}}
\label{\detokenize{cmd_line_tools/LIS_cmd_line_tools:tdlisdetif}}
Takes an input LIS79 file and writes out a new one without TIF markers.


\subparagraph{Arguments}
\label{\detokenize{cmd_line_tools/LIS_cmd_line_tools:id16}}\begin{enumerate}
\item {} 
The path to the input LIS file.

\item {} 
The path to the output LIS file, any directories will be created as necessary.

\end{enumerate}


\subparagraph{Options}
\label{\detokenize{cmd_line_tools/LIS_cmd_line_tools:id17}}

\begin{savenotes}\sphinxattablestart
\centering
\begin{tabulary}{\linewidth}[t]{|T|T|}
\hline
\sphinxstylethead{\sphinxstyletheadfamily 
Option
\unskip}\relax &\sphinxstylethead{\sphinxstyletheadfamily 
Description
\unskip}\relax \\
\hline
\sphinxcode{-{-}version}
&
Show program’s version number and exit
\\
\hline
\sphinxcode{-h, -{-}help}
&
Show this help message and exit.
\\
\hline
\sphinxcode{-n, -{-}nervous}
&
Nervous mode (do no harm). {[}default: False{]}
\\
\hline
\sphinxcode{-l LOGLEVEL, -{-}loglevel=LOGLEVEL}
&
Log Level (debug=10, info=20, warning=30, error=40, critical=50) {[}default: 20{]}
\\
\hline
\end{tabulary}
\par
\sphinxattableend\end{savenotes}


\subparagraph{Examples}
\label{\detokenize{cmd_line_tools/LIS_cmd_line_tools:id18}}
DeTIF with nervous mode just examines the file:

\begin{sphinxVerbatim}[commandchars=\\\{\}]
\PYGZdl{} {}`{}`tdlisdetif{}`{}` \PYGZhy{}n RW.lis RW\PYGZus{}No\PYGZus{}TIF.lis
Cmd: DeTif.py \PYGZhy{}n RW.lis RW\PYGZus{}No\PYGZus{}TIF.lis
stripTif(): Tell: 0x00000000 Len: 0x00000000 TIF: TIF  True \PYGZgt{}:  0x       0  0x       0  0x       0
stripTif(): Tell: 0x0000000c Len: 0x0000003e TIF: TIF  True \PYGZgt{}:  0x       0  0x       0  0x      4a
stripTif(): Tell: 0x00000056 Len: 0x00000400 TIF: TIF  True \PYGZgt{}:  0x       0  0x       0  0x     456
stripTif(): Tell: 0x00000462 Len: 0x00000400 TIF: TIF  True \PYGZgt{}:  0x       0  0x      4a  0x     862
stripTif(): Tell: 0x0000086e Len: 0x00000400 TIF: TIF  True \PYGZgt{}:  0x       0  0x     456  0x     c6e
stripTif(): Tell: 0x00000c7a Len: 0x00000400 TIF: TIF  True \PYGZgt{}:  0x       0  0x     862  0x    107a
stripTif(): Tell: 0x00001086 Len: 0x00000400 TIF: TIF  True \PYGZgt{}:  0x       0  0x     c6e  0x    1486
stripTif(): Tell: 0x00001492 Len: 0x00000400 TIF: TIF  True \PYGZgt{}:  0x       0  0x    107a  0x    1892
stripTif(): Tell: 0x0000189e Len: 0x00000400 TIF: TIF  True \PYGZgt{}:  0x       0  0x    1486  0x    1c9e
...
stripTif(): Tell: 0x000d2b4e Len: 0x000000f6 TIF: TIF  True \PYGZgt{}:  0x       0  0x   d2748  0x   d2c44
stripTif(): Tell: 0x000d2c50 Len: 0x0000003e TIF: TIF  True \PYGZgt{}:  0x       0  0x   d2b42  0x   d2c8e
  CPU time =    0.022 (S)
Exec. time =    0.022 (S)
Bye, bye!
\end{sphinxVerbatim}

DeTIF with write:

\begin{sphinxVerbatim}[commandchars=\\\{\}]
\PYGZdl{} {}`{}`tdlisdetif{}`{}` RW.lis RW\PYGZus{}No\PYGZus{}TIF.lis
Cmd: DeTif.py RW.lis RW\PYGZus{}No\PYGZus{}TIF.lis
  CPU time =    0.019 (S)
Exec. time =    0.019 (S)
Bye, bye!
\end{sphinxVerbatim}


\paragraph{\sphinxstyleliteralintitle{tdlisindex}}
\label{\detokenize{cmd_line_tools/LIS_cmd_line_tools:tdlisindex}}
This indexes a LIS file and prints out the result. It can also provide some performance measurements of the indexing operation. See {\hyperref[\detokenize{tech/LIS_indexing:totaldepth-tech-indexing}]{\sphinxcrossref{\DUrole{std,std-ref}{Indexing LIS Files}}}} for more information about the design and performance of LIS indexing.


\subparagraph{Arguments}
\label{\detokenize{cmd_line_tools/LIS_cmd_line_tools:id19}}\begin{enumerate}
\item {} 
The path to a LIS file or a directory of LIS files.

\end{enumerate}


\subparagraph{Options}
\label{\detokenize{cmd_line_tools/LIS_cmd_line_tools:id20}}

\begin{savenotes}\sphinxattablestart
\centering
\begin{tabulary}{\linewidth}[t]{|T|T|}
\hline
\sphinxstylethead{\sphinxstyletheadfamily 
Option
\unskip}\relax &\sphinxstylethead{\sphinxstyletheadfamily 
Description
\unskip}\relax \\
\hline
\sphinxcode{-{-}version}
&
Show program’s version number and exit
\\
\hline
\sphinxcode{-h, -{-}help}
&
Show this help message and exit.
\\
\hline
\sphinxcode{-k, -{-}keep-going}
&
Keep going as far as sensible. {[}default: False{]}
\\
\hline
\sphinxcode{-l LOGLEVEL, -{-}loglevel=LOGLEVEL}
&
Log Level (debug=10, info=20, warning=30, error=40, critical=50) {[}default: 20{]}
\\
\hline
\sphinxcode{-j JOBS, -{-}jobs=JOBS}
&
Max processes when multiprocessing. Zero uses number of native CPUs {[}8{]}.
-1 disables multiprocessing. {[}default: -1{]}
\\
\hline
\sphinxcode{-t TIMES, -{-}times=TIMES}
&
Number of times to repeat the read {[}default: 1{]}
\\
\hline
\sphinxcode{-s, -{-}statistics}
&
Dump timing statistics. {[}default: False{]}
\\
\hline
\sphinxcode{-v, -{-}verbose}
&
Verbose output, this outputs a representation of table data and DFSRs.
\\
\hline
\sphinxcode{-r, -{-}recursive}
&
Process input recursively. {[}default: False{]}
\\
\hline
\end{tabulary}
\par
\sphinxattableend\end{savenotes}


\subparagraph{Examples}
\label{\detokenize{cmd_line_tools/LIS_cmd_line_tools:id21}}
Simple scan of a single file:

\begin{sphinxVerbatim}[commandchars=\\\{\}]
\PYGZdl{} {}`{}`tdlisindex{}`{}` Simple/LIS.lis
Cmd: Index.py Simple/LIS.lis
2012\PYGZhy{}02\PYGZhy{}09 09:36:28,039 INFO     Index.indexFile(): Simple/LIS.lis
File size: 17708 (0.017 MB) Reference Time: 0.002459 (s) for Simple/LIS.lis pickleLen=4351 jsonLen=\PYGZhy{}1
Summary:
Results:        1
 Errors:        0
  Total:        1
CPU time =    0.004 (S)
Bye, bye!
\end{sphinxVerbatim}

Simple scan of a single file with verbose output:

\begin{sphinxVerbatim}[commandchars=\\\{\}]
\PYGZdl{} {}`{}`tdlisindex{}`{}` \PYGZhy{}v Simple/LIS.lis
Cmd: Index.py \PYGZhy{}v Simple/LIS.lis
2012\PYGZhy{}02\PYGZhy{}09 09:39:29,493 INFO     Index.indexFile(): Simple/LIS.lis
\PYGZlt{}TotalDepth.LIS.core.FileIndexer.FileIndex object at 0x10197fdd0\PYGZgt{} \PYGZdq{}Simple/LIS.lis\PYGZdq{} [4]:
  tell: 0x00000000 type=128 \PYGZlt{}TotalDepth.LIS.core.FileIndexer.IndexFileHead object at 0x10197fe10\PYGZgt{}
  tell: 0x0000003e type= 34 name=b\PYGZsq{}CONS\PYGZsq{} \PYGZlt{}TotalDepth.LIS.core.FileIndexer.IndexTable object at 0x10197fe90\PYGZgt{}
  \PYGZlt{}TotalDepth.LIS.core.LogPass.LogPass object at 0x101b071d0\PYGZgt{}
  tell: 0x000044ee type=129 \PYGZlt{}TotalDepth.LIS.core.FileIndexer.IndexFileTail object at 0x101b07790\PYGZgt{}
=============================== All records ===============================
tell: 0x00000000 type=128 \PYGZlt{}TotalDepth.LIS.core.FileIndexer.IndexFileHead object at 0x10197fe10\PYGZgt{}
tell: 0x0000003e type= 34 name=b\PYGZsq{}CONS\PYGZsq{} \PYGZlt{}TotalDepth.LIS.core.FileIndexer.IndexTable object at 0x10197fe90\PYGZgt{}
\PYGZlt{}TotalDepth.LIS.core.LogPass.LogPass object at 0x101b071d0\PYGZgt{}
tell: 0x000044ee type=129 \PYGZlt{}TotalDepth.LIS.core.FileIndexer.IndexFileTail object at 0x101b07790\PYGZgt{}
============================= All records DONE ============================
================================ Log Passes ===============================
LogPass \PYGZlt{}TotalDepth.LIS.core.LogPass.LogPass object at 0x101b071d0\PYGZgt{}:
       DFSR: \PYGZlt{}TotalDepth.LIS.core.LogiRec.LrDFSRRead object at 0x10197ff90\PYGZgt{}: \PYGZdq{}Data format specification record\PYGZdq{}
 Frame plan: \PYGZlt{}TotalDepth.LIS.core.Type01Plan.FrameSetPlan object at 0x101b07210\PYGZgt{}: indr=0 frame length=24 channels=6
   Channels: [b\PYGZsq{}DEPT\PYGZsq{}, b\PYGZsq{}SP  \PYGZsq{}, b\PYGZsq{}SN  \PYGZsq{}, b\PYGZsq{}ILD \PYGZsq{}, b\PYGZsq{}CILD\PYGZsq{}, b\PYGZsq{}DT  \PYGZsq{}]
        RLE: \PYGZlt{}TotalDepth.LIS.core.Rle.RLEType01 object at 0x101b07250\PYGZgt{}: func=None: [RLEItemType01: datum=8592 stride=1014 repeat=7 frames=42, RLEItemType01: datum=16704 stride=None repeat=0 frames=39]
     X axis: first=2052.983 last=1995.986 frames=375 overall spacing=\PYGZhy{}0.1524 in optical units=b\PYGZsq{}M   \PYGZsq{} (actual units=b\PYGZsq{}M   \PYGZsq{})
  Frame set: None

============================= Log Passes DONE =============================
=============================== Plot Records ==============================
============================ Plot Records DONE ============================
   Min: 0.003 (s)
   Max: 0.003 (s)
  Mean: 0.003 (s)
File size: 17708 (0.017 MB) Reference Time: 0.002529 (s) for Simple/LIS.lis pickleLen=4351 jsonLen=\PYGZhy{}1
Summary:
Results:        1
 Errors:        0
  Total:        1
CPU time =    0.004 (S)
Bye, bye!
\end{sphinxVerbatim}

Scan of a directory (recursively) indexing each file 11 times and writing out statistics:

\begin{sphinxVerbatim}[commandchars=\\\{\}]
\PYGZdl{} {}`{}`tdlisindex{}`{}` \PYGZhy{}t11 \PYGZhy{}s \PYGZhy{}l 40 Simple/
Cmd: Index.py \PYGZhy{}t11 \PYGZhy{}s \PYGZhy{}l 40 ../../../../TDTestData/LIS/Simple
File size: 17708 (0.017 MB) Reference Time: 0.001670 (s) for Simple/LIS.lis pickleLen=4351 jsonLen=\PYGZhy{}1
File size: 863374 (0.823 MB) Reference Time: 0.043411 (s) for Simple/RW.lis pickleLen=18231 jsonLen=\PYGZhy{}1
File size: 853030 (0.814 MB) Reference Time: 0.039238 (s) for Simple/RW\PYGZus{}No\PYGZus{}TIF.lis pickleLen=18238 jsonLen=\PYGZhy{}1
Summary:
Size (kb)   Time (s)
17.293      0.001670
843.139     0.043411
833.037     0.039238

Files: 3
Errors: 0
CPU time =    0.938 (S)
Bye, bye!
\end{sphinxVerbatim}


\paragraph{\sphinxstyleliteralintitle{tdXlisrandomframesetread}}
\label{\detokenize{cmd_line_tools/LIS_cmd_line_tools:tdxlisrandomframesetread}}
For developers only. This is designed to measure the performance of loading and iterating across a frame-set.


\section{Usage In a Python Project}
\label{\detokenize{usage:usage-in-a-python-project}}\label{\detokenize{usage::doc}}
To use TotalDepth:

\begin{sphinxVerbatim}[commandchars=\\\{\}]
\PYG{k+kn}{import} \PYG{n+nn}{TotalDepth}
\end{sphinxVerbatim}


\subsection{Important Concepts}
\label{\detokenize{usage:important-concepts}}
Due to the size of data TotalDepth makes extensive use of lazy evaluation and generators, this means that compute power is only used where necesary.


\subsubsection{Log Pass}
\label{\detokenize{usage:log-pass}}
A {\hyperref[\detokenize{glossary:term-log-pass}]{\sphinxtermref{\DUrole{xref,std,std-term}{Log Pass}}}} represents a typical, independent, logging recording consisting of a format specification plus binary data. The format specification defines how to interpret the binary data (number of channels, name and units for each channel). In the LIS format the formatting record is known as the {\hyperref[\detokenize{glossary:term-data-format-specification-record}]{\sphinxtermref{\DUrole{xref,std,std-term}{Data Format Specification Record}}}} ({\hyperref[\detokenize{glossary:term-dfsr}]{\sphinxtermref{\DUrole{xref,std,std-term}{DFSR}}}}). A “Repeat Section” plus a “Main Log” are two, distinct, independent Log Passes.


\subsubsection{Frame Set}
\label{\detokenize{usage:frame-set}}
A {\hyperref[\detokenize{glossary:term-frame-set}]{\sphinxtermref{\DUrole{xref,std,std-term}{Frame Set}}}} is the logging data converted to the platforms internal types (usually C doubles) and stored in memory as a table where each row represents a particular depth (or time) and each column is the outptu of a specific channel. Each row is often referred to as a {\hyperref[\detokenize{glossary:term-frame}]{\sphinxtermref{\DUrole{xref,std,std-term}{Frame}}}}


\section{Processing LIS Files}
\label{\detokenize{usage:processing-lis-files}}

\subsection{Important LIS Concepts}
\label{\detokenize{usage:important-lis-concepts}}

\subsubsection{LIS File}
\label{\detokenize{usage:lis-file}}
A LIS file is a wrapper round the platforms native I/O system.

Reference: {\hyperref[\detokenize{ref/LIS/core/File:TotalDepth.LIS.core.File.FileRead}]{\sphinxcrossref{\sphinxcode{TotalDepth.LIS.core.File.FileRead}}}}


\subsubsection{LIS Index}
\label{\detokenize{usage:lis-index}}
The LIS file format is designed for recording rather than processing. As such it is a sequential self anouncing format where to understand any part of it you have to have processed all of the preceeding data. This becomes very expensive with large files.

To avoid this an index is first created for the file that then allows random access to any part of that file including an individual frame. All access to the file is using this index. This index is very fast to create and its size is typically 1\% of the original file size.

Reference: {\hyperref[\detokenize{ref/LIS/core/FileIndexer:TotalDepth.LIS.core.FileIndexer.FileIndex}]{\sphinxcrossref{\sphinxcode{TotalDepth.LIS.core.FileIndexer.FileIndex}}}}


\subsection{Reading a LIS File Table Data}
\label{\detokenize{usage:reading-a-lis-file-table-data}}
Suppose that we have a file in \sphinxcode{\textasciitilde{}/tmp/LIS.lis} and you want to read the well site data. First import what we need:

\begin{sphinxVerbatim}[commandchars=\\\{\}]
\PYG{g+gp}{\PYGZgt{}\PYGZgt{}\PYGZgt{} }\PYG{k+kn}{import} \PYG{n+nn}{TotalDepth}
\PYG{g+gp}{\PYGZgt{}\PYGZgt{}\PYGZgt{} }\PYG{k+kn}{from} \PYG{n+nn}{TotalDepth}\PYG{n+nn}{.}\PYG{n+nn}{LIS}\PYG{n+nn}{.}\PYG{n+nn}{core} \PYG{k}{import} \PYG{n}{File}
\PYG{g+gp}{\PYGZgt{}\PYGZgt{}\PYGZgt{} }\PYG{k+kn}{from} \PYG{n+nn}{TotalDepth}\PYG{n+nn}{.}\PYG{n+nn}{LIS}\PYG{n+nn}{.}\PYG{n+nn}{core} \PYG{k}{import} \PYG{n}{FileIndexer}
\PYG{g+gp}{\PYGZgt{}\PYGZgt{}\PYGZgt{} }\PYG{k+kn}{import} \PYG{n+nn}{os}
\end{sphinxVerbatim}

Then open the LIS file and create an index from it:

\begin{sphinxVerbatim}[commandchars=\\\{\}]
\PYG{g+gp}{\PYGZgt{}\PYGZgt{}\PYGZgt{} }\PYG{n}{fpath} \PYG{o}{=} \PYG{n}{os}\PYG{o}{.}\PYG{n}{path}\PYG{o}{.}\PYG{n}{expanduser}\PYG{p}{(}\PYG{l+s+s1}{\PYGZsq{}}\PYG{l+s+s1}{\PYGZti{}/tmp/LIS.lis}\PYG{l+s+s1}{\PYGZsq{}}\PYG{p}{)}
\PYG{g+gp}{\PYGZgt{}\PYGZgt{}\PYGZgt{} }\PYG{n}{lis\PYGZus{}file} \PYG{o}{=} \PYG{n}{TotalDepth}\PYG{o}{.}\PYG{n}{LIS}\PYG{o}{.}\PYG{n}{core}\PYG{o}{.}\PYG{n}{File}\PYG{o}{.}\PYG{n}{FileRead}\PYG{p}{(}\PYG{n}{fpath}\PYG{p}{)}
\PYG{g+gp}{\PYGZgt{}\PYGZgt{}\PYGZgt{} }\PYG{n}{lis\PYGZus{}index} \PYG{o}{=} \PYG{n}{TotalDepth}\PYG{o}{.}\PYG{n}{LIS}\PYG{o}{.}\PYG{n}{core}\PYG{o}{.}\PYG{n}{FileIndexer}\PYG{o}{.}\PYG{n}{FileIndex}\PYG{p}{(}\PYG{n}{lis\PYGZus{}file}\PYG{p}{)}
\end{sphinxVerbatim}

There is a method on the index \sphinxcode{genAll()} that iterates through all the records, filter this for only table data:

\begin{sphinxVerbatim}[commandchars=\\\{\}]
\PYG{g+gp}{\PYGZgt{}\PYGZgt{}\PYGZgt{} }\PYG{k+kn}{from} \PYG{n+nn}{TotalDepth}\PYG{n+nn}{.}\PYG{n+nn}{LIS}\PYG{n+nn}{.}\PYG{n+nn}{core} \PYG{k}{import} \PYG{n}{LogiRec}
\PYG{g+gp}{\PYGZgt{}\PYGZgt{}\PYGZgt{} }\PYG{n}{cons\PYGZus{}records} \PYG{o}{=} \PYG{p}{[}\PYG{n}{lr} \PYG{k}{for} \PYG{n}{lr} \PYG{o+ow}{in} \PYG{n}{lis\PYGZus{}index}\PYG{o}{.}\PYG{n}{genAll}\PYG{p}{(}\PYG{p}{)} \PYG{k}{if} \PYG{n}{lr}\PYG{o}{.}\PYG{n}{lrType} \PYG{o+ow}{in} \PYG{n}{LogiRec}\PYG{o}{.}\PYG{n}{LR\PYGZus{}TYPE\PYGZus{}TABLE\PYGZus{}DATA}\PYG{p}{]}
\PYG{g+gp}{\PYGZgt{}\PYGZgt{}\PYGZgt{} }\PYG{n}{cons\PYGZus{}records}
\PYG{g+go}{[\PYGZlt{}TotalDepth.LIS.core.FileIndexer.IndexTable object at 0x103ac3dd8\PYGZgt{}]}
\end{sphinxVerbatim}

In this file there is only one table record. Now read it:

\begin{sphinxVerbatim}[commandchars=\\\{\}]
\PYG{g+gp}{\PYGZgt{}\PYGZgt{}\PYGZgt{} }\PYG{n}{lis\PYGZus{}file}\PYG{o}{.}\PYG{n}{seekLr}\PYG{p}{(}\PYG{n}{cons\PYGZus{}records}\PYG{p}{[}\PYG{l+m+mi}{0}\PYG{p}{]}\PYG{o}{.}\PYG{n}{tell}\PYG{p}{)}
\PYG{g+gp}{\PYGZgt{}\PYGZgt{}\PYGZgt{} }\PYG{n}{table} \PYG{o}{=} \PYG{n}{LogiRec}\PYG{o}{.}\PYG{n}{LrTableRead}\PYG{p}{(}\PYG{n}{lis\PYGZus{}file}\PYG{p}{)}
\end{sphinxVerbatim}

Now we can explore the table:

\begin{sphinxVerbatim}[commandchars=\\\{\}]
\PYG{g+gp}{\PYGZgt{}\PYGZgt{}\PYGZgt{} }\PYG{n}{table}\PYG{o}{.}\PYG{n}{desc}
\PYG{g+go}{\PYGZsq{}Well site data\PYGZsq{}}
\PYG{g+gp}{\PYGZgt{}\PYGZgt{}\PYGZgt{} }\PYG{n}{table}\PYG{o}{.}\PYG{n}{value}
\PYG{g+go}{b\PYGZsq{}CONS\PYGZsq{}}
\PYG{g+gp}{\PYGZgt{}\PYGZgt{}\PYGZgt{} }\PYG{n}{table}\PYG{o}{.}\PYG{n}{colLabels}\PYG{p}{(}\PYG{p}{)}
\PYG{g+go}{odict\PYGZus{}keys([b\PYGZsq{}MNEM\PYGZsq{}, b\PYGZsq{}ALLO\PYGZsq{}, b\PYGZsq{}PUNI\PYGZsq{}, b\PYGZsq{}TUNI\PYGZsq{}, b\PYGZsq{}VALU\PYGZsq{}])}
\PYG{g+gp}{\PYGZgt{}\PYGZgt{}\PYGZgt{} }\PYG{n}{table}\PYG{o}{.}\PYG{n}{rowLabels}\PYG{p}{(}\PYG{p}{)}
\PYG{g+go}{dict\PYGZus{}keys([b\PYGZsq{}HIDE\PYGZsq{}, b\PYGZsq{}HID1\PYGZsq{}, b\PYGZsq{}HID2\PYGZsq{}, b\PYGZsq{}CN  \PYGZsq{}, b\PYGZsq{}WN  \PYGZsq{}, ..., b\PYGZsq{}C30 \PYGZsq{}])}
\end{sphinxVerbatim}

Notice all the entries are represented as Python bytes objects (\sphinxcode{b'...'}), this is because LIS does not support Unicode. LIS is also a bit shouty.

To get a specific value, say the well name:

\begin{sphinxVerbatim}[commandchars=\\\{\}]
\PYG{g+gp}{\PYGZgt{}\PYGZgt{}\PYGZgt{} }\PYG{n+nb}{print}\PYG{p}{(}\PYG{n}{table}\PYG{p}{[}\PYG{l+s+sa}{b}\PYG{l+s+s1}{\PYGZsq{}}\PYG{l+s+s1}{WN  }\PYG{l+s+s1}{\PYGZsq{}}\PYG{p}{]}\PYG{p}{[}\PYG{l+s+sa}{b}\PYG{l+s+s1}{\PYGZsq{}}\PYG{l+s+s1}{VALU}\PYG{l+s+s1}{\PYGZsq{}}\PYG{p}{]}\PYG{p}{)}
\PYG{g+go}{CB: type=69 rc=65 size=16 mnem=b\PYGZsq{}VALU\PYGZsq{} EngValRc: b\PYGZsq{}GUSHER\PYGZsq{}}
\PYG{g+gp}{\PYGZgt{}\PYGZgt{}\PYGZgt{} }\PYG{n}{table}\PYG{p}{[}\PYG{l+s+sa}{b}\PYG{l+s+s1}{\PYGZsq{}}\PYG{l+s+s1}{WN  }\PYG{l+s+s1}{\PYGZsq{}}\PYG{p}{]}\PYG{p}{[}\PYG{l+s+sa}{b}\PYG{l+s+s1}{\PYGZsq{}}\PYG{l+s+s1}{VALU}\PYG{l+s+s1}{\PYGZsq{}}\PYG{p}{]}\PYG{o}{.}\PYG{n}{value}
\PYG{g+go}{b\PYGZsq{}GUSHER\PYGZsq{}}
\end{sphinxVerbatim}

You can index by integer:

\begin{sphinxVerbatim}[commandchars=\\\{\}]
\PYG{g+gp}{\PYGZgt{}\PYGZgt{}\PYGZgt{} }\PYG{n}{table}\PYG{p}{[}\PYG{l+m+mi}{4}\PYG{p}{]}\PYG{p}{[}\PYG{l+m+mi}{0}\PYG{p}{]}\PYG{o}{.}\PYG{n}{value}
\PYG{g+go}{b\PYGZsq{}WN  \PYGZsq{}}
\PYG{g+gp}{\PYGZgt{}\PYGZgt{}\PYGZgt{} }\PYG{n}{table}\PYG{p}{[}\PYG{l+m+mi}{4}\PYG{p}{]}\PYG{p}{[}\PYG{l+m+mi}{4}\PYG{p}{]}\PYG{o}{.}\PYG{n}{value}
\PYG{g+go}{b\PYGZsq{}GUSHER\PYGZsq{}}
\PYG{g+gp}{\PYGZgt{}\PYGZgt{}\PYGZgt{} }\PYG{p}{[}\PYG{n}{v}\PYG{o}{.}\PYG{n}{value} \PYG{k}{for} \PYG{n}{v} \PYG{o+ow}{in} \PYG{n}{table}\PYG{p}{[}\PYG{l+m+mi}{4}\PYG{p}{]}\PYG{p}{]}
\PYG{g+go}{[b\PYGZsq{}WN  \PYGZsq{}, b\PYGZsq{}ALLO\PYGZsq{}, b\PYGZsq{}    \PYGZsq{}, b\PYGZsq{}    \PYGZsq{}, b\PYGZsq{}GROSSENKNETEN Z2\PYGZsq{}]}
\end{sphinxVerbatim}

You can index by slice:

\begin{sphinxVerbatim}[commandchars=\\\{\}]
\PYG{g+gp}{\PYGZgt{}\PYGZgt{}\PYGZgt{} }\PYG{p}{[}\PYG{n}{v}\PYG{o}{.}\PYG{n}{value} \PYG{k}{for} \PYG{n}{v} \PYG{o+ow}{in} \PYG{n}{table}\PYG{p}{[}\PYG{l+m+mi}{4}\PYG{p}{]}\PYG{p}{[}\PYG{p}{:}\PYG{l+m+mi}{2}\PYG{p}{]}\PYG{p}{]}
\PYG{g+go}{[b\PYGZsq{}WN  \PYGZsq{}, b\PYGZsq{}ALLO\PYGZsq{}]}
\end{sphinxVerbatim}

To print the whole table there are some generators for this:

\begin{sphinxVerbatim}[commandchars=\\\{\}]
\PYG{g+gp}{\PYGZgt{}\PYGZgt{}\PYGZgt{} }\PYG{k}{for} \PYG{n}{row} \PYG{o+ow}{in} \PYG{n}{table}\PYG{o}{.}\PYG{n}{genRows}\PYG{p}{(}\PYG{p}{)}\PYG{p}{:}
\PYG{g+gp}{... }    \PYG{k}{for} \PYG{n}{col} \PYG{o+ow}{in} \PYG{n}{row}\PYG{o}{.}\PYG{n}{genCells}\PYG{p}{(}\PYG{p}{)}\PYG{p}{:}
\PYG{g+gp}{... }        \PYG{n+nb}{print}\PYG{p}{(}\PYG{n}{col}\PYG{o}{.}\PYG{n}{value}\PYG{p}{,} \PYG{l+s+s1}{\PYGZsq{}}\PYG{l+s+s1}{ }\PYG{l+s+s1}{\PYGZsq{}}\PYG{p}{,} \PYG{n}{end}\PYG{o}{=}\PYG{l+s+s1}{\PYGZsq{}}\PYG{l+s+s1}{\PYGZsq{}}\PYG{p}{)}
\PYG{g+gp}{... }    \PYG{n+nb}{print}\PYG{p}{(}\PYG{p}{)}
\PYG{g+gp}{...}
\PYG{g+go}{b\PYGZsq{}HIDE\PYGZsq{}  b\PYGZsq{}ALLO\PYGZsq{}  b\PYGZsq{}    \PYGZsq{}  b\PYGZsq{}    \PYGZsq{}  b\PYGZsq{}MAIN LOG\PYGZsq{}}
\PYG{g+go}{b\PYGZsq{}HID1\PYGZsq{}  b\PYGZsq{}ALLO\PYGZsq{}  b\PYGZsq{}    \PYGZsq{}  b\PYGZsq{}    \PYGZsq{}  b\PYGZsq{}RAW DATA\PYGZsq{}}
\PYG{g+go}{b\PYGZsq{}HID2\PYGZsq{}  b\PYGZsq{}ALLO\PYGZsq{}  b\PYGZsq{}    \PYGZsq{}  b\PYGZsq{}    \PYGZsq{}  b\PYGZsq{}\PYGZsq{}}
\PYG{g+go}{b\PYGZsq{}CN  \PYGZsq{}  b\PYGZsq{}ALLO\PYGZsq{}  b\PYGZsq{}    \PYGZsq{}  b\PYGZsq{}    \PYGZsq{}  b\PYGZsq{}BIG COMPANY\PYGZsq{}}
\PYG{g+go}{b\PYGZsq{}WN  \PYGZsq{}  b\PYGZsq{}ALLO\PYGZsq{}  b\PYGZsq{}    \PYGZsq{}  b\PYGZsq{}    \PYGZsq{}  b\PYGZsq{}GUSHER\PYGZsq{}}
\PYG{g+gp}{...}
\end{sphinxVerbatim}

Reference: {\hyperref[\detokenize{ref/LIS/core/LogiRec:TotalDepth.LIS.core.LogiRec.LrTable}]{\sphinxcrossref{\sphinxcode{TotalDepth.LIS.core.LogiRec.LrTable}}}}


\subsection{Reading a LIS File Log Data}
\label{\detokenize{usage:reading-a-lis-file-log-data}}
Suppose that we have a file in \sphinxcode{\textasciitilde{}/tmp/LIS.lis} and you want to read the frame data from a particular log pass, first import what we need:

\begin{sphinxVerbatim}[commandchars=\\\{\}]
\PYG{g+gp}{\PYGZgt{}\PYGZgt{}\PYGZgt{} }\PYG{k+kn}{import} \PYG{n+nn}{TotalDepth}
\PYG{g+gp}{\PYGZgt{}\PYGZgt{}\PYGZgt{} }\PYG{k+kn}{from} \PYG{n+nn}{TotalDepth}\PYG{n+nn}{.}\PYG{n+nn}{LIS}\PYG{n+nn}{.}\PYG{n+nn}{core} \PYG{k}{import} \PYG{n}{File}
\PYG{g+gp}{\PYGZgt{}\PYGZgt{}\PYGZgt{} }\PYG{k+kn}{from} \PYG{n+nn}{TotalDepth}\PYG{n+nn}{.}\PYG{n+nn}{LIS}\PYG{n+nn}{.}\PYG{n+nn}{core} \PYG{k}{import} \PYG{n}{FileIndexer}
\PYG{g+gp}{\PYGZgt{}\PYGZgt{}\PYGZgt{} }\PYG{k+kn}{import} \PYG{n+nn}{os}
\end{sphinxVerbatim}

Then open the LIS file and create an index from it:

\begin{sphinxVerbatim}[commandchars=\\\{\}]
\PYG{g+gp}{\PYGZgt{}\PYGZgt{}\PYGZgt{} }\PYG{n}{fpath} \PYG{o}{=} \PYG{n}{os}\PYG{o}{.}\PYG{n}{path}\PYG{o}{.}\PYG{n}{expanduser}\PYG{p}{(}\PYG{l+s+s1}{\PYGZsq{}}\PYG{l+s+s1}{\PYGZti{}/tmp/LIS.lis}\PYG{l+s+s1}{\PYGZsq{}}\PYG{p}{)}
\PYG{g+gp}{\PYGZgt{}\PYGZgt{}\PYGZgt{} }\PYG{n}{lis\PYGZus{}file} \PYG{o}{=} \PYG{n}{TotalDepth}\PYG{o}{.}\PYG{n}{LIS}\PYG{o}{.}\PYG{n}{core}\PYG{o}{.}\PYG{n}{File}\PYG{o}{.}\PYG{n}{FileRead}\PYG{p}{(}\PYG{n}{fpath}\PYG{p}{)}
\PYG{g+gp}{\PYGZgt{}\PYGZgt{}\PYGZgt{} }\PYG{n}{lis\PYGZus{}index} \PYG{o}{=} \PYG{n}{TotalDepth}\PYG{o}{.}\PYG{n}{LIS}\PYG{o}{.}\PYG{n}{core}\PYG{o}{.}\PYG{n}{FileIndexer}\PYG{o}{.}\PYG{n}{FileIndex}\PYG{p}{(}\PYG{n}{lis\PYGZus{}file}\PYG{p}{)}
\end{sphinxVerbatim}

There is a method on the index \sphinxcode{genLogPasses()} that iterates through the log passes, lets get them all:

\begin{sphinxVerbatim}[commandchars=\\\{\}]
\PYG{g+gp}{\PYGZgt{}\PYGZgt{}\PYGZgt{} }\PYG{n}{log\PYGZus{}passes} \PYG{o}{=} \PYG{n+nb}{list}\PYG{p}{(}\PYG{n}{lis\PYGZus{}index}\PYG{o}{.}\PYG{n}{genLogPasses}\PYG{p}{(}\PYG{p}{)}\PYG{p}{)}
\PYG{g+gp}{\PYGZgt{}\PYGZgt{}\PYGZgt{} }\PYG{n+nb}{print}\PYG{p}{(}\PYG{n}{log\PYGZus{}passes}\PYG{p}{)}
\PYG{g+go}{[\PYGZlt{}TotalDepth.LIS.core.FileIndexer.IndexLogPass object at 0x103ac3e80\PYGZgt{}]}
\end{sphinxVerbatim}

In this file there is only one log pass, we can get the description of it using \sphinxcode{longstr()}:

\begin{sphinxVerbatim}[commandchars=\\\{\}]
\PYG{g+gp}{\PYGZgt{}\PYGZgt{}\PYGZgt{} }\PYG{n+nb}{print}\PYG{p}{(}\PYG{n}{log\PYGZus{}passes}\PYG{p}{[}\PYG{l+m+mi}{0}\PYG{p}{]}\PYG{o}{.}\PYG{n}{logPass}\PYG{o}{.}\PYG{n}{longStr}\PYG{p}{(}\PYG{p}{)}\PYG{p}{)}
\PYG{g+go}{\PYGZlt{}TotalDepth.LIS.core.LogPass.LogPass object at 0x103ae10b8\PYGZgt{}:}
\PYG{g+go}{       DFSR: \PYGZlt{}TotalDepth.LIS.core.LogiRec.LrDFSRRead object at 0x103ac3eb8\PYGZgt{}: \PYGZdq{}Data format specification record\PYGZdq{}}
\PYG{g+go}{ Frame plan: \PYGZlt{}TotalDepth.LIS.core.Type01Plan.FrameSetPlan object at 0x103ae10f0\PYGZgt{}: indr=0 frame length=24 channels=6}
\PYG{g+go}{   Channels: [b\PYGZsq{}DEPT\PYGZsq{}, b\PYGZsq{}SP  \PYGZsq{}, b\PYGZsq{}SN  \PYGZsq{}, b\PYGZsq{}ILD \PYGZsq{}, b\PYGZsq{}CILD\PYGZsq{}, b\PYGZsq{}DT  \PYGZsq{}]}
\PYG{g+go}{        RLE: \PYGZlt{}TotalDepth.LIS.core.Rle.RLEType01 object at 0x103ae1128\PYGZgt{}: func=None: [RLEItemType01: datum=8592 stride=1014 repeat=7 frames=42, RLEItemType01: datum=16704 stride=None repeat=0 frames=39]}
\PYG{g+go}{     X axis: first=2052.983 last=1995.986 frames=375 overall spacing=\PYGZhy{}0.1524 in optical units=b\PYGZsq{}M   \PYGZsq{} (actual units=b\PYGZsq{}M   \PYGZsq{})}
\PYG{g+go}{  Frame set: None}
\end{sphinxVerbatim}

Note the last line \sphinxcode{Frame set: None}, this is because the log pass is a lightweight object which does not (yet) contain all the frame data. To read all the frame data from the file we call \sphinxcode{setFrameData(LisFile)} on the log pass:

\begin{sphinxVerbatim}[commandchars=\\\{\}]
\PYG{g+gp}{\PYGZgt{}\PYGZgt{}\PYGZgt{} }\PYG{n}{log\PYGZus{}passes}\PYG{p}{[}\PYG{l+m+mi}{0}\PYG{p}{]}\PYG{o}{.}\PYG{n}{logPass}\PYG{o}{.}\PYG{n}{setFrameSet}\PYG{p}{(}\PYG{n}{lis\PYGZus{}file}\PYG{p}{)}
\end{sphinxVerbatim}

Now the frame set is fully populated:

\begin{sphinxVerbatim}[commandchars=\\\{\}]
\PYG{g+gp}{\PYGZgt{}\PYGZgt{}\PYGZgt{} }\PYG{n+nb}{print}\PYG{p}{(}\PYG{n+nb}{list}\PYG{p}{(}\PYG{n}{log\PYGZus{}passes}\PYG{p}{[}\PYG{l+m+mi}{0}\PYG{p}{]}\PYG{o}{.}\PYG{n}{logPass}\PYG{o}{.}\PYG{n}{genFrameSetScNameUnit}\PYG{p}{(}\PYG{p}{)}\PYG{p}{)}\PYG{p}{)}
\PYG{g+go}{[(\PYGZsq{}DEPT\PYGZsq{}, \PYGZsq{}M   \PYGZsq{}), (\PYGZsq{}SP  \PYGZsq{}, \PYGZsq{}MV  \PYGZsq{}), (\PYGZsq{}SN  \PYGZsq{}, \PYGZsq{}OHMM\PYGZsq{}), (\PYGZsq{}ILD \PYGZsq{}, \PYGZsq{}OHMM\PYGZsq{}), (\PYGZsq{}CILD\PYGZsq{}, \PYGZsq{}MMHO\PYGZsq{}), (\PYGZsq{}DT  \PYGZsq{}, \PYGZsq{}US/M\PYGZsq{})]}
\end{sphinxVerbatim}

To get the actual vales in the frame we can access the numpy array directly:

\begin{sphinxVerbatim}[commandchars=\\\{\}]
\PYG{g+gp}{\PYGZgt{}\PYGZgt{}\PYGZgt{} }\PYG{n}{data} \PYG{o}{=} \PYG{n}{log\PYGZus{}passes}\PYG{p}{[}\PYG{l+m+mi}{0}\PYG{p}{]}\PYG{o}{.}\PYG{n}{logPass}\PYG{o}{.}\PYG{n}{frameSet}\PYG{o}{.}\PYG{n}{frames}
\PYG{g+gp}{\PYGZgt{}\PYGZgt{}\PYGZgt{} }\PYG{n}{data}
\PYG{g+go}{array([[  2.05298340e+03,  \PYGZhy{}4.54907703e+00,   1.34538269e+00,}
\PYG{g+go}{          1.26347518e+00,   3.86598633e+02,  \PYGZhy{}9.99250000e+02],}
\PYG{g+go}{       [  2.05283105e+03,  \PYGZhy{}5.13720322e+00,   1.36061692e+00,}
\PYG{g+go}{          1.29521227e+00,   5.00510803e+02,  \PYGZhy{}9.99250000e+02],}
\PYG{g+go}{       [  2.05267871e+03,  \PYGZhy{}6.66747475e+00,   1.38543439e+00,}
\PYG{g+go}{          1.45785594e+00,   5.95623291e+02,  \PYGZhy{}9.99250000e+02],}
\PYG{g+go}{       ...,}
\PYG{g+go}{       [  1.99629077e+03,  \PYGZhy{}9.99250000e+02,  \PYGZhy{}9.99250000e+02,}
\PYG{g+go}{         \PYGZhy{}9.99250000e+02,  \PYGZhy{}9.99250000e+02,  \PYGZhy{}9.99250000e+02],}
\PYG{g+go}{       [  1.99613843e+03,  \PYGZhy{}9.99250000e+02,  \PYGZhy{}9.99250000e+02,}
\PYG{g+go}{         \PYGZhy{}9.99250000e+02,  \PYGZhy{}9.99250000e+02,  \PYGZhy{}9.99250000e+02],}
\PYG{g+go}{       [  1.99598608e+03,  \PYGZhy{}9.99250000e+02,  \PYGZhy{}9.99250000e+02,}
\PYG{g+go}{         \PYGZhy{}9.99250000e+02,  \PYGZhy{}9.99250000e+02,  \PYGZhy{}9.99250000e+02]])}
\end{sphinxVerbatim}

Now, if you are familiar with numpy then all normal operations are possible, for example get the X axis:

\begin{sphinxVerbatim}[commandchars=\\\{\}]
\PYG{g+gp}{\PYGZgt{}\PYGZgt{}\PYGZgt{} }\PYG{n}{data}\PYG{p}{[}\PYG{p}{:}\PYG{p}{,}\PYG{l+m+mi}{0}\PYG{p}{]}
\PYG{g+go}{array([ 2052.98339844,  2052.83105469,  2052.67871094,  2052.52636719,}
\PYG{g+go}{        2052.37402344,  2052.22167969,  2052.06933594,  2051.91650391,}
\PYG{g+go}{        2051.76416016,  2051.61181641,  2051.45947266,  2051.30712891,}
\PYG{g+go}{        ...}
\PYG{g+go}{        1996.29077148,  1996.13842773,  1995.98608398])}
\end{sphinxVerbatim}

Find the min, mean, max:

\begin{sphinxVerbatim}[commandchars=\\\{\}]
\PYG{g+gp}{\PYGZgt{}\PYGZgt{}\PYGZgt{} }\PYG{n}{data}\PYG{o}{.}\PYG{n}{min}\PYG{p}{(}\PYG{n}{axis}\PYG{o}{=}\PYG{l+m+mi}{0}\PYG{p}{)}
\PYG{g+go}{array([ 1995.98608398,  \PYGZhy{}999.25      ,  \PYGZhy{}999.25      ,  \PYGZhy{}999.25      ,}
\PYG{g+go}{        \PYGZhy{}999.25      ,  \PYGZhy{}999.25      ])}
\PYG{g+gp}{\PYGZgt{}\PYGZgt{}\PYGZgt{} }\PYG{n}{data}\PYG{o}{.}\PYG{n}{mean}\PYG{p}{(}\PYG{n}{axis}\PYG{o}{=}\PYG{l+m+mi}{0}\PYG{p}{)}
\PYG{g+go}{array([ 2024.48480339,  \PYGZhy{}305.07223682,  \PYGZhy{}326.84234802,  \PYGZhy{}324.20620109,}
\PYG{g+go}{         206.05499658,    28.2555695 ])}
\PYG{g+gp}{\PYGZgt{}\PYGZgt{}\PYGZgt{} }\PYG{n}{data}\PYG{o}{.}\PYG{n}{max}\PYG{p}{(}\PYG{n}{axis}\PYG{o}{=}\PYG{l+m+mi}{0}\PYG{p}{)}
\PYG{g+go}{array([  2.05298340e+03,  \PYGZhy{}7.69491196e\PYGZhy{}01,   1.98412299e+00,}
\PYG{g+go}{         2.34852839e+00,   1.75242944e+03,   4.59522583e+02])}
\end{sphinxVerbatim}

References:

LogPass: {\hyperref[\detokenize{ref/LIS/core/LogPass:TotalDepth.LIS.core.LogPass.LogPass}]{\sphinxcrossref{\sphinxcode{TotalDepth.LIS.core.LogPass.LogPass}}}}

FrameSet: {\hyperref[\detokenize{ref/LIS/core/FrameSet:TotalDepth.LIS.core.FrameSet.FrameSet}]{\sphinxcrossref{\sphinxcode{TotalDepth.LIS.core.FrameSet.FrameSet}}}}


\section{Processing LAS Files}
\label{\detokenize{usage:processing-las-files}}
TODO:


\section{TotalDepth Technical Notes}
\label{\detokenize{tech/index_tech:totaldepth-technical-notes}}\label{\detokenize{tech/index_tech::doc}}
This collect together various technical descriptions of how TotalDepth solves certain problems.

Contents:


\subsection{Indexing LIS Files}
\label{\detokenize{tech/LIS_indexing:indexing-lis-files}}\label{\detokenize{tech/LIS_indexing::doc}}\label{\detokenize{tech/LIS_indexing:totaldepth-tech-indexing}}
Most petrophysical files are recorded in real time and the recording format is sequential, thus everything depends on what has gone before. This is not very satisfactory for the user who might well wish to access the data in a arbitrary manner - “give me these curves over this interval”. The solution is to create an index to the original file so that it can be accessed \sphinxstyleemphasis{as if} it is a random access file.

Here we describe how that indexing works and the performance it achieves for the user.


\subsubsection{Introduction}
\label{\detokenize{tech/LIS_indexing:introduction}}
The LIS file format is a binary, self describing, sequential format with multiple layers of encoding and, in practice, no forward references. LIS files can be large and, generally speaking, the greater part of the file consists of frame data whose format is invariant within any particular {\hyperref[\detokenize{glossary:term-log-pass}]{\sphinxtermref{\DUrole{xref,std,std-term}{Log Pass}}}}.

In a dynamic situation, such as a user reading the file, the \sphinxstyleemphasis{instantaneous} amount of data needed from a LIS file is small compared with the file size. For example plotting 200 feet of a 50Mb file might need only 1/3000 of the data in the file. As the LIS file format is geared to sequential recording, not random access, accessing such a small amount efficiently needs additional software cunning.

TotalDepth’s approach to this is to use \sphinxstyleemphasis{indexing}. The essential requirements for an indexer are:
\begin{itemize}
\item {} 
Fast to create an index.

\item {} 
The index is small.

\item {} 
The index has sufficiently useful granularity.

\item {} 
The index can be serialised in a number of ways (as a minimum; XML, binary (e.g. ‘pickle’), JSON?).

\item {} 
The indexer design is flexible and extensible.

\end{itemize}

Apart from the cost of design and coding a solution the cost/benefit of indexing can be measured thus:
\begin{itemize}
\item {} 
The time to creating an index.

\item {} 
The space required by the index, in-memory or serialised in some form.

\item {} 
The time, in O(N), terms of accessing N bytes of data.

\end{itemize}

A design that has low time/space requirements is regarded as a ‘good’ design.


\subsubsection{Indexing Design}
\label{\detokenize{tech/LIS_indexing:indexing-design}}
TotalDepth’s LIS indexer works on several levels:


\begin{savenotes}\sphinxattablestart
\centering
\begin{tabulary}{\linewidth}[t]{|T|T|}
\hline
\sphinxstylethead{\sphinxstyletheadfamily 
Level
\unskip}\relax &\sphinxstylethead{\sphinxstyletheadfamily 
Description
\unskip}\relax \\
\hline
File Level
&
Indexing of the postion of all Logical Records
\\
\hline
Logical Records with frame data
&
Runs of these can be efficiently indexed with Run Length Encoding
\\
\hline
Within a Logical Record
containing frame data
&
Accessing a particular frame and channel can be by computation with the help
of a frame index. In particular this can identify seek/read sequences of any
value from any channel in O(1) time.
\\
\hline
\end{tabulary}
\par
\sphinxattableend\end{savenotes}


\paragraph{Indexing a LIS File}
\label{\detokenize{tech/LIS_indexing:indexing-a-lis-file}}
The intentions is to find the start position of each Logical Record and a minimal amount of information of that Logical Record. The start position is the \sphinxcode{size\_t} value of the file index of, either, the start of the TIF marker for the first Physical Record in the Logical Record (if TIF encoded), or, the start of the first Physical Record in the Logical Record (if \sphinxstyleemphasis{not} TIF encoded).

As well as recording the file position of the Logical Record the indexer retrieves part or all of the Logical Record contents. This is specialised by Logical Record type thus:


\begin{savenotes}\sphinxattablestart
\centering
\begin{tabulary}{\linewidth}[t]{|T|T|}
\hline
\sphinxstylethead{\sphinxstyletheadfamily 
Logical Record
\unskip}\relax &\sphinxstylethead{\sphinxstyletheadfamily 
Index Contents
\unskip}\relax \\
\hline
Reel/Tape/File Header/Trailer
&
The complete contents of the Logical Record.
These are small (58 or 128 bytes) but possibly important.
\\
\hline
Table records
&
The Logical Record header and the first Component Block.
These are variable, large, but possibly important.
\\
\hline
DFSR
&
The complete contents of the Logical Record.
These are variable size, not particularly large and very important.
\\
\hline
Logical Records with frame data
&
Indexed as part of a Log Pass with a RLE object. This records the Logical
Record header and the first data word.
\\
\hline
All other Logical Records
&
The Logical Record header only.
\\
\hline
\end{tabulary}
\par
\sphinxattableend\end{savenotes}


\subparagraph{Module}
\label{\detokenize{tech/LIS_indexing:module}}
The Python module that performs file indexing is \sphinxcode{TotalDepth.LIS.core.FileIndexer}

For reference documentation see: {\hyperref[\detokenize{ref/LIS/core/FileIndexer:totaldepth-lis-core-fileindexer}]{\sphinxcrossref{\DUrole{std,std-ref}{FileIndexer}}}}.


\subparagraph{Example}
\label{\detokenize{tech/LIS_indexing:example}}
The LIS package has an Index.py module that will index any LIS file. Here is some (selected) output of a single file:

\begin{sphinxVerbatim}[commandchars=\\\{\}]
\PYG{o}{==}\PYG{o}{==}\PYG{o}{==}\PYG{o}{==}\PYG{o}{==}\PYG{o}{==}\PYG{o}{==}\PYG{o}{==}\PYG{o}{==}\PYG{o}{==}\PYG{o}{==}\PYG{o}{==}\PYG{o}{==}\PYG{o}{==}\PYG{o}{==}\PYG{o}{=} \PYG{n}{All} \PYG{n}{records} \PYG{o}{==}\PYG{o}{==}\PYG{o}{==}\PYG{o}{==}\PYG{o}{==}\PYG{o}{==}\PYG{o}{==}\PYG{o}{==}\PYG{o}{==}\PYG{o}{==}\PYG{o}{==}\PYG{o}{==}\PYG{o}{==}\PYG{o}{==}\PYG{o}{==}\PYG{o}{=}
\PYG{n}{tell}\PYG{p}{:} \PYG{l+m+mh}{0x00000000} \PYG{n+nb}{type}\PYG{o}{=}\PYG{l+m+mi}{128} \PYG{o}{\PYGZlt{}}\PYG{n}{TotalDepth}\PYG{o}{.}\PYG{n}{LIS}\PYG{o}{.}\PYG{n}{core}\PYG{o}{.}\PYG{n}{FileIndexer}\PYG{o}{.}\PYG{n}{IndexFileHead} \PYG{n+nb}{object} \PYG{n}{at} \PYG{l+m+mh}{0x101b09050}\PYG{o}{\PYGZgt{}}
\PYG{n}{tell}\PYG{p}{:} \PYG{l+m+mh}{0x0000004a} \PYG{n+nb}{type}\PYG{o}{=} \PYG{l+m+mi}{34} \PYG{n}{name}\PYG{o}{=}\PYG{l+s+sa}{b}\PYG{l+s+s1}{\PYGZsq{}}\PYG{l+s+s1}{CONS}\PYG{l+s+s1}{\PYGZsq{}} \PYG{o}{\PYGZlt{}}\PYG{n}{TotalDepth}\PYG{o}{.}\PYG{n}{LIS}\PYG{o}{.}\PYG{n}{core}\PYG{o}{.}\PYG{n}{FileIndexer}\PYG{o}{.}\PYG{n}{IndexTable} \PYG{n+nb}{object} \PYG{n}{at} \PYG{l+m+mh}{0x101b090d0}\PYG{o}{\PYGZgt{}}
\PYG{o}{\PYGZlt{}}\PYG{n}{TotalDepth}\PYG{o}{.}\PYG{n}{LIS}\PYG{o}{.}\PYG{n}{core}\PYG{o}{.}\PYG{n}{LogPass}\PYG{o}{.}\PYG{n}{LogPass} \PYG{n+nb}{object} \PYG{n}{at} \PYG{l+m+mh}{0x101b09450}\PYG{o}{\PYGZgt{}}
\PYG{n}{tell}\PYG{p}{:} \PYG{l+m+mh}{0x0006141c} \PYG{n+nb}{type}\PYG{o}{=}\PYG{l+m+mi}{129} \PYG{o}{\PYGZlt{}}\PYG{n}{TotalDepth}\PYG{o}{.}\PYG{n}{LIS}\PYG{o}{.}\PYG{n}{core}\PYG{o}{.}\PYG{n}{FileIndexer}\PYG{o}{.}\PYG{n}{IndexFileTail} \PYG{n+nb}{object} \PYG{n}{at} \PYG{l+m+mh}{0x101b0c110}\PYG{o}{\PYGZgt{}}
\PYG{n}{tell}\PYG{p}{:} \PYG{l+m+mh}{0x00061466} \PYG{n+nb}{type}\PYG{o}{=}\PYG{l+m+mi}{128} \PYG{o}{\PYGZlt{}}\PYG{n}{TotalDepth}\PYG{o}{.}\PYG{n}{LIS}\PYG{o}{.}\PYG{n}{core}\PYG{o}{.}\PYG{n}{FileIndexer}\PYG{o}{.}\PYG{n}{IndexFileHead} \PYG{n+nb}{object} \PYG{n}{at} \PYG{l+m+mh}{0x101b0c190}\PYG{o}{\PYGZgt{}}
\PYG{n}{tell}\PYG{p}{:} \PYG{l+m+mh}{0x000614b0} \PYG{n+nb}{type}\PYG{o}{=} \PYG{l+m+mi}{34} \PYG{n}{name}\PYG{o}{=}\PYG{l+s+sa}{b}\PYG{l+s+s1}{\PYGZsq{}}\PYG{l+s+s1}{CONS}\PYG{l+s+s1}{\PYGZsq{}} \PYG{o}{\PYGZlt{}}\PYG{n}{TotalDepth}\PYG{o}{.}\PYG{n}{LIS}\PYG{o}{.}\PYG{n}{core}\PYG{o}{.}\PYG{n}{FileIndexer}\PYG{o}{.}\PYG{n}{IndexTable} \PYG{n+nb}{object} \PYG{n}{at} \PYG{l+m+mh}{0x101b0c210}\PYG{o}{\PYGZgt{}}
\PYG{o}{\PYGZlt{}}\PYG{n}{TotalDepth}\PYG{o}{.}\PYG{n}{LIS}\PYG{o}{.}\PYG{n}{core}\PYG{o}{.}\PYG{n}{LogPass}\PYG{o}{.}\PYG{n}{LogPass} \PYG{n+nb}{object} \PYG{n}{at} \PYG{l+m+mh}{0x101b0c510}\PYG{o}{\PYGZgt{}}
\PYG{n}{tell}\PYG{p}{:} \PYG{l+m+mh}{0x00065a44} \PYG{n+nb}{type}\PYG{o}{=}\PYG{l+m+mi}{129} \PYG{o}{\PYGZlt{}}\PYG{n}{TotalDepth}\PYG{o}{.}\PYG{n}{LIS}\PYG{o}{.}\PYG{n}{core}\PYG{o}{.}\PYG{n}{FileIndexer}\PYG{o}{.}\PYG{n}{IndexFileTail} \PYG{n+nb}{object} \PYG{n}{at} \PYG{l+m+mh}{0x101b0cad0}\PYG{o}{\PYGZgt{}}
\PYG{n}{tell}\PYG{p}{:} \PYG{l+m+mh}{0x00065a8e} \PYG{n+nb}{type}\PYG{o}{=}\PYG{l+m+mi}{128} \PYG{o}{\PYGZlt{}}\PYG{n}{TotalDepth}\PYG{o}{.}\PYG{n}{LIS}\PYG{o}{.}\PYG{n}{core}\PYG{o}{.}\PYG{n}{FileIndexer}\PYG{o}{.}\PYG{n}{IndexFileHead} \PYG{n+nb}{object} \PYG{n}{at} \PYG{l+m+mh}{0x101b0cb50}\PYG{o}{\PYGZgt{}}
\PYG{n}{tell}\PYG{p}{:} \PYG{l+m+mh}{0x00065ad8} \PYG{n+nb}{type}\PYG{o}{=} \PYG{l+m+mi}{34} \PYG{n}{name}\PYG{o}{=}\PYG{l+s+sa}{b}\PYG{l+s+s1}{\PYGZsq{}}\PYG{l+s+s1}{CONS}\PYG{l+s+s1}{\PYGZsq{}} \PYG{o}{\PYGZlt{}}\PYG{n}{TotalDepth}\PYG{o}{.}\PYG{n}{LIS}\PYG{o}{.}\PYG{n}{core}\PYG{o}{.}\PYG{n}{FileIndexer}\PYG{o}{.}\PYG{n}{IndexTable} \PYG{n+nb}{object} \PYG{n}{at} \PYG{l+m+mh}{0x101b0cbd0}\PYG{o}{\PYGZgt{}}
\PYG{o}{\PYGZlt{}}\PYG{n}{TotalDepth}\PYG{o}{.}\PYG{n}{LIS}\PYG{o}{.}\PYG{n}{core}\PYG{o}{.}\PYG{n}{LogPass}\PYG{o}{.}\PYG{n}{LogPass} \PYG{n+nb}{object} \PYG{n}{at} \PYG{l+m+mh}{0x101b0cfd0}\PYG{o}{\PYGZgt{}}
\PYG{n}{tell}\PYG{p}{:} \PYG{l+m+mh}{0x000d2c44} \PYG{n+nb}{type}\PYG{o}{=}\PYG{l+m+mi}{129} \PYG{o}{\PYGZlt{}}\PYG{n}{TotalDepth}\PYG{o}{.}\PYG{n}{LIS}\PYG{o}{.}\PYG{n}{core}\PYG{o}{.}\PYG{n}{FileIndexer}\PYG{o}{.}\PYG{n}{IndexFileTail} \PYG{n+nb}{object} \PYG{n}{at} \PYG{l+m+mh}{0x101b10450}\PYG{o}{\PYGZgt{}}
\PYG{o}{==}\PYG{o}{==}\PYG{o}{==}\PYG{o}{==}\PYG{o}{==}\PYG{o}{==}\PYG{o}{==}\PYG{o}{==}\PYG{o}{==}\PYG{o}{==}\PYG{o}{==}\PYG{o}{==}\PYG{o}{==}\PYG{o}{==}\PYG{o}{=} \PYG{n}{All} \PYG{n}{records} \PYG{n}{DONE} \PYG{o}{==}\PYG{o}{==}\PYG{o}{==}\PYG{o}{==}\PYG{o}{==}\PYG{o}{==}\PYG{o}{==}\PYG{o}{==}\PYG{o}{==}\PYG{o}{==}\PYG{o}{==}\PYG{o}{==}\PYG{o}{==}\PYG{o}{==}
\end{sphinxVerbatim}

Note that the indexing of a Log Pass is separate since it covers an EFLR and zero or more IFLRs. This is the subject of the next section.


\paragraph{Indexing a Log Pass}
\label{\detokenize{tech/LIS_indexing:indexing-a-log-pass}}
A Log Pass is described by an EFLR and zero or more IFLRs. In LIS terms this means a DFSR and the associated binary IFLRs. This is a candidate for fairly agressive optimisation since:
\begin{itemize}
\item {} 
In practice the Logical Records containing the IFLRs are normally adjacent and regular (same size).

\item {} 
The frame structure is invariant.

\end{itemize}

Supposing a Log Pass has 18 data channels using representation code 68 (frame length 72 bytes) and each IFLR record contains 24 frames (24*72=1728 bytes). If there are 72 Logical Records then it is fairly easy to to calculate where, in logical space, logical record x is in the file; start + x * 1728 %
\begin{footnote}[1]\sphinxAtStartFootnote
Illustrative only, it is slightly more complicated than this.
%
\end{footnote}.

NOTE: The following describes some of the internals of Log Pass indexing, it is for information only as the LogPass and IndexLogPass objects do this automatically.


\subparagraph{Run Length Encoding of Logical Records}
\label{\detokenize{tech/LIS_indexing:run-length-encoding-of-logical-records}}
Run Length Encoding (RLE) is an encoding system that is highly efficient at describing regularly spaced intervals. Since LIS is primarily a recording format the recording software will create and suitably sized buffer for an integer number of frames and other environmental parameters. During recording the buffer will be filled frame by frame. When the buffer is full the buffer will be flushed to file. So a continuos series of IFLRs tends to consist of a series of adjacent Logical Records of the same size followed, possibly, by a single terminating Logical Record that is shorter (by an integer number of frames) than the others.


\subparagraph{Module}
\label{\detokenize{tech/LIS_indexing:id2}}
The Python module that performs RLE of Logical Records is \sphinxcode{TotalDepth.LIS.core.Rle}

For reference documentation see: {\hyperref[\detokenize{ref/LIS/core/Rle:totaldepth-lis-core-rle}]{\sphinxcrossref{\DUrole{std,std-ref}{Run Length Encoding Module}}}}.


\subparagraph{Indexing Frame Data}
\label{\detokenize{tech/LIS_indexing:indexing-frame-data}}
This describes how an index is created to find arbitrary values in a Logical Record containing Frame Data.

Supposing a Log Pass has 18 data channels using representation code 68 (frame length 72 bytes) and each IFLR record contains 24 frames. If there are 72 Logical Records then it is a simple %
\begin{footnote}[2]\sphinxAtStartFootnote
Well not that simple, there is indirect X-axis to take into account and several other things.
%
\end{footnote} task to calculate where, in logical space, the bytes for channel x and frame y are.


\subparagraph{Module}
\label{\detokenize{tech/LIS_indexing:id4}}
The Python module that performs indexing within a Logical Record containing frame data is \sphinxcode{TotalDepth.LIS.core.Type01Plan}.

For reference documentation see: {\hyperref[\detokenize{ref/LIS/core/Type01Plan:totaldepth-lis-core-type01plan}]{\sphinxcrossref{\DUrole{std,std-ref}{Type01Plan}}}}.


\subsubsection{Indexing Performance}
\label{\detokenize{tech/LIS_indexing:indexing-performance}}
As mentioned above the cost of indexing can be measured with these independent measures:
\begin{itemize}
\item {} 
The time to creating an index.

\item {} 
The space required by the index, in-memory or serialised in some form.

\item {} 
The time per byte to access N bytes of data.

\item {} 
Any deviation from O(N) performance, N being total file size or size of data read.

\end{itemize}

The following data was measured using 300+ LIS files totalling of around 300+Mb, the largest being around 50Mb. The LIS module \sphinxcode{TotalDepth.LIS.RandomFrameSetRead.py} was used to conduct the tests.

The original targets for the cost of TotalDepth LIS indexing were:


\begin{savenotes}\sphinxattablestart
\centering
\begin{tabulary}{\linewidth}[t]{|T|T|}
\hline
\sphinxstylethead{\sphinxstyletheadfamily 
Cost ms/Mb
\unskip}\relax &\sphinxstylethead{\sphinxstyletheadfamily 
Result
\unskip}\relax \\
\hline
\textless{}15
&
Excellent
\\
\hline
15-50
&
Good
\\
\hline
50-100
&
Satisfactory
\\
\hline
\textgreater{}100
&
Unsatisfactory
\\
\hline
\end{tabulary}
\par
\sphinxattableend\end{savenotes}


\paragraph{Index Time}
\label{\detokenize{tech/LIS_indexing:index-time}}
This is simply the actual time taken to create a file level index:

\noindent\sphinxincludegraphics{{IndexTime}.png}

The best fit of all points is the cyan line (index(x) on the plot legend). The best fit for the lines is:


\begin{savenotes}\sphinxattablestart
\centering
\begin{tabulary}{\linewidth}[t]{|T|T|T|T|T|T|}
\hline
\sphinxstylethead{\sphinxstyletheadfamily 
Line
\unskip}\relax &\sphinxstylethead{\sphinxstyletheadfamily 
Description
\unskip}\relax &\sphinxstylethead{\sphinxstyletheadfamily 
Colour
\unskip}\relax &\sphinxstylethead{\sphinxstyletheadfamily 
Latency (ms)
\unskip}\relax &\sphinxstylethead{\sphinxstyletheadfamily 
Cost (ms/Mb)
\unskip}\relax &\sphinxstylethead{\sphinxstyletheadfamily 
Result
\unskip}\relax \\
\hline
PR\_1kB(x)
&
Physical Record length = 1kb
&
Green
&
1.3
&
80
&
Satisfactory
\\
\hline
PR\_8kB(x)
&
Physical Record length = 8kb
&
Blue
&
1.2
&
12
&
Excellent
\\
\hline
PR\_DontKnow
&
Upper bound, worst case
&
Magenta
&
3.0
&
90
&
Satisfactory
\\
\hline
index(x)
&
Regression fit on all points
&
Cyan
&
1.2
&
39.9
&
Good
\\
\hline
\end{tabulary}
\par
\sphinxattableend\end{savenotes}

However there are some other trends that can be teased out when separating files that have Physical Records of 1kb and those in the data set that have 8kb Physical Records (there were no larger Physical Record sizes in the data set). The green fit (PR\_1kb(x) on the plot legend) is the best fit for those files that have Physical Records of 1kb. The blue line (PR\_8kb(x) on the plot legend) is the best fit for files that have Physical Records of 8kb, it is almost exactly 8 times faster.

In fact in other tests, not presented here, it was shown that there was a linear speedup with Physical Record size up to their maximum size of 64kb. In fact the dominant factor in indexing time was the \sphinxstyleemphasis{number of physical records}. This suggests a couple of things:
\begin{itemize}
\item {} 
Before archiving or processing LIS files rewrite them to their maximum Physical Record size (64kb). This will pay off later when indexing prior to any read operation.

\item {} 
Rewrite the Physical Record handler (and below) in C or Cython as it seems to be the bottleneck for indexing.

\end{itemize}

This is actual time taken to create a file level index divided by the file size. This gives some measure of ‘cost’ which is defined here as ms per Mb of data processed, here the size in Mb is the total file size:

\noindent\sphinxincludegraphics{{IndexCost}.png}

The rising costs for smaller files is no real cause for concern, this represents the startup cost of indexing and reading of around 1 to 2 ms. This graph clearly demonstrates O(N), or better, behaviour.

Again the advantage in having 8kb Physical Records is evident in the lower right hand corner. The cost would be further reduced to around 1 to 1.5 ms/Mb by having 64kb Physical Records.


\paragraph{Multi-Processing}
\label{\detokenize{tech/LIS_indexing:multi-processing}}\label{\detokenize{tech/LIS_indexing:totaldepth-tech-indexing-indexsize}}
Most of TotalDepth software supports parallel processing. The LIS command line tool Index.py can create indexes in parallel. The following graph shows the file size plotted against total time to index when indexing around 300 LIS files with different numbers of simultaneous processes. The red line is the best fit for single process indexing that costs about 40 ms/Mb:

\noindent\sphinxincludegraphics{{IndexMultiProcessing}.png}

The index time rises slightly with increasing number of processes and the scatter rises dramatically with 8 processes presumably because of I/O contention.

Actual wall clock times for indexing all those LIS files for different numbers of simultaneous processes on a four core machine with hyper-threading are:


\begin{savenotes}\sphinxattablestart
\centering
\begin{tabulary}{\linewidth}[t]{|T|T|T|}
\hline
\sphinxstylethead{\sphinxstyletheadfamily 
Processes
\unskip}\relax &\sphinxstylethead{\sphinxstyletheadfamily 
Total time to index (s)
\unskip}\relax &\sphinxstylethead{\sphinxstyletheadfamily 
Improvement over No multiprocessing
\unskip}\relax \\
\hline
No multiprocessing
&
18.4
&
Datum
\\
\hline
1
&
18.3
&
0.995
\\
\hline
2
&
9.9
&
0.538
\\
\hline
4
&
8.9
&
0.484
\\
\hline
8
&
7.6
&
0.413
\\
\hline
\end{tabulary}
\par
\sphinxattableend\end{savenotes}

Moving to two processes gives an almost linear speedup, moving from two to four or eight processes gives only slight improvement presumably because the execution time becomes I/O bound.


\paragraph{Index Size}
\label{\detokenize{tech/LIS_indexing:index-size}}
It is envisaged that the index will be persisted in some form. Once persisted then the LIS file would only be accessed via the index, any file write operation requires a suitable adjustment to the index.

Persistence techniques could be, for example:
\begin{itemize}
\item {} 
Stored in a database.

\item {} 
Serialised in binary form onto the file system.

\item {} 
Serialised in human readable form, such as XML, onto the file system or database %
\begin{footnote}[3]\sphinxAtStartFootnote
JSON can not handle the index in its current form.
%
\end{footnote}.

\item {} 
Serialised in binary form and attached as a Logical Record to the LIS file it refers to.

\end{itemize}


\subparagraph{TotalDepth and SaaS}
\label{\detokenize{tech/LIS_indexing:totaldepth-and-saas}}
There is a further imperative to understanding index size; if TotalDepth were to be used, as it was always intended, as Software as a Service (SaaS) where the bulk of the processing is with the data file on the client machine and the processing done on the server then part of the bootstrap process of any transaction is for the client to index the file and send the index to the server. In that case it is important to keep the index size small.

Thus the size of the index content is a significant consideration.

The following graph measures the size of the index when serialised with Python’s \sphinxcode{cPickle} module, Pickled size is in kb (red markers, left scale) and compared as a percentage of the original file size (green markers, right scale):

\noindent\sphinxincludegraphics{{PickleSize}.png}

The pickled size shows a wide range that is representative of the wide range of LIS inputs. The best fit for size (albeit with a large scatter) is 5kb + 0.6\% of the file size which is entirely satisfactory {[}the upper bound being 42kb + 2.3\% and the lower bound being just 1.2kb + 0.08\% with larger files tending towards the lower bound{]}.

The relative size of the index shows a strong downward trend (blue line) for files below 4Mb, before levelling off at the 0.1 to 3\% mark. This is quite satisfactory for the use cases described above including SaaS. This graph clearly demonstrates O(N), or better, behaviour.


\paragraph{Indexing Performance Improvements}
\label{\detokenize{tech/LIS_indexing:indexing-performance-improvements}}\label{\detokenize{tech/LIS_indexing:totaldepth-tech-indexing-perf-improve}}
As noted above there is a substantial improvement in indexing when large Physical Record sizes are used.

It is also likely that significant improvement could be made if the RawStream, TifMarker and PhysRec were to be rewritten in C/C++ or Cython. The PhysRec module has a dependency on the RepCode module (easily removed). All three modules have a dependency on the struct module so the limited functionality that they use from there would have to be reproduced, thats pretty easy since it only involves integer manipulation.

So given an ‘average’ cost of indexing of 40 ms/Mb (i.e. a ‘good’ rating) the performance improvements could be:
\begin{itemize}
\item {} 
Moving to larger Physical Record sizes: x8?

\item {} 
Integrate the existing (in another project) code in C that handles Raw Stream/TIF/Physical Record handling into this project. This is known to be about x100 faster (and the index has a lower memory footprint).

\end{itemize}

The performance improvements would not necessarily combine as they are mutually dependent but the combination might reduce the cost to around 2 to 8 ms/MB, an exceptionally good performance indeed.

See {\hyperref[\detokenize{tech/performance:totaldepth-tech-perf-improve}]{\sphinxcrossref{\DUrole{std,std-ref}{Performance Improvements}}}} for other performance improvements


\paragraph{LIS Read Performance via an Index}
\label{\detokenize{tech/LIS_indexing:lis-read-performance-via-an-index}}
This is described here {\hyperref[\detokenize{tech/performance:totaldepth-tech-lis-read-perf}]{\sphinxcrossref{\DUrole{std,std-ref}{Performance of TotalDepth}}}}


\paragraph{Summary}
\label{\detokenize{tech/LIS_indexing:summary}}
Indexing is not free, it incurs an overhead, but this overhead is acceptable. The overhead is worst for small data sizes where the performance is high in any case. The overhead is low, and the benefit is very great for large or complex data sizes where the performance, without indexing, could be very poor indeed.


\subsection{Performance of TotalDepth}
\label{\detokenize{tech/performance:totaldepth-tech-lis-read-perf}}\label{\detokenize{tech/performance::doc}}\label{\detokenize{tech/performance:performance-of-totaldepth}}
This technical note presents the results of some actual performance tests on TotalDepth.


\subsubsection{Measuring Performance}
\label{\detokenize{tech/performance:measuring-performance}}
This describes some principles used in establishing TotalDepth’s performance and setting performance targets.


\paragraph{User’s Perception of Performance}
\label{\detokenize{tech/performance:user-s-perception-of-performance}}
Users only want to pay for what they get and experienced users have a rough idea of the cost of what they ask the software to do. For example most users would regard these a cheap operations, and would not expect them to take much time. If they did the user is likely to regard the application as ‘slow’:
\begin{itemize}
\item {} 
Load file

\item {} 
Show log header

\item {} 
Plot small section

\end{itemize}

Most users would regard these more expensive operations and is more likely to accept that they would take more time.:
\begin{itemize}
\item {} 
Cross correlate multiple curves

\item {} 
Dipmeter processing

\item {} 
Deconvolution

\end{itemize}

Most users are aware of the size of the data set with which the are operating and appreciate that operations on larger data sets take longer time. Users do not appreciate O(N $^{\text{2}}$) or worse behaviour. We don’t like it either %
\begin{footnote}[1]\sphinxAtStartFootnote
It is very easy for software developers to fail to see this kind of behaviour. For example if the time for an operation is: a + b * N + c * N $^{\text{2}}$ and c \textless{}\textless{} b. If the software test suit tests an insufficiently small size of N then it appears that the operation is O(N). Along comes a user with a large data set and they see O(N $^{\text{2}}$) behaviour. This (or worse) is quite commonly observed in many software products.
%
\end{footnote}.

TotalDepth measures the cost of operations, in:

\begin{sphinxVerbatim}[commandchars=\\\{\}]
\PYG{n}{Execution} \PYG{n}{time} \PYG{p}{(}\PYG{n}{ms}\PYG{p}{)}
\PYG{o}{\PYGZhy{}}\PYG{o}{\PYGZhy{}}\PYG{o}{\PYGZhy{}}\PYG{o}{\PYGZhy{}}\PYG{o}{\PYGZhy{}}\PYG{o}{\PYGZhy{}}\PYG{o}{\PYGZhy{}}\PYG{o}{\PYGZhy{}}\PYG{o}{\PYGZhy{}}\PYG{o}{\PYGZhy{}}\PYG{o}{\PYGZhy{}}\PYG{o}{\PYGZhy{}}\PYG{o}{\PYGZhy{}}\PYG{o}{\PYGZhy{}}\PYG{o}{\PYGZhy{}}\PYG{o}{\PYGZhy{}}\PYG{o}{\PYGZhy{}}\PYG{o}{\PYGZhy{}}\PYG{o}{\PYGZhy{}}
\PYG{n}{Size} \PYG{n}{of} \PYG{n+nb}{input} \PYG{p}{(}\PYG{n}{Mb}\PYG{p}{)}
\end{sphinxVerbatim}


\subparagraph{Example of LIS cost}
\label{\detokenize{tech/performance:example-of-lis-cost}}
Our measure is ms/Mb of input. 1Mb of LIS data is typically 250,000 values, or, to put this in context 200 feet of 10 curves (6” sampling) is 0.015 Mb. So, just as an example, the cost of plotting such data from a 20Mb file \sphinxstyleemphasis{might} work out as:


\begin{savenotes}\sphinxattablestart
\centering
\begin{tabular}[t]{|*{5}{\X{1}{5}|}}
\hline
\sphinxstylethead{\sphinxstyletheadfamily 
Operation
\unskip}\relax &\sphinxstylethead{\sphinxstyletheadfamily 
Cost (ms/Mb)
\unskip}\relax &\sphinxstylethead{\sphinxstyletheadfamily 
Data Size (Mb)
\unskip}\relax &\sphinxstylethead{\sphinxstyletheadfamily 
User’s Time (ms)
\unskip}\relax &\sphinxstylethead{\sphinxstyletheadfamily 
Notes
\unskip}\relax \\
\hline
Index a 20Mb file
&
12
&
20
&
240
&
One-off exercise
\\
\hline
Reading 200 feet, 10 curves
&
1500
&
0.015
&
22
&\\
\hline
Plotting what has been read
&
4000
&
0.015
&
60
&\\
\hline
\sphinxstylestrong{Total}
&\begin{itemize}
\item {} 
\end{itemize}
&\begin{itemize}
\item {} 
\end{itemize}
&
\sphinxstylestrong{322}
&\\
\hline
\end{tabular}
\par
\sphinxattableend\end{savenotes}

The rest of this tech note describes the performance of reading LIS files in these ms/Mb terms.


\subsubsection{Actual LIS Performance}
\label{\detokenize{tech/performance:actual-lis-performance}}
Actual performance results of some LIS file operations.


\paragraph{File Read Cost}
\label{\detokenize{tech/performance:file-read-cost}}
The measure here is low level reading and conversion from binary words to internal number formats. Three operations are being measured:
\begin{itemize}
\item {} 
Raw read speed of bytes.

\item {} 
Above, and converting to internal integers and floats.

\item {} 
Above and converting from Rep Code 68 to an internal float.

\end{itemize}

The input data was about 100 binary files with random data of around 100 Mb in total.

The x-axis of the graph below is the buffer length, in other words how much data is swallowed in a single atomic read.

\noindent\sphinxincludegraphics{{ReadUnpackRc68}.png}


\subparagraph{Raw File Read}
\label{\detokenize{tech/performance:raw-file-read}}
The ‘Read all’ graph shows the cost of reading bytes at different buffer sizes from a file and discarding it. This is the baseline and represents the crude storage access time. Large buffer sizes achieve \textless{}1 ms/Mb (1 Gb/s).


\subparagraph{Raw File seek()}
\label{\detokenize{tech/performance:raw-file-seek}}
The ‘Seek all’ graph shows the cost of traversing a file using seek() at different seek lengths. No data is read. Combined with the read information this can be used to estimate the minimum time to make a series of read/seek operations.


\subparagraph{Using the \sphinxstyleliteralintitle{struct} module}
\label{\detokenize{tech/performance:using-the-struct-module}}
The “Read int (compiled)” and “Read float (compiled)” data shows the cost of reading bytes into a buffer and then converting them to a list of integers and floats using the \sphinxcode{struct} module (the type declaration is compiled). They show an asymptotic cost, in addition to the read cost, of 8 and 3 ms/Mb respectively. This shows very good performance for reading builtin types with the struct module. As most RP66 types are builtins (or types supported by the struct module) then the struct module should be used for RP66.

It is slightly surprising that reading an integer is slower than reading a floating point number.


\subparagraph{Reading Rep Code 68}
\label{\detokenize{tech/performance:reading-rep-code-68}}
This establishes the cost of converting a bytes object in Representation Code 68 to a internal floating point value.

The cost of using Python code and Cython code is examined and in each case the the total cost or read+convert is presented and then the integer \sphinxcode{struct} read cost (above) is subtracted to get the conversion cost.


\subparagraph{Python}
\label{\detokenize{tech/performance:python}}
The data sets “Read Rep Code 68 (Python)” and “Convert to Rep Code 68 (Python)” plots to the cost of reading bytes and converting them to Representation code 68, the latter curve has the read cost subtracted (from above) so it represents merely the conversion cost of in-memory from bytes to Rep Code 68. This is asymptotic to around 270 ms/Mb.


\subparagraph{Cython}
\label{\detokenize{tech/performance:cython}}
The data sets “Read Rep Code 68 (Cython)” and “Convert to Rep Code 68 (Cython)” is as immediately above but using Cython code rather than Python code. This is asymptotic to around 72 ms/Mb. This is significantly longer than using the struct module to interpret a IEEE float so it could be that some improvement could be made with a pure C implementation.


\subparagraph{Cython vs Python}
\label{\detokenize{tech/performance:cython-vs-python}}
The “Cython/Python cost (right axis)” plots the ratio of converting Rep Code 68 with Python and Cython. The Cython code takes 0.27 of the cost of the Python code. Well worth it.


\paragraph{Frame Read Cost}
\label{\detokenize{tech/performance:frame-read-cost}}
The operation being timed here is, given a LIS file index, read a certain set of frame data then convert each value to an internal floating point number and then populate the internal FrameSet object. This is a basic operation before doing any plotting, channel editing, recalibration etc. The amount of data read is the cumulative size in LIS bytes of all the values read from the file. 250,000 values would constitute typically 1Mb of LIS data.

A number of different methods were use to read the LIS data:
\begin{itemize}
\item {} 
Method A: All frames all channels. Here indexing provides no optimisation as it is a simple sequential read operation of everything.

\item {} 
Method C: A sequential subset of the frames and a sequential subset of the channels. This requires a logical seek operation to the start of the frames then, within each frame a seek then read then seek operation.

\item {} 
Method D: A sequential subset of the frames and a non-sequential subset of the channels. This requires a logical seek operation to the start of the frames then, within each frame multiple seek/read operations.

\item {} 
Method E: A non-sequential subset of the frames and a non-sequential subset of the channels. This requires a set of logical seek/read operations to each frame then, within each frame multiple seek/read operations.

\end{itemize}

Indexing can provide an optimisation for methods C-E as it hugely reduces the amount of read operations to that just sufficient for the required frames and channels. The important measure is that the cost of indexing, computing and executing seek/read operations does not outweigh the reduction in data read.

For example if a frame has 40 channels and only 5 are required then, if the cost per byte stays the same then the user sees a 8 fold speed improvement. If the cost per byte doubles then the user still sees a 4 fold speed improvement. If the cost per byte rises 8 fold there is no advantage over doing a sequential read of all channels. Beyond an eight fold rise in indexing cost the user sees a performance \sphinxstyleemphasis{reduction}.


\subparagraph{A: All Frames, all Channels}
\label{\detokenize{tech/performance:a-all-frames-all-channels}}
This is the baseline, indexing plays no part as this is a straightforward sequential read operation of every channel in every frame.

\noindent\sphinxincludegraphics{{ReadCost_A}.png}

Although there is quite a spread there is, gratifyingly, no obvious trend for the cost to increase as the file size does which would indicate worse than O(N) behaviour. The behaviour is clearly O(N). The cost in ms/Mb is:


\begin{savenotes}\sphinxattablestart
\centering
\begin{tabulary}{\linewidth}[t]{|T|T|}
\hline
\sphinxstylethead{\sphinxstyletheadfamily 
A:
\unskip}\relax &\sphinxstylethead{\sphinxstyletheadfamily 
Cost ms/Mb
\unskip}\relax \\
\hline
Minimum
&
541.8
\\
\hline
Mean
&
1446.9
\\
\hline
Maximum
&
4003.3
\\
\hline
\end{tabulary}
\par
\sphinxattableend\end{savenotes}


\subparagraph{C: Adjacent Frames and Channels}
\label{\detokenize{tech/performance:c-adjacent-frames-and-channels}}
This is taking a random, but sequential, subset of the frames and a random, but sequential, subset of the channels. This requires the indexer to do a logical seek operation to the start of the frames then, within each frame a seek then read then seek operation.

The results are plotted compared to the baseline and the ratio of the cost of C relative to the cost of A for each file is plotted on the right hand axis.

\noindent\sphinxincludegraphics{{ReadCost_A-C}.png}

The cost in ms/Mb is:


\begin{savenotes}\sphinxattablestart
\centering
\begin{tabulary}{\linewidth}[t]{|T|T|T|T|}
\hline
\sphinxstylethead{\sphinxstyletheadfamily 
Value
\unskip}\relax &\sphinxstylethead{\sphinxstyletheadfamily 
A: Cost ms/Mb
\unskip}\relax &\sphinxstylethead{\sphinxstyletheadfamily 
C: Cost ms/Mb
\unskip}\relax &\sphinxstylethead{\sphinxstyletheadfamily 
Ratio C/A
\unskip}\relax \\
\hline
Minimum
&
541.8
&
545.7
&
0.57
\\
\hline
\sphinxstylestrong{Mean}
&
\sphinxstylestrong{1446.9}
&
\sphinxstylestrong{1561.3}
&
\sphinxstylestrong{1.13}
\\
\hline
Maximum
&
4003.3
&
4485.8
&
4.25
\\
\hline
\end{tabulary}
\par
\sphinxattableend\end{savenotes}

This means that a small increase in cost (13\%) is the price of a hugely reduced data set.


\subparagraph{D: Adjacent Frames, any Channels}
\label{\detokenize{tech/performance:d-adjacent-frames-any-channels}}
This is taking a random, but sequential, subset of the frames and a random, but non-sequential, subset of the channels. This requires the indexer to do a logical seek operation to the start of the frames then, within each frame multiple seek/read operations.

This is highly representative of the indexing operation performed when plotting, say, any 200 foot section of a number of selected channels.

The results are plotted compared to the baseline and the ratio of the cost of D relative to the cost of A for each file is plotted on the right hand axis.

\noindent\sphinxincludegraphics{{ReadCost_A-D}.png}

The cost in ms/Mb is:


\begin{savenotes}\sphinxattablestart
\centering
\begin{tabulary}{\linewidth}[t]{|T|T|T|T|}
\hline
\sphinxstylethead{\sphinxstyletheadfamily 
Value
\unskip}\relax &\sphinxstylethead{\sphinxstyletheadfamily 
A: Cost ms/Mb
\unskip}\relax &\sphinxstylethead{\sphinxstyletheadfamily 
D: Cost ms/Mb
\unskip}\relax &\sphinxstylethead{\sphinxstyletheadfamily 
Ratio D/A
\unskip}\relax \\
\hline
Minimum
&
541.8
&
581.8
&
0.98
\\
\hline
\sphinxstylestrong{Mean}
&
\sphinxstylestrong{1446.9}
&
\sphinxstylestrong{3410.5}
&
\sphinxstylestrong{2.55}
\\
\hline
Maximum
&
4003.3
&
11928.6
&
7.44
\\
\hline
\end{tabulary}
\par
\sphinxattableend\end{savenotes}

It should be noted that this test is deliberately harsh in that non-sequential channels are always chosen. If some of the channels are adjacent then the cost reduces, for those channels, more towards C (typically half the cost of D).

Evens so provided the number of channels of interest is less than 40\% of the total indexing this way provides a real speed improvement. The important point being this may not produce an improvement (and may actually be slower) for small frame sizes; they are not the problem precisely because they are small. For the seriously problematic large frame sizes then indexing will always be faster, often dramatically so.


\subparagraph{E: Any Frames, any Channels}
\label{\detokenize{tech/performance:e-any-frames-any-channels}}
This is taking a random, non-sequential, subset of the frames and a random, but non-sequential, subset of the channels. This requires the indexer to perform a set of logical seek/read operations to each frame then, within each frame multiple seek/read operations.

Essentially this is the same as test D but skipping intermediate frames.

The results are plotted compared to the baseline and the ratio of the cost of E relative to the cost of A for each file is plotted on the right hand axis.

\noindent\sphinxincludegraphics{{ReadCost_A-E}.png}

The cost in ms/Mb is:


\begin{savenotes}\sphinxattablestart
\centering
\begin{tabulary}{\linewidth}[t]{|T|T|T|T|}
\hline
\sphinxstylethead{\sphinxstyletheadfamily 
Value
\unskip}\relax &\sphinxstylethead{\sphinxstyletheadfamily 
A: Cost ms/Mb
\unskip}\relax &\sphinxstylethead{\sphinxstyletheadfamily 
E: Cost ms/Mb
\unskip}\relax &\sphinxstylethead{\sphinxstyletheadfamily 
Ratio E/A
\unskip}\relax \\
\hline
Minimum
&
541.8
&
619.9
&
1.03
\\
\hline
\sphinxstylestrong{Mean}
&
\sphinxstylestrong{1446.9}
&
\sphinxstylestrong{3491.7}
&
\sphinxstylestrong{2.57}
\\
\hline
Maximum
&
4003.3
&
13552.5
&
7.98
\\
\hline
\end{tabulary}
\par
\sphinxattableend\end{savenotes}

There is a very slight (around 1-5\%) additional cost over D but the huge benefit is being able to skip intermediate frames. This means that a low resolution plot (say 1:500) of a high resolution log (say sampling every 1.2 inches) could read every fifth frame (six inch sampling) and that would be plotted every 0.012” (say 1 pixel) as a speedup of almost 500\%.


\subparagraph{All Measurements}
\label{\detokenize{tech/performance:all-measurements}}
For the sake of completeness, here are all the results:

\noindent\sphinxincludegraphics{{ReadCost}.png}


\paragraph{Performance Improvements}
\label{\detokenize{tech/performance:performance-improvements}}\label{\detokenize{tech/performance:totaldepth-tech-perf-improve}}
The low level performance of TotalDepth is pretty good. FrameSet performance is satisfactory. Further improvement is certain for {\hyperref[\detokenize{tech/LIS_indexing:totaldepth-tech-indexing-perf-improve}]{\sphinxcrossref{\DUrole{std,std-ref}{Indexing Performance Improvements}}}} once the existing C code (in another project) is integrated into this one.

Populating the frame is a costly exercise and the current solution takes this path:

\begin{sphinxVerbatim}[commandchars=\\\{\}]
\PYG{n}{File} \PYG{n+nb}{bytes} \PYG{o}{\PYGZhy{}}\PYG{o}{\PYGZgt{}} \PYG{n}{Cython} \PYG{n}{convert} \PYG{n}{to} \PYG{n}{C} \PYG{n}{double} \PYG{o}{\PYGZhy{}}\PYG{o}{\PYGZgt{}} \PYG{n}{convert} \PYG{n}{to} \PYG{n}{Python} \PYG{n+nb}{float} \PYG{o}{\PYGZhy{}}\PYG{o}{\PYGZgt{}} \PYG{n}{insert} \PYG{n}{into} \PYG{n}{a} \PYG{n}{numpy} \PYG{n}{array}\PYG{o}{.}
\end{sphinxVerbatim}

All this boxing and unboxing is expensive and a faster (but with more code complexity) is to populate the numpy array directly so this all happens in C code:

\begin{sphinxVerbatim}[commandchars=\\\{\}]
\PYG{n}{File} \PYG{n+nb}{bytes} \PYG{o}{\PYGZhy{}}\PYG{o}{\PYGZgt{}} \PYG{n}{Convert} \PYG{n}{to} \PYG{n}{C} \PYG{n}{double} \PYG{o}{\PYGZhy{}}\PYG{o}{\PYGZgt{}} \PYG{n}{copy} \PYG{n}{directly} \PYG{n}{into} \PYG{n}{numpy} \PYG{n}{memory} \PYG{n}{space}
\end{sphinxVerbatim}

This should provide a great speedup.

The SVG creation is also worth looking at.
\phantomsection\label{\detokenize{tech/plotting:totaldepth-tech-plotting}}

\subsection{Plotting Wireline Data}
\label{\detokenize{tech/plotting::doc}}\label{\detokenize{tech/plotting:plotting-wireline-data}}
Aside from the LogPass (the data structure that holds the frame data) the plot layout needs to be specified. There are two ways that TotalDepth does this:
\begin{itemize}
\item {} 
A plot specification that is \sphinxstyleemphasis{integral} to the file. For example; LIS files may well have FILM and PRES tables within them that configure the plot as the recording engineer (or software) intended. In this case TotalDepth can make those plots directly.

\item {} 
A plot specification that is specified \sphinxstyleemphasis{externally} to the file. TotalDepth supports one such way; using LgFormat XML files and these are described below.

\end{itemize}


\subsubsection{Wireline Files With Integral Plot Data}
\label{\detokenize{tech/plotting:wireline-files-with-integral-plot-data}}
This applies to LIS (and possibly RP66 files). The LAS file specification does not describe any plot specification at all.


\paragraph{LIS Plot Specification}
\label{\detokenize{tech/plotting:lis-plot-specification}}
There are a number of table-like Logical Records (type 34) within a Log Pass that can specify the plot layout. As a minimum TotalDepth.LIS needs a \sphinxcode{FILM} and a \sphinxcode{PRES} record.


\subparagraph{FILM Record}
\label{\detokenize{tech/plotting:film-record}}
A dump of a \sphinxcode{FILM} Logical Record might look like this:

\begin{sphinxVerbatim}[commandchars=\\\{\}]
\PYG{n}{Table} \PYG{n}{record} \PYG{p}{(}\PYG{n+nb}{type} \PYG{l+m+mi}{34}\PYG{p}{)} \PYG{n+nb}{type}\PYG{p}{:} \PYG{n}{FILM}

\PYG{n}{MNEM}  \PYG{n}{GCOD}  \PYG{n}{GDEC}  \PYG{n}{DEST}  \PYG{n}{DSCA}
\PYG{o}{\PYGZhy{}}\PYG{o}{\PYGZhy{}}\PYG{o}{\PYGZhy{}}\PYG{o}{\PYGZhy{}}\PYG{o}{\PYGZhy{}}\PYG{o}{\PYGZhy{}}\PYG{o}{\PYGZhy{}}\PYG{o}{\PYGZhy{}}\PYG{o}{\PYGZhy{}}\PYG{o}{\PYGZhy{}}\PYG{o}{\PYGZhy{}}\PYG{o}{\PYGZhy{}}\PYG{o}{\PYGZhy{}}\PYG{o}{\PYGZhy{}}\PYG{o}{\PYGZhy{}}\PYG{o}{\PYGZhy{}}\PYG{o}{\PYGZhy{}}\PYG{o}{\PYGZhy{}}\PYG{o}{\PYGZhy{}}\PYG{o}{\PYGZhy{}}\PYG{o}{\PYGZhy{}}\PYG{o}{\PYGZhy{}}\PYG{o}{\PYGZhy{}}\PYG{o}{\PYGZhy{}}\PYG{o}{\PYGZhy{}}\PYG{o}{\PYGZhy{}}\PYG{o}{\PYGZhy{}}\PYG{o}{\PYGZhy{}}\PYG{o}{\PYGZhy{}}

\PYG{l+m+mi}{1}     \PYG{n}{EEE}   \PYG{o}{\PYGZhy{}}\PYG{o}{\PYGZhy{}}\PYG{o}{\PYGZhy{}}\PYG{o}{\PYGZhy{}}  \PYG{n}{PF2}   \PYG{n}{D200}
\PYG{l+m+mi}{2}     \PYG{n}{E4E}   \PYG{o}{\PYGZhy{}}\PYG{l+m+mi}{4}\PYG{o}{\PYGZhy{}}\PYG{o}{\PYGZhy{}}  \PYG{n}{PF1}   \PYG{n}{DM}
\end{sphinxVerbatim}

The columns are:


\begin{savenotes}\sphinxattablestart
\centering
\begin{tabulary}{\linewidth}[t]{|T|T|}
\hline
\sphinxstylethead{\sphinxstyletheadfamily 
Column
\unskip}\relax &\sphinxstylethead{\sphinxstyletheadfamily 
Description
\unskip}\relax \\
\hline
\sphinxcode{MNEM}
&
The (logical) name of the output, TotalDepth uses this as part of the plot file name.
\\
\hline
\sphinxcode{GCOD}
&
The coding of the tracks. \sphinxcode{E4E} means three tracks which have linear, log (4 decades), linear.
With three tracks the X axis track (depth for example) appears between the first and second tracks.
\\
\hline
\sphinxcode{GDEC}
&
The number of logarithmic decades for each track, \sphinxcode{-} for linear.
\\
\hline
\sphinxcode{DEST}
&
The physical destination (in the logging unit) of the plot. Ignored.
\\
\hline
\sphinxcode{DSCA}
&
The depth scale, encoded. For example \sphinxcode{D200} means 1:200.
\\
\hline
\end{tabulary}
\par
\sphinxattableend\end{savenotes}


\subparagraph{PRES Record}
\label{\detokenize{tech/plotting:pres-record}}
A dump of a \sphinxcode{PRES} Logical Record might look like this:

\begin{sphinxVerbatim}[commandchars=\\\{\}]
 \PYG{n}{Table} \PYG{n}{record} \PYG{p}{(}\PYG{n+nb}{type} \PYG{l+m+mi}{34}\PYG{p}{)} \PYG{n+nb}{type}\PYG{p}{:} \PYG{n}{PRES}

\PYG{n}{MNEM}  \PYG{n}{OUTP}  \PYG{n}{STAT}  \PYG{n}{TRAC}  \PYG{n}{CODI}  \PYG{n}{DEST}  \PYG{n}{MODE}      \PYG{n}{FILT}          \PYG{n}{LEDG}          \PYG{n}{REDG}
\PYG{o}{\PYGZhy{}}\PYG{o}{\PYGZhy{}}\PYG{o}{\PYGZhy{}}\PYG{o}{\PYGZhy{}}\PYG{o}{\PYGZhy{}}\PYG{o}{\PYGZhy{}}\PYG{o}{\PYGZhy{}}\PYG{o}{\PYGZhy{}}\PYG{o}{\PYGZhy{}}\PYG{o}{\PYGZhy{}}\PYG{o}{\PYGZhy{}}\PYG{o}{\PYGZhy{}}\PYG{o}{\PYGZhy{}}\PYG{o}{\PYGZhy{}}\PYG{o}{\PYGZhy{}}\PYG{o}{\PYGZhy{}}\PYG{o}{\PYGZhy{}}\PYG{o}{\PYGZhy{}}\PYG{o}{\PYGZhy{}}\PYG{o}{\PYGZhy{}}\PYG{o}{\PYGZhy{}}\PYG{o}{\PYGZhy{}}\PYG{o}{\PYGZhy{}}\PYG{o}{\PYGZhy{}}\PYG{o}{\PYGZhy{}}\PYG{o}{\PYGZhy{}}\PYG{o}{\PYGZhy{}}\PYG{o}{\PYGZhy{}}\PYG{o}{\PYGZhy{}}\PYG{o}{\PYGZhy{}}\PYG{o}{\PYGZhy{}}\PYG{o}{\PYGZhy{}}\PYG{o}{\PYGZhy{}}\PYG{o}{\PYGZhy{}}\PYG{o}{\PYGZhy{}}\PYG{o}{\PYGZhy{}}\PYG{o}{\PYGZhy{}}\PYG{o}{\PYGZhy{}}\PYG{o}{\PYGZhy{}}\PYG{o}{\PYGZhy{}}\PYG{o}{\PYGZhy{}}\PYG{o}{\PYGZhy{}}\PYG{o}{\PYGZhy{}}\PYG{o}{\PYGZhy{}}\PYG{o}{\PYGZhy{}}\PYG{o}{\PYGZhy{}}\PYG{o}{\PYGZhy{}}\PYG{o}{\PYGZhy{}}\PYG{o}{\PYGZhy{}}\PYG{o}{\PYGZhy{}}\PYG{o}{\PYGZhy{}}\PYG{o}{\PYGZhy{}}\PYG{o}{\PYGZhy{}}\PYG{o}{\PYGZhy{}}\PYG{o}{\PYGZhy{}}\PYG{o}{\PYGZhy{}}\PYG{o}{\PYGZhy{}}\PYG{o}{\PYGZhy{}}\PYG{o}{\PYGZhy{}}\PYG{o}{\PYGZhy{}}\PYG{o}{\PYGZhy{}}\PYG{o}{\PYGZhy{}}\PYG{o}{\PYGZhy{}}\PYG{o}{\PYGZhy{}}\PYG{o}{\PYGZhy{}}\PYG{o}{\PYGZhy{}}\PYG{o}{\PYGZhy{}}\PYG{o}{\PYGZhy{}}\PYG{o}{\PYGZhy{}}\PYG{o}{\PYGZhy{}}\PYG{o}{\PYGZhy{}}\PYG{o}{\PYGZhy{}}\PYG{o}{\PYGZhy{}}\PYG{o}{\PYGZhy{}}\PYG{o}{\PYGZhy{}}\PYG{o}{\PYGZhy{}}\PYG{o}{\PYGZhy{}}\PYG{o}{\PYGZhy{}}\PYG{o}{\PYGZhy{}}\PYG{o}{\PYGZhy{}}\PYG{o}{\PYGZhy{}}\PYG{o}{\PYGZhy{}}\PYG{o}{\PYGZhy{}}

\PYG{n}{SP}    \PYG{n}{SP}    \PYG{n}{ALLO}  \PYG{n}{T1}    \PYG{n}{LLIN}  \PYG{l+m+mi}{1}     \PYG{n}{SHIF}      \PYG{l+m+mf}{0.500000}      \PYG{o}{\PYGZhy{}}\PYG{l+m+mf}{80.0000}       \PYG{l+m+mf}{20.0000}
\PYG{n}{CALI}  \PYG{n}{CALI}  \PYG{n}{ALLO}  \PYG{n}{T1}    \PYG{n}{LDAS}  \PYG{l+m+mi}{1}     \PYG{n}{SHIF}      \PYG{l+m+mf}{0.500000}       \PYG{l+m+mf}{5.00000}       \PYG{l+m+mf}{15.0000}
\PYG{n}{MINV}  \PYG{n}{MINV}  \PYG{n}{DISA}  \PYG{n}{T1}    \PYG{n}{LLIN}  \PYG{l+m+mi}{1}     \PYG{n}{SHIF}      \PYG{l+m+mf}{0.500000}       \PYG{l+m+mf}{30.0000}       \PYG{l+m+mf}{0.00000}
\PYG{n}{MNOR}  \PYG{n}{MNOR}  \PYG{n}{DISA}  \PYG{n}{T1}    \PYG{n}{LDAS}  \PYG{l+m+mi}{1}     \PYG{n}{SHIF}      \PYG{l+m+mf}{0.500000}       \PYG{l+m+mf}{30.0000}       \PYG{l+m+mf}{0.00000}
\PYG{n}{LLD}   \PYG{n}{LLD}   \PYG{n}{ALLO}  \PYG{n}{T23}   \PYG{n}{LDAS}  \PYG{l+m+mi}{1}     \PYG{n}{GRAD}      \PYG{l+m+mf}{0.500000}      \PYG{l+m+mf}{0.200000}       \PYG{l+m+mf}{2000.00}
\PYG{n}{LLDB}  \PYG{n}{LLD}   \PYG{n}{ALLO}  \PYG{n}{T2}    \PYG{n}{HDAS}  \PYG{l+m+mi}{1}     \PYG{n}{GRAD}      \PYG{l+m+mf}{0.500000}       \PYG{l+m+mf}{2000.00}       \PYG{l+m+mf}{200000.}
\PYG{n}{LLG}   \PYG{n}{LLG}   \PYG{n}{DISA}  \PYG{n}{T23}   \PYG{n}{LDAS}  \PYG{l+m+mi}{1}     \PYG{n}{GRAD}      \PYG{l+m+mf}{0.500000}      \PYG{l+m+mf}{0.200000}       \PYG{l+m+mf}{2000.00}
\PYG{n}{LLGB}  \PYG{n}{LLG}   \PYG{n}{DISA}  \PYG{n}{T2}    \PYG{n}{HDAS}  \PYG{l+m+mi}{1}     \PYG{n}{GRAD}      \PYG{l+m+mf}{0.500000}       \PYG{l+m+mf}{2000.00}       \PYG{l+m+mf}{200000.}
\PYG{n}{LLS}   \PYG{n}{LLS}   \PYG{n}{ALLO}  \PYG{n}{T23}   \PYG{n}{LSPO}  \PYG{l+m+mi}{1}     \PYG{n}{GRAD}      \PYG{l+m+mf}{0.500000}      \PYG{l+m+mf}{0.200000}       \PYG{l+m+mf}{2000.00}
\PYG{n}{LLSB}  \PYG{n}{LLS}   \PYG{n}{ALLO}  \PYG{n}{T2}    \PYG{n}{HSPO}  \PYG{l+m+mi}{1}     \PYG{n}{GRAD}      \PYG{l+m+mf}{0.500000}       \PYG{l+m+mf}{2000.00}       \PYG{l+m+mf}{200000.}
\PYG{n}{MSFL}  \PYG{n}{MSFL}  \PYG{n}{ALLO}  \PYG{n}{T23}   \PYG{n}{LLIN}  \PYG{l+m+mi}{1}     \PYG{n}{GRAD}      \PYG{l+m+mf}{0.500000}      \PYG{l+m+mf}{0.200000}       \PYG{l+m+mf}{2000.00}
\PYG{l+m+mi}{11}    \PYG{n}{DUMM}  \PYG{n}{DISA}  \PYG{n}{T1}    \PYG{n}{LLIN}  \PYG{n}{NEIT}  \PYG{n}{NB}        \PYG{l+m+mf}{0.500000}       \PYG{l+m+mf}{0.00000}       \PYG{l+m+mf}{1.00000}
\PYG{l+m+mi}{12}    \PYG{n}{DUMM}  \PYG{n}{DISA}  \PYG{n}{T1}    \PYG{n}{LLIN}  \PYG{n}{NEIT}  \PYG{n}{NB}        \PYG{l+m+mf}{0.500000}       \PYG{l+m+mf}{0.00000}       \PYG{l+m+mf}{1.00000}
\PYG{o}{.}\PYG{o}{.}\PYG{o}{.}
\end{sphinxVerbatim}


\begin{savenotes}\sphinxattablestart
\centering
\begin{tabulary}{\linewidth}[t]{|T|T|}
\hline
\sphinxstylethead{\sphinxstyletheadfamily 
Column
\unskip}\relax &\sphinxstylethead{\sphinxstyletheadfamily 
Description
\unskip}\relax \\
\hline
\sphinxcode{MNEM}
&
The (logical) name of the output curve. Note multiple curves might come from one \sphinxcode{OUTP}.
\\
\hline
\sphinxcode{OUTP}
&
The source of the curve.
\\
\hline
\sphinxcode{STAT}
&
Status, is this curve to be plotted.
\\
\hline
\sphinxcode{TRAC}
&
Which track the curve should be plotted on.
\\
\hline
\sphinxcode{CODE}
&
The line coding, dot, dash etc.
\\
\hline
\sphinxcode{DEST}
&
The logical film that this curve is sent to. Can be \sphinxcode{BOTH} and so on.
\\
\hline
\sphinxcode{MODE}
&
What to do when the curve goes off scale (wrap round the track for example).
\\
\hline
\sphinxcode{FILT}
&
Filtering. Unused.
\\
\hline
\sphinxcode{LEDG}
&
Value of the left edge of the plot.
\\
\hline
\sphinxcode{REDG}
&
Value of the right edge of the plot.
\\
\hline
\end{tabulary}
\par
\sphinxattableend\end{savenotes}


\subparagraph{Records Needed for a LIS Plot}
\label{\detokenize{tech/plotting:records-needed-for-a-lis-plot}}
As a minimum TotalDepth needs a \sphinxcode{FILM} and \sphinxcode{PRES} plot, using these TotalDepth can produce a plot identical to that when the data was recorded.

In future releases TotalDepth might also support these Logical Records for plotting:
\begin{itemize}
\item {} 
\sphinxcode{AREA}: Describes shading between curves on a plot.

\item {} 
\sphinxcode{PIP}: Describes integration marks on the plot such as integrated transit time, integrated hole volume etc.

\item {} 
\sphinxcode{VDL}: Describes presentation of waveform amplitude data.

\end{itemize}


\subsubsection{Wireline Files With External Plot Data}
\label{\detokenize{tech/plotting:wireline-files-with-external-plot-data}}
This can apply to any wireline file and it means using an external configuration file to decide the plot layout. For example; LAS files do not have any plot specification (the LAS standard precludes this). The presentation of the plot is made with an external plot specification.

These external plot specifications are numerous and varied. TotalDepth currently supports one such specification: the XML \sphinxtitleref{LgFormat} or \sphinxtitleref{LgSchema} files.


\paragraph{LgFormat XML Files}
\label{\detokenize{tech/plotting:lgformat-xml-files}}
These appear to originate from some of Schlumberger’s “free” software.

Here is an example LgFormat XML file for plotting \sphinxhref{../copy\_to\_html/plots/HDT.xml}{HDT logs} .TotalDepth supplies an number of examples of these.

There are a number of problems %
\begin{footnote}[1]\sphinxAtStartFootnote
For example the \sphinxcode{LgCurve} element content of \sphinxcode{\textless{}LeftLimit\textgreater{}} and \sphinxcode{\textless{}RightLimit\textgreater{}} is numeric, without units. This robs a compliant implementation of the opportunity of converting frame data, say in \sphinxcode{"V/V "} to plot data say \sphinxcode{"PU  "} which might be far more appropriate. Truth in advertising prompts me to say that LIS \sphinxcode{PRES} tables have the same flaw but they have a reasonable excuse that they are \sphinxstyleemphasis{per file} specification rather than a global specification.
%
\end{footnote} with the LgFormat which is why TotalDepth only supports a subset of it while we look for something better.


\subsubsection{Internal Plot Algorithms}
\label{\detokenize{tech/plotting:internal-plot-algorithms}}
This describes some algorithms used by the \sphinxcode{Plot} module. This is for information only as these algorithms are internal and the user has little or no control over them.


\paragraph{Plotting the Legends (Scales)}
\label{\detokenize{tech/plotting:tech-plotting-legends}}\label{\detokenize{tech/plotting:plotting-the-legends-scales}}
These at each end of the plot per curve and may span a number of tracks or be limited to half a track. Here is an example which has seven curves:

TODO: Finish this

\noindent\sphinxincludegraphics{{LegendStacking}.png}

There are two important methods that implement this algorithm: \sphinxcode{Plot.Plot.\_retCurvePlotScaleOrder()} and \sphinxcode{Plot.Plot.\_plotScale()}


\paragraph{Interpolating Backups}
\label{\detokenize{tech/plotting:interpolating-backups}}
Industry standard wireline plots have the notion of curve backups where if the the trace for a particular curve goes off scale on one side of the track it might well reappear from the alternate side.

Here is a simulated example generated by \sphinxcode{TestPlot.py} where a single curve (a sin curve with an amplitude of +/- 40) is plotted in full on track one. The same curve is plotted fully backed up:
\begin{itemize}
\item {} 
In green on track two with a scale of +/- 5

\item {} 
In blue on track two with a scale of +/- 20

\item {} 
In red on track three with a scale of +/- 2.5

\item {} 
In aqua on track three with a scale of +/- 10

\end{itemize}

\noindent\sphinxincludegraphics{{SinWrap}.png}

The algorithm that does this is probably best illustrated by example. The diagram below shows a curve (in the white track on the left) that goes from from point A goes off-scale to the right to point B in ‘real’ space. The line A-B is subdivided by track widths and these \sphinxstyleemphasis{crossing} lines are plotted by interpolation in the track space making the line space A-B’ (so-called ‘FlyBack’ lines are omitted in the diagram). If the {\hyperref[\detokenize{glossary:term-backup-mode}]{\sphinxtermref{\DUrole{xref,std,std-term}{Backup Mode}}}} is \sphinxstyleemphasis{wrap} then both red and green lines in the track are plotted. In this case if the {\hyperref[\detokenize{glossary:term-backup-mode}]{\sphinxtermref{\DUrole{xref,std,std-term}{Backup Mode}}}} is \sphinxstyleemphasis{1} then only the red lines are plotted in the track i.e. only one backup allowed.

\noindent\sphinxincludegraphics{{BackupAlgo}.png}

On the return B-C a similar process happens in reverse.


\subparagraph{Backup Specification}
\label{\detokenize{tech/plotting:backup-specification}}
A backup specification is expected for any plot \sphinxstyleemphasis{per-curve}. Internally a backup specification is a pair of integers. If the virtual track position is less than the value (if left) or greater than the value (if right) then it is off scale. Zero is a special case in that all virtual track positions are on scale.

Various values have been predefined and there are existing mappings from LIS data and LgFormat XML files. The following table describes all of this.


\begin{savenotes}\sphinxattablestart
\centering
\begin{tabulary}{\linewidth}[t]{|T|T|T|T|T|}
\hline
\sphinxstylethead{\sphinxstyletheadfamily 
Spec.
\unskip}\relax &\sphinxstylethead{\sphinxstyletheadfamily 
Symbol
\unskip}\relax &\sphinxstylethead{\sphinxstyletheadfamily 
Description
\unskip}\relax &\sphinxstylethead{\sphinxstyletheadfamily 
LIS Mapping
\unskip}\relax &\sphinxstylethead{\sphinxstyletheadfamily 
LgFormat Mapping
\unskip}\relax \\
\hline
(1, -1)
&
BACKUP\_NONE
&
No backup
&
‘NB  ‘, ‘GRAD’
&\\
\hline
(0,  0)
&
BACKUP\_ALL
&
Every backup i.e. ‘wrap’ Note: Plot.py has a
way of limiting ludicrous backup lines to a
sensible number; say 4
&
‘WRAP’
&
‘LG\_WRAPPED’
‘LG\_X10’
\\
\hline
(-1, 1)
&
BACKUP\_ONCE
&
Single backup to left or right
&
‘SHIF’
&
‘1’
\\
\hline
(-2, 2)
&
BACKUP\_TWICE
&
Two backups to left or right
&&
‘2’
\\
\hline
(-1, 1)
&
BACKUP\_LEFT
&
Single backup to left only
&&
‘LG\_LEFT\_WRAPPED’
\\
\hline
(1, 1)
&
BACKUP\_RIGHT
&
Single backup to right only
&&
‘LG\_RIGHT\_WRAPPED’
\\
\hline
\end{tabulary}
\par
\sphinxattableend\end{savenotes}

Note that it is also common to have one curve with no backup and a second curve driven from the same output with a different scale acting as a backup. This has the advantage that a different coding and colour can be assigned to it. See {\hyperref[\detokenize{tech/plotting:tech-plotting-legends}]{\sphinxcrossref{\DUrole{std,std-ref}{Plotting the Legends (Scales)}}}} above for an example with a Laterlog plot.


\subparagraph{Algorithm}
\label{\detokenize{tech/plotting:algorithm}}
The backup algorithm works as follows:
\begin{itemize}
\item {} 
For every point plotted an integer wrap value is calculated.

\item {} 
If the wrap value is different from the previous wrap value then some interpolation is required and the \sphinxcode{Plot.\_interpolateBackup()} method is called (below).

\item {} 
Otherwise the point is plotted.

\end{itemize}


\subparagraph{\_interpolateBackup}
\label{\detokenize{tech/plotting:interpolatebackup}}
The \sphinxcode{Plot.\_interpolateBackup()} method manages all the different cases where a backup is needed given the backup configuration. It relies on \sphinxcode{Plot.\_retInterpolateWrapPoints()} to generate a list of interpolated lines, including \sphinxstyleemphasis{crossing lines},  between one point and another.


\subparagraph{\_filterCrossLineList}
\label{\detokenize{tech/plotting:filtercrosslinelist}}
A further refinement to the algorithm is limiting the number of \sphinxstyleemphasis{crossing lines}. Consider the SP curve represented by this data taken from a real example:

\begin{sphinxVerbatim}[commandchars=\\\{\}]
\PYG{n}{Depth} \PYG{p}{(}\PYG{n}{ft}\PYG{p}{)} \PYG{n}{SP} \PYG{p}{(}\PYG{n}{mV}\PYG{p}{)}
\PYG{l+m+mf}{10245.0}    \PYG{l+m+mf}{332.724}
\PYG{l+m+mf}{10245.5}    \PYG{l+m+mf}{13716948.000}
\PYG{l+m+mf}{10246.0}    \PYG{l+m+mf}{41150204.000}
\PYG{l+m+mf}{10246.5}    \PYG{l+m+mf}{68583456.000}
\PYG{l+m+mf}{10247.0}    \PYG{l+m+mf}{96016712.000}
\PYG{l+m+mf}{10247.5}    \PYG{l+m+mf}{334.176}
\end{sphinxVerbatim}

Clearly some episode happened between 10245.0 and 10245.5 feet that caused a jump of 13716615.276 mV. This could have been an recording disturbance (unlikely as where would you find 96kV at 10247 feet?) or an editing error. In any case on a scale of -20 to 80 mV means 137,166 (mostly spurious) wrap lines crossing the track. This can turn a 1.6Mb file into a 91Mb file! To stop this a arbitrary limit is made to the number of \sphinxstyleemphasis{crossing lines} (e.g. 4) between each Xaxis interval. This limit filters the \sphinxstyleemphasis{crossing line} list to an evenly distributed list of 4 (or thereabouts).

The method that does this is The \sphinxcode{Plot.\_filterCrossLineList()} method


\subsection{TotalDepth and SaaS}
\label{\detokenize{tech/saas:totaldepth-and-saas}}\label{\detokenize{tech/saas::doc}}
This design shows the potential for running TotalDepth as \sphinxstyleemphasis{Software as a Service} (SaaS) or, if you prefer, \sphinxstyleemphasis{Cloud Computing}.

The design envisages a server running the full TotalDepth software and any number of clients running sub-sets of that software. Client software could be in-browser or on-device.

Network speed (i.e. latency+bandwidth) is crucial to the performance of this system and the network connection between the client and the server could be any of these (in order of slowness):
\begin{itemize}
\item {} 
A client and a server both running on the client’s machine.

\item {} 
A client device and a server running on a high speed LAN.

\item {} 
A client device communicating with a server via a VPN.

\item {} 
A client device communicating with a remote server via the internet.

\end{itemize}

TotalDepth accommodates all of these.


\subsubsection{Processing Server Side Files}
\label{\detokenize{tech/saas:processing-server-side-files}}
There is no particular challenge here as the server has all it needs to process any client instruction. The big disadvantage is that the client has to upload the file to the server before they can do anything. For small jobs on large files that can impose an disproportionate cost.


\subsubsection{Processing Files that Remain on the Client’s Device}
\label{\detokenize{tech/saas:processing-files-that-remain-on-the-client-s-device}}
The real challenge is: the client has a (very large?) file and wants to do something small (or at least incremental) with it, for example “Plot curves XYZ from 5840 to 5600 feet using format ABC” and the server only needs a tiny fraction of that file. Regardless of the network connection there should be some cunning in extracting the data that the client needs in minimal time. TotalDepth’s indexing technique makes this possible.

This design shows how the client can work with the server without the tedious business of the client having to upload the complete file to the server. Instead the client and server cooperate to make sure that the minimum data is transferred for the immediate task in hand.

The design is illustrated below, the client is on the left:

\noindent\sphinxincludegraphics{{Slide12}.png}


\paragraph{Starting the session}
\label{\detokenize{tech/saas:starting-the-session}}
The session is initialised by the user requesting the client to operate on a particular file on the clients device. The client passes basic file information to the server and receives a unique session ID.


\subparagraph{The Session ID}
\label{\detokenize{tech/saas:the-session-id}}
The session ID could be a portable file ID, such as a checksum. Thus two users using the same file could benefit from each others read transactions. This would require copy-on-write and session ID modification.

Another optimisation when using a checksum is that if the server already has an exact copy of the file (using the checksum as a proxy for that) then the server could provide data from its own copy of the file without the user having to transfer any data to the server at all.


\subparagraph{Session Bootstrapping}
\label{\detokenize{tech/saas:session-bootstrapping}}
Given a valid session ID the client code reads the minimum structural information for the server to comprehend (and mirror) the file structure. This will vary according to the file format (and will vary according to file size).


\subparagraph{LIS}
\label{\detokenize{tech/saas:lis}}
The client code must be capable of processing TIF markers, Physical Record headers and trailers and Logical Record headers. The client passes this data as an ordered list of pairs \sphinxcode{{[}(tell, bytes), ...{]}} where tell is the \sphinxcode{size\_t} position in the of the Physical Record (or TIF marker) that starts the Logical Record and \sphinxcode{bytes} is the raw two bytes of the Logical Record Header.


\subparagraph{LAS}
\label{\detokenize{tech/saas:las}}
As LAS files are small and do not lend themselves easily to efficient indexing then, most likely, it is worth passing the
entire client file to the server.


\subparagraph{RP66}
\label{\detokenize{tech/saas:rp66}}
This will be similar to LIS bootstrapping with the client code processing Visible Record Headers and Logical Record segments.


\subparagraph{Index Completion}
\label{\detokenize{tech/saas:index-completion}}
The bootstrap information passed to the server allows the server to construct a virtual, partial, logical image of the file. The server may request further information in the form of an ordered list of pairs \sphinxcode{{[}(file position, number of bytes), ...{]}}. The client does a series of seek/read operations and passes the data back to the server as a list of pairs \sphinxcode{{[}(file position, bytes), ...{]}}. This allows the server to complete the file index.

The total size of the data transferred to the server at this stage is typically 0.6\% of the file size. See {\hyperref[\detokenize{tech/LIS_indexing:totaldepth-tech-indexing}]{\sphinxcrossref{\DUrole{std,std-ref}{Indexing LIS Files}}}} for a description of how LIS files are indexed and {\hyperref[\detokenize{tech/LIS_indexing:totaldepth-tech-indexing-indexsize}]{\sphinxcrossref{\DUrole{std,std-ref}{Multi-Processing}}}} for a study of index sizes.


\paragraph{Rest of Session}
\label{\detokenize{tech/saas:rest-of-session}}
Once bootstrapped any user request is passed by the client to the server, for example: “Plot curves ABC from 5840 to 5600 feet”.

The server then consults its virtual file image to see if it has the data to satisfy the request. If so then the result of the request is sent to the client.

If the server does not have sufficient data in its virtual file image then the server will send to the client an (ordered) list of pairs \sphinxcode{{[}(file position, number of bytes), ...{]}}. The client does a series of seek/read operations on each pair and and passes the data back to the server as a list of pairs \sphinxcode{{[}(file position, bytes), ...{]}}. The server adds these to its virtual file image. The server then can complete the request.

Over time the virtual file image would grow but only as fast as the user’s (new) demands.


\paragraph{Data Persistence}
\label{\detokenize{tech/saas:data-persistence}}
It is possible for the server to persist the virtual file image over any number of sessions, this would improve the performance even further.


\subsubsection{SaaS Pros and Cons}
\label{\detokenize{tech/saas:saas-pros-and-cons}}

\paragraph{Advantages}
\label{\detokenize{tech/saas:advantages}}\begin{itemize}
\item {} 
Low barrier to use: browser based, no installation.

\item {} 
Cross Platform: desktop, tablet, mobile etc.

\item {} 
Minimum client code.

\item {} 
Continuous software version update from the server.

\item {} 
Integrated with other Internet services for example mapping data.

\item {} 
Tailored appearance per client.

\item {} 
Cloud availability behind the server.

\item {} 
All usage data is available on server logs.

\end{itemize}


\paragraph{Disadvantages}
\label{\detokenize{tech/saas:disadvantages}}\begin{itemize}
\item {} 
Requires network connection (could have a server version running locally).

\item {} 
User agent variability.

\item {} 
Speed/performance limited by network.

\item {} 
Server infrastructure and investment.

\item {} 
Continuous maintenance and support.

\item {} 
Security.

\end{itemize}


\section{TotalDepth and Software Test}
\label{\detokenize{testing/index_test:totaldepth-and-software-test}}\label{\detokenize{testing/index_test::doc}}
Contents:


\subsection{Unit Tests}
\label{\detokenize{testing/unit_tests::doc}}\label{\detokenize{testing/unit_tests:unit-tests}}
This describes how unit testing is done in TotalDepth.


\subsubsection{Module Unit Testing}
\label{\detokenize{testing/unit_tests:module-unit-testing}}
Most modules have their unit tests in the \sphinxcode{test/} directory in the package the module is in.
The test code is usually in \sphinxcode{Test\sphinxstyleemphasis{Module}.py}.


\paragraph{Example}
\label{\detokenize{testing/unit_tests:example}}
For the module \sphinxcode{TotalDepth.LIS.core.LogPass} the unit tests are in \sphinxcode{TotalDepth/LIS/core/test/TestLogPass.Py}
and they can be run thus:

\begin{sphinxVerbatim}[commandchars=\\\{\}]
\PYGZdl{} cd \PYGZlt{}TOTALDEPTH\PYGZgt{}/src/TotalDepth/LIS/core
\PYGZdl{} python3 test/TestLogPass.py
\end{sphinxVerbatim}

The output is:

\begin{sphinxVerbatim}[commandchars=\\\{\}]
TestClass.py script version \PYGZdq{}0.8.0\PYGZdq{}, dated 10 Jan 2011
Author: Paul Ross
Copyright (c) Paul Ross

testSetUpTearDown (\PYGZus{}\PYGZus{}main\PYGZus{}\PYGZus{}.TestLogPass\PYGZus{}LowLevel)
TestLogPass\PYGZus{}LowLevel: Tests setUp() and tearDown(). ... ok
test\PYGZus{}00 (\PYGZus{}\PYGZus{}main\PYGZus{}\PYGZus{}.TestLogPass\PYGZus{}LowLevel)
TestLogPass\PYGZus{}LowLevel.test\PYGZus{}00(): Construction. ... ok
test\PYGZus{}01 (\PYGZus{}\PYGZus{}main\PYGZus{}\PYGZus{}.TestLogPass\PYGZus{}LowLevel)

... 8\PYGZlt{}\PYGZhy{}\PYGZhy{}\PYGZhy{}\PYGZhy{}\PYGZhy{}\PYGZhy{}\PYGZhy{}\PYGZhy{} Many lines snipped \PYGZhy{}\PYGZhy{}\PYGZhy{}\PYGZhy{}\PYGZhy{}\PYGZhy{}\PYGZhy{}\PYGZhy{}\PYGZgt{}8

TestLogPass\PYGZus{}UpIndirect.test\PYGZus{}02(): 3 LR, 5 fr, 4 ch. setFrameSet() theFrSl=slice(2,4,2). ... ok
test\PYGZus{}03 (\PYGZus{}\PYGZus{}main\PYGZus{}\PYGZus{}.TestLogPass\PYGZus{}UpIndirect)
TestLogPass\PYGZus{}UpIndirect.test\PYGZus{}03(): 3 LR, 5 fr, 4 ch. setFrameSet() theFrSl=slice(1,5,2). ... ok
test\PYGZus{}10 (\PYGZus{}\PYGZus{}main\PYGZus{}\PYGZus{}.TestLogPass\PYGZus{}UpIndirect)
TestLogPass\PYGZus{}UpIndirect.test\PYGZus{}10(): genFrameSetHeadings() ... ok

\PYGZhy{}\PYGZhy{}\PYGZhy{}\PYGZhy{}\PYGZhy{}\PYGZhy{}\PYGZhy{}\PYGZhy{}\PYGZhy{}\PYGZhy{}\PYGZhy{}\PYGZhy{}\PYGZhy{}\PYGZhy{}\PYGZhy{}\PYGZhy{}\PYGZhy{}\PYGZhy{}\PYGZhy{}\PYGZhy{}\PYGZhy{}\PYGZhy{}\PYGZhy{}\PYGZhy{}\PYGZhy{}\PYGZhy{}\PYGZhy{}\PYGZhy{}\PYGZhy{}\PYGZhy{}\PYGZhy{}\PYGZhy{}\PYGZhy{}\PYGZhy{}\PYGZhy{}\PYGZhy{}\PYGZhy{}\PYGZhy{}\PYGZhy{}\PYGZhy{}\PYGZhy{}\PYGZhy{}\PYGZhy{}\PYGZhy{}\PYGZhy{}\PYGZhy{}\PYGZhy{}\PYGZhy{}\PYGZhy{}\PYGZhy{}\PYGZhy{}\PYGZhy{}\PYGZhy{}\PYGZhy{}\PYGZhy{}\PYGZhy{}\PYGZhy{}\PYGZhy{}\PYGZhy{}\PYGZhy{}\PYGZhy{}\PYGZhy{}\PYGZhy{}\PYGZhy{}\PYGZhy{}\PYGZhy{}\PYGZhy{}\PYGZhy{}\PYGZhy{}\PYGZhy{}
Ran 54 tests in 0.271s

OK
CPU time =    0.275 (S)
Bye, bye!
\end{sphinxVerbatim}


\subsubsection{Package Unit Testing}
\label{\detokenize{testing/unit_tests:package-unit-testing}}
Most packages have their unit tests in the \sphinxcode{test/UnitTests.py} in the package the module is in.


\paragraph{Example}
\label{\detokenize{testing/unit_tests:id1}}
For the module \sphinxcode{TotalDepth.LAS.core} the unit tests are in \sphinxcode{TotalDepth/LAS/core/test/UnitTests.Py}
and they can be run thus:

\begin{sphinxVerbatim}[commandchars=\\\{\}]
\PYGZdl{} python3 test/UnitTests.py
\end{sphinxVerbatim}

And the result is:

\begin{sphinxVerbatim}[commandchars=\\\{\}]
UnitTests.py script version \PYGZdq{}0.8.0\PYGZdq{}, dated 2009\PYGZhy{}09\PYGZhy{}15
Author: Paul Ross
Copyright (c) Paul Ross

Command line:
test/UnitTests.py

(\PYGZsq{}File:\PYGZsq{}, \PYGZsq{}test/UnitTests.py\PYGZsq{})
Testing TestEngVal
\PYGZhy{}\PYGZhy{}\PYGZhy{}\PYGZhy{}\PYGZhy{}\PYGZhy{}\PYGZhy{}\PYGZhy{}\PYGZhy{}\PYGZhy{}\PYGZhy{}\PYGZhy{}\PYGZhy{}\PYGZhy{}\PYGZhy{}\PYGZhy{}\PYGZhy{}\PYGZhy{}\PYGZhy{}\PYGZhy{}\PYGZhy{}\PYGZhy{}\PYGZhy{}\PYGZhy{}\PYGZhy{}\PYGZhy{}\PYGZhy{}\PYGZhy{}\PYGZhy{}\PYGZhy{}\PYGZhy{}\PYGZhy{}\PYGZhy{}\PYGZhy{}\PYGZhy{}\PYGZhy{}\PYGZhy{}\PYGZhy{}\PYGZhy{}\PYGZhy{}\PYGZhy{}\PYGZhy{}\PYGZhy{}\PYGZhy{}\PYGZhy{}\PYGZhy{}\PYGZhy{}\PYGZhy{}\PYGZhy{}\PYGZhy{}\PYGZhy{}\PYGZhy{}\PYGZhy{}\PYGZhy{}\PYGZhy{}\PYGZhy{}\PYGZhy{}\PYGZhy{}\PYGZhy{}\PYGZhy{}\PYGZhy{}\PYGZhy{}\PYGZhy{}\PYGZhy{}\PYGZhy{}\PYGZhy{}\PYGZhy{}\PYGZhy{}\PYGZhy{}\PYGZhy{}
Ran 55 tests in 0.002s

OK

8\PYGZlt{}\PYGZhy{}\PYGZhy{}\PYGZhy{}\PYGZhy{}\PYGZhy{}\PYGZhy{}\PYGZhy{}\PYGZhy{} Many lines snipped \PYGZhy{}\PYGZhy{}\PYGZhy{}\PYGZhy{}\PYGZhy{}\PYGZhy{}\PYGZhy{}\PYGZhy{}\PYGZgt{}8

OK
Testing TestUnits
Time:    0.952 rate    102.547 k/S Time:    0.389 rate    250.863 k/S Time:    0.175 rate    556.827 k/S \PYGZhy{}\PYGZhy{}\PYGZhy{}\PYGZhy{}\PYGZhy{}\PYGZhy{}\PYGZhy{}\PYGZhy{}\PYGZhy{}\PYGZhy{}\PYGZhy{}\PYGZhy{}\PYGZhy{}\PYGZhy{}\PYGZhy{}\PYGZhy{}\PYGZhy{}\PYGZhy{}\PYGZhy{}\PYGZhy{}\PYGZhy{}\PYGZhy{}\PYGZhy{}\PYGZhy{}\PYGZhy{}\PYGZhy{}\PYGZhy{}\PYGZhy{}\PYGZhy{}\PYGZhy{}\PYGZhy{}\PYGZhy{}\PYGZhy{}\PYGZhy{}\PYGZhy{}\PYGZhy{}\PYGZhy{}\PYGZhy{}\PYGZhy{}\PYGZhy{}\PYGZhy{}\PYGZhy{}\PYGZhy{}\PYGZhy{}\PYGZhy{}\PYGZhy{}\PYGZhy{}\PYGZhy{}\PYGZhy{}\PYGZhy{}\PYGZhy{}\PYGZhy{}\PYGZhy{}\PYGZhy{}\PYGZhy{}\PYGZhy{}\PYGZhy{}\PYGZhy{}\PYGZhy{}\PYGZhy{}\PYGZhy{}\PYGZhy{}\PYGZhy{}\PYGZhy{}\PYGZhy{}\PYGZhy{}\PYGZhy{}\PYGZhy{}\PYGZhy{}\PYGZhy{}
Ran 8 tests in 1.517s

OK
Results
   Tests: 803
  Errors: 0
Failures: 0

CPU time =  215.986 (S)
Bye, bye!
\end{sphinxVerbatim}


\subsubsection{Unit Testing and \sphinxstyleliteralintitle{coverage}}
\label{\detokenize{testing/unit_tests:unit-testing-and-coverage}}
Ned Batchelor’s \sphinxstyleemphasis{excellent} coverage tool can be used with these unit tests:

\begin{sphinxVerbatim}[commandchars=\\\{\}]
\PYGZdl{} coverage run test/TestMnem.py
\end{sphinxVerbatim}

And the result is:

\begin{sphinxVerbatim}[commandchars=\\\{\}]
TestMnem.py script version \PYGZdq{}0.8.0\PYGZdq{}, dated May 26, 2011
Author: Paul Ross
Copyright (c) 2011 Paul Ross.

test\PYGZus{}00 (\PYGZus{}\PYGZus{}main\PYGZus{}\PYGZus{}.TestMnem)
TestMnem.test\PYGZus{}00(): Tests setUp() and tearDown(). ... ok
test\PYGZus{}01 (\PYGZus{}\PYGZus{}main\PYGZus{}\PYGZus{}.TestMnem)
TestMnem.test\PYGZus{}01(): Basic constructor. ... ok
test\PYGZus{}02 (\PYGZus{}\PYGZus{}main\PYGZus{}\PYGZus{}.TestMnem)
TestMnem.test\PYGZus{}02(): Truncation of \PYGZgt{}4 chars. ... ok
test\PYGZus{}03 (\PYGZus{}\PYGZus{}main\PYGZus{}\PYGZus{}.TestMnem)
TestMnem.test\PYGZus{}03(): Padding of \PYGZlt{}4 chars. ... ok
...
test\PYGZus{}70 (\PYGZus{}\PYGZus{}main\PYGZus{}\PYGZus{}.TestMnem)
TestMnem.test\PYGZus{}70(): Construction with a string. ... ok

\PYGZhy{}\PYGZhy{}\PYGZhy{}\PYGZhy{}\PYGZhy{}\PYGZhy{}\PYGZhy{}\PYGZhy{}\PYGZhy{}\PYGZhy{}\PYGZhy{}\PYGZhy{}\PYGZhy{}\PYGZhy{}\PYGZhy{}\PYGZhy{}\PYGZhy{}\PYGZhy{}\PYGZhy{}\PYGZhy{}\PYGZhy{}\PYGZhy{}\PYGZhy{}\PYGZhy{}\PYGZhy{}\PYGZhy{}\PYGZhy{}\PYGZhy{}\PYGZhy{}\PYGZhy{}\PYGZhy{}\PYGZhy{}\PYGZhy{}\PYGZhy{}\PYGZhy{}\PYGZhy{}\PYGZhy{}\PYGZhy{}\PYGZhy{}\PYGZhy{}\PYGZhy{}\PYGZhy{}\PYGZhy{}\PYGZhy{}\PYGZhy{}\PYGZhy{}\PYGZhy{}\PYGZhy{}\PYGZhy{}\PYGZhy{}\PYGZhy{}\PYGZhy{}\PYGZhy{}\PYGZhy{}\PYGZhy{}\PYGZhy{}\PYGZhy{}\PYGZhy{}\PYGZhy{}\PYGZhy{}\PYGZhy{}\PYGZhy{}\PYGZhy{}\PYGZhy{}\PYGZhy{}\PYGZhy{}\PYGZhy{}\PYGZhy{}\PYGZhy{}\PYGZhy{}
Ran 21 tests in 0.002s

OK
CPU time =    0.003 (S)
Bye, bye!
\end{sphinxVerbatim}

Reporting coverage:

\begin{sphinxVerbatim}[commandchars=\\\{\}]
\PYGZdl{} coverage report \PYGZhy{}m
\end{sphinxVerbatim}

And the result is:

\begin{sphinxVerbatim}[commandchars=\\\{\}]
\PYG{n}{Name}                                   \PYG{n}{Stmts}   \PYG{n}{Miss}  \PYG{n}{Cover}   \PYG{n}{Missing}
\PYG{o}{\PYGZhy{}}\PYG{o}{\PYGZhy{}}\PYG{o}{\PYGZhy{}}\PYG{o}{\PYGZhy{}}\PYG{o}{\PYGZhy{}}\PYG{o}{\PYGZhy{}}\PYG{o}{\PYGZhy{}}\PYG{o}{\PYGZhy{}}\PYG{o}{\PYGZhy{}}\PYG{o}{\PYGZhy{}}\PYG{o}{\PYGZhy{}}\PYG{o}{\PYGZhy{}}\PYG{o}{\PYGZhy{}}\PYG{o}{\PYGZhy{}}\PYG{o}{\PYGZhy{}}\PYG{o}{\PYGZhy{}}\PYG{o}{\PYGZhy{}}\PYG{o}{\PYGZhy{}}\PYG{o}{\PYGZhy{}}\PYG{o}{\PYGZhy{}}\PYG{o}{\PYGZhy{}}\PYG{o}{\PYGZhy{}}\PYG{o}{\PYGZhy{}}\PYG{o}{\PYGZhy{}}\PYG{o}{\PYGZhy{}}\PYG{o}{\PYGZhy{}}\PYG{o}{\PYGZhy{}}\PYG{o}{\PYGZhy{}}\PYG{o}{\PYGZhy{}}\PYG{o}{\PYGZhy{}}\PYG{o}{\PYGZhy{}}\PYG{o}{\PYGZhy{}}\PYG{o}{\PYGZhy{}}\PYG{o}{\PYGZhy{}}\PYG{o}{\PYGZhy{}}\PYG{o}{\PYGZhy{}}\PYG{o}{\PYGZhy{}}\PYG{o}{\PYGZhy{}}\PYG{o}{\PYGZhy{}}\PYG{o}{\PYGZhy{}}\PYG{o}{\PYGZhy{}}\PYG{o}{\PYGZhy{}}\PYG{o}{\PYGZhy{}}\PYG{o}{\PYGZhy{}}\PYG{o}{\PYGZhy{}}\PYG{o}{\PYGZhy{}}\PYG{o}{\PYGZhy{}}\PYG{o}{\PYGZhy{}}\PYG{o}{\PYGZhy{}}\PYG{o}{\PYGZhy{}}\PYG{o}{\PYGZhy{}}\PYG{o}{\PYGZhy{}}\PYG{o}{\PYGZhy{}}\PYG{o}{\PYGZhy{}}\PYG{o}{\PYGZhy{}}\PYG{o}{\PYGZhy{}}\PYG{o}{\PYGZhy{}}\PYG{o}{\PYGZhy{}}\PYG{o}{\PYGZhy{}}\PYG{o}{\PYGZhy{}}\PYG{o}{\PYGZhy{}}\PYG{o}{\PYGZhy{}}\PYG{o}{\PYGZhy{}}\PYG{o}{\PYGZhy{}}\PYG{o}{\PYGZhy{}}\PYG{o}{\PYGZhy{}}\PYG{o}{\PYGZhy{}}\PYG{o}{\PYGZhy{}}\PYG{o}{\PYGZhy{}}\PYG{o}{\PYGZhy{}}\PYG{o}{\PYGZhy{}}\PYG{o}{\PYGZhy{}}\PYG{o}{\PYGZhy{}}\PYG{o}{\PYGZhy{}}\PYG{o}{\PYGZhy{}}\PYG{o}{\PYGZhy{}}\PYG{o}{\PYGZhy{}}\PYG{o}{\PYGZhy{}}\PYG{o}{\PYGZhy{}}\PYG{o}{\PYGZhy{}}\PYG{o}{\PYGZhy{}}\PYG{o}{\PYGZhy{}}\PYG{o}{\PYGZhy{}}\PYG{o}{\PYGZhy{}}\PYG{o}{\PYGZhy{}}\PYG{o}{\PYGZhy{}}\PYG{o}{\PYGZhy{}}\PYG{o}{\PYGZhy{}}\PYG{o}{\PYGZhy{}}\PYG{o}{\PYGZhy{}}\PYG{o}{\PYGZhy{}}
\PYG{n}{TotalDepth}\PYG{o}{/}\PYG{n}{src}\PYG{o}{/}\PYG{n}{TotalDepth}\PYG{o}{/}\PYG{n}{LIS}\PYG{o}{/}\PYG{n+nf+fm}{\PYGZus{}\PYGZus{}init\PYGZus{}\PYGZus{}}     \PYG{l+m+mi}{7}      \PYG{l+m+mi}{0}   \PYG{l+m+mi}{100}\PYG{o}{\PYGZpc{}}
\PYG{n}{TotalDepth}\PYG{o}{/}\PYG{n}{src}\PYG{o}{/}\PYG{n}{TotalDepth}\PYG{o}{/}\PYG{n+nf+fm}{\PYGZus{}\PYGZus{}init\PYGZus{}\PYGZus{}}         \PYG{l+m+mi}{6}      \PYG{l+m+mi}{0}   \PYG{l+m+mi}{100}\PYG{o}{\PYGZpc{}}
\PYG{n}{Mnem}                                      \PYG{l+m+mi}{52}      \PYG{l+m+mi}{0}   \PYG{l+m+mi}{100}\PYG{o}{\PYGZpc{}}
\PYG{n+nf+fm}{\PYGZus{}\PYGZus{}init\PYGZus{}\PYGZus{}}                                   \PYG{l+m+mi}{0}      \PYG{l+m+mi}{0}   \PYG{l+m+mi}{100}\PYG{o}{\PYGZpc{}}
\PYG{n}{test}\PYG{o}{/}\PYG{n}{TestMnem}                            \PYG{l+m+mi}{161}     \PYG{l+m+mi}{13}    \PYG{l+m+mi}{92}\PYG{o}{\PYGZpc{}}   \PYG{l+m+mi}{259}\PYG{p}{,} \PYG{l+m+mi}{286}\PYG{o}{\PYGZhy{}}\PYG{l+m+mi}{289}\PYG{p}{,} \PYG{l+m+mi}{292}\PYG{o}{\PYGZhy{}}\PYG{l+m+mi}{296}\PYG{p}{,} \PYG{l+m+mi}{298}\PYG{o}{\PYGZhy{}}\PYG{l+m+mi}{300}
\PYG{o}{\PYGZhy{}}\PYG{o}{\PYGZhy{}}\PYG{o}{\PYGZhy{}}\PYG{o}{\PYGZhy{}}\PYG{o}{\PYGZhy{}}\PYG{o}{\PYGZhy{}}\PYG{o}{\PYGZhy{}}\PYG{o}{\PYGZhy{}}\PYG{o}{\PYGZhy{}}\PYG{o}{\PYGZhy{}}\PYG{o}{\PYGZhy{}}\PYG{o}{\PYGZhy{}}\PYG{o}{\PYGZhy{}}\PYG{o}{\PYGZhy{}}\PYG{o}{\PYGZhy{}}\PYG{o}{\PYGZhy{}}\PYG{o}{\PYGZhy{}}\PYG{o}{\PYGZhy{}}\PYG{o}{\PYGZhy{}}\PYG{o}{\PYGZhy{}}\PYG{o}{\PYGZhy{}}\PYG{o}{\PYGZhy{}}\PYG{o}{\PYGZhy{}}\PYG{o}{\PYGZhy{}}\PYG{o}{\PYGZhy{}}\PYG{o}{\PYGZhy{}}\PYG{o}{\PYGZhy{}}\PYG{o}{\PYGZhy{}}\PYG{o}{\PYGZhy{}}\PYG{o}{\PYGZhy{}}\PYG{o}{\PYGZhy{}}\PYG{o}{\PYGZhy{}}\PYG{o}{\PYGZhy{}}\PYG{o}{\PYGZhy{}}\PYG{o}{\PYGZhy{}}\PYG{o}{\PYGZhy{}}\PYG{o}{\PYGZhy{}}\PYG{o}{\PYGZhy{}}\PYG{o}{\PYGZhy{}}\PYG{o}{\PYGZhy{}}\PYG{o}{\PYGZhy{}}\PYG{o}{\PYGZhy{}}\PYG{o}{\PYGZhy{}}\PYG{o}{\PYGZhy{}}\PYG{o}{\PYGZhy{}}\PYG{o}{\PYGZhy{}}\PYG{o}{\PYGZhy{}}\PYG{o}{\PYGZhy{}}\PYG{o}{\PYGZhy{}}\PYG{o}{\PYGZhy{}}\PYG{o}{\PYGZhy{}}\PYG{o}{\PYGZhy{}}\PYG{o}{\PYGZhy{}}\PYG{o}{\PYGZhy{}}\PYG{o}{\PYGZhy{}}\PYG{o}{\PYGZhy{}}\PYG{o}{\PYGZhy{}}\PYG{o}{\PYGZhy{}}\PYG{o}{\PYGZhy{}}\PYG{o}{\PYGZhy{}}\PYG{o}{\PYGZhy{}}\PYG{o}{\PYGZhy{}}\PYG{o}{\PYGZhy{}}\PYG{o}{\PYGZhy{}}\PYG{o}{\PYGZhy{}}\PYG{o}{\PYGZhy{}}\PYG{o}{\PYGZhy{}}\PYG{o}{\PYGZhy{}}\PYG{o}{\PYGZhy{}}\PYG{o}{\PYGZhy{}}\PYG{o}{\PYGZhy{}}\PYG{o}{\PYGZhy{}}\PYG{o}{\PYGZhy{}}\PYG{o}{\PYGZhy{}}\PYG{o}{\PYGZhy{}}\PYG{o}{\PYGZhy{}}\PYG{o}{\PYGZhy{}}\PYG{o}{\PYGZhy{}}\PYG{o}{\PYGZhy{}}\PYG{o}{\PYGZhy{}}\PYG{o}{\PYGZhy{}}\PYG{o}{\PYGZhy{}}\PYG{o}{\PYGZhy{}}\PYG{o}{\PYGZhy{}}\PYG{o}{\PYGZhy{}}\PYG{o}{\PYGZhy{}}\PYG{o}{\PYGZhy{}}\PYG{o}{\PYGZhy{}}\PYG{o}{\PYGZhy{}}\PYG{o}{\PYGZhy{}}\PYG{o}{\PYGZhy{}}
\PYG{n}{TOTAL}                                    \PYG{l+m+mi}{226}     \PYG{l+m+mi}{13}    \PYG{l+m+mi}{94}\PYG{o}{\PYGZpc{}}
\end{sphinxVerbatim}


\subsection{Testing the Plot Package}
\label{\detokenize{testing/test_plot:testing-the-plot-package}}\label{\detokenize{testing/test_plot::doc}}
This is a comprehensive testing of your installation to see if LIS/LAS files can be written, read and plotted. The code writes specific LIS/LAS files in memory, reads them back with the appropriate parser then plots them as SVG files.


\subsubsection{Running the test}
\label{\detokenize{testing/test_plot:running-the-test}}
TestPlot is a unit test which also has performance tests within it. It is invoked thus:

\begin{sphinxVerbatim}[commandchars=\\\{\}]
\PYGZdl{} python tests/unit/test\PYGZus{}util/test\PYGZus{}plot/TestPlot.py
\end{sphinxVerbatim}

The response should be something like:

\begin{sphinxVerbatim}[commandchars=\\\{\}]
TestClass.py script version \PYGZdq{}0.1.0\PYGZdq{}, dated 2010\PYGZhy{}08\PYGZhy{}02
Author: Paul Ross
Copyright (c) Paul Ross

test\PYGZus{}00 (\PYGZus{}\PYGZus{}main\PYGZus{}\PYGZus{}.TestPlotRollStatic)
TestPlotRollStatic.test\PYGZus{}00(): Tests setUp() and tearDown(). ... ok
test\PYGZus{}01 (\PYGZus{}\PYGZus{}main\PYGZus{}\PYGZus{}.TestPlotRollStatic)
TestPlotRollStatic.test\PYGZus{}01(): viewBox. ... ok

... 8\PYGZlt{}\PYGZhy{}\PYGZhy{}\PYGZhy{}\PYGZhy{}\PYGZhy{}\PYGZhy{}\PYGZhy{}\PYGZhy{} snip, snip, snip \PYGZhy{}\PYGZhy{}\PYGZhy{}\PYGZhy{}\PYGZhy{}\PYGZhy{}\PYGZhy{}\PYGZhy{}\PYGZgt{}8

test\PYGZus{}01 (\PYGZus{}\PYGZus{}main\PYGZus{}\PYGZus{}.TestPlotReadLIS\PYGZus{}SingleSinCurve) ...  ExecTimerList [6]:
 Loading FrameSet          Size:    0.002 (MB) Time:    0.010 (s) Cost:   3838.393 (ms/MB)
 Initialising LIS plot     Size:    0.000 (MB) Time:    0.001 (s) Cost:        N/A (ms/MB)
 Plotting Tracks           Size:    0.000 (MB) Time:    0.005 (s) Cost:        N/A (ms/MB)
 Plotting XGrid            Size:    0.000 (MB) Time:    0.023 (s) Cost:        N/A (ms/MB)
 Plotting scales (legends) Size:    0.000 (MB) Time:    0.013 (s) Cost:        N/A (ms/MB)
 Plotting curves           Size:    0.002 (MB) Time:    0.141 (s) Cost:   9338.957 (ms/MB)  ok
test\PYGZus{}00 (\PYGZus{}\PYGZus{}main\PYGZus{}\PYGZus{}.TestPlotReadLIS\PYGZus{}SingleSquareCurveLowFreq)
TestPlotReadLIS\PYGZus{}SingleSquareCurve.test\PYGZus{}00(): Tests setUp() and tearDown(). ... ok

... 8\PYGZlt{}\PYGZhy{}\PYGZhy{}\PYGZhy{}\PYGZhy{}\PYGZhy{}\PYGZhy{}\PYGZhy{}\PYGZhy{} snip, snip \PYGZhy{}\PYGZhy{}\PYGZhy{}\PYGZhy{}\PYGZhy{}\PYGZhy{}\PYGZhy{}\PYGZhy{}\PYGZgt{}8

test\PYGZus{}12 (\PYGZus{}\PYGZus{}main\PYGZus{}\PYGZus{}.TestPlotReadLAS\PYGZus{}XML\PYGZus{}LgFormat)
TestPlotReadLAS\PYGZus{}XML\PYGZus{}LgFormat.test\PYGZus{}12(): Plot from XML LgFormat files \PYGZhy{} density, porosity and multiple gamma ray curves. ...  ExecTimerList [6]:
 Initialising LAS plot: \PYGZdq{}Porosity\PYGZus{}GR\PYGZus{}3Track\PYGZdq{} Size:    0.000 (MB) Time:    0.001 (s) Cost:        N/A (ms/MB)
 Plotting API Header                         Size:    0.000 (MB) Time:    0.016 (s) Cost:        N/A (ms/MB)
 Plotting Tracks                             Size:    0.000 (MB) Time:    0.005 (s) Cost:        N/A (ms/MB)
 Plotting XGrid                              Size:    0.000 (MB) Time:    0.055 (s) Cost:        N/A (ms/MB)
 Plotting scales (legends)                   Size:    0.000 (MB) Time:    0.013 (s) Cost:        N/A (ms/MB)
 Plotting curves                             Size:    0.020 (MB) Time:    0.096 (s) Cost:   5009.433 (ms/MB)  ok

\PYGZhy{}\PYGZhy{}\PYGZhy{}\PYGZhy{}\PYGZhy{}\PYGZhy{}\PYGZhy{}\PYGZhy{}\PYGZhy{}\PYGZhy{}\PYGZhy{}\PYGZhy{}\PYGZhy{}\PYGZhy{}\PYGZhy{}\PYGZhy{}\PYGZhy{}\PYGZhy{}\PYGZhy{}\PYGZhy{}\PYGZhy{}\PYGZhy{}\PYGZhy{}\PYGZhy{}\PYGZhy{}\PYGZhy{}\PYGZhy{}\PYGZhy{}\PYGZhy{}\PYGZhy{}\PYGZhy{}\PYGZhy{}\PYGZhy{}\PYGZhy{}\PYGZhy{}\PYGZhy{}\PYGZhy{}\PYGZhy{}\PYGZhy{}\PYGZhy{}\PYGZhy{}\PYGZhy{}\PYGZhy{}\PYGZhy{}\PYGZhy{}\PYGZhy{}\PYGZhy{}\PYGZhy{}\PYGZhy{}\PYGZhy{}\PYGZhy{}\PYGZhy{}\PYGZhy{}\PYGZhy{}\PYGZhy{}\PYGZhy{}\PYGZhy{}\PYGZhy{}\PYGZhy{}\PYGZhy{}\PYGZhy{}\PYGZhy{}\PYGZhy{}\PYGZhy{}\PYGZhy{}\PYGZhy{}\PYGZhy{}\PYGZhy{}\PYGZhy{}\PYGZhy{}
Ran 107 tests in 34.318s

OK
CPU time =   34.179 (S)
Bye, bye!
\end{sphinxVerbatim}

The important thing is that there should be no reported failures.


\subsubsection{Results}
\label{\detokenize{testing/test_plot:results}}
You should find in the directory \sphinxcode{\sphinxstyleemphasis{\textless{}TOTALDEPTH\textgreater{}}/tests/unit/test\_util/test\_plot/test\_svg} a set of test plots.


\paragraph{Index}
\label{\detokenize{testing/test_plot:index}}
This directory has an \sphinxcode{\sphinxstyleemphasis{\textless{}TOTALDEPTH\textgreater{}}/tests/unit/test\_util/test\_plot/test\_svg/index.html} that looks like this:

\noindent\sphinxincludegraphics{{TestPlotIndex}.png}


\paragraph{Example Plot}
\label{\detokenize{testing/test_plot:example-plot}}
Navigate to a typical test LIS plot from that index such as:

\noindent\sphinxincludegraphics{{TestPlotExample_00}.png}

Cool. If you see that then your TotalDepth installation is pretty good!


\section{API Reference}
\label{\detokenize{ref/index_ref::doc}}\label{\detokenize{ref/index_ref:api-reference}}
Contents:


\subsection{TotalDepth.LIS API Reference}
\label{\detokenize{ref/LIS/index_LIS::doc}}\label{\detokenize{ref/LIS/index_LIS:totaldepth-lis-api-reference}}
Contents:


\subsubsection{LIS Handling of Representation Codes}
\label{\detokenize{ref/LIS/core/RepCode:lis-handling-of-representation-codes}}\label{\detokenize{ref/LIS/core/RepCode::doc}}
This describes how TotalDepth.LIS handles representation codes, it covers several modules.


\paragraph{Design}
\label{\detokenize{ref/LIS/core/RepCode:design}}
There is a top level module \sphinxcode{RepCode} that imports all sub-modules and provides some fundamental definitions. Sub-modules \sphinxcode{pRepCode} and \sphinxcode{cRepCOde} provide alternative implementations in Python or Cython.


\paragraph{\sphinxstyleliteralintitle{RepCode} Module}
\label{\detokenize{ref/LIS/core/RepCode:repcode-module}}

\subparagraph{Description}
\label{\detokenize{ref/LIS/core/RepCode:description}}
This provides the main interface to Representation Code processing as well as some fundamental definitions.

This module is the top level module that imports other sub-modules implemented in Python, Cython or C/C++. The Cython implementations take precedence as this module imports the sub-modules thus:

\begin{sphinxVerbatim}[commandchars=\\\{\}]
\PYG{k+kn}{from} \PYG{n+nn}{TotalDepth}\PYG{n+nn}{.}\PYG{n+nn}{LIS}\PYG{n+nn}{.}\PYG{n+nn}{core}\PYG{n+nn}{.}\PYG{n+nn}{pRepCode} \PYG{k}{import} \PYG{o}{*}
\PYG{k+kn}{from} \PYG{n+nn}{TotalDepth}\PYG{n+nn}{.}\PYG{n+nn}{LIS}\PYG{n+nn}{.}\PYG{n+nn}{core}\PYG{n+nn}{.}\PYG{n+nn}{cRepCode} \PYG{k}{import} \PYG{o}{*}
\end{sphinxVerbatim}


\subparagraph{Usage}
\label{\detokenize{ref/LIS/core/RepCode:usage}}
It is designed to use thus:

\begin{sphinxVerbatim}[commandchars=\\\{\}]
\PYG{k+kn}{from} \PYG{n+nn}{TotalDepth}\PYG{n+nn}{.}\PYG{n+nn}{LIS}\PYG{n+nn}{.}\PYG{n+nn}{core} \PYG{k}{import} \PYG{n}{RepCode}
\end{sphinxVerbatim}


\subparagraph{Reference}
\label{\detokenize{ref/LIS/core/RepCode:module-TotalDepth.LIS.core.RepCode}}\label{\detokenize{ref/LIS/core/RepCode:reference}}\index{TotalDepth.LIS.core.RepCode (module)}
Module for translating raw LIS to an appropriate internal type for LIS
Representation Codes (‘RepCodes’).

This aggregates pRepCode and cRepCode with the latter overwriting the former.

Created on 11 Nov 2010

@author: p2ross
\index{ExceptionRepCodeNoLength}

\begin{fulllineitems}
\phantomsection\label{\detokenize{ref/LIS/core/RepCode:TotalDepth.LIS.core.RepCode.ExceptionRepCodeNoLength}}\pysigline{\sphinxbfcode{exception }\sphinxcode{TotalDepth.LIS.core.RepCode.}\sphinxbfcode{ExceptionRepCodeNoLength}}
Exception for indeterminate length when using Rep Code 65.

\end{fulllineitems}

\index{ExceptionRepCodeRead}

\begin{fulllineitems}
\phantomsection\label{\detokenize{ref/LIS/core/RepCode:TotalDepth.LIS.core.RepCode.ExceptionRepCodeRead}}\pysigline{\sphinxbfcode{exception }\sphinxcode{TotalDepth.LIS.core.RepCode.}\sphinxbfcode{ExceptionRepCodeRead}}
Exception for unknown Representation codes in look up tables.

\end{fulllineitems}

\index{ExceptionRepCodeWrite}

\begin{fulllineitems}
\phantomsection\label{\detokenize{ref/LIS/core/RepCode:TotalDepth.LIS.core.RepCode.ExceptionRepCodeWrite}}\pysigline{\sphinxbfcode{exception }\sphinxcode{TotalDepth.LIS.core.RepCode.}\sphinxbfcode{ExceptionRepCodeWrite}}
Exception for unknown Representation codes in look up tables.

\end{fulllineitems}

\index{read49() (in module TotalDepth.LIS.core.RepCode)}

\begin{fulllineitems}
\phantomsection\label{\detokenize{ref/LIS/core/RepCode:TotalDepth.LIS.core.RepCode.read49}}\pysiglinewithargsret{\sphinxcode{TotalDepth.LIS.core.RepCode.}\sphinxbfcode{read49}}{\emph{theFile}}{}
Returns a Representation Code 49 value from a File object.

\end{fulllineitems}

\index{read50() (in module TotalDepth.LIS.core.RepCode)}

\begin{fulllineitems}
\phantomsection\label{\detokenize{ref/LIS/core/RepCode:TotalDepth.LIS.core.RepCode.read50}}\pysiglinewithargsret{\sphinxcode{TotalDepth.LIS.core.RepCode.}\sphinxbfcode{read50}}{\emph{theFile}}{}
Returns a Representation Code 50 value from a File object.

\end{fulllineitems}

\index{read56() (in module TotalDepth.LIS.core.RepCode)}

\begin{fulllineitems}
\phantomsection\label{\detokenize{ref/LIS/core/RepCode:TotalDepth.LIS.core.RepCode.read56}}\pysiglinewithargsret{\sphinxcode{TotalDepth.LIS.core.RepCode.}\sphinxbfcode{read56}}{\emph{theFile}}{}
Returns a Representation Code 56 value from a File object.

\end{fulllineitems}

\index{read66() (in module TotalDepth.LIS.core.RepCode)}

\begin{fulllineitems}
\phantomsection\label{\detokenize{ref/LIS/core/RepCode:TotalDepth.LIS.core.RepCode.read66}}\pysiglinewithargsret{\sphinxcode{TotalDepth.LIS.core.RepCode.}\sphinxbfcode{read66}}{\emph{theFile}}{}
Returns a Representation Code 66 value from a File object.

\end{fulllineitems}

\index{read68() (in module TotalDepth.LIS.core.RepCode)}

\begin{fulllineitems}
\phantomsection\label{\detokenize{ref/LIS/core/RepCode:TotalDepth.LIS.core.RepCode.read68}}\pysiglinewithargsret{\sphinxcode{TotalDepth.LIS.core.RepCode.}\sphinxbfcode{read68}}{\emph{theFile}}{}
Returns a Representation Code 68 value from a File object.

\end{fulllineitems}

\index{read70() (in module TotalDepth.LIS.core.RepCode)}

\begin{fulllineitems}
\phantomsection\label{\detokenize{ref/LIS/core/RepCode:TotalDepth.LIS.core.RepCode.read70}}\pysiglinewithargsret{\sphinxcode{TotalDepth.LIS.core.RepCode.}\sphinxbfcode{read70}}{\emph{theFile}}{}
Returns a Representation Code 70 value from a File object.

\end{fulllineitems}

\index{read73() (in module TotalDepth.LIS.core.RepCode)}

\begin{fulllineitems}
\phantomsection\label{\detokenize{ref/LIS/core/RepCode:TotalDepth.LIS.core.RepCode.read73}}\pysiglinewithargsret{\sphinxcode{TotalDepth.LIS.core.RepCode.}\sphinxbfcode{read73}}{\emph{theFile}}{}
Returns a Representation Code 73 value from a File object.

\end{fulllineitems}

\index{read77() (in module TotalDepth.LIS.core.RepCode)}

\begin{fulllineitems}
\phantomsection\label{\detokenize{ref/LIS/core/RepCode:TotalDepth.LIS.core.RepCode.read77}}\pysiglinewithargsret{\sphinxcode{TotalDepth.LIS.core.RepCode.}\sphinxbfcode{read77}}{\emph{theFile}}{}
Returns a Representation Code 77 value from a File object.

\end{fulllineitems}

\index{read79() (in module TotalDepth.LIS.core.RepCode)}

\begin{fulllineitems}
\phantomsection\label{\detokenize{ref/LIS/core/RepCode:TotalDepth.LIS.core.RepCode.read79}}\pysiglinewithargsret{\sphinxcode{TotalDepth.LIS.core.RepCode.}\sphinxbfcode{read79}}{\emph{theFile}}{}
Returns a Representation Code 79 value from a File object.

\end{fulllineitems}

\index{readBytes() (in module TotalDepth.LIS.core.RepCode)}

\begin{fulllineitems}
\phantomsection\label{\detokenize{ref/LIS/core/RepCode:TotalDepth.LIS.core.RepCode.readBytes}}\pysiglinewithargsret{\sphinxcode{TotalDepth.LIS.core.RepCode.}\sphinxbfcode{readBytes}}{\emph{theRc}, \emph{theB}, \emph{theLen=None}}{}
Reads a Representation Code from the a bytes() object. If theRc is 65
(string) then theLen must be supplied, up to that length of bytes will be
returned.

\end{fulllineitems}

\index{readBytes130() (in module TotalDepth.LIS.core.RepCode)}

\begin{fulllineitems}
\phantomsection\label{\detokenize{ref/LIS/core/RepCode:TotalDepth.LIS.core.RepCode.readBytes130}}\pysiglinewithargsret{\sphinxcode{TotalDepth.LIS.core.RepCode.}\sphinxbfcode{readBytes130}}{\emph{by}}{}
Reads Dipmeter RepCode 130 and returns a list of (integer) values.

\end{fulllineitems}

\index{readBytes234() (in module TotalDepth.LIS.core.RepCode)}

\begin{fulllineitems}
\phantomsection\label{\detokenize{ref/LIS/core/RepCode:TotalDepth.LIS.core.RepCode.readBytes234}}\pysiglinewithargsret{\sphinxcode{TotalDepth.LIS.core.RepCode.}\sphinxbfcode{readBytes234}}{\emph{by}}{}
Reads Dipmeter RepCode 234 and returns a list of (integer) values.

\end{fulllineitems}

\index{readBytes49() (in module TotalDepth.LIS.core.RepCode)}

\begin{fulllineitems}
\phantomsection\label{\detokenize{ref/LIS/core/RepCode:TotalDepth.LIS.core.RepCode.readBytes49}}\pysiglinewithargsret{\sphinxcode{TotalDepth.LIS.core.RepCode.}\sphinxbfcode{readBytes49}}{\emph{arg}}{}
Returns a Representation Code 49 value from a bytes object.

\end{fulllineitems}

\index{readBytes50() (in module TotalDepth.LIS.core.RepCode)}

\begin{fulllineitems}
\phantomsection\label{\detokenize{ref/LIS/core/RepCode:TotalDepth.LIS.core.RepCode.readBytes50}}\pysiglinewithargsret{\sphinxcode{TotalDepth.LIS.core.RepCode.}\sphinxbfcode{readBytes50}}{\emph{arg}}{}
Returns a Representation Code 50 value from a bytes object.

\end{fulllineitems}

\index{readBytes56() (in module TotalDepth.LIS.core.RepCode)}

\begin{fulllineitems}
\phantomsection\label{\detokenize{ref/LIS/core/RepCode:TotalDepth.LIS.core.RepCode.readBytes56}}\pysiglinewithargsret{\sphinxcode{TotalDepth.LIS.core.RepCode.}\sphinxbfcode{readBytes56}}{\emph{arg}}{}
Returns a Representation Code 56 value from a bytes object.

\end{fulllineitems}

\index{readBytes66() (in module TotalDepth.LIS.core.RepCode)}

\begin{fulllineitems}
\phantomsection\label{\detokenize{ref/LIS/core/RepCode:TotalDepth.LIS.core.RepCode.readBytes66}}\pysiglinewithargsret{\sphinxcode{TotalDepth.LIS.core.RepCode.}\sphinxbfcode{readBytes66}}{\emph{arg}}{}
Returns a Representation Code 66 value from a bytes object.

\end{fulllineitems}

\index{readBytes68() (in module TotalDepth.LIS.core.RepCode)}

\begin{fulllineitems}
\phantomsection\label{\detokenize{ref/LIS/core/RepCode:TotalDepth.LIS.core.RepCode.readBytes68}}\pysiglinewithargsret{\sphinxcode{TotalDepth.LIS.core.RepCode.}\sphinxbfcode{readBytes68}}{\emph{arg}}{}
Returns a Representation Code 68 value from a bytes object.

\end{fulllineitems}

\index{readBytes70() (in module TotalDepth.LIS.core.RepCode)}

\begin{fulllineitems}
\phantomsection\label{\detokenize{ref/LIS/core/RepCode:TotalDepth.LIS.core.RepCode.readBytes70}}\pysiglinewithargsret{\sphinxcode{TotalDepth.LIS.core.RepCode.}\sphinxbfcode{readBytes70}}{\emph{arg}}{}
Returns a Representation Code 70 value from a bytes object.

\end{fulllineitems}

\index{readBytes73() (in module TotalDepth.LIS.core.RepCode)}

\begin{fulllineitems}
\phantomsection\label{\detokenize{ref/LIS/core/RepCode:TotalDepth.LIS.core.RepCode.readBytes73}}\pysiglinewithargsret{\sphinxcode{TotalDepth.LIS.core.RepCode.}\sphinxbfcode{readBytes73}}{\emph{arg}}{}
Returns a Representation Code 73 value from a bytes object.

\end{fulllineitems}

\index{readBytes77() (in module TotalDepth.LIS.core.RepCode)}

\begin{fulllineitems}
\phantomsection\label{\detokenize{ref/LIS/core/RepCode:TotalDepth.LIS.core.RepCode.readBytes77}}\pysiglinewithargsret{\sphinxcode{TotalDepth.LIS.core.RepCode.}\sphinxbfcode{readBytes77}}{\emph{arg}}{}
Returns a Representation Code 77 value from a bytes object.

\end{fulllineitems}

\index{readBytes79() (in module TotalDepth.LIS.core.RepCode)}

\begin{fulllineitems}
\phantomsection\label{\detokenize{ref/LIS/core/RepCode:TotalDepth.LIS.core.RepCode.readBytes79}}\pysiglinewithargsret{\sphinxcode{TotalDepth.LIS.core.RepCode.}\sphinxbfcode{readBytes79}}{\emph{arg}}{}
Returns a Representation Code 79 value from a bytes object.

\end{fulllineitems}

\index{readRepCode() (in module TotalDepth.LIS.core.RepCode)}

\begin{fulllineitems}
\phantomsection\label{\detokenize{ref/LIS/core/RepCode:TotalDepth.LIS.core.RepCode.readRepCode}}\pysiglinewithargsret{\sphinxcode{TotalDepth.LIS.core.RepCode.}\sphinxbfcode{readRepCode}}{\emph{theRc}, \emph{theFile}, \emph{theLen=None}}{}
Reads a Representation Code from the file and returns a value.
If theRc is 65 (string) then theLen must be supplied, up to that length of
bytes will be returned.

\end{fulllineitems}

\index{writeBytes() (in module TotalDepth.LIS.core.RepCode)}

\begin{fulllineitems}
\phantomsection\label{\detokenize{ref/LIS/core/RepCode:TotalDepth.LIS.core.RepCode.writeBytes}}\pysiglinewithargsret{\sphinxcode{TotalDepth.LIS.core.RepCode.}\sphinxbfcode{writeBytes}}{\emph{v}, \emph{r}}{}
Takes a value v and a Representation Code r and converts this to a
bytes() object.

\end{fulllineitems}

\index{writeBytes66() (in module TotalDepth.LIS.core.RepCode)}

\begin{fulllineitems}
\phantomsection\label{\detokenize{ref/LIS/core/RepCode:TotalDepth.LIS.core.RepCode.writeBytes66}}\pysiglinewithargsret{\sphinxcode{TotalDepth.LIS.core.RepCode.}\sphinxbfcode{writeBytes66}}{\emph{v}}{}
Converts a value to a Rep Code 66 and returns the bytes.

\end{fulllineitems}

\index{writeBytes68() (in module TotalDepth.LIS.core.RepCode)}

\begin{fulllineitems}
\phantomsection\label{\detokenize{ref/LIS/core/RepCode:TotalDepth.LIS.core.RepCode.writeBytes68}}\pysiglinewithargsret{\sphinxcode{TotalDepth.LIS.core.RepCode.}\sphinxbfcode{writeBytes68}}{\emph{v}}{}
Converts a value to a Rep Code 68 and returns the bytes.

\end{fulllineitems}

\index{writeBytes73() (in module TotalDepth.LIS.core.RepCode)}

\begin{fulllineitems}
\phantomsection\label{\detokenize{ref/LIS/core/RepCode:TotalDepth.LIS.core.RepCode.writeBytes73}}\pysiglinewithargsret{\sphinxcode{TotalDepth.LIS.core.RepCode.}\sphinxbfcode{writeBytes73}}{\emph{v}}{}
Converts a value to a Rep Code 73 and returns the bytes.

\end{fulllineitems}



\paragraph{\sphinxstyleliteralintitle{pRepCode} Module}
\label{\detokenize{ref/LIS/core/RepCode:prepcode-module}}
This contains Python implementations.


\subparagraph{Usage}
\label{\detokenize{ref/LIS/core/RepCode:id1}}
This module is not designed to imported directly, use \sphinxcode{RepCode} instead. This module can be imported only for test purposes thus:

\begin{sphinxVerbatim}[commandchars=\\\{\}]
\PYG{k+kn}{from} \PYG{n+nn}{TotalDepth}\PYG{n+nn}{.}\PYG{n+nn}{LIS}\PYG{n+nn}{.}\PYG{n+nn}{core} \PYG{k}{import} \PYG{n}{pRepCode}
\end{sphinxVerbatim}


\subparagraph{Reference}
\label{\detokenize{ref/LIS/core/RepCode:id2}}\phantomsection\label{\detokenize{ref/LIS/core/RepCode:module-TotalDepth.LIS.core.pRepCode}}\index{TotalDepth.LIS.core.pRepCode (module)}
Python module for translating raw LIS to an appropriate internal type for LIS
Representation Codes (‘RepCodes’).

Created on 11 Nov 2010

@author: p2ross
\index{ExceptionRepCode}

\begin{fulllineitems}
\phantomsection\label{\detokenize{ref/LIS/core/RepCode:TotalDepth.LIS.core.pRepCode.ExceptionRepCode}}\pysigline{\sphinxbfcode{exception }\sphinxcode{TotalDepth.LIS.core.pRepCode.}\sphinxbfcode{ExceptionRepCode}}
Specialisation of exception for Representation codes.

\end{fulllineitems}

\index{ExceptionRepCodeUnknown}

\begin{fulllineitems}
\phantomsection\label{\detokenize{ref/LIS/core/RepCode:TotalDepth.LIS.core.pRepCode.ExceptionRepCodeUnknown}}\pysigline{\sphinxbfcode{exception }\sphinxcode{TotalDepth.LIS.core.pRepCode.}\sphinxbfcode{ExceptionRepCodeUnknown}}
Specialisation of exception for unknown Representation codes.

\end{fulllineitems}

\index{from49() (in module TotalDepth.LIS.core.pRepCode)}

\begin{fulllineitems}
\phantomsection\label{\detokenize{ref/LIS/core/RepCode:TotalDepth.LIS.core.pRepCode.from49}}\pysiglinewithargsret{\sphinxcode{TotalDepth.LIS.core.pRepCode.}\sphinxbfcode{from49}}{\emph{theWord}}{}
Returns a double from Rep code 49 0x31, 16bit floating point representation.
Value +153 is 0100 1100 1000 1000 or 0x4C88.
Value -153 is 1011 0011 1000 1000 or 0xB388.

\end{fulllineitems}

\index{from50() (in module TotalDepth.LIS.core.pRepCode)}

\begin{fulllineitems}
\phantomsection\label{\detokenize{ref/LIS/core/RepCode:TotalDepth.LIS.core.pRepCode.from50}}\pysiglinewithargsret{\sphinxcode{TotalDepth.LIS.core.pRepCode.}\sphinxbfcode{from50}}{\emph{theWord}}{}
Returns a double from Rep code 0x32, 32bit floating point representation.
Value +153 is 0x00084C80
Value -153 is 0x0008B380

\end{fulllineitems}

\index{from56() (in module TotalDepth.LIS.core.pRepCode)}

\begin{fulllineitems}
\phantomsection\label{\detokenize{ref/LIS/core/RepCode:TotalDepth.LIS.core.pRepCode.from56}}\pysiglinewithargsret{\sphinxcode{TotalDepth.LIS.core.pRepCode.}\sphinxbfcode{from56}}{\emph{theWord}}{}
Returns an integer from Rep code 0x38, signed char representation.

\end{fulllineitems}

\index{from66() (in module TotalDepth.LIS.core.pRepCode)}

\begin{fulllineitems}
\phantomsection\label{\detokenize{ref/LIS/core/RepCode:TotalDepth.LIS.core.pRepCode.from66}}\pysiglinewithargsret{\sphinxcode{TotalDepth.LIS.core.pRepCode.}\sphinxbfcode{from66}}{\emph{theWord}}{}
Returns a Rep code 0x42, unsigned byte from theWord, expected to be an integer.

\end{fulllineitems}

\index{from68() (in module TotalDepth.LIS.core.pRepCode)}

\begin{fulllineitems}
\phantomsection\label{\detokenize{ref/LIS/core/RepCode:TotalDepth.LIS.core.pRepCode.from68}}\pysiglinewithargsret{\sphinxcode{TotalDepth.LIS.core.pRepCode.}\sphinxbfcode{from68}}{\emph{theWord}}{}
Returns a double from a Rep code 68 word (a 32 bit integer).

\end{fulllineitems}

\index{from70() (in module TotalDepth.LIS.core.pRepCode)}

\begin{fulllineitems}
\phantomsection\label{\detokenize{ref/LIS/core/RepCode:TotalDepth.LIS.core.pRepCode.from70}}\pysiglinewithargsret{\sphinxcode{TotalDepth.LIS.core.pRepCode.}\sphinxbfcode{from70}}{\emph{theWord}}{}
Returns a double from a Rep code 0x46, 32bit fixed point.

\end{fulllineitems}

\index{from73() (in module TotalDepth.LIS.core.pRepCode)}

\begin{fulllineitems}
\phantomsection\label{\detokenize{ref/LIS/core/RepCode:TotalDepth.LIS.core.pRepCode.from73}}\pysiglinewithargsret{\sphinxcode{TotalDepth.LIS.core.pRepCode.}\sphinxbfcode{from73}}{\emph{theWord}}{}
Returns a integer from a Rep code 0x49, 32bit signed integer.

\end{fulllineitems}

\index{from77() (in module TotalDepth.LIS.core.pRepCode)}

\begin{fulllineitems}
\phantomsection\label{\detokenize{ref/LIS/core/RepCode:TotalDepth.LIS.core.pRepCode.from77}}\pysiglinewithargsret{\sphinxcode{TotalDepth.LIS.core.pRepCode.}\sphinxbfcode{from77}}{\emph{theWord}}{}
Returns a Rep code 0x4D, 8bit mask from theWord, expected to be an integer.

\end{fulllineitems}

\index{from79() (in module TotalDepth.LIS.core.pRepCode)}

\begin{fulllineitems}
\phantomsection\label{\detokenize{ref/LIS/core/RepCode:TotalDepth.LIS.core.pRepCode.from79}}\pysiglinewithargsret{\sphinxcode{TotalDepth.LIS.core.pRepCode.}\sphinxbfcode{from79}}{\emph{theWord}}{}
Returns an integer from Rep code 0x4F, 16bit signed integer.

\end{fulllineitems}

\index{isInt() (in module TotalDepth.LIS.core.pRepCode)}

\begin{fulllineitems}
\phantomsection\label{\detokenize{ref/LIS/core/RepCode:TotalDepth.LIS.core.pRepCode.isInt}}\pysiglinewithargsret{\sphinxcode{TotalDepth.LIS.core.pRepCode.}\sphinxbfcode{isInt}}{\emph{r}}{}
Returns True if the Rep Code is represented by an integer.

\end{fulllineitems}

\index{lisSize() (in module TotalDepth.LIS.core.pRepCode)}

\begin{fulllineitems}
\phantomsection\label{\detokenize{ref/LIS/core/RepCode:TotalDepth.LIS.core.pRepCode.lisSize}}\pysiglinewithargsret{\sphinxcode{TotalDepth.LIS.core.pRepCode.}\sphinxbfcode{lisSize}}{\emph{r}}{}
Returns the size in bytes for a single instance of a representation code.
Zero means variable length. May raise ExceptionRepCodeUnknown.

\end{fulllineitems}

\index{maxValue() (in module TotalDepth.LIS.core.pRepCode)}

\begin{fulllineitems}
\phantomsection\label{\detokenize{ref/LIS/core/RepCode:TotalDepth.LIS.core.pRepCode.maxValue}}\pysiglinewithargsret{\sphinxcode{TotalDepth.LIS.core.pRepCode.}\sphinxbfcode{maxValue}}{\emph{r}}{}
Returns the maximum value for various representation codes.
May raise ExceptionRepCodeUnknown.

\end{fulllineitems}

\index{minMaxValue() (in module TotalDepth.LIS.core.pRepCode)}

\begin{fulllineitems}
\phantomsection\label{\detokenize{ref/LIS/core/RepCode:TotalDepth.LIS.core.pRepCode.minMaxValue}}\pysiglinewithargsret{\sphinxcode{TotalDepth.LIS.core.pRepCode.}\sphinxbfcode{minMaxValue}}{\emph{r}}{}
Returns a pair; (minimum value, maximum value) for various
representation codes. May raise ExceptionRepCodeUnknown.

\end{fulllineitems}

\index{minValue() (in module TotalDepth.LIS.core.pRepCode)}

\begin{fulllineitems}
\phantomsection\label{\detokenize{ref/LIS/core/RepCode:TotalDepth.LIS.core.pRepCode.minValue}}\pysiglinewithargsret{\sphinxcode{TotalDepth.LIS.core.pRepCode.}\sphinxbfcode{minValue}}{\emph{r}}{}
Returns the minimum value for various representation codes.
May raise ExceptionRepCodeUnknown.

\end{fulllineitems}

\index{read49() (in module TotalDepth.LIS.core.pRepCode)}

\begin{fulllineitems}
\phantomsection\label{\detokenize{ref/LIS/core/RepCode:TotalDepth.LIS.core.pRepCode.read49}}\pysiglinewithargsret{\sphinxcode{TotalDepth.LIS.core.pRepCode.}\sphinxbfcode{read49}}{\emph{theFile}}{}
Returns a Representation Code 49 value from a File object.

\end{fulllineitems}

\index{read50() (in module TotalDepth.LIS.core.pRepCode)}

\begin{fulllineitems}
\phantomsection\label{\detokenize{ref/LIS/core/RepCode:TotalDepth.LIS.core.pRepCode.read50}}\pysiglinewithargsret{\sphinxcode{TotalDepth.LIS.core.pRepCode.}\sphinxbfcode{read50}}{\emph{theFile}}{}
Returns a Representation Code 50 value from a File object.

\end{fulllineitems}

\index{read56() (in module TotalDepth.LIS.core.pRepCode)}

\begin{fulllineitems}
\phantomsection\label{\detokenize{ref/LIS/core/RepCode:TotalDepth.LIS.core.pRepCode.read56}}\pysiglinewithargsret{\sphinxcode{TotalDepth.LIS.core.pRepCode.}\sphinxbfcode{read56}}{\emph{theFile}}{}
Returns a Representation Code 56 value from a File object.

\end{fulllineitems}

\index{read66() (in module TotalDepth.LIS.core.pRepCode)}

\begin{fulllineitems}
\phantomsection\label{\detokenize{ref/LIS/core/RepCode:TotalDepth.LIS.core.pRepCode.read66}}\pysiglinewithargsret{\sphinxcode{TotalDepth.LIS.core.pRepCode.}\sphinxbfcode{read66}}{\emph{theFile}}{}
Returns a Representation Code 66 value from a File object.

\end{fulllineitems}

\index{read68() (in module TotalDepth.LIS.core.pRepCode)}

\begin{fulllineitems}
\phantomsection\label{\detokenize{ref/LIS/core/RepCode:TotalDepth.LIS.core.pRepCode.read68}}\pysiglinewithargsret{\sphinxcode{TotalDepth.LIS.core.pRepCode.}\sphinxbfcode{read68}}{\emph{theFile}}{}
Returns a Representation Code 68 value from a File object.

\end{fulllineitems}

\index{read70() (in module TotalDepth.LIS.core.pRepCode)}

\begin{fulllineitems}
\phantomsection\label{\detokenize{ref/LIS/core/RepCode:TotalDepth.LIS.core.pRepCode.read70}}\pysiglinewithargsret{\sphinxcode{TotalDepth.LIS.core.pRepCode.}\sphinxbfcode{read70}}{\emph{theFile}}{}
Returns a Representation Code 70 value from a File object.

\end{fulllineitems}

\index{read73() (in module TotalDepth.LIS.core.pRepCode)}

\begin{fulllineitems}
\phantomsection\label{\detokenize{ref/LIS/core/RepCode:TotalDepth.LIS.core.pRepCode.read73}}\pysiglinewithargsret{\sphinxcode{TotalDepth.LIS.core.pRepCode.}\sphinxbfcode{read73}}{\emph{theFile}}{}
Returns a Representation Code 73 value from a File object.

\end{fulllineitems}

\index{read77() (in module TotalDepth.LIS.core.pRepCode)}

\begin{fulllineitems}
\phantomsection\label{\detokenize{ref/LIS/core/RepCode:TotalDepth.LIS.core.pRepCode.read77}}\pysiglinewithargsret{\sphinxcode{TotalDepth.LIS.core.pRepCode.}\sphinxbfcode{read77}}{\emph{theFile}}{}
Returns a Representation Code 77 value from a File object.

\end{fulllineitems}

\index{read79() (in module TotalDepth.LIS.core.pRepCode)}

\begin{fulllineitems}
\phantomsection\label{\detokenize{ref/LIS/core/RepCode:TotalDepth.LIS.core.pRepCode.read79}}\pysiglinewithargsret{\sphinxcode{TotalDepth.LIS.core.pRepCode.}\sphinxbfcode{read79}}{\emph{theFile}}{}
Returns a Representation Code 79 value from a File object.

\end{fulllineitems}

\index{to49() (in module TotalDepth.LIS.core.pRepCode)}

\begin{fulllineitems}
\phantomsection\label{\detokenize{ref/LIS/core/RepCode:TotalDepth.LIS.core.pRepCode.to49}}\pysiglinewithargsret{\sphinxcode{TotalDepth.LIS.core.pRepCode.}\sphinxbfcode{to49}}{\emph{theVal}}{}
Converts a double to Rep code 49 0x31, 16bit floating point representation.
Value +153 is 0100 1100 1000 1000 or 0x4C88.
Value -153 is 1011 0011 1000 1000 or 0xB388.

\end{fulllineitems}

\index{to50() (in module TotalDepth.LIS.core.pRepCode)}

\begin{fulllineitems}
\phantomsection\label{\detokenize{ref/LIS/core/RepCode:TotalDepth.LIS.core.pRepCode.to50}}\pysiglinewithargsret{\sphinxcode{TotalDepth.LIS.core.pRepCode.}\sphinxbfcode{to50}}{\emph{theVal}}{}
Converts a double to Rep code 0x32, 32bit floating point representation.

\end{fulllineitems}

\index{to56() (in module TotalDepth.LIS.core.pRepCode)}

\begin{fulllineitems}
\phantomsection\label{\detokenize{ref/LIS/core/RepCode:TotalDepth.LIS.core.pRepCode.to56}}\pysiglinewithargsret{\sphinxcode{TotalDepth.LIS.core.pRepCode.}\sphinxbfcode{to56}}{\emph{theVal}}{}
Converts a double to Rep code 0x38, signed char representation.

\end{fulllineitems}

\index{to66() (in module TotalDepth.LIS.core.pRepCode)}

\begin{fulllineitems}
\phantomsection\label{\detokenize{ref/LIS/core/RepCode:TotalDepth.LIS.core.pRepCode.to66}}\pysiglinewithargsret{\sphinxcode{TotalDepth.LIS.core.pRepCode.}\sphinxbfcode{to66}}{\emph{theVal}}{}
Converts theVal to representation code 66 integer from theWord.

\end{fulllineitems}

\index{to68() (in module TotalDepth.LIS.core.pRepCode)}

\begin{fulllineitems}
\phantomsection\label{\detokenize{ref/LIS/core/RepCode:TotalDepth.LIS.core.pRepCode.to68}}\pysiglinewithargsret{\sphinxcode{TotalDepth.LIS.core.pRepCode.}\sphinxbfcode{to68}}{\emph{v}}{}
Returns Representation code 68 as a 32 bit integer from a double.

\end{fulllineitems}

\index{to70() (in module TotalDepth.LIS.core.pRepCode)}

\begin{fulllineitems}
\phantomsection\label{\detokenize{ref/LIS/core/RepCode:TotalDepth.LIS.core.pRepCode.to70}}\pysiglinewithargsret{\sphinxcode{TotalDepth.LIS.core.pRepCode.}\sphinxbfcode{to70}}{\emph{v}}{}
Returns Rep code 0x46, 32bit fixed point from an int or double.

\end{fulllineitems}

\index{to73() (in module TotalDepth.LIS.core.pRepCode)}

\begin{fulllineitems}
\phantomsection\label{\detokenize{ref/LIS/core/RepCode:TotalDepth.LIS.core.pRepCode.to73}}\pysiglinewithargsret{\sphinxcode{TotalDepth.LIS.core.pRepCode.}\sphinxbfcode{to73}}{\emph{v}}{}
Returns Rep code 0x49, 32bit signed integer from an int or double.

\end{fulllineitems}

\index{to77() (in module TotalDepth.LIS.core.pRepCode)}

\begin{fulllineitems}
\phantomsection\label{\detokenize{ref/LIS/core/RepCode:TotalDepth.LIS.core.pRepCode.to77}}\pysiglinewithargsret{\sphinxcode{TotalDepth.LIS.core.pRepCode.}\sphinxbfcode{to77}}{\emph{theVal}}{}
Converts theVal to Rep code 0x4D, 8bit mask from theVal.

\end{fulllineitems}

\index{to79() (in module TotalDepth.LIS.core.pRepCode)}

\begin{fulllineitems}
\phantomsection\label{\detokenize{ref/LIS/core/RepCode:TotalDepth.LIS.core.pRepCode.to79}}\pysiglinewithargsret{\sphinxcode{TotalDepth.LIS.core.pRepCode.}\sphinxbfcode{to79}}{\emph{theVal}}{}
Converts a double to Rep code 0x4F, 16bit signed integer.

\end{fulllineitems}

\index{wordLength() (in module TotalDepth.LIS.core.pRepCode)}

\begin{fulllineitems}
\phantomsection\label{\detokenize{ref/LIS/core/RepCode:TotalDepth.LIS.core.pRepCode.wordLength}}\pysiglinewithargsret{\sphinxcode{TotalDepth.LIS.core.pRepCode.}\sphinxbfcode{wordLength}}{\emph{r}}{}
Returns the word length in bytes used by a representation code.
NOTE: This is subtly different from lisSize as it take into account dipmeter
sub-channels
Zero means variable length. May raise ExceptionRepCodeUnknown.

\end{fulllineitems}



\paragraph{\sphinxstyleliteralintitle{cRepCode} Module}
\label{\detokenize{ref/LIS/core/RepCode:crepcode-module}}
This contains Cython implementations. By the import mechanism used be \sphinxcode{RepCode} these implementations take precedence over the implementations in \sphinxcode{pRepCode}.


\subparagraph{Usage}
\label{\detokenize{ref/LIS/core/RepCode:id3}}
This module is not designed to imported directly, use \sphinxcode{RepCode} instead. This module can be imported only for test purposes thus:

\begin{sphinxVerbatim}[commandchars=\\\{\}]
\PYG{k+kn}{from} \PYG{n+nn}{TotalDepth}\PYG{n+nn}{.}\PYG{n+nn}{LIS}\PYG{n+nn}{.}\PYG{n+nn}{core} \PYG{k}{import} \PYG{n}{cRepCode}
\end{sphinxVerbatim}


\subparagraph{Reference}
\label{\detokenize{ref/LIS/core/RepCode:id4}}\phantomsection\label{\detokenize{ref/LIS/core/RepCode:module-TotalDepth.LIS.core.cRepCode}}\index{TotalDepth.LIS.core.cRepCode (module)}
Cython module for translating raw LIS to an appropriate internal type for LIS
Representation Codes (‘RepCodes’).

Internal types are double, int or string.


\paragraph{Testing}
\label{\detokenize{ref/LIS/core/RepCode:testing}}
The unit tests are in \sphinxcode{test/TestRepCode.py} and \sphinxcode{test/TestRepCode68.py}.


\subsubsection{Units}
\label{\detokenize{ref/LIS/core/Units:units}}\label{\detokenize{ref/LIS/core/Units::doc}}\phantomsection\label{\detokenize{ref/LIS/core/Units:module-TotalDepth.LIS.core.Units}}\index{TotalDepth.LIS.core.Units (module)}
Provides unit conversion for LIS79.

The \_\_RAW\_UNIT\_MAP is the master map from which all other data is derived.
Its format is as follows:

key - The unit category as four bytes representing uppercase ASCII characters.
value - A tuple of three fields:
\begin{itemize}
\item {} 
{[}0{]} - Descriptive string of the unit category.

\item {} 
{[}1{]} - The base unit name as four bytes representing uppercase ASCII characters.

\item {} 
{[}2{]} - A tuple the contents of which is a four or five item tuple:
\begin{itemize}
\item {} \begin{description}
\item[{If four members:}] \leavevmode\begin{itemize}
\item {} 
{[}2{]}{[}0{]} - The unit name as four bytes representing uppercase ASCII characters.

\item {} 
{[}2{]}{[}1{]} - The multiplier as a float.

\item {} 
{[}2{]}{[}2{]} - Descriptive string of the units.

\item {} \begin{description}
\item[{{[}2{]}{[}3{]} - The unit name that this is an alternate for as four bytes}] \leavevmode
representing uppercase ASCII characters, or four spaces.

\end{description}

\end{itemize}

\end{description}

\item {} \begin{description}
\item[{If five members:}] \leavevmode\begin{itemize}
\item {} 
{[}2{]}{[}0{]} - The unit name as four bytes representing uppercase ASCII characters.

\item {} 
{[}2{]}{[}1{]} - The multiplier as a float.

\item {} 
{[}2{]}{[}2{]} - The offset as a float.

\item {} 
{[}2{]}{[}3{]} - Descriptive string of the units.

\item {} \begin{description}
\item[{{[}2{]}{[}4{]} - The unit name that this is an alternate for as four bytes}] \leavevmode
representing uppercase ASCII characters or four spaces.

\end{description}

\end{itemize}

\end{description}

\end{itemize}

\end{itemize}

The unit name should also be unique.

TODO: Clean up units by making reciprocal e.g. 1/6.0 rather than 0.166666…

TODO: Check each unit for errors.
\index{ExceptionUnits}

\begin{fulllineitems}
\phantomsection\label{\detokenize{ref/LIS/core/Units:TotalDepth.LIS.core.Units.ExceptionUnits}}\pysigline{\sphinxbfcode{exception }\sphinxcode{TotalDepth.LIS.core.Units.}\sphinxbfcode{ExceptionUnits}}
Specialisation of exception for Unit conversion.

\end{fulllineitems}

\index{ExceptionUnitsMissmatchedCategory}

\begin{fulllineitems}
\phantomsection\label{\detokenize{ref/LIS/core/Units:TotalDepth.LIS.core.Units.ExceptionUnitsMissmatchedCategory}}\pysigline{\sphinxbfcode{exception }\sphinxcode{TotalDepth.LIS.core.Units.}\sphinxbfcode{ExceptionUnitsMissmatchedCategory}}
When a two units do not exist in the same category.

\end{fulllineitems}

\index{ExceptionUnitsNoUnitInCategory}

\begin{fulllineitems}
\phantomsection\label{\detokenize{ref/LIS/core/Units:TotalDepth.LIS.core.Units.ExceptionUnitsNoUnitInCategory}}\pysigline{\sphinxbfcode{exception }\sphinxcode{TotalDepth.LIS.core.Units.}\sphinxbfcode{ExceptionUnitsNoUnitInCategory}}
When a unit does not exist in a category.

\end{fulllineitems}

\index{ExceptionUnitsUnknownCategory}

\begin{fulllineitems}
\phantomsection\label{\detokenize{ref/LIS/core/Units:TotalDepth.LIS.core.Units.ExceptionUnitsUnknownCategory}}\pysigline{\sphinxbfcode{exception }\sphinxcode{TotalDepth.LIS.core.Units.}\sphinxbfcode{ExceptionUnitsUnknownCategory}}
When a unit category does not exist.

\end{fulllineitems}

\index{ExceptionUnitsUnknownUnit}

\begin{fulllineitems}
\phantomsection\label{\detokenize{ref/LIS/core/Units:TotalDepth.LIS.core.Units.ExceptionUnitsUnknownUnit}}\pysigline{\sphinxbfcode{exception }\sphinxcode{TotalDepth.LIS.core.Units.}\sphinxbfcode{ExceptionUnitsUnknownUnit}}
When a unit does not exist.

\end{fulllineitems}

\index{UnitConvert (class in TotalDepth.LIS.core.Units)}

\begin{fulllineitems}
\phantomsection\label{\detokenize{ref/LIS/core/Units:TotalDepth.LIS.core.Units.UnitConvert}}\pysiglinewithargsret{\sphinxbfcode{class }\sphinxcode{TotalDepth.LIS.core.Units.}\sphinxbfcode{UnitConvert}}{\emph{tup}}{}
Internal data structure for this module that representas a particular unit of measure.
Takes a 4 or 5 member tuple from \_\_RAW\_UNIT\_MAP.
\index{convert() (TotalDepth.LIS.core.Units.UnitConvert method)}

\begin{fulllineitems}
\phantomsection\label{\detokenize{ref/LIS/core/Units:TotalDepth.LIS.core.Units.UnitConvert.convert}}\pysiglinewithargsret{\sphinxbfcode{convert}}{\emph{val}, \emph{other}}{}
Convert a value from me to the other where other is a UnitConvert object.

\end{fulllineitems}


\end{fulllineitems}

\index{UnitConvertCategory (class in TotalDepth.LIS.core.Units)}

\begin{fulllineitems}
\phantomsection\label{\detokenize{ref/LIS/core/Units:TotalDepth.LIS.core.Units.UnitConvertCategory}}\pysiglinewithargsret{\sphinxbfcode{class }\sphinxcode{TotalDepth.LIS.core.Units.}\sphinxbfcode{UnitConvertCategory}}{\emph{theCat}, \emph{theDesc}, \emph{theBaseUnitName}, \emph{theUnitS}}{}
Internal module data structure that represents a category of units such as linear length.

theCat is the unit category.

theDesc is the description of that category.

theBaseUnitName is the name of the base units for the category.
For example for linear lenght this is b’M   ‘.

theUnitS is a list of unit names.
\index{convert() (TotalDepth.LIS.core.Units.UnitConvertCategory method)}

\begin{fulllineitems}
\phantomsection\label{\detokenize{ref/LIS/core/Units:TotalDepth.LIS.core.Units.UnitConvertCategory.convert}}\pysiglinewithargsret{\sphinxbfcode{convert}}{\emph{v}, \emph{u\_1}, \emph{u\_2}}{}
Returns a value converted from one units to another.
e.g. convert(1.2, “FEET”, “INCH”)

\end{fulllineitems}

\index{unitConvertor() (TotalDepth.LIS.core.Units.UnitConvertCategory method)}

\begin{fulllineitems}
\phantomsection\label{\detokenize{ref/LIS/core/Units:TotalDepth.LIS.core.Units.UnitConvertCategory.unitConvertor}}\pysiglinewithargsret{\sphinxbfcode{unitConvertor}}{\emph{u}}{}
Returns a UnitConvert object corresponding to the name u.
Will raise a ExceptionUnitsNoUnitInCategory if not found.

\end{fulllineitems}

\index{units() (TotalDepth.LIS.core.Units.UnitConvertCategory method)}

\begin{fulllineitems}
\phantomsection\label{\detokenize{ref/LIS/core/Units:TotalDepth.LIS.core.Units.UnitConvertCategory.units}}\pysiglinewithargsret{\sphinxbfcode{units}}{}{}
Reuturns a list of unit names for this category.

\end{fulllineitems}


\end{fulllineitems}

\index{category() (in module TotalDepth.LIS.core.Units)}

\begin{fulllineitems}
\phantomsection\label{\detokenize{ref/LIS/core/Units:TotalDepth.LIS.core.Units.category}}\pysiglinewithargsret{\sphinxcode{TotalDepth.LIS.core.Units.}\sphinxbfcode{category}}{\emph{unit}}{}
Returns the category of the unit. May raise a ExceptionUnitsUnknownUnit.

\end{fulllineitems}

\index{categoryDescription() (in module TotalDepth.LIS.core.Units)}

\begin{fulllineitems}
\phantomsection\label{\detokenize{ref/LIS/core/Units:TotalDepth.LIS.core.Units.categoryDescription}}\pysiglinewithargsret{\sphinxcode{TotalDepth.LIS.core.Units.}\sphinxbfcode{categoryDescription}}{\emph{theCat}}{}
Returns the description of a unit category.

\end{fulllineitems}

\index{convert() (in module TotalDepth.LIS.core.Units)}

\begin{fulllineitems}
\phantomsection\label{\detokenize{ref/LIS/core/Units:TotalDepth.LIS.core.Units.convert}}\pysiglinewithargsret{\sphinxcode{TotalDepth.LIS.core.Units.}\sphinxbfcode{convert}}{\emph{v}, \emph{u\_1}, \emph{u\_2}}{}
Returns a value converted from one units to another. e.g. convert(1.2, b”FEET”, b”INCH”).

Will raise an ExceptionUnitsUnknownUnit if either unit is unknown.

Will raise an ExceptionUnitsMissmatchedCategory is both units doe not belong is the same unit category.

\end{fulllineitems}

\index{hasUnit() (in module TotalDepth.LIS.core.Units)}

\begin{fulllineitems}
\phantomsection\label{\detokenize{ref/LIS/core/Units:TotalDepth.LIS.core.Units.hasUnit}}\pysiglinewithargsret{\sphinxcode{TotalDepth.LIS.core.Units.}\sphinxbfcode{hasUnit}}{\emph{u}}{}
Returns True if I have that unit e,g, b”FEET”.

\end{fulllineitems}

\index{hasUnitCategory() (in module TotalDepth.LIS.core.Units)}

\begin{fulllineitems}
\phantomsection\label{\detokenize{ref/LIS/core/Units:TotalDepth.LIS.core.Units.hasUnitCategory}}\pysiglinewithargsret{\sphinxcode{TotalDepth.LIS.core.Units.}\sphinxbfcode{hasUnitCategory}}{\emph{c}}{}
Returns True if I have that unit category e.g. b”TIME”.

\end{fulllineitems}

\index{opticalUnits() (in module TotalDepth.LIS.core.Units)}

\begin{fulllineitems}
\phantomsection\label{\detokenize{ref/LIS/core/Units:TotalDepth.LIS.core.Units.opticalUnits}}\pysiglinewithargsret{\sphinxcode{TotalDepth.LIS.core.Units.}\sphinxbfcode{opticalUnits}}{\emph{u}}{}
If possible returns the ‘optical’ units i.e. user friendly units.
For example the ‘optical’ units of b’.1IN’ are b’FEET’.
Failure returns the argument.

\end{fulllineitems}

\index{realUnitName() (in module TotalDepth.LIS.core.Units)}

\begin{fulllineitems}
\phantomsection\label{\detokenize{ref/LIS/core/Units:TotalDepth.LIS.core.Units.realUnitName}}\pysiglinewithargsret{\sphinxcode{TotalDepth.LIS.core.Units.}\sphinxbfcode{realUnitName}}{\emph{u}}{}
Returns the real unit name or None if u is the ‘real’ unit
e.g. the ‘real’ unit name for b”FT  ” is b”FEET”.
May raise a ExceptionUnits or descendent.

\end{fulllineitems}

\index{retUnitConvert() (in module TotalDepth.LIS.core.Units)}

\begin{fulllineitems}
\phantomsection\label{\detokenize{ref/LIS/core/Units:TotalDepth.LIS.core.Units.retUnitConvert}}\pysiglinewithargsret{\sphinxcode{TotalDepth.LIS.core.Units.}\sphinxbfcode{retUnitConvert}}{\emph{u}}{}
Returns a UnitConvert object for the unit.
May raise a ExceptionUnits or descendent.

\end{fulllineitems}

\index{retUnitConvertCategory() (in module TotalDepth.LIS.core.Units)}

\begin{fulllineitems}
\phantomsection\label{\detokenize{ref/LIS/core/Units:TotalDepth.LIS.core.Units.retUnitConvertCategory}}\pysiglinewithargsret{\sphinxcode{TotalDepth.LIS.core.Units.}\sphinxbfcode{retUnitConvertCategory}}{\emph{c}}{}
Returns a UnitConvertCategory object for the category.
May raise a ExceptionUnits or descendent.

\end{fulllineitems}

\index{unitCategories() (in module TotalDepth.LIS.core.Units)}

\begin{fulllineitems}
\phantomsection\label{\detokenize{ref/LIS/core/Units:TotalDepth.LIS.core.Units.unitCategories}}\pysiglinewithargsret{\sphinxcode{TotalDepth.LIS.core.Units.}\sphinxbfcode{unitCategories}}{}{}
Returns a list of the unit categories.

\end{fulllineitems}

\index{unitDescription() (in module TotalDepth.LIS.core.Units)}

\begin{fulllineitems}
\phantomsection\label{\detokenize{ref/LIS/core/Units:TotalDepth.LIS.core.Units.unitDescription}}\pysiglinewithargsret{\sphinxcode{TotalDepth.LIS.core.Units.}\sphinxbfcode{unitDescription}}{\emph{u}}{}
Returns the description of the unit.
e.g. Given “.1IN” returns “Tenth-inches”.
May raise a ExceptionUnits or descendent.

\end{fulllineitems}

\index{units() (in module TotalDepth.LIS.core.Units)}

\begin{fulllineitems}
\phantomsection\label{\detokenize{ref/LIS/core/Units:TotalDepth.LIS.core.Units.units}}\pysiglinewithargsret{\sphinxcode{TotalDepth.LIS.core.Units.}\sphinxbfcode{units}}{\emph{theCat=None}}{}
Returns an unordered list of unit names. If category is None all unit
names are returned, otherwise the unit names for a particular category are
returned. This may raise a ExceptionUnitsUnknownCategory if the category
does not exist.

\end{fulllineitems}



\paragraph{Examples}
\label{\detokenize{ref/LIS/core/Units:examples}}
Converting bytes objects:

\begin{sphinxVerbatim}[commandchars=\\\{\}]
\PYG{k+kn}{from} \PYG{n+nn}{TotalDepth}\PYG{n+nn}{.}\PYG{n+nn}{LIS}\PYG{n+nn}{.}\PYG{n+nn}{core} \PYG{k}{import} \PYG{n}{Units}

\PYG{n}{v} \PYG{o}{=} \PYG{n}{Units}\PYG{o}{.}\PYG{n}{convert}\PYG{p}{(}\PYG{l+m+mf}{1.0}\PYG{p}{,} \PYG{l+s+sa}{b}\PYG{l+s+s2}{\PYGZdq{}}\PYG{l+s+s2}{M   }\PYG{l+s+s2}{\PYGZdq{}}\PYG{p}{,} \PYG{l+s+sa}{b}\PYG{l+s+s2}{\PYGZdq{}}\PYG{l+s+s2}{FEET}\PYG{l+s+s2}{\PYGZdq{}}\PYG{p}{)}
\PYG{c+c1}{\PYGZsh{} v is now 3.281}
\end{sphinxVerbatim}


\paragraph{Testing}
\label{\detokenize{ref/LIS/core/Units:testing}}
The unit tests are in \sphinxcode{test/TestUnits.py}.


\subsubsection{EngVal}
\label{\detokenize{ref/LIS/core/EngVal:engval}}\label{\detokenize{ref/LIS/core/EngVal::doc}}\label{\detokenize{ref/LIS/core/EngVal:module-TotalDepth.LIS.core.EngVal}}\index{TotalDepth.LIS.core.EngVal (module)}
This module handles engineering values i.e. a real numeric value
associated with a unit-of-measure.


\paragraph{Operator Overloading}
\label{\detokenize{ref/LIS/core/EngVal:operator-overloading}}
EngVals can be operated on by a real number or an other EngVal.


\subparagraph{Addition and subtraction}
\label{\detokenize{ref/LIS/core/EngVal:addition-and-subtraction}}
The operators \sphinxcode{+, -, +=, -=} work on mixed real numbers and EngVals as expected.
If two EngVal objects then unit conversion is performed before the operation.
If the two EngVals have units in different categories an ExceptionUnit will be raised.


\subparagraph{Division}
\label{\detokenize{ref/LIS/core/EngVal:division}}
Operators \sphinxcode{/, /=} are implemented.

Division of an EngVal by a real number preserves the EngVal units.

Division of an EngVal by an EngVal results in the following:
\begin{itemize}
\item {} 
It preserves the EngVal units if the denominator units are DIMENSIONLESS.

\item {} \begin{description}
\item[{The result will have DIMENSIONLESS EngVal units if the numerator and denominator}] \leavevmode
have units that are freely convertible i.e. in the same unit category.

\end{description}

\item {} 
An ExceptionUnit will be raised.

\end{itemize}


\subparagraph{Multiplication}
\label{\detokenize{ref/LIS/core/EngVal:multiplication}}
Operators \sphinxcode{*, *=} are implemented for reals and EngVals.
If an EngVal the units must be DIMENSIONLESS.


\paragraph{Notes}
\label{\detokenize{ref/LIS/core/EngVal:notes}}
rval operators are implemented and this can result in type promotion.
For example the result of 4.0 * EngVal(16, b’INCH’) is an EngVal(64.0, b’INCH’).

Created on 24 Nov 2010


\paragraph{EngVal Reference}
\label{\detokenize{ref/LIS/core/EngVal:engval-reference}}\index{EngVal (class in TotalDepth.LIS.core.EngVal)}

\begin{fulllineitems}
\phantomsection\label{\detokenize{ref/LIS/core/EngVal:TotalDepth.LIS.core.EngVal.EngVal}}\pysiglinewithargsret{\sphinxbfcode{class }\sphinxcode{TotalDepth.LIS.core.EngVal.}\sphinxbfcode{EngVal}}{\emph{theVal}, \emph{theUom=Mnem(b'x00x00x00x00')}}{}
Represents an engineering value that consists of a numeric value and a
unit of measure (usually a string).
\index{\_\_add\_\_() (TotalDepth.LIS.core.EngVal.EngVal method)}

\begin{fulllineitems}
\phantomsection\label{\detokenize{ref/LIS/core/EngVal:TotalDepth.LIS.core.EngVal.EngVal.__add__}}\pysiglinewithargsret{\sphinxbfcode{\_\_add\_\_}}{\emph{other}}{}
Overload self+other, returned result has the sum of self and other.
other can be an EngVal or a Real number.
The units chosen are self’s.

\end{fulllineitems}

\index{\_\_eq\_\_() (TotalDepth.LIS.core.EngVal.EngVal method)}

\begin{fulllineitems}
\phantomsection\label{\detokenize{ref/LIS/core/EngVal:TotalDepth.LIS.core.EngVal.EngVal.__eq__}}\pysiglinewithargsret{\sphinxbfcode{\_\_eq\_\_}}{\emph{other}}{}
True if self == other False otherwise.
If other is an EngVal unit conversion is performed which may raise an ExceptionUnit.
If other is a real number then self units are ignored.

\end{fulllineitems}

\index{\_\_ge\_\_() (TotalDepth.LIS.core.EngVal.EngVal method)}

\begin{fulllineitems}
\phantomsection\label{\detokenize{ref/LIS/core/EngVal:TotalDepth.LIS.core.EngVal.EngVal.__ge__}}\pysiglinewithargsret{\sphinxbfcode{\_\_ge\_\_}}{\emph{other}}{}
True if self \textgreater{}= other False otherwise.
If other is an EngVal unit conversion is performed which may raise an ExceptionUnit.
If other is a real number then self units are ignored.

\end{fulllineitems}

\index{\_\_gt\_\_() (TotalDepth.LIS.core.EngVal.EngVal method)}

\begin{fulllineitems}
\phantomsection\label{\detokenize{ref/LIS/core/EngVal:TotalDepth.LIS.core.EngVal.EngVal.__gt__}}\pysiglinewithargsret{\sphinxbfcode{\_\_gt\_\_}}{\emph{other}}{}
True if self \textgreater{} other False otherwise.
If other is an EngVal unit conversion is performed which may raise an ExceptionUnit.
If other is a real number then self units are ignored.

\end{fulllineitems}

\index{\_\_iadd\_\_() (TotalDepth.LIS.core.EngVal.EngVal method)}

\begin{fulllineitems}
\phantomsection\label{\detokenize{ref/LIS/core/EngVal:TotalDepth.LIS.core.EngVal.EngVal.__iadd__}}\pysiglinewithargsret{\sphinxbfcode{\_\_iadd\_\_}}{\emph{other}}{}
Addition in place, other must be a real number or an EngVal where
it is converted to my units.

\end{fulllineitems}

\index{\_\_imul\_\_() (TotalDepth.LIS.core.EngVal.EngVal method)}

\begin{fulllineitems}
\phantomsection\label{\detokenize{ref/LIS/core/EngVal:TotalDepth.LIS.core.EngVal.EngVal.__imul__}}\pysiglinewithargsret{\sphinxbfcode{\_\_imul\_\_}}{\emph{other}}{}
Overload self *= other. other must be a real number or a dimensionless EngVal.

\end{fulllineitems}

\index{\_\_isub\_\_() (TotalDepth.LIS.core.EngVal.EngVal method)}

\begin{fulllineitems}
\phantomsection\label{\detokenize{ref/LIS/core/EngVal:TotalDepth.LIS.core.EngVal.EngVal.__isub__}}\pysiglinewithargsret{\sphinxbfcode{\_\_isub\_\_}}{\emph{other}}{}
Subtraction in place, other must be a real number or an EngVal where
it is converted to my units.

\end{fulllineitems}

\index{\_\_itruediv\_\_() (TotalDepth.LIS.core.EngVal.EngVal method)}

\begin{fulllineitems}
\phantomsection\label{\detokenize{ref/LIS/core/EngVal:TotalDepth.LIS.core.EngVal.EngVal.__itruediv__}}\pysiglinewithargsret{\sphinxbfcode{\_\_itruediv\_\_}}{\emph{other}}{}
Overload self /= other. other must be a real number or a dimensionless EngVal.

\end{fulllineitems}

\index{\_\_le\_\_() (TotalDepth.LIS.core.EngVal.EngVal method)}

\begin{fulllineitems}
\phantomsection\label{\detokenize{ref/LIS/core/EngVal:TotalDepth.LIS.core.EngVal.EngVal.__le__}}\pysiglinewithargsret{\sphinxbfcode{\_\_le\_\_}}{\emph{other}}{}
True if self \textless{}= other False otherwise.
If other is an EngVal unit conversion is performed which may raise an ExceptionUnit.
If other is a real number then self units are ignored.

\end{fulllineitems}

\index{\_\_lt\_\_() (TotalDepth.LIS.core.EngVal.EngVal method)}

\begin{fulllineitems}
\phantomsection\label{\detokenize{ref/LIS/core/EngVal:TotalDepth.LIS.core.EngVal.EngVal.__lt__}}\pysiglinewithargsret{\sphinxbfcode{\_\_lt\_\_}}{\emph{other}}{}
True if self \textless{} other False otherwise.
If other is an EngVal unit conversion is performed which may raise an ExceptionUnit.
If other is a real number then self units are ignored.

\end{fulllineitems}

\index{\_\_mul\_\_() (TotalDepth.LIS.core.EngVal.EngVal method)}

\begin{fulllineitems}
\phantomsection\label{\detokenize{ref/LIS/core/EngVal:TotalDepth.LIS.core.EngVal.EngVal.__mul__}}\pysiglinewithargsret{\sphinxbfcode{\_\_mul\_\_}}{\emph{other}}{}
Overload self*other. other must be a real number or a dimensionless EngVal.

\end{fulllineitems}

\index{\_\_ne\_\_() (TotalDepth.LIS.core.EngVal.EngVal method)}

\begin{fulllineitems}
\phantomsection\label{\detokenize{ref/LIS/core/EngVal:TotalDepth.LIS.core.EngVal.EngVal.__ne__}}\pysiglinewithargsret{\sphinxbfcode{\_\_ne\_\_}}{\emph{other}}{}
True if self != other False otherwise.
If other is an EngVal unit conversion is performed which may raise an ExceptionUnit.
If other is a real number then self units are ignored.

\end{fulllineitems}

\index{\_\_radd\_\_() (TotalDepth.LIS.core.EngVal.EngVal method)}

\begin{fulllineitems}
\phantomsection\label{\detokenize{ref/LIS/core/EngVal:TotalDepth.LIS.core.EngVal.EngVal.__radd__}}\pysiglinewithargsret{\sphinxbfcode{\_\_radd\_\_}}{\emph{other}}{}
Right value addition, see \_\_add\_\_().

\end{fulllineitems}

\index{\_\_rmul\_\_() (TotalDepth.LIS.core.EngVal.EngVal method)}

\begin{fulllineitems}
\phantomsection\label{\detokenize{ref/LIS/core/EngVal:TotalDepth.LIS.core.EngVal.EngVal.__rmul__}}\pysiglinewithargsret{\sphinxbfcode{\_\_rmul\_\_}}{\emph{other}}{}
Right value multiplication, see \_\_mul\_\_().

\end{fulllineitems}

\index{\_\_rsub\_\_() (TotalDepth.LIS.core.EngVal.EngVal method)}

\begin{fulllineitems}
\phantomsection\label{\detokenize{ref/LIS/core/EngVal:TotalDepth.LIS.core.EngVal.EngVal.__rsub__}}\pysiglinewithargsret{\sphinxbfcode{\_\_rsub\_\_}}{\emph{other}}{}
Right value subtraction, see \_\_sub\_\_().

\end{fulllineitems}

\index{\_\_rtruediv\_\_() (TotalDepth.LIS.core.EngVal.EngVal method)}

\begin{fulllineitems}
\phantomsection\label{\detokenize{ref/LIS/core/EngVal:TotalDepth.LIS.core.EngVal.EngVal.__rtruediv__}}\pysiglinewithargsret{\sphinxbfcode{\_\_rtruediv\_\_}}{\emph{other}}{}
Right value true division, see \_\_truediv\_\_().

\end{fulllineitems}

\index{\_\_str\_\_() (TotalDepth.LIS.core.EngVal.EngVal method)}

\begin{fulllineitems}
\phantomsection\label{\detokenize{ref/LIS/core/EngVal:TotalDepth.LIS.core.EngVal.EngVal.__str__}}\pysiglinewithargsret{\sphinxbfcode{\_\_str\_\_}}{}{}
String representation.

\end{fulllineitems}

\index{\_\_sub\_\_() (TotalDepth.LIS.core.EngVal.EngVal method)}

\begin{fulllineitems}
\phantomsection\label{\detokenize{ref/LIS/core/EngVal:TotalDepth.LIS.core.EngVal.EngVal.__sub__}}\pysiglinewithargsret{\sphinxbfcode{\_\_sub\_\_}}{\emph{other}}{}
Overload self-other, returned result has the sum of self and other.
The units chosen are self’s.

\end{fulllineitems}

\index{\_\_truediv\_\_() (TotalDepth.LIS.core.EngVal.EngVal method)}

\begin{fulllineitems}
\phantomsection\label{\detokenize{ref/LIS/core/EngVal:TotalDepth.LIS.core.EngVal.EngVal.__truediv__}}\pysiglinewithargsret{\sphinxbfcode{\_\_truediv\_\_}}{\emph{other}}{}
Overload self/other. other must be a real number or an EngVal. If
the units of other are dimensionless then treat as a real number.
If the units are in the same category as me convert them and return a
dimensionless EngVal.

\end{fulllineitems}

\index{\_\_weakref\_\_ (TotalDepth.LIS.core.EngVal.EngVal attribute)}

\begin{fulllineitems}
\phantomsection\label{\detokenize{ref/LIS/core/EngVal:TotalDepth.LIS.core.EngVal.EngVal.__weakref__}}\pysigline{\sphinxbfcode{\_\_weakref\_\_}}
list of weak references to the object (if defined)

\end{fulllineitems}

\index{convert() (TotalDepth.LIS.core.EngVal.EngVal method)}

\begin{fulllineitems}
\phantomsection\label{\detokenize{ref/LIS/core/EngVal:TotalDepth.LIS.core.EngVal.EngVal.convert}}\pysiglinewithargsret{\sphinxbfcode{convert}}{\emph{theUnits}}{}
Convert my value to the supplied units in-place. May raise an ExceptionUnits.

\end{fulllineitems}

\index{dimensionless() (TotalDepth.LIS.core.EngVal.EngVal method)}

\begin{fulllineitems}
\phantomsection\label{\detokenize{ref/LIS/core/EngVal:TotalDepth.LIS.core.EngVal.EngVal.dimensionless}}\pysiglinewithargsret{\sphinxbfcode{dimensionless}}{}{}
Returns True if the measure is dimensionless.

\end{fulllineitems}

\index{getInUnits() (TotalDepth.LIS.core.EngVal.EngVal method)}

\begin{fulllineitems}
\phantomsection\label{\detokenize{ref/LIS/core/EngVal:TotalDepth.LIS.core.EngVal.EngVal.getInUnits}}\pysiglinewithargsret{\sphinxbfcode{getInUnits}}{\emph{theUnits}}{}
Returns my value to the supplied units. May raise an ExceptionUnits.

\end{fulllineitems}

\index{newEngValInOpticalUnits() (TotalDepth.LIS.core.EngVal.EngVal method)}

\begin{fulllineitems}
\phantomsection\label{\detokenize{ref/LIS/core/EngVal:TotalDepth.LIS.core.EngVal.EngVal.newEngValInOpticalUnits}}\pysiglinewithargsret{\sphinxbfcode{newEngValInOpticalUnits}}{}{}
Returns a new EngVal converting me to the ‘optical’ units if possible.
For example a value in b’.1IN” will be converted to b’FEET’.

\end{fulllineitems}

\index{newEngValInUnits() (TotalDepth.LIS.core.EngVal.EngVal method)}

\begin{fulllineitems}
\phantomsection\label{\detokenize{ref/LIS/core/EngVal:TotalDepth.LIS.core.EngVal.EngVal.newEngValInUnits}}\pysiglinewithargsret{\sphinxbfcode{newEngValInUnits}}{\emph{theUnits}}{}
Returns a new EngVal converting me to the supplied units.
May raise an ExceptionUnits.

\end{fulllineitems}

\index{pStr() (TotalDepth.LIS.core.EngVal.EngVal method)}

\begin{fulllineitems}
\phantomsection\label{\detokenize{ref/LIS/core/EngVal:TotalDepth.LIS.core.EngVal.EngVal.pStr}}\pysiglinewithargsret{\sphinxbfcode{pStr}}{}{}
Returns a ‘pretty’ ASCII string.

\end{fulllineitems}

\index{strFormat() (TotalDepth.LIS.core.EngVal.EngVal method)}

\begin{fulllineitems}
\phantomsection\label{\detokenize{ref/LIS/core/EngVal:TotalDepth.LIS.core.EngVal.EngVal.strFormat}}\pysiglinewithargsret{\sphinxbfcode{strFormat}}{\emph{theFormat}, \emph{incPrefix=True}}{}
Returns as as string with the value to the specified format (must be
capable of handling floating point values.

\end{fulllineitems}


\end{fulllineitems}

\index{EngValRc (class in TotalDepth.LIS.core.EngVal)}

\begin{fulllineitems}
\phantomsection\label{\detokenize{ref/LIS/core/EngVal:TotalDepth.LIS.core.EngVal.EngValRc}}\pysiglinewithargsret{\sphinxbfcode{class }\sphinxcode{TotalDepth.LIS.core.EngVal.}\sphinxbfcode{EngValRc}}{\emph{theVal}, \emph{theUom}, \emph{theRc=None}}{}
An engineering value with a integer Representation Code.
\index{\_\_str\_\_() (TotalDepth.LIS.core.EngVal.EngValRc method)}

\begin{fulllineitems}
\phantomsection\label{\detokenize{ref/LIS/core/EngVal:TotalDepth.LIS.core.EngVal.EngValRc.__str__}}\pysiglinewithargsret{\sphinxbfcode{\_\_str\_\_}}{}{}
String representation.

\end{fulllineitems}

\index{encode() (TotalDepth.LIS.core.EngVal.EngValRc method)}

\begin{fulllineitems}
\phantomsection\label{\detokenize{ref/LIS/core/EngVal:TotalDepth.LIS.core.EngVal.EngValRc.encode}}\pysiglinewithargsret{\sphinxbfcode{encode}}{}{}
Encode my value to my RepCode returning a bytes object. May raise an ExceptionEngVal.

\end{fulllineitems}


\end{fulllineitems}



\subsubsection{Mnem}
\label{\detokenize{ref/LIS/core/Mnem:mnem}}\label{\detokenize{ref/LIS/core/Mnem::doc}}\label{\detokenize{ref/LIS/core/Mnem:module-TotalDepth.LIS.core.Mnem}}\index{TotalDepth.LIS.core.Mnem (module)}
Represents a MNEM (Mnemonic) as bytes.
Created on May 26, 2011

@author: paulross
\index{ORDS\_WS (in module TotalDepth.LIS.core.Mnem)}

\begin{fulllineitems}
\phantomsection\label{\detokenize{ref/LIS/core/Mnem:TotalDepth.LIS.core.Mnem.ORDS_WS}}\pysigline{\sphinxcode{TotalDepth.LIS.core.Mnem.}\sphinxbfcode{ORDS\_WS}\sphinxbfcode{ = (32, 9, 10, 13, 11, 12)}}
A tuple of the ordinal values of whitespace characters

\end{fulllineitems}

\index{ORDS\_REPLACE (in module TotalDepth.LIS.core.Mnem)}

\begin{fulllineitems}
\phantomsection\label{\detokenize{ref/LIS/core/Mnem:TotalDepth.LIS.core.Mnem.ORDS_REPLACE}}\pysigline{\sphinxcode{TotalDepth.LIS.core.Mnem.}\sphinxbfcode{ORDS\_REPLACE}\sphinxbfcode{ = (0, 32, 9, 10, 13, 11, 12)}}
A tuple of the ordinal values of characters that can be replaced with PAD\_CHAR

\end{fulllineitems}

\index{LEN\_MNEM (in module TotalDepth.LIS.core.Mnem)}

\begin{fulllineitems}
\phantomsection\label{\detokenize{ref/LIS/core/Mnem:TotalDepth.LIS.core.Mnem.LEN_MNEM}}\pysigline{\sphinxcode{TotalDepth.LIS.core.Mnem.}\sphinxbfcode{LEN\_MNEM}\sphinxbfcode{ = 4}}
Standard LIS mnemonic length

\end{fulllineitems}

\index{Mnem (class in TotalDepth.LIS.core.Mnem)}

\begin{fulllineitems}
\phantomsection\label{\detokenize{ref/LIS/core/Mnem:TotalDepth.LIS.core.Mnem.Mnem}}\pysiglinewithargsret{\sphinxbfcode{class }\sphinxcode{TotalDepth.LIS.core.Mnem.}\sphinxbfcode{Mnem}}{\emph{m}, \emph{len\_mnem=4}}{}
Represents a four byte mnemonic where tailing nulls and spaces are not
considered significant. This preserves original length but replaces trailing
and whitespace characters with the PAD\_CHAR.

m must be a bytes object or an ‘ascii’ str that can be converted to a bytes object.

If len\_mnem is positive then m is truncated or padded as necessary to
achieve that length.

If len\_mnem is zero then all of m is considered significant, no padding
is performed.

If len\_mnem is less than zero then all characters of m are considered
significant and padding up to -1*len\_mnem characters of m is 
performed if m smaller than that.
\index{\_\_init\_\_() (TotalDepth.LIS.core.Mnem.Mnem method)}

\begin{fulllineitems}
\phantomsection\label{\detokenize{ref/LIS/core/Mnem:TotalDepth.LIS.core.Mnem.Mnem.__init__}}\pysiglinewithargsret{\sphinxbfcode{\_\_init\_\_}}{\emph{m}, \emph{len\_mnem=4}}{}
Constructor that prunes trailing nulls and spaces.

\end{fulllineitems}

\index{m (TotalDepth.LIS.core.Mnem.Mnem attribute)}

\begin{fulllineitems}
\phantomsection\label{\detokenize{ref/LIS/core/Mnem:TotalDepth.LIS.core.Mnem.Mnem.m}}\pysigline{\sphinxbfcode{m}}
The raw bytes of the mnemonic.

\end{fulllineitems}

\index{\_\_str\_\_() (TotalDepth.LIS.core.Mnem.Mnem method)}

\begin{fulllineitems}
\phantomsection\label{\detokenize{ref/LIS/core/Mnem:TotalDepth.LIS.core.Mnem.Mnem.__str__}}\pysiglinewithargsret{\sphinxbfcode{\_\_str\_\_}}{}{}
String representation.

\end{fulllineitems}

\index{pStr() (TotalDepth.LIS.core.Mnem.Mnem method)}

\begin{fulllineitems}
\phantomsection\label{\detokenize{ref/LIS/core/Mnem:TotalDepth.LIS.core.Mnem.Mnem.pStr}}\pysiglinewithargsret{\sphinxbfcode{pStr}}{\emph{strip=False}}{}
Returns a ‘pretty’ ascii string. If strip then trailing padding is removed.

\end{fulllineitems}

\index{\_\_repr\_\_() (TotalDepth.LIS.core.Mnem.Mnem method)}

\begin{fulllineitems}
\phantomsection\label{\detokenize{ref/LIS/core/Mnem:TotalDepth.LIS.core.Mnem.Mnem.__repr__}}\pysiglinewithargsret{\sphinxbfcode{\_\_repr\_\_}}{}{}
repr() representation.

\end{fulllineitems}

\index{\_\_hash\_\_() (TotalDepth.LIS.core.Mnem.Mnem method)}

\begin{fulllineitems}
\phantomsection\label{\detokenize{ref/LIS/core/Mnem:TotalDepth.LIS.core.Mnem.Mnem.__hash__}}\pysiglinewithargsret{\sphinxbfcode{\_\_hash\_\_}}{}{}
Hashing, this makes bytes() and Mnem() objects interchangeable.

\end{fulllineitems}

\index{\_\_weakref\_\_ (TotalDepth.LIS.core.Mnem.Mnem attribute)}

\begin{fulllineitems}
\phantomsection\label{\detokenize{ref/LIS/core/Mnem:TotalDepth.LIS.core.Mnem.Mnem.__weakref__}}\pysigline{\sphinxbfcode{\_\_weakref\_\_}}
list of weak references to the object (if defined)

\end{fulllineitems}

\index{\_\_eq\_\_() (TotalDepth.LIS.core.Mnem.Mnem method)}

\begin{fulllineitems}
\phantomsection\label{\detokenize{ref/LIS/core/Mnem:TotalDepth.LIS.core.Mnem.Mnem.__eq__}}\pysiglinewithargsret{\sphinxbfcode{\_\_eq\_\_}}{\emph{other}}{}
True if self == other False otherwise.
If other is not a Mnem it is coerced into one before the comparison is made..

\end{fulllineitems}

\index{\_\_ne\_\_() (TotalDepth.LIS.core.Mnem.Mnem method)}

\begin{fulllineitems}
\phantomsection\label{\detokenize{ref/LIS/core/Mnem:TotalDepth.LIS.core.Mnem.Mnem.__ne__}}\pysiglinewithargsret{\sphinxbfcode{\_\_ne\_\_}}{\emph{other}}{}
True if self != other False otherwise.
If other is not a Mnem it is coerced into one before the comparison is made..

\end{fulllineitems}

\index{\_\_lt\_\_() (TotalDepth.LIS.core.Mnem.Mnem method)}

\begin{fulllineitems}
\phantomsection\label{\detokenize{ref/LIS/core/Mnem:TotalDepth.LIS.core.Mnem.Mnem.__lt__}}\pysiglinewithargsret{\sphinxbfcode{\_\_lt\_\_}}{\emph{other}}{}
True if self \textless{} other False otherwise.
If other is not a Mnem it is coerced into one before the comparison is made..

\end{fulllineitems}

\index{\_\_iter\_\_() (TotalDepth.LIS.core.Mnem.Mnem method)}

\begin{fulllineitems}
\phantomsection\label{\detokenize{ref/LIS/core/Mnem:TotalDepth.LIS.core.Mnem.Mnem.__iter__}}\pysiglinewithargsret{\sphinxbfcode{\_\_iter\_\_}}{}{}
Byte by byte iteration.

\end{fulllineitems}


\end{fulllineitems}



\subsubsection{Raw Stream Handler}
\label{\detokenize{ref/LIS/core/RawStream:raw-stream-handler}}\label{\detokenize{ref/LIS/core/RawStream::doc}}\phantomsection\label{\detokenize{ref/LIS/core/RawStream:module-TotalDepth.LIS.core.RawStream}}\index{TotalDepth.LIS.core.RawStream (module)}
The RawStream handler provides low-level stream I/O functionality.
\index{ExceptionRawStream}

\begin{fulllineitems}
\phantomsection\label{\detokenize{ref/LIS/core/RawStream:TotalDepth.LIS.core.RawStream.ExceptionRawStream}}\pysigline{\sphinxbfcode{exception }\sphinxcode{TotalDepth.LIS.core.RawStream.}\sphinxbfcode{ExceptionRawStream}}
Specialisation of exception for RawStream.

\end{fulllineitems}

\index{ExceptionRawStreamEOF}

\begin{fulllineitems}
\phantomsection\label{\detokenize{ref/LIS/core/RawStream:TotalDepth.LIS.core.RawStream.ExceptionRawStreamEOF}}\pysigline{\sphinxbfcode{exception }\sphinxcode{TotalDepth.LIS.core.RawStream.}\sphinxbfcode{ExceptionRawStreamEOF}}
RawStream premature EOF.

\end{fulllineitems}

\index{RawStream (class in TotalDepth.LIS.core.RawStream)}

\begin{fulllineitems}
\phantomsection\label{\detokenize{ref/LIS/core/RawStream:TotalDepth.LIS.core.RawStream.RawStream}}\pysiglinewithargsret{\sphinxbfcode{class }\sphinxcode{TotalDepth.LIS.core.RawStream.}\sphinxbfcode{RawStream}}{\emph{f}, \emph{mode='rb'}, \emph{fileId=None}}{}
Class that creates a I/O stream from a file path or file-like object
and provides various low level functionality on it such as unpacking.

f - A file like object or string, if the latter it assumed to be a path.

mode - The file mode, defaults to binary read.

fileId - If f is a string is this is present then this is used as the
file name. If f is not a string then f.name is used with
fileId as a fallback.
\index{\_\_init\_\_() (TotalDepth.LIS.core.RawStream.RawStream method)}

\begin{fulllineitems}
\phantomsection\label{\detokenize{ref/LIS/core/RawStream:TotalDepth.LIS.core.RawStream.RawStream.__init__}}\pysiglinewithargsret{\sphinxbfcode{\_\_init\_\_}}{\emph{f}, \emph{mode='rb'}, \emph{fileId=None}}{}
Construct with:
f - A file like object or string, if the latter it assumed to be a path.
mode - The file mode, defaults to binary read.
fileId - If f is a string is this is present then this is used as the
file name. If f is not a string then f.name is used with
fileId as a fallback.

\end{fulllineitems}

\index{\_\_enter\_\_() (TotalDepth.LIS.core.RawStream.RawStream method)}

\begin{fulllineitems}
\phantomsection\label{\detokenize{ref/LIS/core/RawStream:TotalDepth.LIS.core.RawStream.RawStream.__enter__}}\pysiglinewithargsret{\sphinxbfcode{\_\_enter\_\_}}{}{}
Context Manager support.

\end{fulllineitems}

\index{\_\_exit\_\_() (TotalDepth.LIS.core.RawStream.RawStream method)}

\begin{fulllineitems}
\phantomsection\label{\detokenize{ref/LIS/core/RawStream:TotalDepth.LIS.core.RawStream.RawStream.__exit__}}\pysiglinewithargsret{\sphinxbfcode{\_\_exit\_\_}}{\emph{exc\_type}, \emph{exc\_value}, \emph{traceback}}{}
Context manager finalisation, this closes the underlying stream.

\end{fulllineitems}

\index{stream (TotalDepth.LIS.core.RawStream.RawStream attribute)}

\begin{fulllineitems}
\phantomsection\label{\detokenize{ref/LIS/core/RawStream:TotalDepth.LIS.core.RawStream.RawStream.stream}}\pysigline{\sphinxbfcode{stream}}
Exposes the underlying stream.

\end{fulllineitems}

\index{tell() (TotalDepth.LIS.core.RawStream.RawStream method)}

\begin{fulllineitems}
\phantomsection\label{\detokenize{ref/LIS/core/RawStream:TotalDepth.LIS.core.RawStream.RawStream.tell}}\pysiglinewithargsret{\sphinxbfcode{tell}}{}{}
Return the file’s current position, like stdio’s ftell.

\end{fulllineitems}

\index{seek() (TotalDepth.LIS.core.RawStream.RawStream method)}

\begin{fulllineitems}
\phantomsection\label{\detokenize{ref/LIS/core/RawStream:TotalDepth.LIS.core.RawStream.RawStream.seek}}\pysiglinewithargsret{\sphinxbfcode{seek}}{\emph{offset}, \emph{whence=0}}{}
Set the file’s current position, like stdio’s fseek. The whence
argument is optional and defaults to os.SEEK\_SET or 0 (absolute file
positioning); other values are os.SEEK\_CUR or 1 (seek relative to the
current position) and os.SEEK\_END or 2 (seek relative to the file’s end).
There is no return value.
Not all file objects are seekable.

\end{fulllineitems}

\index{read() (TotalDepth.LIS.core.RawStream.RawStream method)}

\begin{fulllineitems}
\phantomsection\label{\detokenize{ref/LIS/core/RawStream:TotalDepth.LIS.core.RawStream.RawStream.read}}\pysiglinewithargsret{\sphinxbfcode{read}}{\emph{theLen}}{}
Reads and returns theLen bytes.

\end{fulllineitems}

\index{\_\_weakref\_\_ (TotalDepth.LIS.core.RawStream.RawStream attribute)}

\begin{fulllineitems}
\phantomsection\label{\detokenize{ref/LIS/core/RawStream:TotalDepth.LIS.core.RawStream.RawStream.__weakref__}}\pysigline{\sphinxbfcode{\_\_weakref\_\_}}
list of weak references to the object (if defined)

\end{fulllineitems}

\index{write() (TotalDepth.LIS.core.RawStream.RawStream method)}

\begin{fulllineitems}
\phantomsection\label{\detokenize{ref/LIS/core/RawStream:TotalDepth.LIS.core.RawStream.RawStream.write}}\pysiglinewithargsret{\sphinxbfcode{write}}{\emph{theB}}{}
Writes theB bytes.

\end{fulllineitems}

\index{close() (TotalDepth.LIS.core.RawStream.RawStream method)}

\begin{fulllineitems}
\phantomsection\label{\detokenize{ref/LIS/core/RawStream:TotalDepth.LIS.core.RawStream.RawStream.close}}\pysiglinewithargsret{\sphinxbfcode{close}}{}{}
Closes the underlying stream.

\end{fulllineitems}

\index{readAndUnpack() (TotalDepth.LIS.core.RawStream.RawStream method)}

\begin{fulllineitems}
\phantomsection\label{\detokenize{ref/LIS/core/RawStream:TotalDepth.LIS.core.RawStream.RawStream.readAndUnpack}}\pysiglinewithargsret{\sphinxbfcode{readAndUnpack}}{\emph{theStruct}}{}
Reads from the stream and unpacks binary data according to the
struct module format. This returns a tuple.

theStruct - A formated instance of struct.Struct().

\end{fulllineitems}

\index{packAndWrite() (TotalDepth.LIS.core.RawStream.RawStream method)}

\begin{fulllineitems}
\phantomsection\label{\detokenize{ref/LIS/core/RawStream:TotalDepth.LIS.core.RawStream.RawStream.packAndWrite}}\pysiglinewithargsret{\sphinxbfcode{packAndWrite}}{\emph{theStruct}, \emph{*args}}{}
Packs binary data from args and writes it to the stream.

theStruct - A formated instance of struct.Struct().

args - The data to write.

\end{fulllineitems}


\end{fulllineitems}



\subsubsection{LIS File}
\label{\detokenize{ref/LIS/core/File:lis-file}}\label{\detokenize{ref/LIS/core/File::doc}}\phantomsection\label{\detokenize{ref/LIS/core/File:module-TotalDepth.LIS.core.File}}\index{TotalDepth.LIS.core.File (module)}
Reads or writes LIS data to a physical file.

Created on 14 Nov 2010
\index{ExceptionFile}

\begin{fulllineitems}
\phantomsection\label{\detokenize{ref/LIS/core/File:TotalDepth.LIS.core.File.ExceptionFile}}\pysigline{\sphinxbfcode{exception }\sphinxcode{TotalDepth.LIS.core.File.}\sphinxbfcode{ExceptionFile}}
Specialisation of exception for Physical files.

\end{fulllineitems}

\index{ExceptionFileRead}

\begin{fulllineitems}
\phantomsection\label{\detokenize{ref/LIS/core/File:TotalDepth.LIS.core.File.ExceptionFileRead}}\pysigline{\sphinxbfcode{exception }\sphinxcode{TotalDepth.LIS.core.File.}\sphinxbfcode{ExceptionFileRead}}
Specialisation of exception for reading Physical files.

\end{fulllineitems}

\index{ExceptionFileWrite}

\begin{fulllineitems}
\phantomsection\label{\detokenize{ref/LIS/core/File:TotalDepth.LIS.core.File.ExceptionFileWrite}}\pysigline{\sphinxbfcode{exception }\sphinxcode{TotalDepth.LIS.core.File.}\sphinxbfcode{ExceptionFileWrite}}
Specialisation of exception for writing Physical files.

\end{fulllineitems}

\index{FileBase (class in TotalDepth.LIS.core.File)}

\begin{fulllineitems}
\phantomsection\label{\detokenize{ref/LIS/core/File:TotalDepth.LIS.core.File.FileBase}}\pysiglinewithargsret{\sphinxbfcode{class }\sphinxcode{TotalDepth.LIS.core.File.}\sphinxbfcode{FileBase}}{\emph{theFile}, \emph{theFileId}, \emph{mode}, \emph{keepGoing}}{}
LIS file handler. This handles Physical Records (and TIF records).

\end{fulllineitems}

\index{FileRead (class in TotalDepth.LIS.core.File)}

\begin{fulllineitems}
\phantomsection\label{\detokenize{ref/LIS/core/File:TotalDepth.LIS.core.File.FileRead}}\pysiglinewithargsret{\sphinxbfcode{class }\sphinxcode{TotalDepth.LIS.core.File.}\sphinxbfcode{FileRead}}{\emph{theFile}, \emph{theFileId=None}, \emph{keepGoing=False}}{}
LIS file reader, this offers the caller a number of incremental
read operations. This handles Physical Records (and TIF records).

theFile - A file like object or string, if the latter it assumed to be a path.

theFileId - File identifier, this could be a path for example. If None the RawStream will try and cope with it.

keepGoing - If True we do our best to keep going.
\index{readLrBytes() (TotalDepth.LIS.core.File.FileRead method)}

\begin{fulllineitems}
\phantomsection\label{\detokenize{ref/LIS/core/File:TotalDepth.LIS.core.File.FileRead.readLrBytes}}\pysiglinewithargsret{\sphinxbfcode{readLrBytes}}{\emph{theLen=-1}}{}
Reads theLen LogicalData bytes and returns it or None if nothing
left to read for this logical record.
If theLen is -1 all the remaining Logical data is read.

\end{fulllineitems}

\index{skipLrBytes() (TotalDepth.LIS.core.File.FileRead method)}

\begin{fulllineitems}
\phantomsection\label{\detokenize{ref/LIS/core/File:TotalDepth.LIS.core.File.FileRead.skipLrBytes}}\pysiglinewithargsret{\sphinxbfcode{skipLrBytes}}{\emph{theLen=-1}}{}
Skips logical data and returns a count of skipped bytes.
If theLen is -1 all the remaining Logical data is read.

\end{fulllineitems}

\index{seekCurrentLrStart() (TotalDepth.LIS.core.File.FileRead method)}

\begin{fulllineitems}
\phantomsection\label{\detokenize{ref/LIS/core/File:TotalDepth.LIS.core.File.FileRead.seekCurrentLrStart}}\pysiglinewithargsret{\sphinxbfcode{seekCurrentLrStart}}{}{}
Setting the file position directly to the beginning of
a PRH or TIF marker (if present) for the current Logical Record.

\end{fulllineitems}

\index{skipToNextLr() (TotalDepth.LIS.core.File.FileRead method)}

\begin{fulllineitems}
\phantomsection\label{\detokenize{ref/LIS/core/File:TotalDepth.LIS.core.File.FileRead.skipToNextLr}}\pysiglinewithargsret{\sphinxbfcode{skipToNextLr}}{}{}
Skips the rest of the current Logical Data and positions the file at
the start of the next Logical Record.

\end{fulllineitems}

\index{tellLr() (TotalDepth.LIS.core.File.FileRead method)}

\begin{fulllineitems}
\phantomsection\label{\detokenize{ref/LIS/core/File:TotalDepth.LIS.core.File.FileRead.tellLr}}\pysiglinewithargsret{\sphinxbfcode{tellLr}}{}{}
Returns the absolute file position of the start current Logical
record. This value can be safely used in seekLr.

\end{fulllineitems}

\index{tell() (TotalDepth.LIS.core.File.FileRead method)}

\begin{fulllineitems}
\phantomsection\label{\detokenize{ref/LIS/core/File:TotalDepth.LIS.core.File.FileRead.tell}}\pysiglinewithargsret{\sphinxbfcode{tell}}{}{}
Returns the absolute position of the file.

\end{fulllineitems}

\index{ldIndex() (TotalDepth.LIS.core.File.FileRead method)}

\begin{fulllineitems}
\phantomsection\label{\detokenize{ref/LIS/core/File:TotalDepth.LIS.core.File.FileRead.ldIndex}}\pysiglinewithargsret{\sphinxbfcode{ldIndex}}{}{}
Returns the index position in the current logical data.

\end{fulllineitems}

\index{seekLr() (TotalDepth.LIS.core.File.FileRead method)}

\begin{fulllineitems}
\phantomsection\label{\detokenize{ref/LIS/core/File:TotalDepth.LIS.core.File.FileRead.seekLr}}\pysiglinewithargsret{\sphinxbfcode{seekLr}}{\emph{offset}}{}
External setting of file position directly to the beginning of
a PRH or TIF marker (if present). The caller is fully responsible
for getting this right!

\end{fulllineitems}

\index{hasLd() (TotalDepth.LIS.core.File.FileRead method)}

\begin{fulllineitems}
\phantomsection\label{\detokenize{ref/LIS/core/File:TotalDepth.LIS.core.File.FileRead.hasLd}}\pysiglinewithargsret{\sphinxbfcode{hasLd}}{}{}
Returns True if there is logical data to be read, False otherwise.
NOTE: This will return False on file initialisation and only return
True once the Physical Record Header (i.e. one or more logical bytes)
has been read.

\end{fulllineitems}

\index{rewind() (TotalDepth.LIS.core.File.FileRead method)}

\begin{fulllineitems}
\phantomsection\label{\detokenize{ref/LIS/core/File:TotalDepth.LIS.core.File.FileRead.rewind}}\pysiglinewithargsret{\sphinxbfcode{rewind}}{}{}
Sets the file position to the beginning of file.

\end{fulllineitems}

\index{isEOF (TotalDepth.LIS.core.File.FileRead attribute)}

\begin{fulllineitems}
\phantomsection\label{\detokenize{ref/LIS/core/File:TotalDepth.LIS.core.File.FileRead.isEOF}}\pysigline{\sphinxbfcode{isEOF}}
True if at EOF.

\end{fulllineitems}

\index{unpack() (TotalDepth.LIS.core.File.FileRead method)}

\begin{fulllineitems}
\phantomsection\label{\detokenize{ref/LIS/core/File:TotalDepth.LIS.core.File.FileRead.unpack}}\pysiglinewithargsret{\sphinxbfcode{unpack}}{\emph{theStruct}}{}
Unpack some logical bytes using the supplied format.
format \textasciitilde{} a struct.Struct object.
Returns a tuple of the number of objects specified by the format or None.

\end{fulllineitems}


\end{fulllineitems}

\index{FileWrite (class in TotalDepth.LIS.core.File)}

\begin{fulllineitems}
\phantomsection\label{\detokenize{ref/LIS/core/File:TotalDepth.LIS.core.File.FileWrite}}\pysiglinewithargsret{\sphinxbfcode{class }\sphinxcode{TotalDepth.LIS.core.File.}\sphinxbfcode{FileWrite}}{\emph{theFile}, \emph{theFileId=None}, \emph{keepGoing=False}, \emph{hasTif=False}, \emph{thePrLen=65535}, \emph{thePrt=\textless{}TotalDepth.LIS.core.PhysRec.PhysRecTail object\textgreater{}}}{}
LIS file writer. This handles Physical Records (and TIF records).

theFile - A file like object or string, if the latter it assumed to be a path.

theFileId - File identifier, this could be a path for example. If None the RawStream will try and cope with it.

keepGoing - If True we do our best to keep going.

hasTif - Insert TIF markers or not.

thePrLen - Max Physical Record length, defaults to the maximum possible length.

thePrt - Physical Records Trailer settings (defaults to PhysRec.PhysRecTail()).
\index{write() (TotalDepth.LIS.core.File.FileWrite method)}

\begin{fulllineitems}
\phantomsection\label{\detokenize{ref/LIS/core/File:TotalDepth.LIS.core.File.FileWrite.write}}\pysiglinewithargsret{\sphinxbfcode{write}}{\emph{theLr}}{}
Writes the Logical Record to the file. Returns the tell() of the
start of the LR.

\end{fulllineitems}

\index{close() (TotalDepth.LIS.core.File.FileWrite method)}

\begin{fulllineitems}
\phantomsection\label{\detokenize{ref/LIS/core/File:TotalDepth.LIS.core.File.FileWrite.close}}\pysiglinewithargsret{\sphinxbfcode{close}}{}{}
Closes the file.

\end{fulllineitems}


\end{fulllineitems}



\subsubsection{TIF Marker Handling}
\label{\detokenize{ref/LIS/core/TifMarker:tif-marker-handling}}\label{\detokenize{ref/LIS/core/TifMarker::doc}}\phantomsection\label{\detokenize{ref/LIS/core/TifMarker:module-TotalDepth.LIS.core.TifMarker}}\index{TotalDepth.LIS.core.TifMarker (module)}

\paragraph{TIF Markers}
\label{\detokenize{ref/LIS/core/TifMarker:tif-markers}}
These are 3x32bit big-endian integers at the beginning of each Physical Record.


\subparagraph{Word{[}0{]}}
\label{\detokenize{ref/LIS/core/TifMarker:word-0}}
This is the TIF set type, 0 for a normal TIF set. 1 for an EOF set.


\subparagraph{Word{[}1{]}}
\label{\detokenize{ref/LIS/core/TifMarker:word-1}}
This is the physical file location of the start of the previous set.


\subparagraph{Word{[}2{]}}
\label{\detokenize{ref/LIS/core/TifMarker:word-2}}
This is the physical file location of the start of the next set.

A dump looks like this:

\begin{sphinxVerbatim}[commandchars=\\\{\}]
           \PYG{l+m+mi}{0}  \PYG{l+m+mi}{1}  \PYG{l+m+mi}{2}  \PYG{l+m+mi}{3}  \PYG{l+m+mi}{4}  \PYG{l+m+mi}{5}  \PYG{l+m+mi}{6}  \PYG{l+m+mi}{7}  \PYG{l+m+mi}{8}  \PYG{l+m+mi}{9}  \PYG{n}{A}  \PYG{n}{B}  \PYG{n}{C}  \PYG{n}{D}  \PYG{n}{E}  \PYG{n}{F}
\PYG{l+m+mi}{00000000}\PYG{p}{:} \PYG{l+m+mi}{00} \PYG{l+m+mi}{00} \PYG{l+m+mi}{00} \PYG{l+m+mi}{00} \PYG{l+m+mi}{00} \PYG{l+m+mi}{00} \PYG{l+m+mi}{00} \PYG{l+m+mi}{00} \PYG{l+m+mi}{4}\PYG{n}{A} \PYG{l+m+mi}{00} \PYG{l+m+mi}{00} \PYG{l+m+mi}{00} \PYG{l+m+mi}{00} \PYG{l+m+mi}{3}\PYG{n}{E} \PYG{l+m+mi}{00} \PYG{l+m+mi}{00}
\PYG{l+m+mi}{00000010}\PYG{p}{:} \PYG{l+m+mi}{80} \PYG{l+m+mi}{00} \PYG{l+m+mi}{32} \PYG{l+m+mi}{30} \PYG{l+m+mi}{30} \PYG{l+m+mi}{30} \PYG{l+m+mi}{39} \PYG{l+m+mi}{39} \PYG{l+m+mi}{2}\PYG{n}{E} \PYG{l+m+mi}{44} \PYG{l+m+mi}{41} \PYG{l+m+mi}{54} \PYG{l+m+mi}{20} \PYG{l+m+mi}{20} \PYG{l+m+mi}{20} \PYG{l+m+mi}{20}
\PYG{l+m+mi}{00000020}\PYG{p}{:} \PYG{l+m+mi}{20} \PYG{l+m+mi}{20} \PYG{l+m+mi}{20} \PYG{l+m+mi}{20} \PYG{l+m+mi}{20} \PYG{l+m+mi}{20} \PYG{l+m+mi}{20} \PYG{l+m+mi}{20} \PYG{l+m+mi}{20} \PYG{l+m+mi}{20} \PYG{l+m+mi}{20} \PYG{l+m+mi}{20} \PYG{l+m+mi}{39} \PYG{l+m+mi}{39} \PYG{l+m+mi}{2}\PYG{n}{F} \PYG{l+m+mi}{30}
\PYG{l+m+mi}{00000030}\PYG{p}{:} \PYG{l+m+mi}{34} \PYG{l+m+mi}{2}\PYG{n}{F} \PYG{l+m+mi}{32} \PYG{l+m+mi}{39} \PYG{l+m+mi}{20} \PYG{l+m+mi}{20} \PYG{l+m+mi}{31} \PYG{l+m+mi}{30} \PYG{l+m+mi}{32} \PYG{l+m+mi}{34} \PYG{l+m+mi}{20} \PYG{l+m+mi}{20} \PYG{l+m+mi}{20} \PYG{l+m+mi}{20} \PYG{l+m+mi}{20} \PYG{l+m+mi}{20}
\PYG{l+m+mi}{00000040}\PYG{p}{:} \PYG{l+m+mi}{20} \PYG{l+m+mi}{20} \PYG{l+m+mi}{20} \PYG{l+m+mi}{20} \PYG{l+m+mi}{20} \PYG{l+m+mi}{20} \PYG{l+m+mi}{2}\PYG{n}{E} \PYG{l+m+mi}{20} \PYG{l+m+mi}{20} \PYG{l+m+mi}{20} \PYG{l+m+mi}{00} \PYG{l+m+mi}{00} \PYG{l+m+mi}{00} \PYG{l+m+mi}{00} \PYG{l+m+mi}{00} \PYG{l+m+mi}{00}
\PYG{l+m+mi}{00000050}\PYG{p}{:} \PYG{l+m+mi}{00} \PYG{l+m+mi}{00} \PYG{l+m+mi}{56} \PYG{l+m+mi}{04} \PYG{l+m+mi}{00} \PYG{l+m+mi}{00} \PYG{l+m+mi}{04} \PYG{l+m+mi}{00} \PYG{l+m+mi}{00} \PYG{l+m+mi}{01} \PYG{l+m+mi}{40} \PYG{l+m+mi}{00} \PYG{l+m+mi}{01} \PYG{l+m+mi}{01} \PYG{l+m+mi}{42} \PYG{l+m+mi}{00}
\PYG{l+m+mi}{00000060}\PYG{p}{:} \PYG{l+m+mi}{02} \PYG{l+m+mi}{01} \PYG{l+m+mi}{42} \PYG{l+m+mi}{00} \PYG{l+m+mi}{03} \PYG{l+m+mi}{04} \PYG{l+m+mi}{49} \PYG{l+m+mi}{00} \PYG{l+m+mi}{00} \PYG{l+m+mi}{02} \PYG{l+m+mi}{34} \PYG{l+m+mi}{04} \PYG{l+m+mi}{01} \PYG{l+m+mi}{42} \PYG{l+m+mi}{01} \PYG{l+m+mi}{05}
\PYG{o}{.}\PYG{o}{.}\PYG{o}{.}
\PYG{l+m+mi}{00000450}\PYG{p}{:} \PYG{l+m+mi}{20} \PYG{l+m+mi}{20} \PYG{l+m+mi}{44} \PYG{l+m+mi}{45} \PYG{l+m+mi}{47} \PYG{l+m+mi}{20} \PYG{l+m+mi}{00} \PYG{l+m+mi}{00} \PYG{l+m+mi}{00} \PYG{l+m+mi}{00} \PYG{l+m+mi}{4}\PYG{n}{A} \PYG{l+m+mi}{00} \PYG{l+m+mi}{00} \PYG{l+m+mi}{00} \PYG{l+m+mi}{62} \PYG{l+m+mi}{08}
\PYG{l+m+mi}{00000460}\PYG{p}{:} \PYG{l+m+mi}{00} \PYG{l+m+mi}{00} \PYG{l+m+mi}{04} \PYG{l+m+mi}{00} \PYG{l+m+mi}{00} \PYG{l+m+mi}{03} \PYG{l+m+mi}{00} \PYG{l+m+mi}{00} \PYG{l+m+mi}{00} \PYG{l+m+mi}{00} \PYG{l+m+mi}{00} \PYG{l+m+mi}{00} \PYG{l+m+mi}{00} \PYG{l+m+mi}{04} \PYG{l+m+mi}{20} \PYG{l+m+mi}{20}
\PYG{l+m+mi}{00000470}\PYG{p}{:} \PYG{l+m+mi}{00} \PYG{l+m+mi}{01} \PYG{l+m+mi}{44} \PYG{l+m+mi}{20} \PYG{l+m+mi}{20} \PYG{l+m+mi}{20} \PYG{l+m+mi}{20} \PYG{l+m+mi}{20} \PYG{l+m+mi}{31} \PYG{l+m+mi}{30} \PYG{l+m+mi}{41} \PYG{l+m+mi}{20} \PYG{l+m+mi}{20} \PYG{l+m+mi}{20} \PYG{l+m+mi}{20} \PYG{l+m+mi}{20}
\PYG{o}{.}\PYG{o}{.}\PYG{o}{.}
\PYG{l+m+mi}{00000860}\PYG{p}{:} \PYG{l+m+mi}{46} \PYG{l+m+mi}{4}\PYG{n}{E} \PYG{l+m+mi}{00} \PYG{l+m+mi}{00} \PYG{l+m+mi}{00} \PYG{l+m+mi}{00} \PYG{l+m+mi}{56} \PYG{l+m+mi}{04} \PYG{l+m+mi}{00} \PYG{l+m+mi}{00} \PYG{l+m+mi}{6}\PYG{n}{E} \PYG{l+m+mi}{0}\PYG{n}{C} \PYG{l+m+mi}{00} \PYG{l+m+mi}{00} \PYG{l+m+mi}{04} \PYG{l+m+mi}{00}
\PYG{l+m+mi}{00000870}\PYG{p}{:} \PYG{l+m+mi}{00} \PYG{l+m+mi}{03} \PYG{l+m+mi}{4}\PYG{n}{F} \PYG{l+m+mi}{52} \PYG{l+m+mi}{20} \PYG{l+m+mi}{20} \PYG{l+m+mi}{20} \PYG{l+m+mi}{20} \PYG{l+m+mi}{20} \PYG{l+m+mi}{20} \PYG{l+m+mi}{20} \PYG{l+m+mi}{20} \PYG{l+m+mi}{20} \PYG{l+m+mi}{20} \PYG{l+m+mi}{20} \PYG{l+m+mi}{20}
\PYG{o}{.}\PYG{o}{.}\PYG{o}{.}
\PYG{o}{.}\PYG{o}{.}\PYG{o}{.}
\PYG{o}{.}\PYG{o}{.}\PYG{o}{.}
\PYG{l+m+mf}{0016E770}\PYG{p}{:} \PYG{l+m+mi}{18} \PYG{l+m+mi}{00} \PYG{n}{BA} \PYG{l+m+mi}{83} \PYG{l+m+mi}{18} \PYG{l+m+mi}{00} \PYG{l+m+mi}{00} \PYG{l+m+mi}{00} \PYG{l+m+mi}{00} \PYG{l+m+mi}{00} \PYG{l+m+mi}{30} \PYG{n}{E5} \PYG{l+m+mi}{16} \PYG{l+m+mi}{00} \PYG{n}{C0} \PYG{n}{E7}
\PYG{l+m+mf}{0016E780}\PYG{p}{:} \PYG{l+m+mi}{16} \PYG{l+m+mi}{00} \PYG{l+m+mi}{00} \PYG{l+m+mi}{3}\PYG{n}{E} \PYG{l+m+mi}{00} \PYG{l+m+mi}{00} \PYG{l+m+mi}{81} \PYG{l+m+mi}{00} \PYG{l+m+mi}{32} \PYG{l+m+mi}{30} \PYG{l+m+mi}{30} \PYG{l+m+mi}{30} \PYG{l+m+mi}{39} \PYG{l+m+mi}{39} \PYG{l+m+mi}{2}\PYG{n}{E} \PYG{l+m+mi}{44}
\PYG{o}{.}\PYG{o}{.}\PYG{o}{.}
\PYG{l+m+mf}{0016E7}\PYG{n}{B0}\PYG{p}{:} \PYG{l+m+mi}{20} \PYG{l+m+mi}{20} \PYG{l+m+mi}{20} \PYG{l+m+mi}{20} \PYG{l+m+mi}{20} \PYG{l+m+mi}{20} \PYG{l+m+mi}{20} \PYG{l+m+mi}{20} \PYG{l+m+mi}{20} \PYG{l+m+mi}{20} \PYG{l+m+mi}{20} \PYG{l+m+mi}{20} \PYG{l+m+mi}{2}\PYG{n}{E} \PYG{l+m+mi}{20} \PYG{l+m+mi}{20} \PYG{l+m+mi}{20}
\PYG{l+m+mf}{0016E7}\PYG{n}{C0}\PYG{p}{:} \PYG{l+m+mi}{01} \PYG{l+m+mi}{00} \PYG{l+m+mi}{00} \PYG{l+m+mi}{00} \PYG{l+m+mi}{76} \PYG{n}{E7} \PYG{l+m+mi}{16} \PYG{l+m+mi}{00} \PYG{n}{CC} \PYG{n}{E7} \PYG{l+m+mi}{16} \PYG{l+m+mi}{00} \PYG{l+m+mi}{01} \PYG{l+m+mi}{00} \PYG{l+m+mi}{00} \PYG{l+m+mi}{00}
\PYG{l+m+mf}{0016E7}\PYG{n}{D0}\PYG{p}{:} \PYG{n}{C0} \PYG{n}{E7} \PYG{l+m+mi}{16} \PYG{l+m+mi}{00} \PYG{n}{D8} \PYG{n}{E7} \PYG{l+m+mi}{16} \PYG{l+m+mi}{00}                        
\end{sphinxVerbatim}

Or:

\begin{sphinxVerbatim}[commandchars=\\\{\}]
\PYG{n}{tell}\PYG{p}{(}\PYG{p}{)}       \PYG{n}{TIF} \PYG{n+nb}{type}           \PYG{n}{TIF} \PYG{n}{back}           \PYG{n}{TIF} \PYG{n+nb}{next}
\PYG{l+m+mi}{00000000}\PYG{p}{:}    \PYG{l+m+mi}{00} \PYG{l+m+mi}{00} \PYG{l+m+mi}{00} \PYG{l+m+mi}{00}        \PYG{l+m+mi}{00} \PYG{l+m+mi}{00} \PYG{l+m+mi}{00} \PYG{l+m+mi}{00}        \PYG{l+m+mi}{00} \PYG{l+m+mi}{00} \PYG{l+m+mi}{00} \PYG{l+m+mi}{4}\PYG{n}{A}
\PYG{l+m+mi}{0000004}\PYG{n}{A}\PYG{p}{:}    \PYG{l+m+mi}{00} \PYG{l+m+mi}{00} \PYG{l+m+mi}{00} \PYG{l+m+mi}{00}        \PYG{l+m+mi}{00} \PYG{l+m+mi}{00} \PYG{l+m+mi}{00} \PYG{l+m+mi}{00}        \PYG{l+m+mi}{00} \PYG{l+m+mi}{00} \PYG{l+m+mi}{04} \PYG{l+m+mi}{56}
\PYG{l+m+mi}{00000456}\PYG{p}{:}    \PYG{l+m+mi}{00} \PYG{l+m+mi}{00} \PYG{l+m+mi}{00} \PYG{l+m+mi}{00}        \PYG{l+m+mi}{00} \PYG{l+m+mi}{00} \PYG{l+m+mi}{00} \PYG{l+m+mi}{4}\PYG{n}{A}        \PYG{l+m+mi}{00} \PYG{l+m+mi}{00} \PYG{l+m+mi}{08} \PYG{l+m+mi}{62}
\PYG{l+m+mi}{00000862}\PYG{p}{:}    \PYG{l+m+mi}{00} \PYG{l+m+mi}{00} \PYG{l+m+mi}{00} \PYG{l+m+mi}{00}        \PYG{l+m+mi}{00} \PYG{l+m+mi}{00} \PYG{l+m+mi}{04} \PYG{l+m+mi}{56}        \PYG{l+m+mi}{00} \PYG{l+m+mi}{00} \PYG{l+m+mi}{0}\PYG{n}{C} \PYG{l+m+mi}{6}\PYG{n}{E}
\PYG{o}{.}\PYG{o}{.}\PYG{o}{.}
\PYG{l+m+mf}{0016E776}\PYG{p}{:}    \PYG{l+m+mi}{00} \PYG{l+m+mi}{00} \PYG{l+m+mi}{00} \PYG{l+m+mi}{00}        \PYG{l+m+mi}{00} \PYG{l+m+mi}{16} \PYG{n}{E5} \PYG{l+m+mi}{30}        \PYG{l+m+mi}{00} \PYG{l+m+mi}{16} \PYG{n}{E7} \PYG{n}{C0}
\PYG{l+m+mf}{0016E7}\PYG{n}{C0}\PYG{p}{:}    \PYG{l+m+mi}{00} \PYG{l+m+mi}{00} \PYG{l+m+mi}{00} \PYG{l+m+mi}{01}        \PYG{l+m+mi}{00} \PYG{l+m+mi}{16} \PYG{n}{E7} \PYG{l+m+mi}{76}        \PYG{l+m+mi}{00} \PYG{l+m+mi}{16} \PYG{n}{E7} \PYG{n}{CC}
\PYG{l+m+mf}{0016E7}\PYG{n}{CC}\PYG{p}{:}    \PYG{l+m+mi}{00} \PYG{l+m+mi}{00} \PYG{l+m+mi}{00} \PYG{l+m+mi}{01}        \PYG{l+m+mi}{00} \PYG{l+m+mi}{16} \PYG{n}{E7} \PYG{n}{C0}        \PYG{l+m+mi}{00} \PYG{l+m+mi}{16} \PYG{n}{E7} \PYG{n}{D8}
\PYG{l+m+mf}{0016E7}\PYG{n}{D8}\PYG{p}{:} \PYG{n}{EOF}
\end{sphinxVerbatim}
\index{ExceptionTifMarker}

\begin{fulllineitems}
\phantomsection\label{\detokenize{ref/LIS/core/TifMarker:TotalDepth.LIS.core.TifMarker.ExceptionTifMarker}}\pysigline{\sphinxbfcode{exception }\sphinxcode{TotalDepth.LIS.core.TifMarker.}\sphinxbfcode{ExceptionTifMarker}}
Specialisation of exception for Physical Records.

\end{fulllineitems}

\index{TIF\_FIRST\_WORD\_LIMIT (in module TotalDepth.LIS.core.TifMarker)}

\begin{fulllineitems}
\phantomsection\label{\detokenize{ref/LIS/core/TifMarker:TotalDepth.LIS.core.TifMarker.TIF_FIRST_WORD_LIMIT}}\pysigline{\sphinxcode{TotalDepth.LIS.core.TifMarker.}\sphinxbfcode{TIF\_FIRST\_WORD\_LIMIT}\sphinxbfcode{ = 65547}}
The maximum possible size of the first ‘next’ word. If larger than this then
the words are written wrongly as little-endian and need to be reversed
This is calculated as the maximum PR length + TIF bytes.

\end{fulllineitems}

\index{TIF\_NUM\_WORDS (in module TotalDepth.LIS.core.TifMarker)}

\begin{fulllineitems}
\phantomsection\label{\detokenize{ref/LIS/core/TifMarker:TotalDepth.LIS.core.TifMarker.TIF_NUM_WORDS}}\pysigline{\sphinxcode{TotalDepth.LIS.core.TifMarker.}\sphinxbfcode{TIF\_NUM\_WORDS}\sphinxbfcode{ = 3}}
Number of words in a TIF marker

\end{fulllineitems}

\index{TIF\_TOTAL\_BYTES (in module TotalDepth.LIS.core.TifMarker)}

\begin{fulllineitems}
\phantomsection\label{\detokenize{ref/LIS/core/TifMarker:TotalDepth.LIS.core.TifMarker.TIF_TOTAL_BYTES}}\pysigline{\sphinxcode{TotalDepth.LIS.core.TifMarker.}\sphinxbfcode{TIF\_TOTAL\_BYTES}\sphinxbfcode{ = 12}}
Number of bytes in a TIF marker

\end{fulllineitems}

\index{TIF\_WORD\_ALL\_FORMAT (in module TotalDepth.LIS.core.TifMarker)}

\begin{fulllineitems}
\phantomsection\label{\detokenize{ref/LIS/core/TifMarker:TotalDepth.LIS.core.TifMarker.TIF_WORD_ALL_FORMAT}}\pysigline{\sphinxcode{TotalDepth.LIS.core.TifMarker.}\sphinxbfcode{TIF\_WORD\_ALL\_FORMAT}\sphinxbfcode{ = \textless{}Struct object\textgreater{}}}
struct.Struct() format for a TIF marker

\end{fulllineitems}

\index{TIF\_WORD\_ALL\_FORMAT\_WRONG\_SEX (in module TotalDepth.LIS.core.TifMarker)}

\begin{fulllineitems}
\phantomsection\label{\detokenize{ref/LIS/core/TifMarker:TotalDepth.LIS.core.TifMarker.TIF_WORD_ALL_FORMAT_WRONG_SEX}}\pysigline{\sphinxcode{TotalDepth.LIS.core.TifMarker.}\sphinxbfcode{TIF\_WORD\_ALL\_FORMAT\_WRONG\_SEX}\sphinxbfcode{ = \textless{}Struct object\textgreater{}}}
struct.Struct() format for a TIF marker written wrongly as little-endian

\end{fulllineitems}

\index{TIF\_WORD\_BYTES (in module TotalDepth.LIS.core.TifMarker)}

\begin{fulllineitems}
\phantomsection\label{\detokenize{ref/LIS/core/TifMarker:TotalDepth.LIS.core.TifMarker.TIF_WORD_BYTES}}\pysigline{\sphinxcode{TotalDepth.LIS.core.TifMarker.}\sphinxbfcode{TIF\_WORD\_BYTES}\sphinxbfcode{ = 4}}
Number of bytes in a TIF word

\end{fulllineitems}

\index{TIF\_WORD\_FORMAT (in module TotalDepth.LIS.core.TifMarker)}

\begin{fulllineitems}
\phantomsection\label{\detokenize{ref/LIS/core/TifMarker:TotalDepth.LIS.core.TifMarker.TIF_WORD_FORMAT}}\pysigline{\sphinxcode{TotalDepth.LIS.core.TifMarker.}\sphinxbfcode{TIF\_WORD\_FORMAT}\sphinxbfcode{ = \textless{}Struct object\textgreater{}}}
struct.Struct() format for a TIF word

\end{fulllineitems}

\index{TIF\_WORD\_FORMAT\_WRONG\_SEX (in module TotalDepth.LIS.core.TifMarker)}

\begin{fulllineitems}
\phantomsection\label{\detokenize{ref/LIS/core/TifMarker:TotalDepth.LIS.core.TifMarker.TIF_WORD_FORMAT_WRONG_SEX}}\pysigline{\sphinxcode{TotalDepth.LIS.core.TifMarker.}\sphinxbfcode{TIF\_WORD\_FORMAT\_WRONG\_SEX}\sphinxbfcode{ = \textless{}Struct object\textgreater{}}}
struct.Struct() format for a TIF word written wrongly as little-endian

\end{fulllineitems}

\index{TifMarkerBase (class in TotalDepth.LIS.core.TifMarker)}

\begin{fulllineitems}
\phantomsection\label{\detokenize{ref/LIS/core/TifMarker:TotalDepth.LIS.core.TifMarker.TifMarkerBase}}\pysiglinewithargsret{\sphinxbfcode{class }\sphinxcode{TotalDepth.LIS.core.TifMarker.}\sphinxbfcode{TifMarkerBase}}{\emph{raiseOnError=True}}{}
Base class for TIF markers.
\index{eof (TotalDepth.LIS.core.TifMarker.TifMarkerBase attribute)}

\begin{fulllineitems}
\phantomsection\label{\detokenize{ref/LIS/core/TifMarker:TotalDepth.LIS.core.TifMarker.TifMarkerBase.eof}}\pysigline{\sphinxbfcode{eof}}
True if I have encountered a EOF marker.

\end{fulllineitems}

\index{markers() (TotalDepth.LIS.core.TifMarker.TifMarkerBase method)}

\begin{fulllineitems}
\phantomsection\label{\detokenize{ref/LIS/core/TifMarker:TotalDepth.LIS.core.TifMarker.TifMarkerBase.markers}}\pysiglinewithargsret{\sphinxbfcode{markers}}{}{}
Current values of markers as a tuple of three integers.

\end{fulllineitems}

\index{reportError() (TotalDepth.LIS.core.TifMarker.TifMarkerBase method)}

\begin{fulllineitems}
\phantomsection\label{\detokenize{ref/LIS/core/TifMarker:TotalDepth.LIS.core.TifMarker.TifMarkerBase.reportError}}\pysiglinewithargsret{\sphinxbfcode{reportError}}{\emph{theMsg}}{}
Reports the error. I constructed with raiseOnError as True this will
raise a ExceptionTifMarker otherwise it will write the error to the log.

\end{fulllineitems}

\index{reset() (TotalDepth.LIS.core.TifMarker.TifMarkerBase method)}

\begin{fulllineitems}
\phantomsection\label{\detokenize{ref/LIS/core/TifMarker:TotalDepth.LIS.core.TifMarker.TifMarkerBase.reset}}\pysiglinewithargsret{\sphinxbfcode{reset}}{}{}
Resets the TIF markers to all zero, this means hasPrevious is False.

\end{fulllineitems}

\index{strHeader() (TotalDepth.LIS.core.TifMarker.TifMarkerBase method)}

\begin{fulllineitems}
\phantomsection\label{\detokenize{ref/LIS/core/TifMarker:TotalDepth.LIS.core.TifMarker.TifMarkerBase.strHeader}}\pysiglinewithargsret{\sphinxbfcode{strHeader}}{}{}
Header string for an ASCII dump.

\end{fulllineitems}


\end{fulllineitems}

\index{TifMarkerRead (class in TotalDepth.LIS.core.TifMarker)}

\begin{fulllineitems}
\phantomsection\label{\detokenize{ref/LIS/core/TifMarker:TotalDepth.LIS.core.TifMarker.TifMarkerRead}}\pysiglinewithargsret{\sphinxbfcode{class }\sphinxcode{TotalDepth.LIS.core.TifMarker.}\sphinxbfcode{TifMarkerRead}}{\emph{theStream}, \emph{allowPrPadding=False}}{}
Class for reading TIF markers. This will automatically determine if TIF
markers are present and automatically correct ill-formed little-endian TIF markers.

theStream - the file stream.

allowPrPadding - If True this will consume spurious padding bytes after the
Physical Record tail i.e. the TIF markers determine the Physical Record structure
rather than the Physical Record Headers.
\index{hasPrevious (TotalDepth.LIS.core.TifMarker.TifMarkerRead attribute)}

\begin{fulllineitems}
\phantomsection\label{\detokenize{ref/LIS/core/TifMarker:TotalDepth.LIS.core.TifMarker.TifMarkerRead.hasPrevious}}\pysigline{\sphinxbfcode{hasPrevious}}
True if a Physical Record has been read, cleared on reset().

\end{fulllineitems}

\index{read() (TotalDepth.LIS.core.TifMarker.TifMarkerRead method)}

\begin{fulllineitems}
\phantomsection\label{\detokenize{ref/LIS/core/TifMarker:TotalDepth.LIS.core.TifMarker.TifMarkerRead.read}}\pysiglinewithargsret{\sphinxbfcode{read}}{\emph{theStream}}{}
Read TIF markers from a RawStream object. Returns the stream tell()
or None  of the start of the TIF marker. This is not necessarily the
same as the stream tell() seen by the caller as we might consume PR padding.

\end{fulllineitems}

\index{reset() (TotalDepth.LIS.core.TifMarker.TifMarkerRead method)}

\begin{fulllineitems}
\phantomsection\label{\detokenize{ref/LIS/core/TifMarker:TotalDepth.LIS.core.TifMarker.TifMarkerRead.reset}}\pysiglinewithargsret{\sphinxbfcode{reset}}{}{}
Calling reset() means that the caller is probably randomly
accessing the file so we can not error check the previous marker in
the same way that we can if we are reading the file linearly.

\end{fulllineitems}


\end{fulllineitems}

\index{TifMarkerWrite (class in TotalDepth.LIS.core.TifMarker)}

\begin{fulllineitems}
\phantomsection\label{\detokenize{ref/LIS/core/TifMarker:TotalDepth.LIS.core.TifMarker.TifMarkerWrite}}\pysigline{\sphinxbfcode{class }\sphinxcode{TotalDepth.LIS.core.TifMarker.}\sphinxbfcode{TifMarkerWrite}}
Class for writing TIF markers.
\index{close() (TotalDepth.LIS.core.TifMarker.TifMarkerWrite method)}

\begin{fulllineitems}
\phantomsection\label{\detokenize{ref/LIS/core/TifMarker:TotalDepth.LIS.core.TifMarker.TifMarkerWrite.close}}\pysiglinewithargsret{\sphinxbfcode{close}}{\emph{theStream}}{}
Write TIF EOF markers.

\end{fulllineitems}

\index{write() (TotalDepth.LIS.core.TifMarker.TifMarkerWrite method)}

\begin{fulllineitems}
\phantomsection\label{\detokenize{ref/LIS/core/TifMarker:TotalDepth.LIS.core.TifMarker.TifMarkerWrite.write}}\pysiglinewithargsret{\sphinxbfcode{write}}{\emph{theStream}, \emph{theLen}}{}
Write TIF markers to a RawStream object. theLen must be the length
of the Physical Record including the PRH and PRT.

\end{fulllineitems}


\end{fulllineitems}



\subsubsection{Physical Record Handler}
\label{\detokenize{ref/LIS/core/PhysRec::doc}}\label{\detokenize{ref/LIS/core/PhysRec:physical-record-handler}}\phantomsection\label{\detokenize{ref/LIS/core/PhysRec:module-TotalDepth.LIS.core.PhysRec}}\index{TotalDepth.LIS.core.PhysRec (module)}

\paragraph{Physical Record Attributes}
\label{\detokenize{ref/LIS/core/PhysRec:physical-record-attributes}}

\begin{savenotes}\sphinxattablestart
\centering
\begin{tabulary}{\linewidth}[t]{|T|T|}
\hline
\sphinxstylethead{\sphinxstyletheadfamily 
Bit
\unskip}\relax &\sphinxstylethead{\sphinxstyletheadfamily 
Description
\unskip}\relax \\
\hline
15
&
Unused, reserved
\\
\hline
14
&
Physical record type (only 0 is defined)
\\
\hline
13-12
&
00 - No checksum, 01 - 16bit checksum, 10, 11 - Undefined
\\
\hline
11
&
Unused, reserved
\\
\hline
10
&
If 1 File number is present in trailer
\\
\hline
09
&
If 1 Record number is present in trailer
\\
\hline
08
&
Unused
\\
\hline
07
&
Unused, reserved
\\
\hline
06
&
If 1 then a previous parity error has occurred
\\
\hline
05
&
If 1 then a previous checksum error has occurred
\\
\hline
04
&
Unused
\\
\hline
03
&
Unused, reserved
\\
\hline
02
&
Unused
\\
\hline
01
&
If 1 there is a predecessor Physical Record
\\
\hline
00
&
If 1 there is a succcessor Physical Record
\\
\hline
\end{tabulary}
\par
\sphinxattableend\end{savenotes}
\index{ExceptionPhysRec}

\begin{fulllineitems}
\phantomsection\label{\detokenize{ref/LIS/core/PhysRec:TotalDepth.LIS.core.PhysRec.ExceptionPhysRec}}\pysigline{\sphinxbfcode{exception }\sphinxcode{TotalDepth.LIS.core.PhysRec.}\sphinxbfcode{ExceptionPhysRec}}
Specialisation of exception for Physical Records.

\end{fulllineitems}

\index{ExceptionPhysRecEOF}

\begin{fulllineitems}
\phantomsection\label{\detokenize{ref/LIS/core/PhysRec:TotalDepth.LIS.core.PhysRec.ExceptionPhysRecEOF}}\pysigline{\sphinxbfcode{exception }\sphinxcode{TotalDepth.LIS.core.PhysRec.}\sphinxbfcode{ExceptionPhysRecEOF}}
Physical Record unexpected EOF.

\end{fulllineitems}

\index{ExceptionPhysRecUndefinedChecksum}

\begin{fulllineitems}
\phantomsection\label{\detokenize{ref/LIS/core/PhysRec:TotalDepth.LIS.core.PhysRec.ExceptionPhysRecUndefinedChecksum}}\pysigline{\sphinxbfcode{exception }\sphinxcode{TotalDepth.LIS.core.PhysRec.}\sphinxbfcode{ExceptionPhysRecUndefinedChecksum}}
Physical Record encountered undefined checksum bit.

\end{fulllineitems}

\index{ExceptionPhysRecUnknownType}

\begin{fulllineitems}
\phantomsection\label{\detokenize{ref/LIS/core/PhysRec:TotalDepth.LIS.core.PhysRec.ExceptionPhysRecUnknownType}}\pysigline{\sphinxbfcode{exception }\sphinxcode{TotalDepth.LIS.core.PhysRec.}\sphinxbfcode{ExceptionPhysRecUnknownType}}
Encountered unknown type 1 Physical Record.

\end{fulllineitems}

\index{ExceptionPhysRecWrite}

\begin{fulllineitems}
\phantomsection\label{\detokenize{ref/LIS/core/PhysRec:TotalDepth.LIS.core.PhysRec.ExceptionPhysRecWrite}}\pysigline{\sphinxbfcode{exception }\sphinxcode{TotalDepth.LIS.core.PhysRec.}\sphinxbfcode{ExceptionPhysRecWrite}}
Physical Record writing.

\end{fulllineitems}

\index{ExceptionPhysRecTail}

\begin{fulllineitems}
\phantomsection\label{\detokenize{ref/LIS/core/PhysRec:TotalDepth.LIS.core.PhysRec.ExceptionPhysRecTail}}\pysigline{\sphinxbfcode{exception }\sphinxcode{TotalDepth.LIS.core.PhysRec.}\sphinxbfcode{ExceptionPhysRecTail}}
Physical Record Trailer exception.

\end{fulllineitems}

\index{PR\_PRH\_LEN\_FORMAT (in module TotalDepth.LIS.core.PhysRec)}

\begin{fulllineitems}
\phantomsection\label{\detokenize{ref/LIS/core/PhysRec:TotalDepth.LIS.core.PhysRec.PR_PRH_LEN_FORMAT}}\pysigline{\sphinxcode{TotalDepth.LIS.core.PhysRec.}\sphinxbfcode{PR\_PRH\_LEN\_FORMAT}\sphinxbfcode{ = \textless{}Struct object\textgreater{}}}
PR Header 4 bytes long, two big-endian 16 bit numbers
The struct.Struct() format for the Physical Record Header

\end{fulllineitems}

\index{PR\_PRH\_ATTR\_FORMAT (in module TotalDepth.LIS.core.PhysRec)}

\begin{fulllineitems}
\phantomsection\label{\detokenize{ref/LIS/core/PhysRec:TotalDepth.LIS.core.PhysRec.PR_PRH_ATTR_FORMAT}}\pysigline{\sphinxcode{TotalDepth.LIS.core.PhysRec.}\sphinxbfcode{PR\_PRH\_ATTR\_FORMAT}\sphinxbfcode{ = \textless{}Struct object\textgreater{}}}
The struct.Struct() format for the Physical Record Header attributes

\end{fulllineitems}

\index{PR\_PRH\_LENGTH (in module TotalDepth.LIS.core.PhysRec)}

\begin{fulllineitems}
\phantomsection\label{\detokenize{ref/LIS/core/PhysRec:TotalDepth.LIS.core.PhysRec.PR_PRH_LENGTH}}\pysigline{\sphinxcode{TotalDepth.LIS.core.PhysRec.}\sphinxbfcode{PR\_PRH\_LENGTH}\sphinxbfcode{ = 4}}
The length of the Physical Record Header

\end{fulllineitems}

\index{PR\_ATTRIBUTE\_BITS (in module TotalDepth.LIS.core.PhysRec)}

\begin{fulllineitems}
\phantomsection\label{\detokenize{ref/LIS/core/PhysRec:TotalDepth.LIS.core.PhysRec.PR_ATTRIBUTE_BITS}}\pysigline{\sphinxcode{TotalDepth.LIS.core.PhysRec.}\sphinxbfcode{PR\_ATTRIBUTE\_BITS}\sphinxbfcode{ = 16}}
Number of bits in the 2 byte attributes

\end{fulllineitems}

\index{PR\_SUCCESSOR\_ATTRIBUTE\_BIT (in module TotalDepth.LIS.core.PhysRec)}

\begin{fulllineitems}
\phantomsection\label{\detokenize{ref/LIS/core/PhysRec:TotalDepth.LIS.core.PhysRec.PR_SUCCESSOR_ATTRIBUTE_BIT}}\pysigline{\sphinxcode{TotalDepth.LIS.core.PhysRec.}\sphinxbfcode{PR\_SUCCESSOR\_ATTRIBUTE\_BIT}\sphinxbfcode{ = 0}}
Successor bit position

\end{fulllineitems}

\index{PR\_PREDECESSOR\_ATTRIBUTE\_BIT (in module TotalDepth.LIS.core.PhysRec)}

\begin{fulllineitems}
\phantomsection\label{\detokenize{ref/LIS/core/PhysRec:TotalDepth.LIS.core.PhysRec.PR_PREDECESSOR_ATTRIBUTE_BIT}}\pysigline{\sphinxcode{TotalDepth.LIS.core.PhysRec.}\sphinxbfcode{PR\_PREDECESSOR\_ATTRIBUTE\_BIT}\sphinxbfcode{ = 1}}
Predessor bit position

\end{fulllineitems}

\index{PR\_OLD\_CHECK\_ERROR\_BIT (in module TotalDepth.LIS.core.PhysRec)}

\begin{fulllineitems}
\phantomsection\label{\detokenize{ref/LIS/core/PhysRec:TotalDepth.LIS.core.PhysRec.PR_OLD_CHECK_ERROR_BIT}}\pysigline{\sphinxcode{TotalDepth.LIS.core.PhysRec.}\sphinxbfcode{PR\_OLD\_CHECK\_ERROR\_BIT}\sphinxbfcode{ = 5}}
Checksum error bit position

\end{fulllineitems}

\index{PR\_OLD\_PARITY\_ERROR\_BIT (in module TotalDepth.LIS.core.PhysRec)}

\begin{fulllineitems}
\phantomsection\label{\detokenize{ref/LIS/core/PhysRec:TotalDepth.LIS.core.PhysRec.PR_OLD_PARITY_ERROR_BIT}}\pysigline{\sphinxcode{TotalDepth.LIS.core.PhysRec.}\sphinxbfcode{PR\_OLD\_PARITY\_ERROR\_BIT}\sphinxbfcode{ = 6}}
Parity error bit position

\end{fulllineitems}

\index{PR\_RECORD\_NUMBER\_BIT (in module TotalDepth.LIS.core.PhysRec)}

\begin{fulllineitems}
\phantomsection\label{\detokenize{ref/LIS/core/PhysRec:TotalDepth.LIS.core.PhysRec.PR_RECORD_NUMBER_BIT}}\pysigline{\sphinxcode{TotalDepth.LIS.core.PhysRec.}\sphinxbfcode{PR\_RECORD\_NUMBER\_BIT}\sphinxbfcode{ = 9}}
Bit position to indicate there is a record number in the trailer

\end{fulllineitems}

\index{PR\_FILE\_NUMBER\_BIT (in module TotalDepth.LIS.core.PhysRec)}

\begin{fulllineitems}
\phantomsection\label{\detokenize{ref/LIS/core/PhysRec:TotalDepth.LIS.core.PhysRec.PR_FILE_NUMBER_BIT}}\pysigline{\sphinxcode{TotalDepth.LIS.core.PhysRec.}\sphinxbfcode{PR\_FILE\_NUMBER\_BIT}\sphinxbfcode{ = 10}}
Bit position to indicate there is a file number in the trailer

\end{fulllineitems}

\index{PR\_CHECKSUM\_BIT (in module TotalDepth.LIS.core.PhysRec)}

\begin{fulllineitems}
\phantomsection\label{\detokenize{ref/LIS/core/PhysRec:TotalDepth.LIS.core.PhysRec.PR_CHECKSUM_BIT}}\pysigline{\sphinxcode{TotalDepth.LIS.core.PhysRec.}\sphinxbfcode{PR\_CHECKSUM\_BIT}\sphinxbfcode{ = 12}}
Bit position to indicate there is a 16bit checksum in the trailer

\end{fulllineitems}

\index{PR\_CHECKSUM\_UNDEFINED\_BIT (in module TotalDepth.LIS.core.PhysRec)}

\begin{fulllineitems}
\phantomsection\label{\detokenize{ref/LIS/core/PhysRec:TotalDepth.LIS.core.PhysRec.PR_CHECKSUM_UNDEFINED_BIT}}\pysigline{\sphinxcode{TotalDepth.LIS.core.PhysRec.}\sphinxbfcode{PR\_CHECKSUM\_UNDEFINED\_BIT}\sphinxbfcode{ = 13}}
Bit position to indicate checksum is undefined

\end{fulllineitems}

\index{PR\_TYPE\_BIT (in module TotalDepth.LIS.core.PhysRec)}

\begin{fulllineitems}
\phantomsection\label{\detokenize{ref/LIS/core/PhysRec:TotalDepth.LIS.core.PhysRec.PR_TYPE_BIT}}\pysigline{\sphinxcode{TotalDepth.LIS.core.PhysRec.}\sphinxbfcode{PR\_TYPE\_BIT}\sphinxbfcode{ = 14}}
Bit position to indicate Physical Record Type

\end{fulllineitems}

\index{PR\_ATTRIBUTE\_UNUSED\_ONLY\_MASK (in module TotalDepth.LIS.core.PhysRec)}

\begin{fulllineitems}
\phantomsection\label{\detokenize{ref/LIS/core/PhysRec:TotalDepth.LIS.core.PhysRec.PR_ATTRIBUTE_UNUSED_ONLY_MASK}}\pysigline{\sphinxcode{TotalDepth.LIS.core.PhysRec.}\sphinxbfcode{PR\_ATTRIBUTE\_UNUSED\_ONLY\_MASK}\sphinxbfcode{ = 276}}
Unused only bits - i.e. Unused but not Unused, reserved

\end{fulllineitems}

\index{PR\_ATTRIBUTE\_UNUSED\_RESERVED\_MASK (in module TotalDepth.LIS.core.PhysRec)}

\begin{fulllineitems}
\phantomsection\label{\detokenize{ref/LIS/core/PhysRec:TotalDepth.LIS.core.PhysRec.PR_ATTRIBUTE_UNUSED_RESERVED_MASK}}\pysigline{\sphinxcode{TotalDepth.LIS.core.PhysRec.}\sphinxbfcode{PR\_ATTRIBUTE\_UNUSED\_RESERVED\_MASK}\sphinxbfcode{ = 34952}}
Unused, reserved bit mask

\end{fulllineitems}

\index{PR\_ATTRIBUTE\_UNUSED\_MASK (in module TotalDepth.LIS.core.PhysRec)}

\begin{fulllineitems}
\phantomsection\label{\detokenize{ref/LIS/core/PhysRec:TotalDepth.LIS.core.PhysRec.PR_ATTRIBUTE_UNUSED_MASK}}\pysigline{\sphinxcode{TotalDepth.LIS.core.PhysRec.}\sphinxbfcode{PR\_ATTRIBUTE\_UNUSED\_MASK}\sphinxbfcode{ = 0}}
Unused and Unused, reserved bits, 0x899C

\end{fulllineitems}

\index{PR\_PRT\_REC\_NUM\_FORMAT (in module TotalDepth.LIS.core.PhysRec)}

\begin{fulllineitems}
\phantomsection\label{\detokenize{ref/LIS/core/PhysRec:TotalDepth.LIS.core.PhysRec.PR_PRT_REC_NUM_FORMAT}}\pysigline{\sphinxcode{TotalDepth.LIS.core.PhysRec.}\sphinxbfcode{PR\_PRT\_REC\_NUM\_FORMAT}\sphinxbfcode{ = \textless{}Struct object\textgreater{}}}
The struct.Struct() format for the Physical Record Trailer record number

\end{fulllineitems}

\index{PR\_PRT\_REC\_NUM\_LEN (in module TotalDepth.LIS.core.PhysRec)}

\begin{fulllineitems}
\phantomsection\label{\detokenize{ref/LIS/core/PhysRec:TotalDepth.LIS.core.PhysRec.PR_PRT_REC_NUM_LEN}}\pysigline{\sphinxcode{TotalDepth.LIS.core.PhysRec.}\sphinxbfcode{PR\_PRT\_REC\_NUM\_LEN}\sphinxbfcode{ = 2}}
The length of the Physical Record Trailer for the record number

\end{fulllineitems}

\index{PR\_PRT\_REC\_NUM\_MIN (in module TotalDepth.LIS.core.PhysRec)}

\begin{fulllineitems}
\phantomsection\label{\detokenize{ref/LIS/core/PhysRec:TotalDepth.LIS.core.PhysRec.PR_PRT_REC_NUM_MIN}}\pysigline{\sphinxcode{TotalDepth.LIS.core.PhysRec.}\sphinxbfcode{PR\_PRT\_REC\_NUM\_MIN}\sphinxbfcode{ = -32768}}
The minimum record number

\end{fulllineitems}

\index{PR\_PRT\_REC\_NUM\_MAX (in module TotalDepth.LIS.core.PhysRec)}

\begin{fulllineitems}
\phantomsection\label{\detokenize{ref/LIS/core/PhysRec:TotalDepth.LIS.core.PhysRec.PR_PRT_REC_NUM_MAX}}\pysigline{\sphinxcode{TotalDepth.LIS.core.PhysRec.}\sphinxbfcode{PR\_PRT\_REC\_NUM\_MAX}\sphinxbfcode{ = 32767}}
The maximum record number

\end{fulllineitems}

\index{PR\_PRT\_FILE\_NUM\_FORMAT (in module TotalDepth.LIS.core.PhysRec)}

\begin{fulllineitems}
\phantomsection\label{\detokenize{ref/LIS/core/PhysRec:TotalDepth.LIS.core.PhysRec.PR_PRT_FILE_NUM_FORMAT}}\pysigline{\sphinxcode{TotalDepth.LIS.core.PhysRec.}\sphinxbfcode{PR\_PRT\_FILE\_NUM\_FORMAT}\sphinxbfcode{ = \textless{}Struct object\textgreater{}}}
The struct.Struct() format for the Physical Record Trailer file number

\end{fulllineitems}

\index{PR\_PRT\_FILE\_NUM\_LEN (in module TotalDepth.LIS.core.PhysRec)}

\begin{fulllineitems}
\phantomsection\label{\detokenize{ref/LIS/core/PhysRec:TotalDepth.LIS.core.PhysRec.PR_PRT_FILE_NUM_LEN}}\pysigline{\sphinxcode{TotalDepth.LIS.core.PhysRec.}\sphinxbfcode{PR\_PRT\_FILE\_NUM\_LEN}\sphinxbfcode{ = 2}}
The length of the Physical Record Trailer for the file number

\end{fulllineitems}

\index{PR\_PRT\_FILE\_NUM\_MIN (in module TotalDepth.LIS.core.PhysRec)}

\begin{fulllineitems}
\phantomsection\label{\detokenize{ref/LIS/core/PhysRec:TotalDepth.LIS.core.PhysRec.PR_PRT_FILE_NUM_MIN}}\pysigline{\sphinxcode{TotalDepth.LIS.core.PhysRec.}\sphinxbfcode{PR\_PRT\_FILE\_NUM\_MIN}\sphinxbfcode{ = -32768}}
The minimum file number

\end{fulllineitems}

\index{PR\_PRT\_FILE\_NUM\_MAX (in module TotalDepth.LIS.core.PhysRec)}

\begin{fulllineitems}
\phantomsection\label{\detokenize{ref/LIS/core/PhysRec:TotalDepth.LIS.core.PhysRec.PR_PRT_FILE_NUM_MAX}}\pysigline{\sphinxcode{TotalDepth.LIS.core.PhysRec.}\sphinxbfcode{PR\_PRT\_FILE\_NUM\_MAX}\sphinxbfcode{ = 32767}}
The maximum file number

\end{fulllineitems}

\index{PR\_PRT\_CHECKSUM\_FORMAT (in module TotalDepth.LIS.core.PhysRec)}

\begin{fulllineitems}
\phantomsection\label{\detokenize{ref/LIS/core/PhysRec:TotalDepth.LIS.core.PhysRec.PR_PRT_CHECKSUM_FORMAT}}\pysigline{\sphinxcode{TotalDepth.LIS.core.PhysRec.}\sphinxbfcode{PR\_PRT\_CHECKSUM\_FORMAT}\sphinxbfcode{ = \textless{}Struct object\textgreater{}}}
The struct.Struct() format for the Physical Record Trailer checksum

\end{fulllineitems}

\index{PR\_PRT\_CHECKSUM\_LEN (in module TotalDepth.LIS.core.PhysRec)}

\begin{fulllineitems}
\phantomsection\label{\detokenize{ref/LIS/core/PhysRec:TotalDepth.LIS.core.PhysRec.PR_PRT_CHECKSUM_LEN}}\pysigline{\sphinxcode{TotalDepth.LIS.core.PhysRec.}\sphinxbfcode{PR\_PRT\_CHECKSUM\_LEN}\sphinxbfcode{ = 2}}
The length of the Physical Record Trailer for the checksum

\end{fulllineitems}

\index{PR\_MAX\_LENGTH (in module TotalDepth.LIS.core.PhysRec)}

\begin{fulllineitems}
\phantomsection\label{\detokenize{ref/LIS/core/PhysRec:TotalDepth.LIS.core.PhysRec.PR_MAX_LENGTH}}\pysigline{\sphinxcode{TotalDepth.LIS.core.PhysRec.}\sphinxbfcode{PR\_MAX\_LENGTH}\sphinxbfcode{ = 65535}}
Maximum possible Physical Record length represented by an unsigned 16 bit int

\end{fulllineitems}

\index{PhysRecBase (class in TotalDepth.LIS.core.PhysRec)}

\begin{fulllineitems}
\phantomsection\label{\detokenize{ref/LIS/core/PhysRec:TotalDepth.LIS.core.PhysRec.PhysRecBase}}\pysiglinewithargsret{\sphinxbfcode{class }\sphinxcode{TotalDepth.LIS.core.PhysRec.}\sphinxbfcode{PhysRecBase}}{\emph{theFileId}, \emph{keepGoing}}{}
Base class for physical record read and write.
TODO: Checksum reading, writing and testing.
\index{close() (TotalDepth.LIS.core.PhysRec.PhysRecBase method)}

\begin{fulllineitems}
\phantomsection\label{\detokenize{ref/LIS/core/PhysRec:TotalDepth.LIS.core.PhysRec.PhysRecBase.close}}\pysiglinewithargsret{\sphinxbfcode{close}}{}{}
Close the underlying stream, further operations will raise a
ValueError.

\end{fulllineitems}

\index{strHeader() (TotalDepth.LIS.core.PhysRec.PhysRecBase method)}

\begin{fulllineitems}
\phantomsection\label{\detokenize{ref/LIS/core/PhysRec:TotalDepth.LIS.core.PhysRec.PhysRecBase.strHeader}}\pysiglinewithargsret{\sphinxbfcode{strHeader}}{}{}
Returns the header string to go at the top of a list of \_\_str\_\_().

\end{fulllineitems}


\end{fulllineitems}

\index{PhysRecRead (class in TotalDepth.LIS.core.PhysRec)}

\begin{fulllineitems}
\phantomsection\label{\detokenize{ref/LIS/core/PhysRec:TotalDepth.LIS.core.PhysRec.PhysRecRead}}\pysiglinewithargsret{\sphinxbfcode{class }\sphinxcode{TotalDepth.LIS.core.PhysRec.}\sphinxbfcode{PhysRecRead}}{\emph{theFile}, \emph{theFileId=None}, \emph{keepGoing=False}}{}
Specialisation of PhysRecBase for reading streams.
\index{readLrBytes() (TotalDepth.LIS.core.PhysRec.PhysRecRead method)}

\begin{fulllineitems}
\phantomsection\label{\detokenize{ref/LIS/core/PhysRec:TotalDepth.LIS.core.PhysRec.PhysRecRead.readLrBytes}}\pysiglinewithargsret{\sphinxbfcode{readLrBytes}}{\emph{theSize=-1}, \emph{theLd=None}}{}
Reads theSize logical data bytes and returns it as a bytes() object.
If theSize is -1 all logical data for this logical record is returned.
If theLd is not None it is extended and returned, otherwise a new
bytes() object is created and returned.
Returns None on end of logical record.

\end{fulllineitems}

\index{skipLrBytes() (TotalDepth.LIS.core.PhysRec.PhysRecRead method)}

\begin{fulllineitems}
\phantomsection\label{\detokenize{ref/LIS/core/PhysRec:TotalDepth.LIS.core.PhysRec.PhysRecRead.skipLrBytes}}\pysiglinewithargsret{\sphinxbfcode{skipLrBytes}}{\emph{theSize=-1}}{}
Skips logical data and returns a count of skipped bytes.
If theSize is -1 all logical data for this logical record is skipped
positioning the stream at end of this logical record (Note: not the 
beginning of the next logical record).
Returns 0 on end of logical record.

\end{fulllineitems}

\index{skipToNextLr() (TotalDepth.LIS.core.PhysRec.PhysRecRead method)}

\begin{fulllineitems}
\phantomsection\label{\detokenize{ref/LIS/core/PhysRec:TotalDepth.LIS.core.PhysRec.PhysRecRead.skipToNextLr}}\pysiglinewithargsret{\sphinxbfcode{skipToNextLr}}{}{}
Skips all remaining logical data, the PR trailer, and the next PR
header. This positions stream at the start of the next logical record.
May raise an ExceptionPhysRecEOF if there is no further Logical or
Physical record.
Returns the number of Logical Data bytes skipped.

\end{fulllineitems}

\index{tellLr() (TotalDepth.LIS.core.PhysRec.PhysRecRead method)}

\begin{fulllineitems}
\phantomsection\label{\detokenize{ref/LIS/core/PhysRec:TotalDepth.LIS.core.PhysRec.PhysRecRead.tellLr}}\pysiglinewithargsret{\sphinxbfcode{tellLr}}{}{}
Returns the absolute file position of the start current Logical
record. This value can be safely used in seekLr.

\end{fulllineitems}

\index{tell() (TotalDepth.LIS.core.PhysRec.PhysRecRead method)}

\begin{fulllineitems}
\phantomsection\label{\detokenize{ref/LIS/core/PhysRec:TotalDepth.LIS.core.PhysRec.PhysRecRead.tell}}\pysiglinewithargsret{\sphinxbfcode{tell}}{}{}
Returns the absolute position of the file.

\end{fulllineitems}

\index{seekLr() (TotalDepth.LIS.core.PhysRec.PhysRecRead method)}

\begin{fulllineitems}
\phantomsection\label{\detokenize{ref/LIS/core/PhysRec:TotalDepth.LIS.core.PhysRec.PhysRecRead.seekLr}}\pysiglinewithargsret{\sphinxbfcode{seekLr}}{\emph{offset}}{}
External setting of file position directly to the beginning of
a PRH or TIF marker (if present). The caller is fully responsible
for getting this right!

\end{fulllineitems}

\index{seekCurrentLrStart() (TotalDepth.LIS.core.PhysRec.PhysRecRead method)}

\begin{fulllineitems}
\phantomsection\label{\detokenize{ref/LIS/core/PhysRec:TotalDepth.LIS.core.PhysRec.PhysRecRead.seekCurrentLrStart}}\pysiglinewithargsret{\sphinxbfcode{seekCurrentLrStart}}{}{}
Setting the file position directly to the beginning of
a PRH or TIF marker (if present) for the current Logical Rcord.

\end{fulllineitems}

\index{hasLd() (TotalDepth.LIS.core.PhysRec.PhysRecRead method)}

\begin{fulllineitems}
\phantomsection\label{\detokenize{ref/LIS/core/PhysRec:TotalDepth.LIS.core.PhysRec.PhysRecRead.hasLd}}\pysiglinewithargsret{\sphinxbfcode{hasLd}}{}{}
Returns True if there is logical data to be read, False otherwise.
NOTE: This will return False on file initialisation and only return
True once the Physical Record Header has been read.

\end{fulllineitems}

\index{ldRemaingInPr() (TotalDepth.LIS.core.PhysRec.PhysRecRead method)}

\begin{fulllineitems}
\phantomsection\label{\detokenize{ref/LIS/core/PhysRec:TotalDepth.LIS.core.PhysRec.PhysRecRead.ldRemaingInPr}}\pysiglinewithargsret{\sphinxbfcode{ldRemaingInPr}}{}{}
Returns the number of bytes remaining in this particular Physical
Record. NOTE: The can be 0 and hasLd() be True if at the end of a
Physical Record that has a successor record.

\end{fulllineitems}

\index{genLd() (TotalDepth.LIS.core.PhysRec.PhysRecRead method)}

\begin{fulllineitems}
\phantomsection\label{\detokenize{ref/LIS/core/PhysRec:TotalDepth.LIS.core.PhysRec.PhysRecRead.genLd}}\pysiglinewithargsret{\sphinxbfcode{genLd}}{}{}
A generator that produces a tuple of (logical data, isLrStart)
where:
logical data - A bytes() object for the logical data in the current PR.
isLrStart - A boolean that is True of that LogicalData is the start of
a logical record.
NOTE: This rewinds the current state of this instance.

\end{fulllineitems}


\end{fulllineitems}

\index{PhysRecTail (class in TotalDepth.LIS.core.PhysRec)}

\begin{fulllineitems}
\phantomsection\label{\detokenize{ref/LIS/core/PhysRec:TotalDepth.LIS.core.PhysRec.PhysRecTail}}\pysiglinewithargsret{\sphinxbfcode{class }\sphinxcode{TotalDepth.LIS.core.PhysRec.}\sphinxbfcode{PhysRecTail}}{\emph{hasRecNum=False}, \emph{fileNum=None}, \emph{hasCheckSum=False}}{}
Represents Physical Record Tail fields.
\index{prhAttr (TotalDepth.LIS.core.PhysRec.PhysRecTail attribute)}

\begin{fulllineitems}
\phantomsection\label{\detokenize{ref/LIS/core/PhysRec:TotalDepth.LIS.core.PhysRec.PhysRecTail.prhAttr}}\pysigline{\sphinxbfcode{prhAttr}}
Returns the PRH attributes, to be or’d with any other attributes.

\end{fulllineitems}

\index{hasTail() (TotalDepth.LIS.core.PhysRec.PhysRecTail method)}

\begin{fulllineitems}
\phantomsection\label{\detokenize{ref/LIS/core/PhysRec:TotalDepth.LIS.core.PhysRec.PhysRecTail.hasTail}}\pysiglinewithargsret{\sphinxbfcode{hasTail}}{}{}
Returns True if any PRT field is present, False otherwise.

\end{fulllineitems}

\index{computeCheckSum() (TotalDepth.LIS.core.PhysRec.PhysRecTail method)}

\begin{fulllineitems}
\phantomsection\label{\detokenize{ref/LIS/core/PhysRec:TotalDepth.LIS.core.PhysRec.PhysRecTail.computeCheckSum}}\pysiglinewithargsret{\sphinxbfcode{computeCheckSum}}{\emph{theB}}{}
Computes the checksum of the byte stream.

\end{fulllineitems}


\end{fulllineitems}

\index{PhysRecWrite (class in TotalDepth.LIS.core.PhysRec)}

\begin{fulllineitems}
\phantomsection\label{\detokenize{ref/LIS/core/PhysRec:TotalDepth.LIS.core.PhysRec.PhysRecWrite}}\pysiglinewithargsret{\sphinxbfcode{class }\sphinxcode{TotalDepth.LIS.core.PhysRec.}\sphinxbfcode{PhysRecWrite}}{\emph{theFile}, \emph{theFileId=None}, \emph{keepGoing=False}, \emph{hasTif=False}, \emph{thePrLen=65535}, \emph{thePrt=\textless{}TotalDepth.LIS.core.PhysRec.PhysRecTail object\textgreater{}}}{}
Specialisation of PhysRecBase for writing to files.
\index{close() (TotalDepth.LIS.core.PhysRec.PhysRecWrite method)}

\begin{fulllineitems}
\phantomsection\label{\detokenize{ref/LIS/core/PhysRec:TotalDepth.LIS.core.PhysRec.PhysRecWrite.close}}\pysiglinewithargsret{\sphinxbfcode{close}}{}{}
Close the Physical Record Handler and the underlying stream.

\end{fulllineitems}

\index{writeLr() (TotalDepth.LIS.core.PhysRec.PhysRecWrite method)}

\begin{fulllineitems}
\phantomsection\label{\detokenize{ref/LIS/core/PhysRec:TotalDepth.LIS.core.PhysRec.PhysRecWrite.writeLr}}\pysiglinewithargsret{\sphinxbfcode{writeLr}}{\emph{theLr}}{}
Splits a Logical Record into into Physical Records and writes them
to the stream. These Physical Records have trailer records if required.
Returns the tell() of the start of the LR.

\end{fulllineitems}


\end{fulllineitems}



\subsubsection{Logical Record Handler}
\label{\detokenize{ref/LIS/core/LogiRec:logical-record-handler}}\label{\detokenize{ref/LIS/core/LogiRec::doc}}\phantomsection\label{\detokenize{ref/LIS/core/LogiRec:module-TotalDepth.LIS.core.LogiRec}}\index{TotalDepth.LIS.core.LogiRec (module)}
Handles LIS Logical Records.
\index{ExceptionLr}

\begin{fulllineitems}
\phantomsection\label{\detokenize{ref/LIS/core/LogiRec:TotalDepth.LIS.core.LogiRec.ExceptionLr}}\pysigline{\sphinxbfcode{exception }\sphinxcode{TotalDepth.LIS.core.LogiRec.}\sphinxbfcode{ExceptionLr}}
Specialisation of exception for Logical Records.

\end{fulllineitems}

\index{ExceptionLrNotImplemented}

\begin{fulllineitems}
\phantomsection\label{\detokenize{ref/LIS/core/LogiRec:TotalDepth.LIS.core.LogiRec.ExceptionLrNotImplemented}}\pysigline{\sphinxbfcode{exception }\sphinxcode{TotalDepth.LIS.core.LogiRec.}\sphinxbfcode{ExceptionLrNotImplemented}}
Logical Records that have no implementation here.

\end{fulllineitems}

\index{ExceptionCbWrite}

\begin{fulllineitems}
\phantomsection\label{\detokenize{ref/LIS/core/LogiRec:TotalDepth.LIS.core.LogiRec.ExceptionCbWrite}}\pysigline{\sphinxbfcode{exception }\sphinxcode{TotalDepth.LIS.core.LogiRec.}\sphinxbfcode{ExceptionCbWrite}}
Raised when creating a component block with Python native types that has illogical or conflicting data.

\end{fulllineitems}

\index{LR\_TYPE\_NORMAL\_DATA (in module TotalDepth.LIS.core.LogiRec)}

\begin{fulllineitems}
\phantomsection\label{\detokenize{ref/LIS/core/LogiRec:TotalDepth.LIS.core.LogiRec.LR_TYPE_NORMAL_DATA}}\pysigline{\sphinxcode{TotalDepth.LIS.core.LogiRec.}\sphinxbfcode{LR\_TYPE\_NORMAL\_DATA}\sphinxbfcode{ = 0}}
Normal data record containing log data

\end{fulllineitems}

\index{LR\_TYPE\_ALTERNATE\_DATA (in module TotalDepth.LIS.core.LogiRec)}

\begin{fulllineitems}
\phantomsection\label{\detokenize{ref/LIS/core/LogiRec:TotalDepth.LIS.core.LogiRec.LR_TYPE_ALTERNATE_DATA}}\pysigline{\sphinxcode{TotalDepth.LIS.core.LogiRec.}\sphinxbfcode{LR\_TYPE\_ALTERNATE\_DATA}\sphinxbfcode{ = 1}}
Alternate data.

\end{fulllineitems}

\index{LR\_TYPE\_JOB\_ID (in module TotalDepth.LIS.core.LogiRec)}

\begin{fulllineitems}
\phantomsection\label{\detokenize{ref/LIS/core/LogiRec:TotalDepth.LIS.core.LogiRec.LR_TYPE_JOB_ID}}\pysigline{\sphinxcode{TotalDepth.LIS.core.LogiRec.}\sphinxbfcode{LR\_TYPE\_JOB\_ID}\sphinxbfcode{ = 32}}
Job identification

\end{fulllineitems}

\index{LR\_TYPE\_WELL\_DATA (in module TotalDepth.LIS.core.LogiRec)}

\begin{fulllineitems}
\phantomsection\label{\detokenize{ref/LIS/core/LogiRec:TotalDepth.LIS.core.LogiRec.LR_TYPE_WELL_DATA}}\pysigline{\sphinxcode{TotalDepth.LIS.core.LogiRec.}\sphinxbfcode{LR\_TYPE\_WELL\_DATA}\sphinxbfcode{ = 34}}
Well site data

\end{fulllineitems}

\index{LR\_TYPE\_TOOL\_INFO (in module TotalDepth.LIS.core.LogiRec)}

\begin{fulllineitems}
\phantomsection\label{\detokenize{ref/LIS/core/LogiRec:TotalDepth.LIS.core.LogiRec.LR_TYPE_TOOL_INFO}}\pysigline{\sphinxcode{TotalDepth.LIS.core.LogiRec.}\sphinxbfcode{LR\_TYPE\_TOOL\_INFO}\sphinxbfcode{ = 39}}
Tool string info

\end{fulllineitems}

\index{LR\_TYPE\_ENCRYPTED\_TABLE (in module TotalDepth.LIS.core.LogiRec)}

\begin{fulllineitems}
\phantomsection\label{\detokenize{ref/LIS/core/LogiRec:TotalDepth.LIS.core.LogiRec.LR_TYPE_ENCRYPTED_TABLE}}\pysigline{\sphinxcode{TotalDepth.LIS.core.LogiRec.}\sphinxbfcode{LR\_TYPE\_ENCRYPTED\_TABLE}\sphinxbfcode{ = 42}}
Encrypted table dump

\end{fulllineitems}

\index{LR\_TYPE\_TABLE\_DUMP (in module TotalDepth.LIS.core.LogiRec)}

\begin{fulllineitems}
\phantomsection\label{\detokenize{ref/LIS/core/LogiRec:TotalDepth.LIS.core.LogiRec.LR_TYPE_TABLE_DUMP}}\pysigline{\sphinxcode{TotalDepth.LIS.core.LogiRec.}\sphinxbfcode{LR\_TYPE\_TABLE\_DUMP}\sphinxbfcode{ = 47}}
Table dump

\end{fulllineitems}

\index{LR\_TYPE\_DATA\_FORMAT (in module TotalDepth.LIS.core.LogiRec)}

\begin{fulllineitems}
\phantomsection\label{\detokenize{ref/LIS/core/LogiRec:TotalDepth.LIS.core.LogiRec.LR_TYPE_DATA_FORMAT}}\pysigline{\sphinxcode{TotalDepth.LIS.core.LogiRec.}\sphinxbfcode{LR\_TYPE\_DATA\_FORMAT}\sphinxbfcode{ = 64}}
Data format specification record

\end{fulllineitems}

\index{LR\_TYPE\_DATA\_DESCRIPTOR (in module TotalDepth.LIS.core.LogiRec)}

\begin{fulllineitems}
\phantomsection\label{\detokenize{ref/LIS/core/LogiRec:TotalDepth.LIS.core.LogiRec.LR_TYPE_DATA_DESCRIPTOR}}\pysigline{\sphinxcode{TotalDepth.LIS.core.LogiRec.}\sphinxbfcode{LR\_TYPE\_DATA\_DESCRIPTOR}\sphinxbfcode{ = 65}}
Data descriptor (not defined in the LIS79 Description Reference Manual)

\end{fulllineitems}

\index{LR\_TYPE\_TU10\_BOOT (in module TotalDepth.LIS.core.LogiRec)}

\begin{fulllineitems}
\phantomsection\label{\detokenize{ref/LIS/core/LogiRec:TotalDepth.LIS.core.LogiRec.LR_TYPE_TU10_BOOT}}\pysigline{\sphinxcode{TotalDepth.LIS.core.LogiRec.}\sphinxbfcode{LR\_TYPE\_TU10\_BOOT}\sphinxbfcode{ = 95}}
TU10 software boot

\end{fulllineitems}

\index{LR\_TYPE\_BOOTSTRAP\_LOADER (in module TotalDepth.LIS.core.LogiRec)}

\begin{fulllineitems}
\phantomsection\label{\detokenize{ref/LIS/core/LogiRec:TotalDepth.LIS.core.LogiRec.LR_TYPE_BOOTSTRAP_LOADER}}\pysigline{\sphinxcode{TotalDepth.LIS.core.LogiRec.}\sphinxbfcode{LR\_TYPE\_BOOTSTRAP\_LOADER}\sphinxbfcode{ = 96}}
Bootstrap loader

\end{fulllineitems}

\index{LR\_TYPE\_CP\_KERNEL (in module TotalDepth.LIS.core.LogiRec)}

\begin{fulllineitems}
\phantomsection\label{\detokenize{ref/LIS/core/LogiRec:TotalDepth.LIS.core.LogiRec.LR_TYPE_CP_KERNEL}}\pysigline{\sphinxcode{TotalDepth.LIS.core.LogiRec.}\sphinxbfcode{LR\_TYPE\_CP\_KERNEL}\sphinxbfcode{ = 97}}
CP-kernel loader boot

\end{fulllineitems}

\index{LR\_TYPE\_PROGRAM\_FILE\_HEAD (in module TotalDepth.LIS.core.LogiRec)}

\begin{fulllineitems}
\phantomsection\label{\detokenize{ref/LIS/core/LogiRec:TotalDepth.LIS.core.LogiRec.LR_TYPE_PROGRAM_FILE_HEAD}}\pysigline{\sphinxcode{TotalDepth.LIS.core.LogiRec.}\sphinxbfcode{LR\_TYPE\_PROGRAM\_FILE\_HEAD}\sphinxbfcode{ = 100}}
Program file header

\end{fulllineitems}

\index{LR\_TYPE\_PROGRAM\_OVER\_HEAD (in module TotalDepth.LIS.core.LogiRec)}

\begin{fulllineitems}
\phantomsection\label{\detokenize{ref/LIS/core/LogiRec:TotalDepth.LIS.core.LogiRec.LR_TYPE_PROGRAM_OVER_HEAD}}\pysigline{\sphinxcode{TotalDepth.LIS.core.LogiRec.}\sphinxbfcode{LR\_TYPE\_PROGRAM\_OVER\_HEAD}\sphinxbfcode{ = 101}}
Program overlay header

\end{fulllineitems}

\index{LR\_TYPE\_PROGRAM\_OVER\_LOAD (in module TotalDepth.LIS.core.LogiRec)}

\begin{fulllineitems}
\phantomsection\label{\detokenize{ref/LIS/core/LogiRec:TotalDepth.LIS.core.LogiRec.LR_TYPE_PROGRAM_OVER_LOAD}}\pysigline{\sphinxcode{TotalDepth.LIS.core.LogiRec.}\sphinxbfcode{LR\_TYPE\_PROGRAM\_OVER\_LOAD}\sphinxbfcode{ = 102}}
Program overlay load

\end{fulllineitems}

\index{LR\_TYPE\_FILE\_HEAD (in module TotalDepth.LIS.core.LogiRec)}

\begin{fulllineitems}
\phantomsection\label{\detokenize{ref/LIS/core/LogiRec:TotalDepth.LIS.core.LogiRec.LR_TYPE_FILE_HEAD}}\pysigline{\sphinxcode{TotalDepth.LIS.core.LogiRec.}\sphinxbfcode{LR\_TYPE\_FILE\_HEAD}\sphinxbfcode{ = 128}}
File header

\end{fulllineitems}

\index{LR\_TYPE\_FILE\_TAIL (in module TotalDepth.LIS.core.LogiRec)}

\begin{fulllineitems}
\phantomsection\label{\detokenize{ref/LIS/core/LogiRec:TotalDepth.LIS.core.LogiRec.LR_TYPE_FILE_TAIL}}\pysigline{\sphinxcode{TotalDepth.LIS.core.LogiRec.}\sphinxbfcode{LR\_TYPE\_FILE\_TAIL}\sphinxbfcode{ = 129}}
File trailer

\end{fulllineitems}

\index{LR\_TYPE\_TAPE\_HEAD (in module TotalDepth.LIS.core.LogiRec)}

\begin{fulllineitems}
\phantomsection\label{\detokenize{ref/LIS/core/LogiRec:TotalDepth.LIS.core.LogiRec.LR_TYPE_TAPE_HEAD}}\pysigline{\sphinxcode{TotalDepth.LIS.core.LogiRec.}\sphinxbfcode{LR\_TYPE\_TAPE\_HEAD}\sphinxbfcode{ = 130}}
Tape header

\end{fulllineitems}

\index{LR\_TYPE\_TAPE\_TAIL (in module TotalDepth.LIS.core.LogiRec)}

\begin{fulllineitems}
\phantomsection\label{\detokenize{ref/LIS/core/LogiRec:TotalDepth.LIS.core.LogiRec.LR_TYPE_TAPE_TAIL}}\pysigline{\sphinxcode{TotalDepth.LIS.core.LogiRec.}\sphinxbfcode{LR\_TYPE\_TAPE\_TAIL}\sphinxbfcode{ = 131}}
Tape trailer

\end{fulllineitems}

\index{LR\_TYPE\_REEL\_HEAD (in module TotalDepth.LIS.core.LogiRec)}

\begin{fulllineitems}
\phantomsection\label{\detokenize{ref/LIS/core/LogiRec:TotalDepth.LIS.core.LogiRec.LR_TYPE_REEL_HEAD}}\pysigline{\sphinxcode{TotalDepth.LIS.core.LogiRec.}\sphinxbfcode{LR\_TYPE\_REEL\_HEAD}\sphinxbfcode{ = 132}}
Reel header

\end{fulllineitems}

\index{LR\_TYPE\_REEL\_TAIL (in module TotalDepth.LIS.core.LogiRec)}

\begin{fulllineitems}
\phantomsection\label{\detokenize{ref/LIS/core/LogiRec:TotalDepth.LIS.core.LogiRec.LR_TYPE_REEL_TAIL}}\pysigline{\sphinxcode{TotalDepth.LIS.core.LogiRec.}\sphinxbfcode{LR\_TYPE\_REEL\_TAIL}\sphinxbfcode{ = 133}}
Reel trailer

\end{fulllineitems}

\index{LR\_TYPE\_EOF (in module TotalDepth.LIS.core.LogiRec)}

\begin{fulllineitems}
\phantomsection\label{\detokenize{ref/LIS/core/LogiRec:TotalDepth.LIS.core.LogiRec.LR_TYPE_EOF}}\pysigline{\sphinxcode{TotalDepth.LIS.core.LogiRec.}\sphinxbfcode{LR\_TYPE\_EOF}\sphinxbfcode{ = 137}}
Logical EOF (end of file)

\end{fulllineitems}

\index{LR\_TYPE\_BOT (in module TotalDepth.LIS.core.LogiRec)}

\begin{fulllineitems}
\phantomsection\label{\detokenize{ref/LIS/core/LogiRec:TotalDepth.LIS.core.LogiRec.LR_TYPE_BOT}}\pysigline{\sphinxcode{TotalDepth.LIS.core.LogiRec.}\sphinxbfcode{LR\_TYPE\_BOT}\sphinxbfcode{ = 138}}
Logical BOT (beginning of tape)

\end{fulllineitems}

\index{LR\_TYPE\_EOT (in module TotalDepth.LIS.core.LogiRec)}

\begin{fulllineitems}
\phantomsection\label{\detokenize{ref/LIS/core/LogiRec:TotalDepth.LIS.core.LogiRec.LR_TYPE_EOT}}\pysigline{\sphinxcode{TotalDepth.LIS.core.LogiRec.}\sphinxbfcode{LR\_TYPE\_EOT}\sphinxbfcode{ = 139}}
Logical EOT (end of tape)

\end{fulllineitems}

\index{LR\_TYPE\_EOM (in module TotalDepth.LIS.core.LogiRec)}

\begin{fulllineitems}
\phantomsection\label{\detokenize{ref/LIS/core/LogiRec:TotalDepth.LIS.core.LogiRec.LR_TYPE_EOM}}\pysigline{\sphinxcode{TotalDepth.LIS.core.LogiRec.}\sphinxbfcode{LR\_TYPE\_EOM}\sphinxbfcode{ = 141}}
Logical EOM (end of medium)

\end{fulllineitems}

\index{LR\_TYPE\_OPERATOR\_INPUT (in module TotalDepth.LIS.core.LogiRec)}

\begin{fulllineitems}
\phantomsection\label{\detokenize{ref/LIS/core/LogiRec:TotalDepth.LIS.core.LogiRec.LR_TYPE_OPERATOR_INPUT}}\pysigline{\sphinxcode{TotalDepth.LIS.core.LogiRec.}\sphinxbfcode{LR\_TYPE\_OPERATOR\_INPUT}\sphinxbfcode{ = 224}}
Operator command inputs

\end{fulllineitems}

\index{LR\_TYPE\_OPERATOR\_RESPONSE (in module TotalDepth.LIS.core.LogiRec)}

\begin{fulllineitems}
\phantomsection\label{\detokenize{ref/LIS/core/LogiRec:TotalDepth.LIS.core.LogiRec.LR_TYPE_OPERATOR_RESPONSE}}\pysigline{\sphinxcode{TotalDepth.LIS.core.LogiRec.}\sphinxbfcode{LR\_TYPE\_OPERATOR\_RESPONSE}\sphinxbfcode{ = 225}}
Operator response inputs

\end{fulllineitems}

\index{LR\_TYPE\_SYSTEM\_OUTPUT (in module TotalDepth.LIS.core.LogiRec)}

\begin{fulllineitems}
\phantomsection\label{\detokenize{ref/LIS/core/LogiRec:TotalDepth.LIS.core.LogiRec.LR_TYPE_SYSTEM_OUTPUT}}\pysigline{\sphinxcode{TotalDepth.LIS.core.LogiRec.}\sphinxbfcode{LR\_TYPE\_SYSTEM\_OUTPUT}\sphinxbfcode{ = 227}}
System outputs to operator

\end{fulllineitems}

\index{LR\_TYPE\_FLIC\_COMMENT (in module TotalDepth.LIS.core.LogiRec)}

\begin{fulllineitems}
\phantomsection\label{\detokenize{ref/LIS/core/LogiRec:TotalDepth.LIS.core.LogiRec.LR_TYPE_FLIC_COMMENT}}\pysigline{\sphinxcode{TotalDepth.LIS.core.LogiRec.}\sphinxbfcode{LR\_TYPE\_FLIC\_COMMENT}\sphinxbfcode{ = 232}}
FLIC comment

\end{fulllineitems}

\index{LR\_TYPE\_BLANK\_RECORD (in module TotalDepth.LIS.core.LogiRec)}

\begin{fulllineitems}
\phantomsection\label{\detokenize{ref/LIS/core/LogiRec:TotalDepth.LIS.core.LogiRec.LR_TYPE_BLANK_RECORD}}\pysigline{\sphinxcode{TotalDepth.LIS.core.LogiRec.}\sphinxbfcode{LR\_TYPE\_BLANK\_RECORD}\sphinxbfcode{ = 234}}
Blank record/CSU comment

\end{fulllineitems}

\index{LR\_TYPE\_PICTURE (in module TotalDepth.LIS.core.LogiRec)}

\begin{fulllineitems}
\phantomsection\label{\detokenize{ref/LIS/core/LogiRec:TotalDepth.LIS.core.LogiRec.LR_TYPE_PICTURE}}\pysigline{\sphinxcode{TotalDepth.LIS.core.LogiRec.}\sphinxbfcode{LR\_TYPE\_PICTURE}\sphinxbfcode{ = 85}}
Picture

\end{fulllineitems}

\index{LR\_TYPE\_IMAGE (in module TotalDepth.LIS.core.LogiRec)}

\begin{fulllineitems}
\phantomsection\label{\detokenize{ref/LIS/core/LogiRec:TotalDepth.LIS.core.LogiRec.LR_TYPE_IMAGE}}\pysigline{\sphinxcode{TotalDepth.LIS.core.LogiRec.}\sphinxbfcode{LR\_TYPE\_IMAGE}\sphinxbfcode{ = 86}}
Image

\end{fulllineitems}

\index{LR\_TYPE\_ALL (in module TotalDepth.LIS.core.LogiRec)}

\begin{fulllineitems}
\phantomsection\label{\detokenize{ref/LIS/core/LogiRec:TotalDepth.LIS.core.LogiRec.LR_TYPE_ALL}}\pysigline{\sphinxcode{TotalDepth.LIS.core.LogiRec.}\sphinxbfcode{LR\_TYPE\_ALL}\sphinxbfcode{ = (0, 1, 32, 34, 39, 42, 47, 64, 65, 95, 96, 97, 100, 101, 102, 128, 129, 130, 131, 132, 133, 137, 138, 139, 141, 224, 225, 227, 232, 234, 85, 86)}}
All possible Logical Records Types

\end{fulllineitems}

\index{LR\_TYPE\_LOG\_DATA (in module TotalDepth.LIS.core.LogiRec)}

\begin{fulllineitems}
\phantomsection\label{\detokenize{ref/LIS/core/LogiRec:TotalDepth.LIS.core.LogiRec.LR_TYPE_LOG_DATA}}\pysigline{\sphinxcode{TotalDepth.LIS.core.LogiRec.}\sphinxbfcode{LR\_TYPE\_LOG\_DATA}\sphinxbfcode{ = (0, 1)}}
Logical Records Types for Log data

\end{fulllineitems}

\index{LR\_TYPE\_TABLE\_DATA (in module TotalDepth.LIS.core.LogiRec)}

\begin{fulllineitems}
\phantomsection\label{\detokenize{ref/LIS/core/LogiRec:TotalDepth.LIS.core.LogiRec.LR_TYPE_TABLE_DATA}}\pysigline{\sphinxcode{TotalDepth.LIS.core.LogiRec.}\sphinxbfcode{LR\_TYPE\_TABLE\_DATA}\sphinxbfcode{ = (32, 34, 39)}}
Logical Records Types for Table data

\end{fulllineitems}

\index{LR\_TYPE\_DELIMITER\_START (in module TotalDepth.LIS.core.LogiRec)}

\begin{fulllineitems}
\phantomsection\label{\detokenize{ref/LIS/core/LogiRec:TotalDepth.LIS.core.LogiRec.LR_TYPE_DELIMITER_START}}\pysigline{\sphinxcode{TotalDepth.LIS.core.LogiRec.}\sphinxbfcode{LR\_TYPE\_DELIMITER\_START}\sphinxbfcode{ = (132, 130, 128)}}
Logical Records Types for delimiter start records.

\end{fulllineitems}

\index{LR\_TYPE\_DELIMITER\_END (in module TotalDepth.LIS.core.LogiRec)}

\begin{fulllineitems}
\phantomsection\label{\detokenize{ref/LIS/core/LogiRec:TotalDepth.LIS.core.LogiRec.LR_TYPE_DELIMITER_END}}\pysigline{\sphinxcode{TotalDepth.LIS.core.LogiRec.}\sphinxbfcode{LR\_TYPE\_DELIMITER\_END}\sphinxbfcode{ = (133, 131, 129)}}
Logical Records Types for delimiter end records.

\end{fulllineitems}

\index{LR\_TYPE\_DELIMITER (in module TotalDepth.LIS.core.LogiRec)}

\begin{fulllineitems}
\phantomsection\label{\detokenize{ref/LIS/core/LogiRec:TotalDepth.LIS.core.LogiRec.LR_TYPE_DELIMITER}}\pysigline{\sphinxcode{TotalDepth.LIS.core.LogiRec.}\sphinxbfcode{LR\_TYPE\_DELIMITER}\sphinxbfcode{ = (132, 130, 128, 133, 131, 129)}}
Logical Records Types for all delimiter records.
Delimeter records ‘bookend’ dynamic and static data that makes up a LIS file.

\end{fulllineitems}

\index{LR\_TYPE\_MARKER (in module TotalDepth.LIS.core.LogiRec)}

\begin{fulllineitems}
\phantomsection\label{\detokenize{ref/LIS/core/LogiRec:TotalDepth.LIS.core.LogiRec.LR_TYPE_MARKER}}\pysigline{\sphinxcode{TotalDepth.LIS.core.LogiRec.}\sphinxbfcode{LR\_TYPE\_MARKER}\sphinxbfcode{ = (137, 138, 139, 141)}}
Logical Records Types for all marker records.
Marker records terminate original physical media such as a tape.

\end{fulllineitems}

\index{LR\_TYPE\_DELIMITER\_MARKER (in module TotalDepth.LIS.core.LogiRec)}

\begin{fulllineitems}
\phantomsection\label{\detokenize{ref/LIS/core/LogiRec:TotalDepth.LIS.core.LogiRec.LR_TYPE_DELIMITER_MARKER}}\pysigline{\sphinxcode{TotalDepth.LIS.core.LogiRec.}\sphinxbfcode{LR\_TYPE\_DELIMITER\_MARKER}\sphinxbfcode{ = (132, 130, 128, 133, 131, 129, 137, 138, 139, 141)}}
Logical Records Types for all delimiter and marker records.

\end{fulllineitems}

\index{isDelimiter() (in module TotalDepth.LIS.core.LogiRec)}

\begin{fulllineitems}
\phantomsection\label{\detokenize{ref/LIS/core/LogiRec:TotalDepth.LIS.core.LogiRec.isDelimiter}}\pysiglinewithargsret{\sphinxcode{TotalDepth.LIS.core.LogiRec.}\sphinxbfcode{isDelimiter}}{\emph{theType}}{}
Returns True if the Logical Record Type is a Delimiter record.

\end{fulllineitems}

\index{LR\_TYPE\_UNKNOWN\_INTERNAL\_FORMAT (in module TotalDepth.LIS.core.LogiRec)}

\begin{fulllineitems}
\phantomsection\label{\detokenize{ref/LIS/core/LogiRec:TotalDepth.LIS.core.LogiRec.LR_TYPE_UNKNOWN_INTERNAL_FORMAT}}\pysigline{\sphinxcode{TotalDepth.LIS.core.LogiRec.}\sphinxbfcode{LR\_TYPE\_UNKNOWN\_INTERNAL\_FORMAT}\sphinxbfcode{ = (42, 47, 95, 96, 97, 100, 101, 102, 224, 225, 227, 232, 234, 85, 86)}}
Logical Records Types with no known format so just treat these as unformatted binary data

\end{fulllineitems}

\index{LR\_DESCRIPTION\_MAP (in module TotalDepth.LIS.core.LogiRec)}

\begin{fulllineitems}
\phantomsection\label{\detokenize{ref/LIS/core/LogiRec:TotalDepth.LIS.core.LogiRec.LR_DESCRIPTION_MAP}}\pysigline{\sphinxcode{TotalDepth.LIS.core.LogiRec.}\sphinxbfcode{LR\_DESCRIPTION\_MAP}\sphinxbfcode{ = \{0: 'Normal data record containing log data', 1: 'Alternate data.', 32: 'Job identification', 34: 'Well site data', 39: 'Tool string info', 42: 'Encrypted table dump', 47: 'Table dump', 64: 'Data format specification record', 65: 'Data descriptor (not defined in the LIS79 Description Reference Manual)', 95: 'TU10 software boot', 96: 'Bootstrap loader', 97: 'CP-kernel loader boot', 100: 'Program file header', 101: 'Program overlay header', 102: 'Program overlay load', 128: 'File header', 129: 'File trailer', 130: 'Tape header', 131: 'Tape trailer', 132: 'Reel header', 133: 'Reel trailer', 137: 'Logical EOF (end of file)', 138: 'Logical BOT (beginning of tape)', 139: 'Logical EOT (end of tape)', 141: 'Logical EOM (end of medium)', 224: 'Operator command inputs', 225: 'Operator response inputs', 227: 'System outputs to operator', 232: 'FLIC comment', 234: 'Blank record/CSU comment', 85: 'Picture', 86: 'Image'\}}}
Map of \{Logical Records Type : description, …\}

\end{fulllineitems}

\index{LR\_DESCRIPTION\_UNKNOWN (in module TotalDepth.LIS.core.LogiRec)}

\begin{fulllineitems}
\phantomsection\label{\detokenize{ref/LIS/core/LogiRec:TotalDepth.LIS.core.LogiRec.LR_DESCRIPTION_UNKNOWN}}\pysigline{\sphinxcode{TotalDepth.LIS.core.LogiRec.}\sphinxbfcode{LR\_DESCRIPTION\_UNKNOWN}\sphinxbfcode{ = 'Unknown Logical Record type.'}}
Description string for unknown Logical Records Type

\end{fulllineitems}

\index{STRUCT\_LR\_HEAD (in module TotalDepth.LIS.core.LogiRec)}

\begin{fulllineitems}
\phantomsection\label{\detokenize{ref/LIS/core/LogiRec:TotalDepth.LIS.core.LogiRec.STRUCT_LR_HEAD}}\pysigline{\sphinxcode{TotalDepth.LIS.core.LogiRec.}\sphinxbfcode{STRUCT\_LR\_HEAD}\sphinxbfcode{ = \textless{}Struct object\textgreater{}}}
Logical Record header (type and attributes)

\end{fulllineitems}

\index{STRUCT\_LR\_FILE\_HEAD\_TAIL (in module TotalDepth.LIS.core.LogiRec)}

\begin{fulllineitems}
\phantomsection\label{\detokenize{ref/LIS/core/LogiRec:TotalDepth.LIS.core.LogiRec.STRUCT_LR_FILE_HEAD_TAIL}}\pysigline{\sphinxcode{TotalDepth.LIS.core.LogiRec.}\sphinxbfcode{STRUCT\_LR\_FILE\_HEAD\_TAIL}\sphinxbfcode{ = \textless{}Struct object\textgreater{}}}
Logical Record field interpretation via the struct module

\end{fulllineitems}

\index{STRUCT\_LR\_REEL\_TAPE\_HEAD\_TAIL (in module TotalDepth.LIS.core.LogiRec)}

\begin{fulllineitems}
\phantomsection\label{\detokenize{ref/LIS/core/LogiRec:TotalDepth.LIS.core.LogiRec.STRUCT_LR_REEL_TAPE_HEAD_TAIL}}\pysigline{\sphinxcode{TotalDepth.LIS.core.LogiRec.}\sphinxbfcode{STRUCT\_LR\_REEL\_TAPE\_HEAD\_TAIL}\sphinxbfcode{ = \textless{}Struct object\textgreater{}}}
Logical Record reel/tape head/tail via the struct module

\end{fulllineitems}

\index{STRUCT\_COMPONENT\_BLOCK\_PREAMBLE (in module TotalDepth.LIS.core.LogiRec)}

\begin{fulllineitems}
\phantomsection\label{\detokenize{ref/LIS/core/LogiRec:TotalDepth.LIS.core.LogiRec.STRUCT_COMPONENT_BLOCK_PREAMBLE}}\pysigline{\sphinxcode{TotalDepth.LIS.core.LogiRec.}\sphinxbfcode{STRUCT\_COMPONENT\_BLOCK\_PREAMBLE}\sphinxbfcode{ = \textless{}Struct object\textgreater{}}}
Component Block preamble as a struct.Struct()

\end{fulllineitems}

\index{STRUCT\_ENTRY\_BLOCK\_PREAMBLE (in module TotalDepth.LIS.core.LogiRec)}

\begin{fulllineitems}
\phantomsection\label{\detokenize{ref/LIS/core/LogiRec:TotalDepth.LIS.core.LogiRec.STRUCT_ENTRY_BLOCK_PREAMBLE}}\pysigline{\sphinxcode{TotalDepth.LIS.core.LogiRec.}\sphinxbfcode{STRUCT\_ENTRY\_BLOCK\_PREAMBLE}\sphinxbfcode{ = \textless{}Struct object\textgreater{}}}
Entry Block preamble as a struct.Struct()

\end{fulllineitems}

\index{STRUCT\_DSB (in module TotalDepth.LIS.core.LogiRec)}

\begin{fulllineitems}
\phantomsection\label{\detokenize{ref/LIS/core/LogiRec:TotalDepth.LIS.core.LogiRec.STRUCT_DSB}}\pysigline{\sphinxcode{TotalDepth.LIS.core.LogiRec.}\sphinxbfcode{STRUCT\_DSB}\sphinxbfcode{ = \textless{}Struct object\textgreater{}}}
Datum Specification Block structure.
NOTE: Due to the funny way API codes are done we read then as a single
32 bit unsigned integer and then do a decimal masking operation to
extract the four sub-fields

\end{fulllineitems}

\index{LrBase (class in TotalDepth.LIS.core.LogiRec)}

\begin{fulllineitems}
\phantomsection\label{\detokenize{ref/LIS/core/LogiRec:TotalDepth.LIS.core.LogiRec.LrBase}}\pysiglinewithargsret{\sphinxbfcode{class }\sphinxcode{TotalDepth.LIS.core.LogiRec.}\sphinxbfcode{LrBase}}{\emph{theType}, \emph{theAttr}}{}
Base class for Logical Records.
Constructed with and integer type and integer attributes.
\index{\_\_init\_\_() (TotalDepth.LIS.core.LogiRec.LrBase method)}

\begin{fulllineitems}
\phantomsection\label{\detokenize{ref/LIS/core/LogiRec:TotalDepth.LIS.core.LogiRec.LrBase.__init__}}\pysiglinewithargsret{\sphinxbfcode{\_\_init\_\_}}{\emph{theType}, \emph{theAttr}}{}
Base class constructor, theType and theAttr are bytes.

\end{fulllineitems}

\index{init() (TotalDepth.LIS.core.LogiRec.LrBase method)}

\begin{fulllineitems}
\phantomsection\label{\detokenize{ref/LIS/core/LogiRec:TotalDepth.LIS.core.LogiRec.LrBase.init}}\pysiglinewithargsret{\sphinxbfcode{init}}{\emph{theLen}}{}
Returns a string of spaces of the supplied length.

\end{fulllineitems}

\index{\_\_str\_\_() (TotalDepth.LIS.core.LogiRec.LrBase method)}

\begin{fulllineitems}
\phantomsection\label{\detokenize{ref/LIS/core/LogiRec:TotalDepth.LIS.core.LogiRec.LrBase.__str__}}\pysiglinewithargsret{\sphinxbfcode{\_\_str\_\_}}{}{}
String representation.

\end{fulllineitems}

\index{desc (TotalDepth.LIS.core.LogiRec.LrBase attribute)}

\begin{fulllineitems}
\phantomsection\label{\detokenize{ref/LIS/core/LogiRec:TotalDepth.LIS.core.LogiRec.LrBase.desc}}\pysigline{\sphinxbfcode{desc}}
Description ot the LR type.

\end{fulllineitems}

\index{\_\_weakref\_\_ (TotalDepth.LIS.core.LogiRec.LrBase attribute)}

\begin{fulllineitems}
\phantomsection\label{\detokenize{ref/LIS/core/LogiRec:TotalDepth.LIS.core.LogiRec.LrBase.__weakref__}}\pysigline{\sphinxbfcode{\_\_weakref\_\_}}
list of weak references to the object (if defined)

\end{fulllineitems}


\end{fulllineitems}

\index{LrMarker (class in TotalDepth.LIS.core.LogiRec)}

\begin{fulllineitems}
\phantomsection\label{\detokenize{ref/LIS/core/LogiRec:TotalDepth.LIS.core.LogiRec.LrMarker}}\pysiglinewithargsret{\sphinxbfcode{class }\sphinxcode{TotalDepth.LIS.core.LogiRec.}\sphinxbfcode{LrMarker}}{\emph{theType}, \emph{theAttr}}{}
A marker record such as EOF BOT EOT EOM.

\end{fulllineitems}

\index{LrEOF (class in TotalDepth.LIS.core.LogiRec)}

\begin{fulllineitems}
\phantomsection\label{\detokenize{ref/LIS/core/LogiRec:TotalDepth.LIS.core.LogiRec.LrEOF}}\pysiglinewithargsret{\sphinxbfcode{class }\sphinxcode{TotalDepth.LIS.core.LogiRec.}\sphinxbfcode{LrEOF}}{\emph{attr=0}}{}
A EOF marker record.

\end{fulllineitems}

\index{LrEOFRead (class in TotalDepth.LIS.core.LogiRec)}

\begin{fulllineitems}
\phantomsection\label{\detokenize{ref/LIS/core/LogiRec:TotalDepth.LIS.core.LogiRec.LrEOFRead}}\pysiglinewithargsret{\sphinxbfcode{class }\sphinxcode{TotalDepth.LIS.core.LogiRec.}\sphinxbfcode{LrEOFRead}}{\emph{theFile}}{}
A EOF marker record read from a file.

\end{fulllineitems}

\index{LrBOT (class in TotalDepth.LIS.core.LogiRec)}

\begin{fulllineitems}
\phantomsection\label{\detokenize{ref/LIS/core/LogiRec:TotalDepth.LIS.core.LogiRec.LrBOT}}\pysiglinewithargsret{\sphinxbfcode{class }\sphinxcode{TotalDepth.LIS.core.LogiRec.}\sphinxbfcode{LrBOT}}{\emph{attr=0}}{}
A BOT marker record.

\end{fulllineitems}

\index{LrBOTRead (class in TotalDepth.LIS.core.LogiRec)}

\begin{fulllineitems}
\phantomsection\label{\detokenize{ref/LIS/core/LogiRec:TotalDepth.LIS.core.LogiRec.LrBOTRead}}\pysiglinewithargsret{\sphinxbfcode{class }\sphinxcode{TotalDepth.LIS.core.LogiRec.}\sphinxbfcode{LrBOTRead}}{\emph{theFile}}{}
A BOT marker record read from a file.

\end{fulllineitems}

\index{LrEOT (class in TotalDepth.LIS.core.LogiRec)}

\begin{fulllineitems}
\phantomsection\label{\detokenize{ref/LIS/core/LogiRec:TotalDepth.LIS.core.LogiRec.LrEOT}}\pysiglinewithargsret{\sphinxbfcode{class }\sphinxcode{TotalDepth.LIS.core.LogiRec.}\sphinxbfcode{LrEOT}}{\emph{attr=0}}{}
A EOT marker record.

\end{fulllineitems}

\index{LrEOTRead (class in TotalDepth.LIS.core.LogiRec)}

\begin{fulllineitems}
\phantomsection\label{\detokenize{ref/LIS/core/LogiRec:TotalDepth.LIS.core.LogiRec.LrEOTRead}}\pysiglinewithargsret{\sphinxbfcode{class }\sphinxcode{TotalDepth.LIS.core.LogiRec.}\sphinxbfcode{LrEOTRead}}{\emph{theFile}}{}
A EOT marker record read from a file.

\end{fulllineitems}

\index{LrEOM (class in TotalDepth.LIS.core.LogiRec)}

\begin{fulllineitems}
\phantomsection\label{\detokenize{ref/LIS/core/LogiRec:TotalDepth.LIS.core.LogiRec.LrEOM}}\pysiglinewithargsret{\sphinxbfcode{class }\sphinxcode{TotalDepth.LIS.core.LogiRec.}\sphinxbfcode{LrEOM}}{\emph{attr=0}}{}
A EOM marker record.

\end{fulllineitems}

\index{LrEOMRead (class in TotalDepth.LIS.core.LogiRec)}

\begin{fulllineitems}
\phantomsection\label{\detokenize{ref/LIS/core/LogiRec:TotalDepth.LIS.core.LogiRec.LrEOMRead}}\pysiglinewithargsret{\sphinxbfcode{class }\sphinxcode{TotalDepth.LIS.core.LogiRec.}\sphinxbfcode{LrEOMRead}}{\emph{theFile}}{}
A EOM marker record read from a file.

\end{fulllineitems}

\index{LrWithDateField (class in TotalDepth.LIS.core.LogiRec)}

\begin{fulllineitems}
\phantomsection\label{\detokenize{ref/LIS/core/LogiRec:TotalDepth.LIS.core.LogiRec.LrWithDateField}}\pysiglinewithargsret{\sphinxbfcode{class }\sphinxcode{TotalDepth.LIS.core.LogiRec.}\sphinxbfcode{LrWithDateField}}{\emph{theType}, \emph{theAttr}}{}
ABC for classes that have the YY/MM/DD date field.
\index{ymd (TotalDepth.LIS.core.LogiRec.LrWithDateField attribute)}

\begin{fulllineitems}
\phantomsection\label{\detokenize{ref/LIS/core/LogiRec:TotalDepth.LIS.core.LogiRec.LrWithDateField.ymd}}\pysigline{\sphinxbfcode{ymd}}
Returns the YY/MM/DD date field to a year, month, day tuple or None.

\end{fulllineitems}


\end{fulllineitems}

\index{LrFileHeadTail (class in TotalDepth.LIS.core.LogiRec)}

\begin{fulllineitems}
\phantomsection\label{\detokenize{ref/LIS/core/LogiRec:TotalDepth.LIS.core.LogiRec.LrFileHeadTail}}\pysiglinewithargsret{\sphinxbfcode{class }\sphinxcode{TotalDepth.LIS.core.LogiRec.}\sphinxbfcode{LrFileHeadTail}}{\emph{theType}, \emph{theAttr}}{}
Parent class of FileHead, FileTail that have identical structure.
\index{read() (TotalDepth.LIS.core.LogiRec.LrFileHeadTail method)}

\begin{fulllineitems}
\phantomsection\label{\detokenize{ref/LIS/core/LogiRec:TotalDepth.LIS.core.LogiRec.LrFileHeadTail.read}}\pysiglinewithargsret{\sphinxbfcode{read}}{\emph{theFile}}{}
Read from a LIS physical file.

\end{fulllineitems}

\index{contFileName (TotalDepth.LIS.core.LogiRec.LrFileHeadTail attribute)}

\begin{fulllineitems}
\phantomsection\label{\detokenize{ref/LIS/core/LogiRec:TotalDepth.LIS.core.LogiRec.LrFileHeadTail.contFileName}}\pysigline{\sphinxbfcode{contFileName}}
Continuation file name.

\end{fulllineitems}


\end{fulllineitems}

\index{LrFileHead (class in TotalDepth.LIS.core.LogiRec)}

\begin{fulllineitems}
\phantomsection\label{\detokenize{ref/LIS/core/LogiRec:TotalDepth.LIS.core.LogiRec.LrFileHead}}\pysiglinewithargsret{\sphinxbfcode{class }\sphinxcode{TotalDepth.LIS.core.LogiRec.}\sphinxbfcode{LrFileHead}}{\emph{theType}, \emph{theAttr}}{}
Specific class of File Head.
\index{prevFileName (TotalDepth.LIS.core.LogiRec.LrFileHead attribute)}

\begin{fulllineitems}
\phantomsection\label{\detokenize{ref/LIS/core/LogiRec:TotalDepth.LIS.core.LogiRec.LrFileHead.prevFileName}}\pysigline{\sphinxbfcode{prevFileName}}
Previous file name.

\end{fulllineitems}


\end{fulllineitems}

\index{LrFileHeadRead (class in TotalDepth.LIS.core.LogiRec)}

\begin{fulllineitems}
\phantomsection\label{\detokenize{ref/LIS/core/LogiRec:TotalDepth.LIS.core.LogiRec.LrFileHeadRead}}\pysiglinewithargsret{\sphinxbfcode{class }\sphinxcode{TotalDepth.LIS.core.LogiRec.}\sphinxbfcode{LrFileHeadRead}}{\emph{theFile}}{}
Specific class of File head read from a file.

\end{fulllineitems}

\index{LrFileTail (class in TotalDepth.LIS.core.LogiRec)}

\begin{fulllineitems}
\phantomsection\label{\detokenize{ref/LIS/core/LogiRec:TotalDepth.LIS.core.LogiRec.LrFileTail}}\pysiglinewithargsret{\sphinxbfcode{class }\sphinxcode{TotalDepth.LIS.core.LogiRec.}\sphinxbfcode{LrFileTail}}{\emph{theType}, \emph{theAttr}}{}
Specific class of File Tail.
\index{nextFileName (TotalDepth.LIS.core.LogiRec.LrFileTail attribute)}

\begin{fulllineitems}
\phantomsection\label{\detokenize{ref/LIS/core/LogiRec:TotalDepth.LIS.core.LogiRec.LrFileTail.nextFileName}}\pysigline{\sphinxbfcode{nextFileName}}
Next file name.

\end{fulllineitems}


\end{fulllineitems}

\index{LrFileTailRead (class in TotalDepth.LIS.core.LogiRec)}

\begin{fulllineitems}
\phantomsection\label{\detokenize{ref/LIS/core/LogiRec:TotalDepth.LIS.core.LogiRec.LrFileTailRead}}\pysiglinewithargsret{\sphinxbfcode{class }\sphinxcode{TotalDepth.LIS.core.LogiRec.}\sphinxbfcode{LrFileTailRead}}{\emph{theFile}}{}
Specific class of File tail read from a file.

\end{fulllineitems}

\index{LrReelTapeHeadTail (class in TotalDepth.LIS.core.LogiRec)}

\begin{fulllineitems}
\phantomsection\label{\detokenize{ref/LIS/core/LogiRec:TotalDepth.LIS.core.LogiRec.LrReelTapeHeadTail}}\pysiglinewithargsret{\sphinxbfcode{class }\sphinxcode{TotalDepth.LIS.core.LogiRec.}\sphinxbfcode{LrReelTapeHeadTail}}{\emph{theType}, \emph{theAttr}}{}
Parent class of Reel/Tape Head/Tail that have identical structure.

\end{fulllineitems}

\index{LrTapeHeadTail (class in TotalDepth.LIS.core.LogiRec)}

\begin{fulllineitems}
\phantomsection\label{\detokenize{ref/LIS/core/LogiRec:TotalDepth.LIS.core.LogiRec.LrTapeHeadTail}}\pysiglinewithargsret{\sphinxbfcode{class }\sphinxcode{TotalDepth.LIS.core.LogiRec.}\sphinxbfcode{LrTapeHeadTail}}{\emph{theType}, \emph{theAttr}}{}
Tape head or tail Logical Record.

\end{fulllineitems}

\index{LrTapeHead (class in TotalDepth.LIS.core.LogiRec)}

\begin{fulllineitems}
\phantomsection\label{\detokenize{ref/LIS/core/LogiRec:TotalDepth.LIS.core.LogiRec.LrTapeHead}}\pysiglinewithargsret{\sphinxbfcode{class }\sphinxcode{TotalDepth.LIS.core.LogiRec.}\sphinxbfcode{LrTapeHead}}{\emph{theType}, \emph{theAttr}}{}
Tape head Logical Record.
\index{prevTapeName (TotalDepth.LIS.core.LogiRec.LrTapeHead attribute)}

\begin{fulllineitems}
\phantomsection\label{\detokenize{ref/LIS/core/LogiRec:TotalDepth.LIS.core.LogiRec.LrTapeHead.prevTapeName}}\pysigline{\sphinxbfcode{prevTapeName}}
Previous tape name.

\end{fulllineitems}


\end{fulllineitems}

\index{LrTapeHeadRead (class in TotalDepth.LIS.core.LogiRec)}

\begin{fulllineitems}
\phantomsection\label{\detokenize{ref/LIS/core/LogiRec:TotalDepth.LIS.core.LogiRec.LrTapeHeadRead}}\pysiglinewithargsret{\sphinxbfcode{class }\sphinxcode{TotalDepth.LIS.core.LogiRec.}\sphinxbfcode{LrTapeHeadRead}}{\emph{theFile}}{}
Specific class of Tape head read from a file.

\end{fulllineitems}

\index{LrTapeTail (class in TotalDepth.LIS.core.LogiRec)}

\begin{fulllineitems}
\phantomsection\label{\detokenize{ref/LIS/core/LogiRec:TotalDepth.LIS.core.LogiRec.LrTapeTail}}\pysiglinewithargsret{\sphinxbfcode{class }\sphinxcode{TotalDepth.LIS.core.LogiRec.}\sphinxbfcode{LrTapeTail}}{\emph{theType}, \emph{theAttr}}{}
Tape tail Logical Record.
\index{nextTapeName (TotalDepth.LIS.core.LogiRec.LrTapeTail attribute)}

\begin{fulllineitems}
\phantomsection\label{\detokenize{ref/LIS/core/LogiRec:TotalDepth.LIS.core.LogiRec.LrTapeTail.nextTapeName}}\pysigline{\sphinxbfcode{nextTapeName}}
Next tape name.

\end{fulllineitems}


\end{fulllineitems}

\index{LrTapeTailRead (class in TotalDepth.LIS.core.LogiRec)}

\begin{fulllineitems}
\phantomsection\label{\detokenize{ref/LIS/core/LogiRec:TotalDepth.LIS.core.LogiRec.LrTapeTailRead}}\pysiglinewithargsret{\sphinxbfcode{class }\sphinxcode{TotalDepth.LIS.core.LogiRec.}\sphinxbfcode{LrTapeTailRead}}{\emph{theFile}}{}
Specific class of Tape tail read from a file.

\end{fulllineitems}

\index{LrReelHeadTail (class in TotalDepth.LIS.core.LogiRec)}

\begin{fulllineitems}
\phantomsection\label{\detokenize{ref/LIS/core/LogiRec:TotalDepth.LIS.core.LogiRec.LrReelHeadTail}}\pysiglinewithargsret{\sphinxbfcode{class }\sphinxcode{TotalDepth.LIS.core.LogiRec.}\sphinxbfcode{LrReelHeadTail}}{\emph{theType}, \emph{theAttr}}{}
Reel head or tail Logical Record.

\end{fulllineitems}

\index{LrReelHead (class in TotalDepth.LIS.core.LogiRec)}

\begin{fulllineitems}
\phantomsection\label{\detokenize{ref/LIS/core/LogiRec:TotalDepth.LIS.core.LogiRec.LrReelHead}}\pysiglinewithargsret{\sphinxbfcode{class }\sphinxcode{TotalDepth.LIS.core.LogiRec.}\sphinxbfcode{LrReelHead}}{\emph{theType}, \emph{theAttr}}{}
Reel head Logical Record.
\index{prevReelName (TotalDepth.LIS.core.LogiRec.LrReelHead attribute)}

\begin{fulllineitems}
\phantomsection\label{\detokenize{ref/LIS/core/LogiRec:TotalDepth.LIS.core.LogiRec.LrReelHead.prevReelName}}\pysigline{\sphinxbfcode{prevReelName}}
Previous reel name.

\end{fulllineitems}


\end{fulllineitems}

\index{LrReelHeadRead (class in TotalDepth.LIS.core.LogiRec)}

\begin{fulllineitems}
\phantomsection\label{\detokenize{ref/LIS/core/LogiRec:TotalDepth.LIS.core.LogiRec.LrReelHeadRead}}\pysiglinewithargsret{\sphinxbfcode{class }\sphinxcode{TotalDepth.LIS.core.LogiRec.}\sphinxbfcode{LrReelHeadRead}}{\emph{theFile}}{}
Specific class of Reel head read from a file.

\end{fulllineitems}

\index{LrReelTail (class in TotalDepth.LIS.core.LogiRec)}

\begin{fulllineitems}
\phantomsection\label{\detokenize{ref/LIS/core/LogiRec:TotalDepth.LIS.core.LogiRec.LrReelTail}}\pysiglinewithargsret{\sphinxbfcode{class }\sphinxcode{TotalDepth.LIS.core.LogiRec.}\sphinxbfcode{LrReelTail}}{\emph{theType}, \emph{theAttr}}{}
Reel tail Logical Record.
\index{nextReelName (TotalDepth.LIS.core.LogiRec.LrReelTail attribute)}

\begin{fulllineitems}
\phantomsection\label{\detokenize{ref/LIS/core/LogiRec:TotalDepth.LIS.core.LogiRec.LrReelTail.nextReelName}}\pysigline{\sphinxbfcode{nextReelName}}
Next reel name.

\end{fulllineitems}


\end{fulllineitems}

\index{LrReelTailRead (class in TotalDepth.LIS.core.LogiRec)}

\begin{fulllineitems}
\phantomsection\label{\detokenize{ref/LIS/core/LogiRec:TotalDepth.LIS.core.LogiRec.LrReelTailRead}}\pysiglinewithargsret{\sphinxbfcode{class }\sphinxcode{TotalDepth.LIS.core.LogiRec.}\sphinxbfcode{LrReelTailRead}}{\emph{theFile}}{}
Specific class of Reel tail read from a file.

\end{fulllineitems}

\index{LrMisc (class in TotalDepth.LIS.core.LogiRec)}

\begin{fulllineitems}
\phantomsection\label{\detokenize{ref/LIS/core/LogiRec:TotalDepth.LIS.core.LogiRec.LrMisc}}\pysiglinewithargsret{\sphinxbfcode{class }\sphinxcode{TotalDepth.LIS.core.LogiRec.}\sphinxbfcode{LrMisc}}{\emph{theType}, \emph{theAttr}}{}
Miscellaneous Logical Record.

\end{fulllineitems}

\index{LrMiscRead (class in TotalDepth.LIS.core.LogiRec)}

\begin{fulllineitems}
\phantomsection\label{\detokenize{ref/LIS/core/LogiRec:TotalDepth.LIS.core.LogiRec.LrMiscRead}}\pysiglinewithargsret{\sphinxbfcode{class }\sphinxcode{TotalDepth.LIS.core.LogiRec.}\sphinxbfcode{LrMiscRead}}{\emph{theFile}}{}
Miscellaneous Logical Record read from a LIS file.

\end{fulllineitems}

\index{COMPONENT\_BLOCK\_TABLE (in module TotalDepth.LIS.core.LogiRec)}

\begin{fulllineitems}
\phantomsection\label{\detokenize{ref/LIS/core/LogiRec:TotalDepth.LIS.core.LogiRec.COMPONENT_BLOCK_TABLE}}\pysigline{\sphinxcode{TotalDepth.LIS.core.LogiRec.}\sphinxbfcode{COMPONENT\_BLOCK\_TABLE}\sphinxbfcode{ = 73}}
Type of first Component Blocks in the table

\end{fulllineitems}

\index{COMPONENT\_BLOCK\_DATUM\_BLOCK\_START (in module TotalDepth.LIS.core.LogiRec)}

\begin{fulllineitems}
\phantomsection\label{\detokenize{ref/LIS/core/LogiRec:TotalDepth.LIS.core.LogiRec.COMPONENT_BLOCK_DATUM_BLOCK_START}}\pysigline{\sphinxcode{TotalDepth.LIS.core.LogiRec.}\sphinxbfcode{COMPONENT\_BLOCK\_DATUM\_BLOCK\_START}\sphinxbfcode{ = 0}}
Type of first Component Blocks in the Datum Block i.e. row

\end{fulllineitems}

\index{COMPONENT\_BLOCK\_DATUM\_BLOCK\_ENTRY (in module TotalDepth.LIS.core.LogiRec)}

\begin{fulllineitems}
\phantomsection\label{\detokenize{ref/LIS/core/LogiRec:TotalDepth.LIS.core.LogiRec.COMPONENT_BLOCK_DATUM_BLOCK_ENTRY}}\pysigline{\sphinxcode{TotalDepth.LIS.core.LogiRec.}\sphinxbfcode{COMPONENT\_BLOCK\_DATUM\_BLOCK\_ENTRY}\sphinxbfcode{ = 69}}
Component Block type that describes an entry in a Datum Block i.e. a cell

\end{fulllineitems}

\index{ExceptionLrTable}

\begin{fulllineitems}
\phantomsection\label{\detokenize{ref/LIS/core/LogiRec:TotalDepth.LIS.core.LogiRec.ExceptionLrTable}}\pysigline{\sphinxbfcode{exception }\sphinxcode{TotalDepth.LIS.core.LogiRec.}\sphinxbfcode{ExceptionLrTable}}
Specialisation of exception for Table Logical Records.

\end{fulllineitems}

\index{ExceptionLrTableInit}

\begin{fulllineitems}
\phantomsection\label{\detokenize{ref/LIS/core/LogiRec:TotalDepth.LIS.core.LogiRec.ExceptionLrTableInit}}\pysigline{\sphinxbfcode{exception }\sphinxcode{TotalDepth.LIS.core.LogiRec.}\sphinxbfcode{ExceptionLrTableInit}}
Table \_\_init\_\_() issues.

\end{fulllineitems}

\index{ExceptionLrTableInternaStructuresCorrupt}

\begin{fulllineitems}
\phantomsection\label{\detokenize{ref/LIS/core/LogiRec:TotalDepth.LIS.core.LogiRec.ExceptionLrTableInternaStructuresCorrupt}}\pysigline{\sphinxbfcode{exception }\sphinxcode{TotalDepth.LIS.core.LogiRec.}\sphinxbfcode{ExceptionLrTableInternaStructuresCorrupt}}
Raised when there are inconsistencies with the IR of the table.

\end{fulllineitems}

\index{ExceptionLrTableCompose}

\begin{fulllineitems}
\phantomsection\label{\detokenize{ref/LIS/core/LogiRec:TotalDepth.LIS.core.LogiRec.ExceptionLrTableCompose}}\pysigline{\sphinxbfcode{exception }\sphinxcode{TotalDepth.LIS.core.LogiRec.}\sphinxbfcode{ExceptionLrTableCompose}}
Table creation (not from file) issues.

\end{fulllineitems}

\index{ExceptionLrTableRow}

\begin{fulllineitems}
\phantomsection\label{\detokenize{ref/LIS/core/LogiRec:TotalDepth.LIS.core.LogiRec.ExceptionLrTableRow}}\pysigline{\sphinxbfcode{exception }\sphinxcode{TotalDepth.LIS.core.LogiRec.}\sphinxbfcode{ExceptionLrTableRow}}
TableRow issues.

\end{fulllineitems}

\index{ExceptionLrTableRowInit}

\begin{fulllineitems}
\phantomsection\label{\detokenize{ref/LIS/core/LogiRec:TotalDepth.LIS.core.LogiRec.ExceptionLrTableRowInit}}\pysigline{\sphinxbfcode{exception }\sphinxcode{TotalDepth.LIS.core.LogiRec.}\sphinxbfcode{ExceptionLrTableRowInit}}
TableRow \_\_init\_\_() issues.

\end{fulllineitems}

\index{ExceptionCbEngValInit}

\begin{fulllineitems}
\phantomsection\label{\detokenize{ref/LIS/core/LogiRec:TotalDepth.LIS.core.LogiRec.ExceptionCbEngValInit}}\pysigline{\sphinxbfcode{exception }\sphinxcode{TotalDepth.LIS.core.LogiRec.}\sphinxbfcode{ExceptionCbEngValInit}}
CbEngVal.\_\_init\_\_() issues such as unknown rep code.

\end{fulllineitems}

\index{CbEngVal (class in TotalDepth.LIS.core.LogiRec)}

\begin{fulllineitems}
\phantomsection\label{\detokenize{ref/LIS/core/LogiRec:TotalDepth.LIS.core.LogiRec.CbEngVal}}\pysigline{\sphinxbfcode{class }\sphinxcode{TotalDepth.LIS.core.LogiRec.}\sphinxbfcode{CbEngVal}}
Contains the data from a Component Block and has an EngVal
\index{CB\_TYPES (TotalDepth.LIS.core.LogiRec.CbEngVal attribute)}

\begin{fulllineitems}
\phantomsection\label{\detokenize{ref/LIS/core/LogiRec:TotalDepth.LIS.core.LogiRec.CbEngVal.CB_TYPES}}\pysigline{\sphinxbfcode{CB\_TYPES}\sphinxbfcode{ = (73, 0, 69)}}
Allowable Component Block types

\end{fulllineitems}

\index{setValue() (TotalDepth.LIS.core.LogiRec.CbEngVal method)}

\begin{fulllineitems}
\phantomsection\label{\detokenize{ref/LIS/core/LogiRec:TotalDepth.LIS.core.LogiRec.CbEngVal.setValue}}\pysiglinewithargsret{\sphinxbfcode{setValue}}{\emph{v}}{}
Sets the value to v.

\end{fulllineitems}

\index{lisBytes() (TotalDepth.LIS.core.LogiRec.CbEngVal method)}

\begin{fulllineitems}
\phantomsection\label{\detokenize{ref/LIS/core/LogiRec:TotalDepth.LIS.core.LogiRec.CbEngVal.lisBytes}}\pysiglinewithargsret{\sphinxbfcode{lisBytes}}{}{}
Return a bytes() array from the internal representation.

\end{fulllineitems}

\index{value (TotalDepth.LIS.core.LogiRec.CbEngVal attribute)}

\begin{fulllineitems}
\phantomsection\label{\detokenize{ref/LIS/core/LogiRec:TotalDepth.LIS.core.LogiRec.CbEngVal.value}}\pysigline{\sphinxbfcode{value}}
The value of the Component Block or None.

\end{fulllineitems}

\index{status (TotalDepth.LIS.core.LogiRec.CbEngVal attribute)}

\begin{fulllineitems}
\phantomsection\label{\detokenize{ref/LIS/core/LogiRec:TotalDepth.LIS.core.LogiRec.CbEngVal.status}}\pysigline{\sphinxbfcode{status}}
Returns True if the value is b’ALLO’, False otherwise.

\end{fulllineitems}

\index{\_\_weakref\_\_ (TotalDepth.LIS.core.LogiRec.CbEngVal attribute)}

\begin{fulllineitems}
\phantomsection\label{\detokenize{ref/LIS/core/LogiRec:TotalDepth.LIS.core.LogiRec.CbEngVal.__weakref__}}\pysigline{\sphinxbfcode{\_\_weakref\_\_}}
list of weak references to the object (if defined)

\end{fulllineitems}


\end{fulllineitems}

\index{CbEngValRead (class in TotalDepth.LIS.core.LogiRec)}

\begin{fulllineitems}
\phantomsection\label{\detokenize{ref/LIS/core/LogiRec:TotalDepth.LIS.core.LogiRec.CbEngValRead}}\pysiglinewithargsret{\sphinxbfcode{class }\sphinxcode{TotalDepth.LIS.core.LogiRec.}\sphinxbfcode{CbEngValRead}}{\emph{theFile}}{}
Contains the data from a Component Block and has an EngVal read from a file.
\index{\_\_init\_\_() (TotalDepth.LIS.core.LogiRec.CbEngValRead method)}

\begin{fulllineitems}
\phantomsection\label{\detokenize{ref/LIS/core/LogiRec:TotalDepth.LIS.core.LogiRec.CbEngValRead.__init__}}\pysiglinewithargsret{\sphinxbfcode{\_\_init\_\_}}{\emph{theFile}}{}
Initialise. This will raise a TypeError if theFile.unpack returns None
i.e. when not enough data to create a Component Block.

\end{fulllineitems}


\end{fulllineitems}

\index{CbEngValWrite (class in TotalDepth.LIS.core.LogiRec)}

\begin{fulllineitems}
\phantomsection\label{\detokenize{ref/LIS/core/LogiRec:TotalDepth.LIS.core.LogiRec.CbEngValWrite}}\pysiglinewithargsret{\sphinxbfcode{class }\sphinxcode{TotalDepth.LIS.core.LogiRec.}\sphinxbfcode{CbEngValWrite}}{\emph{t}, \emph{v}, \emph{m}, \emph{**kwargs}}{}
A Component Block and has an EngVal created directly.
\index{\_\_init\_\_() (TotalDepth.LIS.core.LogiRec.CbEngValWrite method)}

\begin{fulllineitems}
\phantomsection\label{\detokenize{ref/LIS/core/LogiRec:TotalDepth.LIS.core.LogiRec.CbEngValWrite.__init__}}\pysiglinewithargsret{\sphinxbfcode{\_\_init\_\_}}{\emph{t}, \emph{v}, \emph{m}, \emph{**kwargs}}{}
Initialise component block with type, value, mnemonic and optional key words.

\end{fulllineitems}


\end{fulllineitems}

\index{TableRow (class in TotalDepth.LIS.core.LogiRec)}

\begin{fulllineitems}
\phantomsection\label{\detokenize{ref/LIS/core/LogiRec:TotalDepth.LIS.core.LogiRec.TableRow}}\pysiglinewithargsret{\sphinxbfcode{class }\sphinxcode{TotalDepth.LIS.core.LogiRec.}\sphinxbfcode{TableRow}}{\emph{theCb}}{}
Represents a row of a table and consists of CbEngVal objects.
\index{genCells() (TotalDepth.LIS.core.LogiRec.TableRow method)}

\begin{fulllineitems}
\phantomsection\label{\detokenize{ref/LIS/core/LogiRec:TotalDepth.LIS.core.LogiRec.TableRow.genCells}}\pysiglinewithargsret{\sphinxbfcode{genCells}}{}{}
yields each CbEngVal.

\end{fulllineitems}

\index{\_\_len\_\_() (TotalDepth.LIS.core.LogiRec.TableRow method)}

\begin{fulllineitems}
\phantomsection\label{\detokenize{ref/LIS/core/LogiRec:TotalDepth.LIS.core.LogiRec.TableRow.__len__}}\pysiglinewithargsret{\sphinxbfcode{\_\_len\_\_}}{}{}
The number of cells in the row.

\end{fulllineitems}

\index{value (TotalDepth.LIS.core.LogiRec.TableRow attribute)}

\begin{fulllineitems}
\phantomsection\label{\detokenize{ref/LIS/core/LogiRec:TotalDepth.LIS.core.LogiRec.TableRow.value}}\pysigline{\sphinxbfcode{value}}
The name of the row i.e. the value of block 0.

\end{fulllineitems}

\index{addCb() (TotalDepth.LIS.core.LogiRec.TableRow method)}

\begin{fulllineitems}
\phantomsection\label{\detokenize{ref/LIS/core/LogiRec:TotalDepth.LIS.core.LogiRec.TableRow.addCb}}\pysiglinewithargsret{\sphinxbfcode{addCb}}{\emph{theCb}}{}
Adds a component block onto the end of the row.
Returns the mnemonic field from the component block.

\end{fulllineitems}

\index{\_\_getitem\_\_() (TotalDepth.LIS.core.LogiRec.TableRow method)}

\begin{fulllineitems}
\phantomsection\label{\detokenize{ref/LIS/core/LogiRec:TotalDepth.LIS.core.LogiRec.TableRow.__getitem__}}\pysiglinewithargsret{\sphinxbfcode{\_\_getitem\_\_}}{\emph{key}}{}
If key is an integer or slice this returns a CbEngVal by index(es).
If key is a bytes() object then this returns a CbEngVal by label.
May raise a KeyError or IndexError.

\end{fulllineitems}

\index{\_\_contains\_\_() (TotalDepth.LIS.core.LogiRec.TableRow method)}

\begin{fulllineitems}
\phantomsection\label{\detokenize{ref/LIS/core/LogiRec:TotalDepth.LIS.core.LogiRec.TableRow.__contains__}}\pysiglinewithargsret{\sphinxbfcode{\_\_contains\_\_}}{\emph{key}}{}
Returns True if this row has a column named key.

\end{fulllineitems}

\index{\_\_weakref\_\_ (TotalDepth.LIS.core.LogiRec.TableRow attribute)}

\begin{fulllineitems}
\phantomsection\label{\detokenize{ref/LIS/core/LogiRec:TotalDepth.LIS.core.LogiRec.TableRow.__weakref__}}\pysigline{\sphinxbfcode{\_\_weakref\_\_}}
list of weak references to the object (if defined)

\end{fulllineitems}


\end{fulllineitems}

\index{LrTable (class in TotalDepth.LIS.core.LogiRec)}

\begin{fulllineitems}
\phantomsection\label{\detokenize{ref/LIS/core/LogiRec:TotalDepth.LIS.core.LogiRec.LrTable}}\pysiglinewithargsret{\sphinxbfcode{class }\sphinxcode{TotalDepth.LIS.core.LogiRec.}\sphinxbfcode{LrTable}}{\emph{theType}, \emph{theAttr}}{}
Table-like Logical Record.
\index{\_\_init\_\_() (TotalDepth.LIS.core.LogiRec.LrTable method)}

\begin{fulllineitems}
\phantomsection\label{\detokenize{ref/LIS/core/LogiRec:TotalDepth.LIS.core.LogiRec.LrTable.__init__}}\pysiglinewithargsret{\sphinxbfcode{\_\_init\_\_}}{\emph{theType}, \emph{theAttr}}{}
Base class constructor. theType, theAttr are byte objects i.e. integers 0 to 255

\end{fulllineitems}

\index{genRows() (TotalDepth.LIS.core.LogiRec.LrTable method)}

\begin{fulllineitems}
\phantomsection\label{\detokenize{ref/LIS/core/LogiRec:TotalDepth.LIS.core.LogiRec.LrTable.genRows}}\pysiglinewithargsret{\sphinxbfcode{genRows}}{}{}
yields each TableRow.
TODO: parameter to sort or reverse.

\end{fulllineitems}

\index{genRowNames() (TotalDepth.LIS.core.LogiRec.LrTable method)}

\begin{fulllineitems}
\phantomsection\label{\detokenize{ref/LIS/core/LogiRec:TotalDepth.LIS.core.LogiRec.LrTable.genRowNames}}\pysiglinewithargsret{\sphinxbfcode{genRowNames}}{\emph{sort=0}}{}
yields each TableRow name.
If sort \textgreater{} 0 then the results are sorted. If sort \textless{} 0 the results are reverse sorted

\end{fulllineitems}

\index{retRowByMnem() (TotalDepth.LIS.core.LogiRec.LrTable method)}

\begin{fulllineitems}
\phantomsection\label{\detokenize{ref/LIS/core/LogiRec:TotalDepth.LIS.core.LogiRec.LrTable.retRowByMnem}}\pysiglinewithargsret{\sphinxbfcode{retRowByMnem}}{\emph{m}}{}
Returns the TableRow from a Mnem.Mnem object. i.e. the table row that has a b’MNEM’
column whose value matches m.
May raise a KeyError. Note: an IndexError would mean that self.\_\_mnemRowIndex is corrupt.

\end{fulllineitems}

\index{isSingleParam (TotalDepth.LIS.core.LogiRec.LrTable attribute)}

\begin{fulllineitems}
\phantomsection\label{\detokenize{ref/LIS/core/LogiRec:TotalDepth.LIS.core.LogiRec.LrTable.isSingleParam}}\pysigline{\sphinxbfcode{isSingleParam}}
True it this table is a list of single parameters (i.e. type 0 blocks).

\end{fulllineitems}

\index{value (TotalDepth.LIS.core.LogiRec.LrTable attribute)}

\begin{fulllineitems}
\phantomsection\label{\detokenize{ref/LIS/core/LogiRec:TotalDepth.LIS.core.LogiRec.LrTable.value}}\pysigline{\sphinxbfcode{value}}
The name of the table or None i.e. the value of the first block.

\end{fulllineitems}

\index{\_\_getitem\_\_() (TotalDepth.LIS.core.LogiRec.LrTable method)}

\begin{fulllineitems}
\phantomsection\label{\detokenize{ref/LIS/core/LogiRec:TotalDepth.LIS.core.LogiRec.LrTable.__getitem__}}\pysiglinewithargsret{\sphinxbfcode{\_\_getitem\_\_}}{\emph{key}}{}
If key is an integer or slice this returns block by index(es).
If key is a bytes() object then this returns row by label.
May raise a KeyError or IndexError.

\end{fulllineitems}

\index{\_\_contains\_\_() (TotalDepth.LIS.core.LogiRec.LrTable method)}

\begin{fulllineitems}
\phantomsection\label{\detokenize{ref/LIS/core/LogiRec:TotalDepth.LIS.core.LogiRec.LrTable.__contains__}}\pysiglinewithargsret{\sphinxbfcode{\_\_contains\_\_}}{\emph{key}}{}
Returns True if this row has a row named key.

\end{fulllineitems}

\index{\_\_len\_\_() (TotalDepth.LIS.core.LogiRec.LrTable method)}

\begin{fulllineitems}
\phantomsection\label{\detokenize{ref/LIS/core/LogiRec:TotalDepth.LIS.core.LogiRec.LrTable.__len__}}\pysiglinewithargsret{\sphinxbfcode{\_\_len\_\_}}{}{}
Number of rows in the table.

\end{fulllineitems}

\index{rowLabels() (TotalDepth.LIS.core.LogiRec.LrTable method)}

\begin{fulllineitems}
\phantomsection\label{\detokenize{ref/LIS/core/LogiRec:TotalDepth.LIS.core.LogiRec.LrTable.rowLabels}}\pysiglinewithargsret{\sphinxbfcode{rowLabels}}{}{}
Returns dictionary view of row values (unordered).

\end{fulllineitems}

\index{rowMnems() (TotalDepth.LIS.core.LogiRec.LrTable method)}

\begin{fulllineitems}
\phantomsection\label{\detokenize{ref/LIS/core/LogiRec:TotalDepth.LIS.core.LogiRec.LrTable.rowMnems}}\pysiglinewithargsret{\sphinxbfcode{rowMnems}}{}{}
Returns dictionary view of row Mnem.Mnem objects (unordered).

\end{fulllineitems}

\index{colLabels() (TotalDepth.LIS.core.LogiRec.LrTable method)}

\begin{fulllineitems}
\phantomsection\label{\detokenize{ref/LIS/core/LogiRec:TotalDepth.LIS.core.LogiRec.LrTable.colLabels}}\pysiglinewithargsret{\sphinxbfcode{colLabels}}{}{}
Returns ordered super-set of column values. Not all rows have these.

\end{fulllineitems}

\index{startNewRow() (TotalDepth.LIS.core.LogiRec.LrTable method)}

\begin{fulllineitems}
\phantomsection\label{\detokenize{ref/LIS/core/LogiRec:TotalDepth.LIS.core.LogiRec.LrTable.startNewRow}}\pysiglinewithargsret{\sphinxbfcode{startNewRow}}{\emph{theCbEv}}{}
Starts a new row from the component block. Returns theCbEv.mnem value.

\end{fulllineitems}

\index{addDatumBlock() (TotalDepth.LIS.core.LogiRec.LrTable method)}

\begin{fulllineitems}
\phantomsection\label{\detokenize{ref/LIS/core/LogiRec:TotalDepth.LIS.core.LogiRec.LrTable.addDatumBlock}}\pysiglinewithargsret{\sphinxbfcode{addDatumBlock}}{\emph{theCbEv}}{}
Adds a component block to the last row. Returns theCbEv value.

\end{fulllineitems}

\index{genRowValuesInColOrder() (TotalDepth.LIS.core.LogiRec.LrTable method)}

\begin{fulllineitems}
\phantomsection\label{\detokenize{ref/LIS/core/LogiRec:TotalDepth.LIS.core.LogiRec.LrTable.genRowValuesInColOrder}}\pysiglinewithargsret{\sphinxbfcode{genRowValuesInColOrder}}{\emph{theRow}}{}
Yields table cells in a particular row in column order.

\end{fulllineitems}

\index{genLisBytes() (TotalDepth.LIS.core.LogiRec.LrTable method)}

\begin{fulllineitems}
\phantomsection\label{\detokenize{ref/LIS/core/LogiRec:TotalDepth.LIS.core.LogiRec.LrTable.genLisBytes}}\pysiglinewithargsret{\sphinxbfcode{genLisBytes}}{}{}
Yields chunks of binary LIS data (actually each component block).

\end{fulllineitems}


\end{fulllineitems}

\index{LrTableRead (class in TotalDepth.LIS.core.LogiRec)}

\begin{fulllineitems}
\phantomsection\label{\detokenize{ref/LIS/core/LogiRec:TotalDepth.LIS.core.LogiRec.LrTableRead}}\pysiglinewithargsret{\sphinxbfcode{class }\sphinxcode{TotalDepth.LIS.core.LogiRec.}\sphinxbfcode{LrTableRead}}{\emph{theFile}}{}
A table-like Logical Record read from a LIS file.

\end{fulllineitems}

\index{LrTableWrite (class in TotalDepth.LIS.core.LogiRec)}

\begin{fulllineitems}
\phantomsection\label{\detokenize{ref/LIS/core/LogiRec:TotalDepth.LIS.core.LogiRec.LrTableWrite}}\pysiglinewithargsret{\sphinxbfcode{class }\sphinxcode{TotalDepth.LIS.core.LogiRec.}\sphinxbfcode{LrTableWrite}}{\emph{theType}, \emph{theName}, \emph{theMnemS}, \emph{theTable}}{}
Creates a table from internal Python data structures.

theType is the table type e.g. b’FILM’

theMnemS is the list of column names, theTable members must fit this size.

If an element of the table is a tuple or a list it is assumed to be (value, units).
\index{\_\_init\_\_() (TotalDepth.LIS.core.LogiRec.LrTableWrite method)}

\begin{fulllineitems}
\phantomsection\label{\detokenize{ref/LIS/core/LogiRec:TotalDepth.LIS.core.LogiRec.LrTableWrite.__init__}}\pysiglinewithargsret{\sphinxbfcode{\_\_init\_\_}}{\emph{theType}, \emph{theName}, \emph{theMnemS}, \emph{theTable}}{}
Construct a table from internal data that must be bytes/float/int.
theType is the table type e.g. b’FILM’
theMnemS is the list of column names, theTable members must fit this size.
If an element of the table is a tuple or a list it is assumed to be (value, units).

\end{fulllineitems}


\end{fulllineitems}

\index{ExceptionEntryBlock}

\begin{fulllineitems}
\phantomsection\label{\detokenize{ref/LIS/core/LogiRec:TotalDepth.LIS.core.LogiRec.ExceptionEntryBlock}}\pysigline{\sphinxbfcode{exception }\sphinxcode{TotalDepth.LIS.core.LogiRec.}\sphinxbfcode{ExceptionEntryBlock}}
Specialisation of exception for Entry Blocks.

\end{fulllineitems}

\index{ExceptionEntryBlockSetInit}

\begin{fulllineitems}
\phantomsection\label{\detokenize{ref/LIS/core/LogiRec:TotalDepth.LIS.core.LogiRec.ExceptionEntryBlockSetInit}}\pysigline{\sphinxbfcode{exception }\sphinxcode{TotalDepth.LIS.core.LogiRec.}\sphinxbfcode{ExceptionEntryBlockSetInit}}
Exception for EntryBlockSet.\_\_init\_\_().

\end{fulllineitems}

\index{EntryBlockRead (class in TotalDepth.LIS.core.LogiRec)}

\begin{fulllineitems}
\phantomsection\label{\detokenize{ref/LIS/core/LogiRec:TotalDepth.LIS.core.LogiRec.EntryBlockRead}}\pysigline{\sphinxbfcode{class }\sphinxcode{TotalDepth.LIS.core.LogiRec.}\sphinxbfcode{EntryBlockRead}}
An entry block read from a LIS file.

\end{fulllineitems}

\index{EntryBlockSet (class in TotalDepth.LIS.core.LogiRec)}

\begin{fulllineitems}
\phantomsection\label{\detokenize{ref/LIS/core/LogiRec:TotalDepth.LIS.core.LogiRec.EntryBlockSet}}\pysigline{\sphinxbfcode{class }\sphinxcode{TotalDepth.LIS.core.LogiRec.}\sphinxbfcode{EntryBlockSet}}
Represents the set of Entry Blocks in a DFSR.
\index{ATTR\_MAP (TotalDepth.LIS.core.LogiRec.EntryBlockSet attribute)}

\begin{fulllineitems}
\phantomsection\label{\detokenize{ref/LIS/core/LogiRec:TotalDepth.LIS.core.LogiRec.EntryBlockSet.ATTR_MAP}}\pysigline{\sphinxbfcode{ATTR\_MAP}\sphinxbfcode{ = \{'dataType': 1, 'dsbType': 2, 'upDown': 4, 'optLogScale': 5, 'frameSpacing': 8, 'frameSpacingUnits': 9, 'absentValue': 12, 'recordingMode': 13, 'depthUnits': 14, 'depthRepCode': 15\}}}
Map of supported attributes i.e. those that are ‘interesting’

\end{fulllineitems}

\index{EB\_DOC (TotalDepth.LIS.core.LogiRec.EntryBlockSet attribute)}

\begin{fulllineitems}
\phantomsection\label{\detokenize{ref/LIS/core/LogiRec:TotalDepth.LIS.core.LogiRec.EntryBlockSet.EB_DOC}}\pysigline{\sphinxbfcode{EB\_DOC}\sphinxbfcode{ = \{0: 'Terminator, size is chosen to make total size even.', 1: 'Data Record Type. The type of the IFLR (0 \textbar{} 1) that this describes.', 2: 'Datum Spec Block Type. How to interpret the DSBs. Only 0 is defined.', 3: 'Data Frame Size. Not required.', 4: 'Up/Down. 1=up, 255=down, 0=neither.', 5: 'Optical Log Scale. 1=Feet, 255=Meters, 0=Time.', 6: 'Data Reference Point. The distance of the data reference point above the tool reference point. Essentially add this to depth to find the depth axis of un-memorised data such as tension.', 7: 'Units for Data Reference Point.', 8: 'Frame Spacing.', 9: 'Units for Frame Spacing.', 10: 'Undefined.', 11: 'Maximum Frames per Record.', 12: 'Absent value.', 13: 'Depth Recording Mode. 1=Indirect X, 0=Direct X', 14: 'Units of Depth when depth Recording Mode=1', 15: 'Representation Code for depth when Depth Recording Mode=1', 16: 'Datum Spec Block sub-Type. How to interpret the DSBs (0 \textbar{} 1)'\}}}
Documentation about each Entry Block

\end{fulllineitems}

\index{BLOCKS\_TO\_SKIP (TotalDepth.LIS.core.LogiRec.EntryBlockSet attribute)}

\begin{fulllineitems}
\phantomsection\label{\detokenize{ref/LIS/core/LogiRec:TotalDepth.LIS.core.LogiRec.EntryBlockSet.BLOCKS_TO_SKIP}}\pysigline{\sphinxbfcode{BLOCKS\_TO\_SKIP}\sphinxbfcode{ = (10,)}}
List of block numbers that are not written out, also \_setLisSizeEven()
and lisSize() ignore these.

\end{fulllineitems}

\index{\_\_getattr\_\_() (TotalDepth.LIS.core.LogiRec.EntryBlockSet method)}

\begin{fulllineitems}
\phantomsection\label{\detokenize{ref/LIS/core/LogiRec:TotalDepth.LIS.core.LogiRec.EntryBlockSet.__getattr__}}\pysiglinewithargsret{\sphinxbfcode{\_\_getattr\_\_}}{\emph{name}}{}
Returns the Entry Block corresponding to the name.

\end{fulllineitems}

\index{\_\_getitem\_\_() (TotalDepth.LIS.core.LogiRec.EntryBlockSet method)}

\begin{fulllineitems}
\phantomsection\label{\detokenize{ref/LIS/core/LogiRec:TotalDepth.LIS.core.LogiRec.EntryBlockSet.__getitem__}}\pysiglinewithargsret{\sphinxbfcode{\_\_getitem\_\_}}{\emph{key}}{}
This returns an Entry block by integer index.

\end{fulllineitems}

\index{logUp (TotalDepth.LIS.core.LogiRec.EntryBlockSet attribute)}

\begin{fulllineitems}
\phantomsection\label{\detokenize{ref/LIS/core/LogiRec:TotalDepth.LIS.core.LogiRec.EntryBlockSet.logUp}}\pysigline{\sphinxbfcode{logUp}}
True if the logging direction is up (X decreasing).
Note: not logUp and not logDown is possible to be True e.g. time log.

\end{fulllineitems}

\index{logDown (TotalDepth.LIS.core.LogiRec.EntryBlockSet attribute)}

\begin{fulllineitems}
\phantomsection\label{\detokenize{ref/LIS/core/LogiRec:TotalDepth.LIS.core.LogiRec.EntryBlockSet.logDown}}\pysigline{\sphinxbfcode{logDown}}
True if the logging direction is down (X increasing).
Note: not logUp and not logDown is possible to be True e.g. time log.

\end{fulllineitems}

\index{xInc (TotalDepth.LIS.core.LogiRec.EntryBlockSet attribute)}

\begin{fulllineitems}
\phantomsection\label{\detokenize{ref/LIS/core/LogiRec:TotalDepth.LIS.core.LogiRec.EntryBlockSet.xInc}}\pysigline{\sphinxbfcode{xInc}}
True if the logging X increases (down or time log).

\end{fulllineitems}

\index{opticalLogScale (TotalDepth.LIS.core.LogiRec.EntryBlockSet attribute)}

\begin{fulllineitems}
\phantomsection\label{\detokenize{ref/LIS/core/LogiRec:TotalDepth.LIS.core.LogiRec.EntryBlockSet.opticalLogScale}}\pysigline{\sphinxbfcode{opticalLogScale}}
Returns the Units corresponding to Entry Block 5: ‘Optical Log Scale’
This will be a LENG or TIME unit or empty if undefined.

\end{fulllineitems}

\index{lisSize() (TotalDepth.LIS.core.LogiRec.EntryBlockSet method)}

\begin{fulllineitems}
\phantomsection\label{\detokenize{ref/LIS/core/LogiRec:TotalDepth.LIS.core.LogiRec.EntryBlockSet.lisSize}}\pysiglinewithargsret{\sphinxbfcode{lisSize}}{}{}
Returns the totla size of the Entry Block set.

\end{fulllineitems}

\index{setEntryBlock() (TotalDepth.LIS.core.LogiRec.EntryBlockSet method)}

\begin{fulllineitems}
\phantomsection\label{\detokenize{ref/LIS/core/LogiRec:TotalDepth.LIS.core.LogiRec.EntryBlockSet.setEntryBlock}}\pysiglinewithargsret{\sphinxbfcode{setEntryBlock}}{\emph{theEb}}{}
Sets an Entry Block.

\end{fulllineitems}

\index{readFromFile() (TotalDepth.LIS.core.LogiRec.EntryBlockSet method)}

\begin{fulllineitems}
\phantomsection\label{\detokenize{ref/LIS/core/LogiRec:TotalDepth.LIS.core.LogiRec.EntryBlockSet.readFromFile}}\pysiglinewithargsret{\sphinxbfcode{readFromFile}}{\emph{theFile}}{}
Reads from a File object. NOTE: theFile.hasLd() must be True so
the Logical Record Header must have been read already.

\end{fulllineitems}

\index{lisBytes() (TotalDepth.LIS.core.LogiRec.EntryBlockSet method)}

\begin{fulllineitems}
\phantomsection\label{\detokenize{ref/LIS/core/LogiRec:TotalDepth.LIS.core.LogiRec.EntryBlockSet.lisBytes}}\pysiglinewithargsret{\sphinxbfcode{lisBytes}}{}{}
Returns the Entry Block set as an array of bytes.

\end{fulllineitems}

\index{lisByteList() (TotalDepth.LIS.core.LogiRec.EntryBlockSet method)}

\begin{fulllineitems}
\phantomsection\label{\detokenize{ref/LIS/core/LogiRec:TotalDepth.LIS.core.LogiRec.EntryBlockSet.lisByteList}}\pysiglinewithargsret{\sphinxbfcode{lisByteList}}{}{}
Returns a list of bytes() objects, one for each entry block.

\end{fulllineitems}

\index{\_\_weakref\_\_ (TotalDepth.LIS.core.LogiRec.EntryBlockSet attribute)}

\begin{fulllineitems}
\phantomsection\label{\detokenize{ref/LIS/core/LogiRec:TotalDepth.LIS.core.LogiRec.EntryBlockSet.__weakref__}}\pysigline{\sphinxbfcode{\_\_weakref\_\_}}
list of weak references to the object (if defined)

\end{fulllineitems}


\end{fulllineitems}

\index{ExceptionDatumSpecBlock}

\begin{fulllineitems}
\phantomsection\label{\detokenize{ref/LIS/core/LogiRec:TotalDepth.LIS.core.LogiRec.ExceptionDatumSpecBlock}}\pysigline{\sphinxbfcode{exception }\sphinxcode{TotalDepth.LIS.core.LogiRec.}\sphinxbfcode{ExceptionDatumSpecBlock}}
Specialisation of exception for Datum Specification Blocks.

\end{fulllineitems}

\index{DatumSpecBlock (class in TotalDepth.LIS.core.LogiRec)}

\begin{fulllineitems}
\phantomsection\label{\detokenize{ref/LIS/core/LogiRec:TotalDepth.LIS.core.LogiRec.DatumSpecBlock}}\pysigline{\sphinxbfcode{class }\sphinxcode{TotalDepth.LIS.core.LogiRec.}\sphinxbfcode{DatumSpecBlock}}
This represents as Datum Specification Block.
\index{isNull (TotalDepth.LIS.core.LogiRec.DatumSpecBlock attribute)}

\begin{fulllineitems}
\phantomsection\label{\detokenize{ref/LIS/core/LogiRec:TotalDepth.LIS.core.LogiRec.DatumSpecBlock.isNull}}\pysigline{\sphinxbfcode{isNull}}
True if this block is compromised in any way and should be ignored
when composing a DFSR. The critical test is whether the data from this
channel will be in the frame.

\end{fulllineitems}

\index{samples() (TotalDepth.LIS.core.LogiRec.DatumSpecBlock method)}

\begin{fulllineitems}
\phantomsection\label{\detokenize{ref/LIS/core/LogiRec:TotalDepth.LIS.core.LogiRec.DatumSpecBlock.samples}}\pysiglinewithargsret{\sphinxbfcode{samples}}{\emph{theSc}}{}
Returns the number of samples in a sub-channel.

\end{fulllineitems}

\index{bursts() (TotalDepth.LIS.core.LogiRec.DatumSpecBlock method)}

\begin{fulllineitems}
\phantomsection\label{\detokenize{ref/LIS/core/LogiRec:TotalDepth.LIS.core.LogiRec.DatumSpecBlock.bursts}}\pysiglinewithargsret{\sphinxbfcode{bursts}}{\emph{theSc}}{}
Returns the (samples, burst) for a sub-channel (bursts are invariant over sub-channels).

\end{fulllineitems}

\index{values() (TotalDepth.LIS.core.LogiRec.DatumSpecBlock method)}

\begin{fulllineitems}
\phantomsection\label{\detokenize{ref/LIS/core/LogiRec:TotalDepth.LIS.core.LogiRec.DatumSpecBlock.values}}\pysiglinewithargsret{\sphinxbfcode{values}}{}{}
Returns the total number of discrete values per frame for a single channel.

\end{fulllineitems}

\index{subChMnem() (TotalDepth.LIS.core.LogiRec.DatumSpecBlock method)}

\begin{fulllineitems}
\phantomsection\label{\detokenize{ref/LIS/core/LogiRec:TotalDepth.LIS.core.LogiRec.DatumSpecBlock.subChMnem}}\pysiglinewithargsret{\sphinxbfcode{subChMnem}}{\emph{theSc}}{}
Returns the curve Mnemonic for a particular sub-channel or None if unknown.

\end{fulllineitems}

\index{\_\_weakref\_\_ (TotalDepth.LIS.core.LogiRec.DatumSpecBlock attribute)}

\begin{fulllineitems}
\phantomsection\label{\detokenize{ref/LIS/core/LogiRec:TotalDepth.LIS.core.LogiRec.DatumSpecBlock.__weakref__}}\pysigline{\sphinxbfcode{\_\_weakref\_\_}}
list of weak references to the object (if defined)

\end{fulllineitems}


\end{fulllineitems}

\index{DatumSpecBlockRead (class in TotalDepth.LIS.core.LogiRec)}

\begin{fulllineitems}
\phantomsection\label{\detokenize{ref/LIS/core/LogiRec:TotalDepth.LIS.core.LogiRec.DatumSpecBlockRead}}\pysiglinewithargsret{\sphinxbfcode{class }\sphinxcode{TotalDepth.LIS.core.LogiRec.}\sphinxbfcode{DatumSpecBlockRead}}{\emph{theF}}{}
This represents as Datum Specification Block read from a file.

\end{fulllineitems}

\index{LrDFSR (class in TotalDepth.LIS.core.LogiRec)}

\begin{fulllineitems}
\phantomsection\label{\detokenize{ref/LIS/core/LogiRec:TotalDepth.LIS.core.LogiRec.LrDFSR}}\pysiglinewithargsret{\sphinxbfcode{class }\sphinxcode{TotalDepth.LIS.core.LogiRec.}\sphinxbfcode{LrDFSR}}{\emph{theType}, \emph{theAttr}}{}
Data Format Specification Record.

\end{fulllineitems}

\index{LrDFSRRead (class in TotalDepth.LIS.core.LogiRec)}

\begin{fulllineitems}
\phantomsection\label{\detokenize{ref/LIS/core/LogiRec:TotalDepth.LIS.core.LogiRec.LrDFSRRead}}\pysiglinewithargsret{\sphinxbfcode{class }\sphinxcode{TotalDepth.LIS.core.LogiRec.}\sphinxbfcode{LrDFSRRead}}{\emph{theFile}}{}
Data Format Specification Record read from a file.

\end{fulllineitems}

\index{LrNormalAlternateData (class in TotalDepth.LIS.core.LogiRec)}

\begin{fulllineitems}
\phantomsection\label{\detokenize{ref/LIS/core/LogiRec:TotalDepth.LIS.core.LogiRec.LrNormalAlternateData}}\pysiglinewithargsret{\sphinxbfcode{class }\sphinxcode{TotalDepth.LIS.core.LogiRec.}\sphinxbfcode{LrNormalAlternateData}}{\emph{theType}, \emph{theAttr}}{}
Class for Normal and Alternate data i.e. curve data.

\end{fulllineitems}

\index{LrNormalAlternateDataRead (class in TotalDepth.LIS.core.LogiRec)}

\begin{fulllineitems}
\phantomsection\label{\detokenize{ref/LIS/core/LogiRec:TotalDepth.LIS.core.LogiRec.LrNormalAlternateDataRead}}\pysiglinewithargsret{\sphinxbfcode{class }\sphinxcode{TotalDepth.LIS.core.LogiRec.}\sphinxbfcode{LrNormalAlternateDataRead}}{\emph{theFile}}{}
Class for Normal and Alternate data i.e. curve data.

\end{fulllineitems}

\index{LrFactory (class in TotalDepth.LIS.core.LogiRec)}

\begin{fulllineitems}
\phantomsection\label{\detokenize{ref/LIS/core/LogiRec:TotalDepth.LIS.core.LogiRec.LrFactory}}\pysigline{\sphinxbfcode{class }\sphinxcode{TotalDepth.LIS.core.LogiRec.}\sphinxbfcode{LrFactory}}
Provides a despatch mechanism for generating Logical Records.
This can be sub-classed to create different sets of Logical Records.
For example the Indexer creates minimal logical records.
\index{\_\_weakref\_\_ (TotalDepth.LIS.core.LogiRec.LrFactory attribute)}

\begin{fulllineitems}
\phantomsection\label{\detokenize{ref/LIS/core/LogiRec:TotalDepth.LIS.core.LogiRec.LrFactory.__weakref__}}\pysigline{\sphinxbfcode{\_\_weakref\_\_}}
list of weak references to the object (if defined)

\end{fulllineitems}


\end{fulllineitems}

\index{LrFactoryRead (class in TotalDepth.LIS.core.LogiRec)}

\begin{fulllineitems}
\phantomsection\label{\detokenize{ref/LIS/core/LogiRec:TotalDepth.LIS.core.LogiRec.LrFactoryRead}}\pysigline{\sphinxbfcode{class }\sphinxcode{TotalDepth.LIS.core.LogiRec.}\sphinxbfcode{LrFactoryRead}}
A factory for generating complete Logical Records from a file.
\index{retLrFromFile() (TotalDepth.LIS.core.LogiRec.LrFactoryRead method)}

\begin{fulllineitems}
\phantomsection\label{\detokenize{ref/LIS/core/LogiRec:TotalDepth.LIS.core.LogiRec.LrFactoryRead.retLrFromFile}}\pysiglinewithargsret{\sphinxbfcode{retLrFromFile}}{\emph{theFile}}{}
Given a LIS file this reads one Logical Record, and returns the
appropriate Logical Record object or None.

\end{fulllineitems}


\end{fulllineitems}



\subsubsection{FileIndexer}
\label{\detokenize{ref/LIS/core/FileIndexer:totaldepth-lis-core-fileindexer}}\label{\detokenize{ref/LIS/core/FileIndexer::doc}}\label{\detokenize{ref/LIS/core/FileIndexer:fileindexer}}
\#Contents:
\phantomsection\label{\detokenize{ref/LIS/core/FileIndexer:module-TotalDepth.LIS.core.FileIndexer}}\index{TotalDepth.LIS.core.FileIndexer (module)}
Indexes LIS files.

Created on 10 Feb 2011

@author: p2ross
\index{ExceptionFileIndex}

\begin{fulllineitems}
\phantomsection\label{\detokenize{ref/LIS/core/FileIndexer:TotalDepth.LIS.core.FileIndexer.ExceptionFileIndex}}\pysigline{\sphinxbfcode{exception }\sphinxcode{TotalDepth.LIS.core.FileIndexer.}\sphinxbfcode{ExceptionFileIndex}}
Specialisation of exception for the LIS file indexer.

\end{fulllineitems}

\index{FileIndex (class in TotalDepth.LIS.core.FileIndexer)}

\begin{fulllineitems}
\phantomsection\label{\detokenize{ref/LIS/core/FileIndexer:TotalDepth.LIS.core.FileIndexer.FileIndex}}\pysiglinewithargsret{\sphinxbfcode{class }\sphinxcode{TotalDepth.LIS.core.FileIndexer.}\sphinxbfcode{FileIndex}}{\emph{theF}, \emph{xAxisIndex=0}}{}
Create an index for the LIS file, theF is a LIS File object.

xAxisIndex is the channel index that is regarded as the X axis (default 0).
This is currently ignored in the absence of a reasonable use case.
\index{genAll() (TotalDepth.LIS.core.FileIndexer.FileIndex method)}

\begin{fulllineitems}
\phantomsection\label{\detokenize{ref/LIS/core/FileIndexer:TotalDepth.LIS.core.FileIndexer.FileIndex.genAll}}\pysiglinewithargsret{\sphinxbfcode{genAll}}{}{}
Generates each index object (child class of IndexObjBase).

\end{fulllineitems}

\index{genLogPasses() (TotalDepth.LIS.core.FileIndexer.FileIndex method)}

\begin{fulllineitems}
\phantomsection\label{\detokenize{ref/LIS/core/FileIndexer:TotalDepth.LIS.core.FileIndexer.FileIndex.genLogPasses}}\pysiglinewithargsret{\sphinxbfcode{genLogPasses}}{}{}
Yields all IndexLogPass objects in the index.

\end{fulllineitems}

\index{genPlotRecords() (TotalDepth.LIS.core.FileIndexer.FileIndex method)}

\begin{fulllineitems}
\phantomsection\label{\detokenize{ref/LIS/core/FileIndexer:TotalDepth.LIS.core.FileIndexer.FileIndex.genPlotRecords}}\pysiglinewithargsret{\sphinxbfcode{genPlotRecords}}{\emph{fromInternalRecords=True}}{}
This provides the minimal information for creating a Plot. It yields
a PlotRecordSet that has (tell\_FILM, tell\_PRES, tell\_AREA, tell\_PIP, IndexLogPass) objects that are not
separated by delimiter records.
tell\_FILM is the file offset of the FILM record.
tell\_PRES is the file offset of the PRES record.
tell\_AREA is None or the file offset of the AREA record.
tell\_PIP is None or the file offset of the PIP record.
IndexLogPass is a IndexLogPass object.

\end{fulllineitems}

\index{jsonObject() (TotalDepth.LIS.core.FileIndexer.FileIndex method)}

\begin{fulllineitems}
\phantomsection\label{\detokenize{ref/LIS/core/FileIndexer:TotalDepth.LIS.core.FileIndexer.FileIndex.jsonObject}}\pysiglinewithargsret{\sphinxbfcode{jsonObject}}{}{}
Return an Python object that can be JSON encoded.

\end{fulllineitems}

\index{longDesc() (TotalDepth.LIS.core.FileIndexer.FileIndex method)}

\begin{fulllineitems}
\phantomsection\label{\detokenize{ref/LIS/core/FileIndexer:TotalDepth.LIS.core.FileIndexer.FileIndex.longDesc}}\pysiglinewithargsret{\sphinxbfcode{longDesc}}{}{}
Returns a string that is the long description of this object.

\end{fulllineitems}

\index{lrTypeS (TotalDepth.LIS.core.FileIndexer.FileIndex attribute)}

\begin{fulllineitems}
\phantomsection\label{\detokenize{ref/LIS/core/FileIndexer:TotalDepth.LIS.core.FileIndexer.FileIndex.lrTypeS}}\pysigline{\sphinxbfcode{lrTypeS}}
Returns a list of Logical Record types (integers) that is in the index.

\end{fulllineitems}

\index{numLogPasses() (TotalDepth.LIS.core.FileIndexer.FileIndex method)}

\begin{fulllineitems}
\phantomsection\label{\detokenize{ref/LIS/core/FileIndexer:TotalDepth.LIS.core.FileIndexer.FileIndex.numLogPasses}}\pysiglinewithargsret{\sphinxbfcode{numLogPasses}}{}{}
Returns the number of IndexLogPass objects in the index.

\end{fulllineitems}


\end{fulllineitems}

\index{IndexFileHead (class in TotalDepth.LIS.core.FileIndexer)}

\begin{fulllineitems}
\phantomsection\label{\detokenize{ref/LIS/core/FileIndexer:TotalDepth.LIS.core.FileIndexer.IndexFileHead}}\pysiglinewithargsret{\sphinxbfcode{class }\sphinxcode{TotalDepth.LIS.core.FileIndexer.}\sphinxbfcode{IndexFileHead}}{\emph{tell}, \emph{lrType}, \emph{theF}}{}
Indexes a File header. The full Logical record is retained.

\end{fulllineitems}

\index{IndexFileHeadTail (class in TotalDepth.LIS.core.FileIndexer)}

\begin{fulllineitems}
\phantomsection\label{\detokenize{ref/LIS/core/FileIndexer:TotalDepth.LIS.core.FileIndexer.IndexFileHeadTail}}\pysiglinewithargsret{\sphinxbfcode{class }\sphinxcode{TotalDepth.LIS.core.FileIndexer.}\sphinxbfcode{IndexFileHeadTail}}{\emph{tell}, \emph{lrType}, \emph{theF}, \emph{theClass}}{}
Indexes a File header or trailer. The full Logical record is retained.
\index{tocStr() (TotalDepth.LIS.core.FileIndexer.IndexFileHeadTail method)}

\begin{fulllineitems}
\phantomsection\label{\detokenize{ref/LIS/core/FileIndexer:TotalDepth.LIS.core.FileIndexer.IndexFileHeadTail.tocStr}}\pysiglinewithargsret{\sphinxbfcode{tocStr}}{}{}
Returns a ‘pretty’ string suitable for a table of contents.

\end{fulllineitems}


\end{fulllineitems}

\index{IndexFileTail (class in TotalDepth.LIS.core.FileIndexer)}

\begin{fulllineitems}
\phantomsection\label{\detokenize{ref/LIS/core/FileIndexer:TotalDepth.LIS.core.FileIndexer.IndexFileTail}}\pysiglinewithargsret{\sphinxbfcode{class }\sphinxcode{TotalDepth.LIS.core.FileIndexer.}\sphinxbfcode{IndexFileTail}}{\emph{tell}, \emph{lrType}, \emph{theF}}{}
Indexes a File trailer. The full Logical record is retained.

\end{fulllineitems}

\index{IndexLogPass (class in TotalDepth.LIS.core.FileIndexer)}

\begin{fulllineitems}
\phantomsection\label{\detokenize{ref/LIS/core/FileIndexer:TotalDepth.LIS.core.FileIndexer.IndexLogPass}}\pysiglinewithargsret{\sphinxbfcode{class }\sphinxcode{TotalDepth.LIS.core.FileIndexer.}\sphinxbfcode{IndexLogPass}}{\emph{tell}, \emph{lrType}, \emph{theF}, \emph{xAxisIndex=0}}{}
The index of a Log Pass. This contains a LogPass object.

xAxisIndex is the channel index that is regarded as the X axis.
This is the indirect axis if present or defaults to channel 0.
\index{add() (TotalDepth.LIS.core.FileIndexer.IndexLogPass method)}

\begin{fulllineitems}
\phantomsection\label{\detokenize{ref/LIS/core/FileIndexer:TotalDepth.LIS.core.FileIndexer.IndexLogPass.add}}\pysiglinewithargsret{\sphinxbfcode{add}}{\emph{tell}, \emph{lrType}, \emph{theF}}{}
Add an IFLR.

\end{fulllineitems}

\index{canAdd() (TotalDepth.LIS.core.FileIndexer.IndexLogPass method)}

\begin{fulllineitems}
\phantomsection\label{\detokenize{ref/LIS/core/FileIndexer:TotalDepth.LIS.core.FileIndexer.IndexLogPass.canAdd}}\pysiglinewithargsret{\sphinxbfcode{canAdd}}{\emph{iflrType}}{}
Returns the IFLR type that this EFLR can describe.

\end{fulllineitems}

\index{iflrType() (TotalDepth.LIS.core.FileIndexer.IndexLogPass method)}

\begin{fulllineitems}
\phantomsection\label{\detokenize{ref/LIS/core/FileIndexer:TotalDepth.LIS.core.FileIndexer.IndexLogPass.iflrType}}\pysiglinewithargsret{\sphinxbfcode{iflrType}}{}{}
Returns the IFLR type that this EFLR can describe.

\end{fulllineitems}

\index{jsonObject() (TotalDepth.LIS.core.FileIndexer.IndexLogPass method)}

\begin{fulllineitems}
\phantomsection\label{\detokenize{ref/LIS/core/FileIndexer:TotalDepth.LIS.core.FileIndexer.IndexLogPass.jsonObject}}\pysiglinewithargsret{\sphinxbfcode{jsonObject}}{}{}
Return an Python object that can be JSON encoded.

\end{fulllineitems}

\index{logPass (TotalDepth.LIS.core.FileIndexer.IndexLogPass attribute)}

\begin{fulllineitems}
\phantomsection\label{\detokenize{ref/LIS/core/FileIndexer:TotalDepth.LIS.core.FileIndexer.IndexLogPass.logPass}}\pysigline{\sphinxbfcode{logPass}}
The LogPass object.

\end{fulllineitems}

\index{tocStr() (TotalDepth.LIS.core.FileIndexer.IndexLogPass method)}

\begin{fulllineitems}
\phantomsection\label{\detokenize{ref/LIS/core/FileIndexer:TotalDepth.LIS.core.FileIndexer.IndexLogPass.tocStr}}\pysiglinewithargsret{\sphinxbfcode{tocStr}}{}{}
Returns a ‘pretty’ string suitable for a table of contents.

\end{fulllineitems}


\end{fulllineitems}

\index{IndexLrFull (class in TotalDepth.LIS.core.FileIndexer)}

\begin{fulllineitems}
\phantomsection\label{\detokenize{ref/LIS/core/FileIndexer:TotalDepth.LIS.core.FileIndexer.IndexLrFull}}\pysiglinewithargsret{\sphinxbfcode{class }\sphinxcode{TotalDepth.LIS.core.FileIndexer.}\sphinxbfcode{IndexLrFull}}{\emph{tell}, \emph{lrType}, \emph{theF}, \emph{theClass}}{}
Takes a full LR and assigns it to self.\_lr.

theClass is a cls that is use to instantiate a Logical Record object at the current file position.
\index{setLogicalRecord() (TotalDepth.LIS.core.FileIndexer.IndexLrFull method)}

\begin{fulllineitems}
\phantomsection\label{\detokenize{ref/LIS/core/FileIndexer:TotalDepth.LIS.core.FileIndexer.IndexLrFull.setLogicalRecord}}\pysiglinewithargsret{\sphinxbfcode{setLogicalRecord}}{\emph{theFile}}{}
Sets the logicalRecord property to an LogiRec object from theFile.

\end{fulllineitems}


\end{fulllineitems}

\index{IndexNone (class in TotalDepth.LIS.core.FileIndexer)}

\begin{fulllineitems}
\phantomsection\label{\detokenize{ref/LIS/core/FileIndexer:TotalDepth.LIS.core.FileIndexer.IndexNone}}\pysiglinewithargsret{\sphinxbfcode{class }\sphinxcode{TotalDepth.LIS.core.FileIndexer.}\sphinxbfcode{IndexNone}}{\emph{tell}, \emph{lrType}, \emph{theF}}{}
NULL class just takes the LR information and skips to next LR.

\end{fulllineitems}

\index{IndexObjBase (class in TotalDepth.LIS.core.FileIndexer)}

\begin{fulllineitems}
\phantomsection\label{\detokenize{ref/LIS/core/FileIndexer:TotalDepth.LIS.core.FileIndexer.IndexObjBase}}\pysiglinewithargsret{\sphinxbfcode{class }\sphinxcode{TotalDepth.LIS.core.FileIndexer.}\sphinxbfcode{IndexObjBase}}{\emph{tell}, \emph{lrType}, \emph{theF}}{}
Base class for indexed objects.

tell - The file position of the Logical Record as an integer.

lrType - The Logical Record type as an integer.

theF - The LIS File object. The file ID is recorded for later error checking.
\index{canAdd() (TotalDepth.LIS.core.FileIndexer.IndexObjBase method)}

\begin{fulllineitems}
\phantomsection\label{\detokenize{ref/LIS/core/FileIndexer:TotalDepth.LIS.core.FileIndexer.IndexObjBase.canAdd}}\pysiglinewithargsret{\sphinxbfcode{canAdd}}{\emph{iflrType}}{}
Returns True if this can accumulate another IFLR.

\end{fulllineitems}

\index{iflrType() (TotalDepth.LIS.core.FileIndexer.IndexObjBase method)}

\begin{fulllineitems}
\phantomsection\label{\detokenize{ref/LIS/core/FileIndexer:TotalDepth.LIS.core.FileIndexer.IndexObjBase.iflrType}}\pysiglinewithargsret{\sphinxbfcode{iflrType}}{}{}
Returns the IFLR type that this EFLR can describe.

\end{fulllineitems}

\index{isDelimiter (TotalDepth.LIS.core.FileIndexer.IndexObjBase attribute)}

\begin{fulllineitems}
\phantomsection\label{\detokenize{ref/LIS/core/FileIndexer:TotalDepth.LIS.core.FileIndexer.IndexObjBase.isDelimiter}}\pysigline{\sphinxbfcode{isDelimiter}}
True if this represents a delimiter record e.g. File Head/Tail.

\end{fulllineitems}

\index{jsonObject() (TotalDepth.LIS.core.FileIndexer.IndexObjBase method)}

\begin{fulllineitems}
\phantomsection\label{\detokenize{ref/LIS/core/FileIndexer:TotalDepth.LIS.core.FileIndexer.IndexObjBase.jsonObject}}\pysiglinewithargsret{\sphinxbfcode{jsonObject}}{}{}
Return an Python object that can be JSON encoded.

\end{fulllineitems}

\index{logicalRecord (TotalDepth.LIS.core.FileIndexer.IndexObjBase attribute)}

\begin{fulllineitems}
\phantomsection\label{\detokenize{ref/LIS/core/FileIndexer:TotalDepth.LIS.core.FileIndexer.IndexObjBase.logicalRecord}}\pysigline{\sphinxbfcode{logicalRecord}}
The underlying LogiRec object or None.

\end{fulllineitems}

\index{lrType (TotalDepth.LIS.core.FileIndexer.IndexObjBase attribute)}

\begin{fulllineitems}
\phantomsection\label{\detokenize{ref/LIS/core/FileIndexer:TotalDepth.LIS.core.FileIndexer.IndexObjBase.lrType}}\pysigline{\sphinxbfcode{lrType}}
The Logical Record type, in integer.

\end{fulllineitems}

\index{setLogicalRecord() (TotalDepth.LIS.core.FileIndexer.IndexObjBase method)}

\begin{fulllineitems}
\phantomsection\label{\detokenize{ref/LIS/core/FileIndexer:TotalDepth.LIS.core.FileIndexer.IndexObjBase.setLogicalRecord}}\pysiglinewithargsret{\sphinxbfcode{setLogicalRecord}}{\emph{theFile}}{}
Sets the logicalRecord property to an LogiRec object from theFile.

\end{fulllineitems}

\index{tell (TotalDepth.LIS.core.FileIndexer.IndexObjBase attribute)}

\begin{fulllineitems}
\phantomsection\label{\detokenize{ref/LIS/core/FileIndexer:TotalDepth.LIS.core.FileIndexer.IndexObjBase.tell}}\pysigline{\sphinxbfcode{tell}}
The file offset of the Logical Record.

\end{fulllineitems}

\index{tocStr() (TotalDepth.LIS.core.FileIndexer.IndexObjBase method)}

\begin{fulllineitems}
\phantomsection\label{\detokenize{ref/LIS/core/FileIndexer:TotalDepth.LIS.core.FileIndexer.IndexObjBase.tocStr}}\pysiglinewithargsret{\sphinxbfcode{tocStr}}{}{}
Returns a ‘pretty’ string suitable for a table of contents.

\end{fulllineitems}


\end{fulllineitems}

\index{IndexReelHead (class in TotalDepth.LIS.core.FileIndexer)}

\begin{fulllineitems}
\phantomsection\label{\detokenize{ref/LIS/core/FileIndexer:TotalDepth.LIS.core.FileIndexer.IndexReelHead}}\pysiglinewithargsret{\sphinxbfcode{class }\sphinxcode{TotalDepth.LIS.core.FileIndexer.}\sphinxbfcode{IndexReelHead}}{\emph{tell}, \emph{lrType}, \emph{theF}}{}
Indexes a Reel header. The full Logical record is retained.

\end{fulllineitems}

\index{IndexReelHeadTail (class in TotalDepth.LIS.core.FileIndexer)}

\begin{fulllineitems}
\phantomsection\label{\detokenize{ref/LIS/core/FileIndexer:TotalDepth.LIS.core.FileIndexer.IndexReelHeadTail}}\pysiglinewithargsret{\sphinxbfcode{class }\sphinxcode{TotalDepth.LIS.core.FileIndexer.}\sphinxbfcode{IndexReelHeadTail}}{\emph{tell}, \emph{lrType}, \emph{theF}, \emph{theClass}}{}
Indexes a Reel header or trailer. The full Logical record is retained.
\index{tocStr() (TotalDepth.LIS.core.FileIndexer.IndexReelHeadTail method)}

\begin{fulllineitems}
\phantomsection\label{\detokenize{ref/LIS/core/FileIndexer:TotalDepth.LIS.core.FileIndexer.IndexReelHeadTail.tocStr}}\pysiglinewithargsret{\sphinxbfcode{tocStr}}{}{}
Returns a ‘pretty’ string suitable for a table of contents.

\end{fulllineitems}


\end{fulllineitems}

\index{IndexReelTail (class in TotalDepth.LIS.core.FileIndexer)}

\begin{fulllineitems}
\phantomsection\label{\detokenize{ref/LIS/core/FileIndexer:TotalDepth.LIS.core.FileIndexer.IndexReelTail}}\pysiglinewithargsret{\sphinxbfcode{class }\sphinxcode{TotalDepth.LIS.core.FileIndexer.}\sphinxbfcode{IndexReelTail}}{\emph{tell}, \emph{lrType}, \emph{theF}}{}
Indexes a Reel trailer. The full Logical record is retained.

\end{fulllineitems}

\index{IndexTable (class in TotalDepth.LIS.core.FileIndexer)}

\begin{fulllineitems}
\phantomsection\label{\detokenize{ref/LIS/core/FileIndexer:TotalDepth.LIS.core.FileIndexer.IndexTable}}\pysiglinewithargsret{\sphinxbfcode{class }\sphinxcode{TotalDepth.LIS.core.FileIndexer.}\sphinxbfcode{IndexTable}}{\emph{tell}, \emph{lrType}, \emph{theF}}{}
Table type logical records. Here we capture the first component block so
that we know the name of the table.
\index{jsonObject() (TotalDepth.LIS.core.FileIndexer.IndexTable method)}

\begin{fulllineitems}
\phantomsection\label{\detokenize{ref/LIS/core/FileIndexer:TotalDepth.LIS.core.FileIndexer.IndexTable.jsonObject}}\pysiglinewithargsret{\sphinxbfcode{jsonObject}}{}{}
Return an Python object that can be JSON encoded.

\end{fulllineitems}

\index{name (TotalDepth.LIS.core.FileIndexer.IndexTable attribute)}

\begin{fulllineitems}
\phantomsection\label{\detokenize{ref/LIS/core/FileIndexer:TotalDepth.LIS.core.FileIndexer.IndexTable.name}}\pysigline{\sphinxbfcode{name}}
The name of the table from the first component block.

\end{fulllineitems}

\index{setLogicalRecord() (TotalDepth.LIS.core.FileIndexer.IndexTable method)}

\begin{fulllineitems}
\phantomsection\label{\detokenize{ref/LIS/core/FileIndexer:TotalDepth.LIS.core.FileIndexer.IndexTable.setLogicalRecord}}\pysiglinewithargsret{\sphinxbfcode{setLogicalRecord}}{\emph{theFile}}{}
Sets the logicalRecord property to an LogiRec.LrTable() object.

\end{fulllineitems}

\index{tocStr() (TotalDepth.LIS.core.FileIndexer.IndexTable method)}

\begin{fulllineitems}
\phantomsection\label{\detokenize{ref/LIS/core/FileIndexer:TotalDepth.LIS.core.FileIndexer.IndexTable.tocStr}}\pysiglinewithargsret{\sphinxbfcode{tocStr}}{}{}
Returns a ‘pretty’ string suitable for a table of contents.

\end{fulllineitems}


\end{fulllineitems}

\index{IndexTapeHead (class in TotalDepth.LIS.core.FileIndexer)}

\begin{fulllineitems}
\phantomsection\label{\detokenize{ref/LIS/core/FileIndexer:TotalDepth.LIS.core.FileIndexer.IndexTapeHead}}\pysiglinewithargsret{\sphinxbfcode{class }\sphinxcode{TotalDepth.LIS.core.FileIndexer.}\sphinxbfcode{IndexTapeHead}}{\emph{tell}, \emph{lrType}, \emph{theF}}{}
Indexes a Tape header. The full Logical record is retained.

\end{fulllineitems}

\index{IndexTapeHeadTail (class in TotalDepth.LIS.core.FileIndexer)}

\begin{fulllineitems}
\phantomsection\label{\detokenize{ref/LIS/core/FileIndexer:TotalDepth.LIS.core.FileIndexer.IndexTapeHeadTail}}\pysiglinewithargsret{\sphinxbfcode{class }\sphinxcode{TotalDepth.LIS.core.FileIndexer.}\sphinxbfcode{IndexTapeHeadTail}}{\emph{tell}, \emph{lrType}, \emph{theF}, \emph{theClass}}{}
Indexes a Tape header or trailer. The full Logical record is retained.
\index{tocStr() (TotalDepth.LIS.core.FileIndexer.IndexTapeHeadTail method)}

\begin{fulllineitems}
\phantomsection\label{\detokenize{ref/LIS/core/FileIndexer:TotalDepth.LIS.core.FileIndexer.IndexTapeHeadTail.tocStr}}\pysiglinewithargsret{\sphinxbfcode{tocStr}}{}{}
Returns a ‘pretty’ string suitable for a table of contents.

\end{fulllineitems}


\end{fulllineitems}

\index{IndexTapeTail (class in TotalDepth.LIS.core.FileIndexer)}

\begin{fulllineitems}
\phantomsection\label{\detokenize{ref/LIS/core/FileIndexer:TotalDepth.LIS.core.FileIndexer.IndexTapeTail}}\pysiglinewithargsret{\sphinxbfcode{class }\sphinxcode{TotalDepth.LIS.core.FileIndexer.}\sphinxbfcode{IndexTapeTail}}{\emph{tell}, \emph{lrType}, \emph{theF}}{}
Indexes a Tape trailer. The full Logical record is retained.

\end{fulllineitems}

\index{IndexUnknownInternalFormat (class in TotalDepth.LIS.core.FileIndexer)}

\begin{fulllineitems}
\phantomsection\label{\detokenize{ref/LIS/core/FileIndexer:TotalDepth.LIS.core.FileIndexer.IndexUnknownInternalFormat}}\pysiglinewithargsret{\sphinxbfcode{class }\sphinxcode{TotalDepth.LIS.core.FileIndexer.}\sphinxbfcode{IndexUnknownInternalFormat}}{\emph{tell}, \emph{lrType}, \emph{theF}}{}
Binary verbatim class for things like encrypted records, images, raw table dumps and so on.
\index{setLogicalRecord() (TotalDepth.LIS.core.FileIndexer.IndexUnknownInternalFormat method)}

\begin{fulllineitems}
\phantomsection\label{\detokenize{ref/LIS/core/FileIndexer:TotalDepth.LIS.core.FileIndexer.IndexUnknownInternalFormat.setLogicalRecord}}\pysiglinewithargsret{\sphinxbfcode{setLogicalRecord}}{\emph{theFile}}{}
Sets the logicalRecord property to an LogiRec.LrTable() object.

\end{fulllineitems}


\end{fulllineitems}

\index{PlotRecordSet (class in TotalDepth.LIS.core.FileIndexer)}

\begin{fulllineitems}
\phantomsection\label{\detokenize{ref/LIS/core/FileIndexer:TotalDepth.LIS.core.FileIndexer.PlotRecordSet}}\pysigline{\sphinxbfcode{class }\sphinxcode{TotalDepth.LIS.core.FileIndexer.}\sphinxbfcode{PlotRecordSet}}
A POD class that can contain a set of references to the essential (plus
optional) logical records for plotting.
\index{canPlotFromExternalRecords() (TotalDepth.LIS.core.FileIndexer.PlotRecordSet method)}

\begin{fulllineitems}
\phantomsection\label{\detokenize{ref/LIS/core/FileIndexer:TotalDepth.LIS.core.FileIndexer.PlotRecordSet.canPlotFromExternalRecords}}\pysiglinewithargsret{\sphinxbfcode{canPlotFromExternalRecords}}{}{}
True if I have a valid value that could be yielded, i.e. the minimum
information from the file that allows a plot using some external definition
of what has to be plotted. In practice this means a LogPass.

\end{fulllineitems}

\index{canPlotFromInternalRecords() (TotalDepth.LIS.core.FileIndexer.PlotRecordSet method)}

\begin{fulllineitems}
\phantomsection\label{\detokenize{ref/LIS/core/FileIndexer:TotalDepth.LIS.core.FileIndexer.PlotRecordSet.canPlotFromInternalRecords}}\pysiglinewithargsret{\sphinxbfcode{canPlotFromInternalRecords}}{}{}
True if I have a valid value that could be yielded, i.e. the minimum
information from the file that allows a plot. In practice this means a
FILM, PRES record and a LogPass.

\end{fulllineitems}


\end{fulllineitems}



\paragraph{Usage}
\label{\detokenize{ref/LIS/core/FileIndexer:usage}}

\paragraph{Examples}
\label{\detokenize{ref/LIS/core/FileIndexer:examples}}

\subparagraph{Using a LIS Indexer}
\label{\detokenize{ref/LIS/core/FileIndexer:using-a-lis-indexer}}
Source:

\begin{sphinxVerbatim}[commandchars=\\\{\}]
\PYG{n}{myFilePath} \PYG{o}{=} \PYG{l+s+s2}{\PYGZdq{}}\PYG{l+s+s2}{Spam/Eggs.LIS}\PYG{l+s+s2}{\PYGZdq{}}
\PYG{c+c1}{\PYGZsh{} Open a LIS file}
\PYG{n}{myFi} \PYG{o}{=} \PYG{n}{File}\PYG{o}{.}\PYG{n}{FileRead}\PYG{p}{(}\PYG{n}{myFilePath}\PYG{p}{,} \PYG{n}{theFileId}\PYG{o}{=}\PYG{n}{myFilePath}\PYG{p}{,} \PYG{n}{keepGoing}\PYG{o}{=}\PYG{k+kc}{True}\PYG{p}{)}
\PYG{c+c1}{\PYGZsh{} Index the LIS file}
\PYG{n}{myIdx} \PYG{o}{=} \PYG{n}{FileIndexer}\PYG{o}{.}\PYG{n}{FileIndex}\PYG{p}{(}\PYG{n}{myFi}\PYG{p}{)}
\PYG{c+c1}{\PYGZsh{} Ask the index for all the LogPass objects, these do not have the frameSet populated yet}
\PYG{k}{for} \PYG{n}{aLr} \PYG{o+ow}{in} \PYG{n}{myIdx}\PYG{o}{.}\PYG{n}{genAll}\PYG{p}{(}\PYG{p}{)}\PYG{p}{:}
    \PYG{n+nb}{print}\PYG{p}{(}\PYG{n}{aLr}\PYG{p}{)}
\end{sphinxVerbatim}

Typical result:

\begin{sphinxVerbatim}[commandchars=\\\{\}]
\PYG{n}{tell}\PYG{p}{:} \PYG{l+m+mh}{0x00000000} \PYG{n+nb}{type}\PYG{o}{=}\PYG{l+m+mi}{128} \PYG{o}{\PYGZlt{}}\PYG{n}{TotalDepth}\PYG{o}{.}\PYG{n}{LIS}\PYG{o}{.}\PYG{n}{core}\PYG{o}{.}\PYG{n}{FileIndexer}\PYG{o}{.}\PYG{n}{IndexFileHead} \PYG{n+nb}{object} \PYG{n}{at} \PYG{l+m+mh}{0x10197ff90}\PYG{o}{\PYGZgt{}}
\PYG{n}{tell}\PYG{p}{:} \PYG{l+m+mh}{0x00000050} \PYG{n+nb}{type}\PYG{o}{=} \PYG{l+m+mi}{34} \PYG{n}{name}\PYG{o}{=}\PYG{l+s+sa}{b}\PYG{l+s+s1}{\PYGZsq{}}\PYG{l+s+s1}{TOOL}\PYG{l+s+s1}{\PYGZsq{}} \PYG{o}{\PYGZlt{}}\PYG{n}{TotalDepth}\PYG{o}{.}\PYG{n}{LIS}\PYG{o}{.}\PYG{n}{core}\PYG{o}{.}\PYG{n}{FileIndexer}\PYG{o}{.}\PYG{n}{IndexTable} \PYG{n+nb}{object} \PYG{n}{at} \PYG{l+m+mh}{0x10197ffd0}\PYG{o}{\PYGZgt{}}
\PYG{n}{tell}\PYG{p}{:} \PYG{l+m+mh}{0x000001ea} \PYG{n+nb}{type}\PYG{o}{=} \PYG{l+m+mi}{34} \PYG{n}{name}\PYG{o}{=}\PYG{l+s+sa}{b}\PYG{l+s+s1}{\PYGZsq{}}\PYG{l+s+s1}{INPU}\PYG{l+s+s1}{\PYGZsq{}} \PYG{o}{\PYGZlt{}}\PYG{n}{TotalDepth}\PYG{o}{.}\PYG{n}{LIS}\PYG{o}{.}\PYG{n}{core}\PYG{o}{.}\PYG{n}{FileIndexer}\PYG{o}{.}\PYG{n}{IndexTable} \PYG{n+nb}{object} \PYG{n}{at} \PYG{l+m+mh}{0x10197fd10}\PYG{o}{\PYGZgt{}}
\PYG{n}{tell}\PYG{p}{:} \PYG{l+m+mh}{0x00002342} \PYG{n+nb}{type}\PYG{o}{=} \PYG{l+m+mi}{34} \PYG{n}{name}\PYG{o}{=}\PYG{l+s+sa}{b}\PYG{l+s+s1}{\PYGZsq{}}\PYG{l+s+s1}{OUTP}\PYG{l+s+s1}{\PYGZsq{}} \PYG{o}{\PYGZlt{}}\PYG{n}{TotalDepth}\PYG{o}{.}\PYG{n}{LIS}\PYG{o}{.}\PYG{n}{core}\PYG{o}{.}\PYG{n}{FileIndexer}\PYG{o}{.}\PYG{n}{IndexTable} \PYG{n+nb}{object} \PYG{n}{at} \PYG{l+m+mh}{0x101b0b1d0}\PYG{o}{\PYGZgt{}}
\PYG{n}{tell}\PYG{p}{:} \PYG{l+m+mh}{0x00003622} \PYG{n+nb}{type}\PYG{o}{=} \PYG{l+m+mi}{34} \PYG{n}{name}\PYG{o}{=}\PYG{l+s+sa}{b}\PYG{l+s+s1}{\PYGZsq{}}\PYG{l+s+s1}{CONS}\PYG{l+s+s1}{\PYGZsq{}} \PYG{o}{\PYGZlt{}}\PYG{n}{TotalDepth}\PYG{o}{.}\PYG{n}{LIS}\PYG{o}{.}\PYG{n}{core}\PYG{o}{.}\PYG{n}{FileIndexer}\PYG{o}{.}\PYG{n}{IndexTable} \PYG{n+nb}{object} \PYG{n}{at} \PYG{l+m+mh}{0x101b0bed0}\PYG{o}{\PYGZgt{}}
\PYG{n}{tell}\PYG{p}{:} \PYG{l+m+mh}{0x0000456a} \PYG{n+nb}{type}\PYG{o}{=} \PYG{l+m+mi}{34} \PYG{n}{name}\PYG{o}{=}\PYG{l+s+sa}{b}\PYG{l+s+s1}{\PYGZsq{}}\PYG{l+s+s1}{CONS}\PYG{l+s+s1}{\PYGZsq{}} \PYG{o}{\PYGZlt{}}\PYG{n}{TotalDepth}\PYG{o}{.}\PYG{n}{LIS}\PYG{o}{.}\PYG{n}{core}\PYG{o}{.}\PYG{n}{FileIndexer}\PYG{o}{.}\PYG{n}{IndexTable} \PYG{n+nb}{object} \PYG{n}{at} \PYG{l+m+mh}{0x101b0b590}\PYG{o}{\PYGZgt{}}
\PYG{n}{tell}\PYG{p}{:} \PYG{l+m+mh}{0x000064aa} \PYG{n+nb}{type}\PYG{o}{=} \PYG{l+m+mi}{34} \PYG{n}{name}\PYG{o}{=}\PYG{l+s+sa}{b}\PYG{l+s+s1}{\PYGZsq{}}\PYG{l+s+s1}{PRES}\PYG{l+s+s1}{\PYGZsq{}} \PYG{o}{\PYGZlt{}}\PYG{n}{TotalDepth}\PYG{o}{.}\PYG{n}{LIS}\PYG{o}{.}\PYG{n}{core}\PYG{o}{.}\PYG{n}{FileIndexer}\PYG{o}{.}\PYG{n}{IndexTable} \PYG{n+nb}{object} \PYG{n}{at} \PYG{l+m+mh}{0x101b0b050}\PYG{o}{\PYGZgt{}}
\PYG{n}{tell}\PYG{p}{:} \PYG{l+m+mh}{0x00006efe} \PYG{n+nb}{type}\PYG{o}{=} \PYG{l+m+mi}{34} \PYG{n}{name}\PYG{o}{=}\PYG{l+s+sa}{b}\PYG{l+s+s1}{\PYGZsq{}}\PYG{l+s+s1}{FILM}\PYG{l+s+s1}{\PYGZsq{}} \PYG{o}{\PYGZlt{}}\PYG{n}{TotalDepth}\PYG{o}{.}\PYG{n}{LIS}\PYG{o}{.}\PYG{n}{core}\PYG{o}{.}\PYG{n}{FileIndexer}\PYG{o}{.}\PYG{n}{IndexTable} \PYG{n+nb}{object} \PYG{n}{at} \PYG{l+m+mh}{0x101b0bfd0}\PYG{o}{\PYGZgt{}}
\PYG{n}{tell}\PYG{p}{:} \PYG{l+m+mh}{0x00006fc6} \PYG{n+nb}{type}\PYG{o}{=} \PYG{l+m+mi}{34} \PYG{n}{name}\PYG{o}{=}\PYG{l+s+sa}{b}\PYG{l+s+s1}{\PYGZsq{}}\PYG{l+s+s1}{AREA}\PYG{l+s+s1}{\PYGZsq{}} \PYG{o}{\PYGZlt{}}\PYG{n}{TotalDepth}\PYG{o}{.}\PYG{n}{LIS}\PYG{o}{.}\PYG{n}{core}\PYG{o}{.}\PYG{n}{FileIndexer}\PYG{o}{.}\PYG{n}{IndexTable} \PYG{n+nb}{object} \PYG{n}{at} \PYG{l+m+mh}{0x101b0bb50}\PYG{o}{\PYGZgt{}}
\PYG{o}{\PYGZlt{}}\PYG{n}{TotalDepth}\PYG{o}{.}\PYG{n}{LIS}\PYG{o}{.}\PYG{n}{core}\PYG{o}{.}\PYG{n}{LogPass}\PYG{o}{.}\PYG{n}{LogPass} \PYG{n+nb}{object} \PYG{n}{at} \PYG{l+m+mh}{0x101b06490}\PYG{o}{\PYGZgt{}}
\PYG{o}{\PYGZlt{}}\PYG{n}{TotalDepth}\PYG{o}{.}\PYG{n}{LIS}\PYG{o}{.}\PYG{n}{core}\PYG{o}{.}\PYG{n}{LogPass}\PYG{o}{.}\PYG{n}{LogPass} \PYG{n+nb}{object} \PYG{n}{at} \PYG{l+m+mh}{0x101b12410}\PYG{o}{\PYGZgt{}}
\PYG{n}{tell}\PYG{p}{:} \PYG{l+m+mh}{0x0017cbee} \PYG{n+nb}{type}\PYG{o}{=}\PYG{l+m+mi}{129} \PYG{o}{\PYGZlt{}}\PYG{n}{TotalDepth}\PYG{o}{.}\PYG{n}{LIS}\PYG{o}{.}\PYG{n}{core}\PYG{o}{.}\PYG{n}{FileIndexer}\PYG{o}{.}\PYG{n}{IndexFileTail} \PYG{n+nb}{object} \PYG{n}{at} \PYG{l+m+mh}{0x101b173d0}\PYG{o}{\PYGZgt{}}
\end{sphinxVerbatim}


\paragraph{Testing}
\label{\detokenize{ref/LIS/core/FileIndexer:testing}}
The unit tests are in test/TestFileIndexer.Py


\subsubsection{The LIS Log Pass}
\label{\detokenize{ref/LIS/core/LogPass::doc}}\label{\detokenize{ref/LIS/core/LogPass:the-lis-log-pass}}
This describes the LIS LogPass class that encapsulates a Log Pass. A log pass is the binary log data plus the record (DFSR) that describes the format of the binary data. Internally the binary data is converted to a FrameSet that wraps a numpy array.

The LogPass module is located in \sphinxcode{src/TotalDepth/LIS/core} and can be imported thus:

\begin{sphinxVerbatim}[commandchars=\\\{\}]
\PYG{k+kn}{from} \PYG{n+nn}{TotalDepth}\PYG{n+nn}{.}\PYG{n+nn}{LIS}\PYG{n+nn}{.}\PYG{n+nn}{core} \PYG{k}{import} \PYG{n}{LogPass}
\end{sphinxVerbatim}


\paragraph{LogPass API Reference}
\label{\detokenize{ref/LIS/core/LogPass:module-TotalDepth.LIS.core.LogPass}}\label{\detokenize{ref/LIS/core/LogPass:logpass-api-reference}}\index{TotalDepth.LIS.core.LogPass (module)}
The LogPass module contains a single class LogPass that is fundamental to the
way that TotalDepth handles LIS binary data.

A Log Pass is defined as a single continuous recording of log data. “Main Log”,
“Repeat Section” are seperate examples of Log Pass(es).
\index{EVENT\_EXTRAPOLATE (in module TotalDepth.LIS.core.LogPass)}

\begin{fulllineitems}
\phantomsection\label{\detokenize{ref/LIS/core/LogPass:TotalDepth.LIS.core.LogPass.EVENT_EXTRAPOLATE}}\pysigline{\sphinxcode{TotalDepth.LIS.core.LogPass.}\sphinxbfcode{EVENT\_EXTRAPOLATE}\sphinxbfcode{ = 'extrapolate'}}
Event to extrapolate X axis

\end{fulllineitems}

\index{EVENT\_READ (in module TotalDepth.LIS.core.LogPass)}

\begin{fulllineitems}
\phantomsection\label{\detokenize{ref/LIS/core/LogPass:TotalDepth.LIS.core.LogPass.EVENT_READ}}\pysigline{\sphinxcode{TotalDepth.LIS.core.LogPass.}\sphinxbfcode{EVENT\_READ}\sphinxbfcode{ = 'read'}}
Event to read bytes from  frame

\end{fulllineitems}

\index{EVENT\_SEEK\_LR (in module TotalDepth.LIS.core.LogPass)}

\begin{fulllineitems}
\phantomsection\label{\detokenize{ref/LIS/core/LogPass:TotalDepth.LIS.core.LogPass.EVENT_SEEK_LR}}\pysigline{\sphinxcode{TotalDepth.LIS.core.LogPass.}\sphinxbfcode{EVENT\_SEEK\_LR}\sphinxbfcode{ = 'seekLr'}}
Event to seek to the start of a new Logical Record

\end{fulllineitems}

\index{EVENT\_SKIP (in module TotalDepth.LIS.core.LogPass)}

\begin{fulllineitems}
\phantomsection\label{\detokenize{ref/LIS/core/LogPass:TotalDepth.LIS.core.LogPass.EVENT_SKIP}}\pysigline{\sphinxcode{TotalDepth.LIS.core.LogPass.}\sphinxbfcode{EVENT\_SKIP}\sphinxbfcode{ = 'skip'}}
Event to skip bytes in  frame

\end{fulllineitems}

\index{ExceptionLogPass}

\begin{fulllineitems}
\phantomsection\label{\detokenize{ref/LIS/core/LogPass:TotalDepth.LIS.core.LogPass.ExceptionLogPass}}\pysigline{\sphinxbfcode{exception }\sphinxcode{TotalDepth.LIS.core.LogPass.}\sphinxbfcode{ExceptionLogPass}}
Specialisation of exception for LogPass.

\end{fulllineitems}

\index{ExceptionLogPassCtor}

\begin{fulllineitems}
\phantomsection\label{\detokenize{ref/LIS/core/LogPass:TotalDepth.LIS.core.LogPass.ExceptionLogPassCtor}}\pysigline{\sphinxbfcode{exception }\sphinxcode{TotalDepth.LIS.core.LogPass.}\sphinxbfcode{ExceptionLogPassCtor}}
Specialisation of exception for LogPass \_\_init\_\_().

\end{fulllineitems}

\index{ExceptionLogPassKeyError}

\begin{fulllineitems}
\phantomsection\label{\detokenize{ref/LIS/core/LogPass:TotalDepth.LIS.core.LogPass.ExceptionLogPassKeyError}}\pysigline{\sphinxbfcode{exception }\sphinxcode{TotalDepth.LIS.core.LogPass.}\sphinxbfcode{ExceptionLogPassKeyError}}
Raised when and internal KeyError is raised in a FrameSet access.

\end{fulllineitems}

\index{ExceptionLogPassNoFrameSet}

\begin{fulllineitems}
\phantomsection\label{\detokenize{ref/LIS/core/LogPass:TotalDepth.LIS.core.LogPass.ExceptionLogPassNoFrameSet}}\pysigline{\sphinxbfcode{exception }\sphinxcode{TotalDepth.LIS.core.LogPass.}\sphinxbfcode{ExceptionLogPassNoFrameSet}}
Raised when FrameSet access is required but there has been no call to setFrameSet().

\end{fulllineitems}

\index{ExceptionLogPassNoType01Data}

\begin{fulllineitems}
\phantomsection\label{\detokenize{ref/LIS/core/LogPass:TotalDepth.LIS.core.LogPass.ExceptionLogPassNoType01Data}}\pysigline{\sphinxbfcode{exception }\sphinxcode{TotalDepth.LIS.core.LogPass.}\sphinxbfcode{ExceptionLogPassNoType01Data}}
Raised on access when there is no frame data loaded by addType01Data().

\end{fulllineitems}

\index{LogPass (class in TotalDepth.LIS.core.LogPass)}

\begin{fulllineitems}
\phantomsection\label{\detokenize{ref/LIS/core/LogPass:TotalDepth.LIS.core.LogPass.LogPass}}\pysiglinewithargsret{\sphinxbfcode{class }\sphinxcode{TotalDepth.LIS.core.LogPass.}\sphinxbfcode{LogPass}}{\emph{theDfsr}, \emph{theFileId}, \emph{xAxisIndex=0}}{}
Contains the information about a log pass which is defined as a LIS Data
Format Specification Record (DFSR) plus any number of Type 0/1 Logical Records.

A LogPass must be created with a LIS DFSR and a LIS file ID. It can be then informed
with addType01Data() which records the file positions of Type 0/1 records.
When required the actual frame data can be populated with setFrameSet().

theDfsr - The DFSR.

theFileId - The ID of the file, this will be checked against any File object passed to me.

xAxisIndex - The index of the DSB block that describes the X axis, if indirect X this is ignored.
\index{addType01Data() (TotalDepth.LIS.core.LogPass.LogPass method)}

\begin{fulllineitems}
\phantomsection\label{\detokenize{ref/LIS/core/LogPass:TotalDepth.LIS.core.LogPass.LogPass.addType01Data}}\pysiglinewithargsret{\sphinxbfcode{addType01Data}}{\emph{tellLr}, \emph{lrType}, \emph{lrLen}, \emph{xAxisVal}}{}
Add an Type 0/1 logical record entry.

tellLr - the Logical Record start position in the file.

lrType - the type of the IFLR. Will raise an ExceptionLogPass if this does not match the DFSR.

lrLen - the length of the Logical record, not including the LRH.

xAxisVal - the value of the X Axis of the first frame of the Logical Record.

\end{fulllineitems}

\index{curveUnits() (TotalDepth.LIS.core.LogPass.LogPass method)}

\begin{fulllineitems}
\phantomsection\label{\detokenize{ref/LIS/core/LogPass:TotalDepth.LIS.core.LogPass.LogPass.curveUnits}}\pysiglinewithargsret{\sphinxbfcode{curveUnits}}{\emph{chMnem}}{}
Given a curve as a Mnem.Mnem() this returns the units as a bytes object.

\end{fulllineitems}

\index{curveUnitsAsStr() (TotalDepth.LIS.core.LogPass.LogPass method)}

\begin{fulllineitems}
\phantomsection\label{\detokenize{ref/LIS/core/LogPass:TotalDepth.LIS.core.LogPass.LogPass.curveUnitsAsStr}}\pysiglinewithargsret{\sphinxbfcode{curveUnitsAsStr}}{\emph{chMnem}}{}
Given a curve as a Mnem.Mnem() this returns the units as a string.

\end{fulllineitems}

\index{dfsr (TotalDepth.LIS.core.LogPass.LogPass attribute)}

\begin{fulllineitems}
\phantomsection\label{\detokenize{ref/LIS/core/LogPass:TotalDepth.LIS.core.LogPass.LogPass.dfsr}}\pysigline{\sphinxbfcode{dfsr}}
The DFSR used for construction.

\end{fulllineitems}

\index{frameFromX() (TotalDepth.LIS.core.LogPass.LogPass method)}

\begin{fulllineitems}
\phantomsection\label{\detokenize{ref/LIS/core/LogPass:TotalDepth.LIS.core.LogPass.LogPass.frameFromX}}\pysiglinewithargsret{\sphinxbfcode{frameFromX}}{\emph{theEv}}{}
Returns the estimated frame number from the X axis, and EngValue.

\end{fulllineitems}

\index{frameSet (TotalDepth.LIS.core.LogPass.LogPass attribute)}

\begin{fulllineitems}
\phantomsection\label{\detokenize{ref/LIS/core/LogPass:TotalDepth.LIS.core.LogPass.LogPass.frameSet}}\pysigline{\sphinxbfcode{frameSet}}
The Frame Set as a FrameSet.FrameSet() object or None if not initialised.

\end{fulllineitems}

\index{frameSetLongStr() (TotalDepth.LIS.core.LogPass.LogPass method)}

\begin{fulllineitems}
\phantomsection\label{\detokenize{ref/LIS/core/LogPass:TotalDepth.LIS.core.LogPass.LogPass.frameSetLongStr}}\pysiglinewithargsret{\sphinxbfcode{frameSetLongStr}}{}{}
Returns a long (multiline) descriptive string of the Frame Set or N/A if not initialised.

\end{fulllineitems}

\index{genFrameSetHeadings() (TotalDepth.LIS.core.LogPass.LogPass method)}

\begin{fulllineitems}
\phantomsection\label{\detokenize{ref/LIS/core/LogPass:TotalDepth.LIS.core.LogPass.LogPass.genFrameSetHeadings}}\pysiglinewithargsret{\sphinxbfcode{genFrameSetHeadings}}{}{}
This generates a name and units for each value in a frame in the
current frame set. It is useful for heading up a frame dump.

\end{fulllineitems}

\index{genFrameSetScNameUnit() (TotalDepth.LIS.core.LogPass.LogPass method)}

\begin{fulllineitems}
\phantomsection\label{\detokenize{ref/LIS/core/LogPass:TotalDepth.LIS.core.LogPass.LogPass.genFrameSetScNameUnit}}\pysiglinewithargsret{\sphinxbfcode{genFrameSetScNameUnit}}{\emph{toAscii=True}}{}
This generates a name and units for sub-channel in a frame in the
current frame set. It is useful for heading up a accumulate() dump.

\end{fulllineitems}

\index{genOutpPoints() (TotalDepth.LIS.core.LogPass.LogPass method)}

\begin{fulllineitems}
\phantomsection\label{\detokenize{ref/LIS/core/LogPass:TotalDepth.LIS.core.LogPass.LogPass.genOutpPoints}}\pysiglinewithargsret{\sphinxbfcode{genOutpPoints}}{\emph{theMnem}}{}
Wrapper around the frameset generator, in fact this returns exactly that generator.

\end{fulllineitems}

\index{hasOutpMnem() (TotalDepth.LIS.core.LogPass.LogPass method)}

\begin{fulllineitems}
\phantomsection\label{\detokenize{ref/LIS/core/LogPass:TotalDepth.LIS.core.LogPass.LogPass.hasOutpMnem}}\pysiglinewithargsret{\sphinxbfcode{hasOutpMnem}}{\emph{theMnem}}{}
Returns True is theMnem is in this LogPass (i.e. is in the DFSR).

\end{fulllineitems}

\index{iflrType (TotalDepth.LIS.core.LogPass.LogPass attribute)}

\begin{fulllineitems}
\phantomsection\label{\detokenize{ref/LIS/core/LogPass:TotalDepth.LIS.core.LogPass.LogPass.iflrType}}\pysigline{\sphinxbfcode{iflrType}}
Returns the IFLR type that this LogPass describes.

\end{fulllineitems}

\index{isIndirectX (TotalDepth.LIS.core.LogPass.LogPass attribute)}

\begin{fulllineitems}
\phantomsection\label{\detokenize{ref/LIS/core/LogPass:TotalDepth.LIS.core.LogPass.LogPass.isIndirectX}}\pysigline{\sphinxbfcode{isIndirectX}}
True if indirect X axis, False if explicit X axis channel.

\end{fulllineitems}

\index{jsonObject() (TotalDepth.LIS.core.LogPass.LogPass method)}

\begin{fulllineitems}
\phantomsection\label{\detokenize{ref/LIS/core/LogPass:TotalDepth.LIS.core.LogPass.LogPass.jsonObject}}\pysiglinewithargsret{\sphinxbfcode{jsonObject}}{}{}
Return an Python object that can be JSON encoded.

\end{fulllineitems}

\index{longStr() (TotalDepth.LIS.core.LogPass.LogPass method)}

\begin{fulllineitems}
\phantomsection\label{\detokenize{ref/LIS/core/LogPass:TotalDepth.LIS.core.LogPass.LogPass.longStr}}\pysiglinewithargsret{\sphinxbfcode{longStr}}{}{}
Returns a long (multiline) descriptive string.

\end{fulllineitems}

\index{nullValue (TotalDepth.LIS.core.LogPass.LogPass attribute)}

\begin{fulllineitems}
\phantomsection\label{\detokenize{ref/LIS/core/LogPass:TotalDepth.LIS.core.LogPass.LogPass.nullValue}}\pysigline{\sphinxbfcode{nullValue}}
The NULL or absent value as specified in the DFSR.

\end{fulllineitems}

\index{numBytes (TotalDepth.LIS.core.LogPass.LogPass attribute)}

\begin{fulllineitems}
\phantomsection\label{\detokenize{ref/LIS/core/LogPass:TotalDepth.LIS.core.LogPass.LogPass.numBytes}}\pysigline{\sphinxbfcode{numBytes}}
The number of bytes in the underlying frame set (i.e. LIS) representation for the curve data.
Returns None if the frame set is not initialised.

\end{fulllineitems}

\index{outpMnemS() (TotalDepth.LIS.core.LogPass.LogPass method)}

\begin{fulllineitems}
\phantomsection\label{\detokenize{ref/LIS/core/LogPass:TotalDepth.LIS.core.LogPass.LogPass.outpMnemS}}\pysiglinewithargsret{\sphinxbfcode{outpMnemS}}{}{}
Returns all of the OUTP Mnems in this LogPass (i.e. is in the DFSR).

\end{fulllineitems}

\index{retExtChIndexList() (TotalDepth.LIS.core.LogPass.LogPass method)}

\begin{fulllineitems}
\phantomsection\label{\detokenize{ref/LIS/core/LogPass:TotalDepth.LIS.core.LogPass.LogPass.retExtChIndexList}}\pysiglinewithargsret{\sphinxbfcode{retExtChIndexList}}{\emph{theMnemS}}{}
Returns a sorted, unique list of external channel indexes for a list
of mnemonics. May raise a ExceptionLogPassKeyError.

\end{fulllineitems}

\index{rle (TotalDepth.LIS.core.LogPass.LogPass attribute)}

\begin{fulllineitems}
\phantomsection\label{\detokenize{ref/LIS/core/LogPass:TotalDepth.LIS.core.LogPass.LogPass.rle}}\pysigline{\sphinxbfcode{rle}}
The Logical Record Run Length Encoding as a Rle.RLEType01() object.

\end{fulllineitems}

\index{setFrameSet() (TotalDepth.LIS.core.LogPass.LogPass method)}

\begin{fulllineitems}
\phantomsection\label{\detokenize{ref/LIS/core/LogPass:TotalDepth.LIS.core.LogPass.LogPass.setFrameSet}}\pysiglinewithargsret{\sphinxbfcode{setFrameSet}}{\emph{theFile}, \emph{theFrSl=None}, \emph{theChList=None}}{}
Populates the frames set.

theFile - The File object. Will raise an ExceptionLogPass is the file ID
does not match that in the constructor.

theFrSl - A slice object that describes when LogPass frames are to be used (default all).
Will raise an ExceptionLogPass is there are no frames to load
i.e. addType01Data() has not been called.

theChList - A list of external channel indexes (i.e. DSB block indexes)
to populate the frame set with (default all).

\end{fulllineitems}

\index{setFrameSetChX() (TotalDepth.LIS.core.LogPass.LogPass method)}

\begin{fulllineitems}
\phantomsection\label{\detokenize{ref/LIS/core/LogPass:TotalDepth.LIS.core.LogPass.LogPass.setFrameSetChX}}\pysiglinewithargsret{\sphinxbfcode{setFrameSetChX}}{\emph{theFi}, \emph{theChS}, \emph{Xstart}, \emph{Xstop}, \emph{frStep=1}}{}
Loads a FramesSet using ‘external’ values from a File object.
theChS is a list of channel mnemonics or None for all channels.
Xstart, Xstop are EngVal of the start stop.
frStep is not number of frames to step over, default means all frames.

\end{fulllineitems}

\index{totalFrames (TotalDepth.LIS.core.LogPass.LogPass attribute)}

\begin{fulllineitems}
\phantomsection\label{\detokenize{ref/LIS/core/LogPass:TotalDepth.LIS.core.LogPass.LogPass.totalFrames}}\pysigline{\sphinxbfcode{totalFrames}}
The total number of frames.

\end{fulllineitems}

\index{type01Plan (TotalDepth.LIS.core.LogPass.LogPass attribute)}

\begin{fulllineitems}
\phantomsection\label{\detokenize{ref/LIS/core/LogPass:TotalDepth.LIS.core.LogPass.LogPass.type01Plan}}\pysigline{\sphinxbfcode{type01Plan}}
The Frame plan, a Type01Plan.FrameSetPlan() object.

\end{fulllineitems}

\index{xAxisFirstEngVal (TotalDepth.LIS.core.LogPass.LogPass attribute)}

\begin{fulllineitems}
\phantomsection\label{\detokenize{ref/LIS/core/LogPass:TotalDepth.LIS.core.LogPass.LogPass.xAxisFirstEngVal}}\pysigline{\sphinxbfcode{xAxisFirstEngVal}}
The EngVal (value, units) of the X axis of the first frame.

\end{fulllineitems}

\index{xAxisFirstVal (TotalDepth.LIS.core.LogPass.LogPass attribute)}

\begin{fulllineitems}
\phantomsection\label{\detokenize{ref/LIS/core/LogPass:TotalDepth.LIS.core.LogPass.LogPass.xAxisFirstVal}}\pysigline{\sphinxbfcode{xAxisFirstVal}}
The numerical value of the X axis of the first frame.

\end{fulllineitems}

\index{xAxisFirstValOptical (TotalDepth.LIS.core.LogPass.LogPass attribute)}

\begin{fulllineitems}
\phantomsection\label{\detokenize{ref/LIS/core/LogPass:TotalDepth.LIS.core.LogPass.LogPass.xAxisFirstValOptical}}\pysigline{\sphinxbfcode{xAxisFirstValOptical}}
The numerical value of the X axis of the first frame in ‘optical’ units.

\end{fulllineitems}

\index{xAxisIndex (TotalDepth.LIS.core.LogPass.LogPass attribute)}

\begin{fulllineitems}
\phantomsection\label{\detokenize{ref/LIS/core/LogPass:TotalDepth.LIS.core.LogPass.LogPass.xAxisIndex}}\pysigline{\sphinxbfcode{xAxisIndex}}
The channel index that corresponds to the X axis.

\end{fulllineitems}

\index{xAxisLastEngVal (TotalDepth.LIS.core.LogPass.LogPass attribute)}

\begin{fulllineitems}
\phantomsection\label{\detokenize{ref/LIS/core/LogPass:TotalDepth.LIS.core.LogPass.LogPass.xAxisLastEngVal}}\pysigline{\sphinxbfcode{xAxisLastEngVal}}
The EngVal (value, units) of the X axis of the first frame.

\end{fulllineitems}

\index{xAxisLastVal (TotalDepth.LIS.core.LogPass.LogPass attribute)}

\begin{fulllineitems}
\phantomsection\label{\detokenize{ref/LIS/core/LogPass:TotalDepth.LIS.core.LogPass.LogPass.xAxisLastVal}}\pysigline{\sphinxbfcode{xAxisLastVal}}
The numerical value of the X axis of the last frame.

\end{fulllineitems}

\index{xAxisLastValOptical (TotalDepth.LIS.core.LogPass.LogPass attribute)}

\begin{fulllineitems}
\phantomsection\label{\detokenize{ref/LIS/core/LogPass:TotalDepth.LIS.core.LogPass.LogPass.xAxisLastValOptical}}\pysigline{\sphinxbfcode{xAxisLastValOptical}}
The numerical value of the X axis of the last frame in ‘optical’ units.

\end{fulllineitems}

\index{xAxisSpacing (TotalDepth.LIS.core.LogPass.LogPass attribute)}

\begin{fulllineitems}
\phantomsection\label{\detokenize{ref/LIS/core/LogPass:TotalDepth.LIS.core.LogPass.LogPass.xAxisSpacing}}\pysigline{\sphinxbfcode{xAxisSpacing}}
The numerical value of the X axis frame spacing. This is is the
extreme range of X axis values divided by the number of frames - 1.
This is +ve if the Xaxis increases, -ve if it decreases.

\end{fulllineitems}

\index{xAxisSpacingOptical (TotalDepth.LIS.core.LogPass.LogPass attribute)}

\begin{fulllineitems}
\phantomsection\label{\detokenize{ref/LIS/core/LogPass:TotalDepth.LIS.core.LogPass.LogPass.xAxisSpacingOptical}}\pysigline{\sphinxbfcode{xAxisSpacingOptical}}
The numerical value of the X axis frame spacing. This is is the
extreme range of X axis values divided by the number of frames - 1.

\end{fulllineitems}

\index{xAxisUnits (TotalDepth.LIS.core.LogPass.LogPass attribute)}

\begin{fulllineitems}
\phantomsection\label{\detokenize{ref/LIS/core/LogPass:TotalDepth.LIS.core.LogPass.LogPass.xAxisUnits}}\pysigline{\sphinxbfcode{xAxisUnits}}
The units of the X axis.

\end{fulllineitems}

\index{xAxisUnitsOptical (TotalDepth.LIS.core.LogPass.LogPass attribute)}

\begin{fulllineitems}
\phantomsection\label{\detokenize{ref/LIS/core/LogPass:TotalDepth.LIS.core.LogPass.LogPass.xAxisUnitsOptical}}\pysigline{\sphinxbfcode{xAxisUnitsOptical}}
Returns the actual units to ‘optical’ i.e. user friendly units.
For example if the Xaxis was in b’.1IN’ the ‘optical’ units would be b’FEET”.

\end{fulllineitems}


\end{fulllineitems}



\paragraph{LogPass Usage}
\label{\detokenize{ref/LIS/core/LogPass:logpass-usage}}
Typically a LogPass will be created directly or via a LIS FileIndexer. The latter technique is recommended as it is simpler.


\subparagraph{Direct usage}
\label{\detokenize{ref/LIS/core/LogPass:direct-usage}}
LogPass objects are used via a three step process and this reflects the sequential process of reading a LIS file:
\begin{enumerate}
\item {} 
Construction with a DFSR object (once).

\item {} 
Updating with the location of the binary data records that contain the frame data (once per IFLR).

\item {} 
Populating the FrameSet with real values from the file (many times).

\end{enumerate}


\subparagraph{Construction with a DFSR}
\label{\detokenize{ref/LIS/core/LogPass:construction-with-a-dfsr}}
A LogPass object needs to be constructed with an instance of a LogiRec.LrDFSRRead. The LogPass will take a reference to the DFSR so the caller need not. The caller can always access the DFSR with the \sphinxcode{.dfsr} property.

Assuming myF is a LIS File object positioned at the start of the DFSR:

\begin{sphinxVerbatim}[commandchars=\\\{\}]
\PYG{n}{myLp} \PYG{o}{=} \PYG{n}{LogPass}\PYG{o}{.}\PYG{n}{LogPass}\PYG{p}{(}\PYG{n}{LogiRec}\PYG{o}{.}\PYG{n}{LrDFSRRead}\PYG{p}{(}\PYG{n}{myF}\PYG{p}{)}\PYG{p}{,} \PYG{l+s+s1}{\PYGZsq{}}\PYG{l+s+s1}{FileID}\PYG{l+s+s1}{\PYGZsq{}}\PYG{p}{)}
\end{sphinxVerbatim}


\subparagraph{Resources}
\label{\detokenize{ref/LIS/core/LogPass:resources}}
At this stage no FrameSet is created so the resource usage is minimal.


\subparagraph{Add Binary Data Records}
\label{\detokenize{ref/LIS/core/LogPass:add-binary-data-records}}
The method \sphinxcode{addType01Data(tellLr, lrType, lrLen, xAxisVal)} adds the position of an IFLR that contain frame data. This method takes these arguments:


\begin{savenotes}\sphinxattablestart
\centering
\begin{tabulary}{\linewidth}[t]{|T|T|}
\hline
\sphinxstylethead{\sphinxstyletheadfamily 
Argument
\unskip}\relax &\sphinxstylethead{\sphinxstyletheadfamily 
Description
\unskip}\relax \\
\hline
tellLr
&
The Logical Record start position (as a \sphinxcode{size\_t}) in the file.
\\
\hline
lrType
&
The type of the IFLR {[}0 \textbar{} 1{]}. Will raise an ExceptionLogPass if this does not match the DFSR.
\\
\hline
lrLen
&
The length of the Logical Record in bytes, not including the LRH.
\\
\hline
xAxisVal
&
The value of the X Axis as a number of the first frame of the Logical Record.
\\
\hline
\end{tabulary}
\par
\sphinxattableend\end{savenotes}

This call should be made for each relevant IFLR as they are encountered in sequence.


\subparagraph{Resources}
\label{\detokenize{ref/LIS/core/LogPass:id1}}
At this stage no FrameSet is created and the arguments are encoded into an RLE object so the resource usage is minimal.


\subparagraph{Populating the FrameSet}
\label{\detokenize{ref/LIS/core/LogPass:populating-the-frameset}}
Once the preceding stages have been done the LogPass can be populated any number of times from the LIS file.

The method \sphinxcode{setFrameSet(theFile, theFrSl=None, theChList=None)} constructs a new frame set with the appropriate values.


\begin{savenotes}\sphinxattablestart
\centering
\begin{tabulary}{\linewidth}[t]{|T|T|}
\hline
\sphinxstylethead{\sphinxstyletheadfamily 
Argument
\unskip}\relax &\sphinxstylethead{\sphinxstyletheadfamily 
Description
\unskip}\relax \\
\hline
theFile
&
The File object. Will raise an ExceptionLogPass is the file ID does not match that used in the constructor.
\\
\hline
theFrSl
&
A slice object that describes when LogPass frames are to be used (default all). Will raise an ExceptionLogPass if there are no frames to load i.e. addType01Data() has not been called.
\\
\hline
theChList
&
A list of external channel indexes (i.e. DSB block indexes) to populate the frame set with (default all).
\\
\hline
\end{tabulary}
\par
\sphinxattableend\end{savenotes}


\subparagraph{Examples}
\label{\detokenize{ref/LIS/core/LogPass:examples}}
Setting a frame set for all frames and all channels:

\begin{sphinxVerbatim}[commandchars=\\\{\}]
\PYG{n}{myLogPass}\PYG{o}{.}\PYG{n}{setFrameSet}\PYG{p}{(}\PYG{n}{myFile}\PYG{p}{)}
\PYG{c+c1}{\PYGZsh{} The above line is equivalent to:}
\PYG{n}{myLogPass}\PYG{o}{.}\PYG{n}{setFrameSet}\PYG{p}{(}\PYG{n}{myFile}\PYG{p}{,} \PYG{n}{theFrSl}\PYG{o}{=}\PYG{k+kc}{None}\PYG{p}{,} \PYG{n}{theChList}\PYG{o}{=}\PYG{k+kc}{None}\PYG{p}{)}
\end{sphinxVerbatim}

Setting a frame set for frames {[}0:16:4{]} i.e. frame indexes (0,4,8,12) and channels {[}0, 4, 7{]}

\begin{sphinxVerbatim}[commandchars=\\\{\}]
\PYG{n}{myLogPass}\PYG{o}{.}\PYG{n}{setFrameSet}\PYG{p}{(}\PYG{n}{myFile}\PYG{p}{,} \PYG{n}{theFrSl}\PYG{o}{=}\PYG{n+nb}{slice}\PYG{p}{(}\PYG{l+m+mi}{0}\PYG{p}{,}\PYG{l+m+mi}{16}\PYG{p}{,}\PYG{l+m+mi}{4}\PYG{p}{)}\PYG{p}{,} \PYG{n}{theChList}\PYG{o}{=}\PYG{p}{[}\PYG{l+m+mi}{0}\PYG{p}{,} \PYG{l+m+mi}{4}\PYG{p}{,} \PYG{l+m+mi}{7}\PYG{p}{]}\PYG{p}{)}
\end{sphinxVerbatim}


\subparagraph{Resources}
\label{\detokenize{ref/LIS/core/LogPass:id2}}
Any previous FrameSet will be freed and a new FrameSet of the appropriate dimension is created so the resource usage can be significant.


\subparagraph{Using a LIS Indexer}
\label{\detokenize{ref/LIS/core/LogPass:using-a-lis-indexer}}
A {\hyperref[\detokenize{ref/LIS/core/FileIndexer:totaldepth-lis-core-fileindexer}]{\sphinxcrossref{\DUrole{std,std-ref}{FileIndexer}}}} {[}\sphinxcode{TotalDepth.LIS.core.FileIndexer.FileIndex}{]} object will perform the necessary construction of a LogPass and the population with Logical Record positions with \sphinxcode{addType01Data()} leaving the user just to call \sphinxcode{setFrameSet()}. Thus a FileIndex object imposes low resource usage until the user wishes to populate the frame set.

An indexer will index a LIS file that has multiple Log Passes (e.g. repeat section, main log etc.) so the indexer provides an iteration method for Log Passes:

\begin{sphinxVerbatim}[commandchars=\\\{\}]
\PYG{n}{myFilePath} \PYG{o}{=} \PYG{l+s+s2}{\PYGZdq{}}\PYG{l+s+s2}{Spam/Eggs.LIS}\PYG{l+s+s2}{\PYGZdq{}}
\PYG{c+c1}{\PYGZsh{} Open a LIS file, keepGoing for luck!}
\PYG{n}{myFi} \PYG{o}{=} \PYG{n}{File}\PYG{o}{.}\PYG{n}{FileRead}\PYG{p}{(}\PYG{n}{myFilePath}\PYG{p}{,} \PYG{n}{theFileId}\PYG{o}{=}\PYG{n}{myFilePath}\PYG{p}{,} \PYG{n}{keepGoing}\PYG{o}{=}\PYG{k+kc}{True}\PYG{p}{)}
\PYG{c+c1}{\PYGZsh{} Index the LIS file}
\PYG{n}{myIdx} \PYG{o}{=} \PYG{n}{FileIndexer}\PYG{o}{.}\PYG{n}{FileIndex}\PYG{p}{(}\PYG{n}{myFi}\PYG{p}{)}
\PYG{c+c1}{\PYGZsh{} Ask the index for all the LogPass objects, these do not have the frameSet populated yet}
\PYG{k}{for} \PYG{n}{aLp} \PYG{o+ow}{in} \PYG{n}{myIdx}\PYG{o}{.}\PYG{n}{genLogPasses}\PYG{p}{(}\PYG{p}{)}\PYG{p}{:}
    \PYG{c+c1}{\PYGZsh{} Load the FrameSet, all channels [None], all frames [None]}
    \PYG{n}{aLp}\PYG{o}{.}\PYG{n}{logPass}\PYG{o}{.}\PYG{n}{setFrameSet}\PYG{p}{(}\PYG{n}{myFi}\PYG{p}{,} \PYG{k+kc}{None}\PYG{p}{,} \PYG{k+kc}{None}\PYG{p}{)}
    \PYG{c+c1}{\PYGZsh{} The FrameSet is fully populated here...}
    \PYG{c+c1}{\PYGZsh{} Do something with it...}
\end{sphinxVerbatim}


\paragraph{Testing}
\label{\detokenize{ref/LIS/core/LogPass:testing}}
The unit tests are in \sphinxcode{test/TestLogPass.py}. This should take under a second to execute.

Running the tests under coverage:

\begin{sphinxVerbatim}[commandchars=\\\{\}]
\PYGZdl{} coverage run test/TestLogPass.py
TestClass.py script version \PYGZdq{}0.8.0\PYGZdq{}, dated 10 Jan 2011
Author: Paul Ross
Copyright (c) Paul Ross

testSetUpTearDown (\PYGZus{}\PYGZus{}main\PYGZus{}\PYGZus{}.TestLogPass\PYGZus{}LowLevel)
TestLogPass\PYGZus{}LowLevel: Tests setUp() and tearDown(). ... ok
test\PYGZus{}00 (\PYGZus{}\PYGZus{}main\PYGZus{}\PYGZus{}.TestLogPass\PYGZus{}LowLevel)
TestLogPass\PYGZus{}LowLevel.test\PYGZus{}00(): Construction. ... ok
test\PYGZus{}01 (\PYGZus{}\PYGZus{}main\PYGZus{}\PYGZus{}.TestLogPass\PYGZus{}LowLevel)
TestLogPass\PYGZus{}LowLevel.test\PYGZus{}01(): \PYGZus{}sliceFromList(). ... ok
test\PYGZus{}02 (\PYGZus{}\PYGZus{}main\PYGZus{}\PYGZus{}.TestLogPass\PYGZus{}LowLevel)
TestLogPass\PYGZus{}LowLevel.test\PYGZus{}02(): exercise various properties. ... ok
test\PYGZus{}10 (\PYGZus{}\PYGZus{}main\PYGZus{}\PYGZus{}.TestLogPass\PYGZus{}LowLevel)
TestLogPass\PYGZus{}LowLevel.test\PYGZus{}10(): nullValue(). ... ok

8\PYGZlt{}\PYGZhy{}\PYGZhy{}\PYGZhy{}\PYGZhy{}\PYGZhy{}\PYGZhy{}\PYGZhy{}\PYGZhy{}\PYGZhy{}\PYGZhy{}\PYGZhy{}\PYGZhy{}\PYGZhy{}\PYGZhy{} snip \PYGZhy{}\PYGZhy{}\PYGZhy{}\PYGZhy{}\PYGZhy{}\PYGZhy{}\PYGZhy{}\PYGZhy{}\PYGZhy{}\PYGZhy{}\PYGZhy{}\PYGZhy{}\PYGZhy{}\PYGZhy{}\PYGZhy{}\PYGZhy{}\PYGZhy{}\PYGZhy{}\PYGZgt{}8

test\PYGZus{}00 (\PYGZus{}\PYGZus{}main\PYGZus{}\PYGZus{}.TestLogPass\PYGZus{}UpIndirect)
TestLogPass\PYGZus{}UpIndirect.test\PYGZus{}00(): 3 LR, 5 fr, 4 ch. setFrameSet() All. ... ok
test\PYGZus{}01 (\PYGZus{}\PYGZus{}main\PYGZus{}\PYGZus{}.TestLogPass\PYGZus{}UpIndirect)
TestLogPass\PYGZus{}UpIndirect.test\PYGZus{}01(): 3 LR, 5 fr, 4 ch. setFrameSet() theFrSl=slice(0,16,2). ... ok
test\PYGZus{}02 (\PYGZus{}\PYGZus{}main\PYGZus{}\PYGZus{}.TestLogPass\PYGZus{}UpIndirect)
TestLogPass\PYGZus{}UpIndirect.test\PYGZus{}02(): 3 LR, 5 fr, 4 ch. setFrameSet() theFrSl=slice(2,4,2). ... ok
test\PYGZus{}03 (\PYGZus{}\PYGZus{}main\PYGZus{}\PYGZus{}.TestLogPass\PYGZus{}UpIndirect)
TestLogPass\PYGZus{}UpIndirect.test\PYGZus{}03(): 3 LR, 5 fr, 4 ch. setFrameSet() theFrSl=slice(1,5,2). ... ok
test\PYGZus{}10 (\PYGZus{}\PYGZus{}main\PYGZus{}\PYGZus{}.TestLogPass\PYGZus{}UpIndirect)
TestLogPass\PYGZus{}UpIndirect.test\PYGZus{}10(): genFrameSetHeadings() ... ok

\PYGZhy{}\PYGZhy{}\PYGZhy{}\PYGZhy{}\PYGZhy{}\PYGZhy{}\PYGZhy{}\PYGZhy{}\PYGZhy{}\PYGZhy{}\PYGZhy{}\PYGZhy{}\PYGZhy{}\PYGZhy{}\PYGZhy{}\PYGZhy{}\PYGZhy{}\PYGZhy{}\PYGZhy{}\PYGZhy{}\PYGZhy{}\PYGZhy{}\PYGZhy{}\PYGZhy{}\PYGZhy{}\PYGZhy{}\PYGZhy{}\PYGZhy{}\PYGZhy{}\PYGZhy{}\PYGZhy{}\PYGZhy{}\PYGZhy{}\PYGZhy{}\PYGZhy{}\PYGZhy{}\PYGZhy{}\PYGZhy{}\PYGZhy{}\PYGZhy{}\PYGZhy{}\PYGZhy{}\PYGZhy{}\PYGZhy{}\PYGZhy{}\PYGZhy{}\PYGZhy{}\PYGZhy{}\PYGZhy{}\PYGZhy{}\PYGZhy{}\PYGZhy{}\PYGZhy{}\PYGZhy{}\PYGZhy{}\PYGZhy{}\PYGZhy{}\PYGZhy{}\PYGZhy{}\PYGZhy{}\PYGZhy{}\PYGZhy{}\PYGZhy{}\PYGZhy{}\PYGZhy{}\PYGZhy{}\PYGZhy{}\PYGZhy{}\PYGZhy{}\PYGZhy{}
Ran 54 tests in 0.507s

OK
CPU time =    0.510 (S)
Bye, bye!

\PYGZdl{} coverage report \PYGZhy{}m
Name     Stmts   Miss  Cover   Missing
\PYGZhy{}\PYGZhy{}\PYGZhy{}\PYGZhy{}\PYGZhy{}\PYGZhy{}\PYGZhy{}\PYGZhy{}\PYGZhy{}\PYGZhy{}\PYGZhy{}\PYGZhy{}\PYGZhy{}\PYGZhy{}\PYGZhy{}\PYGZhy{}\PYGZhy{}\PYGZhy{}\PYGZhy{}\PYGZhy{}\PYGZhy{}\PYGZhy{}\PYGZhy{}\PYGZhy{}\PYGZhy{}\PYGZhy{}\PYGZhy{}\PYGZhy{}\PYGZhy{}\PYGZhy{}\PYGZhy{}\PYGZhy{}\PYGZhy{}\PYGZhy{}\PYGZhy{}\PYGZhy{}\PYGZhy{}\PYGZhy{}\PYGZhy{}\PYGZhy{}\PYGZhy{}\PYGZhy{}\PYGZhy{}\PYGZhy{}\PYGZhy{}\PYGZhy{}\PYGZhy{}\PYGZhy{}\PYGZhy{}\PYGZhy{}\PYGZhy{}\PYGZhy{}\PYGZhy{}\PYGZhy{}\PYGZhy{}\PYGZhy{}\PYGZhy{}\PYGZhy{}\PYGZhy{}\PYGZhy{}\PYGZhy{}\PYGZhy{}\PYGZhy{}\PYGZhy{}\PYGZhy{}\PYGZhy{}\PYGZhy{}\PYGZhy{}\PYGZhy{}\PYGZhy{}\PYGZhy{}\PYGZhy{}\PYGZhy{}\PYGZhy{}\PYGZhy{}\PYGZhy{}\PYGZhy{}\PYGZhy{}\PYGZhy{}\PYGZhy{}\PYGZhy{}\PYGZhy{}\PYGZhy{}\PYGZhy{}\PYGZhy{}\PYGZhy{}\PYGZhy{}\PYGZhy{}\PYGZhy{}\PYGZhy{}\PYGZhy{}\PYGZhy{}\PYGZhy{}\PYGZhy{}\PYGZhy{}\PYGZhy{}\PYGZhy{}\PYGZhy{}\PYGZhy{}\PYGZhy{}\PYGZhy{}\PYGZhy{}\PYGZhy{}\PYGZhy{}\PYGZhy{}\PYGZhy{}

8\PYGZlt{}\PYGZhy{}\PYGZhy{}\PYGZhy{}\PYGZhy{}\PYGZhy{}\PYGZhy{}\PYGZhy{}\PYGZhy{}\PYGZhy{}\PYGZhy{}\PYGZhy{}\PYGZhy{}\PYGZhy{}\PYGZhy{} snip \PYGZhy{}\PYGZhy{}\PYGZhy{}\PYGZhy{}\PYGZhy{}\PYGZhy{}\PYGZhy{}\PYGZhy{}\PYGZhy{}\PYGZhy{}\PYGZhy{}\PYGZhy{}\PYGZhy{}\PYGZhy{}\PYGZhy{}\PYGZhy{}\PYGZhy{}\PYGZhy{}\PYGZgt{}8

LogPass    273     27    90\PYGZpc{}   142, 167, 182, 187, 203, 216\PYGZhy{}218, 232, 258, 273\PYGZhy{}275, 319, 324, 412, 416, 457\PYGZhy{}459, 513, 527, 541, 557, 572, 646\PYGZhy{}647

8\PYGZlt{}\PYGZhy{}\PYGZhy{}\PYGZhy{}\PYGZhy{}\PYGZhy{}\PYGZhy{}\PYGZhy{}\PYGZhy{}\PYGZhy{}\PYGZhy{}\PYGZhy{}\PYGZhy{}\PYGZhy{}\PYGZhy{} snip \PYGZhy{}\PYGZhy{}\PYGZhy{}\PYGZhy{}\PYGZhy{}\PYGZhy{}\PYGZhy{}\PYGZhy{}\PYGZhy{}\PYGZhy{}\PYGZhy{}\PYGZhy{}\PYGZhy{}\PYGZhy{}\PYGZhy{}\PYGZhy{}\PYGZhy{}\PYGZhy{}\PYGZgt{}8

\PYGZhy{}\PYGZhy{}\PYGZhy{}\PYGZhy{}\PYGZhy{}\PYGZhy{}\PYGZhy{}\PYGZhy{}\PYGZhy{}\PYGZhy{}\PYGZhy{}\PYGZhy{}\PYGZhy{}\PYGZhy{}\PYGZhy{}\PYGZhy{}\PYGZhy{}\PYGZhy{}\PYGZhy{}\PYGZhy{}\PYGZhy{}\PYGZhy{}\PYGZhy{}\PYGZhy{}\PYGZhy{}\PYGZhy{}\PYGZhy{}\PYGZhy{}\PYGZhy{}\PYGZhy{}\PYGZhy{}\PYGZhy{}\PYGZhy{}\PYGZhy{}\PYGZhy{}\PYGZhy{}\PYGZhy{}\PYGZhy{}\PYGZhy{}\PYGZhy{}\PYGZhy{}\PYGZhy{}\PYGZhy{}\PYGZhy{}\PYGZhy{}\PYGZhy{}\PYGZhy{}\PYGZhy{}\PYGZhy{}\PYGZhy{}\PYGZhy{}\PYGZhy{}\PYGZhy{}\PYGZhy{}\PYGZhy{}\PYGZhy{}\PYGZhy{}\PYGZhy{}\PYGZhy{}\PYGZhy{}\PYGZhy{}\PYGZhy{}\PYGZhy{}\PYGZhy{}\PYGZhy{}\PYGZhy{}\PYGZhy{}\PYGZhy{}\PYGZhy{}\PYGZhy{}\PYGZhy{}\PYGZhy{}\PYGZhy{}\PYGZhy{}\PYGZhy{}\PYGZhy{}\PYGZhy{}\PYGZhy{}\PYGZhy{}\PYGZhy{}\PYGZhy{}\PYGZhy{}\PYGZhy{}\PYGZhy{}\PYGZhy{}\PYGZhy{}\PYGZhy{}\PYGZhy{}\PYGZhy{}\PYGZhy{}\PYGZhy{}\PYGZhy{}\PYGZhy{}\PYGZhy{}\PYGZhy{}\PYGZhy{}\PYGZhy{}\PYGZhy{}\PYGZhy{}\PYGZhy{}\PYGZhy{}\PYGZhy{}\PYGZhy{}\PYGZhy{}\PYGZhy{}\PYGZhy{}
TOTAL     4586   1783    61\PYGZpc{}
\end{sphinxVerbatim}


\subsubsection{The LIS Frame Set}
\label{\detokenize{ref/LIS/core/FrameSet:the-lis-frame-set}}\label{\detokenize{ref/LIS/core/FrameSet::doc}}
This describes the LIS FrameSet that contains the binary frame data. It is effectively a wrapper around a 2-D numpy array with specific APIs to interface that with a channel specific shape. Importantly FrameSets can be partial in that they need not hold all the data for every frame and every channel. Instead they can hold data for the frames and channels specified by the caller.

The FrameSet module is located in \sphinxcode{src/TotalDepth/LIS/core} and can be imported thus:

\begin{sphinxVerbatim}[commandchars=\\\{\}]
\PYG{k+kn}{from} \PYG{n+nn}{TotalDepth}\PYG{n+nn}{.}\PYG{n+nn}{LIS}\PYG{n+nn}{.}\PYG{n+nn}{core} \PYG{k}{import} \PYG{n}{FrameSet}
\end{sphinxVerbatim}


\paragraph{FrameSet Internals}
\label{\detokenize{ref/LIS/core/FrameSet:frameset-internals}}
The internal representation of the FrameSet uses only Python and numpy structures and types, it is thus ignorant of the LIS file format or data structures apart from that described below.


\subparagraph{LIS Dependancies}
\label{\detokenize{ref/LIS/core/FrameSet:lis-dependancies}}
In the current version of the code there are a number of dependencies on knowledge of the LIS file format (or at least TotalDepth’s representation of that format):
\begin{itemize}
\item {} 
\sphinxcode{ChArTe} objects are constructed from LIS Datum Specification Block objects.

\item {} 
\sphinxcode{XAxisDecl} consists of a straight copy of various Entry Block values.

\item {} 
\sphinxcode{XAxisDecl} relies on the LIS value of the up/down flag.

\item {} 
\sphinxcode{FrameSet} is constructed with a DFSR and extracts the absent value form the DFSR as well as constructing ChArTe and XAxisDecl objects.

\item {} \begin{description}
\item[{\sphinxcode{FrameSet} has quite a lot of dependencies of Representation codes these are mainly used in two ways.}] \leavevmode\begin{itemize}
\item {} 
Interpreting Dipmeter sub-channels.

\item {} 
\sphinxcode{setFrameBytes()} uses the RepCode module directly to convert bytes to values for a channel.

\end{itemize}

\end{description}

\end{itemize}

These dependencies restrict the use of FrameSet to processing LIS data, if they were removed then FrameSet could be used for other file formats but there is no obvious use case for that as, apart from LIS, TotalDepth supports LAS (trivially simple frame sets) and will support RP66 at some point. The latter may trigger a refactoring of the FrameSet module.


\paragraph{FrameSet API Reference}
\label{\detokenize{ref/LIS/core/FrameSet:module-TotalDepth.LIS.core.FrameSet}}\label{\detokenize{ref/LIS/core/FrameSet:frameset-api-reference}}\index{TotalDepth.LIS.core.FrameSet (module)}
The FrameSet module provides a means of representing LIS frame data.

Created on 10 Jan 2011
\index{ExceptionFrameSet}

\begin{fulllineitems}
\phantomsection\label{\detokenize{ref/LIS/core/FrameSet:TotalDepth.LIS.core.FrameSet.ExceptionFrameSet}}\pysigline{\sphinxbfcode{exception }\sphinxcode{TotalDepth.LIS.core.FrameSet.}\sphinxbfcode{ExceptionFrameSet}}
Specialisation of exception for FrameSet.

\end{fulllineitems}

\index{ExceptionFrameSetEmpty}

\begin{fulllineitems}
\phantomsection\label{\detokenize{ref/LIS/core/FrameSet:TotalDepth.LIS.core.FrameSet.ExceptionFrameSetEmpty}}\pysigline{\sphinxbfcode{exception }\sphinxcode{TotalDepth.LIS.core.FrameSet.}\sphinxbfcode{ExceptionFrameSetEmpty}}
Raised when an illegal operation is performed on a FrameSet.

\end{fulllineitems}

\index{ExceptionFrameSetNULLSpacing}

\begin{fulllineitems}
\phantomsection\label{\detokenize{ref/LIS/core/FrameSet:TotalDepth.LIS.core.FrameSet.ExceptionFrameSetNULLSpacing}}\pysigline{\sphinxbfcode{exception }\sphinxcode{TotalDepth.LIS.core.FrameSet.}\sphinxbfcode{ExceptionFrameSetNULLSpacing}}
Raised when FrameSet depends on a frame spacing that can not be determined.

\end{fulllineitems}

\index{ExceptionFrameSetMixedChannels}

\begin{fulllineitems}
\phantomsection\label{\detokenize{ref/LIS/core/FrameSet:TotalDepth.LIS.core.FrameSet.ExceptionFrameSetMixedChannels}}\pysigline{\sphinxbfcode{exception }\sphinxcode{TotalDepth.LIS.core.FrameSet.}\sphinxbfcode{ExceptionFrameSetMixedChannels}}
Raised when generating values for multiple channels where the channels are not of the same shape i.e. number of samples, butsts.

\end{fulllineitems}

\index{chkIdx() (in module TotalDepth.LIS.core.FrameSet)}

\begin{fulllineitems}
\phantomsection\label{\detokenize{ref/LIS/core/FrameSet:TotalDepth.LIS.core.FrameSet.chkIdx}}\pysiglinewithargsret{\sphinxcode{TotalDepth.LIS.core.FrameSet.}\sphinxbfcode{chkIdx}}{\emph{i}, \emph{l}, \emph{msg}}{}
Global function for checking indexes and raising an IndexError.
msg is expected to have two format fields that take i and l respectively.

\end{fulllineitems}

\index{sliceDefaults() (in module TotalDepth.LIS.core.FrameSet)}

\begin{fulllineitems}
\phantomsection\label{\detokenize{ref/LIS/core/FrameSet:TotalDepth.LIS.core.FrameSet.sliceDefaults}}\pysiglinewithargsret{\sphinxcode{TotalDepth.LIS.core.FrameSet.}\sphinxbfcode{sliceDefaults}}{\emph{theSl}}{}
Returns a new slice with start=None as 0 and step=None as 1.

\end{fulllineitems}

\index{DataSeqBase (class in TotalDepth.LIS.core.FrameSet)}

\begin{fulllineitems}
\phantomsection\label{\detokenize{ref/LIS/core/FrameSet:TotalDepth.LIS.core.FrameSet.DataSeqBase}}\pysigline{\sphinxbfcode{class }\sphinxcode{TotalDepth.LIS.core.FrameSet.}\sphinxbfcode{DataSeqBase}}
Base class for a sequence of objects.

\end{fulllineitems}

\index{SuChArTe (class in TotalDepth.LIS.core.FrameSet)}

\begin{fulllineitems}
\phantomsection\label{\detokenize{ref/LIS/core/FrameSet:TotalDepth.LIS.core.FrameSet.SuChArTe}}\pysigline{\sphinxbfcode{class }\sphinxcode{TotalDepth.LIS.core.FrameSet.}\sphinxbfcode{SuChArTe}}
Sub-channel Array Template.
\index{numValues (TotalDepth.LIS.core.FrameSet.SuChArTe attribute)}

\begin{fulllineitems}
\phantomsection\label{\detokenize{ref/LIS/core/FrameSet:TotalDepth.LIS.core.FrameSet.SuChArTe.numValues}}\pysigline{\sphinxbfcode{numValues}}
The total number of values in this sub-channel.

\end{fulllineitems}

\index{index() (TotalDepth.LIS.core.FrameSet.SuChArTe method)}

\begin{fulllineitems}
\phantomsection\label{\detokenize{ref/LIS/core/FrameSet:TotalDepth.LIS.core.FrameSet.SuChArTe.index}}\pysiglinewithargsret{\sphinxbfcode{index}}{\emph{theS}, \emph{theB}}{}
Returns the index of a particular sample or channel with bounds checking.

WARNING: This not correct for Dipmeter sub-channels.

\end{fulllineitems}


\end{fulllineitems}

\index{ChArTe (class in TotalDepth.LIS.core.FrameSet)}

\begin{fulllineitems}
\phantomsection\label{\detokenize{ref/LIS/core/FrameSet:TotalDepth.LIS.core.FrameSet.ChArTe}}\pysiglinewithargsret{\sphinxbfcode{class }\sphinxcode{TotalDepth.LIS.core.FrameSet.}\sphinxbfcode{ChArTe}}{\emph{theDsb}}{}
Channel Array Template. Constructed with a DatumSpecBlock object.

\sphinxstylestrong{Implementation note:}
\sphinxcode{\_index()} and \sphinxcode{\_dipmeterIndex()} have no bounds checking but are
significantly faster for those routines that access then with
pre-checked limits.
\index{numSubChannels (TotalDepth.LIS.core.FrameSet.ChArTe attribute)}

\begin{fulllineitems}
\phantomsection\label{\detokenize{ref/LIS/core/FrameSet:TotalDepth.LIS.core.FrameSet.ChArTe.numSubChannels}}\pysigline{\sphinxbfcode{numSubChannels}}
Number of sub-channels.

\end{fulllineitems}

\index{lisSize (TotalDepth.LIS.core.FrameSet.ChArTe attribute)}

\begin{fulllineitems}
\phantomsection\label{\detokenize{ref/LIS/core/FrameSet:TotalDepth.LIS.core.FrameSet.ChArTe.lisSize}}\pysigline{\sphinxbfcode{lisSize}}
Number of bytes per frame in the LIS representation.

\end{fulllineitems}

\index{subChOffsRange() (TotalDepth.LIS.core.FrameSet.ChArTe method)}

\begin{fulllineitems}
\phantomsection\label{\detokenize{ref/LIS/core/FrameSet:TotalDepth.LIS.core.FrameSet.ChArTe.subChOffsRange}}\pysiglinewithargsret{\sphinxbfcode{subChOffsRange}}{\emph{sc}}{}
Returns a range object that is the sub-channel offset in the frame
relative to the start of the channel.

\end{fulllineitems}

\index{subChOffsSlice() (TotalDepth.LIS.core.FrameSet.ChArTe method)}

\begin{fulllineitems}
\phantomsection\label{\detokenize{ref/LIS/core/FrameSet:TotalDepth.LIS.core.FrameSet.ChArTe.subChOffsSlice}}\pysiglinewithargsret{\sphinxbfcode{subChOffsSlice}}{\emph{sc}, \emph{chOfs=0}}{}
Returns a slice object that is the sub-channel offset in the frame
relative to the start of the channel.

\end{fulllineitems}

\index{index() (TotalDepth.LIS.core.FrameSet.ChArTe method)}

\begin{fulllineitems}
\phantomsection\label{\detokenize{ref/LIS/core/FrameSet:TotalDepth.LIS.core.FrameSet.ChArTe.index}}\pysiglinewithargsret{\sphinxbfcode{index}}{\emph{theSc}, \emph{theSa}, \emph{theBu}}{}
Returns the index of a particular sub-channel, sample and burst.
Order is (fastest changing first): burst, sample.
For those with sub-channels: sub-channel, sample
This has bounds checking.

\end{fulllineitems}

\index{dipmeterIndex() (TotalDepth.LIS.core.FrameSet.ChArTe method)}

\begin{fulllineitems}
\phantomsection\label{\detokenize{ref/LIS/core/FrameSet:TotalDepth.LIS.core.FrameSet.ChArTe.dipmeterIndex}}\pysiglinewithargsret{\sphinxbfcode{dipmeterIndex}}{\emph{theSc}, \emph{theSa}, \emph{theBu}}{}
Returns dipmeter data index. Bounds checking is performed.

\end{fulllineitems}


\end{fulllineitems}

\index{FrameSet (class in TotalDepth.LIS.core.FrameSet)}

\begin{fulllineitems}
\phantomsection\label{\detokenize{ref/LIS/core/FrameSet:TotalDepth.LIS.core.FrameSet.FrameSet}}\pysiglinewithargsret{\sphinxbfcode{class }\sphinxcode{TotalDepth.LIS.core.FrameSet.}\sphinxbfcode{FrameSet}}{\emph{theDfsr}, \emph{theFrameSlice}, \emph{theChS=None}, \emph{xAxisIndex=0}}{}
Contains the representation of a list of Frames and thus a
representation of and ‘matrix’ (non literal) of:
(frame, channel, sub-channel, sample, burst).
Commonly shortened to: (fr, ch, sc, sa, bu)
Effectively this is a wrapper around a numpy 2-D array that we treat
specially to get our 5-D array (LIS channels are not homogeneous).

Constructed with a DFSR, a slice of frame indexes and an optional
list of external channel indexes (defaults to all channels).
Duplicates in the theChS will be removed and it will be sorted.

xAxisIndex is the external channel index of the X axis (ignored if indirect X).
\index{NUMPY\_DATA\_TYPE (TotalDepth.LIS.core.FrameSet.FrameSet attribute)}

\begin{fulllineitems}
\phantomsection\label{\detokenize{ref/LIS/core/FrameSet:TotalDepth.LIS.core.FrameSet.FrameSet.NUMPY_DATA_TYPE}}\pysigline{\sphinxbfcode{NUMPY\_DATA\_TYPE}\sphinxbfcode{ = 'float64'}}
Data type used in the underlying numpy array.

\end{fulllineitems}

\index{longStr() (TotalDepth.LIS.core.FrameSet.FrameSet method)}

\begin{fulllineitems}
\phantomsection\label{\detokenize{ref/LIS/core/FrameSet:TotalDepth.LIS.core.FrameSet.FrameSet.longStr}}\pysiglinewithargsret{\sphinxbfcode{longStr}}{}{}
Returns a long string that describes me.

\end{fulllineitems}

\index{dumpFrames() (TotalDepth.LIS.core.FrameSet.FrameSet method)}

\begin{fulllineitems}
\phantomsection\label{\detokenize{ref/LIS/core/FrameSet:TotalDepth.LIS.core.FrameSet.FrameSet.dumpFrames}}\pysiglinewithargsret{\sphinxbfcode{dumpFrames}}{\emph{theS=\textless{}\_io.TextIOWrapper name='\textless{}stdout\textgreater{}' mode='w' encoding='UTF-8'\textgreater{}}}{}
Dump the frames to the stream.

\end{fulllineitems}

\index{nbytes (TotalDepth.LIS.core.FrameSet.FrameSet attribute)}

\begin{fulllineitems}
\phantomsection\label{\detokenize{ref/LIS/core/FrameSet:TotalDepth.LIS.core.FrameSet.FrameSet.nbytes}}\pysigline{\sphinxbfcode{nbytes}}
Returns the number of bytes in the underlying array implementation.

\end{fulllineitems}

\index{lisSize (TotalDepth.LIS.core.FrameSet.FrameSet attribute)}

\begin{fulllineitems}
\phantomsection\label{\detokenize{ref/LIS/core/FrameSet:TotalDepth.LIS.core.FrameSet.FrameSet.lisSize}}\pysigline{\sphinxbfcode{lisSize}}
The number of LIS bytes that make up this FrameSet.

\end{fulllineitems}

\index{numFrames (TotalDepth.LIS.core.FrameSet.FrameSet attribute)}

\begin{fulllineitems}
\phantomsection\label{\detokenize{ref/LIS/core/FrameSet:TotalDepth.LIS.core.FrameSet.FrameSet.numFrames}}\pysigline{\sphinxbfcode{numFrames}}
The number of frames currently in this FrameSet.

\end{fulllineitems}

\index{valuesPerFrame (TotalDepth.LIS.core.FrameSet.FrameSet attribute)}

\begin{fulllineitems}
\phantomsection\label{\detokenize{ref/LIS/core/FrameSet:TotalDepth.LIS.core.FrameSet.FrameSet.valuesPerFrame}}\pysigline{\sphinxbfcode{valuesPerFrame}}
The number of values in each frame currently in this FrameSet.

\end{fulllineitems}

\index{numValues (TotalDepth.LIS.core.FrameSet.FrameSet attribute)}

\begin{fulllineitems}
\phantomsection\label{\detokenize{ref/LIS/core/FrameSet:TotalDepth.LIS.core.FrameSet.FrameSet.numValues}}\pysigline{\sphinxbfcode{numValues}}
The total number of values in the array.

\end{fulllineitems}

\index{frames (TotalDepth.LIS.core.FrameSet.FrameSet attribute)}

\begin{fulllineitems}
\phantomsection\label{\detokenize{ref/LIS/core/FrameSet:TotalDepth.LIS.core.FrameSet.FrameSet.frames}}\pysigline{\sphinxbfcode{frames}}
Gives access to the raw numpy array.

\end{fulllineitems}

\index{frame() (TotalDepth.LIS.core.FrameSet.FrameSet method)}

\begin{fulllineitems}
\phantomsection\label{\detokenize{ref/LIS/core/FrameSet:TotalDepth.LIS.core.FrameSet.FrameSet.frame}}\pysiglinewithargsret{\sphinxbfcode{frame}}{\emph{fr}}{}
Returns a specific frame.

\end{fulllineitems}

\index{xAxisValue() (TotalDepth.LIS.core.FrameSet.FrameSet method)}

\begin{fulllineitems}
\phantomsection\label{\detokenize{ref/LIS/core/FrameSet:TotalDepth.LIS.core.FrameSet.FrameSet.xAxisValue}}\pysiglinewithargsret{\sphinxbfcode{xAxisValue}}{\emph{fr}}{}
Returns the X axis value for the frame when indirect X is used or None.

\end{fulllineitems}

\index{xAxisStep() (TotalDepth.LIS.core.FrameSet.FrameSet method)}

\begin{fulllineitems}
\phantomsection\label{\detokenize{ref/LIS/core/FrameSet:TotalDepth.LIS.core.FrameSet.FrameSet.xAxisStep}}\pysiglinewithargsret{\sphinxbfcode{xAxisStep}}{\emph{numFr}}{}
The distance stepped by numFr.

\end{fulllineitems}

\index{intFrameNum() (TotalDepth.LIS.core.FrameSet.FrameSet method)}

\begin{fulllineitems}
\phantomsection\label{\detokenize{ref/LIS/core/FrameSet:TotalDepth.LIS.core.FrameSet.FrameSet.intFrameNum}}\pysiglinewithargsret{\sphinxbfcode{intFrameNum}}{\emph{theExtFrameNum}}{}
Given an external frame number this returns the internal frame number.
Will raise an IndexError if the internal frame number is out of
range or the external frame number is not in the caller specified
slice object.

\end{fulllineitems}

\index{extFrameNum() (TotalDepth.LIS.core.FrameSet.FrameSet method)}

\begin{fulllineitems}
\phantomsection\label{\detokenize{ref/LIS/core/FrameSet:TotalDepth.LIS.core.FrameSet.FrameSet.extFrameNum}}\pysiglinewithargsret{\sphinxbfcode{extFrameNum}}{\emph{theIntFrameNum}}{}
Given an internal frame number this returns the external frame number.
This does \_not\_ test that the internal frame number exists.

\end{fulllineitems}

\index{numChannels (TotalDepth.LIS.core.FrameSet.FrameSet attribute)}

\begin{fulllineitems}
\phantomsection\label{\detokenize{ref/LIS/core/FrameSet:TotalDepth.LIS.core.FrameSet.FrameSet.numChannels}}\pysigline{\sphinxbfcode{numChannels}}
The number of \_internal\_ channels.

\end{fulllineitems}

\index{numSubChannels() (TotalDepth.LIS.core.FrameSet.FrameSet method)}

\begin{fulllineitems}
\phantomsection\label{\detokenize{ref/LIS/core/FrameSet:TotalDepth.LIS.core.FrameSet.FrameSet.numSubChannels}}\pysiglinewithargsret{\sphinxbfcode{numSubChannels}}{\emph{theChExt}}{}
Number of sub-channels for an external channel.

\end{fulllineitems}

\index{numSamples() (TotalDepth.LIS.core.FrameSet.FrameSet method)}

\begin{fulllineitems}
\phantomsection\label{\detokenize{ref/LIS/core/FrameSet:TotalDepth.LIS.core.FrameSet.FrameSet.numSamples}}\pysiglinewithargsret{\sphinxbfcode{numSamples}}{\emph{theChExt}, \emph{theSc}}{}
Number of samples for the external channel, sub-channel.

\end{fulllineitems}

\index{numBursts() (TotalDepth.LIS.core.FrameSet.FrameSet method)}

\begin{fulllineitems}
\phantomsection\label{\detokenize{ref/LIS/core/FrameSet:TotalDepth.LIS.core.FrameSet.FrameSet.numBursts}}\pysiglinewithargsret{\sphinxbfcode{numBursts}}{\emph{theChExt}, \emph{theSc}}{}
Number of bursts for the external channel, sub-channel.

\end{fulllineitems}

\index{isIndirectX (TotalDepth.LIS.core.FrameSet.FrameSet attribute)}

\begin{fulllineitems}
\phantomsection\label{\detokenize{ref/LIS/core/FrameSet:TotalDepth.LIS.core.FrameSet.FrameSet.isIndirectX}}\pysigline{\sphinxbfcode{isIndirectX}}
True if there is an indirect X axis, False otherwise.

\end{fulllineitems}

\index{xAxisDecl (TotalDepth.LIS.core.FrameSet.FrameSet attribute)}

\begin{fulllineitems}
\phantomsection\label{\detokenize{ref/LIS/core/FrameSet:TotalDepth.LIS.core.FrameSet.FrameSet.xAxisDecl}}\pysigline{\sphinxbfcode{xAxisDecl}}
The XAxisDecl object created from the DFSR.

\end{fulllineitems}

\index{setFrameBytes() (TotalDepth.LIS.core.FrameSet.FrameSet method)}

\begin{fulllineitems}
\phantomsection\label{\detokenize{ref/LIS/core/FrameSet:TotalDepth.LIS.core.FrameSet.FrameSet.setFrameBytes}}\pysiglinewithargsret{\sphinxbfcode{setFrameBytes}}{\emph{by}, \emph{fr}, \emph{chFrom}, \emph{chTo}}{}
Given bytes, convert from Rep Codes to ‘float64’ and populates the
appropriate frame and channel(s). This has random write access so the
caller needs to be sensitive to the frame and channel location.
fr is the internal frame position to write the data to.
chFrom, chTo are inclusive and represent external channel indices.

\sphinxstylestrong{WARNING:} It is rare to use this API directly, instead a FrameSet object
is usually a member of a LogPass object and that uses this API as a LogPass is
capable of finding and reading bytes objects in a file (a LogPass uses
RLE and Type01Plan objects to do this efficiently).

\end{fulllineitems}

\index{setIndirectX() (TotalDepth.LIS.core.FrameSet.FrameSet method)}

\begin{fulllineitems}
\phantomsection\label{\detokenize{ref/LIS/core/FrameSet:TotalDepth.LIS.core.FrameSet.FrameSet.setIndirectX}}\pysiglinewithargsret{\sphinxbfcode{setIndirectX}}{\emph{fr}, \emph{val}}{}
Sets an indirect X axis value directly, for example with an EXTRAPOLATE event.

\end{fulllineitems}

\index{internalChIdx() (TotalDepth.LIS.core.FrameSet.FrameSet method)}

\begin{fulllineitems}
\phantomsection\label{\detokenize{ref/LIS/core/FrameSet:TotalDepth.LIS.core.FrameSet.FrameSet.internalChIdx}}\pysiglinewithargsret{\sphinxbfcode{internalChIdx}}{\emph{chIdxExt}}{}
Return the internal channel index from the external one.

\end{fulllineitems}

\index{valueIdxStartExtCh() (TotalDepth.LIS.core.FrameSet.FrameSet method)}

\begin{fulllineitems}
\phantomsection\label{\detokenize{ref/LIS/core/FrameSet:TotalDepth.LIS.core.FrameSet.FrameSet.valueIdxStartExtCh}}\pysiglinewithargsret{\sphinxbfcode{valueIdxStartExtCh}}{\emph{ch}}{}
The index in the frame of the first value for external channel.

\end{fulllineitems}

\index{valueIdxInFrame() (TotalDepth.LIS.core.FrameSet.FrameSet method)}

\begin{fulllineitems}
\phantomsection\label{\detokenize{ref/LIS/core/FrameSet:TotalDepth.LIS.core.FrameSet.FrameSet.valueIdxInFrame}}\pysiglinewithargsret{\sphinxbfcode{valueIdxInFrame}}{\emph{ch}, \emph{sc}, \emph{sa}, \emph{bu}}{}
The index in the frame of the value for (external channel, sub-channel, sample, burst).
TODO: Provide an API that returns a numpy view of a ch/sc on the frameset.

\end{fulllineitems}

\index{value() (TotalDepth.LIS.core.FrameSet.FrameSet method)}

\begin{fulllineitems}
\phantomsection\label{\detokenize{ref/LIS/core/FrameSet:TotalDepth.LIS.core.FrameSet.FrameSet.value}}\pysiglinewithargsret{\sphinxbfcode{value}}{\emph{fr}, \emph{ch}, \emph{sc}, \emph{sa}, \emph{bu}}{}
Returns a single value from an: internal fr, external ch, sc, sa, bu.
This is very good at random access but can be quite slow for iteration
compared to the generators.

\end{fulllineitems}

\index{frameView() (TotalDepth.LIS.core.FrameSet.FrameSet method)}

\begin{fulllineitems}
\phantomsection\label{\detokenize{ref/LIS/core/FrameSet:TotalDepth.LIS.core.FrameSet.FrameSet.frameView}}\pysiglinewithargsret{\sphinxbfcode{frameView}}{\emph{chIdxExt}, \emph{sc}}{}
Returns a numpy array that is a view of the current frame set for an
external channel and sub channel.

\end{fulllineitems}

\index{genExtChIndexes() (TotalDepth.LIS.core.FrameSet.FrameSet method)}

\begin{fulllineitems}
\phantomsection\label{\detokenize{ref/LIS/core/FrameSet:TotalDepth.LIS.core.FrameSet.FrameSet.genExtChIndexes}}\pysiglinewithargsret{\sphinxbfcode{genExtChIndexes}}{}{}
Generates an ordered list of external channel indexes for this
(possibly) partial frameset.

\end{fulllineitems}

\index{genChScValues() (TotalDepth.LIS.core.FrameSet.FrameSet method)}

\begin{fulllineitems}
\phantomsection\label{\detokenize{ref/LIS/core/FrameSet:TotalDepth.LIS.core.FrameSet.FrameSet.genChScValues}}\pysiglinewithargsret{\sphinxbfcode{genChScValues}}{\emph{ch}, \emph{sc=0}, \emph{chIsExternal=True}}{}
Generates values for the external channel and sub channel.
sc is ignored unless the channel has \textgreater{} 1 sub-channels.
If chIsExternal is True then ch is the external channel index otherwise
it is the internal index.
The value order is sample/burst.

\end{fulllineitems}

\index{genChScPoints() (TotalDepth.LIS.core.FrameSet.FrameSet method)}

\begin{fulllineitems}
\phantomsection\label{\detokenize{ref/LIS/core/FrameSet:TotalDepth.LIS.core.FrameSet.FrameSet.genChScPoints}}\pysiglinewithargsret{\sphinxbfcode{genChScPoints}}{\emph{ch}, \emph{sc=0}, \emph{chIsExt=True}}{}
Generates (xAxis, values) as numbers for the external channel and
sub-channel. sc is ignored unless the channel has \textgreater{} 1 sub-channels.
If chIsExt is True then ch is the external channel index otherwise
it is the internal index.

\end{fulllineitems}

\index{genMultipleChScPoints() (TotalDepth.LIS.core.FrameSet.FrameSet method)}

\begin{fulllineitems}
\phantomsection\label{\detokenize{ref/LIS/core/FrameSet:TotalDepth.LIS.core.FrameSet.FrameSet.genMultipleChScPoints}}\pysiglinewithargsret{\sphinxbfcode{genMultipleChScPoints}}{\emph{theChScS}, \emph{chIsExt=True}}{}
Generates (xAxis, (values, …)) as numbers for the list of
(channel, sub-channel) indexes.
An ExceptionFrameSetMixedChannels will be raised if all ch/sc channels
are not of the same form i.e. number of samples and bursts.
If chIsExt is True then ch is the external channel index otherwise
it is the internal index.

\end{fulllineitems}

\index{genAll() (TotalDepth.LIS.core.FrameSet.FrameSet method)}

\begin{fulllineitems}
\phantomsection\label{\detokenize{ref/LIS/core/FrameSet:TotalDepth.LIS.core.FrameSet.FrameSet.genAll}}\pysiglinewithargsret{\sphinxbfcode{genAll}}{}{}
Yields 6 item tuples (fr ext, ch ext, sc, sa, bu value).

\end{fulllineitems}

\index{accumulate() (TotalDepth.LIS.core.FrameSet.FrameSet method)}

\begin{fulllineitems}
\phantomsection\label{\detokenize{ref/LIS/core/FrameSet:TotalDepth.LIS.core.FrameSet.FrameSet.accumulate}}\pysiglinewithargsret{\sphinxbfcode{accumulate}}{\emph{theAccs}}{}
Calls .add() on every accumulator (with a unary function) for every
internal channel and returns an numpy array of (numSubCh, len(theAccs))
doubles. Each accumulator is expected to have \_\_init\_\_(), add() and
value() that returns a double implemented.
Will raise a ExceptionFrameSetEmpty if there are no values to analyse.
Will return None if theAccs is zero length.

\end{fulllineitems}


\end{fulllineitems}

\index{AccMin (class in TotalDepth.LIS.core.FrameSet)}

\begin{fulllineitems}
\phantomsection\label{\detokenize{ref/LIS/core/FrameSet:TotalDepth.LIS.core.FrameSet.AccMin}}\pysigline{\sphinxbfcode{class }\sphinxcode{TotalDepth.LIS.core.FrameSet.}\sphinxbfcode{AccMin}}
Accumulates the minimum value.
\index{add() (TotalDepth.LIS.core.FrameSet.AccMin method)}

\begin{fulllineitems}
\phantomsection\label{\detokenize{ref/LIS/core/FrameSet:TotalDepth.LIS.core.FrameSet.AccMin.add}}\pysiglinewithargsret{\sphinxbfcode{add}}{\emph{v}}{}
Add a new value.

\end{fulllineitems}

\index{value() (TotalDepth.LIS.core.FrameSet.AccMin method)}

\begin{fulllineitems}
\phantomsection\label{\detokenize{ref/LIS/core/FrameSet:TotalDepth.LIS.core.FrameSet.AccMin.value}}\pysiglinewithargsret{\sphinxbfcode{value}}{}{}
Return the result.

\end{fulllineitems}


\end{fulllineitems}

\index{AccMax (class in TotalDepth.LIS.core.FrameSet)}

\begin{fulllineitems}
\phantomsection\label{\detokenize{ref/LIS/core/FrameSet:TotalDepth.LIS.core.FrameSet.AccMax}}\pysigline{\sphinxbfcode{class }\sphinxcode{TotalDepth.LIS.core.FrameSet.}\sphinxbfcode{AccMax}}
Accumulates the maximum value.
\index{add() (TotalDepth.LIS.core.FrameSet.AccMax method)}

\begin{fulllineitems}
\phantomsection\label{\detokenize{ref/LIS/core/FrameSet:TotalDepth.LIS.core.FrameSet.AccMax.add}}\pysiglinewithargsret{\sphinxbfcode{add}}{\emph{v}}{}
Add a new value.

\end{fulllineitems}

\index{value() (TotalDepth.LIS.core.FrameSet.AccMax method)}

\begin{fulllineitems}
\phantomsection\label{\detokenize{ref/LIS/core/FrameSet:TotalDepth.LIS.core.FrameSet.AccMax.value}}\pysiglinewithargsret{\sphinxbfcode{value}}{}{}
Return the result.

\end{fulllineitems}


\end{fulllineitems}

\index{AccMean (class in TotalDepth.LIS.core.FrameSet)}

\begin{fulllineitems}
\phantomsection\label{\detokenize{ref/LIS/core/FrameSet:TotalDepth.LIS.core.FrameSet.AccMean}}\pysigline{\sphinxbfcode{class }\sphinxcode{TotalDepth.LIS.core.FrameSet.}\sphinxbfcode{AccMean}}
Accumulates the mean value.
\index{add() (TotalDepth.LIS.core.FrameSet.AccMean method)}

\begin{fulllineitems}
\phantomsection\label{\detokenize{ref/LIS/core/FrameSet:TotalDepth.LIS.core.FrameSet.AccMean.add}}\pysiglinewithargsret{\sphinxbfcode{add}}{\emph{v}}{}
Add a new value.

\end{fulllineitems}

\index{value() (TotalDepth.LIS.core.FrameSet.AccMean method)}

\begin{fulllineitems}
\phantomsection\label{\detokenize{ref/LIS/core/FrameSet:TotalDepth.LIS.core.FrameSet.AccMean.value}}\pysiglinewithargsret{\sphinxbfcode{value}}{}{}
Return the result.

\end{fulllineitems}


\end{fulllineitems}

\index{AccStDev (class in TotalDepth.LIS.core.FrameSet)}

\begin{fulllineitems}
\phantomsection\label{\detokenize{ref/LIS/core/FrameSet:TotalDepth.LIS.core.FrameSet.AccStDev}}\pysigline{\sphinxbfcode{class }\sphinxcode{TotalDepth.LIS.core.FrameSet.}\sphinxbfcode{AccStDev}}
Accumulates the standard deviation.
\index{add() (TotalDepth.LIS.core.FrameSet.AccStDev method)}

\begin{fulllineitems}
\phantomsection\label{\detokenize{ref/LIS/core/FrameSet:TotalDepth.LIS.core.FrameSet.AccStDev.add}}\pysiglinewithargsret{\sphinxbfcode{add}}{\emph{v}}{}
Add a new value.

\end{fulllineitems}

\index{value() (TotalDepth.LIS.core.FrameSet.AccStDev method)}

\begin{fulllineitems}
\phantomsection\label{\detokenize{ref/LIS/core/FrameSet:TotalDepth.LIS.core.FrameSet.AccStDev.value}}\pysiglinewithargsret{\sphinxbfcode{value}}{}{}
Return the result.

\end{fulllineitems}


\end{fulllineitems}

\index{AccCount (class in TotalDepth.LIS.core.FrameSet)}

\begin{fulllineitems}
\phantomsection\label{\detokenize{ref/LIS/core/FrameSet:TotalDepth.LIS.core.FrameSet.AccCount}}\pysigline{\sphinxbfcode{class }\sphinxcode{TotalDepth.LIS.core.FrameSet.}\sphinxbfcode{AccCount}}
Accumulates the number of values.
\index{add() (TotalDepth.LIS.core.FrameSet.AccCount method)}

\begin{fulllineitems}
\phantomsection\label{\detokenize{ref/LIS/core/FrameSet:TotalDepth.LIS.core.FrameSet.AccCount.add}}\pysiglinewithargsret{\sphinxbfcode{add}}{\emph{v}}{}
Add a new value.

\end{fulllineitems}

\index{value() (TotalDepth.LIS.core.FrameSet.AccCount method)}

\begin{fulllineitems}
\phantomsection\label{\detokenize{ref/LIS/core/FrameSet:TotalDepth.LIS.core.FrameSet.AccCount.value}}\pysiglinewithargsret{\sphinxbfcode{value}}{}{}
Return the result.

\end{fulllineitems}


\end{fulllineitems}

\index{AccDelta (class in TotalDepth.LIS.core.FrameSet)}

\begin{fulllineitems}
\phantomsection\label{\detokenize{ref/LIS/core/FrameSet:TotalDepth.LIS.core.FrameSet.AccDelta}}\pysigline{\sphinxbfcode{class }\sphinxcode{TotalDepth.LIS.core.FrameSet.}\sphinxbfcode{AccDelta}}
Base class for acumulating a count of first order differences.
\index{add() (TotalDepth.LIS.core.FrameSet.AccDelta method)}

\begin{fulllineitems}
\phantomsection\label{\detokenize{ref/LIS/core/FrameSet:TotalDepth.LIS.core.FrameSet.AccDelta.add}}\pysiglinewithargsret{\sphinxbfcode{add}}{\emph{v}}{}
Add a new value.

\end{fulllineitems}

\index{value() (TotalDepth.LIS.core.FrameSet.AccDelta method)}

\begin{fulllineitems}
\phantomsection\label{\detokenize{ref/LIS/core/FrameSet:TotalDepth.LIS.core.FrameSet.AccDelta.value}}\pysiglinewithargsret{\sphinxbfcode{value}}{}{}
Return the result.

\end{fulllineitems}


\end{fulllineitems}

\index{AccInc (class in TotalDepth.LIS.core.FrameSet)}

\begin{fulllineitems}
\phantomsection\label{\detokenize{ref/LIS/core/FrameSet:TotalDepth.LIS.core.FrameSet.AccInc}}\pysigline{\sphinxbfcode{class }\sphinxcode{TotalDepth.LIS.core.FrameSet.}\sphinxbfcode{AccInc}}
Counting how many values are an increase from the previous value.
\index{add() (TotalDepth.LIS.core.FrameSet.AccInc method)}

\begin{fulllineitems}
\phantomsection\label{\detokenize{ref/LIS/core/FrameSet:TotalDepth.LIS.core.FrameSet.AccInc.add}}\pysiglinewithargsret{\sphinxbfcode{add}}{\emph{v}}{}
Add a new value.

\end{fulllineitems}


\end{fulllineitems}

\index{AccEq (class in TotalDepth.LIS.core.FrameSet)}

\begin{fulllineitems}
\phantomsection\label{\detokenize{ref/LIS/core/FrameSet:TotalDepth.LIS.core.FrameSet.AccEq}}\pysigline{\sphinxbfcode{class }\sphinxcode{TotalDepth.LIS.core.FrameSet.}\sphinxbfcode{AccEq}}
Counting how many values are equal to the previous value.
\index{add() (TotalDepth.LIS.core.FrameSet.AccEq method)}

\begin{fulllineitems}
\phantomsection\label{\detokenize{ref/LIS/core/FrameSet:TotalDepth.LIS.core.FrameSet.AccEq.add}}\pysiglinewithargsret{\sphinxbfcode{add}}{\emph{v}}{}
Add a new value.

\end{fulllineitems}


\end{fulllineitems}

\index{AccDec (class in TotalDepth.LIS.core.FrameSet)}

\begin{fulllineitems}
\phantomsection\label{\detokenize{ref/LIS/core/FrameSet:TotalDepth.LIS.core.FrameSet.AccDec}}\pysigline{\sphinxbfcode{class }\sphinxcode{TotalDepth.LIS.core.FrameSet.}\sphinxbfcode{AccDec}}
Counting how many values are less than the previous value.
\index{add() (TotalDepth.LIS.core.FrameSet.AccDec method)}

\begin{fulllineitems}
\phantomsection\label{\detokenize{ref/LIS/core/FrameSet:TotalDepth.LIS.core.FrameSet.AccDec.add}}\pysiglinewithargsret{\sphinxbfcode{add}}{\emph{v}}{}
Add a new value.

\end{fulllineitems}


\end{fulllineitems}

\index{AccBias (class in TotalDepth.LIS.core.FrameSet)}

\begin{fulllineitems}
\phantomsection\label{\detokenize{ref/LIS/core/FrameSet:TotalDepth.LIS.core.FrameSet.AccBias}}\pysigline{\sphinxbfcode{class }\sphinxcode{TotalDepth.LIS.core.FrameSet.}\sphinxbfcode{AccBias}}
Measures increment, equal, decrement and computes bias which is:
(inc - dec) / total.
\index{add() (TotalDepth.LIS.core.FrameSet.AccBias method)}

\begin{fulllineitems}
\phantomsection\label{\detokenize{ref/LIS/core/FrameSet:TotalDepth.LIS.core.FrameSet.AccBias.add}}\pysiglinewithargsret{\sphinxbfcode{add}}{\emph{v}}{}
Add a new value.

\end{fulllineitems}

\index{value() (TotalDepth.LIS.core.FrameSet.AccBias method)}

\begin{fulllineitems}
\phantomsection\label{\detokenize{ref/LIS/core/FrameSet:TotalDepth.LIS.core.FrameSet.AccBias.value}}\pysiglinewithargsret{\sphinxbfcode{value}}{}{}
Return the result.

\end{fulllineitems}


\end{fulllineitems}

\index{AccDrift (class in TotalDepth.LIS.core.FrameSet)}

\begin{fulllineitems}
\phantomsection\label{\detokenize{ref/LIS/core/FrameSet:TotalDepth.LIS.core.FrameSet.AccDrift}}\pysigline{\sphinxbfcode{class }\sphinxcode{TotalDepth.LIS.core.FrameSet.}\sphinxbfcode{AccDrift}}
Measures drift i.e. the movement between the first and the last value.
\index{add() (TotalDepth.LIS.core.FrameSet.AccDrift method)}

\begin{fulllineitems}
\phantomsection\label{\detokenize{ref/LIS/core/FrameSet:TotalDepth.LIS.core.FrameSet.AccDrift.add}}\pysiglinewithargsret{\sphinxbfcode{add}}{\emph{v}}{}
Add a new value.

\end{fulllineitems}

\index{value() (TotalDepth.LIS.core.FrameSet.AccDrift method)}

\begin{fulllineitems}
\phantomsection\label{\detokenize{ref/LIS/core/FrameSet:TotalDepth.LIS.core.FrameSet.AccDrift.value}}\pysiglinewithargsret{\sphinxbfcode{value}}{}{}
Return the result.

\end{fulllineitems}


\end{fulllineitems}

\index{AccActivity (class in TotalDepth.LIS.core.FrameSet)}

\begin{fulllineitems}
\phantomsection\label{\detokenize{ref/LIS/core/FrameSet:TotalDepth.LIS.core.FrameSet.AccActivity}}\pysigline{\sphinxbfcode{class }\sphinxcode{TotalDepth.LIS.core.FrameSet.}\sphinxbfcode{AccActivity}}
Measures curve activity.
\index{add() (TotalDepth.LIS.core.FrameSet.AccActivity method)}

\begin{fulllineitems}
\phantomsection\label{\detokenize{ref/LIS/core/FrameSet:TotalDepth.LIS.core.FrameSet.AccActivity.add}}\pysiglinewithargsret{\sphinxbfcode{add}}{\emph{v}}{}
Add a new value.

\end{fulllineitems}

\index{value() (TotalDepth.LIS.core.FrameSet.AccActivity method)}

\begin{fulllineitems}
\phantomsection\label{\detokenize{ref/LIS/core/FrameSet:TotalDepth.LIS.core.FrameSet.AccActivity.value}}\pysiglinewithargsret{\sphinxbfcode{value}}{}{}
Return the result.

\end{fulllineitems}


\end{fulllineitems}



\paragraph{FrameSet Usage}
\label{\detokenize{ref/LIS/core/FrameSet:frameset-usage}}

\subsubsection{Run Length Encoding Module}
\label{\detokenize{ref/LIS/core/Rle:run-length-encoding-module}}\label{\detokenize{ref/LIS/core/Rle::doc}}\label{\detokenize{ref/LIS/core/Rle:totaldepth-lis-core-rle}}\phantomsection\label{\detokenize{ref/LIS/core/Rle:module-TotalDepth.LIS.core.Rle}}\index{TotalDepth.LIS.core.Rle (module)}
The RLE module provides Run Length Encoding suitable for recording the file
positions of a set of LIS Logical Records that represent frame data.

Created on 5 Jan 2011


\paragraph{class RLEItem()}
\label{\detokenize{ref/LIS/core/Rle:class-rleitem}}
A generic item in a Run Length Encoding list.


\paragraph{class RLE()}
\label{\detokenize{ref/LIS/core/Rle:class-rle}}
A generic Run Length Encoding list.


\paragraph{class RLEItemType01()}
\label{\detokenize{ref/LIS/core/Rle:class-rleitemtype01}}
A specialised item in a Run Length Encoding list for type 0/1 LIS Logical Records.


\paragraph{class RLEType01()}
\label{\detokenize{ref/LIS/core/Rle:class-rletype01}}
A specialised Run Length Encoding list for type 0/1 LIS Logical Records.


\paragraph{API Reference}
\label{\detokenize{ref/LIS/core/Rle:api-reference}}\index{RLEItem (class in TotalDepth.LIS.core.Rle)}

\begin{fulllineitems}
\phantomsection\label{\detokenize{ref/LIS/core/Rle:TotalDepth.LIS.core.Rle.RLEItem}}\pysiglinewithargsret{\sphinxbfcode{class }\sphinxcode{TotalDepth.LIS.core.Rle.}\sphinxbfcode{RLEItem}}{\emph{v}}{}
Class that represents a single entry in a Run Length Encoding set. v - The datum value.
\index{\_\_str\_\_() (TotalDepth.LIS.core.Rle.RLEItem method)}

\begin{fulllineitems}
\phantomsection\label{\detokenize{ref/LIS/core/Rle:TotalDepth.LIS.core.Rle.RLEItem.__str__}}\pysiglinewithargsret{\sphinxbfcode{\_\_str\_\_}}{}{}
String representation.

\end{fulllineitems}

\index{datum (TotalDepth.LIS.core.Rle.RLEItem attribute)}

\begin{fulllineitems}
\phantomsection\label{\detokenize{ref/LIS/core/Rle:TotalDepth.LIS.core.Rle.RLEItem.datum}}\pysigline{\sphinxbfcode{datum}}
The initial datum value.

\end{fulllineitems}

\index{stride (TotalDepth.LIS.core.Rle.RLEItem attribute)}

\begin{fulllineitems}
\phantomsection\label{\detokenize{ref/LIS/core/Rle:TotalDepth.LIS.core.Rle.RLEItem.stride}}\pysigline{\sphinxbfcode{stride}}
The stride as a number or None if there is only one entry.

\end{fulllineitems}

\index{repeat (TotalDepth.LIS.core.Rle.RLEItem attribute)}

\begin{fulllineitems}
\phantomsection\label{\detokenize{ref/LIS/core/Rle:TotalDepth.LIS.core.Rle.RLEItem.repeat}}\pysigline{\sphinxbfcode{repeat}}
The repeat count.

\end{fulllineitems}

\index{numValues() (TotalDepth.LIS.core.Rle.RLEItem method)}

\begin{fulllineitems}
\phantomsection\label{\detokenize{ref/LIS/core/Rle:TotalDepth.LIS.core.Rle.RLEItem.numValues}}\pysiglinewithargsret{\sphinxbfcode{numValues}}{}{}
Total number of record values.

\end{fulllineitems}

\index{add() (TotalDepth.LIS.core.Rle.RLEItem method)}

\begin{fulllineitems}
\phantomsection\label{\detokenize{ref/LIS/core/Rle:TotalDepth.LIS.core.Rle.RLEItem.add}}\pysiglinewithargsret{\sphinxbfcode{add}}{\emph{v}}{}
Returns True if v has been absorbed in this entry. False means a
new entry is required.

\end{fulllineitems}

\index{values() (TotalDepth.LIS.core.Rle.RLEItem method)}

\begin{fulllineitems}
\phantomsection\label{\detokenize{ref/LIS/core/Rle:TotalDepth.LIS.core.Rle.RLEItem.values}}\pysiglinewithargsret{\sphinxbfcode{values}}{}{}
Generates all values.

\end{fulllineitems}

\index{value() (TotalDepth.LIS.core.Rle.RLEItem method)}

\begin{fulllineitems}
\phantomsection\label{\detokenize{ref/LIS/core/Rle:TotalDepth.LIS.core.Rle.RLEItem.value}}\pysiglinewithargsret{\sphinxbfcode{value}}{\emph{i}}{}
Returns a particular value.

\end{fulllineitems}

\index{range() (TotalDepth.LIS.core.Rle.RLEItem method)}

\begin{fulllineitems}
\phantomsection\label{\detokenize{ref/LIS/core/Rle:TotalDepth.LIS.core.Rle.RLEItem.range}}\pysiglinewithargsret{\sphinxbfcode{range}}{}{}
Returns a range object that has (start, stop, step) or None if a single entry.

\end{fulllineitems}

\index{first() (TotalDepth.LIS.core.Rle.RLEItem method)}

\begin{fulllineitems}
\phantomsection\label{\detokenize{ref/LIS/core/Rle:TotalDepth.LIS.core.Rle.RLEItem.first}}\pysiglinewithargsret{\sphinxbfcode{first}}{}{}
Returns the first value.

\end{fulllineitems}

\index{last() (TotalDepth.LIS.core.Rle.RLEItem method)}

\begin{fulllineitems}
\phantomsection\label{\detokenize{ref/LIS/core/Rle:TotalDepth.LIS.core.Rle.RLEItem.last}}\pysiglinewithargsret{\sphinxbfcode{last}}{}{}
Returns the last value.

\end{fulllineitems}

\index{\_\_weakref\_\_ (TotalDepth.LIS.core.Rle.RLEItem attribute)}

\begin{fulllineitems}
\phantomsection\label{\detokenize{ref/LIS/core/Rle:TotalDepth.LIS.core.Rle.RLEItem.__weakref__}}\pysigline{\sphinxbfcode{\_\_weakref\_\_}}
list of weak references to the object (if defined)

\end{fulllineitems}


\end{fulllineitems}

\index{RLE (class in TotalDepth.LIS.core.Rle)}

\begin{fulllineitems}
\phantomsection\label{\detokenize{ref/LIS/core/Rle:TotalDepth.LIS.core.Rle.RLE}}\pysiglinewithargsret{\sphinxbfcode{class }\sphinxcode{TotalDepth.LIS.core.Rle.}\sphinxbfcode{RLE}}{\emph{theFunc=None}}{}
Class that represents Run Length Encoding.

theFunc - optional unary function to convert all values with.
\index{\_\_init\_\_() (TotalDepth.LIS.core.Rle.RLE method)}

\begin{fulllineitems}
\phantomsection\label{\detokenize{ref/LIS/core/Rle:TotalDepth.LIS.core.Rle.RLE.__init__}}\pysiglinewithargsret{\sphinxbfcode{\_\_init\_\_}}{\emph{theFunc=None}}{}
Constructor, optionally takes a unary function to convert all values with.

\end{fulllineitems}

\index{\_\_str\_\_() (TotalDepth.LIS.core.Rle.RLE method)}

\begin{fulllineitems}
\phantomsection\label{\detokenize{ref/LIS/core/Rle:TotalDepth.LIS.core.Rle.RLE.__str__}}\pysiglinewithargsret{\sphinxbfcode{\_\_str\_\_}}{}{}
String representation.

\end{fulllineitems}

\index{\_\_len\_\_() (TotalDepth.LIS.core.Rle.RLE method)}

\begin{fulllineitems}
\phantomsection\label{\detokenize{ref/LIS/core/Rle:TotalDepth.LIS.core.Rle.RLE.__len__}}\pysiglinewithargsret{\sphinxbfcode{\_\_len\_\_}}{}{}
The number of RLEItem(s).

\end{fulllineitems}

\index{\_\_getitem\_\_() (TotalDepth.LIS.core.Rle.RLE method)}

\begin{fulllineitems}
\phantomsection\label{\detokenize{ref/LIS/core/Rle:TotalDepth.LIS.core.Rle.RLE.__getitem__}}\pysiglinewithargsret{\sphinxbfcode{\_\_getitem\_\_}}{\emph{key}}{}
Returns a RLEItem.

\end{fulllineitems}

\index{numValues() (TotalDepth.LIS.core.Rle.RLE method)}

\begin{fulllineitems}
\phantomsection\label{\detokenize{ref/LIS/core/Rle:TotalDepth.LIS.core.Rle.RLE.numValues}}\pysiglinewithargsret{\sphinxbfcode{numValues}}{}{}
Total number of record values.

\end{fulllineitems}

\index{add() (TotalDepth.LIS.core.Rle.RLE method)}

\begin{fulllineitems}
\phantomsection\label{\detokenize{ref/LIS/core/Rle:TotalDepth.LIS.core.Rle.RLE.add}}\pysiglinewithargsret{\sphinxbfcode{add}}{\emph{v}}{}
Adds a value to this RLE object.

\end{fulllineitems}

\index{values() (TotalDepth.LIS.core.Rle.RLE method)}

\begin{fulllineitems}
\phantomsection\label{\detokenize{ref/LIS/core/Rle:TotalDepth.LIS.core.Rle.RLE.values}}\pysiglinewithargsret{\sphinxbfcode{values}}{}{}
Generates all values entered.

\end{fulllineitems}

\index{value() (TotalDepth.LIS.core.Rle.RLE method)}

\begin{fulllineitems}
\phantomsection\label{\detokenize{ref/LIS/core/Rle:TotalDepth.LIS.core.Rle.RLE.value}}\pysiglinewithargsret{\sphinxbfcode{value}}{\emph{i}}{}
Indexing; this returns the i’th value added.

\end{fulllineitems}

\index{rangeList() (TotalDepth.LIS.core.Rle.RLE method)}

\begin{fulllineitems}
\phantomsection\label{\detokenize{ref/LIS/core/Rle:TotalDepth.LIS.core.Rle.RLE.rangeList}}\pysiglinewithargsret{\sphinxbfcode{rangeList}}{}{}
Returns a list of range() or None objects. A None object means that
the RLEItem has a single value.

\end{fulllineitems}

\index{first() (TotalDepth.LIS.core.Rle.RLE method)}

\begin{fulllineitems}
\phantomsection\label{\detokenize{ref/LIS/core/Rle:TotalDepth.LIS.core.Rle.RLE.first}}\pysiglinewithargsret{\sphinxbfcode{first}}{}{}
Returns the first value or None if no values added.

\end{fulllineitems}

\index{last() (TotalDepth.LIS.core.Rle.RLE method)}

\begin{fulllineitems}
\phantomsection\label{\detokenize{ref/LIS/core/Rle:TotalDepth.LIS.core.Rle.RLE.last}}\pysiglinewithargsret{\sphinxbfcode{last}}{}{}
Returns the last value or None if no values added.

\end{fulllineitems}

\index{\_\_weakref\_\_ (TotalDepth.LIS.core.Rle.RLE attribute)}

\begin{fulllineitems}
\phantomsection\label{\detokenize{ref/LIS/core/Rle:TotalDepth.LIS.core.Rle.RLE.__weakref__}}\pysigline{\sphinxbfcode{\_\_weakref\_\_}}
list of weak references to the object (if defined)

\end{fulllineitems}


\end{fulllineitems}

\index{RLEItemType01 (class in TotalDepth.LIS.core.Rle)}

\begin{fulllineitems}
\phantomsection\label{\detokenize{ref/LIS/core/Rle:TotalDepth.LIS.core.Rle.RLEItemType01}}\pysiglinewithargsret{\sphinxbfcode{class }\sphinxcode{TotalDepth.LIS.core.Rle.}\sphinxbfcode{RLEItemType01}}{\emph{tellLrPos}, \emph{numFrameS}, \emph{xAxisValue}}{}
Specialisation of an RLEItem for type 0 and type 1 LIS Logical Records.
This is a RLEItem for the Logical Record but within we have a RLE() object
for the X axis values.

tellLrPos - the position in the LIS file of the start of the Logical Record.

numFrameS - integer number of frames in this Logical Record.

xAxisValue - The value of the X axis of the first frame in the Logical Record.
\index{\_\_str\_\_() (TotalDepth.LIS.core.Rle.RLEItemType01 method)}

\begin{fulllineitems}
\phantomsection\label{\detokenize{ref/LIS/core/Rle:TotalDepth.LIS.core.Rle.RLEItemType01.__str__}}\pysiglinewithargsret{\sphinxbfcode{\_\_str\_\_}}{}{}
String representation.

\end{fulllineitems}

\index{numFrames (TotalDepth.LIS.core.Rle.RLEItemType01 attribute)}

\begin{fulllineitems}
\phantomsection\label{\detokenize{ref/LIS/core/Rle:TotalDepth.LIS.core.Rle.RLEItemType01.numFrames}}\pysigline{\sphinxbfcode{numFrames}}
Total number of frames.

\end{fulllineitems}

\index{add() (TotalDepth.LIS.core.Rle.RLEItemType01 method)}

\begin{fulllineitems}
\phantomsection\label{\detokenize{ref/LIS/core/Rle:TotalDepth.LIS.core.Rle.RLEItemType01.add}}\pysiglinewithargsret{\sphinxbfcode{add}}{\emph{tellLrPos}, \emph{numFrameS}, \emph{xAxisValue}}{}
Returns True if v has been absorbed in this entry. False means a
new entry is required. A new entry is required if the tellLrPos is
not regular or numFrameS is different than before.

\end{fulllineitems}

\index{values() (TotalDepth.LIS.core.Rle.RLEItemType01 method)}

\begin{fulllineitems}
\phantomsection\label{\detokenize{ref/LIS/core/Rle:TotalDepth.LIS.core.Rle.RLEItemType01.values}}\pysiglinewithargsret{\sphinxbfcode{values}}{}{}
Generates ordered tuples of (value, number of frames, xaxis value).

\end{fulllineitems}

\index{value() (TotalDepth.LIS.core.Rle.RLEItemType01 method)}

\begin{fulllineitems}
\phantomsection\label{\detokenize{ref/LIS/core/Rle:TotalDepth.LIS.core.Rle.RLEItemType01.value}}\pysiglinewithargsret{\sphinxbfcode{value}}{\emph{i}}{}
Returns the i’th tuple of (i, (value, number of frames, xaxis value)).

\end{fulllineitems}

\index{totalFrames() (TotalDepth.LIS.core.Rle.RLEItemType01 method)}

\begin{fulllineitems}
\phantomsection\label{\detokenize{ref/LIS/core/Rle:TotalDepth.LIS.core.Rle.RLEItemType01.totalFrames}}\pysiglinewithargsret{\sphinxbfcode{totalFrames}}{}{}
Returns the total number of frames in this RLE item.

\end{fulllineitems}

\index{tellLrForFrame() (TotalDepth.LIS.core.Rle.RLEItemType01 method)}

\begin{fulllineitems}
\phantomsection\label{\detokenize{ref/LIS/core/Rle:TotalDepth.LIS.core.Rle.RLEItemType01.tellLrForFrame}}\pysiglinewithargsret{\sphinxbfcode{tellLrForFrame}}{\emph{fNum}}{}
Returns the Logical Record position that contains the integer frame number.

\end{fulllineitems}

\index{xAxisFirst() (TotalDepth.LIS.core.Rle.RLEItemType01 method)}

\begin{fulllineitems}
\phantomsection\label{\detokenize{ref/LIS/core/Rle:TotalDepth.LIS.core.Rle.RLEItemType01.xAxisFirst}}\pysiglinewithargsret{\sphinxbfcode{xAxisFirst}}{}{}
Returns the first X-axis value loaded.

\end{fulllineitems}

\index{xAxisLast() (TotalDepth.LIS.core.Rle.RLEItemType01 method)}

\begin{fulllineitems}
\phantomsection\label{\detokenize{ref/LIS/core/Rle:TotalDepth.LIS.core.Rle.RLEItemType01.xAxisLast}}\pysiglinewithargsret{\sphinxbfcode{xAxisLast}}{}{}
Returns the first X-axis value loaded.

\end{fulllineitems}


\end{fulllineitems}

\index{RLEType01 (class in TotalDepth.LIS.core.Rle)}

\begin{fulllineitems}
\phantomsection\label{\detokenize{ref/LIS/core/Rle:TotalDepth.LIS.core.Rle.RLEType01}}\pysiglinewithargsret{\sphinxbfcode{class }\sphinxcode{TotalDepth.LIS.core.Rle.}\sphinxbfcode{RLEType01}}{\emph{theXUnits}, \emph{*args}}{}
Class that represents Run Length Encoding for type 0/1 logical records.

theXUnits - the X axis units.
\index{\_\_str\_\_() (TotalDepth.LIS.core.Rle.RLEType01 method)}

\begin{fulllineitems}
\phantomsection\label{\detokenize{ref/LIS/core/Rle:TotalDepth.LIS.core.Rle.RLEType01.__str__}}\pysiglinewithargsret{\sphinxbfcode{\_\_str\_\_}}{}{}
String representation.

\end{fulllineitems}

\index{xAxisUnits (TotalDepth.LIS.core.Rle.RLEType01 attribute)}

\begin{fulllineitems}
\phantomsection\label{\detokenize{ref/LIS/core/Rle:TotalDepth.LIS.core.Rle.RLEType01.xAxisUnits}}\pysigline{\sphinxbfcode{xAxisUnits}}
X axis units.

\end{fulllineitems}

\index{hasXaxisData (TotalDepth.LIS.core.Rle.RLEType01 attribute)}

\begin{fulllineitems}
\phantomsection\label{\detokenize{ref/LIS/core/Rle:TotalDepth.LIS.core.Rle.RLEType01.hasXaxisData}}\pysigline{\sphinxbfcode{hasXaxisData}}
True if there is X axis data.

\end{fulllineitems}

\index{add() (TotalDepth.LIS.core.Rle.RLEType01 method)}

\begin{fulllineitems}
\phantomsection\label{\detokenize{ref/LIS/core/Rle:TotalDepth.LIS.core.Rle.RLEType01.add}}\pysiglinewithargsret{\sphinxbfcode{add}}{\emph{tellLrPos}, \emph{numFrameS}, \emph{xAxisValue}}{}
Adds a value to this RLE object.

\end{fulllineitems}

\index{tellLrForFrame() (TotalDepth.LIS.core.Rle.RLEType01 method)}

\begin{fulllineitems}
\phantomsection\label{\detokenize{ref/LIS/core/Rle:TotalDepth.LIS.core.Rle.RLEType01.tellLrForFrame}}\pysiglinewithargsret{\sphinxbfcode{tellLrForFrame}}{\emph{fNum}}{}
Returns the (lr\_seek, frame\_offset) i.e. the Logical Record position
that contains the integer frame number and the number of excess frames.

\end{fulllineitems}

\index{totalFrames() (TotalDepth.LIS.core.Rle.RLEType01 method)}

\begin{fulllineitems}
\phantomsection\label{\detokenize{ref/LIS/core/Rle:TotalDepth.LIS.core.Rle.RLEType01.totalFrames}}\pysiglinewithargsret{\sphinxbfcode{totalFrames}}{}{}
Returns the total number of frames in this RLE object.

\end{fulllineitems}

\index{xAxisFirst() (TotalDepth.LIS.core.Rle.RLEType01 method)}

\begin{fulllineitems}
\phantomsection\label{\detokenize{ref/LIS/core/Rle:TotalDepth.LIS.core.Rle.RLEType01.xAxisFirst}}\pysiglinewithargsret{\sphinxbfcode{xAxisFirst}}{}{}
Returns the first X-axis value loaded or None if nothing loaded.

\end{fulllineitems}

\index{xAxisLast() (TotalDepth.LIS.core.Rle.RLEType01 method)}

\begin{fulllineitems}
\phantomsection\label{\detokenize{ref/LIS/core/Rle:TotalDepth.LIS.core.Rle.RLEType01.xAxisLast}}\pysiglinewithargsret{\sphinxbfcode{xAxisLast}}{}{}
Returns the last X-axis value at the satart of the last Logical Record loaded or None if nothing loaded.

\end{fulllineitems}

\index{xAxisLastFrame() (TotalDepth.LIS.core.Rle.RLEType01 method)}

\begin{fulllineitems}
\phantomsection\label{\detokenize{ref/LIS/core/Rle:TotalDepth.LIS.core.Rle.RLEType01.xAxisLastFrame}}\pysiglinewithargsret{\sphinxbfcode{xAxisLastFrame}}{}{}
Returns the last X-axis value of the last frame loaded or None if nothing loaded.

\end{fulllineitems}

\index{frameSpacing() (TotalDepth.LIS.core.Rle.RLEType01 method)}

\begin{fulllineitems}
\phantomsection\label{\detokenize{ref/LIS/core/Rle:TotalDepth.LIS.core.Rle.RLEType01.frameSpacing}}\pysiglinewithargsret{\sphinxbfcode{frameSpacing}}{}{}
Returns the frame spacing from the first/last entries, or None if nothing loaded.
Returned value is -ve for decreasing X (up logs), +ve for increasing X
(down and time logs).

\end{fulllineitems}


\end{fulllineitems}



\subsubsection{Type01Plan}
\label{\detokenize{ref/LIS/core/Type01Plan::doc}}\label{\detokenize{ref/LIS/core/Type01Plan:totaldepth-lis-core-type01plan}}\label{\detokenize{ref/LIS/core/Type01Plan:type01plan}}\phantomsection\label{\detokenize{ref/LIS/core/Type01Plan:module-TotalDepth.LIS.core.Type01Plan}}\index{TotalDepth.LIS.core.Type01Plan (module)}
Given a DFSR the FrameSetPlan gives offsets to any part of the Logical Record
for any frame and channel.
\index{ExceptionFrameSetPlan}

\begin{fulllineitems}
\phantomsection\label{\detokenize{ref/LIS/core/Type01Plan:TotalDepth.LIS.core.Type01Plan.ExceptionFrameSetPlan}}\pysigline{\sphinxbfcode{exception }\sphinxcode{TotalDepth.LIS.core.Type01Plan.}\sphinxbfcode{ExceptionFrameSetPlan}}
Specialisation of exception for FrameSetPlan.

\end{fulllineitems}

\index{ExceptionFrameSetPlanNegLen}

\begin{fulllineitems}
\phantomsection\label{\detokenize{ref/LIS/core/Type01Plan:TotalDepth.LIS.core.Type01Plan.ExceptionFrameSetPlanNegLen}}\pysigline{\sphinxbfcode{exception }\sphinxcode{TotalDepth.LIS.core.Type01Plan.}\sphinxbfcode{ExceptionFrameSetPlanNegLen}}
Specialisation of exception for negative/too small length arguments to FrameSetPlan.

\end{fulllineitems}

\index{ExceptionFrameSetPlanOverrun}

\begin{fulllineitems}
\phantomsection\label{\detokenize{ref/LIS/core/Type01Plan:TotalDepth.LIS.core.Type01Plan.ExceptionFrameSetPlanOverrun}}\pysigline{\sphinxbfcode{exception }\sphinxcode{TotalDepth.LIS.core.Type01Plan.}\sphinxbfcode{ExceptionFrameSetPlanOverrun}}
Exception channel number greater than available.

\end{fulllineitems}

\index{EVENT\_READ (in module TotalDepth.LIS.core.Type01Plan)}

\begin{fulllineitems}
\phantomsection\label{\detokenize{ref/LIS/core/Type01Plan:TotalDepth.LIS.core.Type01Plan.EVENT_READ}}\pysigline{\sphinxcode{TotalDepth.LIS.core.Type01Plan.}\sphinxbfcode{EVENT\_READ}\sphinxbfcode{ = 'read'}}
Read event

\end{fulllineitems}

\index{EVENT\_SKIP (in module TotalDepth.LIS.core.Type01Plan)}

\begin{fulllineitems}
\phantomsection\label{\detokenize{ref/LIS/core/Type01Plan:TotalDepth.LIS.core.Type01Plan.EVENT_SKIP}}\pysigline{\sphinxcode{TotalDepth.LIS.core.Type01Plan.}\sphinxbfcode{EVENT\_SKIP}\sphinxbfcode{ = 'skip'}}
Skip event

\end{fulllineitems}

\index{EVENT\_EXTRAPOLATE (in module TotalDepth.LIS.core.Type01Plan)}

\begin{fulllineitems}
\phantomsection\label{\detokenize{ref/LIS/core/Type01Plan:TotalDepth.LIS.core.Type01Plan.EVENT_EXTRAPOLATE}}\pysigline{\sphinxcode{TotalDepth.LIS.core.Type01Plan.}\sphinxbfcode{EVENT\_EXTRAPOLATE}\sphinxbfcode{ = 'extrapolate'}}
Extrapolate event

\end{fulllineitems}

\index{FrameSetPlan (class in TotalDepth.LIS.core.Type01Plan)}

\begin{fulllineitems}
\phantomsection\label{\detokenize{ref/LIS/core/Type01Plan:TotalDepth.LIS.core.Type01Plan.FrameSetPlan}}\pysiglinewithargsret{\sphinxbfcode{class }\sphinxcode{TotalDepth.LIS.core.Type01Plan.}\sphinxbfcode{FrameSetPlan}}{\emph{dfsr}}{}
Given a DFSR the FrameSetPlan gives offsets to any part of the frame set
within a Logical Record.

NOTE: All offsets, lengths etc. are relative to the end of the LRH
i.e. add LR\_HEADER\_LENGTH to get absolute Logical Data position.
\index{indirectSize (TotalDepth.LIS.core.Type01Plan.FrameSetPlan attribute)}

\begin{fulllineitems}
\phantomsection\label{\detokenize{ref/LIS/core/Type01Plan:TotalDepth.LIS.core.Type01Plan.FrameSetPlan.indirectSize}}\pysigline{\sphinxbfcode{indirectSize}}
The size, in bytes of the indirect X axis value. 0 for explicit X axis.

\end{fulllineitems}

\index{frameSize (TotalDepth.LIS.core.Type01Plan.FrameSetPlan attribute)}

\begin{fulllineitems}
\phantomsection\label{\detokenize{ref/LIS/core/Type01Plan:TotalDepth.LIS.core.Type01Plan.FrameSetPlan.frameSize}}\pysigline{\sphinxbfcode{frameSize}}
Frame size in bytes.

\end{fulllineitems}

\index{numChannels (TotalDepth.LIS.core.Type01Plan.FrameSetPlan attribute)}

\begin{fulllineitems}
\phantomsection\label{\detokenize{ref/LIS/core/Type01Plan:TotalDepth.LIS.core.Type01Plan.FrameSetPlan.numChannels}}\pysigline{\sphinxbfcode{numChannels}}
Number of channels.

\end{fulllineitems}

\index{channelSize() (TotalDepth.LIS.core.Type01Plan.FrameSetPlan method)}

\begin{fulllineitems}
\phantomsection\label{\detokenize{ref/LIS/core/Type01Plan:TotalDepth.LIS.core.Type01Plan.FrameSetPlan.channelSize}}\pysiglinewithargsret{\sphinxbfcode{channelSize}}{\emph{i}}{}
The size of the given channel.

\end{fulllineitems}

\index{numFrames() (TotalDepth.LIS.core.Type01Plan.FrameSetPlan method)}

\begin{fulllineitems}
\phantomsection\label{\detokenize{ref/LIS/core/Type01Plan:TotalDepth.LIS.core.Type01Plan.FrameSetPlan.numFrames}}\pysiglinewithargsret{\sphinxbfcode{numFrames}}{\emph{recLen}}{}
Returns the number of frames that will fit into the record length.
May raise a ExceptionFrameSetPlan. May raise ExceptionFrameSetPlanNegLen
if arguments are negative.

NOTE: record length should not include size of LRH.

\end{fulllineitems}

\index{chOffset() (TotalDepth.LIS.core.Type01Plan.FrameSetPlan method)}

\begin{fulllineitems}
\phantomsection\label{\detokenize{ref/LIS/core/Type01Plan:TotalDepth.LIS.core.Type01Plan.FrameSetPlan.chOffset}}\pysiglinewithargsret{\sphinxbfcode{chOffset}}{\emph{frame}, \emph{ch}}{}
Returns the offset into the LR (after LRH) to the start of a particular channel and frame.

Will raise ExceptionFrameSetPlanNegLen if arguments are negative or an IndexError if ch out of range.

\end{fulllineitems}

\index{skipToEndOfFrame() (TotalDepth.LIS.core.Type01Plan.FrameSetPlan method)}

\begin{fulllineitems}
\phantomsection\label{\detokenize{ref/LIS/core/Type01Plan:TotalDepth.LIS.core.Type01Plan.FrameSetPlan.skipToEndOfFrame}}\pysiglinewithargsret{\sphinxbfcode{skipToEndOfFrame}}{\emph{ch}}{}
Returns the skip distance into the LR to the end of frame after a
particular channel.

\end{fulllineitems}

\index{genOffsets() (TotalDepth.LIS.core.Type01Plan.FrameSetPlan method)}

\begin{fulllineitems}
\phantomsection\label{\detokenize{ref/LIS/core/Type01Plan:TotalDepth.LIS.core.Type01Plan.FrameSetPlan.genOffsets}}\pysiglinewithargsret{\sphinxbfcode{genOffsets}}{\emph{theChIndexS}}{}
Yields an indefinite set of (frame, channel, offset) values for the
set of channels.
Will raise ExceptionFrameSetPlanNegLen if any channel negative.
Will raise KeyError if any channel index is out of range.

\end{fulllineitems}

\index{\_\_weakref\_\_ (TotalDepth.LIS.core.Type01Plan.FrameSetPlan attribute)}

\begin{fulllineitems}
\phantomsection\label{\detokenize{ref/LIS/core/Type01Plan:TotalDepth.LIS.core.Type01Plan.FrameSetPlan.__weakref__}}\pysigline{\sphinxbfcode{\_\_weakref\_\_}}
list of weak references to the object (if defined)

\end{fulllineitems}

\index{genEvents() (TotalDepth.LIS.core.Type01Plan.FrameSetPlan method)}

\begin{fulllineitems}
\phantomsection\label{\detokenize{ref/LIS/core/Type01Plan:TotalDepth.LIS.core.Type01Plan.FrameSetPlan.genEvents}}\pysiglinewithargsret{\sphinxbfcode{genEvents}}{\emph{theFSlice}, \emph{theChIndexS}}{}
For a single Logical Record type 0/1 this yields an set of events.

theFSlice is a frame slice object (default start=0, step=1), step 0 is interpreted as step 1.

theChIndexS is as list of indexes of channels that need to be read.

The events are 5 member tuples:
(event\_type, size, frame, channel\_start, channel\_stop)

event\_type is ‘skip’ or ‘read’ or ‘extrapolate’:

skip: (‘skip’,        size,   frame\_number, channel\_start, channel\_stop)

read: (‘read’,        size,   frame\_number, channel\_start, channel\_stop)

extrapolate: (‘extrapolate’, frames, frame\_number, None,          None)

NOTE: No ‘seek’, 0, … events are generated.

size is the number of bytes to read or skip or, for ‘extrapolate’ the
number of frames to extrapolate the implied X axis.

frame\_number is the frame number in the predicted Logical Record.

In the case of ‘skip’ or ‘extrapolate’ its is the frame number in the
Logical Record being moved to. The frame number can be None if there
is no prediction, for example when reading an indirect X at the
beginning of an LR.

channel\_start, channel\_stop are the channel numbers (inclusive) in the
DFSR DSB block None being the indirect X axis.

Thus: (‘read’, size, frame\_number, None, 3)
Means read size bytes and interpret them as channels: None, 0, 1, 2, 3.
‘extrapolate’ events mean project the Xaxis forward by the specified
number of frames (this will not be bestrided with two seek events but
preceded or followed by a single seek event (if any). ‘read’ and ‘seek’
events are synchronous as are ‘read’ and ‘extrapolate’. However ‘seek’
and ‘extrapolate’ are asynchronous i.e. ‘seek’-‘extrapolate’ is
equivalent to ‘extrapolate’-‘seek’.

Will raise ExceptionFrameSetPlanNegLen if any channel negative or fStop \textless{} fStart.
Will raise IndexError if any channel index is out of range.

Be aware: This code is tricky.

\end{fulllineitems}


\end{fulllineitems}



\subsection{TotalDepth.LAS API Reference}
\label{\detokenize{ref/LAS/index_LAS::doc}}\label{\detokenize{ref/LAS/index_LAS:totaldepth-las-api-reference}}
Contents:


\subsubsection{LAS Read}
\label{\detokenize{ref/LAS/core/LASRead:las-read}}\label{\detokenize{ref/LAS/core/LASRead::doc}}\label{\detokenize{ref/LAS/core/LASRead:module-TotalDepth.LAS.core.LASRead}}\index{TotalDepth.LAS.core.LASRead (module)}
Reads LAS files.

Created on Jan 11, 2012
\index{ExceptionLASRead}

\begin{fulllineitems}
\phantomsection\label{\detokenize{ref/LAS/core/LASRead:TotalDepth.LAS.core.LASRead.ExceptionLASRead}}\pysigline{\sphinxbfcode{exception }\sphinxcode{TotalDepth.LAS.core.LASRead.}\sphinxbfcode{ExceptionLASRead}}
Specialisation of exception for LASRead.

\end{fulllineitems}

\index{ExceptionLASReadSection}

\begin{fulllineitems}
\phantomsection\label{\detokenize{ref/LAS/core/LASRead:TotalDepth.LAS.core.LASRead.ExceptionLASReadSection}}\pysigline{\sphinxbfcode{exception }\sphinxcode{TotalDepth.LAS.core.LASRead.}\sphinxbfcode{ExceptionLASReadSection}}
Specialisation of exception for LASRead when handling sections.

\end{fulllineitems}

\index{ExceptionLASReadSectionArray}

\begin{fulllineitems}
\phantomsection\label{\detokenize{ref/LAS/core/LASRead:TotalDepth.LAS.core.LASRead.ExceptionLASReadSectionArray}}\pysigline{\sphinxbfcode{exception }\sphinxcode{TotalDepth.LAS.core.LASRead.}\sphinxbfcode{ExceptionLASReadSectionArray}}
Specialisation of exception for LASRead when handling array section.

\end{fulllineitems}

\index{ExceptionLASReadData}

\begin{fulllineitems}
\phantomsection\label{\detokenize{ref/LAS/core/LASRead:TotalDepth.LAS.core.LASRead.ExceptionLASReadData}}\pysigline{\sphinxbfcode{exception }\sphinxcode{TotalDepth.LAS.core.LASRead.}\sphinxbfcode{ExceptionLASReadData}}
Specialisation of exception for LASRead when reading data.

\end{fulllineitems}

\index{ExceptionLASKeyError}

\begin{fulllineitems}
\phantomsection\label{\detokenize{ref/LAS/core/LASRead:TotalDepth.LAS.core.LASRead.ExceptionLASKeyError}}\pysigline{\sphinxbfcode{exception }\sphinxcode{TotalDepth.LAS.core.LASRead.}\sphinxbfcode{ExceptionLASKeyError}}
Equivalent to KeyError when looking stuff up.

\end{fulllineitems}

\index{LAS\_FILE\_EXT (in module TotalDepth.LAS.core.LASRead)}

\begin{fulllineitems}
\phantomsection\label{\detokenize{ref/LAS/core/LASRead:TotalDepth.LAS.core.LASRead.LAS_FILE_EXT}}\pysigline{\sphinxcode{TotalDepth.LAS.core.LASRead.}\sphinxbfcode{LAS\_FILE\_EXT}\sphinxbfcode{ = '.las'}}
LAS file extension

\end{fulllineitems}

\index{LAS\_FILE\_EXT\_LOWER (in module TotalDepth.LAS.core.LASRead)}

\begin{fulllineitems}
\phantomsection\label{\detokenize{ref/LAS/core/LASRead:TotalDepth.LAS.core.LASRead.LAS_FILE_EXT_LOWER}}\pysigline{\sphinxcode{TotalDepth.LAS.core.LASRead.}\sphinxbfcode{LAS\_FILE\_EXT\_LOWER}\sphinxbfcode{ = '.las'}}
LAS lower case file extension

\end{fulllineitems}

\index{hasLASExtension() (in module TotalDepth.LAS.core.LASRead)}

\begin{fulllineitems}
\phantomsection\label{\detokenize{ref/LAS/core/LASRead:TotalDepth.LAS.core.LASRead.hasLASExtension}}\pysiglinewithargsret{\sphinxcode{TotalDepth.LAS.core.LASRead.}\sphinxbfcode{hasLASExtension}}{\emph{fp}}{}
Returns True if the file extansion is a LAS one.

\end{fulllineitems}

\index{RE\_COMMENT (in module TotalDepth.LAS.core.LASRead)}

\begin{fulllineitems}
\phantomsection\label{\detokenize{ref/LAS/core/LASRead:TotalDepth.LAS.core.LASRead.RE_COMMENT}}\pysigline{\sphinxcode{TotalDepth.LAS.core.LASRead.}\sphinxbfcode{RE\_COMMENT}\sphinxbfcode{ = re.compile('\textasciicircum{}\textbackslash{}\textbackslash{}s*\#(.*)\$')}}
Regex to match a comment

\end{fulllineitems}

\index{DEBUG\_LINE\_BY\_LINE (in module TotalDepth.LAS.core.LASRead)}

\begin{fulllineitems}
\phantomsection\label{\detokenize{ref/LAS/core/LASRead:TotalDepth.LAS.core.LASRead.DEBUG_LINE_BY_LINE}}\pysigline{\sphinxcode{TotalDepth.LAS.core.LASRead.}\sphinxbfcode{DEBUG\_LINE\_BY\_LINE}\sphinxbfcode{ = False}}
logging.debug call here can add about 50\% of the processing time

\end{fulllineitems}

\index{genLines() (in module TotalDepth.LAS.core.LASRead)}

\begin{fulllineitems}
\phantomsection\label{\detokenize{ref/LAS/core/LASRead:TotalDepth.LAS.core.LASRead.genLines}}\pysiglinewithargsret{\sphinxcode{TotalDepth.LAS.core.LASRead.}\sphinxbfcode{genLines}}{\emph{f}}{}
Given an file-like object this generates non-blank, non-comment lines.
It’s a co-routine so can accept a line to put back.

\end{fulllineitems}

\index{SECT\_TYPES (in module TotalDepth.LAS.core.LASRead)}

\begin{fulllineitems}
\phantomsection\label{\detokenize{ref/LAS/core/LASRead:TotalDepth.LAS.core.LASRead.SECT_TYPES}}\pysigline{\sphinxcode{TotalDepth.LAS.core.LASRead.}\sphinxbfcode{SECT\_TYPES}\sphinxbfcode{ = 'VWCPOA'}}
All section identifiers

\end{fulllineitems}

\index{SECT\_TYPES\_WITH\_DATA\_LINES (in module TotalDepth.LAS.core.LASRead)}

\begin{fulllineitems}
\phantomsection\label{\detokenize{ref/LAS/core/LASRead:TotalDepth.LAS.core.LASRead.SECT_TYPES_WITH_DATA_LINES}}\pysigline{\sphinxcode{TotalDepth.LAS.core.LASRead.}\sphinxbfcode{SECT\_TYPES\_WITH\_DATA\_LINES}\sphinxbfcode{ = 'VWCP'}}
Section with data lines

\end{fulllineitems}

\index{SECT\_TYPES\_WITH\_INDEX (in module TotalDepth.LAS.core.LASRead)}

\begin{fulllineitems}
\phantomsection\label{\detokenize{ref/LAS/core/LASRead:TotalDepth.LAS.core.LASRead.SECT_TYPES_WITH_INDEX}}\pysigline{\sphinxcode{TotalDepth.LAS.core.LASRead.}\sphinxbfcode{SECT\_TYPES\_WITH\_INDEX}\sphinxbfcode{ = 'VWCPA'}}
Section with index value in column 0

\end{fulllineitems}

\index{RE\_SECT\_HEAD (in module TotalDepth.LAS.core.LASRead)}

\begin{fulllineitems}
\phantomsection\label{\detokenize{ref/LAS/core/LASRead:TotalDepth.LAS.core.LASRead.RE_SECT_HEAD}}\pysigline{\sphinxcode{TotalDepth.LAS.core.LASRead.}\sphinxbfcode{RE\_SECT\_HEAD}\sphinxbfcode{ = re.compile('\textasciicircum{}\textasciitilde{}({[}VWCPOA{]})(.+)*\$')}}
Regex to match a section heasd

\end{fulllineitems}

\index{SECT\_DESCRIPTION\_MAP (in module TotalDepth.LAS.core.LASRead)}

\begin{fulllineitems}
\phantomsection\label{\detokenize{ref/LAS/core/LASRead:TotalDepth.LAS.core.LASRead.SECT_DESCRIPTION_MAP}}\pysigline{\sphinxcode{TotalDepth.LAS.core.LASRead.}\sphinxbfcode{SECT\_DESCRIPTION\_MAP}\sphinxbfcode{ = \{'V': 'Version Information Section', 'W': 'Well Information Section', 'C': 'Curve Information Section', 'P': 'Parameter Information Section', 'O': 'Other Information Section', 'A': 'ASCII Log Data'\}}}
Map of section identifiers to description

\end{fulllineitems}

\index{RE\_LINE\_FIELD\_0 (in module TotalDepth.LAS.core.LASRead)}

\begin{fulllineitems}
\phantomsection\label{\detokenize{ref/LAS/core/LASRead:TotalDepth.LAS.core.LASRead.RE_LINE_FIELD_0}}\pysigline{\sphinxcode{TotalDepth.LAS.core.LASRead.}\sphinxbfcode{RE\_LINE\_FIELD\_0}\sphinxbfcode{ = re.compile('\textasciicircum{}\textbackslash{}\textbackslash{}s*({[}\textasciicircum{} .:{]}+)\textbackslash{}\textbackslash{}s*\$')}}
The ‘MENM’ field, no spaces, dots or colons.
ONE group

\end{fulllineitems}

\index{RE\_LINE\_FIELD\_1 (in module TotalDepth.LAS.core.LASRead)}

\begin{fulllineitems}
\phantomsection\label{\detokenize{ref/LAS/core/LASRead:TotalDepth.LAS.core.LASRead.RE_LINE_FIELD_1}}\pysigline{\sphinxcode{TotalDepth.LAS.core.LASRead.}\sphinxbfcode{RE\_LINE\_FIELD\_1}\sphinxbfcode{ = re.compile('\textasciicircum{}({[}\textasciicircum{} :{]}+)*(.+)*\$')}}
The data field which is the middle of the line between the first ‘.’ and last ‘:’
This is composed of optional units and value.
Units must follow the dot immediately and contain no colons or spaces
Note: Group 2 may have leading and trailing spaces
TWO groups

\end{fulllineitems}

\index{RE\_LINE\_FIELD\_2 (in module TotalDepth.LAS.core.LASRead)}

\begin{fulllineitems}
\phantomsection\label{\detokenize{ref/LAS/core/LASRead:TotalDepth.LAS.core.LASRead.RE_LINE_FIELD_2}}\pysigline{\sphinxcode{TotalDepth.LAS.core.LASRead.}\sphinxbfcode{RE\_LINE\_FIELD\_2}\sphinxbfcode{ = re.compile("\textbackslash{}n    \textasciicircum{}          \# Start of line\textbackslash{}n    ({[}\textasciicircum{}\{\textbar{}{]}+)*  \# Description, anything but '\{' or '\textbar{}', needs strip()\textbackslash{}n    (\{.+?\})*   \# Optional format \{...\}\textbackslash{}n    \textbackslash{}\textbackslash{}s*        \# Optional space\textbackslash{}n    {[}\textbar{}{]}*       \# Opti, re.VERBOSE)}}
Data field 2 is after the last ‘:’
THREE groups

\end{fulllineitems}

\index{SectLine (class in TotalDepth.LAS.core.LASRead)}

\begin{fulllineitems}
\phantomsection\label{\detokenize{ref/LAS/core/LASRead:TotalDepth.LAS.core.LASRead.SectLine}}\pysiglinewithargsret{\sphinxbfcode{class }\sphinxcode{TotalDepth.LAS.core.LASRead.}\sphinxbfcode{SectLine}}{\emph{mnem}, \emph{unit}, \emph{valu}, \emph{desc}, \emph{format}, \emph{assoc}}{}
Captures the 6 fields: mnem unit valu desc format assoc
\index{\_\_getnewargs\_\_() (TotalDepth.LAS.core.LASRead.SectLine method)}

\begin{fulllineitems}
\phantomsection\label{\detokenize{ref/LAS/core/LASRead:TotalDepth.LAS.core.LASRead.SectLine.__getnewargs__}}\pysiglinewithargsret{\sphinxbfcode{\_\_getnewargs\_\_}}{}{}
Return self as a plain tuple.  Used by copy and pickle.

\end{fulllineitems}

\index{\_\_new\_\_() (TotalDepth.LAS.core.LASRead.SectLine static method)}

\begin{fulllineitems}
\phantomsection\label{\detokenize{ref/LAS/core/LASRead:TotalDepth.LAS.core.LASRead.SectLine.__new__}}\pysiglinewithargsret{\sphinxbfcode{static }\sphinxbfcode{\_\_new\_\_}}{\emph{\_cls}, \emph{mnem}, \emph{unit}, \emph{valu}, \emph{desc}, \emph{format}, \emph{assoc}}{}
Create new instance of SectLine(mnem, unit, valu, desc, format, assoc)

\end{fulllineitems}

\index{\_\_repr\_\_() (TotalDepth.LAS.core.LASRead.SectLine method)}

\begin{fulllineitems}
\phantomsection\label{\detokenize{ref/LAS/core/LASRead:TotalDepth.LAS.core.LASRead.SectLine.__repr__}}\pysiglinewithargsret{\sphinxbfcode{\_\_repr\_\_}}{}{}
Return a nicely formatted representation string

\end{fulllineitems}

\index{assoc (TotalDepth.LAS.core.LASRead.SectLine attribute)}

\begin{fulllineitems}
\phantomsection\label{\detokenize{ref/LAS/core/LASRead:TotalDepth.LAS.core.LASRead.SectLine.assoc}}\pysigline{\sphinxbfcode{assoc}}
Alias for field number 5

\end{fulllineitems}

\index{desc (TotalDepth.LAS.core.LASRead.SectLine attribute)}

\begin{fulllineitems}
\phantomsection\label{\detokenize{ref/LAS/core/LASRead:TotalDepth.LAS.core.LASRead.SectLine.desc}}\pysigline{\sphinxbfcode{desc}}
Alias for field number 3

\end{fulllineitems}

\index{format (TotalDepth.LAS.core.LASRead.SectLine attribute)}

\begin{fulllineitems}
\phantomsection\label{\detokenize{ref/LAS/core/LASRead:TotalDepth.LAS.core.LASRead.SectLine.format}}\pysigline{\sphinxbfcode{format}}
Alias for field number 4

\end{fulllineitems}

\index{mnem (TotalDepth.LAS.core.LASRead.SectLine attribute)}

\begin{fulllineitems}
\phantomsection\label{\detokenize{ref/LAS/core/LASRead:TotalDepth.LAS.core.LASRead.SectLine.mnem}}\pysigline{\sphinxbfcode{mnem}}
Alias for field number 0

\end{fulllineitems}

\index{unit (TotalDepth.LAS.core.LASRead.SectLine attribute)}

\begin{fulllineitems}
\phantomsection\label{\detokenize{ref/LAS/core/LASRead:TotalDepth.LAS.core.LASRead.SectLine.unit}}\pysigline{\sphinxbfcode{unit}}
Alias for field number 1

\end{fulllineitems}

\index{valu (TotalDepth.LAS.core.LASRead.SectLine attribute)}

\begin{fulllineitems}
\phantomsection\label{\detokenize{ref/LAS/core/LASRead:TotalDepth.LAS.core.LASRead.SectLine.valu}}\pysigline{\sphinxbfcode{valu}}
Alias for field number 2

\end{fulllineitems}


\end{fulllineitems}

\index{LASSection (class in TotalDepth.LAS.core.LASRead)}

\begin{fulllineitems}
\phantomsection\label{\detokenize{ref/LAS/core/LASRead:TotalDepth.LAS.core.LASRead.LASSection}}\pysiglinewithargsret{\sphinxbfcode{class }\sphinxcode{TotalDepth.LAS.core.LASRead.}\sphinxbfcode{LASSection}}{\emph{sectType}}{}
Contains data on a section.
\index{\_\_len\_\_() (TotalDepth.LAS.core.LASRead.LASSection method)}

\begin{fulllineitems}
\phantomsection\label{\detokenize{ref/LAS/core/LASRead:TotalDepth.LAS.core.LASRead.LASSection.__len__}}\pysiglinewithargsret{\sphinxbfcode{\_\_len\_\_}}{}{}
Number of members.

\end{fulllineitems}

\index{\_\_contains\_\_() (TotalDepth.LAS.core.LASRead.LASSection method)}

\begin{fulllineitems}
\phantomsection\label{\detokenize{ref/LAS/core/LASRead:TotalDepth.LAS.core.LASRead.LASSection.__contains__}}\pysiglinewithargsret{\sphinxbfcode{\_\_contains\_\_}}{\emph{theMnem}}{}
Membership test.

\end{fulllineitems}

\index{addMemberLine() (TotalDepth.LAS.core.LASRead.LASSection method)}

\begin{fulllineitems}
\phantomsection\label{\detokenize{ref/LAS/core/LASRead:TotalDepth.LAS.core.LASRead.LASSection.addMemberLine}}\pysiglinewithargsret{\sphinxbfcode{addMemberLine}}{\emph{i}, \emph{l}}{}
Given a line this decomposes it to its \_members. i is the position
of the line l in the file.

\end{fulllineitems}

\index{finalise() (TotalDepth.LAS.core.LASRead.LASSection method)}

\begin{fulllineitems}
\phantomsection\label{\detokenize{ref/LAS/core/LASRead:TotalDepth.LAS.core.LASRead.LASSection.finalise}}\pysiglinewithargsret{\sphinxbfcode{finalise}}{}{}
Finalisation of section, this updates the internal representation.

\end{fulllineitems}

\index{\_\_getitem\_\_() (TotalDepth.LAS.core.LASRead.LASSection method)}

\begin{fulllineitems}
\phantomsection\label{\detokenize{ref/LAS/core/LASRead:TotalDepth.LAS.core.LASRead.LASSection.__getitem__}}\pysiglinewithargsret{\sphinxbfcode{\_\_getitem\_\_}}{\emph{key}}{}
Returns an entry, key can be int or str.

\end{fulllineitems}

\index{mnemS() (TotalDepth.LAS.core.LASRead.LASSection method)}

\begin{fulllineitems}
\phantomsection\label{\detokenize{ref/LAS/core/LASRead:TotalDepth.LAS.core.LASRead.LASSection.mnemS}}\pysiglinewithargsret{\sphinxbfcode{mnemS}}{}{}
Returns an list of mnemonics.

\end{fulllineitems}

\index{unitS() (TotalDepth.LAS.core.LASRead.LASSection method)}

\begin{fulllineitems}
\phantomsection\label{\detokenize{ref/LAS/core/LASRead:TotalDepth.LAS.core.LASRead.LASSection.unitS}}\pysiglinewithargsret{\sphinxbfcode{unitS}}{}{}
Returns an list of mnemonic units.

\end{fulllineitems}

\index{keys() (TotalDepth.LAS.core.LASRead.LASSection method)}

\begin{fulllineitems}
\phantomsection\label{\detokenize{ref/LAS/core/LASRead:TotalDepth.LAS.core.LASRead.LASSection.keys}}\pysiglinewithargsret{\sphinxbfcode{keys}}{}{}
Returns the keys in the internal map.

\end{fulllineitems}

\index{find() (TotalDepth.LAS.core.LASRead.LASSection method)}

\begin{fulllineitems}
\phantomsection\label{\detokenize{ref/LAS/core/LASRead:TotalDepth.LAS.core.LASRead.LASSection.find}}\pysiglinewithargsret{\sphinxbfcode{find}}{\emph{m}}{}
Returns the member ordinal for mnemonic m or -1 if not found.
This can be used for finding the array column for a particular curve.

\end{fulllineitems}

\index{\_\_weakref\_\_ (TotalDepth.LAS.core.LASRead.LASSection attribute)}

\begin{fulllineitems}
\phantomsection\label{\detokenize{ref/LAS/core/LASRead:TotalDepth.LAS.core.LASRead.LASSection.__weakref__}}\pysigline{\sphinxbfcode{\_\_weakref\_\_}}
list of weak references to the object (if defined)

\end{fulllineitems}


\end{fulllineitems}

\index{LASSectionArray (class in TotalDepth.LAS.core.LASRead)}

\begin{fulllineitems}
\phantomsection\label{\detokenize{ref/LAS/core/LASRead:TotalDepth.LAS.core.LASRead.LASSectionArray}}\pysiglinewithargsret{\sphinxbfcode{class }\sphinxcode{TotalDepth.LAS.core.LASRead.}\sphinxbfcode{LASSectionArray}}{\emph{sectType}, \emph{wrap}, \emph{curvSect}, \emph{null=-999.25}}{}
Contains data on an array section.
\index{addMemberLine() (TotalDepth.LAS.core.LASRead.LASSectionArray method)}

\begin{fulllineitems}
\phantomsection\label{\detokenize{ref/LAS/core/LASRead:TotalDepth.LAS.core.LASRead.LASSectionArray.addMemberLine}}\pysiglinewithargsret{\sphinxbfcode{addMemberLine}}{\emph{i}, \emph{l}}{}
Process a line in an array section.

\end{fulllineitems}

\index{finalise() (TotalDepth.LAS.core.LASRead.LASSectionArray method)}

\begin{fulllineitems}
\phantomsection\label{\detokenize{ref/LAS/core/LASRead:TotalDepth.LAS.core.LASRead.LASSectionArray.finalise}}\pysiglinewithargsret{\sphinxbfcode{finalise}}{}{}
Finalisation.
TODO: Make a numpy array and get rid of the members. Need to overload \_\_getitem\_\_.

\end{fulllineitems}

\index{frameSize() (TotalDepth.LAS.core.LASRead.LASSectionArray method)}

\begin{fulllineitems}
\phantomsection\label{\detokenize{ref/LAS/core/LASRead:TotalDepth.LAS.core.LASRead.LASSectionArray.frameSize}}\pysiglinewithargsret{\sphinxbfcode{frameSize}}{}{}
Returns the number of data points in a frame.

\end{fulllineitems}


\end{fulllineitems}

\index{LASBase (class in TotalDepth.LAS.core.LASRead)}

\begin{fulllineitems}
\phantomsection\label{\detokenize{ref/LAS/core/LASRead:TotalDepth.LAS.core.LASRead.LASBase}}\pysiglinewithargsret{\sphinxbfcode{class }\sphinxcode{TotalDepth.LAS.core.LASRead.}\sphinxbfcode{LASBase}}{\emph{id}}{}
Base class for LAS reading and writing. This provides common functionality to child classes.
\index{UNITS\_LAS\_TO\_LIS (TotalDepth.LAS.core.LASRead.LASBase attribute)}

\begin{fulllineitems}
\phantomsection\label{\detokenize{ref/LAS/core/LASRead:TotalDepth.LAS.core.LASRead.LASBase.UNITS_LAS_TO_LIS}}\pysigline{\sphinxbfcode{UNITS\_LAS\_TO\_LIS}\sphinxbfcode{ = \{'F': 'FEET', 'mts': 'M   '\}}}
Mapping of commonly seen LAS units to proper LIS units
Both key and value must be strings

\end{fulllineitems}

\index{\_\_len\_\_() (TotalDepth.LAS.core.LASRead.LASBase method)}

\begin{fulllineitems}
\phantomsection\label{\detokenize{ref/LAS/core/LASRead:TotalDepth.LAS.core.LASRead.LASBase.__len__}}\pysiglinewithargsret{\sphinxbfcode{\_\_len\_\_}}{}{}
Number of sections.

\end{fulllineitems}

\index{\_\_getitem\_\_() (TotalDepth.LAS.core.LASRead.LASBase method)}

\begin{fulllineitems}
\phantomsection\label{\detokenize{ref/LAS/core/LASRead:TotalDepth.LAS.core.LASRead.LASBase.__getitem__}}\pysiglinewithargsret{\sphinxbfcode{\_\_getitem\_\_}}{\emph{key}}{}
Returns a section, key can be int or str.

\end{fulllineitems}

\index{numFrames() (TotalDepth.LAS.core.LASRead.LASBase method)}

\begin{fulllineitems}
\phantomsection\label{\detokenize{ref/LAS/core/LASRead:TotalDepth.LAS.core.LASRead.LASBase.numFrames}}\pysiglinewithargsret{\sphinxbfcode{numFrames}}{}{}
Returns the number of frames of data in an ‘A’ record if I have one.

\end{fulllineitems}

\index{numDataPoints() (TotalDepth.LAS.core.LASRead.LASBase method)}

\begin{fulllineitems}
\phantomsection\label{\detokenize{ref/LAS/core/LASRead:TotalDepth.LAS.core.LASRead.LASBase.numDataPoints}}\pysiglinewithargsret{\sphinxbfcode{numDataPoints}}{}{}
Returns the number of frame data points in an ‘A’ record if I have one.

\end{fulllineitems}

\index{genSects() (TotalDepth.LAS.core.LASRead.LASBase method)}

\begin{fulllineitems}
\phantomsection\label{\detokenize{ref/LAS/core/LASRead:TotalDepth.LAS.core.LASRead.LASBase.genSects}}\pysiglinewithargsret{\sphinxbfcode{genSects}}{}{}
Yields up each section.

\end{fulllineitems}

\index{nullValue (TotalDepth.LAS.core.LASRead.LASBase attribute)}

\begin{fulllineitems}
\phantomsection\label{\detokenize{ref/LAS/core/LASRead:TotalDepth.LAS.core.LASRead.LASBase.nullValue}}\pysigline{\sphinxbfcode{nullValue}}
The NULL value, defaults to -999.25.

\end{fulllineitems}

\index{xAxisUnits (TotalDepth.LAS.core.LASRead.LASBase attribute)}

\begin{fulllineitems}
\phantomsection\label{\detokenize{ref/LAS/core/LASRead:TotalDepth.LAS.core.LASRead.LASBase.xAxisUnits}}\pysigline{\sphinxbfcode{xAxisUnits}}
The X axis units.

\end{fulllineitems}

\index{xAxisStart (TotalDepth.LAS.core.LASRead.LASBase attribute)}

\begin{fulllineitems}
\phantomsection\label{\detokenize{ref/LAS/core/LASRead:TotalDepth.LAS.core.LASRead.LASBase.xAxisStart}}\pysigline{\sphinxbfcode{xAxisStart}}
The Xaxis start value as an EngVal.

\end{fulllineitems}

\index{xAxisStop (TotalDepth.LAS.core.LASRead.LASBase attribute)}

\begin{fulllineitems}
\phantomsection\label{\detokenize{ref/LAS/core/LASRead:TotalDepth.LAS.core.LASRead.LASBase.xAxisStop}}\pysigline{\sphinxbfcode{xAxisStop}}
The Xaxis end value as an EngVal.

\end{fulllineitems}

\index{xAxisStep (TotalDepth.LAS.core.LASRead.LASBase attribute)}

\begin{fulllineitems}
\phantomsection\label{\detokenize{ref/LAS/core/LASRead:TotalDepth.LAS.core.LASRead.LASBase.xAxisStep}}\pysigline{\sphinxbfcode{xAxisStep}}
The Xaxis step value as an EngVal.

\end{fulllineitems}

\index{logDown() (TotalDepth.LAS.core.LASRead.LASBase method)}

\begin{fulllineitems}
\phantomsection\label{\detokenize{ref/LAS/core/LASRead:TotalDepth.LAS.core.LASRead.LASBase.logDown}}\pysiglinewithargsret{\sphinxbfcode{logDown}}{}{}
Returns True if X axis is increasing i.e. for time or down log.

\end{fulllineitems}

\index{getWsdMnem() (TotalDepth.LAS.core.LASRead.LASBase method)}

\begin{fulllineitems}
\phantomsection\label{\detokenize{ref/LAS/core/LASRead:TotalDepth.LAS.core.LASRead.LASBase.getWsdMnem}}\pysiglinewithargsret{\sphinxbfcode{getWsdMnem}}{\emph{m}}{}
Returns a tuple of (value, units) for a Mnemonic that may appear in either a
Well section or a Parameter section. Units may be None if empty.
Returns (None, None) if nothing found.

\end{fulllineitems}

\index{getAllWsdMnemonics() (TotalDepth.LAS.core.LASRead.LASBase method)}

\begin{fulllineitems}
\phantomsection\label{\detokenize{ref/LAS/core/LASRead:TotalDepth.LAS.core.LASRead.LASBase.getAllWsdMnemonics}}\pysiglinewithargsret{\sphinxbfcode{getAllWsdMnemonics}}{}{}
Returns a set of mnemonics from the Well section and the Parameter section.

\end{fulllineitems}

\index{hasOutpMnem() (TotalDepth.LAS.core.LASRead.LASBase method)}

\begin{fulllineitems}
\phantomsection\label{\detokenize{ref/LAS/core/LASRead:TotalDepth.LAS.core.LASRead.LASBase.hasOutpMnem}}\pysiglinewithargsret{\sphinxbfcode{hasOutpMnem}}{\emph{theMnem}}{}
Returns True if theMnem, a Mnem.Mnem() object is an output in the Curve section.
It will use the alternate names table LGFORMAT\_LAS from LASConstants to interpret 
curves that are not exact matches.

\end{fulllineitems}

\index{curveMnems() (TotalDepth.LAS.core.LASRead.LASBase method)}

\begin{fulllineitems}
\phantomsection\label{\detokenize{ref/LAS/core/LASRead:TotalDepth.LAS.core.LASRead.LASBase.curveMnems}}\pysiglinewithargsret{\sphinxbfcode{curveMnems}}{\emph{ordered=False}}{}
Returns list of curve names actually declared in the Curve section.
List will be unordered if ordered is False.

\end{fulllineitems}

\index{curveUnitsAsStr() (TotalDepth.LAS.core.LASRead.LASBase method)}

\begin{fulllineitems}
\phantomsection\label{\detokenize{ref/LAS/core/LASRead:TotalDepth.LAS.core.LASRead.LASBase.curveUnitsAsStr}}\pysiglinewithargsret{\sphinxbfcode{curveUnitsAsStr}}{\emph{m}}{}
Given a curve as a Mnem.Mnem() this returns the units as a string.
May raise KeyError.

\end{fulllineitems}

\index{genOutpPoints() (TotalDepth.LAS.core.LASRead.LASBase method)}

\begin{fulllineitems}
\phantomsection\label{\detokenize{ref/LAS/core/LASRead:TotalDepth.LAS.core.LASRead.LASBase.genOutpPoints}}\pysiglinewithargsret{\sphinxbfcode{genOutpPoints}}{\emph{theMnem}}{}
Generates curve values for theMnem, a Mnem.Mnem() object.

\end{fulllineitems}

\index{\_\_weakref\_\_ (TotalDepth.LAS.core.LASRead.LASBase attribute)}

\begin{fulllineitems}
\phantomsection\label{\detokenize{ref/LAS/core/LASRead:TotalDepth.LAS.core.LASRead.LASBase.__weakref__}}\pysigline{\sphinxbfcode{\_\_weakref\_\_}}
list of weak references to the object (if defined)

\end{fulllineitems}


\end{fulllineitems}

\index{LASRead (class in TotalDepth.LAS.core.LASRead)}

\begin{fulllineitems}
\phantomsection\label{\detokenize{ref/LAS/core/LASRead:TotalDepth.LAS.core.LASRead.LASRead}}\pysiglinewithargsret{\sphinxbfcode{class }\sphinxcode{TotalDepth.LAS.core.LASRead.}\sphinxbfcode{LASRead}}{\emph{theFp}, \emph{theFileID=None}}{}
Reads a LAS file.
\index{\_\_init\_\_() (TotalDepth.LAS.core.LASRead.LASRead method)}

\begin{fulllineitems}
\phantomsection\label{\detokenize{ref/LAS/core/LASRead:TotalDepth.LAS.core.LASRead.LASRead.__init__}}\pysiglinewithargsret{\sphinxbfcode{\_\_init\_\_}}{\emph{theFp}, \emph{theFileID=None}}{}
Reads a LAS file from theFp that is either a string (file path) or a file like object.

\end{fulllineitems}


\end{fulllineitems}



\subsection{Utility Package Reference}
\label{\detokenize{ref/util/index_util::doc}}\label{\detokenize{ref/util/index_util:utility-package-reference}}
Contents:


\subsubsection{Common Command Line Options}
\label{\detokenize{ref/util/CmnCmdOpts:common-command-line-options}}\label{\detokenize{ref/util/CmnCmdOpts::doc}}\label{\detokenize{ref/util/CmnCmdOpts:module-TotalDepth.util.CmnCmdOpts}}\index{TotalDepth.util.CmnCmdOpts (module)}
Common command line options, this is to try and present some degree of
interface consistency among command line applications.

Copyright (c) 2010-2011 Paul Ross. All rights reserved.
\index{argParser() (in module TotalDepth.util.CmnCmdOpts)}

\begin{fulllineitems}
\phantomsection\label{\detokenize{ref/util/CmnCmdOpts:TotalDepth.util.CmnCmdOpts.argParser}}\pysiglinewithargsret{\sphinxcode{TotalDepth.util.CmnCmdOpts.}\sphinxbfcode{argParser}}{\emph{desc}, \emph{prog=None}, \emph{version=None}}{}
Return an command line parser with the standard pre-set options.

Standard options are \sphinxcode{-h, -{-}version} and:

\sphinxcode{-j}: Multiprocessing job control.

\sphinxcode{-k}: Flag to indicate that we should keep going as far as sensible.

\sphinxcode{-l}: Log level.

\end{fulllineitems}

\index{argParserIn() (in module TotalDepth.util.CmnCmdOpts)}

\begin{fulllineitems}
\phantomsection\label{\detokenize{ref/util/CmnCmdOpts:TotalDepth.util.CmnCmdOpts.argParserIn}}\pysiglinewithargsret{\sphinxcode{TotalDepth.util.CmnCmdOpts.}\sphinxbfcode{argParserIn}}{\emph{*args}, \emph{**kwargs}}{}
Return an command line parser with the standard pre-set options plus an input path as an argument.

\end{fulllineitems}

\index{argParserInOut() (in module TotalDepth.util.CmnCmdOpts)}

\begin{fulllineitems}
\phantomsection\label{\detokenize{ref/util/CmnCmdOpts:TotalDepth.util.CmnCmdOpts.argParserInOut}}\pysiglinewithargsret{\sphinxcode{TotalDepth.util.CmnCmdOpts.}\sphinxbfcode{argParserInOut}}{\emph{*args}, \emph{**kwargs}}{}
Return an command line parser with the standard pre-set options plus an input and output paths as an arguments.

\end{fulllineitems}



\subsubsection{DictTree}
\label{\detokenize{ref/util/DictTree::doc}}\label{\detokenize{ref/util/DictTree:module-TotalDepth.util.DictTree}}\label{\detokenize{ref/util/DictTree:dicttree}}\index{TotalDepth.util.DictTree (module)}
A dictionary that takes a list of hashables as a key and behaves like a tree.
\index{ExceptionDictTree}

\begin{fulllineitems}
\phantomsection\label{\detokenize{ref/util/DictTree:TotalDepth.util.DictTree.ExceptionDictTree}}\pysigline{\sphinxbfcode{exception }\sphinxcode{TotalDepth.util.DictTree.}\sphinxbfcode{ExceptionDictTree}}
Exception when handling a DictTree object.

\end{fulllineitems}

\index{DictTree (class in TotalDepth.util.DictTree)}

\begin{fulllineitems}
\phantomsection\label{\detokenize{ref/util/DictTree:TotalDepth.util.DictTree.DictTree}}\pysiglinewithargsret{\sphinxbfcode{class }\sphinxcode{TotalDepth.util.DictTree.}\sphinxbfcode{DictTree}}{\emph{valIterable=None}}{}
A dictionary that takes a list of hashables as a key and behaves like a tree
\index{add() (TotalDepth.util.DictTree.DictTree method)}

\begin{fulllineitems}
\phantomsection\label{\detokenize{ref/util/DictTree:TotalDepth.util.DictTree.DictTree.add}}\pysiglinewithargsret{\sphinxbfcode{add}}{\emph{k}, \emph{v}}{}
Add a key/value. k is a list of hashables.

\end{fulllineitems}

\index{remove() (TotalDepth.util.DictTree.DictTree method)}

\begin{fulllineitems}
\phantomsection\label{\detokenize{ref/util/DictTree:TotalDepth.util.DictTree.DictTree.remove}}\pysiglinewithargsret{\sphinxbfcode{remove}}{\emph{k}, \emph{v=None}}{}
Remove a key/value. k is a list of hashables.

\end{fulllineitems}

\index{value() (TotalDepth.util.DictTree.DictTree method)}

\begin{fulllineitems}
\phantomsection\label{\detokenize{ref/util/DictTree:TotalDepth.util.DictTree.DictTree.value}}\pysiglinewithargsret{\sphinxbfcode{value}}{\emph{k}}{}
Value corresponding to a key or None. k is a list of hashables.

\end{fulllineitems}

\index{values() (TotalDepth.util.DictTree.DictTree method)}

\begin{fulllineitems}
\phantomsection\label{\detokenize{ref/util/DictTree:TotalDepth.util.DictTree.DictTree.values}}\pysiglinewithargsret{\sphinxbfcode{values}}{}{}
Returns a list of all values.

\end{fulllineitems}

\index{keys() (TotalDepth.util.DictTree.DictTree method)}

\begin{fulllineitems}
\phantomsection\label{\detokenize{ref/util/DictTree:TotalDepth.util.DictTree.DictTree.keys}}\pysiglinewithargsret{\sphinxbfcode{keys}}{}{}
Return a list of keys where each key is a list of hashables.

\end{fulllineitems}

\index{\_\_len\_\_() (TotalDepth.util.DictTree.DictTree method)}

\begin{fulllineitems}
\phantomsection\label{\detokenize{ref/util/DictTree:TotalDepth.util.DictTree.DictTree.__len__}}\pysiglinewithargsret{\sphinxbfcode{\_\_len\_\_}}{}{}
Returns the number of keys.

\end{fulllineitems}

\index{depth() (TotalDepth.util.DictTree.DictTree method)}

\begin{fulllineitems}
\phantomsection\label{\detokenize{ref/util/DictTree:TotalDepth.util.DictTree.DictTree.depth}}\pysiglinewithargsret{\sphinxbfcode{depth}}{}{}
Returns the maximum tree depth as an integer.

\end{fulllineitems}

\index{\_\_weakref\_\_ (TotalDepth.util.DictTree.DictTree attribute)}

\begin{fulllineitems}
\phantomsection\label{\detokenize{ref/util/DictTree:TotalDepth.util.DictTree.DictTree.__weakref__}}\pysigline{\sphinxbfcode{\_\_weakref\_\_}}
list of weak references to the object (if defined)

\end{fulllineitems}


\end{fulllineitems}

\index{DictTreeHtmlTable (class in TotalDepth.util.DictTree)}

\begin{fulllineitems}
\phantomsection\label{\detokenize{ref/util/DictTree:TotalDepth.util.DictTree.DictTreeHtmlTable}}\pysiglinewithargsret{\sphinxbfcode{class }\sphinxcode{TotalDepth.util.DictTree.}\sphinxbfcode{DictTreeHtmlTable}}{\emph{*args}}{}
A sub-class of DictTree that helps writing HTML row/col span tables
Suppose we have a tree like this:

\begin{sphinxVerbatim}[commandchars=\\\{\}]
                        \PYG{o}{\textbar{}}\PYG{o}{\PYGZhy{}} \PYG{n}{AAA}
                        \PYG{o}{\textbar{}}
                \PYG{o}{\textbar{}}\PYG{o}{\PYGZhy{}} \PYG{n}{AA} \PYG{o}{\PYGZhy{}}\PYG{o}{\PYGZhy{}}\PYG{o}{\textbar{}}\PYG{o}{\PYGZhy{}} \PYG{n}{AAB}
                \PYG{o}{\textbar{}}       \PYG{o}{\textbar{}}
                \PYG{o}{\textbar{}}       \PYG{o}{\textbar{}}\PYG{o}{\PYGZhy{}} \PYG{n}{AAC}
        \PYG{o}{\textbar{}}\PYG{o}{\PYGZhy{}} \PYG{n}{A} \PYG{o}{\PYGZhy{}}\PYG{o}{\PYGZhy{}}\PYG{o}{\PYGZhy{}}\PYG{o}{\textbar{}}
\PYG{n}{Root} \PYG{o}{\PYGZhy{}}\PYG{o}{\PYGZhy{}}\PYG{o}{\PYGZhy{}}\PYG{o}{\textbar{}}       \PYG{o}{\textbar{}}\PYG{o}{\PYGZhy{}} \PYG{n}{AB}
        \PYG{o}{\textbar{}}       \PYG{o}{\textbar{}}
        \PYG{o}{\textbar{}}       \PYG{o}{\textbar{}}\PYG{o}{\PYGZhy{}} \PYG{n}{AC} \PYG{o}{\PYGZhy{}}\PYG{o}{\PYGZhy{}}\PYG{o}{\PYGZhy{}}\PYG{o}{\PYGZhy{}} \PYG{n}{ACA}
        \PYG{o}{\textbar{}}
        \PYG{o}{\textbar{}}\PYG{o}{\PYGZhy{}} \PYG{n}{B}
        \PYG{o}{\textbar{}}
        \PYG{o}{\textbar{}}\PYG{o}{\PYGZhy{}} \PYG{n}{C} \PYG{o}{\PYGZhy{}}\PYG{o}{\PYGZhy{}}\PYG{o}{\PYGZhy{}}\PYG{o}{\PYGZhy{}} \PYG{n}{CA} \PYG{o}{\PYGZhy{}}\PYG{o}{\PYGZhy{}}\PYG{o}{\PYGZhy{}}\PYG{o}{\PYGZhy{}} \PYG{n}{CAA}
\end{sphinxVerbatim}

And we want to represent the tree like this when laid out as
an HTML table:

\begin{sphinxVerbatim}[commandchars=\\\{\}]
\PYG{o}{\textbar{}}\PYG{o}{\PYGZhy{}}\PYG{o}{\PYGZhy{}}\PYG{o}{\PYGZhy{}}\PYG{o}{\PYGZhy{}}\PYG{o}{\PYGZhy{}}\PYG{o}{\PYGZhy{}}\PYG{o}{\PYGZhy{}}\PYG{o}{\PYGZhy{}}\PYG{o}{\PYGZhy{}}\PYG{o}{\PYGZhy{}}\PYG{o}{\PYGZhy{}}\PYG{o}{\PYGZhy{}}\PYG{o}{\PYGZhy{}}\PYG{o}{\PYGZhy{}}\PYG{o}{\PYGZhy{}}\PYG{o}{\PYGZhy{}}\PYG{o}{\PYGZhy{}}\PYG{o}{\PYGZhy{}}\PYG{o}{\PYGZhy{}}\PYG{o}{\PYGZhy{}}\PYG{o}{\PYGZhy{}}\PYG{o}{\PYGZhy{}}\PYG{o}{\PYGZhy{}}\PYG{o}{\textbar{}}
\PYG{o}{\textbar{}} \PYG{n}{A}     \PYG{o}{\textbar{}} \PYG{n}{AA}    \PYG{o}{\textbar{}} \PYG{n}{AAA}   \PYG{o}{\textbar{}}
\PYG{o}{\textbar{}}       \PYG{o}{\textbar{}}       \PYG{o}{\textbar{}}\PYG{o}{\PYGZhy{}}\PYG{o}{\PYGZhy{}}\PYG{o}{\PYGZhy{}}\PYG{o}{\PYGZhy{}}\PYG{o}{\PYGZhy{}}\PYG{o}{\PYGZhy{}}\PYG{o}{\PYGZhy{}}\PYG{o}{\textbar{}}
\PYG{o}{\textbar{}}       \PYG{o}{\textbar{}}       \PYG{o}{\textbar{}} \PYG{n}{AAB}   \PYG{o}{\textbar{}}
\PYG{o}{\textbar{}}       \PYG{o}{\textbar{}}       \PYG{o}{\textbar{}}\PYG{o}{\PYGZhy{}}\PYG{o}{\PYGZhy{}}\PYG{o}{\PYGZhy{}}\PYG{o}{\PYGZhy{}}\PYG{o}{\PYGZhy{}}\PYG{o}{\PYGZhy{}}\PYG{o}{\PYGZhy{}}\PYG{o}{\textbar{}}
\PYG{o}{\textbar{}}       \PYG{o}{\textbar{}}       \PYG{o}{\textbar{}} \PYG{n}{AAC}   \PYG{o}{\textbar{}}
\PYG{o}{\textbar{}}       \PYG{o}{\textbar{}}\PYG{o}{\PYGZhy{}}\PYG{o}{\PYGZhy{}}\PYG{o}{\PYGZhy{}}\PYG{o}{\PYGZhy{}}\PYG{o}{\PYGZhy{}}\PYG{o}{\PYGZhy{}}\PYG{o}{\PYGZhy{}}\PYG{o}{\PYGZhy{}}\PYG{o}{\PYGZhy{}}\PYG{o}{\PYGZhy{}}\PYG{o}{\PYGZhy{}}\PYG{o}{\PYGZhy{}}\PYG{o}{\PYGZhy{}}\PYG{o}{\PYGZhy{}}\PYG{o}{\PYGZhy{}}\PYG{o}{\textbar{}}
\PYG{o}{\textbar{}}       \PYG{o}{\textbar{}} \PYG{n}{AB}            \PYG{o}{\textbar{}}
\PYG{o}{\textbar{}}       \PYG{o}{\textbar{}}\PYG{o}{\PYGZhy{}}\PYG{o}{\PYGZhy{}}\PYG{o}{\PYGZhy{}}\PYG{o}{\PYGZhy{}}\PYG{o}{\PYGZhy{}}\PYG{o}{\PYGZhy{}}\PYG{o}{\PYGZhy{}}\PYG{o}{\PYGZhy{}}\PYG{o}{\PYGZhy{}}\PYG{o}{\PYGZhy{}}\PYG{o}{\PYGZhy{}}\PYG{o}{\PYGZhy{}}\PYG{o}{\PYGZhy{}}\PYG{o}{\PYGZhy{}}\PYG{o}{\PYGZhy{}}\PYG{o}{\textbar{}}
\PYG{o}{\textbar{}}       \PYG{o}{\textbar{}} \PYG{n}{AC}    \PYG{o}{\textbar{}} \PYG{n}{ACA}   \PYG{o}{\textbar{}}
\PYG{o}{\textbar{}}\PYG{o}{\PYGZhy{}}\PYG{o}{\PYGZhy{}}\PYG{o}{\PYGZhy{}}\PYG{o}{\PYGZhy{}}\PYG{o}{\PYGZhy{}}\PYG{o}{\PYGZhy{}}\PYG{o}{\PYGZhy{}}\PYG{o}{\PYGZhy{}}\PYG{o}{\PYGZhy{}}\PYG{o}{\PYGZhy{}}\PYG{o}{\PYGZhy{}}\PYG{o}{\PYGZhy{}}\PYG{o}{\PYGZhy{}}\PYG{o}{\PYGZhy{}}\PYG{o}{\PYGZhy{}}\PYG{o}{\PYGZhy{}}\PYG{o}{\PYGZhy{}}\PYG{o}{\PYGZhy{}}\PYG{o}{\PYGZhy{}}\PYG{o}{\PYGZhy{}}\PYG{o}{\PYGZhy{}}\PYG{o}{\PYGZhy{}}\PYG{o}{\PYGZhy{}}\PYG{o}{\textbar{}}
\PYG{o}{\textbar{}} \PYG{n}{B}                     \PYG{o}{\textbar{}}
\PYG{o}{\textbar{}}\PYG{o}{\PYGZhy{}}\PYG{o}{\PYGZhy{}}\PYG{o}{\PYGZhy{}}\PYG{o}{\PYGZhy{}}\PYG{o}{\PYGZhy{}}\PYG{o}{\PYGZhy{}}\PYG{o}{\PYGZhy{}}\PYG{o}{\PYGZhy{}}\PYG{o}{\PYGZhy{}}\PYG{o}{\PYGZhy{}}\PYG{o}{\PYGZhy{}}\PYG{o}{\PYGZhy{}}\PYG{o}{\PYGZhy{}}\PYG{o}{\PYGZhy{}}\PYG{o}{\PYGZhy{}}\PYG{o}{\PYGZhy{}}\PYG{o}{\PYGZhy{}}\PYG{o}{\PYGZhy{}}\PYG{o}{\PYGZhy{}}\PYG{o}{\PYGZhy{}}\PYG{o}{\PYGZhy{}}\PYG{o}{\PYGZhy{}}\PYG{o}{\PYGZhy{}}\PYG{o}{\textbar{}}
\PYG{o}{\textbar{}} \PYG{n}{C}     \PYG{o}{\textbar{}} \PYG{n}{CA}    \PYG{o}{\textbar{}} \PYG{n}{CAA}   \PYG{o}{\textbar{}}
\PYG{o}{\textbar{}}\PYG{o}{\PYGZhy{}}\PYG{o}{\PYGZhy{}}\PYG{o}{\PYGZhy{}}\PYG{o}{\PYGZhy{}}\PYG{o}{\PYGZhy{}}\PYG{o}{\PYGZhy{}}\PYG{o}{\PYGZhy{}}\PYG{o}{\PYGZhy{}}\PYG{o}{\PYGZhy{}}\PYG{o}{\PYGZhy{}}\PYG{o}{\PYGZhy{}}\PYG{o}{\PYGZhy{}}\PYG{o}{\PYGZhy{}}\PYG{o}{\PYGZhy{}}\PYG{o}{\PYGZhy{}}\PYG{o}{\PYGZhy{}}\PYG{o}{\PYGZhy{}}\PYG{o}{\PYGZhy{}}\PYG{o}{\PYGZhy{}}\PYG{o}{\PYGZhy{}}\PYG{o}{\PYGZhy{}}\PYG{o}{\PYGZhy{}}\PYG{o}{\PYGZhy{}}\PYG{o}{\textbar{}}
\end{sphinxVerbatim}

In this example the tree is loaded branch by branch thus:
myTree = DictTreeHtmlTable()
myTree.add((‘A’, ‘AA’, ‘AAA’), None)
myTree.add((‘A’, ‘AA’, ‘AAB’), None)
myTree.add((‘A’, ‘AA’, ‘AAC’), None)
myTree.add((‘A’, ‘AB’,), None)
myTree.add((‘A’, ‘AC’, ‘ACA’), None)
myTree.add((‘B’,), None)
myTree.add((‘C’, ‘CA’, ‘CAA’), None)

The HTML code generator can be used like this:

\begin{sphinxVerbatim}[commandchars=\\\{\}]
\PYG{c+c1}{\PYGZsh{} Write: \PYGZlt{}table border=\PYGZdq{}2\PYGZdq{} width=\PYGZdq{}100\PYGZpc{}\PYGZdq{}\PYGZgt{}}
\PYG{k}{for} \PYG{n}{anEvent} \PYG{o+ow}{in} \PYG{n}{myTree}\PYG{o}{.}\PYG{n}{genColRowEvents}\PYG{p}{(}\PYG{p}{)}\PYG{p}{:}
    \PYG{k}{if} \PYG{n}{anEvent} \PYG{o}{==} \PYG{n}{myTree}\PYG{o}{.}\PYG{n}{ROW\PYGZus{}OPEN}\PYG{p}{:}
        \PYG{c+c1}{\PYGZsh{} Write out the \PYGZsq{}\PYGZlt{}tr\PYGZgt{}\PYGZsq{} element}
    \PYG{k}{elif} \PYG{n}{anEvent} \PYG{o}{==} \PYG{n}{myTree}\PYG{o}{.}\PYG{n}{ROW\PYGZus{}CLOSE}\PYG{p}{:}
        \PYG{c+c1}{\PYGZsh{} Write out the \PYGZsq{}\PYGZlt{}/tr\PYGZgt{}\PYGZsq{} element}
    \PYG{k}{else}\PYG{p}{:}
        \PYG{n}{k}\PYG{p}{,} \PYG{n}{v}\PYG{p}{,} \PYG{n}{r}\PYG{p}{,} \PYG{n}{c} \PYG{o}{=} \PYG{n}{anEvent}
        \PYG{c+c1}{\PYGZsh{} Write \PYGZsq{}\PYGZlt{}td rowspan=\PYGZdq{}\PYGZpc{}d\PYGZdq{} colspan=\PYGZdq{}\PYGZpc{}d\PYGZdq{}\PYGZgt{}\PYGZpc{}s\PYGZlt{}/td\PYGZgt{}\PYGZsq{} \PYGZpc{} (r, c, v)}
\PYG{c+c1}{\PYGZsh{} Write: \PYGZlt{}/table\PYGZgt{}}
\end{sphinxVerbatim}

And the HTML will look like this:

\begin{sphinxVerbatim}[commandchars=\\\{\}]
\PYG{o}{\PYGZlt{}}\PYG{n}{table} \PYG{n}{border}\PYG{o}{=}\PYG{l+s+s2}{\PYGZdq{}}\PYG{l+s+s2}{2}\PYG{l+s+s2}{\PYGZdq{}} \PYG{n}{width}\PYG{o}{=}\PYG{l+s+s2}{\PYGZdq{}}\PYG{l+s+s2}{100}\PYG{l+s+s2}{\PYGZpc{}}\PYG{l+s+s2}{\PYGZdq{}}\PYG{o}{\PYGZgt{}}
    \PYG{o}{\PYGZlt{}}\PYG{n}{tr} \PYG{n}{valign}\PYG{o}{=}\PYG{l+s+s2}{\PYGZdq{}}\PYG{l+s+s2}{top}\PYG{l+s+s2}{\PYGZdq{}}\PYG{o}{\PYGZgt{}}
        \PYG{o}{\PYGZlt{}}\PYG{n}{td} \PYG{n}{rowspan}\PYG{o}{=}\PYG{l+s+s2}{\PYGZdq{}}\PYG{l+s+s2}{5}\PYG{l+s+s2}{\PYGZdq{}}\PYG{o}{\PYGZgt{}}\PYG{n}{A}\PYG{o}{\PYGZlt{}}\PYG{o}{/}\PYG{n}{td}\PYG{o}{\PYGZgt{}}
        \PYG{o}{\PYGZlt{}}\PYG{n}{td} \PYG{n}{rowspan}\PYG{o}{=}\PYG{l+s+s2}{\PYGZdq{}}\PYG{l+s+s2}{3}\PYG{l+s+s2}{\PYGZdq{}}\PYG{o}{\PYGZgt{}}\PYG{n}{AA}\PYG{o}{\PYGZlt{}}\PYG{o}{/}\PYG{n}{td}\PYG{o}{\PYGZgt{}}
        \PYG{o}{\PYGZlt{}}\PYG{n}{td}\PYG{o}{\PYGZgt{}}\PYG{n}{AAA}\PYG{o}{\PYGZlt{}}\PYG{o}{/}\PYG{n}{td}\PYG{o}{\PYGZgt{}}
    \PYG{o}{\PYGZlt{}}\PYG{o}{/}\PYG{n}{tr}\PYG{o}{\PYGZgt{}}
    \PYG{o}{\PYGZlt{}}\PYG{n}{tr} \PYG{n}{valign}\PYG{o}{=}\PYG{l+s+s2}{\PYGZdq{}}\PYG{l+s+s2}{top}\PYG{l+s+s2}{\PYGZdq{}}\PYG{o}{\PYGZgt{}}
        \PYG{o}{\PYGZlt{}}\PYG{n}{td}\PYG{o}{\PYGZgt{}}\PYG{n}{AAB}\PYG{o}{\PYGZlt{}}\PYG{o}{/}\PYG{n}{td}\PYG{o}{\PYGZgt{}}
    \PYG{o}{\PYGZlt{}}\PYG{o}{/}\PYG{n}{tr}\PYG{o}{\PYGZgt{}}
    \PYG{o}{\PYGZlt{}}\PYG{n}{tr} \PYG{n}{valign}\PYG{o}{=}\PYG{l+s+s2}{\PYGZdq{}}\PYG{l+s+s2}{top}\PYG{l+s+s2}{\PYGZdq{}}\PYG{o}{\PYGZgt{}}
        \PYG{o}{\PYGZlt{}}\PYG{n}{td}\PYG{o}{\PYGZgt{}}\PYG{n}{AAC}\PYG{o}{\PYGZlt{}}\PYG{o}{/}\PYG{n}{td}\PYG{o}{\PYGZgt{}}
    \PYG{o}{\PYGZlt{}}\PYG{o}{/}\PYG{n}{tr}\PYG{o}{\PYGZgt{}}
    \PYG{o}{\PYGZlt{}}\PYG{n}{tr} \PYG{n}{valign}\PYG{o}{=}\PYG{l+s+s2}{\PYGZdq{}}\PYG{l+s+s2}{top}\PYG{l+s+s2}{\PYGZdq{}}\PYG{o}{\PYGZgt{}}
        \PYG{o}{\PYGZlt{}}\PYG{n}{td} \PYG{n}{colspan}\PYG{o}{=}\PYG{l+s+s2}{\PYGZdq{}}\PYG{l+s+s2}{2}\PYG{l+s+s2}{\PYGZdq{}}\PYG{o}{\PYGZgt{}}\PYG{n}{AB}\PYG{o}{\PYGZlt{}}\PYG{o}{/}\PYG{n}{td}\PYG{o}{\PYGZgt{}}
    \PYG{o}{\PYGZlt{}}\PYG{o}{/}\PYG{n}{tr}\PYG{o}{\PYGZgt{}}
    \PYG{o}{\PYGZlt{}}\PYG{n}{tr} \PYG{n}{valign}\PYG{o}{=}\PYG{l+s+s2}{\PYGZdq{}}\PYG{l+s+s2}{top}\PYG{l+s+s2}{\PYGZdq{}}\PYG{o}{\PYGZgt{}}
        \PYG{o}{\PYGZlt{}}\PYG{n}{td}\PYG{o}{\PYGZgt{}}\PYG{n}{AC}\PYG{o}{\PYGZlt{}}\PYG{o}{/}\PYG{n}{td}\PYG{o}{\PYGZgt{}}
        \PYG{o}{\PYGZlt{}}\PYG{n}{td}\PYG{o}{\PYGZgt{}}\PYG{n}{ACA}\PYG{o}{\PYGZlt{}}\PYG{o}{/}\PYG{n}{td}\PYG{o}{\PYGZgt{}}
    \PYG{o}{\PYGZlt{}}\PYG{o}{/}\PYG{n}{tr}\PYG{o}{\PYGZgt{}}
    \PYG{o}{\PYGZlt{}}\PYG{n}{tr} \PYG{n}{valign}\PYG{o}{=}\PYG{l+s+s2}{\PYGZdq{}}\PYG{l+s+s2}{top}\PYG{l+s+s2}{\PYGZdq{}}\PYG{o}{\PYGZgt{}}
        \PYG{o}{\PYGZlt{}}\PYG{n}{td} \PYG{n}{colspan}\PYG{o}{=}\PYG{l+s+s2}{\PYGZdq{}}\PYG{l+s+s2}{3}\PYG{l+s+s2}{\PYGZdq{}}\PYG{o}{\PYGZgt{}}\PYG{n}{B}\PYG{o}{\PYGZlt{}}\PYG{o}{/}\PYG{n}{td}\PYG{o}{\PYGZgt{}}
    \PYG{o}{\PYGZlt{}}\PYG{o}{/}\PYG{n}{tr}\PYG{o}{\PYGZgt{}}
    \PYG{o}{\PYGZlt{}}\PYG{n}{tr} \PYG{n}{valign}\PYG{o}{=}\PYG{l+s+s2}{\PYGZdq{}}\PYG{l+s+s2}{top}\PYG{l+s+s2}{\PYGZdq{}}\PYG{o}{\PYGZgt{}}
        \PYG{o}{\PYGZlt{}}\PYG{n}{td}\PYG{o}{\PYGZgt{}}\PYG{n}{C}\PYG{o}{\PYGZlt{}}\PYG{o}{/}\PYG{n}{td}\PYG{o}{\PYGZgt{}}
        \PYG{o}{\PYGZlt{}}\PYG{n}{td}\PYG{o}{\PYGZgt{}}\PYG{n}{CA}\PYG{o}{\PYGZlt{}}\PYG{o}{/}\PYG{n}{td}\PYG{o}{\PYGZgt{}}
        \PYG{o}{\PYGZlt{}}\PYG{n}{td}\PYG{o}{\PYGZgt{}}\PYG{n}{CAA}\PYG{o}{\PYGZlt{}}\PYG{o}{/}\PYG{n}{td}\PYG{o}{\PYGZgt{}}
    \PYG{o}{\PYGZlt{}}\PYG{o}{/}\PYG{n}{tr}\PYG{o}{\PYGZgt{}}
\PYG{o}{\PYGZlt{}}\PYG{o}{/}\PYG{n}{table}\PYG{o}{\PYGZgt{}}
\end{sphinxVerbatim}
\index{setColRowSpan() (TotalDepth.util.DictTree.DictTreeHtmlTable method)}

\begin{fulllineitems}
\phantomsection\label{\detokenize{ref/util/DictTree:TotalDepth.util.DictTree.DictTreeHtmlTable.setColRowSpan}}\pysiglinewithargsret{\sphinxbfcode{setColRowSpan}}{}{}
Top level call that sets colspan and rowspan attributes.

\end{fulllineitems}

\index{genColRowEvents() (TotalDepth.util.DictTree.DictTreeHtmlTable method)}

\begin{fulllineitems}
\phantomsection\label{\detokenize{ref/util/DictTree:TotalDepth.util.DictTree.DictTreeHtmlTable.genColRowEvents}}\pysiglinewithargsret{\sphinxbfcode{genColRowEvents}}{}{}
Returns a set of events that are quadruples.
(key\_branch, value, rowspan\_int, colspan\_int)
The branch is a list of keys the from the branch of the tree.
The rowspan and colspan are both integers.
At the start of the a \textless{}tr\textgreater{} there will be a ROW\_OPEN
and at row end (\textless{}/tr\textgreater{} a ROW\_CLOSE will be yielded

\end{fulllineitems}


\end{fulllineitems}



\subsubsection{Directory Walking}
\label{\detokenize{ref/util/DirWalk:module-TotalDepth.util.DirWalk}}\label{\detokenize{ref/util/DirWalk:directory-walking}}\label{\detokenize{ref/util/DirWalk::doc}}\index{TotalDepth.util.DirWalk (module)}
Provides various ways of walking a directory tree

Created on Jun 9, 2011
\index{FileInOut (class in TotalDepth.util.DirWalk)}

\begin{fulllineitems}
\phantomsection\label{\detokenize{ref/util/DirWalk:TotalDepth.util.DirWalk.FileInOut}}\pysiglinewithargsret{\sphinxbfcode{class }\sphinxcode{TotalDepth.util.DirWalk.}\sphinxbfcode{FileInOut}}{\emph{filePathIn}, \emph{filePathOut}}{}
A pair of (in, out) file paths
\index{\_\_getnewargs\_\_() (TotalDepth.util.DirWalk.FileInOut method)}

\begin{fulllineitems}
\phantomsection\label{\detokenize{ref/util/DirWalk:TotalDepth.util.DirWalk.FileInOut.__getnewargs__}}\pysiglinewithargsret{\sphinxbfcode{\_\_getnewargs\_\_}}{}{}
Return self as a plain tuple.  Used by copy and pickle.

\end{fulllineitems}

\index{\_\_new\_\_() (TotalDepth.util.DirWalk.FileInOut static method)}

\begin{fulllineitems}
\phantomsection\label{\detokenize{ref/util/DirWalk:TotalDepth.util.DirWalk.FileInOut.__new__}}\pysiglinewithargsret{\sphinxbfcode{static }\sphinxbfcode{\_\_new\_\_}}{\emph{\_cls}, \emph{filePathIn}, \emph{filePathOut}}{}
Create new instance of FileInOut(filePathIn, filePathOut)

\end{fulllineitems}

\index{\_\_repr\_\_() (TotalDepth.util.DirWalk.FileInOut method)}

\begin{fulllineitems}
\phantomsection\label{\detokenize{ref/util/DirWalk:TotalDepth.util.DirWalk.FileInOut.__repr__}}\pysiglinewithargsret{\sphinxbfcode{\_\_repr\_\_}}{}{}
Return a nicely formatted representation string

\end{fulllineitems}

\index{filePathIn (TotalDepth.util.DirWalk.FileInOut attribute)}

\begin{fulllineitems}
\phantomsection\label{\detokenize{ref/util/DirWalk:TotalDepth.util.DirWalk.FileInOut.filePathIn}}\pysigline{\sphinxbfcode{filePathIn}}
Alias for field number 0

\end{fulllineitems}

\index{filePathOut (TotalDepth.util.DirWalk.FileInOut attribute)}

\begin{fulllineitems}
\phantomsection\label{\detokenize{ref/util/DirWalk:TotalDepth.util.DirWalk.FileInOut.filePathOut}}\pysigline{\sphinxbfcode{filePathOut}}
Alias for field number 1

\end{fulllineitems}


\end{fulllineitems}

\index{ExceptionDirWalk}

\begin{fulllineitems}
\phantomsection\label{\detokenize{ref/util/DirWalk:TotalDepth.util.DirWalk.ExceptionDirWalk}}\pysigline{\sphinxbfcode{exception }\sphinxcode{TotalDepth.util.DirWalk.}\sphinxbfcode{ExceptionDirWalk}}
Exception class for this module.

\end{fulllineitems}

\index{genBigFirst() (in module TotalDepth.util.DirWalk)}

\begin{fulllineitems}
\phantomsection\label{\detokenize{ref/util/DirWalk:TotalDepth.util.DirWalk.genBigFirst}}\pysiglinewithargsret{\sphinxcode{TotalDepth.util.DirWalk.}\sphinxbfcode{genBigFirst}}{\emph{d}}{}
Generator that yields the biggest files (name not path) first.
This is fairly simple in that it it only looks the current directory not
only sub-directories. Useful for multiprocessing.

\end{fulllineitems}

\index{dirWalk() (in module TotalDepth.util.DirWalk)}

\begin{fulllineitems}
\phantomsection\label{\detokenize{ref/util/DirWalk:TotalDepth.util.DirWalk.dirWalk}}\pysiglinewithargsret{\sphinxcode{TotalDepth.util.DirWalk.}\sphinxbfcode{dirWalk}}{\emph{theIn}, \emph{theOut=None}, \emph{theFnMatch=None}, \emph{recursive=False}, \emph{bigFirst=False}}{}
Walks a directory tree generating file paths.

theIn - The input directory.

theOut - The output directory. If None then input file paths as strings
will be generated If non-None this function will yield
FileInOut(in, out) objects.
NOTE: This does not create the output directory structure, it is up to
the caller to do that.

theFnMatch - A glob like match pattern for file names (not tested for directory names).

recursive - Boolean to recurse or not.

bigFirst - If True then the largest files in  directory are given first. If False it is alphabetical.

\end{fulllineitems}



\subsubsection{Timing Code Execution}
\label{\detokenize{ref/util/ExecTimer:module-TotalDepth.util.ExecTimer}}\label{\detokenize{ref/util/ExecTimer::doc}}\label{\detokenize{ref/util/ExecTimer:timing-code-execution}}\index{TotalDepth.util.ExecTimer (module)}
Has classes for timing execution
\index{ExceptionExecTimer}

\begin{fulllineitems}
\phantomsection\label{\detokenize{ref/util/ExecTimer:TotalDepth.util.ExecTimer.ExceptionExecTimer}}\pysigline{\sphinxbfcode{exception }\sphinxcode{TotalDepth.util.ExecTimer.}\sphinxbfcode{ExceptionExecTimer}}
Specialisation of exception for this module.

\end{fulllineitems}

\index{ExecEvent (class in TotalDepth.util.ExecTimer)}

\begin{fulllineitems}
\phantomsection\label{\detokenize{ref/util/ExecTimer:TotalDepth.util.ExecTimer.ExecEvent}}\pysiglinewithargsret{\sphinxbfcode{class }\sphinxcode{TotalDepth.util.ExecTimer.}\sphinxbfcode{ExecEvent}}{\emph{desc}}{}
Records the timing of a single event.
\index{stop() (TotalDepth.util.ExecTimer.ExecEvent method)}

\begin{fulllineitems}
\phantomsection\label{\detokenize{ref/util/ExecTimer:TotalDepth.util.ExecTimer.ExecEvent.stop}}\pysiglinewithargsret{\sphinxbfcode{stop}}{\emph{bytes=0}}{}
Stop the timer.

\end{fulllineitems}

\index{hasCompleted (TotalDepth.util.ExecTimer.ExecEvent attribute)}

\begin{fulllineitems}
\phantomsection\label{\detokenize{ref/util/ExecTimer:TotalDepth.util.ExecTimer.ExecEvent.hasCompleted}}\pysigline{\sphinxbfcode{hasCompleted}}
True if the timer has been stopped.

\end{fulllineitems}

\index{tExec (TotalDepth.util.ExecTimer.ExecEvent attribute)}

\begin{fulllineitems}
\phantomsection\label{\detokenize{ref/util/ExecTimer:TotalDepth.util.ExecTimer.ExecEvent.tExec}}\pysigline{\sphinxbfcode{tExec}}
Executions time in seconds.

\end{fulllineitems}

\index{lenDesc() (TotalDepth.util.ExecTimer.ExecEvent method)}

\begin{fulllineitems}
\phantomsection\label{\detokenize{ref/util/ExecTimer:TotalDepth.util.ExecTimer.ExecEvent.lenDesc}}\pysiglinewithargsret{\sphinxbfcode{lenDesc}}{}{}
Length of the task description string.

\end{fulllineitems}

\index{writeToStderr() (TotalDepth.util.ExecTimer.ExecEvent method)}

\begin{fulllineitems}
\phantomsection\label{\detokenize{ref/util/ExecTimer:TotalDepth.util.ExecTimer.ExecEvent.writeToStderr}}\pysiglinewithargsret{\sphinxbfcode{writeToStderr}}{\emph{descWidth=0}}{}
Write self to sys.stderr.

\end{fulllineitems}

\index{writeToStream() (TotalDepth.util.ExecTimer.ExecEvent method)}

\begin{fulllineitems}
\phantomsection\label{\detokenize{ref/util/ExecTimer:TotalDepth.util.ExecTimer.ExecEvent.writeToStream}}\pysiglinewithargsret{\sphinxbfcode{writeToStream}}{\emph{theS}, \emph{descWidth=0}}{}
Write self to provided stream.

\end{fulllineitems}

\index{\_\_weakref\_\_ (TotalDepth.util.ExecTimer.ExecEvent attribute)}

\begin{fulllineitems}
\phantomsection\label{\detokenize{ref/util/ExecTimer:TotalDepth.util.ExecTimer.ExecEvent.__weakref__}}\pysigline{\sphinxbfcode{\_\_weakref\_\_}}
list of weak references to the object (if defined)

\end{fulllineitems}


\end{fulllineitems}

\index{ExecTimerList (class in TotalDepth.util.ExecTimer)}

\begin{fulllineitems}
\phantomsection\label{\detokenize{ref/util/ExecTimer:TotalDepth.util.ExecTimer.ExecTimerList}}\pysigline{\sphinxbfcode{class }\sphinxcode{TotalDepth.util.ExecTimer.}\sphinxbfcode{ExecTimerList}}
Maintains a list of execution time objects
\index{\_\_init\_\_() (TotalDepth.util.ExecTimer.ExecTimerList method)}

\begin{fulllineitems}
\phantomsection\label{\detokenize{ref/util/ExecTimer:TotalDepth.util.ExecTimer.ExecTimerList.__init__}}\pysiglinewithargsret{\sphinxbfcode{\_\_init\_\_}}{}{}
Constructor

\end{fulllineitems}

\index{\_\_len\_\_() (TotalDepth.util.ExecTimer.ExecTimerList method)}

\begin{fulllineitems}
\phantomsection\label{\detokenize{ref/util/ExecTimer:TotalDepth.util.ExecTimer.ExecTimerList.__len__}}\pysiglinewithargsret{\sphinxbfcode{\_\_len\_\_}}{}{}
Number of task timers.

\end{fulllineitems}

\index{startNewTimer() (TotalDepth.util.ExecTimer.ExecTimerList method)}

\begin{fulllineitems}
\phantomsection\label{\detokenize{ref/util/ExecTimer:TotalDepth.util.ExecTimer.ExecTimerList.startNewTimer}}\pysiglinewithargsret{\sphinxbfcode{startNewTimer}}{\emph{theDesc}}{}
Load a new task timer starting right now.

\end{fulllineitems}

\index{timer (TotalDepth.util.ExecTimer.ExecTimerList attribute)}

\begin{fulllineitems}
\phantomsection\label{\detokenize{ref/util/ExecTimer:TotalDepth.util.ExecTimer.ExecTimerList.timer}}\pysigline{\sphinxbfcode{timer}}
The current timer.

\end{fulllineitems}

\index{hasActiveTimer (TotalDepth.util.ExecTimer.ExecTimerList attribute)}

\begin{fulllineitems}
\phantomsection\label{\detokenize{ref/util/ExecTimer:TotalDepth.util.ExecTimer.ExecTimerList.hasActiveTimer}}\pysigline{\sphinxbfcode{hasActiveTimer}}
True if there is a running timer, False if there are either no timers
or the latest timer is halted.

\end{fulllineitems}

\index{stopTimer() (TotalDepth.util.ExecTimer.ExecTimerList method)}

\begin{fulllineitems}
\phantomsection\label{\detokenize{ref/util/ExecTimer:TotalDepth.util.ExecTimer.ExecTimerList.stopTimer}}\pysiglinewithargsret{\sphinxbfcode{stopTimer}}{\emph{bytes=0}}{}
Stop current timer.

\end{fulllineitems}

\index{writeToStderr() (TotalDepth.util.ExecTimer.ExecTimerList method)}

\begin{fulllineitems}
\phantomsection\label{\detokenize{ref/util/ExecTimer:TotalDepth.util.ExecTimer.ExecTimerList.writeToStderr}}\pysiglinewithargsret{\sphinxbfcode{writeToStderr}}{}{}
Write self to sys.stderr.

\end{fulllineitems}

\index{writeToStream() (TotalDepth.util.ExecTimer.ExecTimerList method)}

\begin{fulllineitems}
\phantomsection\label{\detokenize{ref/util/ExecTimer:TotalDepth.util.ExecTimer.ExecTimerList.writeToStream}}\pysiglinewithargsret{\sphinxbfcode{writeToStream}}{\emph{theS}}{}
Write self to provided stream.

\end{fulllineitems}

\index{\_\_weakref\_\_ (TotalDepth.util.ExecTimer.ExecTimerList attribute)}

\begin{fulllineitems}
\phantomsection\label{\detokenize{ref/util/ExecTimer:TotalDepth.util.ExecTimer.ExecTimerList.__weakref__}}\pysigline{\sphinxbfcode{\_\_weakref\_\_}}
list of weak references to the object (if defined)

\end{fulllineitems}


\end{fulllineitems}



\subsubsection{Look-ahead File Buffer}
\label{\detokenize{ref/util/FileBuffer:look-ahead-file-buffer}}\label{\detokenize{ref/util/FileBuffer::doc}}\label{\detokenize{ref/util/FileBuffer:module-TotalDepth.util.FileBuffer}}\index{TotalDepth.util.FileBuffer (module)}
Provides a ‘look ahead’ file buffer where the caller can inspect bytes ahead
of the current position.

Created on Oct 26, 2011
\index{ExceptionFileBuffer}

\begin{fulllineitems}
\phantomsection\label{\detokenize{ref/util/FileBuffer:TotalDepth.util.FileBuffer.ExceptionFileBuffer}}\pysigline{\sphinxbfcode{exception }\sphinxcode{TotalDepth.util.FileBuffer.}\sphinxbfcode{ExceptionFileBuffer}}
Specialisation of Exception for the FileBuffer module.
\index{\_\_weakref\_\_ (TotalDepth.util.FileBuffer.ExceptionFileBuffer attribute)}

\begin{fulllineitems}
\phantomsection\label{\detokenize{ref/util/FileBuffer:TotalDepth.util.FileBuffer.ExceptionFileBuffer.__weakref__}}\pysigline{\sphinxbfcode{\_\_weakref\_\_}}
list of weak references to the object (if defined)

\end{fulllineitems}


\end{fulllineitems}

\index{ExceptionFileBufferEOF}

\begin{fulllineitems}
\phantomsection\label{\detokenize{ref/util/FileBuffer:TotalDepth.util.FileBuffer.ExceptionFileBufferEOF}}\pysigline{\sphinxbfcode{exception }\sphinxcode{TotalDepth.util.FileBuffer.}\sphinxbfcode{ExceptionFileBufferEOF}}
Specialisation of Exception for the FileBuffer EOF.

\end{fulllineitems}

\index{FileBuffer (class in TotalDepth.util.FileBuffer)}

\begin{fulllineitems}
\phantomsection\label{\detokenize{ref/util/FileBuffer:TotalDepth.util.FileBuffer.FileBuffer}}\pysiglinewithargsret{\sphinxbfcode{class }\sphinxcode{TotalDepth.util.FileBuffer.}\sphinxbfcode{FileBuffer}}{\emph{f}}{}
Provides a buffer interface to a file where the user can look ahead
any distance from the current position.
\index{tell() (TotalDepth.util.FileBuffer.FileBuffer method)}

\begin{fulllineitems}
\phantomsection\label{\detokenize{ref/util/FileBuffer:TotalDepth.util.FileBuffer.FileBuffer.tell}}\pysiglinewithargsret{\sphinxbfcode{tell}}{}{}
Current file position.

\end{fulllineitems}

\index{step() (TotalDepth.util.FileBuffer.FileBuffer method)}

\begin{fulllineitems}
\phantomsection\label{\detokenize{ref/util/FileBuffer:TotalDepth.util.FileBuffer.FileBuffer.step}}\pysiglinewithargsret{\sphinxbfcode{step}}{}{}
Increment the file position by one byte, returns the byte just read.

\end{fulllineitems}

\index{\_\_getitem\_\_() (TotalDepth.util.FileBuffer.FileBuffer method)}

\begin{fulllineitems}
\phantomsection\label{\detokenize{ref/util/FileBuffer:TotalDepth.util.FileBuffer.FileBuffer.__getitem__}}\pysiglinewithargsret{\sphinxbfcode{\_\_getitem\_\_}}{\emph{i}}{}
Get an arbitrary byte or slice.

\end{fulllineitems}

\index{\_\_weakref\_\_ (TotalDepth.util.FileBuffer.FileBuffer attribute)}

\begin{fulllineitems}
\phantomsection\label{\detokenize{ref/util/FileBuffer:TotalDepth.util.FileBuffer.FileBuffer.__weakref__}}\pysigline{\sphinxbfcode{\_\_weakref\_\_}}
list of weak references to the object (if defined)

\end{fulllineitems}


\end{fulllineitems}



\subsubsection{Histogram}
\label{\detokenize{ref/util/Histogram:histogram}}\label{\detokenize{ref/util/Histogram:module-TotalDepth.util.Histogram}}\label{\detokenize{ref/util/Histogram::doc}}\index{TotalDepth.util.Histogram (module)}
Produces histograms.

Created on Nov 29, 2011
\index{Histogram (class in TotalDepth.util.Histogram)}

\begin{fulllineitems}
\phantomsection\label{\detokenize{ref/util/Histogram:TotalDepth.util.Histogram.Histogram}}\pysigline{\sphinxbfcode{class }\sphinxcode{TotalDepth.util.Histogram.}\sphinxbfcode{Histogram}}
A histogram class.
\index{add() (TotalDepth.util.Histogram.Histogram method)}

\begin{fulllineitems}
\phantomsection\label{\detokenize{ref/util/Histogram:TotalDepth.util.Histogram.Histogram.add}}\pysiglinewithargsret{\sphinxbfcode{add}}{\emph{x}, \emph{count=1}}{}
Increments the count of value x by count (default 1).

\end{fulllineitems}

\index{\_\_getitem\_\_() (TotalDepth.util.Histogram.Histogram method)}

\begin{fulllineitems}
\phantomsection\label{\detokenize{ref/util/Histogram:TotalDepth.util.Histogram.Histogram.__getitem__}}\pysiglinewithargsret{\sphinxbfcode{\_\_getitem\_\_}}{\emph{x}}{}
Returns the current count of x.

\end{fulllineitems}

\index{strRep() (TotalDepth.util.Histogram.Histogram method)}

\begin{fulllineitems}
\phantomsection\label{\detokenize{ref/util/Histogram:TotalDepth.util.Histogram.Histogram.strRep}}\pysiglinewithargsret{\sphinxbfcode{strRep}}{\emph{width=75}, \emph{chr='+'}, \emph{valTitle=''}, \emph{inclCount=False}}{}
Returns a string representation of the histogram in ASCII.

width - The maximum width to use.

chr - The character to use in the plot.

valTitle - The title to use for values.

inclCount - Include tha tacture count for each value?

\end{fulllineitems}

\index{\_\_weakref\_\_ (TotalDepth.util.Histogram.Histogram attribute)}

\begin{fulllineitems}
\phantomsection\label{\detokenize{ref/util/Histogram:TotalDepth.util.Histogram.Histogram.__weakref__}}\pysigline{\sphinxbfcode{\_\_weakref\_\_}}
list of weak references to the object (if defined)

\end{fulllineitems}


\end{fulllineitems}



\paragraph{Testing}
\label{\detokenize{ref/util/Histogram:testing}}
Tests are in \sphinxcode{test/TestHistogram.py}


\subparagraph{Running Tests}
\label{\detokenize{ref/util/Histogram:running-tests}}
\begin{sphinxVerbatim}[commandchars=\\\{\}]
\PYGZdl{} python3 test/testHistogram.py
\end{sphinxVerbatim}


\subparagraph{Test Coverage}
\label{\detokenize{ref/util/Histogram:test-coverage}}
\begin{sphinxVerbatim}[commandchars=\\\{\}]
\PYGZdl{} coverage run test/testHistogram.py
...
\PYGZdl{} coverage report \PYGZhy{}m
\end{sphinxVerbatim}


\paragraph{Examples}
\label{\detokenize{ref/util/Histogram:examples}}
\begin{sphinxVerbatim}[commandchars=\\\{\}]
\PYG{k+kn}{from} \PYG{n+nn}{TotalDepth}\PYG{n+nn}{.}\PYG{n+nn}{util} \PYG{k}{import} \PYG{n}{Histogram}

\PYG{n}{myH} \PYG{o}{=} \PYG{n}{Histogram}\PYG{o}{.}\PYG{n}{Histogram}\PYG{p}{(}\PYG{p}{)}
\PYG{k}{for} \PYG{n}{x} \PYG{o+ow}{in} \PYG{n+nb}{range}\PYG{p}{(}\PYG{l+m+mi}{1}\PYG{p}{,} \PYG{l+m+mi}{12}\PYG{p}{)}\PYG{p}{:}
    \PYG{n+nb+bp}{self}\PYG{o}{.}\PYG{n}{\PYGZus{}hist}\PYG{o}{.}\PYG{n}{add}\PYG{p}{(}\PYG{n}{x}\PYG{p}{,} \PYG{n}{x}\PYG{p}{)}
\PYG{n+nb}{print}\PYG{p}{(}\PYG{n+nb+bp}{self}\PYG{o}{.}\PYG{n}{\PYGZus{}hist}\PYG{o}{.}\PYG{n}{strRep}\PYG{p}{(}\PYG{p}{)}\PYG{p}{)}
\PYG{c+c1}{\PYGZsh{} Prints \PYGZdq{}\PYGZdq{}\PYGZdq{} 1 \textbar{} ++++++}
 \PYG{l+m+mi}{2} \PYG{o}{\textbar{}} \PYG{o}{+}\PYG{o}{+}\PYG{o}{+}\PYG{o}{+}\PYG{o}{+}\PYG{o}{+}\PYG{o}{+}\PYG{o}{+}\PYG{o}{+}\PYG{o}{+}\PYG{o}{+}\PYG{o}{+}\PYG{o}{+}
 \PYG{l+m+mi}{3} \PYG{o}{\textbar{}} \PYG{o}{+}\PYG{o}{+}\PYG{o}{+}\PYG{o}{+}\PYG{o}{+}\PYG{o}{+}\PYG{o}{+}\PYG{o}{+}\PYG{o}{+}\PYG{o}{+}\PYG{o}{+}\PYG{o}{+}\PYG{o}{+}\PYG{o}{+}\PYG{o}{+}\PYG{o}{+}\PYG{o}{+}\PYG{o}{+}\PYG{o}{+}
 \PYG{l+m+mi}{4} \PYG{o}{\textbar{}} \PYG{o}{+}\PYG{o}{+}\PYG{o}{+}\PYG{o}{+}\PYG{o}{+}\PYG{o}{+}\PYG{o}{+}\PYG{o}{+}\PYG{o}{+}\PYG{o}{+}\PYG{o}{+}\PYG{o}{+}\PYG{o}{+}\PYG{o}{+}\PYG{o}{+}\PYG{o}{+}\PYG{o}{+}\PYG{o}{+}\PYG{o}{+}\PYG{o}{+}\PYG{o}{+}\PYG{o}{+}\PYG{o}{+}\PYG{o}{+}\PYG{o}{+}
 \PYG{l+m+mi}{5} \PYG{o}{\textbar{}} \PYG{o}{+}\PYG{o}{+}\PYG{o}{+}\PYG{o}{+}\PYG{o}{+}\PYG{o}{+}\PYG{o}{+}\PYG{o}{+}\PYG{o}{+}\PYG{o}{+}\PYG{o}{+}\PYG{o}{+}\PYG{o}{+}\PYG{o}{+}\PYG{o}{+}\PYG{o}{+}\PYG{o}{+}\PYG{o}{+}\PYG{o}{+}\PYG{o}{+}\PYG{o}{+}\PYG{o}{+}\PYG{o}{+}\PYG{o}{+}\PYG{o}{+}\PYG{o}{+}\PYG{o}{+}\PYG{o}{+}\PYG{o}{+}\PYG{o}{+}\PYG{o}{+}\PYG{o}{+}
 \PYG{l+m+mi}{6} \PYG{o}{\textbar{}} \PYG{o}{+}\PYG{o}{+}\PYG{o}{+}\PYG{o}{+}\PYG{o}{+}\PYG{o}{+}\PYG{o}{+}\PYG{o}{+}\PYG{o}{+}\PYG{o}{+}\PYG{o}{+}\PYG{o}{+}\PYG{o}{+}\PYG{o}{+}\PYG{o}{+}\PYG{o}{+}\PYG{o}{+}\PYG{o}{+}\PYG{o}{+}\PYG{o}{+}\PYG{o}{+}\PYG{o}{+}\PYG{o}{+}\PYG{o}{+}\PYG{o}{+}\PYG{o}{+}\PYG{o}{+}\PYG{o}{+}\PYG{o}{+}\PYG{o}{+}\PYG{o}{+}\PYG{o}{+}\PYG{o}{+}\PYG{o}{+}\PYG{o}{+}\PYG{o}{+}\PYG{o}{+}\PYG{o}{+}
 \PYG{l+m+mi}{7} \PYG{o}{\textbar{}} \PYG{o}{+}\PYG{o}{+}\PYG{o}{+}\PYG{o}{+}\PYG{o}{+}\PYG{o}{+}\PYG{o}{+}\PYG{o}{+}\PYG{o}{+}\PYG{o}{+}\PYG{o}{+}\PYG{o}{+}\PYG{o}{+}\PYG{o}{+}\PYG{o}{+}\PYG{o}{+}\PYG{o}{+}\PYG{o}{+}\PYG{o}{+}\PYG{o}{+}\PYG{o}{+}\PYG{o}{+}\PYG{o}{+}\PYG{o}{+}\PYG{o}{+}\PYG{o}{+}\PYG{o}{+}\PYG{o}{+}\PYG{o}{+}\PYG{o}{+}\PYG{o}{+}\PYG{o}{+}\PYG{o}{+}\PYG{o}{+}\PYG{o}{+}\PYG{o}{+}\PYG{o}{+}\PYG{o}{+}\PYG{o}{+}\PYG{o}{+}\PYG{o}{+}\PYG{o}{+}\PYG{o}{+}\PYG{o}{+}\PYG{o}{+}
 \PYG{l+m+mi}{8} \PYG{o}{\textbar{}} \PYG{o}{+}\PYG{o}{+}\PYG{o}{+}\PYG{o}{+}\PYG{o}{+}\PYG{o}{+}\PYG{o}{+}\PYG{o}{+}\PYG{o}{+}\PYG{o}{+}\PYG{o}{+}\PYG{o}{+}\PYG{o}{+}\PYG{o}{+}\PYG{o}{+}\PYG{o}{+}\PYG{o}{+}\PYG{o}{+}\PYG{o}{+}\PYG{o}{+}\PYG{o}{+}\PYG{o}{+}\PYG{o}{+}\PYG{o}{+}\PYG{o}{+}\PYG{o}{+}\PYG{o}{+}\PYG{o}{+}\PYG{o}{+}\PYG{o}{+}\PYG{o}{+}\PYG{o}{+}\PYG{o}{+}\PYG{o}{+}\PYG{o}{+}\PYG{o}{+}\PYG{o}{+}\PYG{o}{+}\PYG{o}{+}\PYG{o}{+}\PYG{o}{+}\PYG{o}{+}\PYG{o}{+}\PYG{o}{+}\PYG{o}{+}\PYG{o}{+}\PYG{o}{+}\PYG{o}{+}\PYG{o}{+}\PYG{o}{+}\PYG{o}{+}
 \PYG{l+m+mi}{9} \PYG{o}{\textbar{}} \PYG{o}{+}\PYG{o}{+}\PYG{o}{+}\PYG{o}{+}\PYG{o}{+}\PYG{o}{+}\PYG{o}{+}\PYG{o}{+}\PYG{o}{+}\PYG{o}{+}\PYG{o}{+}\PYG{o}{+}\PYG{o}{+}\PYG{o}{+}\PYG{o}{+}\PYG{o}{+}\PYG{o}{+}\PYG{o}{+}\PYG{o}{+}\PYG{o}{+}\PYG{o}{+}\PYG{o}{+}\PYG{o}{+}\PYG{o}{+}\PYG{o}{+}\PYG{o}{+}\PYG{o}{+}\PYG{o}{+}\PYG{o}{+}\PYG{o}{+}\PYG{o}{+}\PYG{o}{+}\PYG{o}{+}\PYG{o}{+}\PYG{o}{+}\PYG{o}{+}\PYG{o}{+}\PYG{o}{+}\PYG{o}{+}\PYG{o}{+}\PYG{o}{+}\PYG{o}{+}\PYG{o}{+}\PYG{o}{+}\PYG{o}{+}\PYG{o}{+}\PYG{o}{+}\PYG{o}{+}\PYG{o}{+}\PYG{o}{+}\PYG{o}{+}\PYG{o}{+}\PYG{o}{+}\PYG{o}{+}\PYG{o}{+}\PYG{o}{+}\PYG{o}{+}
\PYG{l+m+mi}{10} \PYG{o}{\textbar{}} \PYG{o}{+}\PYG{o}{+}\PYG{o}{+}\PYG{o}{+}\PYG{o}{+}\PYG{o}{+}\PYG{o}{+}\PYG{o}{+}\PYG{o}{+}\PYG{o}{+}\PYG{o}{+}\PYG{o}{+}\PYG{o}{+}\PYG{o}{+}\PYG{o}{+}\PYG{o}{+}\PYG{o}{+}\PYG{o}{+}\PYG{o}{+}\PYG{o}{+}\PYG{o}{+}\PYG{o}{+}\PYG{o}{+}\PYG{o}{+}\PYG{o}{+}\PYG{o}{+}\PYG{o}{+}\PYG{o}{+}\PYG{o}{+}\PYG{o}{+}\PYG{o}{+}\PYG{o}{+}\PYG{o}{+}\PYG{o}{+}\PYG{o}{+}\PYG{o}{+}\PYG{o}{+}\PYG{o}{+}\PYG{o}{+}\PYG{o}{+}\PYG{o}{+}\PYG{o}{+}\PYG{o}{+}\PYG{o}{+}\PYG{o}{+}\PYG{o}{+}\PYG{o}{+}\PYG{o}{+}\PYG{o}{+}\PYG{o}{+}\PYG{o}{+}\PYG{o}{+}\PYG{o}{+}\PYG{o}{+}\PYG{o}{+}\PYG{o}{+}\PYG{o}{+}\PYG{o}{+}\PYG{o}{+}\PYG{o}{+}\PYG{o}{+}\PYG{o}{+}\PYG{o}{+}\PYG{o}{+}
\PYG{l+m+mi}{11} \PYG{o}{\textbar{}} \PYG{o}{+}\PYG{o}{+}\PYG{o}{+}\PYG{o}{+}\PYG{o}{+}\PYG{o}{+}\PYG{o}{+}\PYG{o}{+}\PYG{o}{+}\PYG{o}{+}\PYG{o}{+}\PYG{o}{+}\PYG{o}{+}\PYG{o}{+}\PYG{o}{+}\PYG{o}{+}\PYG{o}{+}\PYG{o}{+}\PYG{o}{+}\PYG{o}{+}\PYG{o}{+}\PYG{o}{+}\PYG{o}{+}\PYG{o}{+}\PYG{o}{+}\PYG{o}{+}\PYG{o}{+}\PYG{o}{+}\PYG{o}{+}\PYG{o}{+}\PYG{o}{+}\PYG{o}{+}\PYG{o}{+}\PYG{o}{+}\PYG{o}{+}\PYG{o}{+}\PYG{o}{+}\PYG{o}{+}\PYG{o}{+}\PYG{o}{+}\PYG{o}{+}\PYG{o}{+}\PYG{o}{+}\PYG{o}{+}\PYG{o}{+}\PYG{o}{+}\PYG{o}{+}\PYG{o}{+}\PYG{o}{+}\PYG{o}{+}\PYG{o}{+}\PYG{o}{+}\PYG{o}{+}\PYG{o}{+}\PYG{o}{+}\PYG{o}{+}\PYG{o}{+}\PYG{o}{+}\PYG{o}{+}\PYG{o}{+}\PYG{o}{+}\PYG{o}{+}\PYG{o}{+}\PYG{o}{+}\PYG{o}{+}\PYG{o}{+}\PYG{o}{+}\PYG{o}{+}\PYG{o}{+}\PYG{o}{+}\PYG{l+s+s2}{\PYGZdq{}\PYGZdq{}\PYGZdq{}}
\end{sphinxVerbatim}


\subsubsection{HTML Utilities}
\label{\detokenize{ref/util/HtmlUtils:html-utilities}}\label{\detokenize{ref/util/HtmlUtils::doc}}\label{\detokenize{ref/util/HtmlUtils:module-TotalDepth.util.HtmlUtils}}\index{TotalDepth.util.HtmlUtils (module)}
HTML utility functions.
\index{retHtmlFileName() (in module TotalDepth.util.HtmlUtils)}

\begin{fulllineitems}
\phantomsection\label{\detokenize{ref/util/HtmlUtils:TotalDepth.util.HtmlUtils.retHtmlFileName}}\pysiglinewithargsret{\sphinxcode{TotalDepth.util.HtmlUtils.}\sphinxbfcode{retHtmlFileName}}{\emph{thePath}}{}
Creates a unique, short, human readable file name base on the input
file path.

\end{fulllineitems}

\index{retHtmlFileLink() (in module TotalDepth.util.HtmlUtils)}

\begin{fulllineitems}
\phantomsection\label{\detokenize{ref/util/HtmlUtils:TotalDepth.util.HtmlUtils.retHtmlFileLink}}\pysiglinewithargsret{\sphinxcode{TotalDepth.util.HtmlUtils.}\sphinxbfcode{retHtmlFileLink}}{\emph{theSrcPath}, \emph{theLineNum}}{}
Returns a link to a file/line.

\end{fulllineitems}

\index{writeHtmlFileLink() (in module TotalDepth.util.HtmlUtils)}

\begin{fulllineitems}
\phantomsection\label{\detokenize{ref/util/HtmlUtils:TotalDepth.util.HtmlUtils.writeHtmlFileLink}}\pysiglinewithargsret{\sphinxcode{TotalDepth.util.HtmlUtils.}\sphinxbfcode{writeHtmlFileLink}}{\emph{theS}, \emph{theSrcPath}, \emph{theLineNum}, \emph{theText=''}, \emph{theClass=None}}{}
Writes a link to another HTML file that represents source code.
theS is an XHTML stream.
theSrcPath is the path of the original source, which will be encoded with retHtmlFileName().
theLineNum is an integer line number in the target.
theText is optional navigation text.
theClass is optional CSS class for the navigation text.

\end{fulllineitems}

\index{writeHtmlFileAnchor() (in module TotalDepth.util.HtmlUtils)}

\begin{fulllineitems}
\phantomsection\label{\detokenize{ref/util/HtmlUtils:TotalDepth.util.HtmlUtils.writeHtmlFileAnchor}}\pysiglinewithargsret{\sphinxcode{TotalDepth.util.HtmlUtils.}\sphinxbfcode{writeHtmlFileAnchor}}{\emph{theS}, \emph{theLineNum}, \emph{theText=''}, \emph{theClass=None}}{}
Write an anchor to the stream.

\end{fulllineitems}

\index{pathSplit() (in module TotalDepth.util.HtmlUtils)}

\begin{fulllineitems}
\phantomsection\label{\detokenize{ref/util/HtmlUtils:TotalDepth.util.HtmlUtils.pathSplit}}\pysiglinewithargsret{\sphinxcode{TotalDepth.util.HtmlUtils.}\sphinxbfcode{pathSplit}}{\emph{p}}{}
Split a path into its components.

\end{fulllineitems}

\index{writeFileListAsTable() (in module TotalDepth.util.HtmlUtils)}

\begin{fulllineitems}
\phantomsection\label{\detokenize{ref/util/HtmlUtils:TotalDepth.util.HtmlUtils.writeFileListAsTable}}\pysiglinewithargsret{\sphinxcode{TotalDepth.util.HtmlUtils.}\sphinxbfcode{writeFileListAsTable}}{\emph{theS}, \emph{theFileLinkS}, \emph{tableAttrs}, \emph{includeKeyTail}}{}
Writes a list of file names as an HTML table looking like a directory
structure. theFileLinkS is a list of pairs (file\_path, href).
The navigation text in the cell will be the basename of the file\_path.

\end{fulllineitems}

\index{writeFileListTrippleAsTable() (in module TotalDepth.util.HtmlUtils)}

\begin{fulllineitems}
\phantomsection\label{\detokenize{ref/util/HtmlUtils:TotalDepth.util.HtmlUtils.writeFileListTrippleAsTable}}\pysiglinewithargsret{\sphinxcode{TotalDepth.util.HtmlUtils.}\sphinxbfcode{writeFileListTrippleAsTable}}{\emph{theS}, \emph{theFileLinkS}, \emph{tableAttrs}, \emph{includeKeyTail}}{}
Writes a list of file names as an HTML table looking like a directory
structure. theFileLinkS is a list of triples (file\_name, href, nav\_text).

\end{fulllineitems}

\index{writeDictTreeAsTable() (in module TotalDepth.util.HtmlUtils)}

\begin{fulllineitems}
\phantomsection\label{\detokenize{ref/util/HtmlUtils:TotalDepth.util.HtmlUtils.writeDictTreeAsTable}}\pysiglinewithargsret{\sphinxcode{TotalDepth.util.HtmlUtils.}\sphinxbfcode{writeDictTreeAsTable}}{\emph{theS}, \emph{theDt}, \emph{tableAttrs}, \emph{includeKeyTail}}{}
Writes a DictTreeHtmlTable object as a table, for example as a directory
structure.

The key list in the DictTreeHtmlTable object is the path to the file
i.e. os.path.abspath(p).split(os.sep) and the value is expected to be a
pair of (link, nav\_text) or None.

\end{fulllineitems}

\index{writeFilePathsAsTable() (in module TotalDepth.util.HtmlUtils)}

\begin{fulllineitems}
\phantomsection\label{\detokenize{ref/util/HtmlUtils:TotalDepth.util.HtmlUtils.writeFilePathsAsTable}}\pysiglinewithargsret{\sphinxcode{TotalDepth.util.HtmlUtils.}\sphinxbfcode{writeFilePathsAsTable}}{\emph{valueType}, \emph{theS}, \emph{theKvS}, \emph{tableStyle}, \emph{fnTd}}{}
Writes file paths as a table, for example as a directory structure.

valueType - The type of the value: None \textbar{} ‘list’ \textbar{} ‘set’

theKvS - A list of pairs (file\_path, value).

tableStyle - The style used for the table.

fnTd - A callback function that is executed for a \textless{}td\textgreater{} element when
there is a non-None value. This is called with the following arguments:
\begin{itemize}
\item {} 
theS - The HTML stream.

\item {} 
attrs - A map of attrs that include the rowspan/colspan for the \textless{}td\textgreater{}

\item {} 
k - The key as a list of path components.

\item {} 
v - The value given by the caller.

\end{itemize}

\end{fulllineitems}



\subsubsection{XML Writer Module}
\label{\detokenize{ref/util/XmlWrite:xml-writer-module}}\label{\detokenize{ref/util/XmlWrite::doc}}

\paragraph{Introduction}
\label{\detokenize{ref/util/XmlWrite:introduction}}
The important classes here are:
\begin{itemize}
\item {} 
\sphinxcode{XmlWrite.XMLStream} (and derived classes): these manage the low level stream output.

\item {} 
\sphinxcode{XmlWrite.XMLElement}: these handles opening and closing elements via the Context Manager. They require an initialised XMLStream in their constructor.

\end{itemize}

Together, if used correctly, these make writing output to a stream fluent in the code. They also make strong guarantees about the encoding and well-formedness of the result.


\paragraph{Reference}
\label{\detokenize{ref/util/XmlWrite:module-TotalDepth.util.XmlWrite}}\label{\detokenize{ref/util/XmlWrite:reference}}\index{TotalDepth.util.XmlWrite (module)}
Writes XML and XHTML.
\index{ExceptionXml}

\begin{fulllineitems}
\phantomsection\label{\detokenize{ref/util/XmlWrite:TotalDepth.util.XmlWrite.ExceptionXml}}\pysigline{\sphinxbfcode{exception }\sphinxcode{TotalDepth.util.XmlWrite.}\sphinxbfcode{ExceptionXml}}
Exception specialisation for the XML writer.

\end{fulllineitems}

\index{ExceptionXmlEndElement}

\begin{fulllineitems}
\phantomsection\label{\detokenize{ref/util/XmlWrite:TotalDepth.util.XmlWrite.ExceptionXmlEndElement}}\pysigline{\sphinxbfcode{exception }\sphinxcode{TotalDepth.util.XmlWrite.}\sphinxbfcode{ExceptionXmlEndElement}}
Exception specialisation for end of element.

\end{fulllineitems}

\index{encodeString() (in module TotalDepth.util.XmlWrite)}

\begin{fulllineitems}
\phantomsection\label{\detokenize{ref/util/XmlWrite:TotalDepth.util.XmlWrite.encodeString}}\pysiglinewithargsret{\sphinxcode{TotalDepth.util.XmlWrite.}\sphinxbfcode{encodeString}}{\emph{theS}, \emph{theCharPrefix='\_'}}{}
Returns a string that is the argument encoded.
RFC3548:

\begin{sphinxVerbatim}[commandchars=\\\{\}]
                   \PYG{n}{Table} \PYG{l+m+mi}{1}\PYG{p}{:} \PYG{n}{The} \PYG{n}{Base} \PYG{l+m+mi}{64} \PYG{n}{Alphabet}
\PYG{n}{Value} \PYG{n}{Encoding}  \PYG{n}{Value} \PYG{n}{Encoding}  \PYG{n}{Value} \PYG{n}{Encoding}  \PYG{n}{Value} \PYG{n}{Encoding}
    \PYG{l+m+mi}{0} \PYG{n}{A}            \PYG{l+m+mi}{17} \PYG{n}{R}            \PYG{l+m+mi}{34} \PYG{n}{i}            \PYG{l+m+mi}{51} \PYG{n}{z}
    \PYG{l+m+mi}{1} \PYG{n}{B}            \PYG{l+m+mi}{18} \PYG{n}{S}            \PYG{l+m+mi}{35} \PYG{n}{j}            \PYG{l+m+mi}{52} \PYG{l+m+mi}{0}
    \PYG{l+m+mi}{2} \PYG{n}{C}            \PYG{l+m+mi}{19} \PYG{n}{T}            \PYG{l+m+mi}{36} \PYG{n}{k}            \PYG{l+m+mi}{53} \PYG{l+m+mi}{1}
    \PYG{l+m+mi}{3} \PYG{n}{D}            \PYG{l+m+mi}{20} \PYG{n}{U}            \PYG{l+m+mi}{37} \PYG{n}{l}            \PYG{l+m+mi}{54} \PYG{l+m+mi}{2}
    \PYG{l+m+mi}{4} \PYG{n}{E}            \PYG{l+m+mi}{21} \PYG{n}{V}            \PYG{l+m+mi}{38} \PYG{n}{m}            \PYG{l+m+mi}{55} \PYG{l+m+mi}{3}
    \PYG{l+m+mi}{5} \PYG{n}{F}            \PYG{l+m+mi}{22} \PYG{n}{W}            \PYG{l+m+mi}{39} \PYG{n}{n}            \PYG{l+m+mi}{56} \PYG{l+m+mi}{4}
    \PYG{l+m+mi}{6} \PYG{n}{G}            \PYG{l+m+mi}{23} \PYG{n}{X}            \PYG{l+m+mi}{40} \PYG{n}{o}            \PYG{l+m+mi}{57} \PYG{l+m+mi}{5}
    \PYG{l+m+mi}{7} \PYG{n}{H}            \PYG{l+m+mi}{24} \PYG{n}{Y}            \PYG{l+m+mi}{41} \PYG{n}{p}            \PYG{l+m+mi}{58} \PYG{l+m+mi}{6}
    \PYG{l+m+mi}{8} \PYG{n}{I}            \PYG{l+m+mi}{25} \PYG{n}{Z}            \PYG{l+m+mi}{42} \PYG{n}{q}            \PYG{l+m+mi}{59} \PYG{l+m+mi}{7}
    \PYG{l+m+mi}{9} \PYG{n}{J}            \PYG{l+m+mi}{26} \PYG{n}{a}            \PYG{l+m+mi}{43} \PYG{n}{r}            \PYG{l+m+mi}{60} \PYG{l+m+mi}{8}
   \PYG{l+m+mi}{10} \PYG{n}{K}            \PYG{l+m+mi}{27} \PYG{n}{b}            \PYG{l+m+mi}{44} \PYG{n}{s}            \PYG{l+m+mi}{61} \PYG{l+m+mi}{9}
   \PYG{l+m+mi}{11} \PYG{n}{L}            \PYG{l+m+mi}{28} \PYG{n}{c}            \PYG{l+m+mi}{45} \PYG{n}{t}            \PYG{l+m+mi}{62} \PYG{o}{+}
   \PYG{l+m+mi}{12} \PYG{n}{M}            \PYG{l+m+mi}{29} \PYG{n}{d}            \PYG{l+m+mi}{46} \PYG{n}{u}            \PYG{l+m+mi}{63} \PYG{o}{/}
   \PYG{l+m+mi}{13} \PYG{n}{N}            \PYG{l+m+mi}{30} \PYG{n}{e}            \PYG{l+m+mi}{47} \PYG{n}{v}
   \PYG{l+m+mi}{14} \PYG{n}{O}            \PYG{l+m+mi}{31} \PYG{n}{f}            \PYG{l+m+mi}{48} \PYG{n}{w}         \PYG{p}{(}\PYG{n}{pad}\PYG{p}{)} \PYG{o}{=}
   \PYG{l+m+mi}{15} \PYG{n}{P}            \PYG{l+m+mi}{32} \PYG{n}{g}            \PYG{l+m+mi}{49} \PYG{n}{x}
   \PYG{l+m+mi}{16} \PYG{n}{Q}            \PYG{l+m+mi}{33} \PYG{n}{h}            \PYG{l+m+mi}{50} \PYG{n}{y}
\end{sphinxVerbatim}

See section 3 of : \sphinxurl{http://www.faqs.org/rfcs/rfc3548.html}

\end{fulllineitems}

\index{decodeString() (in module TotalDepth.util.XmlWrite)}

\begin{fulllineitems}
\phantomsection\label{\detokenize{ref/util/XmlWrite:TotalDepth.util.XmlWrite.decodeString}}\pysiglinewithargsret{\sphinxcode{TotalDepth.util.XmlWrite.}\sphinxbfcode{decodeString}}{\emph{theS}}{}
Returns a string that is the argument decoded. May raise a TypeError.

\end{fulllineitems}

\index{nameFromString() (in module TotalDepth.util.XmlWrite)}

\begin{fulllineitems}
\phantomsection\label{\detokenize{ref/util/XmlWrite:TotalDepth.util.XmlWrite.nameFromString}}\pysiglinewithargsret{\sphinxcode{TotalDepth.util.XmlWrite.}\sphinxbfcode{nameFromString}}{\emph{theStr}}{}
Returns a name from a string.

See \sphinxurl{http://www.w3.org/TR/1999/REC-html401-19991224/types.html\#type-cdata}

“ID and NAME tokens must begin with a letter ({[}A-Za-z{]}) and may be
followed by any number of letters, digits ({[}0-9{]}), hyphens (“-“),
underscores (“\_”), colons (“:”), and periods (“.”).

This also works for in namespaces as ‘:’ is not used in the encoding.

\end{fulllineitems}

\index{XmlStream (class in TotalDepth.util.XmlWrite)}

\begin{fulllineitems}
\phantomsection\label{\detokenize{ref/util/XmlWrite:TotalDepth.util.XmlWrite.XmlStream}}\pysiglinewithargsret{\sphinxbfcode{class }\sphinxcode{TotalDepth.util.XmlWrite.}\sphinxbfcode{XmlStream}}{\emph{theFout}, \emph{theEnc='utf-8'}, \emph{theDtdLocal=None}, \emph{theId=0}}{}
Creates and maintains an XML output stream.
\index{\_\_init\_\_() (TotalDepth.util.XmlWrite.XmlStream method)}

\begin{fulllineitems}
\phantomsection\label{\detokenize{ref/util/XmlWrite:TotalDepth.util.XmlWrite.XmlStream.__init__}}\pysiglinewithargsret{\sphinxbfcode{\_\_init\_\_}}{\emph{theFout}, \emph{theEnc='utf-8'}, \emph{theDtdLocal=None}, \emph{theId=0}}{}
Initialise with a writable file like object or a file path.

theFout - The file-like object or a string. If the latter it will be
closed on \_\_exit\_\_.

theEnc - The encoding to be used.

theDtdLocal - Any local DTD as a string.

id - An integer value to use as an ID string.

\end{fulllineitems}

\index{id (TotalDepth.util.XmlWrite.XmlStream attribute)}

\begin{fulllineitems}
\phantomsection\label{\detokenize{ref/util/XmlWrite:TotalDepth.util.XmlWrite.XmlStream.id}}\pysigline{\sphinxbfcode{id}}
A unique ID in this stream. The ID is incremented on each call.

\end{fulllineitems}

\index{xmlSpacePreserve() (TotalDepth.util.XmlWrite.XmlStream method)}

\begin{fulllineitems}
\phantomsection\label{\detokenize{ref/util/XmlWrite:TotalDepth.util.XmlWrite.XmlStream.xmlSpacePreserve}}\pysiglinewithargsret{\sphinxbfcode{xmlSpacePreserve}}{}{}
Suspends indentation for this element and its descendants.

\end{fulllineitems}

\index{characters() (TotalDepth.util.XmlWrite.XmlStream method)}

\begin{fulllineitems}
\phantomsection\label{\detokenize{ref/util/XmlWrite:TotalDepth.util.XmlWrite.XmlStream.characters}}\pysiglinewithargsret{\sphinxbfcode{characters}}{\emph{theString}}{}
Encodes the string and writes it to the output.

\end{fulllineitems}

\index{literal() (TotalDepth.util.XmlWrite.XmlStream method)}

\begin{fulllineitems}
\phantomsection\label{\detokenize{ref/util/XmlWrite:TotalDepth.util.XmlWrite.XmlStream.literal}}\pysiglinewithargsret{\sphinxbfcode{literal}}{\emph{theString}}{}
Writes theString to the output without encoding.

\end{fulllineitems}

\index{comment() (TotalDepth.util.XmlWrite.XmlStream method)}

\begin{fulllineitems}
\phantomsection\label{\detokenize{ref/util/XmlWrite:TotalDepth.util.XmlWrite.XmlStream.comment}}\pysiglinewithargsret{\sphinxbfcode{comment}}{\emph{theS}}{}
Writes a comment to the output stream.

\end{fulllineitems}

\index{pI() (TotalDepth.util.XmlWrite.XmlStream method)}

\begin{fulllineitems}
\phantomsection\label{\detokenize{ref/util/XmlWrite:TotalDepth.util.XmlWrite.XmlStream.pI}}\pysiglinewithargsret{\sphinxbfcode{pI}}{\emph{theS}}{}
Writes a Processing Instruction to the output stream.

\end{fulllineitems}

\index{endElement() (TotalDepth.util.XmlWrite.XmlStream method)}

\begin{fulllineitems}
\phantomsection\label{\detokenize{ref/util/XmlWrite:TotalDepth.util.XmlWrite.XmlStream.endElement}}\pysiglinewithargsret{\sphinxbfcode{endElement}}{\emph{name}}{}
Ends an element.

\end{fulllineitems}

\index{writeECMAScript() (TotalDepth.util.XmlWrite.XmlStream method)}

\begin{fulllineitems}
\phantomsection\label{\detokenize{ref/util/XmlWrite:TotalDepth.util.XmlWrite.XmlStream.writeECMAScript}}\pysiglinewithargsret{\sphinxbfcode{writeECMAScript}}{\emph{theScript}}{}
Writes the ECMA script.

Example:

\begin{sphinxVerbatim}[commandchars=\\\{\}]
\PYGZlt{}script type=\PYGZdq{}text/ecmascript\PYGZdq{}\PYGZgt{}
//\PYGZlt{}![CDATA[
...
// ]]\PYGZgt{}
\PYGZlt{}/script\PYGZgt{}
\end{sphinxVerbatim}

\end{fulllineitems}

\index{\_\_enter\_\_() (TotalDepth.util.XmlWrite.XmlStream method)}

\begin{fulllineitems}
\phantomsection\label{\detokenize{ref/util/XmlWrite:TotalDepth.util.XmlWrite.XmlStream.__enter__}}\pysiglinewithargsret{\sphinxbfcode{\_\_enter\_\_}}{}{}
Context manager support.

\end{fulllineitems}

\index{\_\_exit\_\_() (TotalDepth.util.XmlWrite.XmlStream method)}

\begin{fulllineitems}
\phantomsection\label{\detokenize{ref/util/XmlWrite:TotalDepth.util.XmlWrite.XmlStream.__exit__}}\pysiglinewithargsret{\sphinxbfcode{\_\_exit\_\_}}{\emph{exc\_type}, \emph{exc\_value}, \emph{traceback}}{}
Context manager support.

\end{fulllineitems}

\index{\_\_weakref\_\_ (TotalDepth.util.XmlWrite.XmlStream attribute)}

\begin{fulllineitems}
\phantomsection\label{\detokenize{ref/util/XmlWrite:TotalDepth.util.XmlWrite.XmlStream.__weakref__}}\pysigline{\sphinxbfcode{\_\_weakref\_\_}}
list of weak references to the object (if defined)

\end{fulllineitems}


\end{fulllineitems}

\index{Element (class in TotalDepth.util.XmlWrite)}

\begin{fulllineitems}
\phantomsection\label{\detokenize{ref/util/XmlWrite:TotalDepth.util.XmlWrite.Element}}\pysiglinewithargsret{\sphinxbfcode{class }\sphinxcode{TotalDepth.util.XmlWrite.}\sphinxbfcode{Element}}{\emph{theXmlStream}, \emph{theElemName}, \emph{theAttrs=None}}{}
Represents an element in a markup stream.
\index{\_\_enter\_\_() (TotalDepth.util.XmlWrite.Element method)}

\begin{fulllineitems}
\phantomsection\label{\detokenize{ref/util/XmlWrite:TotalDepth.util.XmlWrite.Element.__enter__}}\pysiglinewithargsret{\sphinxbfcode{\_\_enter\_\_}}{}{}
Context manager support.

\end{fulllineitems}

\index{\_\_weakref\_\_ (TotalDepth.util.XmlWrite.Element attribute)}

\begin{fulllineitems}
\phantomsection\label{\detokenize{ref/util/XmlWrite:TotalDepth.util.XmlWrite.Element.__weakref__}}\pysigline{\sphinxbfcode{\_\_weakref\_\_}}
list of weak references to the object (if defined)

\end{fulllineitems}

\index{\_\_exit\_\_() (TotalDepth.util.XmlWrite.Element method)}

\begin{fulllineitems}
\phantomsection\label{\detokenize{ref/util/XmlWrite:TotalDepth.util.XmlWrite.Element.__exit__}}\pysiglinewithargsret{\sphinxbfcode{\_\_exit\_\_}}{\emph{excType}, \emph{excValue}, \emph{tb}}{}
Context manager support.

\end{fulllineitems}


\end{fulllineitems}



\paragraph{Examples}
\label{\detokenize{ref/util/XmlWrite:examples}}
All these examples assume these imports:

\begin{sphinxVerbatim}[commandchars=\\\{\}]
\PYG{k+kn}{import} \PYG{n+nn}{io}
\PYG{k+kn}{from} \PYG{n+nn}{TotalDepth}\PYG{n+nn}{.}\PYG{n+nn}{util} \PYG{k}{import} \PYG{n}{XmlWrite}
\end{sphinxVerbatim}


\subparagraph{Creating an XML Stream}
\label{\detokenize{ref/util/XmlWrite:creating-an-xml-stream}}
Create a file-like object and write to it using an \sphinxcode{XmlWrite.XmlStream}:

\begin{sphinxVerbatim}[commandchars=\\\{\}]
myF = io.StringIO()
with XmlWrite.XmlStream(myF):
    pass
print(myF.getvalue())
\PYGZsh{} Results in:
\PYGZlt{}?xml version=\PYGZsq{}1.0\PYGZsq{} encoding=\PYGZdq{}utf\PYGZhy{}8\PYGZdq{}?\PYGZgt{}\PYGZbs{}n
\end{sphinxVerbatim}


\subparagraph{Simple Elements}
\label{\detokenize{ref/util/XmlWrite:simple-elements}}
Create a file-like object and write elements to it using an \sphinxcode{XmlWrite.XmlElement}:

\begin{sphinxVerbatim}[commandchars=\\\{\}]
myF = io.StringIO()
with XmlWrite.XmlStream(myF) as xS:
    with XmlWrite.Element(xS, \PYGZsq{}Root\PYGZsq{}, \PYGZob{}\PYGZsq{}version\PYGZsq{} : \PYGZsq{}12.0\PYGZsq{}\PYGZcb{}):
        with XmlWrite.Element(xS, \PYGZsq{}A\PYGZsq{}, \PYGZob{}\PYGZsq{}attr\PYGZus{}1\PYGZsq{} : \PYGZsq{}1\PYGZsq{}\PYGZcb{}):
            pass
print(myF.getvalue())
\PYGZsh{} Results in:
\PYGZlt{}?xml version=\PYGZsq{}1.0\PYGZsq{} encoding=\PYGZdq{}utf\PYGZhy{}8\PYGZdq{}?\PYGZgt{}
\PYGZlt{}Root version=\PYGZdq{}12.0\PYGZdq{}\PYGZgt{}
    \PYGZlt{}A attr\PYGZus{}1=\PYGZdq{}1\PYGZdq{}/\PYGZgt{}
\PYGZlt{}/Root\PYGZgt{}
\end{sphinxVerbatim}


\subparagraph{Mixed Content}
\label{\detokenize{ref/util/XmlWrite:mixed-content}}
As above but writing text as well:

\begin{sphinxVerbatim}[commandchars=\\\{\}]
myF = io.StringIO()
with XmlWrite.XmlStream(myF) as xS:
    with XmlWrite.Element(xS, \PYGZsq{}Root\PYGZsq{}, \PYGZob{}\PYGZsq{}version\PYGZsq{} : \PYGZsq{}12.0\PYGZsq{}\PYGZcb{}):
        with XmlWrite.Element(xS, \PYGZsq{}A\PYGZsq{}, \PYGZob{}\PYGZsq{}attr\PYGZus{}1\PYGZsq{} : \PYGZsq{}1\PYGZsq{}\PYGZcb{}):
            xS.characters(\PYGZsq{}\PYGZlt{}\PYGZam{}\PYGZgt{}\PYGZsq{})
print(myF.getvalue())
\PYGZsh{} Results in:
\PYGZlt{}?xml version=\PYGZsq{}1.0\PYGZsq{} encoding=\PYGZdq{}utf\PYGZhy{}8\PYGZdq{}?\PYGZgt{}
\PYGZlt{}Root version=\PYGZdq{}12.0\PYGZdq{}\PYGZgt{}
    \PYGZlt{}A attr\PYGZus{}1=\PYGZdq{}1\PYGZdq{}\PYGZgt{}\PYGZam{}lt;\PYGZam{}amp;\PYGZam{}gt;\PYGZlt{}/A\PYGZgt{}
\PYGZlt{}/Root\PYGZgt{}
\end{sphinxVerbatim}


\paragraph{Testing}
\label{\detokenize{ref/util/XmlWrite:testing}}
The unit tests are in \sphinxcode{test/TestWMLWrite.py}.


\subsubsection{Plot Package Reference}
\label{\detokenize{ref/util/plot/index_plot:plot-package-reference}}\label{\detokenize{ref/util/plot/index_plot::doc}}
Contents:


\paragraph{Plot Coordinates}
\label{\detokenize{ref/util/plot/Coord::doc}}\label{\detokenize{ref/util/plot/Coord:plot-coordinates}}

\subparagraph{Main Classes}
\label{\detokenize{ref/util/plot/Coord:main-classes}}
Most classes in this module are \sphinxcode{collections.namedtuple} objects.


\begin{savenotes}\sphinxattablestart
\centering
\begin{tabulary}{\linewidth}[t]{|T|T|T|}
\hline
\sphinxstylethead{\sphinxstyletheadfamily 
Class
\unskip}\relax &\sphinxstylethead{\sphinxstyletheadfamily 
Description
\unskip}\relax &\sphinxstylethead{\sphinxstyletheadfamily 
Attributes
\unskip}\relax \\
\hline
\sphinxcode{Dim}
&
Linear dimension
&
value units
\\
\hline
\sphinxcode{Box}
&
A Box
&
width depth
\\
\hline
\sphinxcode{Pad}
&
Padding around a tree object
&
prev next, parent child
\\
\hline
\sphinxcode{Margin}
&
Padding around an  object
&
left right top bottom
\\
\hline
\sphinxcode{Pt}
&
A point in Cartesian space
&
x y
\\
\hline
\end{tabulary}
\par
\sphinxattableend\end{savenotes}


\subparagraph{Reference}
\label{\detokenize{ref/util/plot/Coord:reference}}\label{\detokenize{ref/util/plot/Coord:module-TotalDepth.util.plot.Coord}}\index{TotalDepth.util.plot.Coord (module)}
Provides a fairly basic two dimensional coordinate system.
\index{ExceptionCoord}

\begin{fulllineitems}
\phantomsection\label{\detokenize{ref/util/plot/Coord:TotalDepth.util.plot.Coord.ExceptionCoord}}\pysigline{\sphinxbfcode{exception }\sphinxcode{TotalDepth.util.plot.Coord.}\sphinxbfcode{ExceptionCoord}}
Exception class for representing Coordinates.

\end{fulllineitems}

\index{ExceptionCoordUnitConvert}

\begin{fulllineitems}
\phantomsection\label{\detokenize{ref/util/plot/Coord:TotalDepth.util.plot.Coord.ExceptionCoordUnitConvert}}\pysigline{\sphinxbfcode{exception }\sphinxcode{TotalDepth.util.plot.Coord.}\sphinxbfcode{ExceptionCoordUnitConvert}}
Exception raised when converting units.

\end{fulllineitems}

\index{BASE\_UNITS (in module TotalDepth.util.plot.Coord)}

\begin{fulllineitems}
\phantomsection\label{\detokenize{ref/util/plot/Coord:TotalDepth.util.plot.Coord.BASE_UNITS}}\pysigline{\sphinxcode{TotalDepth.util.plot.Coord.}\sphinxbfcode{BASE\_UNITS}\sphinxbfcode{ = 'px'}}
Base units for dimensions

\end{fulllineitems}

\index{UNIT\_MAP (in module TotalDepth.util.plot.Coord)}

\begin{fulllineitems}
\phantomsection\label{\detokenize{ref/util/plot/Coord:TotalDepth.util.plot.Coord.UNIT_MAP}}\pysigline{\sphinxcode{TotalDepth.util.plot.Coord.}\sphinxbfcode{UNIT\_MAP}\sphinxbfcode{ = \{None: 1.0, 'px': 1.0, 'pt': 1.0, 'pc': 12.0, 'in': 72.0, 'cm': 28.346456692913385, 'mm': 2.834645669291339\}}}
Map of \{unit name : conversion factor to base units, …\}

\end{fulllineitems}

\index{units() (in module TotalDepth.util.plot.Coord)}

\begin{fulllineitems}
\phantomsection\label{\detokenize{ref/util/plot/Coord:TotalDepth.util.plot.Coord.units}}\pysiglinewithargsret{\sphinxcode{TotalDepth.util.plot.Coord.}\sphinxbfcode{units}}{}{}
Returns the unsorted list of acceptable units.

\end{fulllineitems}

\index{convert() (in module TotalDepth.util.plot.Coord)}

\begin{fulllineitems}
\phantomsection\label{\detokenize{ref/util/plot/Coord:TotalDepth.util.plot.Coord.convert}}\pysiglinewithargsret{\sphinxcode{TotalDepth.util.plot.Coord.}\sphinxbfcode{convert}}{\emph{val}, \emph{unitFrom}, \emph{unitTo}}{}
Convert a value from one set of units to another.

\end{fulllineitems}

\index{Dim (class in TotalDepth.util.plot.Coord)}

\begin{fulllineitems}
\phantomsection\label{\detokenize{ref/util/plot/Coord:TotalDepth.util.plot.Coord.Dim}}\pysigline{\sphinxbfcode{class }\sphinxcode{TotalDepth.util.plot.Coord.}\sphinxbfcode{Dim}}
Represents a dimension as an engineering value i.e. a number and units.
\index{scale() (TotalDepth.util.plot.Coord.Dim method)}

\begin{fulllineitems}
\phantomsection\label{\detokenize{ref/util/plot/Coord:TotalDepth.util.plot.Coord.Dim.scale}}\pysiglinewithargsret{\sphinxbfcode{scale}}{\emph{factor}}{}
Returns a new Dim() multiplied by a factor, units are unchanged.

\end{fulllineitems}

\index{divide() (TotalDepth.util.plot.Coord.Dim method)}

\begin{fulllineitems}
\phantomsection\label{\detokenize{ref/util/plot/Coord:TotalDepth.util.plot.Coord.Dim.divide}}\pysiglinewithargsret{\sphinxbfcode{divide}}{\emph{factor}}{}
Returns a new Dim() divided by a factor, units are unchanged.

\end{fulllineitems}

\index{convert() (TotalDepth.util.plot.Coord.Dim method)}

\begin{fulllineitems}
\phantomsection\label{\detokenize{ref/util/plot/Coord:TotalDepth.util.plot.Coord.Dim.convert}}\pysiglinewithargsret{\sphinxbfcode{convert}}{\emph{u}}{}
Returns a new Dim() with units changed and value converted.

\end{fulllineitems}

\index{\_\_add\_\_() (TotalDepth.util.plot.Coord.Dim method)}

\begin{fulllineitems}
\phantomsection\label{\detokenize{ref/util/plot/Coord:TotalDepth.util.plot.Coord.Dim.__add__}}\pysiglinewithargsret{\sphinxbfcode{\_\_add\_\_}}{\emph{other}}{}
Overload self+other, returned result has the sum of self and other.
The units chosen are self’s unless self’s units are None in which case other’s
units are used (if not None).

\end{fulllineitems}

\index{\_\_sub\_\_() (TotalDepth.util.plot.Coord.Dim method)}

\begin{fulllineitems}
\phantomsection\label{\detokenize{ref/util/plot/Coord:TotalDepth.util.plot.Coord.Dim.__sub__}}\pysiglinewithargsret{\sphinxbfcode{\_\_sub\_\_}}{\emph{other}}{}
Overload self-other, returned result has the difference of self and
other. The units chosen are self’s unless self’s units are None in
which case other’s units are used (if not None).

\end{fulllineitems}

\index{\_\_iadd\_\_() (TotalDepth.util.plot.Coord.Dim method)}

\begin{fulllineitems}
\phantomsection\label{\detokenize{ref/util/plot/Coord:TotalDepth.util.plot.Coord.Dim.__iadd__}}\pysiglinewithargsret{\sphinxbfcode{\_\_iadd\_\_}}{\emph{other}}{}
Addition in place, value of other is converted to my units and added.

\end{fulllineitems}

\index{\_\_isub\_\_() (TotalDepth.util.plot.Coord.Dim method)}

\begin{fulllineitems}
\phantomsection\label{\detokenize{ref/util/plot/Coord:TotalDepth.util.plot.Coord.Dim.__isub__}}\pysiglinewithargsret{\sphinxbfcode{\_\_isub\_\_}}{\emph{other}}{}
Subtraction in place, value of other is subtracted.

\end{fulllineitems}

\index{\_\_lt\_\_() (TotalDepth.util.plot.Coord.Dim method)}

\begin{fulllineitems}
\phantomsection\label{\detokenize{ref/util/plot/Coord:TotalDepth.util.plot.Coord.Dim.__lt__}}\pysiglinewithargsret{\sphinxbfcode{\_\_lt\_\_}}{\emph{other}}{}
Returns true if self value \textless{} other value after unit conversion.

\end{fulllineitems}

\index{\_\_le\_\_() (TotalDepth.util.plot.Coord.Dim method)}

\begin{fulllineitems}
\phantomsection\label{\detokenize{ref/util/plot/Coord:TotalDepth.util.plot.Coord.Dim.__le__}}\pysiglinewithargsret{\sphinxbfcode{\_\_le\_\_}}{\emph{other}}{}
Returns true if self value \textless{}= other value after unit conversion.

\end{fulllineitems}

\index{\_\_eq\_\_() (TotalDepth.util.plot.Coord.Dim method)}

\begin{fulllineitems}
\phantomsection\label{\detokenize{ref/util/plot/Coord:TotalDepth.util.plot.Coord.Dim.__eq__}}\pysiglinewithargsret{\sphinxbfcode{\_\_eq\_\_}}{\emph{other}}{}
Returns true if self value == other value after unit conversion.

\end{fulllineitems}

\index{\_\_ne\_\_() (TotalDepth.util.plot.Coord.Dim method)}

\begin{fulllineitems}
\phantomsection\label{\detokenize{ref/util/plot/Coord:TotalDepth.util.plot.Coord.Dim.__ne__}}\pysiglinewithargsret{\sphinxbfcode{\_\_ne\_\_}}{\emph{other}}{}
Returns true if self value != other value after unit conversion.

\end{fulllineitems}

\index{\_\_gt\_\_() (TotalDepth.util.plot.Coord.Dim method)}

\begin{fulllineitems}
\phantomsection\label{\detokenize{ref/util/plot/Coord:TotalDepth.util.plot.Coord.Dim.__gt__}}\pysiglinewithargsret{\sphinxbfcode{\_\_gt\_\_}}{\emph{other}}{}
Returns true if self value \textgreater{} other value after unit conversion.

\end{fulllineitems}

\index{\_\_ge\_\_() (TotalDepth.util.plot.Coord.Dim method)}

\begin{fulllineitems}
\phantomsection\label{\detokenize{ref/util/plot/Coord:TotalDepth.util.plot.Coord.Dim.__ge__}}\pysiglinewithargsret{\sphinxbfcode{\_\_ge\_\_}}{\emph{other}}{}
Returns true if self value \textgreater{}= other value after unit conversion.

\end{fulllineitems}


\end{fulllineitems}

\index{dimIn() (in module TotalDepth.util.plot.Coord)}

\begin{fulllineitems}
\phantomsection\label{\detokenize{ref/util/plot/Coord:TotalDepth.util.plot.Coord.dimIn}}\pysiglinewithargsret{\sphinxcode{TotalDepth.util.plot.Coord.}\sphinxbfcode{dimIn}}{\emph{v}}{}
Returns a Dim object with the value in inches.

\end{fulllineitems}

\index{Box (class in TotalDepth.util.plot.Coord)}

\begin{fulllineitems}
\phantomsection\label{\detokenize{ref/util/plot/Coord:TotalDepth.util.plot.Coord.Box}}\pysigline{\sphinxbfcode{class }\sphinxcode{TotalDepth.util.plot.Coord.}\sphinxbfcode{Box}}
A named tuple that describes a box with width and depth as Dim() objects.
\index{\_\_str\_\_() (TotalDepth.util.plot.Coord.Box method)}

\begin{fulllineitems}
\phantomsection\label{\detokenize{ref/util/plot/Coord:TotalDepth.util.plot.Coord.Box.__str__}}\pysiglinewithargsret{\sphinxbfcode{\_\_str\_\_}}{}{}
Stringifying.

\end{fulllineitems}


\end{fulllineitems}

\index{Pad (class in TotalDepth.util.plot.Coord)}

\begin{fulllineitems}
\phantomsection\label{\detokenize{ref/util/plot/Coord:TotalDepth.util.plot.Coord.Pad}}\pysigline{\sphinxbfcode{class }\sphinxcode{TotalDepth.util.plot.Coord.}\sphinxbfcode{Pad}}
Padding around another object that forms the Bounding Box.
All 4 attributes are Dim() objects
\index{\_\_str\_\_() (TotalDepth.util.plot.Coord.Pad method)}

\begin{fulllineitems}
\phantomsection\label{\detokenize{ref/util/plot/Coord:TotalDepth.util.plot.Coord.Pad.__str__}}\pysiglinewithargsret{\sphinxbfcode{\_\_str\_\_}}{}{}
Stringifying.

\end{fulllineitems}


\end{fulllineitems}

\index{Margin (class in TotalDepth.util.plot.Coord)}

\begin{fulllineitems}
\phantomsection\label{\detokenize{ref/util/plot/Coord:TotalDepth.util.plot.Coord.Margin}}\pysigline{\sphinxbfcode{class }\sphinxcode{TotalDepth.util.plot.Coord.}\sphinxbfcode{Margin}}
Margin padding around another object. All 4 attributes are Coord.Dim()
objects.
\index{\_\_str\_\_() (TotalDepth.util.plot.Coord.Margin method)}

\begin{fulllineitems}
\phantomsection\label{\detokenize{ref/util/plot/Coord:TotalDepth.util.plot.Coord.Margin.__str__}}\pysiglinewithargsret{\sphinxbfcode{\_\_str\_\_}}{}{}
Stringifying.

\end{fulllineitems}


\end{fulllineitems}

\index{Pt (class in TotalDepth.util.plot.Coord)}

\begin{fulllineitems}
\phantomsection\label{\detokenize{ref/util/plot/Coord:TotalDepth.util.plot.Coord.Pt}}\pysigline{\sphinxbfcode{class }\sphinxcode{TotalDepth.util.plot.Coord.}\sphinxbfcode{Pt}}
A point, an absolute x/y position on the plot area.
Members are Coord.Dim().
\index{\_\_eq\_\_() (TotalDepth.util.plot.Coord.Pt method)}

\begin{fulllineitems}
\phantomsection\label{\detokenize{ref/util/plot/Coord:TotalDepth.util.plot.Coord.Pt.__eq__}}\pysiglinewithargsret{\sphinxbfcode{\_\_eq\_\_}}{\emph{other}}{}
Comparison.

\end{fulllineitems}

\index{\_\_str\_\_() (TotalDepth.util.plot.Coord.Pt method)}

\begin{fulllineitems}
\phantomsection\label{\detokenize{ref/util/plot/Coord:TotalDepth.util.plot.Coord.Pt.__str__}}\pysiglinewithargsret{\sphinxbfcode{\_\_str\_\_}}{}{}
Stringifying.

\end{fulllineitems}

\index{convert() (TotalDepth.util.plot.Coord.Pt method)}

\begin{fulllineitems}
\phantomsection\label{\detokenize{ref/util/plot/Coord:TotalDepth.util.plot.Coord.Pt.convert}}\pysiglinewithargsret{\sphinxbfcode{convert}}{\emph{u}}{}
Returns a new Pt() with units changed and value converted.

\end{fulllineitems}

\index{scale() (TotalDepth.util.plot.Coord.Pt method)}

\begin{fulllineitems}
\phantomsection\label{\detokenize{ref/util/plot/Coord:TotalDepth.util.plot.Coord.Pt.scale}}\pysiglinewithargsret{\sphinxbfcode{scale}}{\emph{factor}}{}
Returns a new Pt() scaled by a factor, units are unchanged.

\end{fulllineitems}


\end{fulllineitems}

\index{baseUnitsDim() (in module TotalDepth.util.plot.Coord)}

\begin{fulllineitems}
\phantomsection\label{\detokenize{ref/util/plot/Coord:TotalDepth.util.plot.Coord.baseUnitsDim}}\pysiglinewithargsret{\sphinxcode{TotalDepth.util.plot.Coord.}\sphinxbfcode{baseUnitsDim}}{\emph{theLen}}{}
Returns a Coord.Dim() of length and units BASE\_UNITS.

\end{fulllineitems}

\index{zeroBaseUnitsDim() (in module TotalDepth.util.plot.Coord)}

\begin{fulllineitems}
\phantomsection\label{\detokenize{ref/util/plot/Coord:TotalDepth.util.plot.Coord.zeroBaseUnitsDim}}\pysiglinewithargsret{\sphinxcode{TotalDepth.util.plot.Coord.}\sphinxbfcode{zeroBaseUnitsDim}}{}{}
Returns a Coord.Dim() of zero length and units BASE\_UNITS.

\end{fulllineitems}

\index{zeroBaseUnitsBox() (in module TotalDepth.util.plot.Coord)}

\begin{fulllineitems}
\phantomsection\label{\detokenize{ref/util/plot/Coord:TotalDepth.util.plot.Coord.zeroBaseUnitsBox}}\pysiglinewithargsret{\sphinxcode{TotalDepth.util.plot.Coord.}\sphinxbfcode{zeroBaseUnitsBox}}{}{}
Returns a Coord.Box() of zero dimensions and units BASE\_UNITS.

\end{fulllineitems}

\index{zeroBaseUnitsPad() (in module TotalDepth.util.plot.Coord)}

\begin{fulllineitems}
\phantomsection\label{\detokenize{ref/util/plot/Coord:TotalDepth.util.plot.Coord.zeroBaseUnitsPad}}\pysiglinewithargsret{\sphinxcode{TotalDepth.util.plot.Coord.}\sphinxbfcode{zeroBaseUnitsPad}}{}{}
Returns a Coord.Pad() of zero dimensions and units BASE\_UNITS.

\end{fulllineitems}

\index{zeroBaseUnitsPt() (in module TotalDepth.util.plot.Coord)}

\begin{fulllineitems}
\phantomsection\label{\detokenize{ref/util/plot/Coord:TotalDepth.util.plot.Coord.zeroBaseUnitsPt}}\pysiglinewithargsret{\sphinxcode{TotalDepth.util.plot.Coord.}\sphinxbfcode{zeroBaseUnitsPt}}{}{}
Returns a Coord.Dim() of zero length and units BASE\_UNITS.

\end{fulllineitems}

\index{newPt() (in module TotalDepth.util.plot.Coord)}

\begin{fulllineitems}
\phantomsection\label{\detokenize{ref/util/plot/Coord:TotalDepth.util.plot.Coord.newPt}}\pysiglinewithargsret{\sphinxcode{TotalDepth.util.plot.Coord.}\sphinxbfcode{newPt}}{\emph{theP}, \emph{incX=None}, \emph{incY=None}}{}
Returns a new Pt object by incrementing existing point incX, incY
that are both Dim() objects or None.

\end{fulllineitems}

\index{convertPt() (in module TotalDepth.util.plot.Coord)}

\begin{fulllineitems}
\phantomsection\label{\detokenize{ref/util/plot/Coord:TotalDepth.util.plot.Coord.convertPt}}\pysiglinewithargsret{\sphinxcode{TotalDepth.util.plot.Coord.}\sphinxbfcode{convertPt}}{\emph{theP}, \emph{theUnits}}{}
Returns a new point with the dimensions of theP converted to theUnits.

\end{fulllineitems}



\subparagraph{Examples}
\label{\detokenize{ref/util/plot/Coord:examples}}

\subparagraph{\sphinxstyleliteralintitle{Coord.Dim()}}
\label{\detokenize{ref/util/plot/Coord:coord-dim}}
Creation, addition and subtraction:

\begin{sphinxVerbatim}[commandchars=\\\{\}]
\PYG{n}{d} \PYG{o}{=} \PYG{n}{Coord}\PYG{o}{.}\PYG{n}{Dim}\PYG{p}{(}\PYG{l+m+mi}{1}\PYG{p}{,} \PYG{l+s+s1}{\PYGZsq{}}\PYG{l+s+s1}{in}\PYG{l+s+s1}{\PYGZsq{}}\PYG{p}{)} \PYG{o}{+} \PYG{n}{Coord}\PYG{o}{.}\PYG{n}{Dim}\PYG{p}{(}\PYG{l+m+mi}{18}\PYG{p}{,} \PYG{l+s+s1}{\PYGZsq{}}\PYG{l+s+s1}{px}\PYG{l+s+s1}{\PYGZsq{}}\PYG{p}{)}
\PYG{c+c1}{\PYGZsh{} d is 1.25 inches}
\PYG{n}{d} \PYG{o}{=} \PYG{n}{Coord}\PYG{o}{.}\PYG{n}{Dim}\PYG{p}{(}\PYG{l+m+mi}{1}\PYG{p}{,} \PYG{l+s+s1}{\PYGZsq{}}\PYG{l+s+s1}{in}\PYG{l+s+s1}{\PYGZsq{}}\PYG{p}{)} \PYG{o}{\PYGZhy{}} \PYG{n}{Coord}\PYG{o}{.}\PYG{n}{Dim}\PYG{p}{(}\PYG{l+m+mi}{18}\PYG{p}{,} \PYG{l+s+s1}{\PYGZsq{}}\PYG{l+s+s1}{px}\PYG{l+s+s1}{\PYGZsq{}}\PYG{p}{)}
\PYG{c+c1}{\PYGZsh{} d is 0.75 inches}
\PYG{n}{d} \PYG{o}{+}\PYG{o}{=} \PYG{n}{Coord}\PYG{o}{.}\PYG{n}{Dim}\PYG{p}{(}\PYG{l+m+mf}{25.4}\PYG{p}{,} \PYG{l+s+s1}{\PYGZsq{}}\PYG{l+s+s1}{mm}\PYG{l+s+s1}{\PYGZsq{}}\PYG{p}{)}
\PYG{c+c1}{\PYGZsh{} d is 1.75 inches}
\end{sphinxVerbatim}

Scaling and unit conversion returns a new object:

\begin{sphinxVerbatim}[commandchars=\\\{\}]
\PYG{n}{a} \PYG{o}{=} \PYG{n}{Coord}\PYG{o}{.}\PYG{n}{Dim}\PYG{p}{(}\PYG{l+m+mi}{12}\PYG{p}{,} \PYG{l+s+s1}{\PYGZsq{}}\PYG{l+s+s1}{px}\PYG{l+s+s1}{\PYGZsq{}}\PYG{p}{)}
\PYG{n}{b} \PYG{o}{=} \PYG{n}{myObj}\PYG{o}{.}\PYG{n}{scale}\PYG{p}{(}\PYG{l+m+mf}{6.0}\PYG{p}{)}
\PYG{c+c1}{\PYGZsh{} b is 72 pixels}
\PYG{n}{c} \PYG{o}{=} \PYG{n}{b}\PYG{o}{.}\PYG{n}{convert}\PYG{p}{(}\PYG{l+s+s1}{\PYGZsq{}}\PYG{l+s+s1}{in}\PYG{l+s+s1}{\PYGZsq{}}\PYG{p}{)}
\PYG{c+c1}{\PYGZsh{} 1 is 1 inch}
\end{sphinxVerbatim}

Comparison:

\begin{sphinxVerbatim}[commandchars=\\\{\}]
\PYG{k}{assert}\PYG{p}{(}\PYG{n}{Coord}\PYG{o}{.}\PYG{n}{Dim}\PYG{p}{(}\PYG{l+m+mi}{1}\PYG{p}{,} \PYG{l+s+s1}{\PYGZsq{}}\PYG{l+s+s1}{in}\PYG{l+s+s1}{\PYGZsq{}}\PYG{p}{)} \PYG{o}{==} \PYG{n}{Coord}\PYG{o}{.}\PYG{n}{Dim}\PYG{p}{(}\PYG{l+m+mi}{72}\PYG{p}{,} \PYG{l+s+s1}{\PYGZsq{}}\PYG{l+s+s1}{px}\PYG{l+s+s1}{\PYGZsq{}}\PYG{p}{)}\PYG{p}{)}
\PYG{k}{assert}\PYG{p}{(}\PYG{n}{Coord}\PYG{o}{.}\PYG{n}{Dim}\PYG{p}{(}\PYG{l+m+mi}{1}\PYG{p}{,} \PYG{l+s+s1}{\PYGZsq{}}\PYG{l+s+s1}{in}\PYG{l+s+s1}{\PYGZsq{}}\PYG{p}{)} \PYG{o}{\PYGZgt{}}\PYG{o}{=} \PYG{n}{Coord}\PYG{o}{.}\PYG{n}{Dim}\PYG{p}{(}\PYG{l+m+mi}{72}\PYG{p}{,} \PYG{l+s+s1}{\PYGZsq{}}\PYG{l+s+s1}{px}\PYG{l+s+s1}{\PYGZsq{}}\PYG{p}{)}\PYG{p}{)}
\PYG{k}{assert}\PYG{p}{(}\PYG{n}{Coord}\PYG{o}{.}\PYG{n}{Dim}\PYG{p}{(}\PYG{l+m+mi}{1}\PYG{p}{,} \PYG{l+s+s1}{\PYGZsq{}}\PYG{l+s+s1}{in}\PYG{l+s+s1}{\PYGZsq{}}\PYG{p}{)} \PYG{o}{\PYGZlt{}}\PYG{o}{=} \PYG{n}{Coord}\PYG{o}{.}\PYG{n}{Dim}\PYG{p}{(}\PYG{l+m+mi}{72}\PYG{p}{,} \PYG{l+s+s1}{\PYGZsq{}}\PYG{l+s+s1}{px}\PYG{l+s+s1}{\PYGZsq{}}\PYG{p}{)}\PYG{p}{)}
\PYG{k}{assert}\PYG{p}{(}\PYG{n}{Coord}\PYG{o}{.}\PYG{n}{Dim}\PYG{p}{(}\PYG{l+m+mi}{1}\PYG{p}{,} \PYG{l+s+s1}{\PYGZsq{}}\PYG{l+s+s1}{in}\PYG{l+s+s1}{\PYGZsq{}}\PYG{p}{)} \PYG{o}{\PYGZgt{}} \PYG{n}{Coord}\PYG{o}{.}\PYG{n}{Dim}\PYG{p}{(}\PYG{l+m+mi}{71}\PYG{p}{,} \PYG{l+s+s1}{\PYGZsq{}}\PYG{l+s+s1}{px}\PYG{l+s+s1}{\PYGZsq{}}\PYG{p}{)}\PYG{p}{)}
\PYG{k}{assert}\PYG{p}{(}\PYG{n}{Coord}\PYG{o}{.}\PYG{n}{Dim}\PYG{p}{(}\PYG{l+m+mi}{1}\PYG{p}{,} \PYG{l+s+s1}{\PYGZsq{}}\PYG{l+s+s1}{in}\PYG{l+s+s1}{\PYGZsq{}}\PYG{p}{)} \PYG{o}{\PYGZlt{}} \PYG{n}{Coord}\PYG{o}{.}\PYG{n}{Dim}\PYG{p}{(}\PYG{l+m+mi}{73}\PYG{p}{,} \PYG{l+s+s1}{\PYGZsq{}}\PYG{l+s+s1}{px}\PYG{l+s+s1}{\PYGZsq{}}\PYG{p}{)}\PYG{p}{)}
\end{sphinxVerbatim}


\subparagraph{\sphinxstyleliteralintitle{Coord.Pt()}}
\label{\detokenize{ref/util/plot/Coord:coord-pt}}
Creation:

\begin{sphinxVerbatim}[commandchars=\\\{\}]
\PYG{n}{p} \PYG{o}{=} \PYG{n}{Coord}\PYG{o}{.}\PYG{n}{Pt}\PYG{p}{(}
        \PYG{n}{Coord}\PYG{o}{.}\PYG{n}{Dim}\PYG{p}{(}\PYG{l+m+mi}{12}\PYG{p}{,} \PYG{l+s+s1}{\PYGZsq{}}\PYG{l+s+s1}{px}\PYG{l+s+s1}{\PYGZsq{}}\PYG{p}{)}\PYG{p}{,}
        \PYG{n}{Coord}\PYG{o}{.}\PYG{n}{Dim}\PYG{p}{(}\PYG{l+m+mi}{24}\PYG{p}{,} \PYG{l+s+s1}{\PYGZsq{}}\PYG{l+s+s1}{px}\PYG{l+s+s1}{\PYGZsq{}}\PYG{p}{)}\PYG{p}{,}
        \PYG{p}{)}
\PYG{n+nb}{print}\PYG{p}{(}\PYG{n}{p}\PYG{p}{)}
\PYG{c+c1}{\PYGZsh{} Prints: \PYGZsq{}Pt(x=Dim(12px), y=Dim(24px))\PYGZsq{}}
\PYG{n}{p}\PYG{o}{.}\PYG{n}{x} \PYG{c+c1}{\PYGZsh{} Coord.Dim(12, \PYGZsq{}px\PYGZsq{}))}
\PYG{n}{p}\PYG{o}{.}\PYG{n}{y} \PYG{c+c1}{\PYGZsh{} Coord.Dim(24, \PYGZsq{}px\PYGZsq{}))}
\PYG{c+c1}{\PYGZsh{} Scale up by 6 and convert units}
\PYG{n}{pIn} \PYG{o}{=} \PYG{n}{p}\PYG{o}{.}\PYG{n}{scale}\PYG{p}{(}\PYG{l+m+mi}{6}\PYG{p}{)}\PYG{o}{.}\PYG{n}{convert}\PYG{p}{(}\PYG{l+s+s1}{\PYGZsq{}}\PYG{l+s+s1}{in}\PYG{l+s+s1}{\PYGZsq{}}\PYG{p}{)}
\PYG{c+c1}{\PYGZsh{} pIn now \PYGZsq{}Pt(x=Dim(1in), y=Dim(2in))\PYGZsq{}}
\end{sphinxVerbatim}


\subparagraph{Testing}
\label{\detokenize{ref/util/plot/Coord:testing}}
The unit tests are in \sphinxcode{test/TestCoord.py}.


\paragraph{Plot Pen Stroke}
\label{\detokenize{ref/util/plot/Stroke:plot-pen-stroke}}\label{\detokenize{ref/util/plot/Stroke:module-TotalDepth.util.plot.Stroke}}\label{\detokenize{ref/util/plot/Stroke::doc}}\index{TotalDepth.util.plot.Stroke (module)}
Defines how the stroke of a pen is represented on a plot.

Created on 7 Mar 2011
\index{Stroke (class in TotalDepth.util.plot.Stroke)}

\begin{fulllineitems}
\phantomsection\label{\detokenize{ref/util/plot/Stroke:TotalDepth.util.plot.Stroke.Stroke}}\pysiglinewithargsret{\sphinxbfcode{class }\sphinxcode{TotalDepth.util.plot.Stroke.}\sphinxbfcode{Stroke}}{\emph{width}, \emph{colour}, \emph{coding}, \emph{opacity}}{}
Stroke basic properties
\index{coding (TotalDepth.util.plot.Stroke.Stroke attribute)}

\begin{fulllineitems}
\phantomsection\label{\detokenize{ref/util/plot/Stroke:TotalDepth.util.plot.Stroke.Stroke.coding}}\pysigline{\sphinxbfcode{coding}}
Alias for field number 2

\end{fulllineitems}

\index{colour (TotalDepth.util.plot.Stroke.Stroke attribute)}

\begin{fulllineitems}
\phantomsection\label{\detokenize{ref/util/plot/Stroke:TotalDepth.util.plot.Stroke.Stroke.colour}}\pysigline{\sphinxbfcode{colour}}
Alias for field number 1

\end{fulllineitems}

\index{opacity (TotalDepth.util.plot.Stroke.Stroke attribute)}

\begin{fulllineitems}
\phantomsection\label{\detokenize{ref/util/plot/Stroke:TotalDepth.util.plot.Stroke.Stroke.opacity}}\pysigline{\sphinxbfcode{opacity}}
Alias for field number 3

\end{fulllineitems}

\index{width (TotalDepth.util.plot.Stroke.Stroke attribute)}

\begin{fulllineitems}
\phantomsection\label{\detokenize{ref/util/plot/Stroke:TotalDepth.util.plot.Stroke.Stroke.width}}\pysigline{\sphinxbfcode{width}}
Alias for field number 0

\end{fulllineitems}


\end{fulllineitems}

\index{StrokeBlackSolid (in module TotalDepth.util.plot.Stroke)}

\begin{fulllineitems}
\phantomsection\label{\detokenize{ref/util/plot/Stroke:TotalDepth.util.plot.Stroke.StrokeBlackSolid}}\pysigline{\sphinxcode{TotalDepth.util.plot.Stroke.}\sphinxbfcode{StrokeBlackSolid}\sphinxbfcode{ = Stroke(width='1', colour='black', coding=None, opacity=1.0)}}
Prototypical black solid stroke
Usage: StrokeBlackSolid.\_replace(width=2.0)

\end{fulllineitems}

\index{retSVGAttrsFromStroke() (in module TotalDepth.util.plot.Stroke)}

\begin{fulllineitems}
\phantomsection\label{\detokenize{ref/util/plot/Stroke:TotalDepth.util.plot.Stroke.retSVGAttrsFromStroke}}\pysiglinewithargsret{\sphinxcode{TotalDepth.util.plot.Stroke.}\sphinxbfcode{retSVGAttrsFromStroke}}{\emph{stroke}}{}
Returns SVG attributes as a dictionary from a Stroke() object.

\end{fulllineitems}



\paragraph{SVG Writer Module}
\label{\detokenize{ref/util/plot/SVGWriter:module-TotalDepth.util.plot.SVGWriter}}\label{\detokenize{ref/util/plot/SVGWriter:svg-writer-module}}\label{\detokenize{ref/util/plot/SVGWriter::doc}}\index{TotalDepth.util.plot.SVGWriter (module)}
An SVG writer.

TODO: Add a float format to reduce the size of the SVG file.
\index{DEFAULT\_VALUE\_FORMAT (in module TotalDepth.util.plot.SVGWriter)}

\begin{fulllineitems}
\phantomsection\label{\detokenize{ref/util/plot/SVGWriter:TotalDepth.util.plot.SVGWriter.DEFAULT_VALUE_FORMAT}}\pysigline{\sphinxcode{TotalDepth.util.plot.SVGWriter.}\sphinxbfcode{DEFAULT\_VALUE\_FORMAT}\sphinxbfcode{ = '\{:.3f\}'}}
Defaults format for points that are specified in inches or such like

\end{fulllineitems}

\index{DEFAULT\_VALUE\_FORMAT\_POINTS (in module TotalDepth.util.plot.SVGWriter)}

\begin{fulllineitems}
\phantomsection\label{\detokenize{ref/util/plot/SVGWriter:TotalDepth.util.plot.SVGWriter.DEFAULT_VALUE_FORMAT_POINTS}}\pysigline{\sphinxcode{TotalDepth.util.plot.SVGWriter.}\sphinxbfcode{DEFAULT\_VALUE\_FORMAT\_POINTS}\sphinxbfcode{ = '\{:.1f\}'}}
Defaults format for points that are specified in pixels

\end{fulllineitems}

\index{ExceptionSVGWriter}

\begin{fulllineitems}
\phantomsection\label{\detokenize{ref/util/plot/SVGWriter:TotalDepth.util.plot.SVGWriter.ExceptionSVGWriter}}\pysigline{\sphinxbfcode{exception }\sphinxcode{TotalDepth.util.plot.SVGWriter.}\sphinxbfcode{ExceptionSVGWriter}}
Exception class for SVGWriter.

\end{fulllineitems}

\index{SVGCircle (class in TotalDepth.util.plot.SVGWriter)}

\begin{fulllineitems}
\phantomsection\label{\detokenize{ref/util/plot/SVGWriter:TotalDepth.util.plot.SVGWriter.SVGCircle}}\pysiglinewithargsret{\sphinxbfcode{class }\sphinxcode{TotalDepth.util.plot.SVGWriter.}\sphinxbfcode{SVGCircle}}{\emph{theXmlStream}, \emph{thePoint}, \emph{theRadius}, \emph{attrs=None}}{}
A circle in SVG.Initialise the circle with a stream, a Coord.Pt() and a
Coord.Dim() objects.

See: \sphinxurl{http://www.w3.org/TR/2003/REC-SVG11-20030114/shapes.html\#CircleElement}
\index{\_\_init\_\_() (TotalDepth.util.plot.SVGWriter.SVGCircle method)}

\begin{fulllineitems}
\phantomsection\label{\detokenize{ref/util/plot/SVGWriter:TotalDepth.util.plot.SVGWriter.SVGCircle.__init__}}\pysiglinewithargsret{\sphinxbfcode{\_\_init\_\_}}{\emph{theXmlStream}, \emph{thePoint}, \emph{theRadius}, \emph{attrs=None}}{}
Initialise the circle with a stream, a Coord.Pt() and a
Coord.Dim() objects.

\end{fulllineitems}


\end{fulllineitems}

\index{SVGElipse (class in TotalDepth.util.plot.SVGWriter)}

\begin{fulllineitems}
\phantomsection\label{\detokenize{ref/util/plot/SVGWriter:TotalDepth.util.plot.SVGWriter.SVGElipse}}\pysiglinewithargsret{\sphinxbfcode{class }\sphinxcode{TotalDepth.util.plot.SVGWriter.}\sphinxbfcode{SVGElipse}}{\emph{theXmlStream}, \emph{ptFrom}, \emph{theRadX}, \emph{theRadY}, \emph{attrs=None}}{}
An elipse in SVG.Initialise the circle with a stream, a Coord.Pt() and a
Coord.Dim() objects.

See: \sphinxurl{http://www.w3.org/TR/2003/REC-SVG11-20030114/shapes.html\#EllipseElement}
\index{\_\_init\_\_() (TotalDepth.util.plot.SVGWriter.SVGElipse method)}

\begin{fulllineitems}
\phantomsection\label{\detokenize{ref/util/plot/SVGWriter:TotalDepth.util.plot.SVGWriter.SVGElipse.__init__}}\pysiglinewithargsret{\sphinxbfcode{\_\_init\_\_}}{\emph{theXmlStream}, \emph{ptFrom}, \emph{theRadX}, \emph{theRadY}, \emph{attrs=None}}{}
Initialise the circle with a stream, a Coord.Pt() and a
Coord.Dim() objects.

\end{fulllineitems}


\end{fulllineitems}

\index{SVGGroup (class in TotalDepth.util.plot.SVGWriter)}

\begin{fulllineitems}
\phantomsection\label{\detokenize{ref/util/plot/SVGWriter:TotalDepth.util.plot.SVGWriter.SVGGroup}}\pysiglinewithargsret{\sphinxbfcode{class }\sphinxcode{TotalDepth.util.plot.SVGWriter.}\sphinxbfcode{SVGGroup}}{\emph{theXmlStream}, \emph{attrs=None}}{}
A group element in SVG.
\index{\_\_init\_\_() (TotalDepth.util.plot.SVGWriter.SVGGroup method)}

\begin{fulllineitems}
\phantomsection\label{\detokenize{ref/util/plot/SVGWriter:TotalDepth.util.plot.SVGWriter.SVGGroup.__init__}}\pysiglinewithargsret{\sphinxbfcode{\_\_init\_\_}}{\emph{theXmlStream}, \emph{attrs=None}}{}
Initialise the group with a stream.

See: \sphinxurl{http://www.w3.org/TR/2003/REC-SVG11-20030114/struct.html\#GElement}

Sadly we can’t use **kwargs because of Python restrictions on keyword
names. \sphinxcode{stroke-width} is not a valid keyword
argument (although \sphinxcode{stroke\_width} would be). So instead we pass in an
optional dictionary \{string : string, …\}

\end{fulllineitems}


\end{fulllineitems}

\index{SVGLine (class in TotalDepth.util.plot.SVGWriter)}

\begin{fulllineitems}
\phantomsection\label{\detokenize{ref/util/plot/SVGWriter:TotalDepth.util.plot.SVGWriter.SVGLine}}\pysiglinewithargsret{\sphinxbfcode{class }\sphinxcode{TotalDepth.util.plot.SVGWriter.}\sphinxbfcode{SVGLine}}{\emph{theXmlStream}, \emph{ptFrom}, \emph{ptTo}, \emph{attrs=None}}{}
A rectangle in SVG. Initialise the line with a stream, and two Coord.Pt() objects.

See: \sphinxurl{http://www.w3.org/TR/2003/REC-SVG11-20030114/shapes.html\#LineElement}
\index{\_\_init\_\_() (TotalDepth.util.plot.SVGWriter.SVGLine method)}

\begin{fulllineitems}
\phantomsection\label{\detokenize{ref/util/plot/SVGWriter:TotalDepth.util.plot.SVGWriter.SVGLine.__init__}}\pysiglinewithargsret{\sphinxbfcode{\_\_init\_\_}}{\emph{theXmlStream}, \emph{ptFrom}, \emph{ptTo}, \emph{attrs=None}}{}
Initialise the line with a stream, and two Coord.Pt() objects.

\end{fulllineitems}


\end{fulllineitems}

\index{SVGPointList (class in TotalDepth.util.plot.SVGWriter)}

\begin{fulllineitems}
\phantomsection\label{\detokenize{ref/util/plot/SVGWriter:TotalDepth.util.plot.SVGWriter.SVGPointList}}\pysiglinewithargsret{\sphinxbfcode{class }\sphinxcode{TotalDepth.util.plot.SVGWriter.}\sphinxbfcode{SVGPointList}}{\emph{theXmlStream}, \emph{name}, \emph{pointS}, \emph{attrs}}{}
An abstract class that takes a list of points, derived by polyline and polygon.

Initialise the element with a stream, a name, and a list of Coord.Pt() objects.

NOTE: The units of the points are ignored, it is up to the caller to convert
them to the User Coordinate System.

TODO: Make the caller supply points as numbers not Coord.Pt as this may be faster??? e.g.:

\begin{sphinxVerbatim}[commandchars=\\\{\}]
\PYG{n}{FMT\PYGZus{}PAIR\PYGZus{}STR} \PYG{o}{=} \PYG{n}{FMT\PYGZus{}STR} \PYG{o}{+} \PYG{l+s+s1}{\PYGZsq{}}\PYG{l+s+s1}{,}\PYG{l+s+s1}{\PYGZsq{}} \PYG{o}{+} \PYG{n}{FMT\PYGZus{}STR}
\PYG{l+s+s1}{\PYGZsq{}}\PYG{l+s+s1}{ }\PYG{l+s+s1}{\PYGZsq{}}\PYG{o}{.}\PYG{n}{join}\PYG{p}{(}\PYG{p}{[}\PYG{n+nb+bp}{self}\PYG{o}{.}\PYG{n}{FMT\PYGZus{}PAIR\PYGZus{}STR}\PYG{o}{.}\PYG{n}{format}\PYG{p}{(}\PYG{n}{x}\PYG{p}{,} \PYG{n}{y}\PYG{p}{)} \PYG{k}{for} \PYG{n}{x}\PYG{p}{,}\PYG{n}{y} \PYG{o+ow}{in} \PYG{n}{pointS}\PYG{p}{]}\PYG{p}{)}
\end{sphinxVerbatim}
\index{\_\_init\_\_() (TotalDepth.util.plot.SVGWriter.SVGPointList method)}

\begin{fulllineitems}
\phantomsection\label{\detokenize{ref/util/plot/SVGWriter:TotalDepth.util.plot.SVGWriter.SVGPointList.__init__}}\pysiglinewithargsret{\sphinxbfcode{\_\_init\_\_}}{\emph{theXmlStream}, \emph{name}, \emph{pointS}, \emph{attrs}}{}
Initialise the element with a stream, a name, and a list of Coord.Pt() objects.
NOTE: The units of the points are ignored, it is up to the caller to convert
them to the User Coordinate System.
TODO: Make the caller supply points as numbers not Coord.Pt as this may be faster???
e.g.
FMT\_PAIR\_STR = FMT\_STR + ‘,’ + FMT\_STR
‘ ‘.join({[}self.FMT\_PAIR\_STR.format(x, y) for x,y in pointS{]})

\end{fulllineitems}


\end{fulllineitems}

\index{SVGPolygon (class in TotalDepth.util.plot.SVGWriter)}

\begin{fulllineitems}
\phantomsection\label{\detokenize{ref/util/plot/SVGWriter:TotalDepth.util.plot.SVGWriter.SVGPolygon}}\pysiglinewithargsret{\sphinxbfcode{class }\sphinxcode{TotalDepth.util.plot.SVGWriter.}\sphinxbfcode{SVGPolygon}}{\emph{theXmlStream}, \emph{pointS}, \emph{attrs=None}}{}
A polygon in SVG. Initialise the polygon with a stream, and a list of Coord.Pt() objects.

NOTE: The units of the points are ignored, it is up to the caller to convert
them to the User Coordinate System.

See: \sphinxurl{http://www.w3.org/TR/2003/REC-SVG11-20030114/shapes.html\#PolygonElement}
\index{\_\_init\_\_() (TotalDepth.util.plot.SVGWriter.SVGPolygon method)}

\begin{fulllineitems}
\phantomsection\label{\detokenize{ref/util/plot/SVGWriter:TotalDepth.util.plot.SVGWriter.SVGPolygon.__init__}}\pysiglinewithargsret{\sphinxbfcode{\_\_init\_\_}}{\emph{theXmlStream}, \emph{pointS}, \emph{attrs=None}}{}
Initialise the polygon with a stream, and a list of Coord.Pt() objects.
NOTE: The units of the points are ignored, it is up to the caller to convert
them to the User Coordinate System.

\end{fulllineitems}


\end{fulllineitems}

\index{SVGPolyline (class in TotalDepth.util.plot.SVGWriter)}

\begin{fulllineitems}
\phantomsection\label{\detokenize{ref/util/plot/SVGWriter:TotalDepth.util.plot.SVGWriter.SVGPolyline}}\pysiglinewithargsret{\sphinxbfcode{class }\sphinxcode{TotalDepth.util.plot.SVGWriter.}\sphinxbfcode{SVGPolyline}}{\emph{theXmlStream}, \emph{pointS}, \emph{attrs=None}}{}
A polyline in SVG. Initialise the polyline with a stream, and a list of Coord.Pt() objects.

NOTE: The units of the points are ignored, it is up to the caller to convert
them to the User Coordinate System.

See: \sphinxurl{http://www.w3.org/TR/2003/REC-SVG11-20030114/shapes.html\#PolylineElement}
\index{\_\_init\_\_() (TotalDepth.util.plot.SVGWriter.SVGPolyline method)}

\begin{fulllineitems}
\phantomsection\label{\detokenize{ref/util/plot/SVGWriter:TotalDepth.util.plot.SVGWriter.SVGPolyline.__init__}}\pysiglinewithargsret{\sphinxbfcode{\_\_init\_\_}}{\emph{theXmlStream}, \emph{pointS}, \emph{attrs=None}}{}
Initialise the polyline with a stream, and a list of Coord.Pt() objects.
NOTE: The units of the points are ignored, it is up to the caller to convert
them to the User Coordinate System.

\end{fulllineitems}


\end{fulllineitems}

\index{SVGRect (class in TotalDepth.util.plot.SVGWriter)}

\begin{fulllineitems}
\phantomsection\label{\detokenize{ref/util/plot/SVGWriter:TotalDepth.util.plot.SVGWriter.SVGRect}}\pysiglinewithargsret{\sphinxbfcode{class }\sphinxcode{TotalDepth.util.plot.SVGWriter.}\sphinxbfcode{SVGRect}}{\emph{theXmlStream}, \emph{thePoint}, \emph{theBox}, \emph{attrs=None}}{}
A rectangle in SVG.
Initialise the rectangle with a stream, a Coord.Pt() and a
Coord.Box() objects.

See: \sphinxurl{http://www.w3.org/TR/2003/REC-SVG11-20030114/shapes.html\#RectElement}

Typical attributes:

\begin{sphinxVerbatim}[commandchars=\\\{\}]
\PYG{p}{\PYGZob{}}\PYG{l+s+s1}{\PYGZsq{}}\PYG{l+s+s1}{fill}\PYG{l+s+s1}{\PYGZsq{}} \PYG{p}{:} \PYG{l+s+s2}{\PYGZdq{}}\PYG{l+s+s2}{blue}\PYG{l+s+s2}{\PYGZdq{}}\PYG{p}{,} \PYG{l+s+s1}{\PYGZsq{}}\PYG{l+s+s1}{stroke}\PYG{l+s+s1}{\PYGZsq{}} \PYG{p}{:} \PYG{l+s+s2}{\PYGZdq{}}\PYG{l+s+s2}{black}\PYG{l+s+s2}{\PYGZdq{}}\PYG{p}{,} \PYG{l+s+s1}{\PYGZsq{}}\PYG{l+s+s1}{stroke\PYGZhy{}width}\PYG{l+s+s1}{\PYGZsq{}} \PYG{p}{:} \PYG{l+s+s2}{\PYGZdq{}}\PYG{l+s+s2}{2}\PYG{l+s+s2}{\PYGZdq{}}\PYG{p}{,} \PYG{p}{\PYGZcb{}}
\end{sphinxVerbatim}
\index{\_\_init\_\_() (TotalDepth.util.plot.SVGWriter.SVGRect method)}

\begin{fulllineitems}
\phantomsection\label{\detokenize{ref/util/plot/SVGWriter:TotalDepth.util.plot.SVGWriter.SVGRect.__init__}}\pysiglinewithargsret{\sphinxbfcode{\_\_init\_\_}}{\emph{theXmlStream}, \emph{thePoint}, \emph{theBox}, \emph{attrs=None}}{}
Initialise the rectangle with a stream, a Coord.Pt() and a
Coord.Box() objects.

See: \sphinxurl{http://www.w3.org/TR/2003/REC-SVG11-20030114/shapes.html\#RectElement}

Typical attributes:

\begin{sphinxVerbatim}[commandchars=\\\{\}]
\PYG{p}{\PYGZob{}}\PYG{l+s+s1}{\PYGZsq{}}\PYG{l+s+s1}{fill}\PYG{l+s+s1}{\PYGZsq{}} \PYG{p}{:} \PYG{l+s+s2}{\PYGZdq{}}\PYG{l+s+s2}{blue}\PYG{l+s+s2}{\PYGZdq{}}\PYG{p}{,} \PYG{l+s+s1}{\PYGZsq{}}\PYG{l+s+s1}{stroke}\PYG{l+s+s1}{\PYGZsq{}} \PYG{p}{:} \PYG{l+s+s2}{\PYGZdq{}}\PYG{l+s+s2}{black}\PYG{l+s+s2}{\PYGZdq{}}\PYG{p}{,} \PYG{l+s+s1}{\PYGZsq{}}\PYG{l+s+s1}{stroke\PYGZhy{}width}\PYG{l+s+s1}{\PYGZsq{}} \PYG{p}{:} \PYG{l+s+s2}{\PYGZdq{}}\PYG{l+s+s2}{2}\PYG{l+s+s2}{\PYGZdq{}}\PYG{p}{,} \PYG{p}{\PYGZcb{}}
\end{sphinxVerbatim}

\end{fulllineitems}


\end{fulllineitems}

\index{SVGText (class in TotalDepth.util.plot.SVGWriter)}

\begin{fulllineitems}
\phantomsection\label{\detokenize{ref/util/plot/SVGWriter:TotalDepth.util.plot.SVGWriter.SVGText}}\pysiglinewithargsret{\sphinxbfcode{class }\sphinxcode{TotalDepth.util.plot.SVGWriter.}\sphinxbfcode{SVGText}}{\emph{theXmlStream}, \emph{thePoint}, \emph{theFont}, \emph{theSize}, \emph{attrs=None}}{}
Text in SVG. Initialise the text with a stream, a Coord.Pt() and font 
as a string and size as an integer.
If thePoint is None then no location will be specified (for example
for use inside a \sphinxcode{\textless{}defs\textgreater{}} element).

See: \sphinxurl{http://www.w3.org/TR/2003/REC-SVG11-20030114/text.html\#TextElement}
\index{\_\_init\_\_() (TotalDepth.util.plot.SVGWriter.SVGText method)}

\begin{fulllineitems}
\phantomsection\label{\detokenize{ref/util/plot/SVGWriter:TotalDepth.util.plot.SVGWriter.SVGText.__init__}}\pysiglinewithargsret{\sphinxbfcode{\_\_init\_\_}}{\emph{theXmlStream}, \emph{thePoint}, \emph{theFont}, \emph{theSize}, \emph{attrs=None}}{}
Initialise the text with a stream, a Coord.Pt() and font 
as a string and size as an integer.
If thePoint is None then no location will be specified (for example
for use inside a \textless{}defs\textgreater{} element.

\end{fulllineitems}


\end{fulllineitems}

\index{dimToTxt() (in module TotalDepth.util.plot.SVGWriter)}

\begin{fulllineitems}
\phantomsection\label{\detokenize{ref/util/plot/SVGWriter:TotalDepth.util.plot.SVGWriter.dimToTxt}}\pysiglinewithargsret{\sphinxcode{TotalDepth.util.plot.SVGWriter.}\sphinxbfcode{dimToTxt}}{\emph{theDim}}{}
Converts a Coord.Dim() object to text for SVG units.

\end{fulllineitems}



\subparagraph{Examples}
\label{\detokenize{ref/util/plot/SVGWriter:examples}}
All these examples assume these imports:

\begin{sphinxVerbatim}[commandchars=\\\{\}]
\PYG{k+kn}{import} \PYG{n+nn}{io}
\PYG{k+kn}{from} \PYG{n+nn}{TotalDepth}\PYG{n+nn}{.}\PYG{n+nn}{util} \PYG{k}{import} \PYG{n}{XmlWrite}
\PYG{k+kn}{from} \PYG{n+nn}{TotalDepth}\PYG{n+nn}{.}\PYG{n+nn}{util}\PYG{n+nn}{.}\PYG{n+nn}{plot} \PYG{k}{import} \PYG{n}{SVGWriter}
\PYG{k+kn}{from} \PYG{n+nn}{TotalDepth}\PYG{n+nn}{.}\PYG{n+nn}{util}\PYG{n+nn}{.}\PYG{n+nn}{plot} \PYG{k}{import} \PYG{n}{Coord}
\end{sphinxVerbatim}


\subparagraph{Construction}
\label{\detokenize{ref/util/plot/SVGWriter:construction}}
Writing to an in-memory file:

\begin{sphinxVerbatim}[commandchars=\\\{\}]
f = io.StringIO()
vp = Coord.Box(
    Coord.Dim(100, \PYGZsq{}mm\PYGZsq{}),
    Coord.Dim(20, \PYGZsq{}mm\PYGZsq{}),
)
with SVGWriter.SVGWriter(myF, myViewPort):
    pass
print(myF.getvalue())
\PYGZsh{} Prints:
\PYGZlt{}?xml version=\PYGZsq{}1.0\PYGZsq{} encoding=\PYGZdq{}utf\PYGZhy{}8\PYGZdq{}?\PYGZgt{}
\PYGZlt{}!DOCTYPE svg PUBLIC \PYGZdq{}\PYGZhy{}//W3C//DTD SVG 1.1//EN\PYGZdq{} \PYGZdq{}http://www.w3.org/Graphics/SVG/1.1/DTD/svg11.dtd\PYGZdq{}\PYGZgt{}
\PYGZlt{}svg height=\PYGZdq{}20.000mm\PYGZdq{} version=\PYGZdq{}1.1\PYGZdq{} width=\PYGZdq{}100.000mm\PYGZdq{} xmlns=\PYGZdq{}http://www.w3.org/2000/svg\PYGZdq{}/\PYGZgt{}
\end{sphinxVerbatim}


\subparagraph{Writing Objects to SVG}
\label{\detokenize{ref/util/plot/SVGWriter:writing-objects-to-svg}}
Writing a rectangles to a stream:

\begin{sphinxVerbatim}[commandchars=\\\{\}]
myF = io.StringIO()
vp = Coord.Box(Coord.Dim(5, \PYGZsq{}cm\PYGZsq{}), Coord.Dim(4, \PYGZsq{}cm\PYGZsq{}))
with SVGWriter.SVGWriter(myF, vp) as xS:
    with XmlWrite.Element(xS, \PYGZsq{}desc\PYGZsq{}):
        xS.characters(\PYGZsq{}A couple of rectangles\PYGZsq{})
    myPt = Coord.Pt(Coord.Dim(0.5, \PYGZsq{}cm\PYGZsq{}), Coord.Dim(0.5, \PYGZsq{}cm\PYGZsq{}))
    myBx = Coord.Box(Coord.Dim(2.0, \PYGZsq{}cm\PYGZsq{}), Coord.Dim(1.0, \PYGZsq{}cm\PYGZsq{}))
    with SVGWriter.SVGRect(xS, myPt, myBx):
        pass
    myPt = Coord.Pt(Coord.Dim(0.01, \PYGZsq{}cm\PYGZsq{}), Coord.Dim(0.01, \PYGZsq{}cm\PYGZsq{}))
    myBx = Coord.Box(Coord.Dim(4.98, \PYGZsq{}cm\PYGZsq{}), Coord.Dim(3.98, \PYGZsq{}cm\PYGZsq{}))
    with SVGWriter.SVGRect(xS, myPt, myBx, attrs= \PYGZob{}\PYGZsq{}fill\PYGZsq{} : \PYGZdq{}none\PYGZdq{}, \PYGZsq{}stroke\PYGZsq{} : \PYGZdq{}blue\PYGZdq{}, \PYGZsq{}stroke\PYGZhy{}width\PYGZsq{} : \PYGZdq{}.02cm\PYGZdq{}\PYGZcb{}):
        pass
print(myF.getvalue())
\PYGZsh{} Prints:
\PYGZlt{}?xml version=\PYGZsq{}1.0\PYGZsq{} encoding=\PYGZdq{}utf\PYGZhy{}8\PYGZdq{}?\PYGZgt{}
\PYGZlt{}!DOCTYPE svg PUBLIC \PYGZdq{}\PYGZhy{}//W3C//DTD SVG 1.1//EN\PYGZdq{} \PYGZdq{}http://www.w3.org/Graphics/SVG/1.1/DTD/svg11.dtd\PYGZdq{}\PYGZgt{}
\PYGZlt{}svg height=\PYGZdq{}4.000cm\PYGZdq{} version=\PYGZdq{}1.1\PYGZdq{} width=\PYGZdq{}5.000cm\PYGZdq{} xmlns=\PYGZdq{}http://www.w3.org/2000/svg\PYGZdq{}\PYGZgt{}
    \PYGZlt{}desc\PYGZgt{}A couple of rectangles\PYGZlt{}/desc\PYGZgt{}
    \PYGZlt{}rect height=\PYGZdq{}1.000cm\PYGZdq{} width=\PYGZdq{}2.000cm\PYGZdq{} x=\PYGZdq{}0.500cm\PYGZdq{} y=\PYGZdq{}0.500cm\PYGZdq{}/\PYGZgt{}
    \PYGZlt{}rect fill=\PYGZdq{}none\PYGZdq{} height=\PYGZdq{}3.980cm\PYGZdq{} stroke=\PYGZdq{}blue\PYGZdq{} stroke\PYGZhy{}width=\PYGZdq{}.02cm\PYGZdq{} width=\PYGZdq{}4.980cm\PYGZdq{} x=\PYGZdq{}0.010cm\PYGZdq{} y=\PYGZdq{}0.010cm\PYGZdq{}/\PYGZgt{}
\PYGZlt{}/svg\PYGZgt{}
\end{sphinxVerbatim}


\paragraph{Plot Constants for Wireline Logs}
\label{\detokenize{ref/util/plot/PlotConstants:plot-constants-for-wireline-logs}}\label{\detokenize{ref/util/plot/PlotConstants::doc}}\label{\detokenize{ref/util/plot/PlotConstants:module-TotalDepth.util.plot.PlotConstants}}\index{TotalDepth.util.plot.PlotConstants (module)}
Some basic constants used in plotting.

Created on Jan 5, 2012
\index{DEFAULT\_PLOT\_UNITS (in module TotalDepth.util.plot.PlotConstants)}

\begin{fulllineitems}
\phantomsection\label{\detokenize{ref/util/plot/PlotConstants:TotalDepth.util.plot.PlotConstants.DEFAULT_PLOT_UNITS}}\pysigline{\sphinxcode{TotalDepth.util.plot.PlotConstants.}\sphinxbfcode{DEFAULT\_PLOT\_UNITS}\sphinxbfcode{ = 'in'}}
Default SVG plot units

\end{fulllineitems}

\index{DEFAULT\_PLOT\_LIS\_UNITS (in module TotalDepth.util.plot.PlotConstants)}

\begin{fulllineitems}
\phantomsection\label{\detokenize{ref/util/plot/PlotConstants:TotalDepth.util.plot.PlotConstants.DEFAULT_PLOT_LIS_UNITS}}\pysigline{\sphinxcode{TotalDepth.util.plot.PlotConstants.}\sphinxbfcode{DEFAULT\_PLOT\_LIS\_UNITS}\sphinxbfcode{ = b'IN  '}}
Units that we can convert to in LIS terms

\end{fulllineitems}

\index{VIEW\_BOX\_UNITS\_PER\_PLOT\_UNITS (in module TotalDepth.util.plot.PlotConstants)}

\begin{fulllineitems}
\phantomsection\label{\detokenize{ref/util/plot/PlotConstants:TotalDepth.util.plot.PlotConstants.VIEW_BOX_UNITS_PER_PLOT_UNITS}}\pysigline{\sphinxcode{TotalDepth.util.plot.PlotConstants.}\sphinxbfcode{VIEW\_BOX\_UNITS\_PER\_PLOT\_UNITS}\sphinxbfcode{ = 96.0}}
Number of pixels per standard plot unit

\end{fulllineitems}

\index{MarginQtrInch (in module TotalDepth.util.plot.PlotConstants)}

\begin{fulllineitems}
\phantomsection\label{\detokenize{ref/util/plot/PlotConstants:TotalDepth.util.plot.PlotConstants.MarginQtrInch}}\pysigline{\sphinxcode{TotalDepth.util.plot.PlotConstants.}\sphinxbfcode{MarginQtrInch}\sphinxbfcode{ = Margin(left=Dim(value=0.25, units='in'), right=Dim(value=0.25, units='in'), top=Dim(value=0.25, units='in'), bottom=Dim(value=0.25, units='in'))}}
Definition of a quarter inch margin

\end{fulllineitems}

\index{STANDARD\_PAPER\_WIDTH (in module TotalDepth.util.plot.PlotConstants)}

\begin{fulllineitems}
\phantomsection\label{\detokenize{ref/util/plot/PlotConstants:TotalDepth.util.plot.PlotConstants.STANDARD_PAPER_WIDTH}}\pysigline{\sphinxcode{TotalDepth.util.plot.PlotConstants.}\sphinxbfcode{STANDARD\_PAPER\_WIDTH}\sphinxbfcode{ = Dim(value=8.5, units='in')}}
Standard wireline log fan paper width i.e. long side.

\end{fulllineitems}

\index{STANDARD\_PAPER\_DEPTH (in module TotalDepth.util.plot.PlotConstants)}

\begin{fulllineitems}
\phantomsection\label{\detokenize{ref/util/plot/PlotConstants:TotalDepth.util.plot.PlotConstants.STANDARD_PAPER_DEPTH}}\pysigline{\sphinxcode{TotalDepth.util.plot.PlotConstants.}\sphinxbfcode{STANDARD\_PAPER\_DEPTH}\sphinxbfcode{ = Dim(value=6.25, units='in')}}
Standard wireline log fan paper depth i.e. short side.

\end{fulllineitems}

\index{STANDARD\_PAPER\_MARGIN (in module TotalDepth.util.plot.PlotConstants)}

\begin{fulllineitems}
\phantomsection\label{\detokenize{ref/util/plot/PlotConstants:TotalDepth.util.plot.PlotConstants.STANDARD_PAPER_MARGIN}}\pysigline{\sphinxcode{TotalDepth.util.plot.PlotConstants.}\sphinxbfcode{STANDARD\_PAPER\_MARGIN}\sphinxbfcode{ = Margin(left=Dim(value=0.25, units='in'), right=Dim(value=0.25, units='in'), top=Dim(value=0.25, units='in'), bottom=Dim(value=0.25, units='in'))}}
Standard wireline log fan paper margin.

\end{fulllineitems}



\paragraph{Track Module}
\label{\detokenize{ref/util/plot/Track:track-module}}\label{\detokenize{ref/util/plot/Track::doc}}\label{\detokenize{ref/util/plot/Track:module-TotalDepth.util.plot.Track}}\index{TotalDepth.util.plot.Track (module)}
Created on 28 Feb 2011

Plotting LIS data requires these components:
* A PlotConfig object
* A data set (e.g. a LogPass with a FrameSet).
* An output driver (e.g. screen, print PDF, web SVG).

User creates a PlotConfig (this reflects a PRES table). This is reusable.
\begin{description}
\item[{User specifies a data set (from, to, channels etc.).}] \leavevmode
e.g. Invoke LogPass.setFrameSet(File, theFrameSlice=theFrameSlice, theChList=theChList)

\item[{User says ‘plot this data set with this configuration to this output device’.}] \leavevmode
e.g. Plot(PlotConfig, LogPass, PlotDevice)
Plot uses LogPass.frameSet.genChScValues(ch, sc) to plot individual curves.

\end{description}


\subparagraph{Lacunae}
\label{\detokenize{ref/util/plot/Track:lacunae}}
Area plotting.
Caching (e.g. SVG fragments - is this worth it?)


\subparagraph{PlotConfig}
\label{\detokenize{ref/util/plot/Track:plotconfig}}

\subparagraph{PlotTracks}
\label{\detokenize{ref/util/plot/Track:plottracks}}
Typically a three track (+depth) have these dimensions in inches:

\begin{sphinxVerbatim}[commandchars=\\\{\}]
\PYG{n}{Track}    \PYG{n}{Left}    \PYG{n}{Right}    \PYG{n}{Width}
\PYG{l+m+mi}{1}        \PYG{l+m+mi}{0}       \PYG{l+m+mf}{2.4}      \PYG{l+m+mf}{2.4}
\PYG{n}{Depth}    \PYG{l+m+mf}{2.4}     \PYG{l+m+mf}{3.2}      \PYG{l+m+mf}{0.8}
\PYG{l+m+mi}{2}        \PYG{l+m+mf}{3.2}     \PYG{l+m+mf}{5.6}      \PYG{l+m+mf}{2.4}
\PYG{l+m+mi}{3}        \PYG{l+m+mf}{5.6}     \PYG{l+m+mf}{8.0}      \PYG{l+m+mf}{2.4}
\end{sphinxVerbatim}

Track names can be split (e.g. LHT1 is left hand track 1) or merged (T23 is
spread across tracks two and three).

Examples of PRES and FILM records:

\begin{sphinxVerbatim}[commandchars=\\\{\}]
 \PYG{n}{Table} \PYG{n}{record} \PYG{p}{(}\PYG{n+nb}{type} \PYG{l+m+mi}{34}\PYG{p}{)} \PYG{n+nb}{type}\PYG{p}{:} \PYG{n}{PRES}

\PYG{n}{MNEM}  \PYG{n}{OUTP}  \PYG{n}{STAT}  \PYG{n}{TRAC}  \PYG{n}{CODI}  \PYG{n}{DEST}  \PYG{n}{MODE}      \PYG{n}{FILT}          \PYG{n}{LEDG}          \PYG{n}{REDG}
\PYG{o}{\PYGZhy{}}\PYG{o}{\PYGZhy{}}\PYG{o}{\PYGZhy{}}\PYG{o}{\PYGZhy{}}\PYG{o}{\PYGZhy{}}\PYG{o}{\PYGZhy{}}\PYG{o}{\PYGZhy{}}\PYG{o}{\PYGZhy{}}\PYG{o}{\PYGZhy{}}\PYG{o}{\PYGZhy{}}\PYG{o}{\PYGZhy{}}\PYG{o}{\PYGZhy{}}\PYG{o}{\PYGZhy{}}\PYG{o}{\PYGZhy{}}\PYG{o}{\PYGZhy{}}\PYG{o}{\PYGZhy{}}\PYG{o}{\PYGZhy{}}\PYG{o}{\PYGZhy{}}\PYG{o}{\PYGZhy{}}\PYG{o}{\PYGZhy{}}\PYG{o}{\PYGZhy{}}\PYG{o}{\PYGZhy{}}\PYG{o}{\PYGZhy{}}\PYG{o}{\PYGZhy{}}\PYG{o}{\PYGZhy{}}\PYG{o}{\PYGZhy{}}\PYG{o}{\PYGZhy{}}\PYG{o}{\PYGZhy{}}\PYG{o}{\PYGZhy{}}\PYG{o}{\PYGZhy{}}\PYG{o}{\PYGZhy{}}\PYG{o}{\PYGZhy{}}\PYG{o}{\PYGZhy{}}\PYG{o}{\PYGZhy{}}\PYG{o}{\PYGZhy{}}\PYG{o}{\PYGZhy{}}\PYG{o}{\PYGZhy{}}\PYG{o}{\PYGZhy{}}\PYG{o}{\PYGZhy{}}\PYG{o}{\PYGZhy{}}\PYG{o}{\PYGZhy{}}\PYG{o}{\PYGZhy{}}\PYG{o}{\PYGZhy{}}\PYG{o}{\PYGZhy{}}\PYG{o}{\PYGZhy{}}\PYG{o}{\PYGZhy{}}\PYG{o}{\PYGZhy{}}\PYG{o}{\PYGZhy{}}\PYG{o}{\PYGZhy{}}\PYG{o}{\PYGZhy{}}\PYG{o}{\PYGZhy{}}\PYG{o}{\PYGZhy{}}\PYG{o}{\PYGZhy{}}\PYG{o}{\PYGZhy{}}\PYG{o}{\PYGZhy{}}\PYG{o}{\PYGZhy{}}\PYG{o}{\PYGZhy{}}\PYG{o}{\PYGZhy{}}\PYG{o}{\PYGZhy{}}\PYG{o}{\PYGZhy{}}\PYG{o}{\PYGZhy{}}\PYG{o}{\PYGZhy{}}\PYG{o}{\PYGZhy{}}\PYG{o}{\PYGZhy{}}\PYG{o}{\PYGZhy{}}\PYG{o}{\PYGZhy{}}\PYG{o}{\PYGZhy{}}\PYG{o}{\PYGZhy{}}\PYG{o}{\PYGZhy{}}\PYG{o}{\PYGZhy{}}\PYG{o}{\PYGZhy{}}\PYG{o}{\PYGZhy{}}\PYG{o}{\PYGZhy{}}\PYG{o}{\PYGZhy{}}\PYG{o}{\PYGZhy{}}\PYG{o}{\PYGZhy{}}\PYG{o}{\PYGZhy{}}\PYG{o}{\PYGZhy{}}\PYG{o}{\PYGZhy{}}\PYG{o}{\PYGZhy{}}\PYG{o}{\PYGZhy{}}\PYG{o}{\PYGZhy{}}\PYG{o}{\PYGZhy{}}

\PYG{n}{SP}    \PYG{n}{SP}    \PYG{n}{ALLO}  \PYG{n}{T1}    \PYG{n}{LLIN}  \PYG{l+m+mi}{1}     \PYG{n}{SHIF}      \PYG{l+m+mf}{0.500000}      \PYG{o}{\PYGZhy{}}\PYG{l+m+mf}{80.0000}       \PYG{l+m+mf}{20.0000}
\PYG{n}{CALI}  \PYG{n}{CALI}  \PYG{n}{ALLO}  \PYG{n}{T1}    \PYG{n}{LDAS}  \PYG{l+m+mi}{1}     \PYG{n}{SHIF}      \PYG{l+m+mf}{0.500000}       \PYG{l+m+mf}{5.00000}       \PYG{l+m+mf}{15.0000}
\PYG{n}{MINV}  \PYG{n}{MINV}  \PYG{n}{DISA}  \PYG{n}{T1}    \PYG{n}{LLIN}  \PYG{l+m+mi}{1}     \PYG{n}{SHIF}      \PYG{l+m+mf}{0.500000}       \PYG{l+m+mf}{30.0000}       \PYG{l+m+mf}{0.00000}
\PYG{n}{MNOR}  \PYG{n}{MNOR}  \PYG{n}{DISA}  \PYG{n}{T1}    \PYG{n}{LDAS}  \PYG{l+m+mi}{1}     \PYG{n}{SHIF}      \PYG{l+m+mf}{0.500000}       \PYG{l+m+mf}{30.0000}       \PYG{l+m+mf}{0.00000}
\PYG{n}{LLD}   \PYG{n}{LLD}   \PYG{n}{ALLO}  \PYG{n}{T23}   \PYG{n}{LDAS}  \PYG{l+m+mi}{1}     \PYG{n}{GRAD}      \PYG{l+m+mf}{0.500000}      \PYG{l+m+mf}{0.200000}       \PYG{l+m+mf}{2000.00}
\PYG{n}{LLDB}  \PYG{n}{LLD}   \PYG{n}{ALLO}  \PYG{n}{T2}    \PYG{n}{HDAS}  \PYG{l+m+mi}{1}     \PYG{n}{GRAD}      \PYG{l+m+mf}{0.500000}       \PYG{l+m+mf}{2000.00}       \PYG{l+m+mf}{200000.}
\PYG{n}{LLG}   \PYG{n}{LLG}   \PYG{n}{DISA}  \PYG{n}{T23}   \PYG{n}{LDAS}  \PYG{l+m+mi}{1}     \PYG{n}{GRAD}      \PYG{l+m+mf}{0.500000}      \PYG{l+m+mf}{0.200000}       \PYG{l+m+mf}{2000.00}
\PYG{n}{LLGB}  \PYG{n}{LLG}   \PYG{n}{DISA}  \PYG{n}{T2}    \PYG{n}{HDAS}  \PYG{l+m+mi}{1}     \PYG{n}{GRAD}      \PYG{l+m+mf}{0.500000}       \PYG{l+m+mf}{2000.00}       \PYG{l+m+mf}{200000.}
\PYG{n}{LLS}   \PYG{n}{LLS}   \PYG{n}{ALLO}  \PYG{n}{T23}   \PYG{n}{LSPO}  \PYG{l+m+mi}{1}     \PYG{n}{GRAD}      \PYG{l+m+mf}{0.500000}      \PYG{l+m+mf}{0.200000}       \PYG{l+m+mf}{2000.00}
\PYG{n}{LLSB}  \PYG{n}{LLS}   \PYG{n}{ALLO}  \PYG{n}{T2}    \PYG{n}{HSPO}  \PYG{l+m+mi}{1}     \PYG{n}{GRAD}      \PYG{l+m+mf}{0.500000}       \PYG{l+m+mf}{2000.00}       \PYG{l+m+mf}{200000.}
\PYG{n}{MSFL}  \PYG{n}{MSFL}  \PYG{n}{ALLO}  \PYG{n}{T23}   \PYG{n}{LLIN}  \PYG{l+m+mi}{1}     \PYG{n}{GRAD}      \PYG{l+m+mf}{0.500000}      \PYG{l+m+mf}{0.200000}       \PYG{l+m+mf}{2000.00}

\PYG{n}{Table} \PYG{n}{record} \PYG{p}{(}\PYG{n+nb}{type} \PYG{l+m+mi}{34}\PYG{p}{)} \PYG{n+nb}{type}\PYG{p}{:} \PYG{n}{FILM}

\PYG{n}{MNEM}  \PYG{n}{GCOD}  \PYG{n}{GDEC}  \PYG{n}{DEST}  \PYG{n}{DSCA}
\PYG{o}{\PYGZhy{}}\PYG{o}{\PYGZhy{}}\PYG{o}{\PYGZhy{}}\PYG{o}{\PYGZhy{}}\PYG{o}{\PYGZhy{}}\PYG{o}{\PYGZhy{}}\PYG{o}{\PYGZhy{}}\PYG{o}{\PYGZhy{}}\PYG{o}{\PYGZhy{}}\PYG{o}{\PYGZhy{}}\PYG{o}{\PYGZhy{}}\PYG{o}{\PYGZhy{}}\PYG{o}{\PYGZhy{}}\PYG{o}{\PYGZhy{}}\PYG{o}{\PYGZhy{}}\PYG{o}{\PYGZhy{}}\PYG{o}{\PYGZhy{}}\PYG{o}{\PYGZhy{}}\PYG{o}{\PYGZhy{}}\PYG{o}{\PYGZhy{}}\PYG{o}{\PYGZhy{}}\PYG{o}{\PYGZhy{}}\PYG{o}{\PYGZhy{}}\PYG{o}{\PYGZhy{}}\PYG{o}{\PYGZhy{}}\PYG{o}{\PYGZhy{}}\PYG{o}{\PYGZhy{}}\PYG{o}{\PYGZhy{}}\PYG{o}{\PYGZhy{}}

\PYG{l+m+mi}{1}     \PYG{n}{E20}   \PYG{o}{\PYGZhy{}}\PYG{l+m+mi}{4}\PYG{o}{\PYGZhy{}}\PYG{o}{\PYGZhy{}}  \PYG{n}{PF1}   \PYG{n}{D200}
\PYG{l+m+mi}{2}     \PYG{n}{EEE}   \PYG{o}{\PYGZhy{}}\PYG{o}{\PYGZhy{}}\PYG{o}{\PYGZhy{}}\PYG{o}{\PYGZhy{}}  \PYG{n}{PF2}   \PYG{n}{D200} 

\PYG{n}{Table} \PYG{n}{record} \PYG{p}{(}\PYG{n+nb}{type} \PYG{l+m+mi}{34}\PYG{p}{)} \PYG{n+nb}{type}\PYG{p}{:} \PYG{n}{PRES}

\PYG{n}{MNEM}  \PYG{n}{OUTP}  \PYG{n}{STAT}  \PYG{n}{TRAC}  \PYG{n}{CODI}  \PYG{n}{DEST}  \PYG{n}{MODE}      \PYG{n}{FILT}          \PYG{n}{LEDG}          \PYG{n}{REDG}
\PYG{o}{\PYGZhy{}}\PYG{o}{\PYGZhy{}}\PYG{o}{\PYGZhy{}}\PYG{o}{\PYGZhy{}}\PYG{o}{\PYGZhy{}}\PYG{o}{\PYGZhy{}}\PYG{o}{\PYGZhy{}}\PYG{o}{\PYGZhy{}}\PYG{o}{\PYGZhy{}}\PYG{o}{\PYGZhy{}}\PYG{o}{\PYGZhy{}}\PYG{o}{\PYGZhy{}}\PYG{o}{\PYGZhy{}}\PYG{o}{\PYGZhy{}}\PYG{o}{\PYGZhy{}}\PYG{o}{\PYGZhy{}}\PYG{o}{\PYGZhy{}}\PYG{o}{\PYGZhy{}}\PYG{o}{\PYGZhy{}}\PYG{o}{\PYGZhy{}}\PYG{o}{\PYGZhy{}}\PYG{o}{\PYGZhy{}}\PYG{o}{\PYGZhy{}}\PYG{o}{\PYGZhy{}}\PYG{o}{\PYGZhy{}}\PYG{o}{\PYGZhy{}}\PYG{o}{\PYGZhy{}}\PYG{o}{\PYGZhy{}}\PYG{o}{\PYGZhy{}}\PYG{o}{\PYGZhy{}}\PYG{o}{\PYGZhy{}}\PYG{o}{\PYGZhy{}}\PYG{o}{\PYGZhy{}}\PYG{o}{\PYGZhy{}}\PYG{o}{\PYGZhy{}}\PYG{o}{\PYGZhy{}}\PYG{o}{\PYGZhy{}}\PYG{o}{\PYGZhy{}}\PYG{o}{\PYGZhy{}}\PYG{o}{\PYGZhy{}}\PYG{o}{\PYGZhy{}}\PYG{o}{\PYGZhy{}}\PYG{o}{\PYGZhy{}}\PYG{o}{\PYGZhy{}}\PYG{o}{\PYGZhy{}}\PYG{o}{\PYGZhy{}}\PYG{o}{\PYGZhy{}}\PYG{o}{\PYGZhy{}}\PYG{o}{\PYGZhy{}}\PYG{o}{\PYGZhy{}}\PYG{o}{\PYGZhy{}}\PYG{o}{\PYGZhy{}}\PYG{o}{\PYGZhy{}}\PYG{o}{\PYGZhy{}}\PYG{o}{\PYGZhy{}}\PYG{o}{\PYGZhy{}}\PYG{o}{\PYGZhy{}}\PYG{o}{\PYGZhy{}}\PYG{o}{\PYGZhy{}}\PYG{o}{\PYGZhy{}}\PYG{o}{\PYGZhy{}}\PYG{o}{\PYGZhy{}}\PYG{o}{\PYGZhy{}}\PYG{o}{\PYGZhy{}}\PYG{o}{\PYGZhy{}}\PYG{o}{\PYGZhy{}}\PYG{o}{\PYGZhy{}}\PYG{o}{\PYGZhy{}}\PYG{o}{\PYGZhy{}}\PYG{o}{\PYGZhy{}}\PYG{o}{\PYGZhy{}}\PYG{o}{\PYGZhy{}}\PYG{o}{\PYGZhy{}}\PYG{o}{\PYGZhy{}}\PYG{o}{\PYGZhy{}}\PYG{o}{\PYGZhy{}}\PYG{o}{\PYGZhy{}}\PYG{o}{\PYGZhy{}}\PYG{o}{\PYGZhy{}}\PYG{o}{\PYGZhy{}}\PYG{o}{\PYGZhy{}}\PYG{o}{\PYGZhy{}}\PYG{o}{\PYGZhy{}}

\PYG{n}{NPHI}  \PYG{n}{NPHI}  \PYG{n}{ALLO}  \PYG{n}{T23}   \PYG{n}{LDAS}  \PYG{l+m+mi}{1}     \PYG{n}{SHIF}      \PYG{l+m+mf}{0.500000}      \PYG{l+m+mf}{0.450000}     \PYG{o}{\PYGZhy{}}\PYG{l+m+mf}{0.150000}
\PYG{n}{DRHO}  \PYG{n}{DRHO}  \PYG{n}{ALLO}  \PYG{n}{T3}    \PYG{n}{LSPO}  \PYG{l+m+mi}{1}     \PYG{n}{NB}        \PYG{l+m+mf}{0.500000}     \PYG{o}{\PYGZhy{}}\PYG{l+m+mf}{0.250000}      \PYG{l+m+mf}{0.250000}
\PYG{n}{PEF}   \PYG{n}{PEF}   \PYG{n}{ALLO}  \PYG{n}{T23}   \PYG{n}{LGAP}  \PYG{l+m+mi}{1}     \PYG{n}{SHIF}      \PYG{l+m+mf}{0.500000}       \PYG{l+m+mf}{0.00000}       \PYG{l+m+mf}{10.0000}
\PYG{n}{SGR}         \PYG{n}{DISA}  \PYG{n}{T1}    \PYG{n}{LLIN}  \PYG{l+m+mi}{1}     \PYG{n}{SHIF}      \PYG{l+m+mf}{0.500000}       \PYG{l+m+mf}{0.00000}       \PYG{l+m+mf}{300.000}
\PYG{n}{CGR}         \PYG{n}{DISA}  \PYG{n}{T1}    \PYG{n}{LGAP}  \PYG{l+m+mi}{1}     \PYG{n}{SHIF}      \PYG{l+m+mf}{0.500000}       \PYG{l+m+mf}{0.00000}       \PYG{l+m+mf}{300.000}
\PYG{n}{TENS}  \PYG{n}{TENS}  \PYG{n}{DISA}  \PYG{n}{T3}    \PYG{n}{LGAP}  \PYG{l+m+mi}{1}     \PYG{n}{SHIF}      \PYG{l+m+mf}{0.500000}       \PYG{l+m+mf}{14000.0}       \PYG{l+m+mf}{4000.00}
\PYG{n}{CAL}   \PYG{n}{CALI}  \PYG{n}{ALLO}  \PYG{n}{T1}    \PYG{n}{LSPO}  \PYG{l+m+mi}{1}     \PYG{n}{SHIF}      \PYG{l+m+mf}{0.500000}       \PYG{l+m+mf}{5.00000}       \PYG{l+m+mf}{15.0000}
\PYG{n}{BS}    \PYG{n}{BS}    \PYG{n}{DISA}  \PYG{n}{T1}    \PYG{n}{LGAP}  \PYG{l+m+mi}{1}     \PYG{n}{SHIF}      \PYG{l+m+mf}{0.500000}       \PYG{l+m+mf}{5.00000}       \PYG{l+m+mf}{15.0000}
\PYG{n}{FFLS}  \PYG{n}{FFLS}  \PYG{n}{DISA}  \PYG{n}{T1}    \PYG{n}{LLIN}  \PYG{l+m+mi}{2}     \PYG{n}{NB}        \PYG{l+m+mf}{0.500000}     \PYG{o}{\PYGZhy{}}\PYG{l+m+mf}{0.150000}      \PYG{l+m+mf}{0.150000}
\PYG{n}{FFSS}  \PYG{n}{FFSS}  \PYG{n}{DISA}  \PYG{n}{T1}    \PYG{n}{LDAS}  \PYG{l+m+mi}{2}     \PYG{n}{NB}        \PYG{l+m+mf}{0.500000}     \PYG{o}{\PYGZhy{}}\PYG{l+m+mf}{0.150000}      \PYG{l+m+mf}{0.150000}
\PYG{n}{LSHV}  \PYG{n}{LSHV}  \PYG{n}{DISA}  \PYG{n}{T3}    \PYG{n}{LLIN}  \PYG{l+m+mi}{2}     \PYG{n}{WRAP}      \PYG{l+m+mf}{0.500000}       \PYG{l+m+mf}{2150.00}       \PYG{l+m+mf}{2250.00}
\PYG{n}{SSHV}  \PYG{n}{SSHV}  \PYG{n}{DISA}  \PYG{n}{T3}    \PYG{n}{LDAS}  \PYG{l+m+mi}{2}     \PYG{n}{WRAP}      \PYG{l+m+mf}{0.500000}       \PYG{l+m+mf}{1950.00}       \PYG{l+m+mf}{2050.00}
\PYG{n}{FLS}   \PYG{n}{FLS}   \PYG{n}{DISA}  \PYG{n}{T2}    \PYG{n}{LLIN}  \PYG{l+m+mi}{2}     \PYG{n}{SHIF}      \PYG{l+m+mf}{0.500000}       \PYG{l+m+mf}{0.00000}       \PYG{l+m+mf}{150.000}
\PYG{n}{FSS}   \PYG{n}{FSS}   \PYG{n}{DISA}  \PYG{n}{T2}    \PYG{n}{LDAS}  \PYG{l+m+mi}{2}     \PYG{n}{SHIF}      \PYG{l+m+mf}{0.500000}       \PYG{l+m+mf}{0.00000}       \PYG{l+m+mf}{150.000}
\PYG{n}{RHOB}  \PYG{n}{RHOB}  \PYG{n}{ALLO}  \PYG{n}{T23}   \PYG{n}{LLIN}  \PYG{l+m+mi}{1}     \PYG{n}{SHIF}      \PYG{l+m+mf}{0.500000}       \PYG{l+m+mf}{1.95000}       \PYG{l+m+mf}{2.95000}
\PYG{n}{PHIX}  \PYG{n}{PHIX}  \PYG{n}{ALLO}  \PYG{n}{T1}    \PYG{n}{LLIN}  \PYG{l+m+mi}{1}     \PYG{n}{NB}        \PYG{l+m+mf}{0.500000}      \PYG{l+m+mf}{0.500000}       \PYG{l+m+mf}{0.00000}

\PYG{n}{Table} \PYG{n}{record} \PYG{p}{(}\PYG{n+nb}{type} \PYG{l+m+mi}{34}\PYG{p}{)} \PYG{n+nb}{type}\PYG{p}{:} \PYG{n}{FILM}

\PYG{n}{MNEM}  \PYG{n}{GCOD}  \PYG{n}{GDEC}  \PYG{n}{DEST}  \PYG{n}{DSCA}
\PYG{o}{\PYGZhy{}}\PYG{o}{\PYGZhy{}}\PYG{o}{\PYGZhy{}}\PYG{o}{\PYGZhy{}}\PYG{o}{\PYGZhy{}}\PYG{o}{\PYGZhy{}}\PYG{o}{\PYGZhy{}}\PYG{o}{\PYGZhy{}}\PYG{o}{\PYGZhy{}}\PYG{o}{\PYGZhy{}}\PYG{o}{\PYGZhy{}}\PYG{o}{\PYGZhy{}}\PYG{o}{\PYGZhy{}}\PYG{o}{\PYGZhy{}}\PYG{o}{\PYGZhy{}}\PYG{o}{\PYGZhy{}}\PYG{o}{\PYGZhy{}}\PYG{o}{\PYGZhy{}}\PYG{o}{\PYGZhy{}}\PYG{o}{\PYGZhy{}}\PYG{o}{\PYGZhy{}}\PYG{o}{\PYGZhy{}}\PYG{o}{\PYGZhy{}}\PYG{o}{\PYGZhy{}}\PYG{o}{\PYGZhy{}}\PYG{o}{\PYGZhy{}}\PYG{o}{\PYGZhy{}}\PYG{o}{\PYGZhy{}}\PYG{o}{\PYGZhy{}}

\PYG{l+m+mi}{1}     \PYG{n}{EEE}   \PYG{o}{\PYGZhy{}}\PYG{o}{\PYGZhy{}}\PYG{o}{\PYGZhy{}}\PYG{o}{\PYGZhy{}}  \PYG{n}{PF2}   \PYG{n}{D200}
\PYG{l+m+mi}{2}     \PYG{n}{EEE}   \PYG{o}{\PYGZhy{}}\PYG{o}{\PYGZhy{}}\PYG{o}{\PYGZhy{}}\PYG{o}{\PYGZhy{}}  \PYG{n}{PF1}   \PYG{n}{DM}
\end{sphinxVerbatim}

Other FILM Table Examples:

\begin{sphinxVerbatim}[commandchars=\\\{\}]
\PYG{n}{Table} \PYG{n}{record} \PYG{p}{(}\PYG{n+nb}{type} \PYG{l+m+mi}{34}\PYG{p}{)} \PYG{n+nb}{type}\PYG{p}{:} \PYG{n}{FILM}

\PYG{n}{MNEM}  \PYG{n}{GCOD}  \PYG{n}{GDEC}  \PYG{n}{DEST}  \PYG{n}{DSCA}
\PYG{o}{\PYGZhy{}}\PYG{o}{\PYGZhy{}}\PYG{o}{\PYGZhy{}}\PYG{o}{\PYGZhy{}}\PYG{o}{\PYGZhy{}}\PYG{o}{\PYGZhy{}}\PYG{o}{\PYGZhy{}}\PYG{o}{\PYGZhy{}}\PYG{o}{\PYGZhy{}}\PYG{o}{\PYGZhy{}}\PYG{o}{\PYGZhy{}}\PYG{o}{\PYGZhy{}}\PYG{o}{\PYGZhy{}}\PYG{o}{\PYGZhy{}}\PYG{o}{\PYGZhy{}}\PYG{o}{\PYGZhy{}}\PYG{o}{\PYGZhy{}}\PYG{o}{\PYGZhy{}}\PYG{o}{\PYGZhy{}}\PYG{o}{\PYGZhy{}}\PYG{o}{\PYGZhy{}}\PYG{o}{\PYGZhy{}}\PYG{o}{\PYGZhy{}}\PYG{o}{\PYGZhy{}}\PYG{o}{\PYGZhy{}}\PYG{o}{\PYGZhy{}}\PYG{o}{\PYGZhy{}}\PYG{o}{\PYGZhy{}}\PYG{o}{\PYGZhy{}}
\PYG{l+m+mi}{1}     \PYG{n}{EEB}   \PYG{o}{\PYGZhy{}}\PYG{o}{\PYGZhy{}}\PYG{o}{\PYGZhy{}}\PYG{o}{\PYGZhy{}}  \PYG{n}{PF1}   \PYG{n}{D200}
\PYG{l+m+mi}{2}     \PYG{n}{EEB}   \PYG{o}{\PYGZhy{}}\PYG{o}{\PYGZhy{}}\PYG{o}{\PYGZhy{}}\PYG{o}{\PYGZhy{}}  \PYG{n}{PF2}   \PYG{n}{DM}

\PYG{n}{Table} \PYG{n}{record} \PYG{p}{(}\PYG{n+nb}{type} \PYG{l+m+mi}{34}\PYG{p}{)} \PYG{n+nb}{type}\PYG{p}{:} \PYG{n}{FILM}

\PYG{n}{MNEM}  \PYG{n}{GCOD}  \PYG{n}{GDEC}  \PYG{n}{DEST}  \PYG{n}{DSCA}
\PYG{o}{\PYGZhy{}}\PYG{o}{\PYGZhy{}}\PYG{o}{\PYGZhy{}}\PYG{o}{\PYGZhy{}}\PYG{o}{\PYGZhy{}}\PYG{o}{\PYGZhy{}}\PYG{o}{\PYGZhy{}}\PYG{o}{\PYGZhy{}}\PYG{o}{\PYGZhy{}}\PYG{o}{\PYGZhy{}}\PYG{o}{\PYGZhy{}}\PYG{o}{\PYGZhy{}}\PYG{o}{\PYGZhy{}}\PYG{o}{\PYGZhy{}}\PYG{o}{\PYGZhy{}}\PYG{o}{\PYGZhy{}}\PYG{o}{\PYGZhy{}}\PYG{o}{\PYGZhy{}}\PYG{o}{\PYGZhy{}}\PYG{o}{\PYGZhy{}}\PYG{o}{\PYGZhy{}}\PYG{o}{\PYGZhy{}}\PYG{o}{\PYGZhy{}}\PYG{o}{\PYGZhy{}}\PYG{o}{\PYGZhy{}}\PYG{o}{\PYGZhy{}}\PYG{o}{\PYGZhy{}}\PYG{o}{\PYGZhy{}}\PYG{o}{\PYGZhy{}}
\PYG{l+m+mi}{1}     \PYG{n}{E20}   \PYG{o}{\PYGZhy{}}\PYG{l+m+mi}{4}\PYG{o}{\PYGZhy{}}\PYG{o}{\PYGZhy{}}  \PYG{n}{PF1}   \PYG{n}{D200}
\PYG{l+m+mi}{2}     \PYG{n}{EEE}   \PYG{o}{\PYGZhy{}}\PYG{o}{\PYGZhy{}}\PYG{o}{\PYGZhy{}}\PYG{o}{\PYGZhy{}}  \PYG{n}{PF2}   \PYG{n}{D200}

\PYG{n}{MNEM}  \PYG{n}{GCOD}  \PYG{n}{GDEC}  \PYG{n}{DEST}  \PYG{n}{DSCA}
\PYG{o}{\PYGZhy{}}\PYG{o}{\PYGZhy{}}\PYG{o}{\PYGZhy{}}\PYG{o}{\PYGZhy{}}\PYG{o}{\PYGZhy{}}\PYG{o}{\PYGZhy{}}\PYG{o}{\PYGZhy{}}\PYG{o}{\PYGZhy{}}\PYG{o}{\PYGZhy{}}\PYG{o}{\PYGZhy{}}\PYG{o}{\PYGZhy{}}\PYG{o}{\PYGZhy{}}\PYG{o}{\PYGZhy{}}\PYG{o}{\PYGZhy{}}\PYG{o}{\PYGZhy{}}\PYG{o}{\PYGZhy{}}\PYG{o}{\PYGZhy{}}\PYG{o}{\PYGZhy{}}\PYG{o}{\PYGZhy{}}\PYG{o}{\PYGZhy{}}\PYG{o}{\PYGZhy{}}\PYG{o}{\PYGZhy{}}\PYG{o}{\PYGZhy{}}\PYG{o}{\PYGZhy{}}\PYG{o}{\PYGZhy{}}\PYG{o}{\PYGZhy{}}\PYG{o}{\PYGZhy{}}\PYG{o}{\PYGZhy{}}\PYG{o}{\PYGZhy{}}
\PYG{l+m+mi}{1}     \PYG{n}{EEE}   \PYG{o}{\PYGZhy{}}\PYG{o}{\PYGZhy{}}\PYG{o}{\PYGZhy{}}\PYG{o}{\PYGZhy{}}  \PYG{n}{PF1}   \PYG{n}{D200}
\PYG{l+m+mi}{2}     \PYG{n}{E1E}   \PYG{o}{\PYGZhy{}}\PYG{l+m+mi}{4}\PYG{o}{\PYGZhy{}}   \PYG{n}{PF2}   \PYG{n}{D200}

\PYG{n}{Table} \PYG{n}{record} \PYG{p}{(}\PYG{n+nb}{type} \PYG{l+m+mi}{34}\PYG{p}{)} \PYG{n+nb}{type}\PYG{p}{:} \PYG{n}{FILM}

\PYG{n}{MNEM}  \PYG{n}{GCOD}  \PYG{n}{GDEC}  \PYG{n}{DEST}  \PYG{n}{DSCA}
\PYG{o}{\PYGZhy{}}\PYG{o}{\PYGZhy{}}\PYG{o}{\PYGZhy{}}\PYG{o}{\PYGZhy{}}\PYG{o}{\PYGZhy{}}\PYG{o}{\PYGZhy{}}\PYG{o}{\PYGZhy{}}\PYG{o}{\PYGZhy{}}\PYG{o}{\PYGZhy{}}\PYG{o}{\PYGZhy{}}\PYG{o}{\PYGZhy{}}\PYG{o}{\PYGZhy{}}\PYG{o}{\PYGZhy{}}\PYG{o}{\PYGZhy{}}\PYG{o}{\PYGZhy{}}\PYG{o}{\PYGZhy{}}\PYG{o}{\PYGZhy{}}\PYG{o}{\PYGZhy{}}\PYG{o}{\PYGZhy{}}\PYG{o}{\PYGZhy{}}\PYG{o}{\PYGZhy{}}\PYG{o}{\PYGZhy{}}\PYG{o}{\PYGZhy{}}\PYG{o}{\PYGZhy{}}\PYG{o}{\PYGZhy{}}\PYG{o}{\PYGZhy{}}\PYG{o}{\PYGZhy{}}\PYG{o}{\PYGZhy{}}\PYG{o}{\PYGZhy{}}

\PYG{n}{D}     \PYG{n}{E3E}   \PYG{o}{\PYGZhy{}}\PYG{l+m+mi}{3}\PYG{o}{\PYGZhy{}}   \PYG{n}{PFD}   \PYG{n}{D200}
\PYG{n}{E}     \PYG{n}{E3E}   \PYG{o}{\PYGZhy{}}\PYG{l+m+mi}{3}\PYG{o}{\PYGZhy{}}   \PYG{n}{PFE}   \PYG{n}{D500}
\PYG{l+m+mi}{5}     \PYG{n}{EB0}   \PYG{o}{\PYGZhy{}}\PYG{o}{\PYGZhy{}}\PYG{o}{\PYGZhy{}}   \PYG{n}{PF5}   \PYG{n}{D200}
\PYG{l+m+mi}{6}     \PYG{n}{EEB}   \PYG{o}{\PYGZhy{}}\PYG{o}{\PYGZhy{}}\PYG{o}{\PYGZhy{}}   \PYG{n}{PF6}   \PYG{n}{D200}

\PYG{n}{Table} \PYG{n}{record} \PYG{p}{(}\PYG{n+nb}{type} \PYG{l+m+mi}{34}\PYG{p}{)} \PYG{n+nb}{type}\PYG{p}{:} \PYG{n}{FILM}

\PYG{n}{MNEM}  \PYG{n}{GCOD}  \PYG{n}{GDEC}  \PYG{n}{DEST}  \PYG{n}{DSCA}
\PYG{o}{\PYGZhy{}}\PYG{o}{\PYGZhy{}}\PYG{o}{\PYGZhy{}}\PYG{o}{\PYGZhy{}}\PYG{o}{\PYGZhy{}}\PYG{o}{\PYGZhy{}}\PYG{o}{\PYGZhy{}}\PYG{o}{\PYGZhy{}}\PYG{o}{\PYGZhy{}}\PYG{o}{\PYGZhy{}}\PYG{o}{\PYGZhy{}}\PYG{o}{\PYGZhy{}}\PYG{o}{\PYGZhy{}}\PYG{o}{\PYGZhy{}}\PYG{o}{\PYGZhy{}}\PYG{o}{\PYGZhy{}}\PYG{o}{\PYGZhy{}}\PYG{o}{\PYGZhy{}}\PYG{o}{\PYGZhy{}}\PYG{o}{\PYGZhy{}}\PYG{o}{\PYGZhy{}}\PYG{o}{\PYGZhy{}}\PYG{o}{\PYGZhy{}}\PYG{o}{\PYGZhy{}}\PYG{o}{\PYGZhy{}}\PYG{o}{\PYGZhy{}}\PYG{o}{\PYGZhy{}}\PYG{o}{\PYGZhy{}}\PYG{o}{\PYGZhy{}}

\PYG{l+m+mi}{8}     \PYG{n}{EB0}   \PYG{o}{\PYGZhy{}}\PYG{o}{\PYGZhy{}}\PYG{o}{\PYGZhy{}}   \PYG{n}{PF8}   \PYG{n}{D200}
\PYG{n}{A}     \PYG{n}{LLLL}  \PYG{l+m+mi}{1111}  \PYG{n}{PFA}   \PYG{n}{DM}
\PYG{n}{E}     \PYG{n}{E4E}   \PYG{o}{\PYGZhy{}}\PYG{l+m+mi}{4}\PYG{o}{\PYGZhy{}}   \PYG{n}{PFE}   \PYG{n}{D200}
\PYG{n}{K}     \PYG{n}{E4E}   \PYG{o}{\PYGZhy{}}\PYG{l+m+mi}{4}\PYG{o}{\PYGZhy{}}   \PYG{n}{PFK}   \PYG{n}{D500}

\PYG{n}{Table} \PYG{n}{record} \PYG{p}{(}\PYG{n+nb}{type} \PYG{l+m+mi}{34}\PYG{p}{)} \PYG{n+nb}{type}\PYG{p}{:} \PYG{n}{FILM}

\PYG{n}{MNEM}      \PYG{n}{CINT}      \PYG{n}{GCOD}  \PYG{n}{GDEC}  \PYG{n}{DEST}  \PYG{n}{DSCA}
\PYG{o}{\PYGZhy{}}\PYG{o}{\PYGZhy{}}\PYG{o}{\PYGZhy{}}\PYG{o}{\PYGZhy{}}\PYG{o}{\PYGZhy{}}\PYG{o}{\PYGZhy{}}\PYG{o}{\PYGZhy{}}\PYG{o}{\PYGZhy{}}\PYG{o}{\PYGZhy{}}\PYG{o}{\PYGZhy{}}\PYG{o}{\PYGZhy{}}\PYG{o}{\PYGZhy{}}\PYG{o}{\PYGZhy{}}\PYG{o}{\PYGZhy{}}\PYG{o}{\PYGZhy{}}\PYG{o}{\PYGZhy{}}\PYG{o}{\PYGZhy{}}\PYG{o}{\PYGZhy{}}\PYG{o}{\PYGZhy{}}\PYG{o}{\PYGZhy{}}\PYG{o}{\PYGZhy{}}\PYG{o}{\PYGZhy{}}\PYG{o}{\PYGZhy{}}\PYG{o}{\PYGZhy{}}\PYG{o}{\PYGZhy{}}\PYG{o}{\PYGZhy{}}\PYG{o}{\PYGZhy{}}\PYG{o}{\PYGZhy{}}\PYG{o}{\PYGZhy{}}\PYG{o}{\PYGZhy{}}\PYG{o}{\PYGZhy{}}\PYG{o}{\PYGZhy{}}\PYG{o}{\PYGZhy{}}\PYG{o}{\PYGZhy{}}\PYG{o}{\PYGZhy{}}\PYG{o}{\PYGZhy{}}\PYG{o}{\PYGZhy{}}\PYG{o}{\PYGZhy{}}\PYG{o}{\PYGZhy{}}\PYG{o}{\PYGZhy{}}\PYG{o}{\PYGZhy{}}\PYG{o}{\PYGZhy{}}\PYG{o}{\PYGZhy{}}

\PYG{l+m+mi}{1}          \PYG{l+m+mf}{0.00000}  \PYG{n}{E2E}   \PYG{o}{\PYGZhy{}}\PYG{l+m+mi}{2}\PYG{o}{\PYGZhy{}}   \PYG{n}{PF1}   \PYG{n}{D200}
\PYG{l+m+mi}{2}          \PYG{l+m+mf}{0.00000}  \PYG{n}{E2E}   \PYG{o}{\PYGZhy{}}\PYG{l+m+mi}{1}\PYG{o}{\PYGZhy{}}   \PYG{n}{PF2}   \PYG{n}{D500}
\PYG{l+m+mi}{3}          \PYG{l+m+mf}{0.00000}  \PYG{n}{EEE}   \PYG{o}{\PYGZhy{}}\PYG{o}{\PYGZhy{}}\PYG{o}{\PYGZhy{}}\PYG{o}{\PYGZhy{}}  \PYG{n}{NEIT}  \PYG{n}{S5}
\PYG{l+m+mi}{4}          \PYG{l+m+mf}{0.00000}  \PYG{n}{EEE}   \PYG{o}{\PYGZhy{}}\PYG{o}{\PYGZhy{}}\PYG{o}{\PYGZhy{}}\PYG{o}{\PYGZhy{}}  \PYG{n}{NEIT}  \PYG{n}{S5}

\PYG{n}{Table} \PYG{n}{record} \PYG{p}{(}\PYG{n+nb}{type} \PYG{l+m+mi}{34}\PYG{p}{)} \PYG{n+nb}{type}\PYG{p}{:} \PYG{n}{FILM}

\PYG{n}{MNEM}      \PYG{n}{CINT}      \PYG{n}{GCOD}  \PYG{n}{GDEC}  \PYG{n}{DEST}  \PYG{n}{DSCA}
\PYG{o}{\PYGZhy{}}\PYG{o}{\PYGZhy{}}\PYG{o}{\PYGZhy{}}\PYG{o}{\PYGZhy{}}\PYG{o}{\PYGZhy{}}\PYG{o}{\PYGZhy{}}\PYG{o}{\PYGZhy{}}\PYG{o}{\PYGZhy{}}\PYG{o}{\PYGZhy{}}\PYG{o}{\PYGZhy{}}\PYG{o}{\PYGZhy{}}\PYG{o}{\PYGZhy{}}\PYG{o}{\PYGZhy{}}\PYG{o}{\PYGZhy{}}\PYG{o}{\PYGZhy{}}\PYG{o}{\PYGZhy{}}\PYG{o}{\PYGZhy{}}\PYG{o}{\PYGZhy{}}\PYG{o}{\PYGZhy{}}\PYG{o}{\PYGZhy{}}\PYG{o}{\PYGZhy{}}\PYG{o}{\PYGZhy{}}\PYG{o}{\PYGZhy{}}\PYG{o}{\PYGZhy{}}\PYG{o}{\PYGZhy{}}\PYG{o}{\PYGZhy{}}\PYG{o}{\PYGZhy{}}\PYG{o}{\PYGZhy{}}\PYG{o}{\PYGZhy{}}\PYG{o}{\PYGZhy{}}\PYG{o}{\PYGZhy{}}\PYG{o}{\PYGZhy{}}\PYG{o}{\PYGZhy{}}\PYG{o}{\PYGZhy{}}\PYG{o}{\PYGZhy{}}\PYG{o}{\PYGZhy{}}\PYG{o}{\PYGZhy{}}\PYG{o}{\PYGZhy{}}\PYG{o}{\PYGZhy{}}\PYG{o}{\PYGZhy{}}\PYG{o}{\PYGZhy{}}\PYG{o}{\PYGZhy{}}\PYG{o}{\PYGZhy{}}

\PYG{l+m+mi}{1}          \PYG{l+m+mf}{300.000}  \PYG{n}{E2E}   \PYG{o}{\PYGZhy{}}\PYG{l+m+mi}{2}\PYG{o}{\PYGZhy{}}   \PYG{n}{PF1}   \PYG{n}{D200}
\PYG{l+m+mi}{2}          \PYG{l+m+mf}{300.000}  \PYG{n}{E2E}   \PYG{o}{\PYGZhy{}}\PYG{l+m+mi}{2}\PYG{o}{\PYGZhy{}}   \PYG{n}{PF2}   \PYG{n}{D500}
\PYG{l+m+mi}{3}          \PYG{l+m+mf}{300.000}  \PYG{n}{EEE}   \PYG{o}{\PYGZhy{}}\PYG{o}{\PYGZhy{}}\PYG{o}{\PYGZhy{}}\PYG{o}{\PYGZhy{}}  \PYG{n}{NEIT}  \PYG{n}{S5}
\PYG{l+m+mi}{4}          \PYG{l+m+mf}{300.000}  \PYG{n}{EEE}   \PYG{o}{\PYGZhy{}}\PYG{o}{\PYGZhy{}}\PYG{o}{\PYGZhy{}}\PYG{o}{\PYGZhy{}}  \PYG{n}{NEIT}  \PYG{n}{S5}

\PYG{n}{Table} \PYG{n}{record} \PYG{p}{(}\PYG{n+nb}{type} \PYG{l+m+mi}{34}\PYG{p}{)} \PYG{n+nb}{type}\PYG{p}{:} \PYG{n}{FILM}

\PYG{n}{MNEM}      \PYG{n}{CINT}      \PYG{n}{GCOD}  \PYG{n}{GDEC}  \PYG{n}{DEST}  \PYG{n}{DSCA}
\PYG{o}{\PYGZhy{}}\PYG{o}{\PYGZhy{}}\PYG{o}{\PYGZhy{}}\PYG{o}{\PYGZhy{}}\PYG{o}{\PYGZhy{}}\PYG{o}{\PYGZhy{}}\PYG{o}{\PYGZhy{}}\PYG{o}{\PYGZhy{}}\PYG{o}{\PYGZhy{}}\PYG{o}{\PYGZhy{}}\PYG{o}{\PYGZhy{}}\PYG{o}{\PYGZhy{}}\PYG{o}{\PYGZhy{}}\PYG{o}{\PYGZhy{}}\PYG{o}{\PYGZhy{}}\PYG{o}{\PYGZhy{}}\PYG{o}{\PYGZhy{}}\PYG{o}{\PYGZhy{}}\PYG{o}{\PYGZhy{}}\PYG{o}{\PYGZhy{}}\PYG{o}{\PYGZhy{}}\PYG{o}{\PYGZhy{}}\PYG{o}{\PYGZhy{}}\PYG{o}{\PYGZhy{}}\PYG{o}{\PYGZhy{}}\PYG{o}{\PYGZhy{}}\PYG{o}{\PYGZhy{}}\PYG{o}{\PYGZhy{}}\PYG{o}{\PYGZhy{}}\PYG{o}{\PYGZhy{}}\PYG{o}{\PYGZhy{}}\PYG{o}{\PYGZhy{}}\PYG{o}{\PYGZhy{}}\PYG{o}{\PYGZhy{}}\PYG{o}{\PYGZhy{}}\PYG{o}{\PYGZhy{}}\PYG{o}{\PYGZhy{}}\PYG{o}{\PYGZhy{}}\PYG{o}{\PYGZhy{}}\PYG{o}{\PYGZhy{}}\PYG{o}{\PYGZhy{}}\PYG{o}{\PYGZhy{}}\PYG{o}{\PYGZhy{}}

\PYG{l+m+mi}{1}          \PYG{l+m+mf}{50.0000}  \PYG{n}{EEE}   \PYG{o}{\PYGZhy{}}\PYG{o}{\PYGZhy{}}\PYG{o}{\PYGZhy{}}\PYG{o}{\PYGZhy{}}  \PYG{n}{PF1}   \PYG{n}{D200}
\PYG{l+m+mi}{2}          \PYG{l+m+mf}{0.00000}  \PYG{n}{EEE}   \PYG{o}{\PYGZhy{}}\PYG{o}{\PYGZhy{}}\PYG{o}{\PYGZhy{}}\PYG{o}{\PYGZhy{}}  \PYG{n}{PF2}   \PYG{n}{D200}
\PYG{l+m+mi}{3}          \PYG{l+m+mf}{0.00000}  \PYG{n}{EEE}   \PYG{o}{\PYGZhy{}}\PYG{o}{\PYGZhy{}}\PYG{o}{\PYGZhy{}}\PYG{o}{\PYGZhy{}}  \PYG{n}{NEIT}  \PYG{n}{D200}
\PYG{l+m+mi}{4}          \PYG{l+m+mf}{0.00000}  \PYG{n}{EEE}   \PYG{o}{\PYGZhy{}}\PYG{o}{\PYGZhy{}}\PYG{o}{\PYGZhy{}}\PYG{o}{\PYGZhy{}}  \PYG{n}{NEIT}  \PYG{n}{D200}
\end{sphinxVerbatim}

Minimal, but not complete interpretation:

\begin{sphinxVerbatim}[commandchars=\\\{\}]
Ignore GDEC as dupe.
E \PYGZhy{} Equi\PYGZhy{}spaced (linear).
n \PYGZhy{} Log with number of decades.
B \PYGZhy{} Blank.
L \PYGZhy{} ?
\end{sphinxVerbatim}

What to do with 0 (continuation?).
Examples: E20   -4\textendash{} means 4 decades over track 23.
\index{ExceptionTotalDepthLISPlotTrack}

\begin{fulllineitems}
\phantomsection\label{\detokenize{ref/util/plot/Track:TotalDepth.util.plot.Track.ExceptionTotalDepthLISPlotTrack}}\pysigline{\sphinxbfcode{exception }\sphinxcode{TotalDepth.util.plot.Track.}\sphinxbfcode{ExceptionTotalDepthLISPlotTrack}}
Exception for plotting tracks.

\end{fulllineitems}

\index{genLinear10() (in module TotalDepth.util.plot.Track)}

\begin{fulllineitems}
\phantomsection\label{\detokenize{ref/util/plot/Track:TotalDepth.util.plot.Track.genLinear10}}\pysiglinewithargsret{\sphinxcode{TotalDepth.util.plot.Track.}\sphinxbfcode{genLinear10}}{\emph{l}, \emph{r}}{}
Generate a linear series of 10 track grid lines as (Stroke, position).

\end{fulllineitems}

\index{genLog10() (in module TotalDepth.util.plot.Track)}

\begin{fulllineitems}
\phantomsection\label{\detokenize{ref/util/plot/Track:TotalDepth.util.plot.Track.genLog10}}\pysiglinewithargsret{\sphinxcode{TotalDepth.util.plot.Track.}\sphinxbfcode{genLog10}}{\emph{l}, \emph{r}, \emph{cycles=1}, \emph{start=1}}{}
Generate a log10 series of track grid lines as (Stroke, position).
cycles is the number of log cycles to split the track up into.
start is the start position of the scale e.g. 2 for common resistivity curves.
So cycles=2, start=2 would give log grid 2 to 20000 i.e. to plot
resistivity in the range 0.2 to 2000.

\end{fulllineitems}

\index{genLog10Decade2 (in module TotalDepth.util.plot.Track)}

\begin{fulllineitems}
\phantomsection\label{\detokenize{ref/util/plot/Track:TotalDepth.util.plot.Track.genLog10Decade2}}\pysigline{\sphinxcode{TotalDepth.util.plot.Track.}\sphinxbfcode{genLog10Decade2}\sphinxbfcode{ = functools.partial(\textless{}function genLog10\textgreater{}, cycles=2, start=1)}}
Generator for 2 decades of log base 10

\end{fulllineitems}

\index{genLog10Decade3 (in module TotalDepth.util.plot.Track)}

\begin{fulllineitems}
\phantomsection\label{\detokenize{ref/util/plot/Track:TotalDepth.util.plot.Track.genLog10Decade3}}\pysigline{\sphinxcode{TotalDepth.util.plot.Track.}\sphinxbfcode{genLog10Decade3}\sphinxbfcode{ = functools.partial(\textless{}function genLog10\textgreater{}, cycles=3, start=1)}}
Generator for 3 decades of log base 10

\end{fulllineitems}

\index{genLog10Decade4 (in module TotalDepth.util.plot.Track)}

\begin{fulllineitems}
\phantomsection\label{\detokenize{ref/util/plot/Track:TotalDepth.util.plot.Track.genLog10Decade4}}\pysigline{\sphinxcode{TotalDepth.util.plot.Track.}\sphinxbfcode{genLog10Decade4}\sphinxbfcode{ = functools.partial(\textless{}function genLog10\textgreater{}, cycles=4, start=1)}}
Generator for 4 decades of log base 10

\end{fulllineitems}

\index{genLog10Decade5 (in module TotalDepth.util.plot.Track)}

\begin{fulllineitems}
\phantomsection\label{\detokenize{ref/util/plot/Track:TotalDepth.util.plot.Track.genLog10Decade5}}\pysigline{\sphinxcode{TotalDepth.util.plot.Track.}\sphinxbfcode{genLog10Decade5}\sphinxbfcode{ = functools.partial(\textless{}function genLog10\textgreater{}, cycles=5, start=1)}}
Generator for 5 decades of log base 10

\end{fulllineitems}

\index{genLog10Decade1Start2 (in module TotalDepth.util.plot.Track)}

\begin{fulllineitems}
\phantomsection\label{\detokenize{ref/util/plot/Track:TotalDepth.util.plot.Track.genLog10Decade1Start2}}\pysigline{\sphinxcode{TotalDepth.util.plot.Track.}\sphinxbfcode{genLog10Decade1Start2}\sphinxbfcode{ = functools.partial(\textless{}function genLog10\textgreater{}, cycles=1, start=2)}}
Generator for 1 decade of log base 10 starting at 2

\end{fulllineitems}

\index{genLog10Decade2Start2 (in module TotalDepth.util.plot.Track)}

\begin{fulllineitems}
\phantomsection\label{\detokenize{ref/util/plot/Track:TotalDepth.util.plot.Track.genLog10Decade2Start2}}\pysigline{\sphinxcode{TotalDepth.util.plot.Track.}\sphinxbfcode{genLog10Decade2Start2}\sphinxbfcode{ = functools.partial(\textless{}function genLog10\textgreater{}, cycles=2, start=2)}}
Generator for 2 decades of log base 10 starting at 2

\end{fulllineitems}

\index{Track (class in TotalDepth.util.plot.Track)}

\begin{fulllineitems}
\phantomsection\label{\detokenize{ref/util/plot/Track:TotalDepth.util.plot.Track.Track}}\pysiglinewithargsret{\sphinxbfcode{class }\sphinxcode{TotalDepth.util.plot.Track.}\sphinxbfcode{Track}}{\emph{leftPos}, \emph{rightPos}, \emph{gridGn}, \emph{plotXLines=True}, \emph{plotXAlpha=False}}{}
Class that represents a single track. The track, as a structural
graphical element, is merely a grid.
The actual curves are plotted on panes that are independent from
a single track (can span multiple tracks for example).

leftPos - A Coord.Dim() object that is the left edge.

rightPos- A Coord.Dim() object that is the right edge.

gridGn - A generator of line positions, None for blank track.

plotXLines - If True then plot X Axis information (depth lines) in this track.

plotXAlpha - If True then plot X Axis information (depth numbers) in this track.
\index{DEPTH\_PER\_CH (TotalDepth.util.plot.Track.Track attribute)}

\begin{fulllineitems}
\phantomsection\label{\detokenize{ref/util/plot/Track:TotalDepth.util.plot.Track.Track.DEPTH_PER_CH}}\pysigline{\sphinxbfcode{DEPTH\_PER\_CH}\sphinxbfcode{ = Dim(value=0.25, units='in')}}
Space for plotting the scale for each curve

\end{fulllineitems}

\index{left (TotalDepth.util.plot.Track.Track attribute)}

\begin{fulllineitems}
\phantomsection\label{\detokenize{ref/util/plot/Track:TotalDepth.util.plot.Track.Track.left}}\pysigline{\sphinxbfcode{left}}
The left edge as a Coord.Dim().

\end{fulllineitems}

\index{right (TotalDepth.util.plot.Track.Track attribute)}

\begin{fulllineitems}
\phantomsection\label{\detokenize{ref/util/plot/Track:TotalDepth.util.plot.Track.Track.right}}\pysigline{\sphinxbfcode{right}}
The left edge as a Coord.Dim().

\end{fulllineitems}

\index{hasGrid (TotalDepth.util.plot.Track.Track attribute)}

\begin{fulllineitems}
\phantomsection\label{\detokenize{ref/util/plot/Track:TotalDepth.util.plot.Track.Track.hasGrid}}\pysigline{\sphinxbfcode{hasGrid}}
True if ther is a grid to be generated for this track.

\end{fulllineitems}

\index{plotSVG() (TotalDepth.util.plot.Track.Track method)}

\begin{fulllineitems}
\phantomsection\label{\detokenize{ref/util/plot/Track:TotalDepth.util.plot.Track.Track.plotSVG}}\pysiglinewithargsret{\sphinxbfcode{plotSVG}}{\emph{topLeft}, \emph{depth}, \emph{theSVGWriter}}{}
Plot the track gridlines.
topLeft - A Coord.Pt() object that is the top left of the canvas.
depth - A Coord.Dim() object that is the total depth of the grid below topLeft.
drive - The plot driver.

\end{fulllineitems}


\end{fulllineitems}



\paragraph{X Axis Gird Line Module}
\label{\detokenize{ref/util/plot/XGrid:x-axis-gird-line-module}}\label{\detokenize{ref/util/plot/XGrid::doc}}\label{\detokenize{ref/util/plot/XGrid:module-TotalDepth.util.plot.XGrid}}\index{TotalDepth.util.plot.XGrid (module)}
Provides plotting support for the X axis grid i.e. Depth grid lines and text.

Created on 28 Feb 2011

TODO: Remove APIs that are not used by Plot or anything. Plot only appears to
use genXAxisRange() and genXAxisTextRange().
\index{ExceptionPlotXGrid}

\begin{fulllineitems}
\phantomsection\label{\detokenize{ref/util/plot/XGrid:TotalDepth.util.plot.XGrid.ExceptionPlotXGrid}}\pysigline{\sphinxbfcode{exception }\sphinxcode{TotalDepth.util.plot.XGrid.}\sphinxbfcode{ExceptionPlotXGrid}}
Exception for plotting.

\end{fulllineitems}

\index{StrokeGreySolid (in module TotalDepth.util.plot.XGrid)}

\begin{fulllineitems}
\phantomsection\label{\detokenize{ref/util/plot/XGrid:TotalDepth.util.plot.XGrid.StrokeGreySolid}}\pysigline{\sphinxcode{TotalDepth.util.plot.XGrid.}\sphinxbfcode{StrokeGreySolid}\sphinxbfcode{ = Stroke(width='1', colour='grey', coding=None, opacity=1.0)}}
Definition of a grey solid stroke
Usage: StrokeBlackGrey.\_replace(width=2.0)

\end{fulllineitems}

\index{XGrid (class in TotalDepth.util.plot.XGrid)}

\begin{fulllineitems}
\phantomsection\label{\detokenize{ref/util/plot/XGrid:TotalDepth.util.plot.XGrid.XGrid}}\pysiglinewithargsret{\sphinxbfcode{class }\sphinxcode{TotalDepth.util.plot.XGrid.}\sphinxbfcode{XGrid}}{\emph{scale}}{}
Class that can generate depth line grid and alphanumeric values.
Constructed with integer scale.
\index{DEFAULT\_INTERVAL\_MAP (TotalDepth.util.plot.XGrid.XGrid attribute)}

\begin{fulllineitems}
\phantomsection\label{\detokenize{ref/util/plot/XGrid:TotalDepth.util.plot.XGrid.XGrid.DEFAULT_INTERVAL_MAP}}\pysigline{\sphinxbfcode{DEFAULT\_INTERVAL\_MAP}\sphinxbfcode{ = \{1: Stroke(width=0.5, colour='black', coding=None, opacity=1.0), 10: Stroke(width=1.0, colour='black', coding=None, opacity=1.0)\}}}
Default for unknown units and scale, this is basically like
simple graph paper

\end{fulllineitems}

\index{DEFAULT\_INTERVAL\_TEXT (TotalDepth.util.plot.XGrid.XGrid attribute)}

\begin{fulllineitems}
\phantomsection\label{\detokenize{ref/util/plot/XGrid:TotalDepth.util.plot.XGrid.XGrid.DEFAULT_INTERVAL_TEXT}}\pysigline{\sphinxbfcode{DEFAULT\_INTERVAL\_TEXT}\sphinxbfcode{ = 100}}
Default position for text on the X axis

\end{fulllineitems}

\index{\_\_init\_\_() (TotalDepth.util.plot.XGrid.XGrid method)}

\begin{fulllineitems}
\phantomsection\label{\detokenize{ref/util/plot/XGrid:TotalDepth.util.plot.XGrid.XGrid.__init__}}\pysiglinewithargsret{\sphinxbfcode{\_\_init\_\_}}{\emph{scale}}{}
Constructor with integer scale. We make scale a constructor argument
as we know that up front. We don’t necessarily know the X units.

\end{fulllineitems}

\index{genXAxisRange() (TotalDepth.util.plot.XGrid.XGrid method)}

\begin{fulllineitems}
\phantomsection\label{\detokenize{ref/util/plot/XGrid:TotalDepth.util.plot.XGrid.XGrid.genXAxisRange}}\pysiglinewithargsret{\sphinxbfcode{genXAxisRange}}{\emph{evFrom}, \emph{evTo}}{}
Generates a bounded series of X axis line plot positions as
(Dim(), Stroke()). evFrom and evTo and EngVal objects.

\end{fulllineitems}

\index{genXPosStroke() (TotalDepth.util.plot.XGrid.XGrid method)}

\begin{fulllineitems}
\phantomsection\label{\detokenize{ref/util/plot/XGrid:TotalDepth.util.plot.XGrid.XGrid.genXPosStroke}}\pysiglinewithargsret{\sphinxbfcode{genXPosStroke}}{\emph{xFrom}, \emph{xInc}, \emph{units}}{}
Generates unbounded series of X line positions as (Dim(), Stroke()).

xFrom - The starting value as a float, positions may not include this
if fractional. First x position generated will be math.ceil()
if xInc, math.floor() otherwise.

xInc - A boolean, True if X increases.

units - Units of X axis e.g. b’FEET’.

\end{fulllineitems}

\index{genEventsUnits() (TotalDepth.util.plot.XGrid.XGrid method)}

\begin{fulllineitems}
\phantomsection\label{\detokenize{ref/util/plot/XGrid:TotalDepth.util.plot.XGrid.XGrid.genEventsUnits}}\pysiglinewithargsret{\sphinxbfcode{genEventsUnits}}{\emph{xFrom}, \emph{xInc}, \emph{units}}{}
Generates events from/to in units.

\end{fulllineitems}

\index{genXAxisTextRange() (TotalDepth.util.plot.XGrid.XGrid method)}

\begin{fulllineitems}
\phantomsection\label{\detokenize{ref/util/plot/XGrid:TotalDepth.util.plot.XGrid.XGrid.genXAxisTextRange}}\pysiglinewithargsret{\sphinxbfcode{genXAxisTextRange}}{\emph{evFrom}, \emph{evTo}}{}
Generates a bounded series of X axis line plot positions as
(Dim(), value). evFrom and evTo and EngVal objects.

\end{fulllineitems}

\index{\_\_weakref\_\_ (TotalDepth.util.plot.XGrid.XGrid attribute)}

\begin{fulllineitems}
\phantomsection\label{\detokenize{ref/util/plot/XGrid:TotalDepth.util.plot.XGrid.XGrid.__weakref__}}\pysigline{\sphinxbfcode{\_\_weakref\_\_}}
list of weak references to the object (if defined)

\end{fulllineitems}


\end{fulllineitems}



\paragraph{XML Plot Description Low-level support}
\label{\detokenize{ref/util/plot/XMLCfg:module-TotalDepth.util.plot.XMLCfg}}\label{\detokenize{ref/util/plot/XMLCfg::doc}}\label{\detokenize{ref/util/plot/XMLCfg:xml-plot-description-low-level-support}}\index{TotalDepth.util.plot.XMLCfg (module)}
Provides low level support for XML plot configurations

Created on Dec 13, 2011
\index{ExceptionXMLCfg}

\begin{fulllineitems}
\phantomsection\label{\detokenize{ref/util/plot/XMLCfg:TotalDepth.util.plot.XMLCfg.ExceptionXMLCfg}}\pysigline{\sphinxbfcode{exception }\sphinxcode{TotalDepth.util.plot.XMLCfg.}\sphinxbfcode{ExceptionXMLCfg}}
Exception class for this module.

\end{fulllineitems}

\index{ExceptionXMLCfgWrongRootElem}

\begin{fulllineitems}
\phantomsection\label{\detokenize{ref/util/plot/XMLCfg:TotalDepth.util.plot.XMLCfg.ExceptionXMLCfgWrongRootElem}}\pysigline{\sphinxbfcode{exception }\sphinxcode{TotalDepth.util.plot.XMLCfg.}\sphinxbfcode{ExceptionXMLCfgWrongRootElem}}
Exception when the root element is wrong.

\end{fulllineitems}

\index{ExceptionXMLCfgMissingElem}

\begin{fulllineitems}
\phantomsection\label{\detokenize{ref/util/plot/XMLCfg:TotalDepth.util.plot.XMLCfg.ExceptionXMLCfgMissingElem}}\pysigline{\sphinxbfcode{exception }\sphinxcode{TotalDepth.util.plot.XMLCfg.}\sphinxbfcode{ExceptionXMLCfgMissingElem}}
Exception when the is a missing mandatory element.

\end{fulllineitems}

\index{ExceptionXMLCfgNoContent}

\begin{fulllineitems}
\phantomsection\label{\detokenize{ref/util/plot/XMLCfg:TotalDepth.util.plot.XMLCfg.ExceptionXMLCfgNoContent}}\pysigline{\sphinxbfcode{exception }\sphinxcode{TotalDepth.util.plot.XMLCfg.}\sphinxbfcode{ExceptionXMLCfgNoContent}}
Exception when the is missing content.

\end{fulllineitems}

\index{LgXMLBase (class in TotalDepth.util.plot.XMLCfg)}

\begin{fulllineitems}
\phantomsection\label{\detokenize{ref/util/plot/XMLCfg:TotalDepth.util.plot.XMLCfg.LgXMLBase}}\pysigline{\sphinxbfcode{class }\sphinxcode{TotalDepth.util.plot.XMLCfg.}\sphinxbfcode{LgXMLBase}}
Base class for XML functionality that can be used by both FILM and PRES XML classes.
\index{str() (TotalDepth.util.plot.XMLCfg.LgXMLBase method)}

\begin{fulllineitems}
\phantomsection\label{\detokenize{ref/util/plot/XMLCfg:TotalDepth.util.plot.XMLCfg.LgXMLBase.str}}\pysiglinewithargsret{\sphinxbfcode{str}}{\emph{e}, \emph{name}, \emph{default=None}}{}
Returns the text in a single child element or None.

\end{fulllineitems}

\index{bool() (TotalDepth.util.plot.XMLCfg.LgXMLBase method)}

\begin{fulllineitems}
\phantomsection\label{\detokenize{ref/util/plot/XMLCfg:TotalDepth.util.plot.XMLCfg.LgXMLBase.bool}}\pysiglinewithargsret{\sphinxbfcode{bool}}{\emph{e}, \emph{name}, \emph{default=False}}{}
Returns the text in a single child element converting 1/0 to True/False.

\end{fulllineitems}

\index{int() (TotalDepth.util.plot.XMLCfg.LgXMLBase method)}

\begin{fulllineitems}
\phantomsection\label{\detokenize{ref/util/plot/XMLCfg:TotalDepth.util.plot.XMLCfg.LgXMLBase.int}}\pysiglinewithargsret{\sphinxbfcode{int}}{\emph{e}, \emph{name}, \emph{default=0}}{}
Returns the text in a single child element as an integer defaulting to 0.

\end{fulllineitems}

\index{float() (TotalDepth.util.plot.XMLCfg.LgXMLBase method)}

\begin{fulllineitems}
\phantomsection\label{\detokenize{ref/util/plot/XMLCfg:TotalDepth.util.plot.XMLCfg.LgXMLBase.float}}\pysiglinewithargsret{\sphinxbfcode{float}}{\emph{e}, \emph{name}, \emph{default=0.0}}{}
Returns the text in a single child element as a float.

\end{fulllineitems}


\end{fulllineitems}



\paragraph{XML Plot Description Discovery}
\label{\detokenize{ref/util/plot/XMLMatches:xml-plot-description-discovery}}\label{\detokenize{ref/util/plot/XMLMatches::doc}}\label{\detokenize{ref/util/plot/XMLMatches:module-TotalDepth.util.plot.XMLMatches}}\index{TotalDepth.util.plot.XMLMatches (module)}
This looks at the available XML LgFormats and works out what curves of a LogPass
or LASFile can be plotted by each one.

Created on Jan 21, 2012
\index{fileCurveMap() (in module TotalDepth.util.plot.XMLMatches)}

\begin{fulllineitems}
\phantomsection\label{\detokenize{ref/util/plot/XMLMatches:TotalDepth.util.plot.XMLMatches.fileCurveMap}}\pysiglinewithargsret{\sphinxcode{TotalDepth.util.plot.XMLMatches.}\sphinxbfcode{fileCurveMap}}{\emph{theLpOrLasFile}, \emph{directory=None}}{}
Returns a map of \sphinxcode{\{FilmID : {[}OUTP, ...{]}, ...\}} which is a list of OUTP in theLpOrLasFile
that could be plotted with that film ID.

\end{fulllineitems}

\index{fileCurveMapFromFILM() (in module TotalDepth.util.plot.XMLMatches)}

\begin{fulllineitems}
\phantomsection\label{\detokenize{ref/util/plot/XMLMatches:TotalDepth.util.plot.XMLMatches.fileCurveMapFromFILM}}\pysiglinewithargsret{\sphinxcode{TotalDepth.util.plot.XMLMatches.}\sphinxbfcode{fileCurveMapFromFILM}}{\emph{theLpOrLasFile}, \emph{theFilmCfg}}{}
Returns a map of \sphinxcode{\{FilmID : {[}OUTP, ...{]}, ...\}} which is a list of OUTP in theLpOrLasFile
that could be plotted with that film ID.

\end{fulllineitems}



\paragraph{AREA Plotting Configuration}
\label{\detokenize{ref/util/plot/AREACfg:area-plotting-configuration}}\label{\detokenize{ref/util/plot/AREACfg::doc}}\label{\detokenize{ref/util/plot/AREACfg:module-TotalDepth.util.plot.AREACfg}}\index{TotalDepth.util.plot.AREACfg (module)}
Module for plotting areas of particular patterns wirth well log data.

Created on 1 Apr 2011

Note on patterns in XML file in: \sphinxcode{formats\textbackslash{}IWWPatterns.xml}

There are elements thus:

\begin{sphinxVerbatim}[commandchars=\\\{\}]
\PYG{o}{\PYGZlt{}}\PYG{n}{Bits}\PYG{o}{\PYGZgt{}}\PYG{o}{/}\PYG{o}{+}\PYG{l+m+mi}{7}\PYG{o}{/}\PYG{l+m+mi}{7}\PYG{n}{v}\PYG{o}{/}\PYG{n}{u}\PYG{o}{/}\PYG{o}{+}\PYG{l+m+mi}{7}\PYG{o}{/}\PYG{l+m+mi}{7}\PYG{n}{gAA7}\PYG{o}{/}\PYG{l+m+mi}{7}\PYG{n}{v}\PYG{o}{/}\PYG{n}{u}\PYG{o}{/}\PYG{o}{+}\PYG{l+m+mi}{7}\PYG{o}{/}\PYG{l+m+mi}{7}\PYG{n}{v}\PYG{o}{/}\PYG{n}{gAA}\PYG{o}{\PYGZlt{}}\PYG{o}{/}\PYG{n}{Bits}\PYG{o}{\PYGZgt{}}
\PYG{o}{\PYGZlt{}}\PYG{n}{PatternHeight}\PYG{o}{\PYGZgt{}}\PYG{l+m+mi}{12}\PYG{o}{\PYGZlt{}}\PYG{o}{/}\PYG{n}{PatternHeight}\PYG{o}{\PYGZgt{}}
\PYG{o}{\PYGZlt{}}\PYG{n}{PatternWidth}\PYG{o}{\PYGZgt{}}\PYG{l+m+mi}{15}\PYG{o}{\PYGZlt{}}\PYG{o}{/}\PYG{n}{PatternWidth}\PYG{o}{\PYGZgt{}}
\end{sphinxVerbatim}

Frequency analysis shows that the characters used are ‘+’, ‘/’, A-Z, a-z, 0-9
so this looks like base64 encoded. Length is always 32 chars (256 bits).
Pattern size suggests 12*15=180 bits.

Examples:

\begin{sphinxVerbatim}[commandchars=\\\{\}]
\PYG{g+gp}{\PYGZgt{}\PYGZgt{}\PYGZgt{} }\PYG{n}{base64}\PYG{o}{.}\PYG{n}{b64decode}\PYG{p}{(}\PYG{l+s+sa}{b}\PYG{l+s+s1}{\PYGZsq{}}\PYG{l+s+s1}{/+7/7v/u/+7/7gAA7/7v/u/+7/7v/gAA}\PYG{l+s+s1}{\PYGZsq{}}\PYG{p}{)}
\PYG{g+go}{b\PYGZsq{}ÿîÿîÿîÿîÿî\PYGZbs{}\PYGZbs{}ïþïþïþïþïþ\PYGZbs{}\PYGZbs{}\PYGZsq{}}

\PYG{g+go}{\PYGZlt{}Description\PYGZgt{}Coal\PYGZhy{}LigNite\PYGZlt{}/Description\PYGZgt{}}
\PYG{g+gp}{\PYGZgt{}\PYGZgt{}\PYGZgt{} }\PYG{n}{base64}\PYG{o}{.}\PYG{n}{b64decode}\PYG{p}{(}\PYG{l+s+sa}{b}\PYG{l+s+s1}{\PYGZsq{}}\PYG{l+s+s1}{AAAAAAAAAAAAAAAAAAAAAAAAAAAAAAAA}\PYG{l+s+s1}{\PYGZsq{}}\PYG{p}{)}
\PYG{g+go}{b\PYGZsq{}\PYGZbs{}\PYGZbs{}\PYGZbs{}\PYGZbs{}\PYGZbs{}\PYGZbs{}\PYGZbs{}\PYGZbs{}\PYGZbs{}\PYGZbs{}\PYGZbs{}\PYGZbs{}\PYGZbs{}\PYGZbs{}\PYGZbs{}\PYGZbs{}\PYGZbs{}\PYGZbs{}\PYGZbs{}\PYGZbs{}\PYGZbs{}\PYGZbs{}\PYGZbs{}\PYGZbs{}\PYGZsq{}}

\PYG{g+go}{\PYGZlt{}Description\PYGZgt{}Void\PYGZlt{}/Description\PYGZgt{}}
\PYG{g+gp}{\PYGZgt{}\PYGZgt{}\PYGZgt{} }\PYG{n}{base64}\PYG{o}{.}\PYG{n}{b64decode}\PYG{p}{(}\PYG{l+s+sa}{b}\PYG{l+s+s1}{\PYGZsq{}}\PYG{l+s+s1}{//7//v/+//7//v/+//7//v/+//7//v/+}\PYG{l+s+s1}{\PYGZsq{}}\PYG{p}{)}
\PYG{g+go}{b\PYGZsq{}ÿþÿþÿþÿþÿþÿþÿþÿþÿþÿþÿþÿþ\PYGZsq{}}

\PYG{g+gp}{\PYGZgt{}\PYGZgt{}\PYGZgt{} }\PYG{n}{base64}\PYG{o}{.}\PYG{n}{b64decode}\PYG{p}{(}\PYG{l+s+sa}{b}\PYG{l+s+s1}{\PYGZsq{}}\PYG{l+s+s1}{2Hzfkrnudnzvst3O2n67gmZ83Z7V5iYa}\PYG{l+s+s1}{\PYGZsq{}}\PYG{p}{)}
\PYG{g+go}{b\PYGZsq{}Ø\textbar{}ߒ\(\sp{\text{1}}\)îv\textbar{}ï\(\sp{\text{2}}\)ÝÎÚ\PYGZti{}»‚f\textbar{}ݞÕæ\PYGZam{}\PYGZsq{}}
\end{sphinxVerbatim}

Length is always 24 bytes (192 bits) 12x16 ?
This looks promising:

\begin{sphinxVerbatim}[commandchars=\\\{\}]
\PYG{g+gp}{\PYGZgt{}\PYGZgt{}\PYGZgt{} }\PYG{n}{l} \PYG{o}{=} \PYG{n+nb}{list}\PYG{p}{(}\PYG{l+s+sa}{b}\PYG{l+s+s1}{\PYGZsq{}}\PYG{l+s+s1}{Ø\textbar{}ߒ\(\sp{\text{1}}\)îv\textbar{}ï\(\sp{\text{2}}\)ÝÎÚ\PYGZti{}»‚f\textbar{}ݞÕæ\PYGZam{}}\PYG{l+s+s1}{\PYGZsq{}}\PYG{p}{)}
\PYG{g+gp}{\PYGZgt{}\PYGZgt{}\PYGZgt{} }\PYG{n}{s} \PYG{o}{=} \PYG{p}{[}\PYG{l+s+s1}{\PYGZsq{}}\PYG{l+s+si}{\PYGZob{}:08b\PYGZcb{}}\PYG{l+s+s1}{\PYGZsq{}}\PYG{o}{.}\PYG{n}{format}\PYG{p}{(}\PYG{n}{v}\PYG{p}{)} \PYG{k}{for} \PYG{n}{v} \PYG{o+ow}{in} \PYG{n}{l}\PYG{p}{]}
\PYG{g+gp}{\PYGZgt{}\PYGZgt{}\PYGZgt{} }\PYG{n}{st} \PYG{o}{=} \PYG{l+s+s1}{\PYGZsq{}}\PYG{l+s+s1}{\PYGZsq{}}\PYG{o}{.}\PYG{n}{join}\PYG{p}{(}\PYG{n}{s}\PYG{p}{)}
\PYG{g+gp}{\PYGZgt{}\PYGZgt{}\PYGZgt{} }\PYG{n}{patS} \PYG{o}{=} \PYG{p}{[}\PYG{n}{st}\PYG{p}{[}\PYG{n}{i}\PYG{p}{:}\PYG{n}{i}\PYG{o}{+}\PYG{l+m+mi}{12}\PYG{p}{]} \PYG{k}{for} \PYG{n}{i} \PYG{o+ow}{in} \PYG{n+nb}{range}\PYG{p}{(}\PYG{l+m+mi}{16}\PYG{p}{)}\PYG{p}{]}
\PYG{g+gp}{\PYGZgt{}\PYGZgt{}\PYGZgt{} }\PYG{n+nb}{print}\PYG{p}{(}\PYG{l+s+s1}{\PYGZsq{}}\PYG{l+s+se}{\PYGZbs{}n}\PYG{l+s+s1}{\PYGZsq{}}\PYG{o}{.}\PYG{n}{join}\PYG{p}{(}\PYG{n}{patS}\PYG{p}{)}\PYG{p}{)}
\PYG{g+go}{110110000111}
\PYG{g+go}{101100001111}
\PYG{g+go}{011000011111}
\PYG{g+go}{110000111110}
\PYG{g+go}{100001111100}
\PYG{g+go}{000011111001}
\PYG{g+go}{000111110011}
\PYG{g+go}{001111100110}
\PYG{g+go}{011111001101}
\PYG{g+go}{111110011011}
\PYG{g+go}{111100110111}
\PYG{g+go}{111001101111}
\PYG{g+go}{110011011111}
\PYG{g+go}{100110111111}
\PYG{g+go}{001101111110}
\PYG{g+go}{011011111100}
\end{sphinxVerbatim}

Or shortened:

\begin{sphinxVerbatim}[commandchars=\\\{\}]
\PYG{n}{st} \PYG{o}{=} \PYG{l+s+s1}{\PYGZsq{}}\PYG{l+s+s1}{\PYGZsq{}}\PYG{o}{.}\PYG{n}{join}\PYG{p}{(}\PYG{p}{[}\PYG{l+s+s1}{\PYGZsq{}}\PYG{l+s+si}{\PYGZob{}:08b\PYGZcb{}}\PYG{l+s+s1}{\PYGZsq{}}\PYG{o}{.}\PYG{n}{format}\PYG{p}{(}\PYG{n}{v}\PYG{p}{)} \PYG{k}{for} \PYG{n}{v} \PYG{o+ow}{in} \PYG{l+s+sa}{b}\PYG{l+s+s1}{\PYGZsq{}}\PYG{l+s+s1}{...}\PYG{l+s+s1}{\PYGZsq{}}\PYG{p}{]}\PYG{p}{)}
\PYG{n+nb}{print}\PYG{p}{(}\PYG{l+s+s1}{\PYGZsq{}}\PYG{l+s+se}{\PYGZbs{}n}\PYG{l+s+s1}{\PYGZsq{}}\PYG{o}{.}\PYG{n}{join}\PYG{p}{(}\PYG{p}{[}\PYG{n}{st}\PYG{p}{[}\PYG{n}{i}\PYG{p}{:}\PYG{n}{i}\PYG{o}{+}\PYG{l+m+mi}{12}\PYG{p}{]} \PYG{k}{for} \PYG{n}{i} \PYG{o+ow}{in} \PYG{n+nb}{range}\PYG{p}{(}\PYG{l+m+mi}{16}\PYG{p}{)}\PYG{p}{]}\PYG{p}{)}\PYG{p}{)}

\PYG{o}{\PYGZlt{}}\PYG{n}{Description}\PYG{o}{\PYGZgt{}}\PYG{n}{Void}\PYG{o}{\PYGZlt{}}\PYG{o}{/}\PYG{n}{Description}\PYG{o}{\PYGZgt{}}
\PYG{o}{\PYGZgt{}\PYGZgt{}}\PYG{o}{\PYGZgt{}} \PYG{n}{base64}\PYG{o}{.}\PYG{n}{b64decode}\PYG{p}{(}\PYG{l+s+sa}{b}\PYG{l+s+s1}{\PYGZsq{}}\PYG{l+s+s1}{//7//v/+//7//v/+//7//v/+//7//v/+}\PYG{l+s+s1}{\PYGZsq{}}\PYG{p}{)}
\PYG{l+s+sa}{b}\PYG{l+s+s1}{\PYGZsq{}}\PYG{l+s+s1}{ÿþÿþÿþÿþÿþÿþÿþÿþÿþÿþÿþÿþ}\PYG{l+s+s1}{\PYGZsq{}}
\PYG{l+m+mi}{111111111111}
\PYG{l+m+mi}{111111111111}
\PYG{l+m+mi}{111111111111}
\PYG{l+m+mi}{111111111111}
\PYG{l+m+mi}{111111111110}
\PYG{l+m+mi}{111111111101}
\PYG{l+m+mi}{111111111011}
\PYG{l+m+mi}{111111110111}
\PYG{l+m+mi}{111111101111}
\PYG{l+m+mi}{111111011111}
\PYG{l+m+mi}{111110111111}
\PYG{l+m+mi}{111101111111}
\PYG{l+m+mi}{111011111111}
\PYG{l+m+mi}{110111111111}
\PYG{l+m+mi}{101111111111}
\PYG{l+m+mi}{011111111111}
\end{sphinxVerbatim}

Hmmm 12 zeros, perhaps it is 12x15 after all:

\begin{sphinxVerbatim}[commandchars=\\\{\}]
\PYG{g+gp}{\PYGZgt{}\PYGZgt{}\PYGZgt{} }\PYG{n}{base64}\PYG{o}{.}\PYG{n}{b64decode}\PYG{p}{(}\PYG{l+s+sa}{b}\PYG{l+s+s1}{\PYGZsq{}}\PYG{l+s+s1}{/+7/7v/u/+7/7gAA7/7v/u/+7/7v/gAA}\PYG{l+s+s1}{\PYGZsq{}}\PYG{p}{)}
\PYG{g+go}{b\PYGZsq{}ÿîÿîÿîÿîÿî\PYGZbs{}\PYGZbs{}ïþïþïþïþïþ\PYGZbs{}\PYGZbs{}\PYGZsq{}}
\PYG{g+go}{111111111110}
\PYG{g+go}{111111111101}
\PYG{g+go}{111111111011}
\PYG{g+go}{111111110111}
\PYG{g+go}{111111101110}
\PYG{g+go}{111111011101}
\PYG{g+go}{111110111011}
\PYG{g+go}{111101110111}
\PYG{g+go}{111011101111}
\PYG{g+go}{110111011111}
\PYG{g+go}{101110111111}
\PYG{g+go}{011101111111}
\PYG{g+go}{111011111111}
\PYG{g+go}{110111111111}
\PYG{g+go}{101111111111}
\PYG{g+go}{011111111111}
\end{sphinxVerbatim}

Full range (sorted):

\begin{sphinxVerbatim}[commandchars=\\\{\}]
\PYG{o}{/}\PYG{o}{/}\PYG{l+m+mi}{4}\PYG{n}{zAP}\PYG{o}{/}\PYG{l+m+mi}{6}\PYG{o}{/}\PYG{o}{/}\PYG{n}{oA9v}\PYG{o}{/}\PYG{o}{+}\PYG{o}{/}\PYG{o}{/}\PYG{l+m+mi}{4}\PYG{n}{zAP}\PYG{o}{/}\PYG{l+m+mi}{8}\PYG{o}{/}\PYG{o}{/}\PYG{n}{oA}\PYG{o}{+}\PYG{n}{v}\PYG{o}{/}\PYG{l+m+mi}{2}
\PYG{o}{/}\PYG{o}{/}\PYG{l+m+mi}{6}\PYG{o}{/}\PYG{n}{AIP}\PYG{o}{+}\PYG{o}{/}\PYG{o}{/}\PYG{l+m+mi}{4}\PYG{n}{Avv}\PYG{o}{+}\PYG{n}{C}\PYG{o}{/}\PYG{o}{/}\PYG{l+m+mi}{6}\PYG{o}{/}\PYG{n}{AIP}\PYG{o}{+}\PYG{o}{/}\PYG{o}{/}\PYG{l+m+mi}{4}\PYG{n}{Avv}\PYG{o}{+}\PYG{n}{C}
\PYG{o}{/}\PYG{o}{/}\PYG{l+m+mi}{60}\PYG{n}{ks}\PYG{o}{/}\PYG{o}{+}\PYG{o}{/}\PYG{o}{/}\PYG{l+m+mi}{6}\PYG{n}{ZaP}\PYG{o}{+}\PYG{n}{e}\PYG{o}{/}\PYG{o}{/}\PYG{l+m+mi}{60}\PYG{n}{ks}\PYG{o}{/}\PYG{o}{+}\PYG{o}{/}\PYG{o}{/}\PYG{l+m+mi}{6}\PYG{n}{ZaP}\PYG{o}{+}\PYG{n}{e}
\PYG{o}{/}\PYG{o}{/}\PYG{l+m+mi}{69}\PYG{n}{fIEA}\PYG{o}{/}\PYG{o}{/}\PYG{l+m+mi}{4}\PYG{n}{QENfW}\PYG{o}{/}\PYG{o}{/}\PYG{l+m+mi}{69}\PYG{n}{fIEA}\PYG{o}{/}\PYG{o}{/}\PYG{l+m+mi}{4}\PYG{n}{QENf2}
\PYG{o}{/}\PYG{o}{/}\PYG{l+m+mi}{6}\PYG{n}{N5r}\PYG{o}{/}\PYG{n}{mj76}\PYG{o}{/}\PYG{n}{urm}\PYG{o}{+}\PYG{n}{ub7v8u7}\PYG{o}{+}\PYG{l+m+mi}{774}\PYG{n}{t8v}\PYG{o}{/}\PYG{n}{y}
\PYG{o}{/}\PYG{o}{/}\PYG{l+m+mi}{6}\PYG{n}{v1vt}\PYG{o}{+}\PYG{n}{rOr}\PYG{o}{/}\PYG{o}{/}\PYG{n}{lq2}\PYG{o}{/}\PYG{l+m+mi}{87}\PYG{o}{/}\PYG{o}{/}\PYG{n}{rrq}\PYG{o}{/}\PYG{o}{/}\PYG{l+m+mi}{7}\PYG{n}{XXu}\PYG{o}{+}\PYG{o}{+}
\PYG{o}{/}\PYG{o}{/}\PYG{l+m+mi}{7}\PYG{o}{/}\PYG{o}{/}\PYG{n}{v}\PYG{o}{/}\PYG{o}{+}\PYG{o}{/}\PYG{o}{/}\PYG{l+m+mi}{7}\PYG{o}{/}\PYG{o}{/}\PYG{n}{v}\PYG{o}{/}\PYG{o}{+}\PYG{o}{/}\PYG{o}{/}\PYG{l+m+mi}{7}\PYG{o}{/}\PYG{o}{/}\PYG{n}{v}\PYG{o}{/}\PYG{o}{+}\PYG{o}{/}\PYG{o}{/}\PYG{l+m+mi}{7}\PYG{o}{/}\PYG{o}{/}\PYG{n}{v}\PYG{o}{/}\PYG{o}{+}
\PYG{o}{/}\PYG{o}{/}\PYG{l+m+mi}{7}\PYG{o}{/}\PYG{l+m+mi}{3}\PYG{n}{t}\PYG{o}{/}\PYG{n}{e397fwsP}\PYG{o}{+}\PYG{o}{/}\PYG{o}{/}\PYG{l+m+mi}{7}\PYG{o}{/}\PYG{l+m+mi}{3}\PYG{n}{t}\PYG{o}{/}\PYG{n}{e397fwsP}\PYG{o}{+}
\PYG{o}{/}\PYG{o}{/}\PYG{l+m+mi}{7}\PYG{o}{/}\PYG{n}{AP}\PYG{o}{/}\PYG{o}{+}\PYG{o}{/}\PYG{o}{/}\PYG{l+m+mi}{4}\PYG{n}{A}\PYG{o}{/}\PYG{n}{v}\PYG{o}{/}\PYG{o}{+}\PYG{o}{/}\PYG{o}{/}\PYG{l+m+mi}{7}\PYG{o}{/}\PYG{n}{AP}\PYG{o}{/}\PYG{o}{+}\PYG{o}{/}\PYG{o}{/}\PYG{l+m+mi}{4}\PYG{n}{A}\PYG{o}{/}\PYG{n}{v}\PYG{o}{/}\PYG{o}{+}
\PYG{o}{/}\PYG{o}{/}\PYG{l+m+mi}{7}\PYG{o}{/}\PYG{n}{AP}\PYG{o}{/}\PYG{o}{+}\PYG{o}{/}\PYG{l+m+mi}{9}\PYG{n}{YA7v}\PYG{o}{/}\PYG{n}{W1}\PYG{o}{/}\PYG{l+m+mi}{7}\PYG{n}{vANf}\PYG{o}{+}\PYG{o}{/}\PYG{l+m+mi}{9}\PYG{n}{YA7v}\PYG{o}{/}\PYG{n}{W}
\PYG{o}{/}\PYG{o}{/}\PYG{l+m+mi}{7}\PYG{o}{/}\PYG{n}{AP}\PYG{o}{/}\PYG{o}{+}\PYG{n}{Wv4A}\PYG{o}{/}\PYG{n}{v}\PYG{o}{/}\PYG{o}{+}\PYG{o}{/}\PYG{o}{/}\PYG{l+m+mi}{7}\PYG{o}{/}\PYG{n}{AP}\PYG{o}{/}\PYG{o}{+}\PYG{n}{Wv4A}\PYG{o}{/}\PYG{n}{v}\PYG{o}{/}\PYG{o}{+}
\PYG{o}{/}\PYG{o}{/}\PYG{l+m+mi}{7}\PYG{o}{/}\PYG{n}{APv}\PYG{o}{+}\PYG{l+m+mi}{9}\PYG{o}{/}\PYG{l+m+mi}{4}\PYG{n}{A}\PYG{o}{/}\PYG{n}{v}\PYG{o}{/}\PYG{l+m+mi}{6}\PYG{o}{/}\PYG{o}{/}\PYG{n}{b}\PYG{o}{/}\PYG{n}{AP3}\PYG{o}{+}\PYG{l+m+mi}{8}\PYG{o}{/}\PYG{l+m+mi}{4}\PYG{n}{A}\PYG{o}{/}\PYG{n}{v}\PYG{o}{/}\PYG{o}{+}
\PYG{o}{/}\PYG{o}{/}\PYG{l+m+mi}{73}\PYG{o}{/}\PYG{n}{uv}\PYG{o}{+}\PYG{n}{wf7}\PYG{o}{/}\PYG{o}{/}\PYG{n}{v}\PYG{o}{/}\PYG{o}{+}\PYG{o}{/}\PYG{o}{/}\PYG{l+m+mi}{7}\PYG{o}{/}\PYG{l+m+mi}{3}\PYG{n}{v}\PYG{o}{+}\PYG{n}{u}\PYG{o}{/}\PYG{n}{wb}\PYG{o}{/}\PYG{o}{/}\PYG{n}{v}\PYG{o}{/}\PYG{o}{+}
\PYG{o}{/}\PYG{o}{/}\PYG{l+m+mi}{732}\PYG{n}{gAAvfbvfAAAu97vegAAve73fAAA}
\PYG{o}{/}\PYG{o}{/}\PYG{l+m+mi}{7}\PYG{n}{bAP}\PYG{o}{/}\PYG{o}{+}\PYG{o}{/}\PYG{o}{/}\PYG{l+m+mi}{4}\PYG{n}{A1v}\PYG{o}{/}\PYG{o}{+}\PYG{o}{/}\PYG{o}{/}\PYG{l+m+mi}{7}\PYG{n}{bAP}\PYG{o}{/}\PYG{o}{+}\PYG{o}{/}\PYG{o}{/}\PYG{l+m+mi}{4}\PYG{n}{A1v}\PYG{o}{/}\PYG{o}{+}
\PYG{o}{/}\PYG{o}{/}\PYG{l+m+mi}{7}\PYG{n}{bAPv}\PYG{o}{+}\PYG{l+m+mi}{9}\PYG{o}{/}\PYG{l+m+mi}{4}\PYG{n}{A1v}\PYG{o}{/}\PYG{l+m+mi}{6}\PYG{o}{/}\PYG{o}{/}\PYG{n}{bbAP3}\PYG{o}{+}\PYG{l+m+mi}{8}\PYG{o}{/}\PYG{l+m+mi}{4}\PYG{n}{A2v}\PYG{o}{/}\PYG{o}{+}
\PYG{o}{/}\PYG{o}{/}\PYG{l+m+mi}{7}\PYG{n}{HHoLugu7HHv}\PYG{o}{/}\PYG{o}{+}\PYG{o}{/}\PYG{o}{/}\PYG{l+m+mi}{6}\PYG{n}{PxneCd4KPxv}\PYG{o}{/}\PYG{o}{+}
\PYG{o}{/}\PYG{o}{/}\PYG{l+m+mi}{7}\PYG{n}{HHrruuu7HHv}\PYG{o}{/}\PYG{o}{+}\PYG{o}{/}\PYG{o}{/}\PYG{l+m+mi}{6}\PYG{n}{Pxne6d7qPxv}\PYG{o}{/}\PYG{o}{+}
\PYG{o}{/}\PYG{o}{/}\PYG{l+m+mi}{7}\PYG{n}{nAP}\PYG{o}{/}\PYG{o}{+}\PYG{o}{/}\PYG{o}{+}\PYG{l+m+mi}{4}\PYG{n}{A7v}\PYG{o}{/}\PYG{n}{u}\PYG{o}{/}\PYG{o}{/}\PYG{l+m+mi}{7}\PYG{n}{nAP}\PYG{o}{/}\PYG{o}{+}\PYG{o}{/}\PYG{o}{+}\PYG{l+m+mi}{4}\PYG{n}{A7v}\PYG{o}{/}\PYG{n}{u}
\PYG{o}{/}\PYG{o}{/}\PYG{l+m+mi}{7}\PYG{n}{nOOc4}\PYG{o}{/}\PYG{o}{/}\PYG{l+m+mi}{45}\PYG{n}{zjnO}\PYG{o}{/}\PYG{o}{/}\PYG{l+m+mi}{7}\PYG{n}{nOOc4}\PYG{o}{/}\PYG{o}{/}\PYG{l+m+mi}{45}\PYG{n}{zjnO}
\PYG{o}{/}\PYG{o}{/}\PYG{l+m+mi}{7}\PYG{n}{nOOc4}\PYG{o}{/}\PYG{o}{/}\PYG{l+m+mi}{45}\PYG{n}{zv}\PYG{o}{/}\PYG{n}{O}\PYG{o}{/}\PYG{o}{/}\PYG{l+m+mi}{7}\PYG{n}{nOP8}\PYG{o}{+}\PYG{o}{/}\PYG{o}{/}\PYG{l+m+mi}{45}\PYG{n}{zvn}\PYG{o}{+}
\PYG{o}{/}\PYG{o}{/}\PYG{l+m+mi}{7}\PYG{n}{vAP}\PYG{o}{/}\PYG{o}{+}\PYG{o}{/}\PYG{o}{/}\PYG{l+m+mi}{4}\PYG{n}{A7v}\PYG{o}{/}\PYG{o}{+}\PYG{o}{/}\PYG{o}{/}\PYG{l+m+mi}{7}\PYG{n}{vAP}\PYG{o}{/}\PYG{o}{+}\PYG{o}{/}\PYG{o}{/}\PYG{l+m+mi}{4}\PYG{n}{A7v}\PYG{o}{/}\PYG{o}{+}
\PYG{o}{/}\PYG{o}{/}\PYG{n}{b}\PYG{o}{/}\PYG{l+m+mi}{9}\PYG{n}{v}\PYG{o}{/}\PYG{l+m+mi}{2}\PYG{o}{+}\PYG{l+m+mi}{7}\PYG{n}{b5Ngqg}\PYG{o}{+}\PYG{l+m+mi}{777}\PYG{n}{vv}\PYG{o}{/}\PYG{o}{+}\PYG{l+m+mi}{9}\PYG{o}{/}\PYG{l+m+mi}{73}\PYG{o}{/}\PYG{n}{gAA}
\PYG{o}{/}\PYG{o}{/}\PYG{n}{b}\PYG{o}{/}\PYG{l+m+mi}{9}\PYG{n}{v}\PYG{o}{/}\PYG{l+m+mi}{2}\PYG{o}{+}\PYG{l+m+mi}{7}\PYG{n}{b7tgqg}\PYG{o}{+}\PYG{n}{T77vv}\PYG{o}{/}\PYG{o}{+}\PYG{l+m+mi}{9}\PYG{o}{/}\PYG{l+m+mi}{73}\PYG{o}{/}\PYG{n}{gAA}
\PYG{o}{/}\PYG{o}{/}\PYG{n}{b}\PYG{o}{/}\PYG{l+m+mi}{9}\PYG{n}{v}\PYG{o}{/}\PYG{l+m+mi}{2}\PYG{o}{+}\PYG{n}{Db79gog}\PYG{o}{+}\PYG{l+m+mi}{774}\PYG{n}{Pv}\PYG{o}{/}\PYG{o}{+}\PYG{l+m+mi}{9}\PYG{o}{/}\PYG{l+m+mi}{73}\PYG{o}{/}\PYG{n}{gAA}
\PYG{o}{/}\PYG{o}{/}\PYG{n}{b}\PYG{o}{/}\PYG{l+m+mi}{9}\PYG{n}{v}\PYG{o}{/}\PYG{l+m+mi}{2}\PYG{o}{+}\PYG{n}{Hb7tghg}\PYG{o}{+}\PYG{l+m+mi}{774}\PYG{n}{fv}\PYG{o}{/}\PYG{o}{+}\PYG{l+m+mi}{9}\PYG{o}{/}\PYG{l+m+mi}{73}\PYG{o}{/}\PYG{n}{gAA}
\PYG{o}{/}\PYG{o}{/}\PYG{n}{b}\PYG{o}{/}\PYG{l+m+mi}{9}\PYG{n}{v}\PYG{o}{/}\PYG{l+m+mi}{2}\PYG{o}{+}\PYG{n}{Pb7dgjA}\PYG{o}{+}\PYG{o}{/}\PYG{l+m+mi}{77}\PYG{o}{/}\PYG{n}{v}\PYG{o}{/}\PYG{o}{+}\PYG{l+m+mi}{9}\PYG{o}{/}\PYG{l+m+mi}{73}\PYG{o}{/}\PYG{n}{gAA}
\PYG{o}{/}\PYG{o}{+}\PYG{l+m+mi}{4}\PYG{n}{zLv}\PYG{o}{/}\PYG{n}{uzOj}\PYG{o}{/}\PYG{l+m+mi}{7}\PYG{n}{gcA3}\PYG{o}{/}\PYG{l+m+mi}{7}\PYG{n}{TMt}\PYG{o}{/}\PYG{o}{+}\PYG{n}{XMzf}\PYG{o}{/}\PYG{n}{gBy}
\PYG{o}{/}\PYG{o}{+}\PYG{l+m+mi}{6}\PYG{n}{B7n7uge7}\PYG{o}{/}\PYG{l+m+mi}{7}\PYG{n}{gAA7}\PYG{o}{/}\PYG{l+m+mi}{7}\PYG{n}{vAu787wLv}\PYG{o}{/}\PYG{n}{gAA}
\PYG{o}{/}\PYG{o}{+}\PYG{l+m+mi}{7}\PYG{o}{/}\PYG{l+m+mi}{7}\PYG{n}{rvux}\PYG{o}{+}\PYG{l+m+mi}{7}\PYG{o}{/}\PYG{l+m+mi}{7}\PYG{n}{gAA7}\PYG{o}{/}\PYG{l+m+mi}{7}\PYG{n}{v}\PYG{o}{/}\PYG{n}{u}\PYG{o}{+}\PYG{l+m+mi}{678}\PYG{n}{bv}\PYG{o}{/}\PYG{n}{gAA}
\PYG{o}{/}\PYG{o}{+}\PYG{l+m+mi}{7}\PYG{o}{/}\PYG{l+m+mi}{7}\PYG{n}{sHu}\PYG{o}{/}\PYG{o}{+}\PYG{l+m+mi}{7}\PYG{o}{/}\PYG{l+m+mi}{7}\PYG{n}{gAA797vvu}\PYG{o}{/}\PYG{o}{+}\PYG{l+m+mi}{7}\PYG{o}{/}\PYG{l+m+mi}{7}\PYG{n}{v}\PYG{o}{/}\PYG{n}{gAA}
\PYG{o}{/}\PYG{o}{+}\PYG{l+m+mi}{7}\PYG{o}{/}\PYG{l+m+mi}{7}\PYG{n}{uDu}\PYG{o}{/}\PYG{o}{+}\PYG{l+m+mi}{7}\PYG{o}{/}\PYG{l+m+mi}{7}\PYG{n}{gAA7}\PYG{o}{/}\PYG{l+m+mi}{7}\PYG{n}{v}\PYG{o}{/}\PYG{n}{u8G7}\PYG{o}{/}\PYG{l+m+mi}{7}\PYG{n}{v}\PYG{o}{/}\PYG{n}{gAA}
\PYG{o}{/}\PYG{o}{+}\PYG{l+m+mi}{7}\PYG{o}{/}\PYG{l+m+mi}{7}\PYG{n}{v}\PYG{o}{/}\PYG{n}{u}\PYG{o}{/}\PYG{o}{+}\PYG{l+m+mi}{7}\PYG{o}{/}\PYG{l+m+mi}{7}\PYG{n}{gAA7}\PYG{o}{/}\PYG{l+m+mi}{7}\PYG{n}{v}\PYG{o}{/}\PYG{n}{u}\PYG{o}{/}\PYG{o}{+}\PYG{l+m+mi}{7}\PYG{o}{/}\PYG{l+m+mi}{7}\PYG{n}{v}\PYG{o}{/}\PYG{n}{gAA}
\PYG{o}{/}\PYG{o}{+}\PYG{l+m+mi}{7}\PYG{o}{/}\PYG{l+m+mi}{7}\PYG{n}{v}\PYG{o}{/}\PYG{n}{u}\PYG{o}{/}\PYG{o}{+}\PYG{l+m+mi}{7}\PYG{o}{/}\PYG{l+m+mi}{7}\PYG{n}{oOC7}\PYG{o}{/}\PYG{l+m+mi}{7}\PYG{n}{v}\PYG{o}{/}\PYG{n}{u}\PYG{o}{/}\PYG{o}{+}\PYG{l+m+mi}{7}\PYG{o}{/}\PYG{l+m+mi}{7}\PYG{n}{v}\PYG{o}{/}\PYG{n}{oOC}
\PYG{o}{/}\PYG{o}{+}\PYG{l+m+mi}{7}\PYG{o}{/}\PYG{l+m+mi}{7}\PYG{n}{v}\PYG{o}{/}\PYG{n}{u}\PYG{o}{/}\PYG{o}{+}\PYG{l+m+mi}{7}\PYG{n}{bbAAA7}\PYG{o}{/}\PYG{l+m+mi}{7}\PYG{n}{v}\PYG{o}{/}\PYG{n}{u}\PYG{o}{/}\PYG{o}{+}\PYG{l+m+mi}{7}\PYG{o}{/}\PYG{l+m+mi}{5}\PYG{n}{ttgAA}
\PYG{o}{/}\PYG{o}{+}\PYG{l+m+mi}{7}\PYG{o}{/}\PYG{l+m+mi}{7}\PYG{n}{v}\PYG{o}{/}\PYG{n}{u}\PYG{o}{/}\PYG{o}{+}\PYG{n}{r}\PYG{o}{/}\PYG{l+m+mi}{5}\PYG{n}{gAA7}\PYG{o}{/}\PYG{l+m+mi}{7}\PYG{n}{v}\PYG{o}{/}\PYG{n}{u}\PYG{o}{/}\PYG{o}{+}\PYG{l+m+mi}{6}\PYG{o}{/}\PYG{l+m+mi}{7}\PYG{n}{n}\PYG{o}{/}\PYG{n}{gAA}
\PYG{o}{/}\PYG{o}{+}\PYG{l+m+mi}{7}\PYG{o}{/}\PYG{l+m+mi}{7}\PYG{n}{vPu}\PYG{o}{/}\PYG{o}{+}\PYG{l+m+mi}{7}\PYG{o}{/}\PYG{l+m+mi}{7}\PYG{n}{gAA7}\PYG{o}{/}\PYG{l+m+mi}{7}\PYG{n}{v}\PYG{o}{/}\PYG{n}{u8}\PYG{o}{+}\PYG{l+m+mi}{7}\PYG{o}{/}\PYG{l+m+mi}{7}\PYG{n}{v}\PYG{o}{/}\PYG{n}{gAA}
\PYG{o}{/}\PYG{o}{+}\PYG{l+m+mi}{7}\PYG{o}{/}\PYG{l+m+mi}{7}\PYG{n}{vPu}\PYG{o}{/}\PYG{o}{+}\PYG{l+m+mi}{7}\PYG{o}{/}\PYG{l+m+mi}{7}\PYG{n}{gAA7}\PYG{o}{/}\PYG{l+m+mi}{7}\PYG{n}{v}\PYG{o}{/}\PYG{n}{u8G7}\PYG{o}{/}\PYG{l+m+mi}{7}\PYG{n}{v}\PYG{o}{/}\PYG{n}{gAA}
\PYG{o}{/}\PYG{o}{+}\PYG{l+m+mi}{7}\PYG{o}{/}\PYG{l+m+mi}{7}\PYG{n}{vPu}\PYG{o}{/}\PYG{o}{+}\PYG{l+m+mi}{7}\PYG{o}{/}\PYG{l+m+mi}{7}\PYG{n}{gAA797vvu}\PYG{o}{/}\PYG{o}{+}\PYG{l+m+mi}{7}\PYG{o}{/}\PYG{l+m+mi}{7}\PYG{n}{v}\PYG{o}{/}\PYG{n}{gAA}
\PYG{o}{/}\PYG{o}{+}\PYG{l+m+mi}{7}\PYG{o}{/}\PYG{n}{AP}\PYG{o}{/}\PYG{o}{+}\PYG{l+m+mi}{7}\PYG{o}{/}\PYG{l+m+mi}{4}\PYG{n}{A}\PYG{o}{/}\PYG{n}{v}\PYG{o}{/}\PYG{o}{+}\PYG{o}{/}\PYG{o}{+}\PYG{l+m+mi}{7}\PYG{o}{/}\PYG{n}{AP}\PYG{o}{/}\PYG{o}{+}\PYG{l+m+mi}{7}\PYG{o}{/}\PYG{l+m+mi}{4}\PYG{n}{A}\PYG{o}{/}\PYG{n}{v}\PYG{o}{/}\PYG{o}{+}
\PYG{o}{/}\PYG{o}{+}\PYG{l+m+mi}{717}\PYG{n}{vvu9e7}\PYG{o}{/}\PYG{l+m+mi}{7}\PYG{n}{gAA7}\PYG{o}{/}\PYG{l+m+mi}{7}\PYG{n}{v1u}\PYG{o}{/}\PYG{n}{u79bv}\PYG{o}{/}\PYG{n}{gAA}
\PYG{o}{/}\PYG{o}{+}\PYG{l+m+mi}{73}\PYG{n}{xuPG44LB}\PYG{o}{/}\PYG{n}{v}\PYG{o}{/}\PYG{o}{+}\PYG{o}{/}\PYG{o}{+}\PYG{l+m+mi}{73}\PYG{n}{xuPG44LB}\PYG{o}{/}\PYG{n}{v}\PYG{o}{/}\PYG{o}{+}
\PYG{o}{/}\PYG{o}{+}\PYG{l+m+mi}{7}\PYG{n}{bAP}\PYG{o}{/}\PYG{o}{+}\PYG{l+m+mi}{7}\PYG{o}{/}\PYG{l+m+mi}{4}\PYG{n}{A1v}\PYG{o}{/}\PYG{o}{+}\PYG{o}{/}\PYG{o}{+}\PYG{l+m+mi}{7}\PYG{n}{bAP}\PYG{o}{/}\PYG{o}{+}\PYG{l+m+mi}{7}\PYG{o}{/}\PYG{l+m+mi}{4}\PYG{n}{A1v}\PYG{o}{/}\PYG{o}{+}
\PYG{o}{/}\PYG{o}{+}\PYG{l+m+mi}{7}\PYG{n}{n7oHu5}\PYG{o}{+}\PYG{l+m+mi}{7}\PYG{o}{/}\PYG{l+m+mi}{7}\PYG{n}{gAA7}\PYG{o}{/}\PYG{l+m+mi}{7}\PYG{n}{vzu8C787v}\PYG{o}{/}\PYG{n}{gAA}
\PYG{o}{/}\PYG{o}{+}\PYG{l+m+mi}{7}\PYG{n}{v7sfu7}\PYG{o}{+}\PYG{l+m+mi}{7}\PYG{o}{/}\PYG{l+m+mi}{7}\PYG{n}{gAA7}\PYG{o}{/}\PYG{l+m+mi}{7}\PYG{n}{v7u}\PYG{o}{/}\PYG{n}{G7}\PYG{o}{+}\PYG{l+m+mi}{7}\PYG{n}{v}\PYG{o}{/}\PYG{n}{gAA}
\PYG{o}{/}\PYG{o}{+}\PYG{l+m+mi}{7}\PYG{n}{v7tfu7}\PYG{o}{+}\PYG{l+m+mi}{7}\PYG{o}{/}\PYG{l+m+mi}{7}\PYG{n}{gAA7}\PYG{o}{/}\PYG{l+m+mi}{7}\PYG{n}{v7u}\PYG{o}{/}\PYG{n}{W7}\PYG{o}{+}\PYG{l+m+mi}{7}\PYG{n}{v}\PYG{o}{/}\PYG{n}{gAA}
\PYG{o}{/}\PYG{o}{+}\PYG{l+m+mi}{7}\PYG{n}{vAP}\PYG{o}{/}\PYG{o}{+}\PYG{l+m+mi}{7}\PYG{o}{/}\PYG{l+m+mi}{4}\PYG{n}{A7v}\PYG{o}{/}\PYG{o}{+}\PYG{o}{/}\PYG{o}{+}\PYG{l+m+mi}{7}\PYG{n}{vAP}\PYG{o}{/}\PYG{o}{+}\PYG{l+m+mi}{7}\PYG{o}{/}\PYG{l+m+mi}{4}\PYG{n}{A7v}\PYG{o}{/}\PYG{o}{+}
\PYG{o}{/}\PYG{o}{+}\PYG{l+m+mi}{7}\PYG{n}{X7oPu1}\PYG{o}{+}\PYG{l+m+mi}{7}\PYG{o}{/}\PYG{l+m+mi}{7}\PYG{n}{gAA7}\PYG{o}{/}\PYG{l+m+mi}{7}\PYG{n}{v1u}\PYG{o}{+}\PYG{n}{C79bv}\PYG{o}{/}\PYG{n}{gAA}
\PYG{o}{/}\PYG{o}{+}\PYG{n}{bP2ofah}\PYG{o}{+}\PYG{n}{bP}\PYG{o}{/}\PYG{n}{v8}\PYG{o}{+}\PYG{o}{/}\PYG{n}{h7}\PYG{o}{+}\PYG{n}{Hs8}\PYG{o}{+}\PYG{n}{t}\PYG{o}{/}\PYG{l+m+mi}{63}\PYG{o}{/}\PYG{n}{s}\PYG{o}{/}\PYG{o}{+}
\PYG{o}{/}\PYG{o}{+}\PYG{n}{bP2rfat}\PYG{o}{+}\PYG{n}{bP}\PYG{o}{/}\PYG{n}{v8}\PYG{o}{+}\PYG{o}{/}\PYG{n}{t7}\PYG{o}{+}\PYG{l+m+mi}{3}\PYG{n}{s8}\PYG{o}{+}\PYG{n}{t}\PYG{o}{/}\PYG{l+m+mi}{63}\PYG{o}{/}\PYG{n}{s}\PYG{o}{/}\PYG{o}{+}
\PYG{o}{/}\PYG{l+m+mi}{34}\PYG{n}{a4Pn}\PYG{o}{+}\PYG{n}{AAD}\PYG{o}{/}\PYG{o}{+}\PYG{n}{sHW}\PYG{o}{/}\PYG{l+m+mi}{84}\PYG{n}{AAO}\PYG{o}{/}\PYG{o}{+}\PYG{n}{Xg4}\PYG{o}{/}\PYG{o}{/}\PYG{n}{gAA}
\PYG{o}{/}\PYG{l+m+mi}{34}\PYG{n}{ekP3}\PYG{o}{+}\PYG{n}{AAD}\PYG{o}{/}\PYG{o}{+}\PYG{n}{gc2}\PYG{o}{/}\PYG{o}{+}\PYG{l+m+mi}{4}\PYG{n}{AAO}\PYG{o}{/}\PYG{o}{+}\PYG{l+m+mi}{3}\PYG{n}{MC}\PYG{o}{/}\PYG{o}{/}\PYG{n}{gAA}
\PYG{o}{/}\PYG{l+m+mi}{34}\PYG{n}{O4P3}\PYG{o}{+}\PYG{n}{AAD}\PYG{o}{/}\PYG{o}{+}\PYG{n}{uD2}\PYG{o}{/}\PYG{o}{+}\PYG{l+m+mi}{4}\PYG{n}{AAO}\PYG{o}{/}\PYG{o}{+}\PYG{l+m+mi}{3}\PYG{n}{wa}\PYG{o}{/}\PYG{o}{/}\PYG{n}{gAA}
\PYG{o}{/}\PYG{l+m+mi}{36}\PYG{n}{a8vn}\PYG{o}{+}\PYG{n}{AAD}\PYG{o}{/}\PYG{o}{+}\PYG{n}{ufW}\PYG{o}{/}\PYG{l+m+mi}{84}\PYG{n}{AAO}\PYG{o}{/}\PYG{o}{+}\PYG{n}{Xz4}\PYG{o}{/}\PYG{o}{/}\PYG{n}{gAA}
\PYG{o}{/}\PYG{l+m+mi}{37}\PYG{o}{+}\PYG{o}{/}\PYG{n}{v3}\PYG{o}{+}\PYG{n}{AAD}\PYG{o}{/}\PYG{o}{+}\PYG{n}{v}\PYG{o}{/}\PYG{l+m+mi}{2}\PYG{o}{/}\PYG{o}{+}\PYG{l+m+mi}{4}\PYG{n}{AAO}\PYG{o}{/}\PYG{o}{+}\PYG{l+m+mi}{3}\PYG{o}{/}\PYG{l+m+mi}{6}\PYG{o}{/}\PYG{o}{/}\PYG{n}{gAA}
\PYG{o}{/}\PYG{l+m+mi}{376}\PYG{o}{/}\PYG{n}{vn}\PYG{o}{+}\PYG{n}{AAD}\PYG{o}{/}\PYG{o}{+}\PYG{n}{v}\PYG{o}{/}\PYG{n}{W}\PYG{o}{/}\PYG{l+m+mi}{84}\PYG{n}{AAO}\PYG{o}{/}\PYG{o}{+}\PYG{n}{X}\PYG{o}{/}\PYG{l+m+mi}{4}\PYG{o}{/}\PYG{o}{/}\PYG{n}{gAA}
\PYG{o}{/}\PYG{l+m+mi}{37}\PYG{n}{Ozv3}\PYG{o}{+}\PYG{n}{AAD}\PYG{o}{/}\PYG{o}{+}\PYG{n}{s}\PYG{o}{/}\PYG{l+m+mi}{2}\PYG{o}{/}\PYG{o}{+}\PYG{l+m+mi}{4}\PYG{n}{AAO}\PYG{o}{/}\PYG{o}{+}\PYG{l+m+mi}{356}\PYG{o}{/}\PYG{o}{/}\PYG{n}{gAA}
\PYG{o}{/}\PYG{l+m+mi}{3}\PYG{n}{b}\PYG{o}{/}\PYG{n}{jv}\PYG{o}{+}\PYG{n}{u}\PYG{o}{/}\PYG{l+m+mi}{47}\PYG{o}{/}\PYG{n}{dv}\PYG{o}{/}\PYG{o}{+}\PYG{n}{u}\PYG{o}{/}\PYG{l+m+mi}{7}\PYG{n}{H}\PYG{o}{/}\PYG{n}{tf}\PYG{o}{+}\PYG{n}{x}\PYG{o}{/}\PYG{l+m+mi}{67}\PYG{o}{/}\PYG{n}{v}\PYG{o}{/}\PYG{o}{+}
\PYG{o}{/}\PYG{l+m+mi}{97}\PYG{o}{/}\PYG{n}{wP}\PYG{o}{/}\PYG{n}{e}\PYG{o}{+}\PYG{n}{N77Xvje}\PYG{o}{+}\PYG{o}{/}\PYG{l+m+mi}{77}\PYG{o}{/}\PYG{n}{v}\PYG{o}{/}\PYG{n}{g}\PYG{o}{/}\PYG{o}{/}\PYG{l+m+mi}{4}\PYG{n}{AHv}\PYG{o}{/}\PYG{o}{+}
\PYG{o}{/}\PYG{n}{d773vvuAAC}\PYG{o}{+}\PYG{o}{/}\PYG{n}{r7}\PYG{o}{+}\PYG{n}{f34AAPfu7}\PYG{o}{/}\PYG{n}{bv9gAA}
\PYG{o}{/}\PYG{n}{wKBer0Cgf7}\PYG{o}{/}\PYG{o}{/}\PYG{n}{v}\PYG{o}{/}\PYG{o}{+}\PYG{o}{/}\PYG{o}{/}\PYG{l+m+mi}{7}\PYG{o}{/}\PYG{n}{AoF6vQKB}\PYG{o}{/}\PYG{n}{v}\PYG{o}{/}\PYG{o}{+}
\PYG{o}{+}\PYG{o}{/}\PYG{l+m+mi}{73}\PYG{n}{MgH}\PYG{o}{+}\PYG{o}{/}\PYG{o}{/}\PYG{n}{Yz7v4C}\PYG{o}{+}\PYG{o}{/}\PYG{l+m+mi}{73}\PYG{n}{MgP}\PYG{o}{+}\PYG{o}{/}\PYG{o}{/}\PYG{n}{Yz7v4C}
\PYG{l+m+mi}{2}\PYG{o}{/}\PYG{l+m+mi}{7}\PYG{n}{nMu}\PYG{o}{/}\PYG{o}{+}\PYG{l+m+mi}{3}\PYG{o}{/}\PYG{l+m+mi}{75}\PYG{n}{mP}\PYG{o}{/}\PYG{o}{+}\PYG{o}{/}\PYG{l+m+mi}{7}\PYG{n}{Yzzv}\PYG{o}{/}\PYG{n}{e}\PYG{o}{/}\PYG{l+m+mi}{76}\PYG{n}{Z8v}\PYG{o}{/}\PYG{o}{+}
\PYG{l+m+mi}{2}\PYG{n}{Hzfkrnudnzvst3O2n67gmZ83Z7V5iYa}
\PYG{l+m+mi}{3}\PYG{o}{/}\PYG{l+m+mi}{4}\PYG{n}{HMv}\PYG{o}{/}\PYG{o}{+}\PYG{o}{/}\PYG{o}{+}\PYG{l+m+mi}{4}\PYG{n}{zgv}\PYG{o}{/}\PYG{o}{+}\PYG{l+m+mi}{3}\PYG{o}{/}\PYG{l+m+mi}{4}\PYG{n}{HMv}\PYG{o}{/}\PYG{o}{+}\PYG{o}{/}\PYG{o}{+}\PYG{l+m+mi}{4}\PYG{n}{zgv}\PYG{o}{/}\PYG{o}{+}
\PYG{l+m+mi}{3}\PYG{o}{/}\PYG{l+m+mi}{6}\PYG{n}{s}\PYG{o}{+}\PYG{n}{N}\PYG{o}{/}\PYG{o}{+}\PYG{o}{/}\PYG{o}{+}\PYG{l+m+mi}{451}\PYG{n}{v}\PYG{o}{/}\PYG{n}{u3}\PYG{o}{/}\PYG{l+m+mi}{6}\PYG{n}{s}\PYG{o}{+}\PYG{n}{N}\PYG{o}{/}\PYG{o}{+}\PYG{o}{/}\PYG{o}{+}\PYG{l+m+mi}{451}\PYG{n}{v}\PYG{o}{/}\PYG{n}{u}
\PYG{l+m+mi}{3}\PYG{n}{d7u7vd2e7q93N7u73b3unvcve7e9u96}
\PYG{l+m+mi}{3}\PYG{n}{d7urvZ2ebq53Nbqz3a3inucvW7e9u16}
\PYG{l+m+mi}{3}\PYG{n}{dz}\PYG{o}{/}\PYG{o}{/}\PYG{n}{v}\PYG{o}{/}\PYG{o}{+}\PYG{n}{d3b}\PYG{o}{/}\PYG{o}{/}\PYG{n}{v}\PYG{o}{/}\PYG{o}{+}\PYG{l+m+mi}{3}\PYG{n}{dz}\PYG{o}{/}\PYG{o}{/}\PYG{n}{v}\PYG{o}{/}\PYG{o}{+}\PYG{n}{d3b}\PYG{o}{/}\PYG{o}{/}\PYG{n}{v}\PYG{o}{/}\PYG{o}{+}
\PYG{l+m+mi}{4}\PYG{n}{er}\PYG{o}{/}\PYG{o}{/}\PYG{n}{tde}\PYG{o}{/}\PYG{o}{/}\PYG{l+m+mi}{764}\PYG{n}{P}\PYG{o}{/}\PYG{o}{+}\PYG{l+m+mi}{117}\PYG{o}{/}\PYG{o}{/}\PYG{n}{uHq}\PYG{o}{/}\PYG{o}{/}\PYG{l+m+mi}{7}\PYG{n}{XXv}\PYG{o}{/}\PYG{o}{+}
\PYG{l+m+mi}{4}\PYG{n}{er}\PYG{o}{/}\PYG{o}{/}\PYG{n}{tdO}\PYG{o}{/}\PYG{o}{+}\PYG{l+m+mi}{764}\PYG{n}{P}\PYG{o}{/}\PYG{o}{+}\PYG{l+m+mi}{117}\PYG{n}{f}\PYG{o}{/}\PYG{n}{sHq}\PYG{o}{/}\PYG{o}{/}\PYG{l+m+mi}{7}\PYG{n}{XXv}\PYG{o}{/}\PYG{o}{+}
\PYG{l+m+mi}{5}\PYG{o}{/}\PYG{l+m+mi}{6}\PYG{n}{BAOf}\PYG{o}{+}\PYG{o}{/}\PYG{o}{+}\PYG{n}{aAgP}\PYG{o}{/}\PYG{n}{m5}\PYG{o}{/}\PYG{l+m+mi}{6}\PYG{n}{BAOf}\PYG{o}{+}\PYG{o}{/}\PYG{o}{+}\PYG{n}{aAgP}\PYG{o}{/}\PYG{n}{m}
\PYG{l+m+mi}{5}\PYG{o}{+}\PYG{n}{aB}\PYG{o}{/}\PYG{n}{uf}\PYG{o}{+}\PYG{o}{/}\PYG{l+m+mi}{87}\PYG{o}{/}\PYG{n}{Ao}\PYG{o}{/}\PYG{n}{OvH6lfrT}\PYG{o}{+}\PYG{n}{hX7}\PYG{o}{/}\PYG{l+m+mi}{5}\PYG{n}{v}\PYG{o}{+}\PYG{n}{A}
\PYG{l+m+mi}{5}\PYG{n}{mb}\PYG{o}{/}\PYG{o}{/}\PYG{n}{szm}\PYG{o}{/}\PYG{o}{/}\PYG{l+m+mi}{6}\PYG{n}{ZmP}\PYG{o}{/}\PYG{o}{+}\PYG{o}{/}\PYG{o}{/}\PYG{l+m+mi}{4}\PYG{n}{zOP}\PYG{o}{/}\PYG{o}{+}\PYG{o}{/}\PYG{o}{/}\PYG{l+m+mi}{45}\PYG{n}{zv}\PYG{o}{/}\PYG{o}{+}
\PYG{l+m+mi}{5}\PYG{n}{zL}\PYG{o}{/}\PYG{o}{/}\PYG{n}{pnO}\PYG{o}{/}\PYG{o}{/}\PYG{l+m+mi}{7}\PYG{n}{nMv}\PYG{o}{/}\PYG{o}{+}\PYG{n}{mc7}\PYG{o}{/}\PYG{o}{/}\PYG{n}{ucy}\PYG{o}{/}\PYG{o}{/}\PYG{l+m+mi}{45}\PYG{n}{zv}\PYG{o}{/}\PYG{o}{+}
\PYG{l+m+mi}{7}\PYG{o}{/}\PYG{l+m+mi}{6}\PYG{n}{rMu}\PYG{o}{/}\PYG{o}{+}\PYG{n}{AADu7qqq7u4AAP7uMqr}\PYG{o}{+}\PYG{l+m+mi}{7}\PYG{n}{gAA}
\PYG{l+m+mi}{7}\PYG{o}{/}\PYG{l+m+mi}{7}\PYG{n}{vAO}\PYG{o}{/}\PYG{o}{+}\PYG{o}{/}\PYG{o}{+}\PYG{l+m+mi}{4}\PYG{n}{A7v}\PYG{o}{/}\PYG{n}{u7}\PYG{o}{/}\PYG{l+m+mi}{7}\PYG{n}{vAO}\PYG{o}{/}\PYG{o}{+}\PYG{o}{/}\PYG{o}{+}\PYG{l+m+mi}{4}\PYG{n}{A7v}\PYG{o}{/}\PYG{n}{u}
\PYG{l+m+mf}{8e7}\PYG{n}{u7u7u7u7x7gAA7x7u7u7u7u7vHgAA}
\PYG{l+m+mi}{9}\PYG{o}{/}\PYG{l+m+mi}{7}\PYG{n}{nANf}\PYG{o}{+}\PYG{o}{/}\PYG{o}{/}\PYG{n}{YA5v}\PYG{o}{/}\PYG{n}{W9}\PYG{o}{/}\PYG{l+m+mi}{7}\PYG{n}{nANf}\PYG{o}{+}\PYG{o}{/}\PYG{o}{/}\PYG{n}{YA5v}\PYG{o}{/}\PYG{n}{W}
\PYG{l+m+mi}{9}\PYG{o}{/}\PYG{l+m+mi}{7}\PYG{n}{nMtf}\PYG{o}{+}\PYG{o}{/}\PYG{o}{+}\PYG{l+m+mi}{4}\PYG{n}{zzv}\PYG{o}{+}\PYG{n}{u9}\PYG{o}{/}\PYG{l+m+mi}{7}\PYG{n}{nMtf}\PYG{o}{+}\PYG{o}{/}\PYG{o}{+}\PYG{l+m+mi}{4}\PYG{n}{zzv}\PYG{o}{+}\PYG{n}{u}
\PYG{l+m+mi}{9}\PYG{o}{/}\PYG{l+m+mi}{7}\PYG{n}{vAN}\PYG{o}{/}\PYG{o}{+}\PYG{o}{/}\PYG{o}{/}\PYG{n}{YA7v}\PYG{o}{/}\PYG{n}{e9}\PYG{o}{/}\PYG{l+m+mi}{7}\PYG{n}{vAN}\PYG{o}{/}\PYG{o}{+}\PYG{o}{/}\PYG{o}{/}\PYG{n}{YA7v}\PYG{o}{/}\PYG{n}{e}
\PYG{l+m+mi}{9}\PYG{n}{fT}\PYG{o}{/}\PYG{o}{/}\PYG{n}{l9e}\PYG{o}{/}\PYG{o}{/}\PYG{l+m+mi}{719}\PYG{n}{P}\PYG{o}{/}\PYG{o}{+}\PYG{n}{X17}\PYG{o}{/}\PYG{o}{/}\PYG{n}{vX0}\PYG{o}{/}\PYG{o}{/}\PYG{l+m+mi}{5}\PYG{n}{fXv}\PYG{o}{/}\PYG{o}{+}
\PYG{n}{AAAAAAAAAAAAAAAAAAAAAAAAAAAAAAAA}
\PYG{n}{AAD}\PYG{o}{/}\PYG{o}{/}\PYG{n}{gAA}\PYG{o}{/}\PYG{o}{/}\PYG{l+m+mi}{4}\PYG{n}{AAP}\PYG{o}{/}\PYG{o}{+}\PYG{n}{AAD}\PYG{o}{/}\PYG{o}{/}\PYG{n}{gAA}\PYG{o}{/}\PYG{o}{/}\PYG{l+m+mi}{4}\PYG{n}{AAP}\PYG{o}{/}\PYG{o}{+}
\PYG{n}{ff67}\PYG{o}{/}\PYG{n}{td877r}\PYG{o}{/}\PYG{l+m+mi}{1}\PYG{n}{v}\PYG{o}{/}\PYG{n}{uff67}\PYG{o}{/}\PYG{n}{td877r}\PYG{o}{/}\PYG{l+m+mi}{1}\PYG{n}{v}\PYG{o}{/}\PYG{n}{u}
\PYG{n}{JJK22pJIJJK22pJIJJK22pJIJJK22pJI}
\PYG{n}{m}\PYG{o}{/}\PYG{l+m+mi}{73}\PYG{n}{Mu}\PYG{o}{/}\PYG{o}{+}\PYG{o}{/}\PYG{o}{/}\PYG{n}{bM7v}\PYG{o}{/}\PYG{n}{eO}\PYG{o}{/}\PYG{l+m+mi}{73}\PYG{n}{Mu}\PYG{o}{/}\PYG{o}{+}\PYG{o}{/}\PYG{o}{/}\PYG{n}{aZ7v}\PYG{o}{/}\PYG{n}{e}
\PYG{n}{MzL}\PYG{o}{/}\PYG{o}{/}\PYG{n}{szO}\PYG{o}{/}\PYG{o}{/}\PYG{l+m+mi}{6}\PYG{n}{DwP}\PYG{o}{/}\PYG{o}{+}\PYG{n}{zM7}\PYG{o}{/}\PYG{o}{/}\PYG{n}{v}\PYG{o}{/}\PYG{o}{+}\PYG{n}{ODj}\PYG{o}{/}\PYG{o}{/}\PYG{n}{s}\PYG{o}{/}\PYG{n}{O}
\PYG{n}{u}\PYG{o}{/}\PYG{l+m+mi}{7}\PYG{n}{WeO}\PYG{o}{/}\PYG{o}{+}\PYG{o}{/}\PYG{l+m+mi}{7}\PYG{n}{o81v}\PYG{o}{/}\PYG{n}{uu}\PYG{o}{/}\PYG{l+m+mi}{7}\PYG{n}{WeO}\PYG{o}{/}\PYG{o}{+}\PYG{o}{/}\PYG{l+m+mi}{7}\PYG{n}{r}\PYG{o}{/}\PYG{l+m+mi}{1}\PYG{n}{jzu}
\PYG{n}{u}\PYG{o}{/}\PYG{l+m+mi}{7}\PYG{n}{WeO}\PYG{o}{/}\PYG{o}{+}\PYG{o}{/}\PYG{l+m+mi}{7}\PYG{n}{o81v}\PYG{o}{/}\PYG{n}{uu}\PYG{o}{/}\PYG{l+m+mi}{7}\PYG{n}{WeO}\PYG{o}{/}\PYG{o}{+}\PYG{o}{/}\PYG{l+m+mi}{7}\PYG{n}{r}\PYG{o}{/}\PYG{l+m+mi}{1}\PYG{n}{rvu}
\PYG{n}{u}\PYG{o}{/}\PYG{l+m+mi}{7}\PYG{n}{XAO}\PYG{o}{/}\PYG{o}{+}\PYG{o}{/}\PYG{l+m+mi}{7}\PYG{n}{oA1v}\PYG{o}{/}\PYG{n}{uu}\PYG{o}{/}\PYG{l+m+mi}{7}\PYG{n}{XAO}\PYG{o}{/}\PYG{o}{+}\PYG{o}{/}\PYG{l+m+mi}{7}\PYG{n}{oA1v}\PYG{o}{/}\PYG{n}{u}
\PYG{n}{uur}\PYG{o}{/}\PYG{o}{/}\PYG{n}{tdc}\PYG{o}{/}\PYG{o}{/}\PYG{l+m+mi}{666}\PYG{n}{v}\PYG{o}{/}\PYG{o}{+}\PYG{l+m+mi}{11}\PYG{n}{z}\PYG{o}{/}\PYG{o}{/}\PYG{n}{rrq}\PYG{o}{/}\PYG{o}{/}\PYG{l+m+mi}{7}\PYG{n}{XXP}\PYG{o}{/}\PYG{o}{+}
\PYG{n}{w}\PYG{o}{+}\PYG{l+m+mi}{6}\PYG{n}{Z7mbuWu7}\PYG{o}{/}\PYG{l+m+mi}{7}\PYG{n}{gAA74bvMu7M7rTv}\PYG{o}{/}\PYG{n}{gAA}
\PYG{n}{x}\PYG{o}{+}\PYG{l+m+mi}{677}\PYG{n}{qvuu}\PYG{o}{+}\PYG{l+m+mi}{7}\PYG{n}{H7gAA78bvuu}\PYG{o}{+}\PYG{n}{q77rvxgAA}
\PYG{n}{z}\PYG{o}{/}\PYG{l+m+mi}{4}\PYG{n}{DMs}\PYG{o}{/}\PYG{o}{+}\PYG{o}{/}\PYG{l+m+mi}{84}\PYG{n}{zAv}\PYG{o}{/}\PYG{n}{Oz}\PYG{o}{/}\PYG{l+m+mi}{4}\PYG{n}{DMs}\PYG{o}{/}\PYG{o}{+}\PYG{o}{/}\PYG{l+m+mi}{84}\PYG{n}{zAv}\PYG{o}{/}\PYG{n}{O}
\PYG{n}{zM7MzszOzM7MzszOzM7MzszOzM7MzszO}
\PYG{n}{zv4}\PYG{o}{/}\PYG{n}{PvnOx}\PYG{o}{/}\PYG{n}{I}\PYG{o}{+}\PYG{o}{/}\PYG{n}{P9}\PYG{o}{+}\PYG{o}{/}\PYG{n}{b7z3s}\PYG{o}{/}\PYG{n}{uvfb}\PYG{o}{+}\PYG{n}{fvu}\PYG{o}{+}
\end{sphinxVerbatim}
\index{PATTERN\_MAP (in module TotalDepth.util.plot.AREACfg)}

\begin{fulllineitems}
\phantomsection\label{\detokenize{ref/util/plot/AREACfg:TotalDepth.util.plot.AREACfg.PATTERN_MAP}}\pysigline{\sphinxcode{TotalDepth.util.plot.AREACfg.}\sphinxbfcode{PATTERN\_MAP}\sphinxbfcode{ = \{'Void': b'//7//v/+//7//v/+//7//v/+//7//v/+', 'Coal-LigNite': b'AAAAAAAAAAAAAAAAAAAAAAAAAAAAAAAA', 'LimeStone': b'/+7/7v/u/+7/7gAA7/7v/u/+7/7v/gAA', 'Dolomitic LimeStone': b'/+7/7v/u/+r/5gAA7/7v/u/+6/7n/gAA', 'Argillaceous Limestone': b'/+7/7uDu/+7/7gAA7/7v/u8G7/7v/gAA', 'Sandy Limestone': b'/+7/7vPu/+7/7gAA7/7v/u8+7/7v/gAA', 'Argillaceous and dolomitic Limestone': b'/+7/7sHu/+7/7gAA797vvu/+7/7v/gAA', 'Argillaceous and sandy Limestone': b'/+7/7vPu/+7/7gAA7/7v/u8G7/7v/gAA', 'Sandy and dolomitic Limestone': b'/+7/7vPu/+7/7gAA797vvu/+7/7v/gAA', 'Chalk': b'/+7/7v/u/+7/7oOC7/7v/u/+7/7v/oOC', 'Oolitic Limestone': b'x+677qvuu+7H7gAA78bvuu+q77rvxgAA', 'Pelletic Limestone': b'8e7u7u7u7u7x7gAA7x7u7u7u7u7vHgAA', 'Nodular Limestone': b'/+6B7n7uge7/7gAA7/7vAu787wLv/gAA', 'Shelly Limestone': b'/+7/7rvux+7/7gAA7/7v/u+678bv/gAA', 'Reefal Limestone': b'w+6Z7mbuWu7/7gAA74bvMu7M7rTv/gAA', 'Bioclastic Limestone': b'/+7X7oPu1+7/7gAA7/7v1u+C79bv/gAA', 'Silicified Limestone': b'/+7/7v/u/+7bbAAA7/7v/u/+7/5ttgAA', 'Cherty Limestone': b'/+7v7sfu7+7/7gAA7/7v7u/G7+7v/gAA', 'Anhydritic Limestone': b'/+717vvu9e7/7gAA7/7v1u/u79bv/gAA', 'Phosphatic Limestone': b'/+7v7tfu7+7/7gAA7/7v7u/W7+7v/gAA', 'Bituminous Limestone': b'/+7n7oHu5+7/7gAA7/7vzu8C787v/gAA', 'Mudstone': b'//b/9v/2+7b5Ngqg+777vv/+9/73/gAA', 'Wackstone': b'//b/9v/2+7b7tgqg+T77vv/+9/73/gAA', 'Packstone': b'//b/9v/2+Pb7dgjA+/77/v/+9/73/gAA', 'Grainstone': b'//b/9v/2+Db79gog+774Pv/+9/73/gAA', 'Bundstone': b'//b/9v/2+Hb7tghg+774fv/+9/73/gAA', 'Dolomite': b'/37+/v3+AAD/+v/2/+4AAO/+3/6//gAA', 'Calacareous Dolomite': b'/376/vn+AAD/+v/W/84AAO/+X/4//gAA', 'Argillaceous Dolomite': b'/34O4P3+AAD/+uD2/+4AAO/+3wa//gAA', 'Sandy Dolomite': b'/37Ozv3+AAD/+s/2/+4AAO/+356//gAA', 'Argillaceous and calcareous Dolomite': b'/34a4Pn+AAD/+sHW/84AAO/+Xg4//gAA', 'Sandy and calcareous Dolomite': b'/36a8vn+AAD/+ufW/84AAO/+Xz4//gAA', 'Argillaceous and Sandy Dolomite': b'/34ekP3+AAD/+gc2/+4AAO/+3MC//gAA', 'Cargneule': b'/d773vvuAAC+/r7+f34AAPfu7/bv9gAA', 'Clay-Shale': b'//7/AP/+//4A/v/+//7/AP/+//4A/v/+', 'Slaty Shale': b'AAD//gAA//4AAP/+AAD//gAA//4AAP/+', 'Plastic Clay-Shale': b'/97/wP/e+N77Xvje+/77/v/g//4AHv/+', 'Calcareous Clay-Shale': b'7/7vAO/+/+4A7v/u7/7vAO/+/+4A7v/u', 'Dolomitic Caly-Shale': b'9/7vAN/+//YA7v/e9/7vAN/+//YA7v/e', 'Sandy Clay-Shale': b'//7vAP/+//4A7v/+//7vAP/+//4A7v/+', 'Silty Clay-Shale': b'//7bAP/+//4A1v/+//7bAP/+//4A1v/+', 'Calcareous and dolomitic Clay-Shale': b'9/7nANf+//YA5v/W9/7nANf+//YA5v/W', 'Sandy and calcareous Clay-Shale': b'//7nAP/+/+4A7v/u//7nAP/+/+4A7v/u', 'Sandy and dolomitic Clay-Shale': b'//4zAP/6//oA9v/+//4zAP/8//oA+v/2', 'Sandy, calcareous and dolomitic Clay-Shale': b'/+7vAP/+7/4A7v/+/+7vAP/+7/4A7v/+', 'Silty, calcareous and dolomitic Clay-Shale': b'/+7bAP/+7/4A1v/+/+7bAP/+7/4A1v/+', 'Marl': b'/+7/AP/+7/4A/v/+/+7/AP/+7/4A/v/+', 'Dolomitic Marl': b'//7/APv+9/4A/v/6//b/AP3+8/4A/v/+', 'Silty Marl': b'/+7bAP/+7/4A1v/+/+7bAP/+7/4A1v/+', 'Silty dolomitic Marl': b'//7bAPv+9/4A1v/6//bbAP3+8/4A2v/+', 'Silicified Claystone': b'//7/AP/+Wv4A/v/+//7/AP/+Wv4A/v/+', 'Anhydritic Clay-Shale': b'//7/AP/+/9YA7v/W1/7vANf+/9YA7v/W', 'Saliferous Clay-Shale': b'//6/AIP+//4Avv+C//6/AIP+//4Avv+C', 'Gypsiferous Clay-Shale': b'u/7XAO/+/7oA1v/uu/7XAO/+/7oA1v/u', 'Bituminous Clay-Shale': b'5/6BAOf+/+aAgP/m5/6BAOf+/+aAgP/m', 'Organic Shale': b'zM7MzszOzM7MzszOzM7MzszOzM7MzszO', 'Silt': b'uur//tdc//666v/+11z//rrq//7XXP/+', 'Fine Sand': b'3dz//v/+d3b//v/+3dz//v/+d3b//v/+', 'Medium Sand': b'5mb//szm//6ZmP/+//4zOP/+//45zv/+', 'Coarse Sand': b'//7nOOc4//45zv/O//7nOP8+//45zvn+', 'Siltstone': b'9fT//l9e//719P/+X17//vX0//5fXv/+', 'Fine Sandstone': b'3dz//v/+d3b//v/+3dz//v/+d3b//v/+', 'Medium Sandstone': b'5zL//pnO//7nMv/+mc7//ucy//45zv/+', 'Coarse Sandstone': b'//7nOOc4//45zjnO//7nOOc4//45zjnO', 'Quartzite': b'7/6rMu/+AADu7qqq7u4AAP7uMqr+7gAA', 'Argillaceous Siltstone': b'4er//tde//764P/+117//uHq//7XXv/+', 'Argillaceous Sandstone': b'MzL//szO//6DwP/+zM7//v/+ODj//s/O', 'Calcareous Sandstone': b'/+4zLv/uzOj/7gcA3/7TMt/+XMzf/gBy', 'Dolomitic Sandstone': b'm/73Mu/+//bM7v/eO/73Mu/+//aZ7v/e', 'Calcareous and dolomitic Sandstone': b'9/7nMtf+/+4zzv+u9/7nMtf+/+4zzv+u', 'Argillaceous and calcareous Sandstone': b'3/4HMv/+/+4zgv/+3/4HMv/+/+4zgv/+', 'Argillaceous and dolomitic Sandstone': b'+/73MgH+//Yz7v4C+/73MgP+//Yz7v4C', 'Arkosic Sandstone': b'2/7nMu/+3/75mP/+/7Yzzv/e/76Z8v/+', 'Shelly Sandstone': b'//60ks/+//6ZaP+e//60ks/+//6ZaP+e', 'Shelly Siltstone': b'//6v1vt+rOr//lq2/87//rrq//7XXu++', 'Diatomite': b'//69fIEA//4QENfW//69fIEA//4QENf2', 'Radiolarite': b'/3b/jv+u/47/dv/+u/7H/tf+x/67/v/+', 'Chert': b'//73/uv+wf7//v/+//7/3v+u/wb//v/+', 'Bedded Chert': b'3d7u7vd2e7q93N7u73b3unvcve7e9u96', 'Anhydritic Sandstone': b'u/7WeO/+/7o81v/uu/7WeO/+/7r/1jzu', 'Anhydritic Siltsone': b'u/7WeO/+/7o81v/uu/7WeO/+/7r/1rvu', 'Saliferous Sandstone': b'4er//tdO/+764P/+117f/sHq//7XXv/+', 'Bituminous Sandstone': b'z/4DMs/+/84zAv/Oz/4DMs/+/84zAv/O', 'Glauconitic Sandstone': b'3/6s+N/+/+451v/u3/6s+N/+/+451v/u', 'Ferruginous Sandstone': b'//6N5r/mj76/urm+ub7v8u7+774t8v/y', 'Monom Conglomerate 4-64 mm': b'//7HHrruuu7HHv/+//6Pxne6d7qPxv/+', 'Polym Conglomerate 4-64 mm': b'//7HHoLugu7HHv/+//6PxneCd4KPxv/+', 'Monom Conglomerate 2-4 mm': b'/+bP2rfat+bP/v8+/t7+3s8+t/63/s/+', 'Polym Conglomerate 2-4 mm': b'/+bP2ofah+bP/v8+/h7+Hs8+t/63/s/+', 'Monogenic Breccia': b'/+73xuPG44LB/v/+/+73xuPG44LB/v/+', 'Polymictic Breccia': b'/+73xuPG44LB/v/+/+73xuPG44LB/v/+', 'Evaporite': b'JJK22pJIJJK22pJIJJK22pJIJJK22pJI', 'Halite': b'//7/3t/e397fwsP+//7/3t/e397fwsP+', 'Anhydrite': b'3d7urvZ2ebq53Nbqz3a3inucvW7e9u16', 'Gypsum': b'ff67/td877r/1v/uff67/td877r/1v/u', 'Sylvite': b'/wKBer0Cgf7//v/+//7/AoF6vQKB/v/+', 'Volcanic Rock': b'zv4/PvnOx/I+/P9+/b7z3s/uvfb+fvu+', 'Basalt': b'//732gAAvfbvfAAAu97vegAAve73fAAA', 'Granit': b'5+aB/uf+/87/Ao/OvH6lfrT+hX7/5v+A', 'Metamorphic Rocks': b'2Hzfkrnudnzvst3O2n67gmZ83Z7V5iYa'\}}}
A map of \{pattern name : bytes, …\}

\end{fulllineitems}

\index{pprintPattern() (in module TotalDepth.util.plot.AREACfg)}

\begin{fulllineitems}
\phantomsection\label{\detokenize{ref/util/plot/AREACfg:TotalDepth.util.plot.AREACfg.pprintPattern}}\pysiglinewithargsret{\sphinxcode{TotalDepth.util.plot.AREACfg.}\sphinxbfcode{pprintPattern}}{\emph{theName}}{}
Pretty print the pattern.

\end{fulllineitems}

\index{pprintAll() (in module TotalDepth.util.plot.AREACfg)}

\begin{fulllineitems}
\phantomsection\label{\detokenize{ref/util/plot/AREACfg:TotalDepth.util.plot.AREACfg.pprintAll}}\pysiglinewithargsret{\sphinxcode{TotalDepth.util.plot.AREACfg.}\sphinxbfcode{pprintAll}}{}{}
Pretty print all patterns.

\end{fulllineitems}



\paragraph{FILM Plotting Configuration}
\label{\detokenize{ref/util/plot/FILMCfg:module-TotalDepth.util.plot.FILMCfg}}\label{\detokenize{ref/util/plot/FILMCfg::doc}}\label{\detokenize{ref/util/plot/FILMCfg:film-plotting-configuration}}\index{TotalDepth.util.plot.FILMCfg (module)}
Represents the part of a plot configuration that, typically, can be obtained
from a LIS FILM table.

Created on 21 Mar 2011

Example of the data in a film table:

\begin{sphinxVerbatim}[commandchars=\\\{\}]
\PYG{n}{Table} \PYG{n}{record} \PYG{p}{(}\PYG{n+nb}{type} \PYG{l+m+mi}{34}\PYG{p}{)} \PYG{n+nb}{type}\PYG{p}{:} \PYG{n}{FILM}

\PYG{n}{MNEM}  \PYG{n}{GCOD}  \PYG{n}{GDEC}  \PYG{n}{DEST}  \PYG{n}{DSCA}
\PYG{o}{\PYGZhy{}}\PYG{o}{\PYGZhy{}}\PYG{o}{\PYGZhy{}}\PYG{o}{\PYGZhy{}}\PYG{o}{\PYGZhy{}}\PYG{o}{\PYGZhy{}}\PYG{o}{\PYGZhy{}}\PYG{o}{\PYGZhy{}}\PYG{o}{\PYGZhy{}}\PYG{o}{\PYGZhy{}}\PYG{o}{\PYGZhy{}}\PYG{o}{\PYGZhy{}}\PYG{o}{\PYGZhy{}}\PYG{o}{\PYGZhy{}}\PYG{o}{\PYGZhy{}}\PYG{o}{\PYGZhy{}}\PYG{o}{\PYGZhy{}}\PYG{o}{\PYGZhy{}}\PYG{o}{\PYGZhy{}}\PYG{o}{\PYGZhy{}}\PYG{o}{\PYGZhy{}}\PYG{o}{\PYGZhy{}}\PYG{o}{\PYGZhy{}}\PYG{o}{\PYGZhy{}}\PYG{o}{\PYGZhy{}}\PYG{o}{\PYGZhy{}}\PYG{o}{\PYGZhy{}}\PYG{o}{\PYGZhy{}}\PYG{o}{\PYGZhy{}}

\PYG{l+m+mi}{1}     \PYG{n}{EEE}   \PYG{o}{\PYGZhy{}}\PYG{o}{\PYGZhy{}}\PYG{o}{\PYGZhy{}}\PYG{o}{\PYGZhy{}}  \PYG{n}{PF2}   \PYG{n}{D200}
\PYG{l+m+mi}{2}     \PYG{n}{EEE}   \PYG{o}{\PYGZhy{}}\PYG{o}{\PYGZhy{}}\PYG{o}{\PYGZhy{}}\PYG{o}{\PYGZhy{}}  \PYG{n}{PF1}   \PYG{n}{DM}
\end{sphinxVerbatim}

Other FILM Table Examples:

\begin{sphinxVerbatim}[commandchars=\\\{\}]
\PYG{n}{Table} \PYG{n}{record} \PYG{p}{(}\PYG{n+nb}{type} \PYG{l+m+mi}{34}\PYG{p}{)} \PYG{n+nb}{type}\PYG{p}{:} \PYG{n}{FILM}

\PYG{n}{MNEM}  \PYG{n}{GCOD}  \PYG{n}{GDEC}  \PYG{n}{DEST}  \PYG{n}{DSCA}
\PYG{o}{\PYGZhy{}}\PYG{o}{\PYGZhy{}}\PYG{o}{\PYGZhy{}}\PYG{o}{\PYGZhy{}}\PYG{o}{\PYGZhy{}}\PYG{o}{\PYGZhy{}}\PYG{o}{\PYGZhy{}}\PYG{o}{\PYGZhy{}}\PYG{o}{\PYGZhy{}}\PYG{o}{\PYGZhy{}}\PYG{o}{\PYGZhy{}}\PYG{o}{\PYGZhy{}}\PYG{o}{\PYGZhy{}}\PYG{o}{\PYGZhy{}}\PYG{o}{\PYGZhy{}}\PYG{o}{\PYGZhy{}}\PYG{o}{\PYGZhy{}}\PYG{o}{\PYGZhy{}}\PYG{o}{\PYGZhy{}}\PYG{o}{\PYGZhy{}}\PYG{o}{\PYGZhy{}}\PYG{o}{\PYGZhy{}}\PYG{o}{\PYGZhy{}}\PYG{o}{\PYGZhy{}}\PYG{o}{\PYGZhy{}}\PYG{o}{\PYGZhy{}}\PYG{o}{\PYGZhy{}}\PYG{o}{\PYGZhy{}}\PYG{o}{\PYGZhy{}}
\PYG{l+m+mi}{1}     \PYG{n}{EEB}   \PYG{o}{\PYGZhy{}}\PYG{o}{\PYGZhy{}}\PYG{o}{\PYGZhy{}}\PYG{o}{\PYGZhy{}}  \PYG{n}{PF1}   \PYG{n}{D200}
\PYG{l+m+mi}{2}     \PYG{n}{EEB}   \PYG{o}{\PYGZhy{}}\PYG{o}{\PYGZhy{}}\PYG{o}{\PYGZhy{}}\PYG{o}{\PYGZhy{}}  \PYG{n}{PF2}   \PYG{n}{DM}

\PYG{n}{Table} \PYG{n}{record} \PYG{p}{(}\PYG{n+nb}{type} \PYG{l+m+mi}{34}\PYG{p}{)} \PYG{n+nb}{type}\PYG{p}{:} \PYG{n}{FILM}

\PYG{n}{MNEM}  \PYG{n}{GCOD}  \PYG{n}{GDEC}  \PYG{n}{DEST}  \PYG{n}{DSCA}
\PYG{o}{\PYGZhy{}}\PYG{o}{\PYGZhy{}}\PYG{o}{\PYGZhy{}}\PYG{o}{\PYGZhy{}}\PYG{o}{\PYGZhy{}}\PYG{o}{\PYGZhy{}}\PYG{o}{\PYGZhy{}}\PYG{o}{\PYGZhy{}}\PYG{o}{\PYGZhy{}}\PYG{o}{\PYGZhy{}}\PYG{o}{\PYGZhy{}}\PYG{o}{\PYGZhy{}}\PYG{o}{\PYGZhy{}}\PYG{o}{\PYGZhy{}}\PYG{o}{\PYGZhy{}}\PYG{o}{\PYGZhy{}}\PYG{o}{\PYGZhy{}}\PYG{o}{\PYGZhy{}}\PYG{o}{\PYGZhy{}}\PYG{o}{\PYGZhy{}}\PYG{o}{\PYGZhy{}}\PYG{o}{\PYGZhy{}}\PYG{o}{\PYGZhy{}}\PYG{o}{\PYGZhy{}}\PYG{o}{\PYGZhy{}}\PYG{o}{\PYGZhy{}}\PYG{o}{\PYGZhy{}}\PYG{o}{\PYGZhy{}}\PYG{o}{\PYGZhy{}}
\PYG{l+m+mi}{1}     \PYG{n}{E20}   \PYG{o}{\PYGZhy{}}\PYG{l+m+mi}{4}\PYG{o}{\PYGZhy{}}\PYG{o}{\PYGZhy{}}  \PYG{n}{PF1}   \PYG{n}{D200}
\PYG{l+m+mi}{2}     \PYG{n}{EEE}   \PYG{o}{\PYGZhy{}}\PYG{o}{\PYGZhy{}}\PYG{o}{\PYGZhy{}}\PYG{o}{\PYGZhy{}}  \PYG{n}{PF2}   \PYG{n}{D200}

\PYG{n}{MNEM}  \PYG{n}{GCOD}  \PYG{n}{GDEC}  \PYG{n}{DEST}  \PYG{n}{DSCA}
\PYG{o}{\PYGZhy{}}\PYG{o}{\PYGZhy{}}\PYG{o}{\PYGZhy{}}\PYG{o}{\PYGZhy{}}\PYG{o}{\PYGZhy{}}\PYG{o}{\PYGZhy{}}\PYG{o}{\PYGZhy{}}\PYG{o}{\PYGZhy{}}\PYG{o}{\PYGZhy{}}\PYG{o}{\PYGZhy{}}\PYG{o}{\PYGZhy{}}\PYG{o}{\PYGZhy{}}\PYG{o}{\PYGZhy{}}\PYG{o}{\PYGZhy{}}\PYG{o}{\PYGZhy{}}\PYG{o}{\PYGZhy{}}\PYG{o}{\PYGZhy{}}\PYG{o}{\PYGZhy{}}\PYG{o}{\PYGZhy{}}\PYG{o}{\PYGZhy{}}\PYG{o}{\PYGZhy{}}\PYG{o}{\PYGZhy{}}\PYG{o}{\PYGZhy{}}\PYG{o}{\PYGZhy{}}\PYG{o}{\PYGZhy{}}\PYG{o}{\PYGZhy{}}\PYG{o}{\PYGZhy{}}\PYG{o}{\PYGZhy{}}\PYG{o}{\PYGZhy{}}
\PYG{l+m+mi}{1}     \PYG{n}{EEE}   \PYG{o}{\PYGZhy{}}\PYG{o}{\PYGZhy{}}\PYG{o}{\PYGZhy{}}\PYG{o}{\PYGZhy{}}  \PYG{n}{PF1}   \PYG{n}{D200}
\PYG{l+m+mi}{2}     \PYG{n}{E1E}   \PYG{o}{\PYGZhy{}}\PYG{l+m+mi}{4}\PYG{o}{\PYGZhy{}}   \PYG{n}{PF2}   \PYG{n}{D200}

\PYG{n}{Table} \PYG{n}{record} \PYG{p}{(}\PYG{n+nb}{type} \PYG{l+m+mi}{34}\PYG{p}{)} \PYG{n+nb}{type}\PYG{p}{:} \PYG{n}{FILM}

\PYG{n}{MNEM}  \PYG{n}{GCOD}  \PYG{n}{GDEC}  \PYG{n}{DEST}  \PYG{n}{DSCA}
\PYG{o}{\PYGZhy{}}\PYG{o}{\PYGZhy{}}\PYG{o}{\PYGZhy{}}\PYG{o}{\PYGZhy{}}\PYG{o}{\PYGZhy{}}\PYG{o}{\PYGZhy{}}\PYG{o}{\PYGZhy{}}\PYG{o}{\PYGZhy{}}\PYG{o}{\PYGZhy{}}\PYG{o}{\PYGZhy{}}\PYG{o}{\PYGZhy{}}\PYG{o}{\PYGZhy{}}\PYG{o}{\PYGZhy{}}\PYG{o}{\PYGZhy{}}\PYG{o}{\PYGZhy{}}\PYG{o}{\PYGZhy{}}\PYG{o}{\PYGZhy{}}\PYG{o}{\PYGZhy{}}\PYG{o}{\PYGZhy{}}\PYG{o}{\PYGZhy{}}\PYG{o}{\PYGZhy{}}\PYG{o}{\PYGZhy{}}\PYG{o}{\PYGZhy{}}\PYG{o}{\PYGZhy{}}\PYG{o}{\PYGZhy{}}\PYG{o}{\PYGZhy{}}\PYG{o}{\PYGZhy{}}\PYG{o}{\PYGZhy{}}\PYG{o}{\PYGZhy{}}

\PYG{n}{D}     \PYG{n}{E3E}   \PYG{o}{\PYGZhy{}}\PYG{l+m+mi}{3}\PYG{o}{\PYGZhy{}}   \PYG{n}{PFD}   \PYG{n}{D200}
\PYG{n}{E}     \PYG{n}{E3E}   \PYG{o}{\PYGZhy{}}\PYG{l+m+mi}{3}\PYG{o}{\PYGZhy{}}   \PYG{n}{PFE}   \PYG{n}{D500}
\PYG{l+m+mi}{5}     \PYG{n}{EB0}   \PYG{o}{\PYGZhy{}}\PYG{o}{\PYGZhy{}}\PYG{o}{\PYGZhy{}}   \PYG{n}{PF5}   \PYG{n}{D200}
\PYG{l+m+mi}{6}     \PYG{n}{EEB}   \PYG{o}{\PYGZhy{}}\PYG{o}{\PYGZhy{}}\PYG{o}{\PYGZhy{}}   \PYG{n}{PF6}   \PYG{n}{D200}

\PYG{n}{Table} \PYG{n}{record} \PYG{p}{(}\PYG{n+nb}{type} \PYG{l+m+mi}{34}\PYG{p}{)} \PYG{n+nb}{type}\PYG{p}{:} \PYG{n}{FILM}

\PYG{n}{MNEM}  \PYG{n}{GCOD}  \PYG{n}{GDEC}  \PYG{n}{DEST}  \PYG{n}{DSCA}
\PYG{o}{\PYGZhy{}}\PYG{o}{\PYGZhy{}}\PYG{o}{\PYGZhy{}}\PYG{o}{\PYGZhy{}}\PYG{o}{\PYGZhy{}}\PYG{o}{\PYGZhy{}}\PYG{o}{\PYGZhy{}}\PYG{o}{\PYGZhy{}}\PYG{o}{\PYGZhy{}}\PYG{o}{\PYGZhy{}}\PYG{o}{\PYGZhy{}}\PYG{o}{\PYGZhy{}}\PYG{o}{\PYGZhy{}}\PYG{o}{\PYGZhy{}}\PYG{o}{\PYGZhy{}}\PYG{o}{\PYGZhy{}}\PYG{o}{\PYGZhy{}}\PYG{o}{\PYGZhy{}}\PYG{o}{\PYGZhy{}}\PYG{o}{\PYGZhy{}}\PYG{o}{\PYGZhy{}}\PYG{o}{\PYGZhy{}}\PYG{o}{\PYGZhy{}}\PYG{o}{\PYGZhy{}}\PYG{o}{\PYGZhy{}}\PYG{o}{\PYGZhy{}}\PYG{o}{\PYGZhy{}}\PYG{o}{\PYGZhy{}}\PYG{o}{\PYGZhy{}}

\PYG{l+m+mi}{8}     \PYG{n}{EB0}   \PYG{o}{\PYGZhy{}}\PYG{o}{\PYGZhy{}}\PYG{o}{\PYGZhy{}}   \PYG{n}{PF8}   \PYG{n}{D200}
\PYG{n}{A}     \PYG{n}{LLLL}  \PYG{l+m+mi}{1111}  \PYG{n}{PFA}   \PYG{n}{DM}
\PYG{n}{E}     \PYG{n}{E4E}   \PYG{o}{\PYGZhy{}}\PYG{l+m+mi}{4}\PYG{o}{\PYGZhy{}}   \PYG{n}{PFE}   \PYG{n}{D200}
\PYG{n}{K}     \PYG{n}{E4E}   \PYG{o}{\PYGZhy{}}\PYG{l+m+mi}{4}\PYG{o}{\PYGZhy{}}   \PYG{n}{PFK}   \PYG{n}{D500}

\PYG{n}{Table} \PYG{n}{record} \PYG{p}{(}\PYG{n+nb}{type} \PYG{l+m+mi}{34}\PYG{p}{)} \PYG{n+nb}{type}\PYG{p}{:} \PYG{n}{FILM}

\PYG{n}{MNEM}      \PYG{n}{CINT}      \PYG{n}{GCOD}  \PYG{n}{GDEC}  \PYG{n}{DEST}  \PYG{n}{DSCA}
\PYG{o}{\PYGZhy{}}\PYG{o}{\PYGZhy{}}\PYG{o}{\PYGZhy{}}\PYG{o}{\PYGZhy{}}\PYG{o}{\PYGZhy{}}\PYG{o}{\PYGZhy{}}\PYG{o}{\PYGZhy{}}\PYG{o}{\PYGZhy{}}\PYG{o}{\PYGZhy{}}\PYG{o}{\PYGZhy{}}\PYG{o}{\PYGZhy{}}\PYG{o}{\PYGZhy{}}\PYG{o}{\PYGZhy{}}\PYG{o}{\PYGZhy{}}\PYG{o}{\PYGZhy{}}\PYG{o}{\PYGZhy{}}\PYG{o}{\PYGZhy{}}\PYG{o}{\PYGZhy{}}\PYG{o}{\PYGZhy{}}\PYG{o}{\PYGZhy{}}\PYG{o}{\PYGZhy{}}\PYG{o}{\PYGZhy{}}\PYG{o}{\PYGZhy{}}\PYG{o}{\PYGZhy{}}\PYG{o}{\PYGZhy{}}\PYG{o}{\PYGZhy{}}\PYG{o}{\PYGZhy{}}\PYG{o}{\PYGZhy{}}\PYG{o}{\PYGZhy{}}\PYG{o}{\PYGZhy{}}\PYG{o}{\PYGZhy{}}\PYG{o}{\PYGZhy{}}\PYG{o}{\PYGZhy{}}\PYG{o}{\PYGZhy{}}\PYG{o}{\PYGZhy{}}\PYG{o}{\PYGZhy{}}\PYG{o}{\PYGZhy{}}\PYG{o}{\PYGZhy{}}\PYG{o}{\PYGZhy{}}\PYG{o}{\PYGZhy{}}\PYG{o}{\PYGZhy{}}\PYG{o}{\PYGZhy{}}\PYG{o}{\PYGZhy{}}

\PYG{l+m+mi}{1}          \PYG{l+m+mf}{0.00000}  \PYG{n}{E2E}   \PYG{o}{\PYGZhy{}}\PYG{l+m+mi}{2}\PYG{o}{\PYGZhy{}}   \PYG{n}{PF1}   \PYG{n}{D200}
\PYG{l+m+mi}{2}          \PYG{l+m+mf}{0.00000}  \PYG{n}{E2E}   \PYG{o}{\PYGZhy{}}\PYG{l+m+mi}{1}\PYG{o}{\PYGZhy{}}   \PYG{n}{PF2}   \PYG{n}{D500}
\PYG{l+m+mi}{3}          \PYG{l+m+mf}{0.00000}  \PYG{n}{EEE}   \PYG{o}{\PYGZhy{}}\PYG{o}{\PYGZhy{}}\PYG{o}{\PYGZhy{}}\PYG{o}{\PYGZhy{}}  \PYG{n}{NEIT}  \PYG{n}{S5}
\PYG{l+m+mi}{4}          \PYG{l+m+mf}{0.00000}  \PYG{n}{EEE}   \PYG{o}{\PYGZhy{}}\PYG{o}{\PYGZhy{}}\PYG{o}{\PYGZhy{}}\PYG{o}{\PYGZhy{}}  \PYG{n}{NEIT}  \PYG{n}{S5}

\PYG{n}{Table} \PYG{n}{record} \PYG{p}{(}\PYG{n+nb}{type} \PYG{l+m+mi}{34}\PYG{p}{)} \PYG{n+nb}{type}\PYG{p}{:} \PYG{n}{FILM}

\PYG{n}{MNEM}      \PYG{n}{CINT}      \PYG{n}{GCOD}  \PYG{n}{GDEC}  \PYG{n}{DEST}  \PYG{n}{DSCA}
\PYG{o}{\PYGZhy{}}\PYG{o}{\PYGZhy{}}\PYG{o}{\PYGZhy{}}\PYG{o}{\PYGZhy{}}\PYG{o}{\PYGZhy{}}\PYG{o}{\PYGZhy{}}\PYG{o}{\PYGZhy{}}\PYG{o}{\PYGZhy{}}\PYG{o}{\PYGZhy{}}\PYG{o}{\PYGZhy{}}\PYG{o}{\PYGZhy{}}\PYG{o}{\PYGZhy{}}\PYG{o}{\PYGZhy{}}\PYG{o}{\PYGZhy{}}\PYG{o}{\PYGZhy{}}\PYG{o}{\PYGZhy{}}\PYG{o}{\PYGZhy{}}\PYG{o}{\PYGZhy{}}\PYG{o}{\PYGZhy{}}\PYG{o}{\PYGZhy{}}\PYG{o}{\PYGZhy{}}\PYG{o}{\PYGZhy{}}\PYG{o}{\PYGZhy{}}\PYG{o}{\PYGZhy{}}\PYG{o}{\PYGZhy{}}\PYG{o}{\PYGZhy{}}\PYG{o}{\PYGZhy{}}\PYG{o}{\PYGZhy{}}\PYG{o}{\PYGZhy{}}\PYG{o}{\PYGZhy{}}\PYG{o}{\PYGZhy{}}\PYG{o}{\PYGZhy{}}\PYG{o}{\PYGZhy{}}\PYG{o}{\PYGZhy{}}\PYG{o}{\PYGZhy{}}\PYG{o}{\PYGZhy{}}\PYG{o}{\PYGZhy{}}\PYG{o}{\PYGZhy{}}\PYG{o}{\PYGZhy{}}\PYG{o}{\PYGZhy{}}\PYG{o}{\PYGZhy{}}\PYG{o}{\PYGZhy{}}\PYG{o}{\PYGZhy{}}

\PYG{l+m+mi}{1}          \PYG{l+m+mf}{300.000}  \PYG{n}{E2E}   \PYG{o}{\PYGZhy{}}\PYG{l+m+mi}{2}\PYG{o}{\PYGZhy{}}   \PYG{n}{PF1}   \PYG{n}{D200}
\PYG{l+m+mi}{2}          \PYG{l+m+mf}{300.000}  \PYG{n}{E2E}   \PYG{o}{\PYGZhy{}}\PYG{l+m+mi}{2}\PYG{o}{\PYGZhy{}}   \PYG{n}{PF2}   \PYG{n}{D500}
\PYG{l+m+mi}{3}          \PYG{l+m+mf}{300.000}  \PYG{n}{EEE}   \PYG{o}{\PYGZhy{}}\PYG{o}{\PYGZhy{}}\PYG{o}{\PYGZhy{}}\PYG{o}{\PYGZhy{}}  \PYG{n}{NEIT}  \PYG{n}{S5}
\PYG{l+m+mi}{4}          \PYG{l+m+mf}{300.000}  \PYG{n}{EEE}   \PYG{o}{\PYGZhy{}}\PYG{o}{\PYGZhy{}}\PYG{o}{\PYGZhy{}}\PYG{o}{\PYGZhy{}}  \PYG{n}{NEIT}  \PYG{n}{S5}

\PYG{n}{Table} \PYG{n}{record} \PYG{p}{(}\PYG{n+nb}{type} \PYG{l+m+mi}{34}\PYG{p}{)} \PYG{n+nb}{type}\PYG{p}{:} \PYG{n}{FILM}

\PYG{n}{MNEM}      \PYG{n}{CINT}      \PYG{n}{GCOD}  \PYG{n}{GDEC}  \PYG{n}{DEST}  \PYG{n}{DSCA}
\PYG{o}{\PYGZhy{}}\PYG{o}{\PYGZhy{}}\PYG{o}{\PYGZhy{}}\PYG{o}{\PYGZhy{}}\PYG{o}{\PYGZhy{}}\PYG{o}{\PYGZhy{}}\PYG{o}{\PYGZhy{}}\PYG{o}{\PYGZhy{}}\PYG{o}{\PYGZhy{}}\PYG{o}{\PYGZhy{}}\PYG{o}{\PYGZhy{}}\PYG{o}{\PYGZhy{}}\PYG{o}{\PYGZhy{}}\PYG{o}{\PYGZhy{}}\PYG{o}{\PYGZhy{}}\PYG{o}{\PYGZhy{}}\PYG{o}{\PYGZhy{}}\PYG{o}{\PYGZhy{}}\PYG{o}{\PYGZhy{}}\PYG{o}{\PYGZhy{}}\PYG{o}{\PYGZhy{}}\PYG{o}{\PYGZhy{}}\PYG{o}{\PYGZhy{}}\PYG{o}{\PYGZhy{}}\PYG{o}{\PYGZhy{}}\PYG{o}{\PYGZhy{}}\PYG{o}{\PYGZhy{}}\PYG{o}{\PYGZhy{}}\PYG{o}{\PYGZhy{}}\PYG{o}{\PYGZhy{}}\PYG{o}{\PYGZhy{}}\PYG{o}{\PYGZhy{}}\PYG{o}{\PYGZhy{}}\PYG{o}{\PYGZhy{}}\PYG{o}{\PYGZhy{}}\PYG{o}{\PYGZhy{}}\PYG{o}{\PYGZhy{}}\PYG{o}{\PYGZhy{}}\PYG{o}{\PYGZhy{}}\PYG{o}{\PYGZhy{}}\PYG{o}{\PYGZhy{}}\PYG{o}{\PYGZhy{}}\PYG{o}{\PYGZhy{}}

\PYG{l+m+mi}{1}          \PYG{l+m+mf}{50.0000}  \PYG{n}{EEE}   \PYG{o}{\PYGZhy{}}\PYG{o}{\PYGZhy{}}\PYG{o}{\PYGZhy{}}\PYG{o}{\PYGZhy{}}  \PYG{n}{PF1}   \PYG{n}{D200}
\PYG{l+m+mi}{2}          \PYG{l+m+mf}{0.00000}  \PYG{n}{EEE}   \PYG{o}{\PYGZhy{}}\PYG{o}{\PYGZhy{}}\PYG{o}{\PYGZhy{}}\PYG{o}{\PYGZhy{}}  \PYG{n}{PF2}   \PYG{n}{D200}
\PYG{l+m+mi}{3}          \PYG{l+m+mf}{0.00000}  \PYG{n}{EEE}   \PYG{o}{\PYGZhy{}}\PYG{o}{\PYGZhy{}}\PYG{o}{\PYGZhy{}}\PYG{o}{\PYGZhy{}}  \PYG{n}{NEIT}  \PYG{n}{D200}
\PYG{l+m+mi}{4}          \PYG{l+m+mf}{0.00000}  \PYG{n}{EEE}   \PYG{o}{\PYGZhy{}}\PYG{o}{\PYGZhy{}}\PYG{o}{\PYGZhy{}}\PYG{o}{\PYGZhy{}}  \PYG{n}{NEIT}  \PYG{n}{D200}
\end{sphinxVerbatim}

Minimal, but not complete interpretation:

Ignore GDEC as dupe.
\begin{itemize}
\item {} 
E - Equi-spaced (linear).

\item {} 
n - Log with number of decades.

\item {} 
B - Blank.

\item {} 
L - ?

\end{itemize}

What to do with 0 (continuation?).
Examples: E20   -4\textendash{} means 4 decades over track 23.

Four tracks, from 200099.S07:

\begin{sphinxVerbatim}[commandchars=\\\{\}]
\PYG{n}{Table} \PYG{n}{record} \PYG{p}{(}\PYG{n+nb}{type} \PYG{l+m+mi}{34}\PYG{p}{)} \PYG{n+nb}{type}\PYG{p}{:} \PYG{n}{FILM}
\PYG{n}{MNEM}  \PYG{n}{GCOD}  \PYG{n}{GDEC}  \PYG{n}{DEST}  \PYG{n}{DSCA}
\PYG{o}{\PYGZhy{}}\PYG{o}{\PYGZhy{}}\PYG{o}{\PYGZhy{}}\PYG{o}{\PYGZhy{}}\PYG{o}{\PYGZhy{}}\PYG{o}{\PYGZhy{}}\PYG{o}{\PYGZhy{}}\PYG{o}{\PYGZhy{}}\PYG{o}{\PYGZhy{}}\PYG{o}{\PYGZhy{}}\PYG{o}{\PYGZhy{}}\PYG{o}{\PYGZhy{}}\PYG{o}{\PYGZhy{}}\PYG{o}{\PYGZhy{}}\PYG{o}{\PYGZhy{}}\PYG{o}{\PYGZhy{}}\PYG{o}{\PYGZhy{}}\PYG{o}{\PYGZhy{}}\PYG{o}{\PYGZhy{}}\PYG{o}{\PYGZhy{}}\PYG{o}{\PYGZhy{}}\PYG{o}{\PYGZhy{}}\PYG{o}{\PYGZhy{}}\PYG{o}{\PYGZhy{}}\PYG{o}{\PYGZhy{}}\PYG{o}{\PYGZhy{}}\PYG{o}{\PYGZhy{}}\PYG{o}{\PYGZhy{}}\PYG{o}{\PYGZhy{}}
\PYG{o}{.}\PYG{o}{.}\PYG{o}{.}
\PYG{n}{A}     \PYG{n}{LLLL}  \PYG{l+m+mi}{1111}  \PYG{n}{PFA}   \PYG{n}{DM}  
\PYG{o}{.}\PYG{o}{.}\PYG{o}{.}

\PYG{n}{Table} \PYG{n}{record} \PYG{p}{(}\PYG{n+nb}{type} \PYG{l+m+mi}{34}\PYG{p}{)} \PYG{n+nb}{type}\PYG{p}{:} \PYG{n}{PRES}
\PYG{n}{MNEM}  \PYG{n}{OUTP}  \PYG{n}{STAT}  \PYG{n}{TRAC}  \PYG{n}{CODI}  \PYG{n}{DEST}  \PYG{n}{MODE}      \PYG{n}{FILT}          \PYG{n}{LEDG}          \PYG{n}{REDG}      \PYG{n}{COLO}
\PYG{o}{\PYGZhy{}}\PYG{o}{\PYGZhy{}}\PYG{o}{\PYGZhy{}}\PYG{o}{\PYGZhy{}}\PYG{o}{\PYGZhy{}}\PYG{o}{\PYGZhy{}}\PYG{o}{\PYGZhy{}}\PYG{o}{\PYGZhy{}}\PYG{o}{\PYGZhy{}}\PYG{o}{\PYGZhy{}}\PYG{o}{\PYGZhy{}}\PYG{o}{\PYGZhy{}}\PYG{o}{\PYGZhy{}}\PYG{o}{\PYGZhy{}}\PYG{o}{\PYGZhy{}}\PYG{o}{\PYGZhy{}}\PYG{o}{\PYGZhy{}}\PYG{o}{\PYGZhy{}}\PYG{o}{\PYGZhy{}}\PYG{o}{\PYGZhy{}}\PYG{o}{\PYGZhy{}}\PYG{o}{\PYGZhy{}}\PYG{o}{\PYGZhy{}}\PYG{o}{\PYGZhy{}}\PYG{o}{\PYGZhy{}}\PYG{o}{\PYGZhy{}}\PYG{o}{\PYGZhy{}}\PYG{o}{\PYGZhy{}}\PYG{o}{\PYGZhy{}}\PYG{o}{\PYGZhy{}}\PYG{o}{\PYGZhy{}}\PYG{o}{\PYGZhy{}}\PYG{o}{\PYGZhy{}}\PYG{o}{\PYGZhy{}}\PYG{o}{\PYGZhy{}}\PYG{o}{\PYGZhy{}}\PYG{o}{\PYGZhy{}}\PYG{o}{\PYGZhy{}}\PYG{o}{\PYGZhy{}}\PYG{o}{\PYGZhy{}}\PYG{o}{\PYGZhy{}}\PYG{o}{\PYGZhy{}}\PYG{o}{\PYGZhy{}}\PYG{o}{\PYGZhy{}}\PYG{o}{\PYGZhy{}}\PYG{o}{\PYGZhy{}}\PYG{o}{\PYGZhy{}}\PYG{o}{\PYGZhy{}}\PYG{o}{\PYGZhy{}}\PYG{o}{\PYGZhy{}}\PYG{o}{\PYGZhy{}}\PYG{o}{\PYGZhy{}}\PYG{o}{\PYGZhy{}}\PYG{o}{\PYGZhy{}}\PYG{o}{\PYGZhy{}}\PYG{o}{\PYGZhy{}}\PYG{o}{\PYGZhy{}}\PYG{o}{\PYGZhy{}}\PYG{o}{\PYGZhy{}}\PYG{o}{\PYGZhy{}}\PYG{o}{\PYGZhy{}}\PYG{o}{\PYGZhy{}}\PYG{o}{\PYGZhy{}}\PYG{o}{\PYGZhy{}}\PYG{o}{\PYGZhy{}}\PYG{o}{\PYGZhy{}}\PYG{o}{\PYGZhy{}}\PYG{o}{\PYGZhy{}}\PYG{o}{\PYGZhy{}}\PYG{o}{\PYGZhy{}}\PYG{o}{\PYGZhy{}}\PYG{o}{\PYGZhy{}}\PYG{o}{\PYGZhy{}}\PYG{o}{\PYGZhy{}}\PYG{o}{\PYGZhy{}}\PYG{o}{\PYGZhy{}}\PYG{o}{\PYGZhy{}}\PYG{o}{\PYGZhy{}}\PYG{o}{\PYGZhy{}}\PYG{o}{\PYGZhy{}}\PYG{o}{\PYGZhy{}}\PYG{o}{\PYGZhy{}}\PYG{o}{\PYGZhy{}}\PYG{o}{\PYGZhy{}}\PYG{o}{\PYGZhy{}}\PYG{o}{\PYGZhy{}}\PYG{o}{\PYGZhy{}}\PYG{o}{\PYGZhy{}}\PYG{o}{\PYGZhy{}}
\PYG{o}{.}\PYG{o}{.}\PYG{o}{.}
\PYG{n}{AX1A}  \PYG{n}{AX}    \PYG{n}{ALLO}  \PYG{n}{F2}    \PYG{n}{LLIN}  \PYG{n}{A}     \PYG{n}{SHIF}      \PYG{l+m+mf}{0.500000}      \PYG{o}{\PYGZhy{}}\PYG{l+m+mf}{9.00000}       \PYG{l+m+mf}{9.00000}  \PYG{l+m+mi}{400} 
\PYG{n}{AY1A}  \PYG{n}{AY}    \PYG{n}{ALLO}  \PYG{n}{F2}    \PYG{n}{LSPO}  \PYG{n}{A}     \PYG{n}{SHIF}      \PYG{l+m+mf}{0.500000}      \PYG{o}{\PYGZhy{}}\PYG{l+m+mf}{9.00000}       \PYG{l+m+mf}{9.00000}  \PYG{l+m+mi}{400} 
\PYG{n}{AZ1A}  \PYG{n}{AZ}    \PYG{n}{ALLO}  \PYG{n}{F2}    \PYG{n}{LGAP}  \PYG{n}{A}     \PYG{n}{NB}        \PYG{l+m+mf}{0.500000}      \PYG{o}{\PYGZhy{}}\PYG{l+m+mf}{9.00000}       \PYG{l+m+mf}{9.00000}  \PYG{l+m+mi}{400} 
\PYG{n}{AN1A}  \PYG{n}{ANOR}  \PYG{n}{ALLO}  \PYG{n}{F4}    \PYG{n}{HDAS}  \PYG{n}{A}     \PYG{n}{NB}        \PYG{l+m+mf}{0.500000}       \PYG{l+m+mf}{9.00000}       \PYG{l+m+mf}{11.0000}  \PYG{l+m+mi}{420} 
\PYG{n}{CS1A}  \PYG{n}{CS}    \PYG{n}{DISA}  \PYG{n}{F1}    \PYG{n}{LDAS}  \PYG{n}{A}     \PYG{n}{NB}        \PYG{l+m+mf}{0.500000}       \PYG{l+m+mf}{0.00000}       \PYG{l+m+mf}{150000.}  \PYG{l+m+mi}{000} 
\PYG{n}{FX1A}  \PYG{n}{FX}    \PYG{n}{ALLO}  \PYG{n}{F3}    \PYG{n}{LLIN}  \PYG{n}{A}     \PYG{n}{NB}        \PYG{l+m+mf}{0.500000}     \PYG{o}{\PYGZhy{}}\PYG{l+m+mf}{0.700000}      \PYG{l+m+mf}{0.700000}  \PYG{l+m+mi}{003} 
\PYG{n}{FY1A}  \PYG{n}{FY}    \PYG{n}{ALLO}  \PYG{n}{F3}    \PYG{n}{LGAP}  \PYG{n}{A}     \PYG{n}{NB}        \PYG{l+m+mf}{0.500000}     \PYG{o}{\PYGZhy{}}\PYG{l+m+mf}{0.700000}      \PYG{l+m+mf}{0.700000}  \PYG{l+m+mi}{003} 
\PYG{n}{FZ1A}  \PYG{n}{FZ}    \PYG{n}{ALLO}  \PYG{n}{F3}    \PYG{n}{LSPO}  \PYG{n}{A}     \PYG{n}{NB}        \PYG{l+m+mf}{0.500000}     \PYG{o}{\PYGZhy{}}\PYG{l+m+mf}{0.700000}      \PYG{l+m+mf}{0.700000}  \PYG{l+m+mi}{003} 
\PYG{n}{FN1A}  \PYG{n}{FNOR}  \PYG{n}{ALLO}  \PYG{n}{F4}    \PYG{n}{LLIN}  \PYG{n}{A}     \PYG{n}{NB}        \PYG{l+m+mf}{0.500000}      \PYG{l+m+mf}{0.200000}      \PYG{l+m+mf}{0.700000}  \PYG{l+m+mi}{420} 
\PYG{n}{FI1A}  \PYG{n}{FINC}  \PYG{n}{ALLO}  \PYG{n}{F4}    \PYG{n}{LGAP}  \PYG{n}{A}     \PYG{n}{NB}         \PYG{l+m+mf}{1.00000}       \PYG{l+m+mf}{0.00000}       \PYG{l+m+mf}{90.0000}  \PYG{l+m+mi}{420} 
\PYG{n}{GA1A}  \PYG{n}{GADZ}  \PYG{n}{DISA}  \PYG{n}{F1}    \PYG{n}{LLIN}  \PYG{n}{A}     \PYG{n}{NB}         \PYG{l+m+mf}{1.00000}      \PYG{o}{\PYGZhy{}}\PYG{l+m+mf}{1.00000}       \PYG{l+m+mf}{1.00000}  \PYG{l+m+mi}{020} 
\PYG{n}{GP1A}  \PYG{n}{GPV}   \PYG{n}{DISA}  \PYG{n}{F1}    \PYG{n}{LGAP}  \PYG{n}{A}     \PYG{n}{NB}         \PYG{l+m+mf}{1.00000}       \PYG{l+m+mf}{14.0000}       \PYG{l+m+mf}{16.0000}  \PYG{l+m+mi}{020} 
\PYG{n}{GN1A}  \PYG{n}{GNV}   \PYG{n}{DISA}  \PYG{n}{F1}    \PYG{n}{LDAS}  \PYG{n}{A}     \PYG{n}{NB}         \PYG{l+m+mf}{1.00000}      \PYG{o}{\PYGZhy{}}\PYG{l+m+mf}{16.0000}      \PYG{o}{\PYGZhy{}}\PYG{l+m+mf}{14.0000}  \PYG{l+m+mi}{020} 
\PYG{n}{GM1A}  \PYG{n}{GMT}   \PYG{n}{DISA}  \PYG{n}{F3}    \PYG{n}{HDAS}  \PYG{n}{A}     \PYG{n}{SHIF}       \PYG{l+m+mf}{1.00000}       \PYG{l+m+mf}{80.0000}       \PYG{l+m+mf}{130.000}  \PYG{l+m+mi}{020} 
\PYG{n}{GA2A}  \PYG{n}{GAT}   \PYG{n}{DISA}  \PYG{n}{F2}    \PYG{n}{HDAS}  \PYG{n}{A}     \PYG{n}{SHIF}       \PYG{l+m+mf}{1.00000}       \PYG{l+m+mf}{80.0000}       \PYG{l+m+mf}{130.000}  \PYG{l+m+mi}{020} 
\PYG{n}{SI1A}  \PYG{n}{SILO}  \PYG{n}{DISA}  \PYG{n}{FD}    \PYG{n}{HLIN}  \PYG{n}{A}     \PYG{n}{NB}         \PYG{l+m+mf}{1.00000}       \PYG{l+m+mf}{0.00000}       \PYG{l+m+mf}{20.0000}  \PYG{l+m+mi}{000} 
\PYG{n}{ST1A}  \PYG{n}{STIT}  \PYG{n}{DISA}  \PYG{n}{FD}    \PYG{n}{LLIN}  \PYG{n}{A}     \PYG{n}{NB}         \PYG{l+m+mf}{1.00000}       \PYG{l+m+mf}{0.00000}       \PYG{l+m+mf}{20.0000}  \PYG{l+m+mi}{000} 
\PYG{n}{ST2A}  \PYG{n}{STIA}  \PYG{n}{ALLO}  \PYG{n}{FD}    \PYG{n}{LLIN}  \PYG{n}{A}     \PYG{n}{NB}         \PYG{l+m+mf}{1.00000}       \PYG{l+m+mf}{0.00000}       \PYG{l+m+mf}{20.0000}  \PYG{l+m+mi}{000} 
\PYG{n}{TE1A}  \PYG{n}{TENS}  \PYG{n}{DISA}  \PYG{n}{FD}    \PYG{n}{LDAS}  \PYG{n}{A}     \PYG{n}{WRAP}      \PYG{l+m+mf}{0.500000}       \PYG{l+m+mf}{2000.00}       \PYG{l+m+mf}{7000.00}  \PYG{l+m+mi}{000} 
\PYG{o}{.}\PYG{o}{.}\PYG{o}{.}
\end{sphinxVerbatim}

TODO: What about CINT and FORM headings?

2011-05-27 Frequency Analysis done on:

\begin{sphinxVerbatim}[commandchars=\\\{\}]
\PYGZdl{} python3 TableHistogram.py \PYGZhy{}k \PYGZhy{}\PYGZhy{}name=FILM \PYGZhy{}l40 \PYGZhy{}\PYGZhy{}col=DEST ../../../pLogicTestData/LIS/
Cmd: TableHistogram.py \PYGZhy{}k \PYGZhy{}\PYGZhy{}name=FILM \PYGZhy{}l40 \PYGZhy{}\PYGZhy{}col=DEST ../../../pLogicTestData/LIS/
2011\PYGZhy{}05\PYGZhy{}27 09:23:23,870 ERROR    Can not read LIS file ../../../pLogicTestData/LIS/13576.S1 with error: Can not fit integer number of frames length 120 into LR length 824, modulo 104 [indirect size 0].
2011\PYGZhy{}05\PYGZhy{}27 09:23:24,649 ERROR    Can not read LIS file ../../../pLogicTestData/LIS/13610.S1 with error: Can not fit integer number of frames length 7176 into LR length 13354, modulo 6178 [indirect size 0].
...
\end{sphinxVerbatim}

GCOD:

\begin{sphinxVerbatim}[commandchars=\\\{\}]
\PYG{o}{==}\PYG{o}{==}\PYG{o}{==}\PYG{o}{==}\PYG{o}{==}\PYG{o}{==}\PYG{o}{==}\PYG{o}{==}\PYG{o}{==}\PYG{o}{==}\PYG{o}{==}\PYG{o}{==} \PYG{n}{Count} \PYG{n}{of} \PYG{n+nb}{all} \PYG{n}{table} \PYG{n}{entries} \PYG{o}{==}\PYG{o}{==}\PYG{o}{==}\PYG{o}{==}\PYG{o}{==}\PYG{o}{==}\PYG{o}{==}\PYG{o}{==}\PYG{o}{==}\PYG{o}{==}\PYG{o}{==}\PYG{o}{=}
\PYG{p}{\PYGZob{}}
\PYG{l+s+s2}{\PYGZdq{}}\PYG{l+s+s2}{(34, b}\PYG{l+s+s2}{\PYGZsq{}}\PYG{l+s+s2}{FILM}\PYG{l+s+s2}{\PYGZsq{}}\PYG{l+s+s2}{, b}\PYG{l+s+s2}{\PYGZsq{}}\PYG{l+s+s2}{GCOD}\PYG{l+s+s2}{\PYGZsq{}}\PYG{l+s+s2}{, b}\PYG{l+s+s2}{\PYGZsq{}}\PYG{l+s+s2}{BBB }\PYG{l+s+s2}{\PYGZsq{}}\PYG{l+s+s2}{)}\PYG{l+s+s2}{\PYGZdq{}}\PYG{p}{:} \PYG{l+m+mi}{2}\PYG{p}{,}
\PYG{l+s+s2}{\PYGZdq{}}\PYG{l+s+s2}{(34, b}\PYG{l+s+s2}{\PYGZsq{}}\PYG{l+s+s2}{FILM}\PYG{l+s+s2}{\PYGZsq{}}\PYG{l+s+s2}{, b}\PYG{l+s+s2}{\PYGZsq{}}\PYG{l+s+s2}{GCOD}\PYG{l+s+s2}{\PYGZsq{}}\PYG{l+s+s2}{, b}\PYG{l+s+s2}{\PYGZsq{}}\PYG{l+s+s2}{E1E }\PYG{l+s+s2}{\PYGZsq{}}\PYG{l+s+s2}{)}\PYG{l+s+s2}{\PYGZdq{}}\PYG{p}{:} \PYG{l+m+mi}{5}\PYG{p}{,}
\PYG{l+s+s2}{\PYGZdq{}}\PYG{l+s+s2}{(34, b}\PYG{l+s+s2}{\PYGZsq{}}\PYG{l+s+s2}{FILM}\PYG{l+s+s2}{\PYGZsq{}}\PYG{l+s+s2}{, b}\PYG{l+s+s2}{\PYGZsq{}}\PYG{l+s+s2}{GCOD}\PYG{l+s+s2}{\PYGZsq{}}\PYG{l+s+s2}{, b}\PYG{l+s+s2}{\PYGZsq{}}\PYG{l+s+s2}{E20 }\PYG{l+s+s2}{\PYGZsq{}}\PYG{l+s+s2}{)}\PYG{l+s+s2}{\PYGZdq{}}\PYG{p}{:} \PYG{l+m+mi}{26}\PYG{p}{,}
\PYG{l+s+s2}{\PYGZdq{}}\PYG{l+s+s2}{(34, b}\PYG{l+s+s2}{\PYGZsq{}}\PYG{l+s+s2}{FILM}\PYG{l+s+s2}{\PYGZsq{}}\PYG{l+s+s2}{, b}\PYG{l+s+s2}{\PYGZsq{}}\PYG{l+s+s2}{GCOD}\PYG{l+s+s2}{\PYGZsq{}}\PYG{l+s+s2}{, b}\PYG{l+s+s2}{\PYGZsq{}}\PYG{l+s+s2}{E2E }\PYG{l+s+s2}{\PYGZsq{}}\PYG{l+s+s2}{)}\PYG{l+s+s2}{\PYGZdq{}}\PYG{p}{:} \PYG{l+m+mi}{40}\PYG{p}{,}
\PYG{l+s+s2}{\PYGZdq{}}\PYG{l+s+s2}{(34, b}\PYG{l+s+s2}{\PYGZsq{}}\PYG{l+s+s2}{FILM}\PYG{l+s+s2}{\PYGZsq{}}\PYG{l+s+s2}{, b}\PYG{l+s+s2}{\PYGZsq{}}\PYG{l+s+s2}{GCOD}\PYG{l+s+s2}{\PYGZsq{}}\PYG{l+s+s2}{, b}\PYG{l+s+s2}{\PYGZsq{}}\PYG{l+s+s2}{E3E }\PYG{l+s+s2}{\PYGZsq{}}\PYG{l+s+s2}{)}\PYG{l+s+s2}{\PYGZdq{}}\PYG{p}{:} \PYG{l+m+mi}{3}\PYG{p}{,}
\PYG{l+s+s2}{\PYGZdq{}}\PYG{l+s+s2}{(34, b}\PYG{l+s+s2}{\PYGZsq{}}\PYG{l+s+s2}{FILM}\PYG{l+s+s2}{\PYGZsq{}}\PYG{l+s+s2}{, b}\PYG{l+s+s2}{\PYGZsq{}}\PYG{l+s+s2}{GCOD}\PYG{l+s+s2}{\PYGZsq{}}\PYG{l+s+s2}{, b}\PYG{l+s+s2}{\PYGZsq{}}\PYG{l+s+s2}{E40 }\PYG{l+s+s2}{\PYGZsq{}}\PYG{l+s+s2}{)}\PYG{l+s+s2}{\PYGZdq{}}\PYG{p}{:} \PYG{l+m+mi}{2}\PYG{p}{,}
\PYG{l+s+s2}{\PYGZdq{}}\PYG{l+s+s2}{(34, b}\PYG{l+s+s2}{\PYGZsq{}}\PYG{l+s+s2}{FILM}\PYG{l+s+s2}{\PYGZsq{}}\PYG{l+s+s2}{, b}\PYG{l+s+s2}{\PYGZsq{}}\PYG{l+s+s2}{GCOD}\PYG{l+s+s2}{\PYGZsq{}}\PYG{l+s+s2}{, b}\PYG{l+s+s2}{\PYGZsq{}}\PYG{l+s+s2}{E4E }\PYG{l+s+s2}{\PYGZsq{}}\PYG{l+s+s2}{)}\PYG{l+s+s2}{\PYGZdq{}}\PYG{p}{:} \PYG{l+m+mi}{3}\PYG{p}{,}
\PYG{l+s+s2}{\PYGZdq{}}\PYG{l+s+s2}{(34, b}\PYG{l+s+s2}{\PYGZsq{}}\PYG{l+s+s2}{FILM}\PYG{l+s+s2}{\PYGZsq{}}\PYG{l+s+s2}{, b}\PYG{l+s+s2}{\PYGZsq{}}\PYG{l+s+s2}{GCOD}\PYG{l+s+s2}{\PYGZsq{}}\PYG{l+s+s2}{, b}\PYG{l+s+s2}{\PYGZsq{}}\PYG{l+s+s2}{EB0 }\PYG{l+s+s2}{\PYGZsq{}}\PYG{l+s+s2}{)}\PYG{l+s+s2}{\PYGZdq{}}\PYG{p}{:} \PYG{l+m+mi}{2}\PYG{p}{,}
\PYG{l+s+s2}{\PYGZdq{}}\PYG{l+s+s2}{(34, b}\PYG{l+s+s2}{\PYGZsq{}}\PYG{l+s+s2}{FILM}\PYG{l+s+s2}{\PYGZsq{}}\PYG{l+s+s2}{, b}\PYG{l+s+s2}{\PYGZsq{}}\PYG{l+s+s2}{GCOD}\PYG{l+s+s2}{\PYGZsq{}}\PYG{l+s+s2}{, b}\PYG{l+s+s2}{\PYGZsq{}}\PYG{l+s+s2}{EEB }\PYG{l+s+s2}{\PYGZsq{}}\PYG{l+s+s2}{)}\PYG{l+s+s2}{\PYGZdq{}}\PYG{p}{:} \PYG{l+m+mi}{22}\PYG{p}{,}
\PYG{l+s+s2}{\PYGZdq{}}\PYG{l+s+s2}{(34, b}\PYG{l+s+s2}{\PYGZsq{}}\PYG{l+s+s2}{FILM}\PYG{l+s+s2}{\PYGZsq{}}\PYG{l+s+s2}{, b}\PYG{l+s+s2}{\PYGZsq{}}\PYG{l+s+s2}{GCOD}\PYG{l+s+s2}{\PYGZsq{}}\PYG{l+s+s2}{, b}\PYG{l+s+s2}{\PYGZsq{}}\PYG{l+s+s2}{EEE }\PYG{l+s+s2}{\PYGZsq{}}\PYG{l+s+s2}{)}\PYG{l+s+s2}{\PYGZdq{}}\PYG{p}{:} \PYG{l+m+mi}{225}\PYG{p}{,}
\PYG{l+s+s2}{\PYGZdq{}}\PYG{l+s+s2}{(34, b}\PYG{l+s+s2}{\PYGZsq{}}\PYG{l+s+s2}{FILM}\PYG{l+s+s2}{\PYGZsq{}}\PYG{l+s+s2}{, b}\PYG{l+s+s2}{\PYGZsq{}}\PYG{l+s+s2}{GCOD}\PYG{l+s+s2}{\PYGZsq{}}\PYG{l+s+s2}{, b}\PYG{l+s+s2}{\PYGZsq{}}\PYG{l+s+s2}{LLLL}\PYG{l+s+s2}{\PYGZsq{}}\PYG{l+s+s2}{)}\PYG{l+s+s2}{\PYGZdq{}}\PYG{p}{:} \PYG{l+m+mi}{1}\PYG{p}{,}
\PYG{p}{\PYGZcb{}}
\PYG{o}{==}\PYG{o}{==}\PYG{o}{==}\PYG{o}{==}\PYG{o}{==}\PYG{o}{==}\PYG{o}{==}\PYG{o}{==}\PYG{o}{==}\PYG{o}{==}\PYG{o}{==} \PYG{n}{Count} \PYG{n}{of} \PYG{n+nb}{all} \PYG{n}{table} \PYG{n}{entries} \PYG{n}{END} \PYG{o}{==}\PYG{o}{==}\PYG{o}{==}\PYG{o}{==}\PYG{o}{==}\PYG{o}{==}\PYG{o}{==}\PYG{o}{==}\PYG{o}{==}\PYG{o}{==}\PYG{o}{=}
\end{sphinxVerbatim}

GDEC:

\begin{sphinxVerbatim}[commandchars=\\\{\}]
\PYG{o}{==}\PYG{o}{==}\PYG{o}{==}\PYG{o}{==}\PYG{o}{==}\PYG{o}{==}\PYG{o}{==}\PYG{o}{==}\PYG{o}{==}\PYG{o}{==}\PYG{o}{==}\PYG{o}{==} \PYG{n}{Count} \PYG{n}{of} \PYG{n+nb}{all} \PYG{n}{table} \PYG{n}{entries} \PYG{o}{==}\PYG{o}{==}\PYG{o}{==}\PYG{o}{==}\PYG{o}{==}\PYG{o}{==}\PYG{o}{==}\PYG{o}{==}\PYG{o}{==}\PYG{o}{==}\PYG{o}{==}\PYG{o}{=}
\PYG{p}{\PYGZob{}}
\PYG{l+s+s2}{\PYGZdq{}}\PYG{l+s+s2}{(34, b}\PYG{l+s+s2}{\PYGZsq{}}\PYG{l+s+s2}{FILM}\PYG{l+s+s2}{\PYGZsq{}}\PYG{l+s+s2}{, b}\PYG{l+s+s2}{\PYGZsq{}}\PYG{l+s+s2}{GDEC}\PYG{l+s+s2}{\PYGZsq{}}\PYG{l+s+s2}{, b}\PYG{l+s+s2}{\PYGZsq{}}\PYG{l+s+s2}{\PYGZhy{}\PYGZhy{}\PYGZhy{} }\PYG{l+s+s2}{\PYGZsq{}}\PYG{l+s+s2}{)}\PYG{l+s+s2}{\PYGZdq{}}\PYG{p}{:} \PYG{l+m+mi}{10}\PYG{p}{,}
\PYG{l+s+s2}{\PYGZdq{}}\PYG{l+s+s2}{(34, b}\PYG{l+s+s2}{\PYGZsq{}}\PYG{l+s+s2}{FILM}\PYG{l+s+s2}{\PYGZsq{}}\PYG{l+s+s2}{, b}\PYG{l+s+s2}{\PYGZsq{}}\PYG{l+s+s2}{GDEC}\PYG{l+s+s2}{\PYGZsq{}}\PYG{l+s+s2}{, b}\PYG{l+s+s2}{\PYGZsq{}}\PYG{l+s+s2}{\PYGZhy{}\PYGZhy{}\PYGZhy{}\PYGZhy{}}\PYG{l+s+s2}{\PYGZsq{}}\PYG{l+s+s2}{)}\PYG{l+s+s2}{\PYGZdq{}}\PYG{p}{:} \PYG{l+m+mi}{227}\PYG{p}{,}
\PYG{l+s+s2}{\PYGZdq{}}\PYG{l+s+s2}{(34, b}\PYG{l+s+s2}{\PYGZsq{}}\PYG{l+s+s2}{FILM}\PYG{l+s+s2}{\PYGZsq{}}\PYG{l+s+s2}{, b}\PYG{l+s+s2}{\PYGZsq{}}\PYG{l+s+s2}{GDEC}\PYG{l+s+s2}{\PYGZsq{}}\PYG{l+s+s2}{, b}\PYG{l+s+s2}{\PYGZsq{}}\PYG{l+s+s2}{\PYGZhy{}1\PYGZhy{} }\PYG{l+s+s2}{\PYGZsq{}}\PYG{l+s+s2}{)}\PYG{l+s+s2}{\PYGZdq{}}\PYG{p}{:} \PYG{l+m+mi}{2}\PYG{p}{,}
\PYG{l+s+s2}{\PYGZdq{}}\PYG{l+s+s2}{(34, b}\PYG{l+s+s2}{\PYGZsq{}}\PYG{l+s+s2}{FILM}\PYG{l+s+s2}{\PYGZsq{}}\PYG{l+s+s2}{, b}\PYG{l+s+s2}{\PYGZsq{}}\PYG{l+s+s2}{GDEC}\PYG{l+s+s2}{\PYGZsq{}}\PYG{l+s+s2}{, b}\PYG{l+s+s2}{\PYGZsq{}}\PYG{l+s+s2}{\PYGZhy{}2\PYGZhy{} }\PYG{l+s+s2}{\PYGZsq{}}\PYG{l+s+s2}{)}\PYG{l+s+s2}{\PYGZdq{}}\PYG{p}{:} \PYG{l+m+mi}{6}\PYG{p}{,}
\PYG{l+s+s2}{\PYGZdq{}}\PYG{l+s+s2}{(34, b}\PYG{l+s+s2}{\PYGZsq{}}\PYG{l+s+s2}{FILM}\PYG{l+s+s2}{\PYGZsq{}}\PYG{l+s+s2}{, b}\PYG{l+s+s2}{\PYGZsq{}}\PYG{l+s+s2}{GDEC}\PYG{l+s+s2}{\PYGZsq{}}\PYG{l+s+s2}{, b}\PYG{l+s+s2}{\PYGZsq{}}\PYG{l+s+s2}{\PYGZhy{}2\PYGZhy{}\PYGZhy{}}\PYG{l+s+s2}{\PYGZsq{}}\PYG{l+s+s2}{)}\PYG{l+s+s2}{\PYGZdq{}}\PYG{p}{:} \PYG{l+m+mi}{32}\PYG{p}{,}
\PYG{l+s+s2}{\PYGZdq{}}\PYG{l+s+s2}{(34, b}\PYG{l+s+s2}{\PYGZsq{}}\PYG{l+s+s2}{FILM}\PYG{l+s+s2}{\PYGZsq{}}\PYG{l+s+s2}{, b}\PYG{l+s+s2}{\PYGZsq{}}\PYG{l+s+s2}{GDEC}\PYG{l+s+s2}{\PYGZsq{}}\PYG{l+s+s2}{, b}\PYG{l+s+s2}{\PYGZsq{}}\PYG{l+s+s2}{\PYGZhy{}3\PYGZhy{} }\PYG{l+s+s2}{\PYGZsq{}}\PYG{l+s+s2}{)}\PYG{l+s+s2}{\PYGZdq{}}\PYG{p}{:} \PYG{l+m+mi}{3}\PYG{p}{,}
\PYG{l+s+s2}{\PYGZdq{}}\PYG{l+s+s2}{(34, b}\PYG{l+s+s2}{\PYGZsq{}}\PYG{l+s+s2}{FILM}\PYG{l+s+s2}{\PYGZsq{}}\PYG{l+s+s2}{, b}\PYG{l+s+s2}{\PYGZsq{}}\PYG{l+s+s2}{GDEC}\PYG{l+s+s2}{\PYGZsq{}}\PYG{l+s+s2}{, b}\PYG{l+s+s2}{\PYGZsq{}}\PYG{l+s+s2}{\PYGZhy{}4\PYGZhy{} }\PYG{l+s+s2}{\PYGZsq{}}\PYG{l+s+s2}{)}\PYG{l+s+s2}{\PYGZdq{}}\PYG{p}{:} \PYG{l+m+mi}{10}\PYG{p}{,}
\PYG{l+s+s2}{\PYGZdq{}}\PYG{l+s+s2}{(34, b}\PYG{l+s+s2}{\PYGZsq{}}\PYG{l+s+s2}{FILM}\PYG{l+s+s2}{\PYGZsq{}}\PYG{l+s+s2}{, b}\PYG{l+s+s2}{\PYGZsq{}}\PYG{l+s+s2}{GDEC}\PYG{l+s+s2}{\PYGZsq{}}\PYG{l+s+s2}{, b}\PYG{l+s+s2}{\PYGZsq{}}\PYG{l+s+s2}{\PYGZhy{}4\PYGZhy{}\PYGZhy{}}\PYG{l+s+s2}{\PYGZsq{}}\PYG{l+s+s2}{)}\PYG{l+s+s2}{\PYGZdq{}}\PYG{p}{:} \PYG{l+m+mi}{26}\PYG{p}{,}
\PYG{l+s+s2}{\PYGZdq{}}\PYG{l+s+s2}{(34, b}\PYG{l+s+s2}{\PYGZsq{}}\PYG{l+s+s2}{FILM}\PYG{l+s+s2}{\PYGZsq{}}\PYG{l+s+s2}{, b}\PYG{l+s+s2}{\PYGZsq{}}\PYG{l+s+s2}{GDEC}\PYG{l+s+s2}{\PYGZsq{}}\PYG{l+s+s2}{, b}\PYG{l+s+s2}{\PYGZsq{}}\PYG{l+s+s2}{1111}\PYG{l+s+s2}{\PYGZsq{}}\PYG{l+s+s2}{)}\PYG{l+s+s2}{\PYGZdq{}}\PYG{p}{:} \PYG{l+m+mi}{1}\PYG{p}{,}
\PYG{l+s+s2}{\PYGZdq{}}\PYG{l+s+s2}{(34, b}\PYG{l+s+s2}{\PYGZsq{}}\PYG{l+s+s2}{FILM}\PYG{l+s+s2}{\PYGZsq{}}\PYG{l+s+s2}{, b}\PYG{l+s+s2}{\PYGZsq{}}\PYG{l+s+s2}{GDEC}\PYG{l+s+s2}{\PYGZsq{}}\PYG{l+s+s2}{, b}\PYG{l+s+s2}{\PYGZsq{}}\PYG{l+s+s2}{EEE }\PYG{l+s+s2}{\PYGZsq{}}\PYG{l+s+s2}{)}\PYG{l+s+s2}{\PYGZdq{}}\PYG{p}{:} \PYG{l+m+mi}{14}\PYG{p}{,}
\PYG{p}{\PYGZcb{}}
\PYG{o}{==}\PYG{o}{==}\PYG{o}{==}\PYG{o}{==}\PYG{o}{==}\PYG{o}{==}\PYG{o}{==}\PYG{o}{==}\PYG{o}{==}\PYG{o}{==}\PYG{o}{==} \PYG{n}{Count} \PYG{n}{of} \PYG{n+nb}{all} \PYG{n}{table} \PYG{n}{entries} \PYG{n}{END} \PYG{o}{==}\PYG{o}{==}\PYG{o}{==}\PYG{o}{==}\PYG{o}{==}\PYG{o}{==}\PYG{o}{==}\PYG{o}{==}\PYG{o}{==}\PYG{o}{==}\PYG{o}{=}
\end{sphinxVerbatim}

DEST:

\begin{sphinxVerbatim}[commandchars=\\\{\}]
\PYG{o}{==}\PYG{o}{==}\PYG{o}{==}\PYG{o}{==}\PYG{o}{==}\PYG{o}{==}\PYG{o}{==}\PYG{o}{==}\PYG{o}{==}\PYG{o}{==}\PYG{o}{==}\PYG{o}{==} \PYG{n}{Count} \PYG{n}{of} \PYG{n+nb}{all} \PYG{n}{table} \PYG{n}{entries} \PYG{o}{==}\PYG{o}{==}\PYG{o}{==}\PYG{o}{==}\PYG{o}{==}\PYG{o}{==}\PYG{o}{==}\PYG{o}{==}\PYG{o}{==}\PYG{o}{==}\PYG{o}{==}\PYG{o}{=}
\PYG{p}{\PYGZob{}}
\PYG{l+s+s2}{\PYGZdq{}}\PYG{l+s+s2}{(34, b}\PYG{l+s+s2}{\PYGZsq{}}\PYG{l+s+s2}{FILM}\PYG{l+s+s2}{\PYGZsq{}}\PYG{l+s+s2}{, b}\PYG{l+s+s2}{\PYGZsq{}}\PYG{l+s+s2}{DEST}\PYG{l+s+s2}{\PYGZsq{}}\PYG{l+s+s2}{, b}\PYG{l+s+s2}{\PYGZsq{}}\PYG{l+s+s2}{NEIT}\PYG{l+s+s2}{\PYGZsq{}}\PYG{l+s+s2}{)}\PYG{l+s+s2}{\PYGZdq{}}\PYG{p}{:} \PYG{l+m+mi}{66}\PYG{p}{,}
\PYG{l+s+s2}{\PYGZdq{}}\PYG{l+s+s2}{(34, b}\PYG{l+s+s2}{\PYGZsq{}}\PYG{l+s+s2}{FILM}\PYG{l+s+s2}{\PYGZsq{}}\PYG{l+s+s2}{, b}\PYG{l+s+s2}{\PYGZsq{}}\PYG{l+s+s2}{DEST}\PYG{l+s+s2}{\PYGZsq{}}\PYG{l+s+s2}{, b}\PYG{l+s+s2}{\PYGZsq{}}\PYG{l+s+s2}{PF1 }\PYG{l+s+s2}{\PYGZsq{}}\PYG{l+s+s2}{)}\PYG{l+s+s2}{\PYGZdq{}}\PYG{p}{:} \PYG{l+m+mi}{124}\PYG{p}{,}
\PYG{l+s+s2}{\PYGZdq{}}\PYG{l+s+s2}{(34, b}\PYG{l+s+s2}{\PYGZsq{}}\PYG{l+s+s2}{FILM}\PYG{l+s+s2}{\PYGZsq{}}\PYG{l+s+s2}{, b}\PYG{l+s+s2}{\PYGZsq{}}\PYG{l+s+s2}{DEST}\PYG{l+s+s2}{\PYGZsq{}}\PYG{l+s+s2}{, b}\PYG{l+s+s2}{\PYGZsq{}}\PYG{l+s+s2}{PF2 }\PYG{l+s+s2}{\PYGZsq{}}\PYG{l+s+s2}{)}\PYG{l+s+s2}{\PYGZdq{}}\PYG{p}{:} \PYG{l+m+mi}{125}\PYG{p}{,}
\PYG{l+s+s2}{\PYGZdq{}}\PYG{l+s+s2}{(34, b}\PYG{l+s+s2}{\PYGZsq{}}\PYG{l+s+s2}{FILM}\PYG{l+s+s2}{\PYGZsq{}}\PYG{l+s+s2}{, b}\PYG{l+s+s2}{\PYGZsq{}}\PYG{l+s+s2}{DEST}\PYG{l+s+s2}{\PYGZsq{}}\PYG{l+s+s2}{, b}\PYG{l+s+s2}{\PYGZsq{}}\PYG{l+s+s2}{PF5 }\PYG{l+s+s2}{\PYGZsq{}}\PYG{l+s+s2}{)}\PYG{l+s+s2}{\PYGZdq{}}\PYG{p}{:} \PYG{l+m+mi}{1}\PYG{p}{,}
\PYG{l+s+s2}{\PYGZdq{}}\PYG{l+s+s2}{(34, b}\PYG{l+s+s2}{\PYGZsq{}}\PYG{l+s+s2}{FILM}\PYG{l+s+s2}{\PYGZsq{}}\PYG{l+s+s2}{, b}\PYG{l+s+s2}{\PYGZsq{}}\PYG{l+s+s2}{DEST}\PYG{l+s+s2}{\PYGZsq{}}\PYG{l+s+s2}{, b}\PYG{l+s+s2}{\PYGZsq{}}\PYG{l+s+s2}{PF6 }\PYG{l+s+s2}{\PYGZsq{}}\PYG{l+s+s2}{)}\PYG{l+s+s2}{\PYGZdq{}}\PYG{p}{:} \PYG{l+m+mi}{5}\PYG{p}{,}
\PYG{l+s+s2}{\PYGZdq{}}\PYG{l+s+s2}{(34, b}\PYG{l+s+s2}{\PYGZsq{}}\PYG{l+s+s2}{FILM}\PYG{l+s+s2}{\PYGZsq{}}\PYG{l+s+s2}{, b}\PYG{l+s+s2}{\PYGZsq{}}\PYG{l+s+s2}{DEST}\PYG{l+s+s2}{\PYGZsq{}}\PYG{l+s+s2}{, b}\PYG{l+s+s2}{\PYGZsq{}}\PYG{l+s+s2}{PF8 }\PYG{l+s+s2}{\PYGZsq{}}\PYG{l+s+s2}{)}\PYG{l+s+s2}{\PYGZdq{}}\PYG{p}{:} \PYG{l+m+mi}{1}\PYG{p}{,}
\PYG{l+s+s2}{\PYGZdq{}}\PYG{l+s+s2}{(34, b}\PYG{l+s+s2}{\PYGZsq{}}\PYG{l+s+s2}{FILM}\PYG{l+s+s2}{\PYGZsq{}}\PYG{l+s+s2}{, b}\PYG{l+s+s2}{\PYGZsq{}}\PYG{l+s+s2}{DEST}\PYG{l+s+s2}{\PYGZsq{}}\PYG{l+s+s2}{, b}\PYG{l+s+s2}{\PYGZsq{}}\PYG{l+s+s2}{PFA }\PYG{l+s+s2}{\PYGZsq{}}\PYG{l+s+s2}{)}\PYG{l+s+s2}{\PYGZdq{}}\PYG{p}{:} \PYG{l+m+mi}{1}\PYG{p}{,}
\PYG{l+s+s2}{\PYGZdq{}}\PYG{l+s+s2}{(34, b}\PYG{l+s+s2}{\PYGZsq{}}\PYG{l+s+s2}{FILM}\PYG{l+s+s2}{\PYGZsq{}}\PYG{l+s+s2}{, b}\PYG{l+s+s2}{\PYGZsq{}}\PYG{l+s+s2}{DEST}\PYG{l+s+s2}{\PYGZsq{}}\PYG{l+s+s2}{, b}\PYG{l+s+s2}{\PYGZsq{}}\PYG{l+s+s2}{PFD }\PYG{l+s+s2}{\PYGZsq{}}\PYG{l+s+s2}{)}\PYG{l+s+s2}{\PYGZdq{}}\PYG{p}{:} \PYG{l+m+mi}{2}\PYG{p}{,}
\PYG{l+s+s2}{\PYGZdq{}}\PYG{l+s+s2}{(34, b}\PYG{l+s+s2}{\PYGZsq{}}\PYG{l+s+s2}{FILM}\PYG{l+s+s2}{\PYGZsq{}}\PYG{l+s+s2}{, b}\PYG{l+s+s2}{\PYGZsq{}}\PYG{l+s+s2}{DEST}\PYG{l+s+s2}{\PYGZsq{}}\PYG{l+s+s2}{, b}\PYG{l+s+s2}{\PYGZsq{}}\PYG{l+s+s2}{PFE }\PYG{l+s+s2}{\PYGZsq{}}\PYG{l+s+s2}{)}\PYG{l+s+s2}{\PYGZdq{}}\PYG{p}{:} \PYG{l+m+mi}{3}\PYG{p}{,}
\PYG{l+s+s2}{\PYGZdq{}}\PYG{l+s+s2}{(34, b}\PYG{l+s+s2}{\PYGZsq{}}\PYG{l+s+s2}{FILM}\PYG{l+s+s2}{\PYGZsq{}}\PYG{l+s+s2}{, b}\PYG{l+s+s2}{\PYGZsq{}}\PYG{l+s+s2}{DEST}\PYG{l+s+s2}{\PYGZsq{}}\PYG{l+s+s2}{, b}\PYG{l+s+s2}{\PYGZsq{}}\PYG{l+s+s2}{PFJ }\PYG{l+s+s2}{\PYGZsq{}}\PYG{l+s+s2}{)}\PYG{l+s+s2}{\PYGZdq{}}\PYG{p}{:} \PYG{l+m+mi}{2}\PYG{p}{,}
\PYG{l+s+s2}{\PYGZdq{}}\PYG{l+s+s2}{(34, b}\PYG{l+s+s2}{\PYGZsq{}}\PYG{l+s+s2}{FILM}\PYG{l+s+s2}{\PYGZsq{}}\PYG{l+s+s2}{, b}\PYG{l+s+s2}{\PYGZsq{}}\PYG{l+s+s2}{DEST}\PYG{l+s+s2}{\PYGZsq{}}\PYG{l+s+s2}{, b}\PYG{l+s+s2}{\PYGZsq{}}\PYG{l+s+s2}{PFK }\PYG{l+s+s2}{\PYGZsq{}}\PYG{l+s+s2}{)}\PYG{l+s+s2}{\PYGZdq{}}\PYG{p}{:} \PYG{l+m+mi}{1}\PYG{p}{,}
\PYG{p}{\PYGZcb{}}
\PYG{o}{==}\PYG{o}{==}\PYG{o}{==}\PYG{o}{==}\PYG{o}{==}\PYG{o}{==}\PYG{o}{==}\PYG{o}{==}\PYG{o}{==}\PYG{o}{==}\PYG{o}{==} \PYG{n}{Count} \PYG{n}{of} \PYG{n+nb}{all} \PYG{n}{table} \PYG{n}{entries} \PYG{n}{END} \PYG{o}{==}\PYG{o}{==}\PYG{o}{==}\PYG{o}{==}\PYG{o}{==}\PYG{o}{==}\PYG{o}{==}\PYG{o}{==}\PYG{o}{==}\PYG{o}{==}\PYG{o}{=}
\end{sphinxVerbatim}

DSCA:

\begin{sphinxVerbatim}[commandchars=\\\{\}]
\PYG{o}{==}\PYG{o}{==}\PYG{o}{==}\PYG{o}{==}\PYG{o}{==}\PYG{o}{==}\PYG{o}{==}\PYG{o}{==}\PYG{o}{==}\PYG{o}{==}\PYG{o}{==}\PYG{o}{==} \PYG{n}{Count} \PYG{n}{of} \PYG{n+nb}{all} \PYG{n}{table} \PYG{n}{entries} \PYG{o}{==}\PYG{o}{==}\PYG{o}{==}\PYG{o}{==}\PYG{o}{==}\PYG{o}{==}\PYG{o}{==}\PYG{o}{==}\PYG{o}{==}\PYG{o}{==}\PYG{o}{==}\PYG{o}{=}
\PYG{p}{\PYGZob{}}
\PYG{l+s+s2}{\PYGZdq{}}\PYG{l+s+s2}{(34, b}\PYG{l+s+s2}{\PYGZsq{}}\PYG{l+s+s2}{FILM}\PYG{l+s+s2}{\PYGZsq{}}\PYG{l+s+s2}{, b}\PYG{l+s+s2}{\PYGZsq{}}\PYG{l+s+s2}{DSCA}\PYG{l+s+s2}{\PYGZsq{}}\PYG{l+s+s2}{, b}\PYG{l+s+s2}{\PYGZsq{}}\PYG{l+s+s2}{D200}\PYG{l+s+s2}{\PYGZsq{}}\PYG{l+s+s2}{)}\PYG{l+s+s2}{\PYGZdq{}}\PYG{p}{:} \PYG{l+m+mi}{156}\PYG{p}{,}
\PYG{l+s+s2}{\PYGZdq{}}\PYG{l+s+s2}{(34, b}\PYG{l+s+s2}{\PYGZsq{}}\PYG{l+s+s2}{FILM}\PYG{l+s+s2}{\PYGZsq{}}\PYG{l+s+s2}{, b}\PYG{l+s+s2}{\PYGZsq{}}\PYG{l+s+s2}{DSCA}\PYG{l+s+s2}{\PYGZsq{}}\PYG{l+s+s2}{, b}\PYG{l+s+s2}{\PYGZsq{}}\PYG{l+s+s2}{D500}\PYG{l+s+s2}{\PYGZsq{}}\PYG{l+s+s2}{)}\PYG{l+s+s2}{\PYGZdq{}}\PYG{p}{:} \PYG{l+m+mi}{17}\PYG{p}{,}
\PYG{l+s+s2}{\PYGZdq{}}\PYG{l+s+s2}{(34, b}\PYG{l+s+s2}{\PYGZsq{}}\PYG{l+s+s2}{FILM}\PYG{l+s+s2}{\PYGZsq{}}\PYG{l+s+s2}{, b}\PYG{l+s+s2}{\PYGZsq{}}\PYG{l+s+s2}{DSCA}\PYG{l+s+s2}{\PYGZsq{}}\PYG{l+s+s2}{, b}\PYG{l+s+s2}{\PYGZsq{}}\PYG{l+s+s2}{DM  }\PYG{l+s+s2}{\PYGZsq{}}\PYG{l+s+s2}{)}\PYG{l+s+s2}{\PYGZdq{}}\PYG{p}{:} \PYG{l+m+mi}{76}\PYG{p}{,}
\PYG{l+s+s2}{\PYGZdq{}}\PYG{l+s+s2}{(34, b}\PYG{l+s+s2}{\PYGZsq{}}\PYG{l+s+s2}{FILM}\PYG{l+s+s2}{\PYGZsq{}}\PYG{l+s+s2}{, b}\PYG{l+s+s2}{\PYGZsq{}}\PYG{l+s+s2}{DSCA}\PYG{l+s+s2}{\PYGZsq{}}\PYG{l+s+s2}{, b}\PYG{l+s+s2}{\PYGZsq{}}\PYG{l+s+s2}{S5  }\PYG{l+s+s2}{\PYGZsq{}}\PYG{l+s+s2}{)}\PYG{l+s+s2}{\PYGZdq{}}\PYG{p}{:} \PYG{l+m+mi}{82}\PYG{p}{,}
\PYG{p}{\PYGZcb{}}
\PYG{o}{==}\PYG{o}{==}\PYG{o}{==}\PYG{o}{==}\PYG{o}{==}\PYG{o}{==}\PYG{o}{==}\PYG{o}{==}\PYG{o}{==}\PYG{o}{==}\PYG{o}{==} \PYG{n}{Count} \PYG{n}{of} \PYG{n+nb}{all} \PYG{n}{table} \PYG{n}{entries} \PYG{n}{END} \PYG{o}{==}\PYG{o}{==}\PYG{o}{==}\PYG{o}{==}\PYG{o}{==}\PYG{o}{==}\PYG{o}{==}\PYG{o}{==}\PYG{o}{==}\PYG{o}{==}\PYG{o}{=}
\end{sphinxVerbatim}
\index{ExceptionFILMCfg}

\begin{fulllineitems}
\phantomsection\label{\detokenize{ref/util/plot/FILMCfg:TotalDepth.util.plot.FILMCfg.ExceptionFILMCfg}}\pysigline{\sphinxbfcode{exception }\sphinxcode{TotalDepth.util.plot.FILMCfg.}\sphinxbfcode{ExceptionFILMCfg}}
Specialisation of exception for this module.

\end{fulllineitems}

\index{ExceptionPhysFilmCfg}

\begin{fulllineitems}
\phantomsection\label{\detokenize{ref/util/plot/FILMCfg:TotalDepth.util.plot.FILMCfg.ExceptionPhysFilmCfg}}\pysigline{\sphinxbfcode{exception }\sphinxcode{TotalDepth.util.plot.FILMCfg.}\sphinxbfcode{ExceptionPhysFilmCfg}}
Specialisation of exception for PhysFilmCfg.

\end{fulllineitems}

\index{ExceptionFilmCfgLISRead}

\begin{fulllineitems}
\phantomsection\label{\detokenize{ref/util/plot/FILMCfg:TotalDepth.util.plot.FILMCfg.ExceptionFilmCfgLISRead}}\pysigline{\sphinxbfcode{exception }\sphinxcode{TotalDepth.util.plot.FILMCfg.}\sphinxbfcode{ExceptionFilmCfgLISRead}}
Specialisation of exception for FilmCfgLISRead in this module.

\end{fulllineitems}

\index{PhysFilmCfg (class in TotalDepth.util.plot.FILMCfg)}

\begin{fulllineitems}
\phantomsection\label{\detokenize{ref/util/plot/FILMCfg:TotalDepth.util.plot.FILMCfg.PhysFilmCfg}}\pysiglinewithargsret{\sphinxbfcode{class }\sphinxcode{TotalDepth.util.plot.FILMCfg.}\sphinxbfcode{PhysFilmCfg}}{\emph{theName}, \emph{theTracks}, \emph{theDest}, \emph{theX}}{}
Contains the configuration equivalent to a single line in a FILM table.

theName is a hashable.

theTracks is a list of Track.Track objects.

theX is an integer scale.
\index{\_\_init\_\_() (TotalDepth.util.plot.FILMCfg.PhysFilmCfg method)}

\begin{fulllineitems}
\phantomsection\label{\detokenize{ref/util/plot/FILMCfg:TotalDepth.util.plot.FILMCfg.PhysFilmCfg.__init__}}\pysiglinewithargsret{\sphinxbfcode{\_\_init\_\_}}{\emph{theName}, \emph{theTracks}, \emph{theDest}, \emph{theX}}{}
Constructor.
theName is a hashable.
theTracks is a list of Track.Track objects.
theX is an integer scale.

\end{fulllineitems}

\index{name (TotalDepth.util.plot.FILMCfg.PhysFilmCfg attribute)}

\begin{fulllineitems}
\phantomsection\label{\detokenize{ref/util/plot/FILMCfg:TotalDepth.util.plot.FILMCfg.PhysFilmCfg.name}}\pysigline{\sphinxbfcode{name}}
Name of the FILM.

\end{fulllineitems}

\index{xScale (TotalDepth.util.plot.FILMCfg.PhysFilmCfg attribute)}

\begin{fulllineitems}
\phantomsection\label{\detokenize{ref/util/plot/FILMCfg:TotalDepth.util.plot.FILMCfg.PhysFilmCfg.xScale}}\pysigline{\sphinxbfcode{xScale}}
The FILM X axis scale as a number.

\end{fulllineitems}

\index{\_\_len\_\_() (TotalDepth.util.plot.FILMCfg.PhysFilmCfg method)}

\begin{fulllineitems}
\phantomsection\label{\detokenize{ref/util/plot/FILMCfg:TotalDepth.util.plot.FILMCfg.PhysFilmCfg.__len__}}\pysiglinewithargsret{\sphinxbfcode{\_\_len\_\_}}{}{}
Number of Track.Track objects.

\end{fulllineitems}

\index{\_\_getitem\_\_() (TotalDepth.util.plot.FILMCfg.PhysFilmCfg method)}

\begin{fulllineitems}
\phantomsection\label{\detokenize{ref/util/plot/FILMCfg:TotalDepth.util.plot.FILMCfg.PhysFilmCfg.__getitem__}}\pysiglinewithargsret{\sphinxbfcode{\_\_getitem\_\_}}{\emph{i}}{}
Returns the Track.Track object at position i.

\end{fulllineitems}

\index{genTracks() (TotalDepth.util.plot.FILMCfg.PhysFilmCfg method)}

\begin{fulllineitems}
\phantomsection\label{\detokenize{ref/util/plot/FILMCfg:TotalDepth.util.plot.FILMCfg.PhysFilmCfg.genTracks}}\pysiglinewithargsret{\sphinxbfcode{genTracks}}{}{}
Generate all tracks.

\end{fulllineitems}

\index{interpretTrac() (TotalDepth.util.plot.FILMCfg.PhysFilmCfg method)}

\begin{fulllineitems}
\phantomsection\label{\detokenize{ref/util/plot/FILMCfg:TotalDepth.util.plot.FILMCfg.PhysFilmCfg.interpretTrac}}\pysiglinewithargsret{\sphinxbfcode{interpretTrac}}{\emph{theTracStr}}{}
Turns TRAC information into left/right positions as Cooord.Dim()
objects and the number of half-tracks of the start and half-tracks
covered. The later two values are used for stacking and packing the
plot header and footer scales so that they take the minimum space.

e.g. b’T23 ‘ returns:
(the left position of T2, right of T3, 4, 4).

e.g. b’T2 ‘ returns:
(the left position of T2, right of T2, 4, 2).

Note: There is some fudging going on here

\end{fulllineitems}

\index{\_\_weakref\_\_ (TotalDepth.util.plot.FILMCfg.PhysFilmCfg attribute)}

\begin{fulllineitems}
\phantomsection\label{\detokenize{ref/util/plot/FILMCfg:TotalDepth.util.plot.FILMCfg.PhysFilmCfg.__weakref__}}\pysigline{\sphinxbfcode{\_\_weakref\_\_}}
list of weak references to the object (if defined)

\end{fulllineitems}


\end{fulllineitems}

\index{PhysFilmCfgLISRead (class in TotalDepth.util.plot.FILMCfg)}

\begin{fulllineitems}
\phantomsection\label{\detokenize{ref/util/plot/FILMCfg:TotalDepth.util.plot.FILMCfg.PhysFilmCfgLISRead}}\pysiglinewithargsret{\sphinxbfcode{class }\sphinxcode{TotalDepth.util.plot.FILMCfg.}\sphinxbfcode{PhysFilmCfgLISRead}}{\emph{theRow}}{}
Tracks from a LIS FILM table, essentially the pair of GCOD and GDEC
defines the left-to-right layout of the plot.

Grid codes from analysis above:
b’BBB ‘,  b’E1E ‘, b’E20 ‘, b’E2E ‘, b’E3E ‘, b’E40 ‘, b’E4E ‘, b’EB0 ‘, b’EEB ‘, b’EEE ‘, b’LLLL’

The DSCA defines the top-to-bottom nature of the plot. 
DSCA codes from analysis above:
b’D200’, b’D500’, b’DM  ‘, b’S5  ‘

But we can guess some others…
\index{DSCA\_MAP (TotalDepth.util.plot.FILMCfg.PhysFilmCfgLISRead attribute)}

\begin{fulllineitems}
\phantomsection\label{\detokenize{ref/util/plot/FILMCfg:TotalDepth.util.plot.FILMCfg.PhysFilmCfgLISRead.DSCA_MAP}}\pysigline{\sphinxbfcode{DSCA\_MAP}\sphinxbfcode{ = \{b'D20 ': 20, b'D40 ': 40, b'D200': 200, b'D240 ': 240, b'D500': 500, b'DM  ': 1000, b'S5  ': 240, b'S2  ': 600\}}}
X axis scale from a LIS FILM table
DSCA codes from analysis above:
b’D200’, b’D500’, b’DM  ‘, b’S5  ‘
But we can guess some others…

\end{fulllineitems}

\index{\_\_init\_\_() (TotalDepth.util.plot.FILMCfg.PhysFilmCfgLISRead method)}

\begin{fulllineitems}
\phantomsection\label{\detokenize{ref/util/plot/FILMCfg:TotalDepth.util.plot.FILMCfg.PhysFilmCfgLISRead.__init__}}\pysiglinewithargsret{\sphinxbfcode{\_\_init\_\_}}{\emph{theRow}}{}
Reads a LogiRec.TableRow object and populates a CurveCfg.

Example:

\begin{sphinxVerbatim}[commandchars=\\\{\}]
\PYG{n}{MNEM}  \PYG{n}{GCOD}  \PYG{n}{GDEC}  \PYG{n}{DEST}  \PYG{n}{DSCA}
\PYG{o}{\PYGZhy{}}\PYG{o}{\PYGZhy{}}\PYG{o}{\PYGZhy{}}\PYG{o}{\PYGZhy{}}\PYG{o}{\PYGZhy{}}\PYG{o}{\PYGZhy{}}\PYG{o}{\PYGZhy{}}\PYG{o}{\PYGZhy{}}\PYG{o}{\PYGZhy{}}\PYG{o}{\PYGZhy{}}\PYG{o}{\PYGZhy{}}\PYG{o}{\PYGZhy{}}\PYG{o}{\PYGZhy{}}\PYG{o}{\PYGZhy{}}\PYG{o}{\PYGZhy{}}\PYG{o}{\PYGZhy{}}\PYG{o}{\PYGZhy{}}\PYG{o}{\PYGZhy{}}\PYG{o}{\PYGZhy{}}\PYG{o}{\PYGZhy{}}\PYG{o}{\PYGZhy{}}\PYG{o}{\PYGZhy{}}\PYG{o}{\PYGZhy{}}\PYG{o}{\PYGZhy{}}\PYG{o}{\PYGZhy{}}\PYG{o}{\PYGZhy{}}\PYG{o}{\PYGZhy{}}\PYG{o}{\PYGZhy{}}\PYG{o}{\PYGZhy{}}
\PYG{l+m+mi}{1}     \PYG{n}{E20}   \PYG{o}{\PYGZhy{}}\PYG{l+m+mi}{4}\PYG{o}{\PYGZhy{}}\PYG{o}{\PYGZhy{}}  \PYG{n}{PF1}   \PYG{n}{D200}
\end{sphinxVerbatim}

\end{fulllineitems}

\index{supportedFilmTracks() (TotalDepth.util.plot.FILMCfg.PhysFilmCfgLISRead method)}

\begin{fulllineitems}
\phantomsection\label{\detokenize{ref/util/plot/FILMCfg:TotalDepth.util.plot.FILMCfg.PhysFilmCfgLISRead.supportedFilmTracks}}\pysiglinewithargsret{\sphinxbfcode{supportedFilmTracks}}{}{}
A list of supported film (name, decade) pairs.

\end{fulllineitems}


\end{fulllineitems}

\index{FilmCfg (class in TotalDepth.util.plot.FILMCfg)}

\begin{fulllineitems}
\phantomsection\label{\detokenize{ref/util/plot/FILMCfg:TotalDepth.util.plot.FILMCfg.FilmCfg}}\pysigline{\sphinxbfcode{class }\sphinxcode{TotalDepth.util.plot.FILMCfg.}\sphinxbfcode{FilmCfg}}
Contains the configuration equivalent to a complete FILM table.
\index{add() (TotalDepth.util.plot.FILMCfg.FilmCfg method)}

\begin{fulllineitems}
\phantomsection\label{\detokenize{ref/util/plot/FILMCfg:TotalDepth.util.plot.FILMCfg.FilmCfg.add}}\pysiglinewithargsret{\sphinxbfcode{add}}{\emph{k}, \emph{thePfc}}{}
Add a PhysFilmCfg object to the map with key k, typically a FILM
mnemonic in bytes such as Mnem.Mnem(b‘1   ‘) or
some other ID such as a string, the filename of an XML file.

\end{fulllineitems}

\index{keys() (TotalDepth.util.plot.FILMCfg.FilmCfg method)}

\begin{fulllineitems}
\phantomsection\label{\detokenize{ref/util/plot/FILMCfg:TotalDepth.util.plot.FILMCfg.FilmCfg.keys}}\pysiglinewithargsret{\sphinxbfcode{keys}}{}{}
All FILM Mnemonics.

\end{fulllineitems}

\index{\_\_len\_\_() (TotalDepth.util.plot.FILMCfg.FilmCfg method)}

\begin{fulllineitems}
\phantomsection\label{\detokenize{ref/util/plot/FILMCfg:TotalDepth.util.plot.FILMCfg.FilmCfg.__len__}}\pysiglinewithargsret{\sphinxbfcode{\_\_len\_\_}}{}{}
Number of unique film destination names.

\end{fulllineitems}

\index{\_\_getitem\_\_() (TotalDepth.util.plot.FILMCfg.FilmCfg method)}

\begin{fulllineitems}
\phantomsection\label{\detokenize{ref/util/plot/FILMCfg:TotalDepth.util.plot.FILMCfg.FilmCfg.__getitem__}}\pysiglinewithargsret{\sphinxbfcode{\_\_getitem\_\_}}{\emph{name}}{}
Returns the PhysFilmCfg object corresponding to name - a Mnem() object.
Will raise KeyError if not exact match. See retFilmDest() for an API that can
handle curve destinations of BOTH, ALL etc.

\end{fulllineitems}

\index{\_\_contains\_\_() (TotalDepth.util.plot.FILMCfg.FilmCfg method)}

\begin{fulllineitems}
\phantomsection\label{\detokenize{ref/util/plot/FILMCfg:TotalDepth.util.plot.FILMCfg.FilmCfg.__contains__}}\pysiglinewithargsret{\sphinxbfcode{\_\_contains\_\_}}{\emph{name}}{}
Membership test.

\end{fulllineitems}

\index{retAllFILMDestS() (TotalDepth.util.plot.FILMCfg.FilmCfg method)}

\begin{fulllineitems}
\phantomsection\label{\detokenize{ref/util/plot/FILMCfg:TotalDepth.util.plot.FILMCfg.FilmCfg.retAllFILMDestS}}\pysiglinewithargsret{\sphinxbfcode{retAllFILMDestS}}{\emph{curveDestID}}{}
Returns an unordered list of FILM destinations for a curve destination.
For example if curveDestID is b’BOTH’ this might return {[}b‘2   ‘, b‘1   ‘{]}

\end{fulllineitems}

\index{retFILMDest() (TotalDepth.util.plot.FILMCfg.FilmCfg method)}

\begin{fulllineitems}
\phantomsection\label{\detokenize{ref/util/plot/FILMCfg:TotalDepth.util.plot.FILMCfg.FilmCfg.retFILMDest}}\pysiglinewithargsret{\sphinxbfcode{retFILMDest}}{\emph{filmDestID}, \emph{curveDestID}}{}
Returns a PhysFilmCfg object by matching curveDestID to the filmDestID.
Returns None on failure. For LIS curveDestID can be 1, BOTH, ALL, NEIT etc.
This is commonly used by the PRESCfg module so that interpretTrac() can
be called on the result and thus build up a map of track positions for
all possible logical film outputs.

\end{fulllineitems}

\index{interpretTrac() (TotalDepth.util.plot.FILMCfg.FilmCfg method)}

\begin{fulllineitems}
\phantomsection\label{\detokenize{ref/util/plot/FILMCfg:TotalDepth.util.plot.FILMCfg.FilmCfg.interpretTrac}}\pysiglinewithargsret{\sphinxbfcode{interpretTrac}}{\emph{filmDestID}, \emph{curveDestID}, \emph{trackStr}}{}
Given a film destination ID and a curve destination (which could be
b’ALL’) and a track string (e.g. b’T23’) this returns the left/right
positions as Cooord.Dim() objects and the number of half-tracks of the
start and half-tracks covered (used for plot header and footer scales).
Returns None if there is no match for the filmDestID/curveDestID (for
example if curveDestID is b’NEIT’). 
e.g. (b‘1   ‘, b’BOTH’, b’T23 ‘) returns:
(the left position of T2, right of T3, 4, 4).

\end{fulllineitems}

\index{\_\_weakref\_\_ (TotalDepth.util.plot.FILMCfg.FilmCfg attribute)}

\begin{fulllineitems}
\phantomsection\label{\detokenize{ref/util/plot/FILMCfg:TotalDepth.util.plot.FILMCfg.FilmCfg.__weakref__}}\pysigline{\sphinxbfcode{\_\_weakref\_\_}}
list of weak references to the object (if defined)

\end{fulllineitems}


\end{fulllineitems}

\index{FilmCfgLISRead (class in TotalDepth.util.plot.FILMCfg)}

\begin{fulllineitems}
\phantomsection\label{\detokenize{ref/util/plot/FILMCfg:TotalDepth.util.plot.FILMCfg.FilmCfgLISRead}}\pysiglinewithargsret{\sphinxbfcode{class }\sphinxcode{TotalDepth.util.plot.FILMCfg.}\sphinxbfcode{FilmCfgLISRead}}{\emph{theLr}}{}
Interprets a FILM table from a LIS Logical Record.
\index{\_\_init\_\_() (TotalDepth.util.plot.FILMCfg.FilmCfgLISRead method)}

\begin{fulllineitems}
\phantomsection\label{\detokenize{ref/util/plot/FILMCfg:TotalDepth.util.plot.FILMCfg.FilmCfgLISRead.__init__}}\pysiglinewithargsret{\sphinxbfcode{\_\_init\_\_}}{\emph{theLr}}{}
Reads a LogiRec.Table object and creates a PhysFilmCfgLISRead for
each row.

Typical FILM table:

\begin{sphinxVerbatim}[commandchars=\\\{\}]
\PYG{n}{MNEM}  \PYG{n}{GCOD}  \PYG{n}{GDEC}  \PYG{n}{DEST}  \PYG{n}{DSCA}
\PYG{o}{\PYGZhy{}}\PYG{o}{\PYGZhy{}}\PYG{o}{\PYGZhy{}}\PYG{o}{\PYGZhy{}}\PYG{o}{\PYGZhy{}}\PYG{o}{\PYGZhy{}}\PYG{o}{\PYGZhy{}}\PYG{o}{\PYGZhy{}}\PYG{o}{\PYGZhy{}}\PYG{o}{\PYGZhy{}}\PYG{o}{\PYGZhy{}}\PYG{o}{\PYGZhy{}}\PYG{o}{\PYGZhy{}}\PYG{o}{\PYGZhy{}}\PYG{o}{\PYGZhy{}}\PYG{o}{\PYGZhy{}}\PYG{o}{\PYGZhy{}}\PYG{o}{\PYGZhy{}}\PYG{o}{\PYGZhy{}}\PYG{o}{\PYGZhy{}}\PYG{o}{\PYGZhy{}}\PYG{o}{\PYGZhy{}}\PYG{o}{\PYGZhy{}}\PYG{o}{\PYGZhy{}}\PYG{o}{\PYGZhy{}}\PYG{o}{\PYGZhy{}}\PYG{o}{\PYGZhy{}}\PYG{o}{\PYGZhy{}}\PYG{o}{\PYGZhy{}}
\PYG{l+m+mi}{1}     \PYG{n}{E20}   \PYG{o}{\PYGZhy{}}\PYG{l+m+mi}{4}\PYG{o}{\PYGZhy{}}\PYG{o}{\PYGZhy{}}  \PYG{n}{PF1}   \PYG{n}{D200}
\PYG{l+m+mi}{2}     \PYG{n}{EEE}   \PYG{o}{\PYGZhy{}}\PYG{o}{\PYGZhy{}}\PYG{o}{\PYGZhy{}}\PYG{o}{\PYGZhy{}}  \PYG{n}{PF2}   \PYG{n}{D200}
\end{sphinxVerbatim}

\end{fulllineitems}

\index{retAllFILMDestS() (TotalDepth.util.plot.FILMCfg.FilmCfgLISRead method)}

\begin{fulllineitems}
\phantomsection\label{\detokenize{ref/util/plot/FILMCfg:TotalDepth.util.plot.FILMCfg.FilmCfgLISRead.retAllFILMDestS}}\pysiglinewithargsret{\sphinxbfcode{retAllFILMDestS}}{\emph{curveDestID}}{}
Returns an unordered list of FILM destinations for a curve destination.

For example if curveDestID is b’BOTH’ this might return {[}b‘2   ‘, b‘1   ‘{]}

\end{fulllineitems}

\index{retFILMDest() (TotalDepth.util.plot.FILMCfg.FilmCfgLISRead method)}

\begin{fulllineitems}
\phantomsection\label{\detokenize{ref/util/plot/FILMCfg:TotalDepth.util.plot.FILMCfg.FilmCfgLISRead.retFILMDest}}\pysiglinewithargsret{\sphinxbfcode{retFILMDest}}{\emph{filmDestID}, \emph{curveDestID}}{}
Returns a PhysFilmCfg object by matching curveDestID to the filmDestID.
Returns None on failure. curveDestID can be 1, BOTH, ALL, NEIT etc.

This is commonly used by the PRESCfg module so that interpretTrac() can
be called on the result and thus build up a map of track positions for
all possible logical film outputs.

\end{fulllineitems}


\end{fulllineitems}



\paragraph{FILM Plotting Configuration from XML files}
\label{\detokenize{ref/util/plot/FILMCfgXML:module-TotalDepth.util.plot.FILMCfgXML}}\label{\detokenize{ref/util/plot/FILMCfgXML::doc}}\label{\detokenize{ref/util/plot/FILMCfgXML:film-plotting-configuration-from-xml-files}}\index{TotalDepth.util.plot.FILMCfgXML (module)}
Creates FILM configurations from LgFormat XML files.

Created on Dec 14, 2011
\index{ExceptionFILMCfgXML}

\begin{fulllineitems}
\phantomsection\label{\detokenize{ref/util/plot/FILMCfgXML:TotalDepth.util.plot.FILMCfgXML.ExceptionFILMCfgXML}}\pysigline{\sphinxbfcode{exception }\sphinxcode{TotalDepth.util.plot.FILMCfgXML.}\sphinxbfcode{ExceptionFILMCfgXML}}
Specialisation of exception for FILMCfgXML module.

\end{fulllineitems}

\index{ExceptionFILMCfgXMLRead}

\begin{fulllineitems}
\phantomsection\label{\detokenize{ref/util/plot/FILMCfgXML:TotalDepth.util.plot.FILMCfgXML.ExceptionFILMCfgXMLRead}}\pysigline{\sphinxbfcode{exception }\sphinxcode{TotalDepth.util.plot.FILMCfgXML.}\sphinxbfcode{ExceptionFILMCfgXMLRead}}
Specialisation of exception for FILMCfgXMLRead module.

\end{fulllineitems}

\index{ExceptionFILMCfgXMLReadLookUp}

\begin{fulllineitems}
\phantomsection\label{\detokenize{ref/util/plot/FILMCfgXML:TotalDepth.util.plot.FILMCfgXML.ExceptionFILMCfgXMLReadLookUp}}\pysigline{\sphinxbfcode{exception }\sphinxcode{TotalDepth.util.plot.FILMCfgXML.}\sphinxbfcode{ExceptionFILMCfgXMLReadLookUp}}
Specialisation of exception for FILMCfgXMLRead module when a lookup
fails that would normally raise a KeyError.

\end{fulllineitems}

\index{PhysFilmCfgXMLRead (class in TotalDepth.util.plot.FILMCfgXML)}

\begin{fulllineitems}
\phantomsection\label{\detokenize{ref/util/plot/FILMCfgXML:TotalDepth.util.plot.FILMCfgXML.PhysFilmCfgXMLRead}}\pysiglinewithargsret{\sphinxbfcode{class }\sphinxcode{TotalDepth.util.plot.FILMCfgXML.}\sphinxbfcode{PhysFilmCfgXMLRead}}{\emph{root}}{}
Extracts FILM information from a single XML file root element.
\index{TRAC\_STANDARD\_SPAN (TotalDepth.util.plot.FILMCfgXML.PhysFilmCfgXMLRead attribute)}

\begin{fulllineitems}
\phantomsection\label{\detokenize{ref/util/plot/FILMCfgXML:TotalDepth.util.plot.FILMCfgXML.PhysFilmCfgXMLRead.TRAC_STANDARD_SPAN}}\pysigline{\sphinxbfcode{TRAC\_STANDARD\_SPAN}\sphinxbfcode{ = 2}}
Default number of half-tracks to span

\end{fulllineitems}

\index{TRAC\_UNIQUEID\_TO\_NON\_STANDARD\_SPAN (TotalDepth.util.plot.FILMCfgXML.PhysFilmCfgXMLRead attribute)}

\begin{fulllineitems}
\phantomsection\label{\detokenize{ref/util/plot/FILMCfgXML:TotalDepth.util.plot.FILMCfgXML.PhysFilmCfgXMLRead.TRAC_UNIQUEID_TO_NON_STANDARD_SPAN}}\pysigline{\sphinxbfcode{TRAC\_UNIQUEID\_TO\_NON\_STANDARD\_SPAN}\sphinxbfcode{ = \{'track12': 4, 'track23': 4, 'TrackFC1': 1, 'TrackFC2': 1, 'TrackFC3': 1, 'TrackFC4': 1, 'trackLHT1': 1, 'trackRHT1': 1\}}}
Map of track IDs to start track location in half-tracks

\end{fulllineitems}

\index{TRAC\_UNIQUEID\_TO\_NON\_STANDARD\_START (TotalDepth.util.plot.FILMCfgXML.PhysFilmCfgXMLRead attribute)}

\begin{fulllineitems}
\phantomsection\label{\detokenize{ref/util/plot/FILMCfgXML:TotalDepth.util.plot.FILMCfgXML.PhysFilmCfgXMLRead.TRAC_UNIQUEID_TO_NON_STANDARD_START}}\pysigline{\sphinxbfcode{TRAC\_UNIQUEID\_TO\_NON\_STANDARD\_START}\sphinxbfcode{ = \{'track12': 2, 'track23': 4, 'TrackFC1': 4, 'TrackFC2': 5, 'TrackFC3': 6, 'TrackFC4': 7, 'trackLHT1': 0, 'trackRHT1': 1\}}}
Number of half-tracks of the start of the track, this is a fudge around
deciding this from the order of the LgTrack elements

\end{fulllineitems}

\index{interpretTrac() (TotalDepth.util.plot.FILMCfgXML.PhysFilmCfgXMLRead method)}

\begin{fulllineitems}
\phantomsection\label{\detokenize{ref/util/plot/FILMCfgXML:TotalDepth.util.plot.FILMCfgXML.PhysFilmCfgXMLRead.interpretTrac}}\pysiglinewithargsret{\sphinxbfcode{interpretTrac}}{\emph{theTracStr}}{}
Turns TRAC information into left/right positions as Cooord.Dim()
objects and the number of half-tracks of the start and half-tracks
covered. The later two values are used for stacking and packing the
plot header and footer scales so that they take the minimum space.
e.g. b’T23 ‘ returns:
(the left position of T2, right of T3, 4, 4).
e.g. b’T2 ‘ returns:
(the left position of T2, right of T2, 4, 2).
Note: There is some fudging going on here

\end{fulllineitems}


\end{fulllineitems}

\index{FilmCfgXMLRead (class in TotalDepth.util.plot.FILMCfgXML)}

\begin{fulllineitems}
\phantomsection\label{\detokenize{ref/util/plot/FILMCfgXML:TotalDepth.util.plot.FILMCfgXML.FilmCfgXMLRead}}\pysiglinewithargsret{\sphinxbfcode{class }\sphinxcode{TotalDepth.util.plot.FILMCfgXML.}\sphinxbfcode{FilmCfgXMLRead}}{\emph{directory=None}}{}
Contains the configuration equivalent to a complete FILM table from a set of XML files.
\index{\_\_init\_\_() (TotalDepth.util.plot.FILMCfgXML.FilmCfgXMLRead method)}

\begin{fulllineitems}
\phantomsection\label{\detokenize{ref/util/plot/FILMCfgXML:TotalDepth.util.plot.FILMCfgXML.FilmCfgXMLRead.__init__}}\pysiglinewithargsret{\sphinxbfcode{\_\_init\_\_}}{\emph{directory=None}}{}
Constructor with a directory, all files in the directory (non-recursive)
are read as LGFormat XML files. If dir is None then the “formats/” directory
relative to this module is searched. If dir is an empty string then no search
will be done - useful for testing with addXMLRoot(etree.fromstring(..)).

\end{fulllineitems}

\index{readDir() (TotalDepth.util.plot.FILMCfgXML.FilmCfgXMLRead method)}

\begin{fulllineitems}
\phantomsection\label{\detokenize{ref/util/plot/FILMCfgXML:TotalDepth.util.plot.FILMCfgXML.FilmCfgXMLRead.readDir}}\pysiglinewithargsret{\sphinxbfcode{readDir}}{\emph{d=None}}{}
Read a directory of XML files (not recursive).

\end{fulllineitems}

\index{addXMLRoot() (TotalDepth.util.plot.FILMCfgXML.FilmCfgXMLRead method)}

\begin{fulllineitems}
\phantomsection\label{\detokenize{ref/util/plot/FILMCfgXML:TotalDepth.util.plot.FILMCfgXML.FilmCfgXMLRead.addXMLRoot}}\pysiglinewithargsret{\sphinxbfcode{addXMLRoot}}{\emph{theRoot}}{}
Adds a parsed LgFormat XML document to the IR.

\end{fulllineitems}

\index{chOutpMnemInFilmId() (TotalDepth.util.plot.FILMCfgXML.FilmCfgXMLRead method)}

\begin{fulllineitems}
\phantomsection\label{\detokenize{ref/util/plot/FILMCfgXML:TotalDepth.util.plot.FILMCfgXML.FilmCfgXMLRead.chOutpMnemInFilmId}}\pysiglinewithargsret{\sphinxbfcode{chOutpMnemInFilmId}}{\emph{chOutp}, \emph{filmID}}{}
Returns True if this curve appears in the film.

\end{fulllineitems}

\index{longStr() (TotalDepth.util.plot.FILMCfgXML.FilmCfgXMLRead method)}

\begin{fulllineitems}
\phantomsection\label{\detokenize{ref/util/plot/FILMCfgXML:TotalDepth.util.plot.FILMCfgXML.FilmCfgXMLRead.longStr}}\pysiglinewithargsret{\sphinxbfcode{longStr}}{\emph{verbose=1}}{}
Returns a long string of the XML UniqueIds and their description.

\end{fulllineitems}

\index{uniqueIdS() (TotalDepth.util.plot.FILMCfgXML.FilmCfgXMLRead method)}

\begin{fulllineitems}
\phantomsection\label{\detokenize{ref/util/plot/FILMCfgXML:TotalDepth.util.plot.FILMCfgXML.FilmCfgXMLRead.uniqueIdS}}\pysiglinewithargsret{\sphinxbfcode{uniqueIdS}}{}{}
Returns the UniqueId values that I know about.

\end{fulllineitems}

\index{description() (TotalDepth.util.plot.FILMCfgXML.FilmCfgXMLRead method)}

\begin{fulllineitems}
\phantomsection\label{\detokenize{ref/util/plot/FILMCfgXML:TotalDepth.util.plot.FILMCfgXML.FilmCfgXMLRead.description}}\pysiglinewithargsret{\sphinxbfcode{description}}{\emph{theUID}}{}
Returns the XML description corresponding to the Unique ID or None if unknown.

\end{fulllineitems}

\index{rootNode() (TotalDepth.util.plot.FILMCfgXML.FilmCfgXMLRead method)}

\begin{fulllineitems}
\phantomsection\label{\detokenize{ref/util/plot/FILMCfgXML:TotalDepth.util.plot.FILMCfgXML.FilmCfgXMLRead.rootNode}}\pysiglinewithargsret{\sphinxbfcode{rootNode}}{\emph{theUID}}{}
Returns the XML root node corresponding to the Unique ID.

\end{fulllineitems}

\index{retAllFILMDestS() (TotalDepth.util.plot.FILMCfgXML.FilmCfgXMLRead method)}

\begin{fulllineitems}
\phantomsection\label{\detokenize{ref/util/plot/FILMCfgXML:TotalDepth.util.plot.FILMCfgXML.FilmCfgXMLRead.retAllFILMDestS}}\pysiglinewithargsret{\sphinxbfcode{retAllFILMDestS}}{\emph{curveDestID}}{}
Returns an unordered list of FILM destinations for a curve destination.
For example if curveDestID is b’BOTH’ this might return {[}b‘2   ‘, b‘1   ‘{]}

\end{fulllineitems}

\index{retFILMDest() (TotalDepth.util.plot.FILMCfgXML.FilmCfgXMLRead method)}

\begin{fulllineitems}
\phantomsection\label{\detokenize{ref/util/plot/FILMCfgXML:TotalDepth.util.plot.FILMCfgXML.FilmCfgXMLRead.retFILMDest}}\pysiglinewithargsret{\sphinxbfcode{retFILMDest}}{\emph{filmDestID}, \emph{curveDestID}}{}
Returns a PhysFilmCfgXMLRead object by matching curveDestID to the filmDestID.
Returns None on failure. filmDestID and curveDestID are implementation
specific e.g. for LIS the curveDestID can be 1, BOTH, ALL, NEIT etc.
This is commonly used by the PRESCfg module so that interpretTrac() can
be called on the result and thus build up a map of track positions for
all possible logical film outputs.

\end{fulllineitems}


\end{fulllineitems}



\paragraph{PRES Plotting Configuration}
\label{\detokenize{ref/util/plot/PRESCfg:pres-plotting-configuration}}\label{\detokenize{ref/util/plot/PRESCfg::doc}}\label{\detokenize{ref/util/plot/PRESCfg:module-TotalDepth.util.plot.PRESCfg}}\index{TotalDepth.util.plot.PRESCfg (module)}
Holds all the information to draw a curve on a plot.

TODO: Split this to:
\begin{itemize}
\item {} 
Generic: PRESCfg

\item {} 
LIS specific: PRESCfgLIS

\item {} 
XML specific: PRESCfgXML

\end{itemize}

Created on 20 Mar 2011

Examples of PRES records summaries:

\begin{sphinxVerbatim}[commandchars=\\\{\}]
 \PYG{n}{Table} \PYG{n}{record} \PYG{p}{(}\PYG{n+nb}{type} \PYG{l+m+mi}{34}\PYG{p}{)} \PYG{n+nb}{type}\PYG{p}{:} \PYG{n}{PRES}

\PYG{n}{MNEM}  \PYG{n}{OUTP}  \PYG{n}{STAT}  \PYG{n}{TRAC}  \PYG{n}{CODI}  \PYG{n}{DEST}  \PYG{n}{MODE}      \PYG{n}{FILT}          \PYG{n}{LEDG}          \PYG{n}{REDG}
\PYG{o}{\PYGZhy{}}\PYG{o}{\PYGZhy{}}\PYG{o}{\PYGZhy{}}\PYG{o}{\PYGZhy{}}\PYG{o}{\PYGZhy{}}\PYG{o}{\PYGZhy{}}\PYG{o}{\PYGZhy{}}\PYG{o}{\PYGZhy{}}\PYG{o}{\PYGZhy{}}\PYG{o}{\PYGZhy{}}\PYG{o}{\PYGZhy{}}\PYG{o}{\PYGZhy{}}\PYG{o}{\PYGZhy{}}\PYG{o}{\PYGZhy{}}\PYG{o}{\PYGZhy{}}\PYG{o}{\PYGZhy{}}\PYG{o}{\PYGZhy{}}\PYG{o}{\PYGZhy{}}\PYG{o}{\PYGZhy{}}\PYG{o}{\PYGZhy{}}\PYG{o}{\PYGZhy{}}\PYG{o}{\PYGZhy{}}\PYG{o}{\PYGZhy{}}\PYG{o}{\PYGZhy{}}\PYG{o}{\PYGZhy{}}\PYG{o}{\PYGZhy{}}\PYG{o}{\PYGZhy{}}\PYG{o}{\PYGZhy{}}\PYG{o}{\PYGZhy{}}\PYG{o}{\PYGZhy{}}\PYG{o}{\PYGZhy{}}\PYG{o}{\PYGZhy{}}\PYG{o}{\PYGZhy{}}\PYG{o}{\PYGZhy{}}\PYG{o}{\PYGZhy{}}\PYG{o}{\PYGZhy{}}\PYG{o}{\PYGZhy{}}\PYG{o}{\PYGZhy{}}\PYG{o}{\PYGZhy{}}\PYG{o}{\PYGZhy{}}\PYG{o}{\PYGZhy{}}\PYG{o}{\PYGZhy{}}\PYG{o}{\PYGZhy{}}\PYG{o}{\PYGZhy{}}\PYG{o}{\PYGZhy{}}\PYG{o}{\PYGZhy{}}\PYG{o}{\PYGZhy{}}\PYG{o}{\PYGZhy{}}\PYG{o}{\PYGZhy{}}\PYG{o}{\PYGZhy{}}\PYG{o}{\PYGZhy{}}\PYG{o}{\PYGZhy{}}\PYG{o}{\PYGZhy{}}\PYG{o}{\PYGZhy{}}\PYG{o}{\PYGZhy{}}\PYG{o}{\PYGZhy{}}\PYG{o}{\PYGZhy{}}\PYG{o}{\PYGZhy{}}\PYG{o}{\PYGZhy{}}\PYG{o}{\PYGZhy{}}\PYG{o}{\PYGZhy{}}\PYG{o}{\PYGZhy{}}\PYG{o}{\PYGZhy{}}\PYG{o}{\PYGZhy{}}\PYG{o}{\PYGZhy{}}\PYG{o}{\PYGZhy{}}\PYG{o}{\PYGZhy{}}\PYG{o}{\PYGZhy{}}\PYG{o}{\PYGZhy{}}\PYG{o}{\PYGZhy{}}\PYG{o}{\PYGZhy{}}\PYG{o}{\PYGZhy{}}\PYG{o}{\PYGZhy{}}\PYG{o}{\PYGZhy{}}\PYG{o}{\PYGZhy{}}\PYG{o}{\PYGZhy{}}\PYG{o}{\PYGZhy{}}\PYG{o}{\PYGZhy{}}\PYG{o}{\PYGZhy{}}\PYG{o}{\PYGZhy{}}\PYG{o}{\PYGZhy{}}\PYG{o}{\PYGZhy{}}\PYG{o}{\PYGZhy{}}
                                                                      
\PYG{n}{SP}    \PYG{n}{SP}    \PYG{n}{ALLO}  \PYG{n}{T1}    \PYG{n}{LLIN}  \PYG{l+m+mi}{1}     \PYG{n}{SHIF}      \PYG{l+m+mf}{0.500000}      \PYG{o}{\PYGZhy{}}\PYG{l+m+mf}{80.0000}       \PYG{l+m+mf}{20.0000} 
\PYG{n}{CALI}  \PYG{n}{CALI}  \PYG{n}{ALLO}  \PYG{n}{T1}    \PYG{n}{LDAS}  \PYG{l+m+mi}{1}     \PYG{n}{SHIF}      \PYG{l+m+mf}{0.500000}       \PYG{l+m+mf}{5.00000}       \PYG{l+m+mf}{15.0000} 
\PYG{n}{MINV}  \PYG{n}{MINV}  \PYG{n}{DISA}  \PYG{n}{T1}    \PYG{n}{LLIN}  \PYG{l+m+mi}{1}     \PYG{n}{SHIF}      \PYG{l+m+mf}{0.500000}       \PYG{l+m+mf}{30.0000}       \PYG{l+m+mf}{0.00000} 
\PYG{n}{MNOR}  \PYG{n}{MNOR}  \PYG{n}{DISA}  \PYG{n}{T1}    \PYG{n}{LDAS}  \PYG{l+m+mi}{1}     \PYG{n}{SHIF}      \PYG{l+m+mf}{0.500000}       \PYG{l+m+mf}{30.0000}       \PYG{l+m+mf}{0.00000} 
\PYG{n}{LLD}   \PYG{n}{LLD}   \PYG{n}{ALLO}  \PYG{n}{T23}   \PYG{n}{LDAS}  \PYG{l+m+mi}{1}     \PYG{n}{GRAD}      \PYG{l+m+mf}{0.500000}      \PYG{l+m+mf}{0.200000}       \PYG{l+m+mf}{2000.00} 
\PYG{n}{LLDB}  \PYG{n}{LLD}   \PYG{n}{ALLO}  \PYG{n}{T2}    \PYG{n}{HDAS}  \PYG{l+m+mi}{1}     \PYG{n}{GRAD}      \PYG{l+m+mf}{0.500000}       \PYG{l+m+mf}{2000.00}       \PYG{l+m+mf}{200000.} 
\PYG{n}{LLG}   \PYG{n}{LLG}   \PYG{n}{DISA}  \PYG{n}{T23}   \PYG{n}{LDAS}  \PYG{l+m+mi}{1}     \PYG{n}{GRAD}      \PYG{l+m+mf}{0.500000}      \PYG{l+m+mf}{0.200000}       \PYG{l+m+mf}{2000.00} 
\PYG{n}{LLGB}  \PYG{n}{LLG}   \PYG{n}{DISA}  \PYG{n}{T2}    \PYG{n}{HDAS}  \PYG{l+m+mi}{1}     \PYG{n}{GRAD}      \PYG{l+m+mf}{0.500000}       \PYG{l+m+mf}{2000.00}       \PYG{l+m+mf}{200000.} 
\PYG{n}{LLS}   \PYG{n}{LLS}   \PYG{n}{ALLO}  \PYG{n}{T23}   \PYG{n}{LSPO}  \PYG{l+m+mi}{1}     \PYG{n}{GRAD}      \PYG{l+m+mf}{0.500000}      \PYG{l+m+mf}{0.200000}       \PYG{l+m+mf}{2000.00} 
\PYG{n}{LLSB}  \PYG{n}{LLS}   \PYG{n}{ALLO}  \PYG{n}{T2}    \PYG{n}{HSPO}  \PYG{l+m+mi}{1}     \PYG{n}{GRAD}      \PYG{l+m+mf}{0.500000}       \PYG{l+m+mf}{2000.00}       \PYG{l+m+mf}{200000.} 
\PYG{n}{MSFL}  \PYG{n}{MSFL}  \PYG{n}{ALLO}  \PYG{n}{T23}   \PYG{n}{LLIN}  \PYG{l+m+mi}{1}     \PYG{n}{GRAD}      \PYG{l+m+mf}{0.500000}      \PYG{l+m+mf}{0.200000}       \PYG{l+m+mf}{2000.00} 
\PYG{l+m+mi}{11}    \PYG{n}{DUMM}  \PYG{n}{DISA}  \PYG{n}{T1}    \PYG{n}{LLIN}  \PYG{n}{NEIT}  \PYG{n}{NB}        \PYG{l+m+mf}{0.500000}       \PYG{l+m+mf}{0.00000}       \PYG{l+m+mf}{1.00000} 
\PYG{l+m+mi}{12}    \PYG{n}{DUMM}  \PYG{n}{DISA}  \PYG{n}{T1}    \PYG{n}{LLIN}  \PYG{n}{NEIT}  \PYG{n}{NB}        \PYG{l+m+mf}{0.500000}       \PYG{l+m+mf}{0.00000}       \PYG{l+m+mf}{1.00000} 
\PYG{l+m+mi}{13}    \PYG{n}{DUMM}  \PYG{n}{DISA}  \PYG{n}{T1}    \PYG{n}{LLIN}  \PYG{n}{NEIT}  \PYG{n}{NB}        \PYG{l+m+mf}{0.500000}       \PYG{l+m+mf}{0.00000}       \PYG{l+m+mf}{1.00000} 
\PYG{l+m+mi}{14}    \PYG{n}{DUMM}  \PYG{n}{DISA}  \PYG{n}{T1}    \PYG{n}{LLIN}  \PYG{n}{NEIT}  \PYG{n}{NB}        \PYG{l+m+mf}{0.500000}       \PYG{l+m+mf}{0.00000}       \PYG{l+m+mf}{1.00000} 
\PYG{l+m+mi}{15}    \PYG{n}{DUMM}  \PYG{n}{DISA}  \PYG{n}{T1}    \PYG{n}{LLIN}  \PYG{n}{NEIT}  \PYG{n}{NB}        \PYG{l+m+mf}{0.500000}       \PYG{l+m+mf}{0.00000}       \PYG{l+m+mf}{1.00000} 
\PYG{l+m+mi}{16}    \PYG{n}{DUMM}  \PYG{n}{DISA}  \PYG{n}{T1}    \PYG{n}{LLIN}  \PYG{n}{NEIT}  \PYG{n}{NB}        \PYG{l+m+mf}{0.500000}       \PYG{l+m+mf}{0.00000}       \PYG{l+m+mf}{1.00000} 
\PYG{l+m+mi}{17}    \PYG{n}{DUMM}  \PYG{n}{DISA}  \PYG{n}{T1}    \PYG{n}{LLIN}  \PYG{n}{NEIT}  \PYG{n}{NB}        \PYG{l+m+mf}{0.500000}       \PYG{l+m+mf}{0.00000}       \PYG{l+m+mf}{1.00000} 
\PYG{l+m+mi}{18}    \PYG{n}{DUMM}  \PYG{n}{DISA}  \PYG{n}{T1}    \PYG{n}{LLIN}  \PYG{n}{NEIT}  \PYG{n}{NB}        \PYG{l+m+mf}{0.500000}       \PYG{l+m+mf}{0.00000}       \PYG{l+m+mf}{1.00000} 
\PYG{l+m+mi}{19}    \PYG{n}{DUMM}  \PYG{n}{DISA}  \PYG{n}{T1}    \PYG{n}{LLIN}  \PYG{n}{NEIT}  \PYG{n}{NB}        \PYG{l+m+mf}{0.500000}       \PYG{l+m+mf}{0.00000}       \PYG{l+m+mf}{1.00000} 

\PYG{n}{Table} \PYG{n}{record} \PYG{p}{(}\PYG{n+nb}{type} \PYG{l+m+mi}{34}\PYG{p}{)} \PYG{n+nb}{type}\PYG{p}{:} \PYG{n}{PRES}

\PYG{n}{MNEM}  \PYG{n}{OUTP}  \PYG{n}{STAT}  \PYG{n}{TRAC}  \PYG{n}{CODI}  \PYG{n}{DEST}  \PYG{n}{MODE}      \PYG{n}{FILT}          \PYG{n}{LEDG}          \PYG{n}{REDG}
\PYG{o}{\PYGZhy{}}\PYG{o}{\PYGZhy{}}\PYG{o}{\PYGZhy{}}\PYG{o}{\PYGZhy{}}\PYG{o}{\PYGZhy{}}\PYG{o}{\PYGZhy{}}\PYG{o}{\PYGZhy{}}\PYG{o}{\PYGZhy{}}\PYG{o}{\PYGZhy{}}\PYG{o}{\PYGZhy{}}\PYG{o}{\PYGZhy{}}\PYG{o}{\PYGZhy{}}\PYG{o}{\PYGZhy{}}\PYG{o}{\PYGZhy{}}\PYG{o}{\PYGZhy{}}\PYG{o}{\PYGZhy{}}\PYG{o}{\PYGZhy{}}\PYG{o}{\PYGZhy{}}\PYG{o}{\PYGZhy{}}\PYG{o}{\PYGZhy{}}\PYG{o}{\PYGZhy{}}\PYG{o}{\PYGZhy{}}\PYG{o}{\PYGZhy{}}\PYG{o}{\PYGZhy{}}\PYG{o}{\PYGZhy{}}\PYG{o}{\PYGZhy{}}\PYG{o}{\PYGZhy{}}\PYG{o}{\PYGZhy{}}\PYG{o}{\PYGZhy{}}\PYG{o}{\PYGZhy{}}\PYG{o}{\PYGZhy{}}\PYG{o}{\PYGZhy{}}\PYG{o}{\PYGZhy{}}\PYG{o}{\PYGZhy{}}\PYG{o}{\PYGZhy{}}\PYG{o}{\PYGZhy{}}\PYG{o}{\PYGZhy{}}\PYG{o}{\PYGZhy{}}\PYG{o}{\PYGZhy{}}\PYG{o}{\PYGZhy{}}\PYG{o}{\PYGZhy{}}\PYG{o}{\PYGZhy{}}\PYG{o}{\PYGZhy{}}\PYG{o}{\PYGZhy{}}\PYG{o}{\PYGZhy{}}\PYG{o}{\PYGZhy{}}\PYG{o}{\PYGZhy{}}\PYG{o}{\PYGZhy{}}\PYG{o}{\PYGZhy{}}\PYG{o}{\PYGZhy{}}\PYG{o}{\PYGZhy{}}\PYG{o}{\PYGZhy{}}\PYG{o}{\PYGZhy{}}\PYG{o}{\PYGZhy{}}\PYG{o}{\PYGZhy{}}\PYG{o}{\PYGZhy{}}\PYG{o}{\PYGZhy{}}\PYG{o}{\PYGZhy{}}\PYG{o}{\PYGZhy{}}\PYG{o}{\PYGZhy{}}\PYG{o}{\PYGZhy{}}\PYG{o}{\PYGZhy{}}\PYG{o}{\PYGZhy{}}\PYG{o}{\PYGZhy{}}\PYG{o}{\PYGZhy{}}\PYG{o}{\PYGZhy{}}\PYG{o}{\PYGZhy{}}\PYG{o}{\PYGZhy{}}\PYG{o}{\PYGZhy{}}\PYG{o}{\PYGZhy{}}\PYG{o}{\PYGZhy{}}\PYG{o}{\PYGZhy{}}\PYG{o}{\PYGZhy{}}\PYG{o}{\PYGZhy{}}\PYG{o}{\PYGZhy{}}\PYG{o}{\PYGZhy{}}\PYG{o}{\PYGZhy{}}\PYG{o}{\PYGZhy{}}\PYG{o}{\PYGZhy{}}\PYG{o}{\PYGZhy{}}\PYG{o}{\PYGZhy{}}\PYG{o}{\PYGZhy{}}\PYG{o}{\PYGZhy{}}
\PYG{n}{NPHI}  \PYG{n}{NPHI}  \PYG{n}{ALLO}  \PYG{n}{T23}   \PYG{n}{LDAS}  \PYG{l+m+mi}{1}     \PYG{n}{SHIF}      \PYG{l+m+mf}{0.500000}      \PYG{l+m+mf}{0.450000}     \PYG{o}{\PYGZhy{}}\PYG{l+m+mf}{0.150000}
\PYG{n}{DRHO}  \PYG{n}{DRHO}  \PYG{n}{ALLO}  \PYG{n}{T3}    \PYG{n}{LSPO}  \PYG{l+m+mi}{1}     \PYG{n}{NB}        \PYG{l+m+mf}{0.500000}     \PYG{o}{\PYGZhy{}}\PYG{l+m+mf}{0.250000}      \PYG{l+m+mf}{0.250000}
\PYG{n}{PEF}   \PYG{n}{PEF}   \PYG{n}{ALLO}  \PYG{n}{T23}   \PYG{n}{LGAP}  \PYG{l+m+mi}{1}     \PYG{n}{SHIF}      \PYG{l+m+mf}{0.500000}       \PYG{l+m+mf}{0.00000}       \PYG{l+m+mf}{10.0000}
\PYG{n}{SGR}         \PYG{n}{DISA}  \PYG{n}{T1}    \PYG{n}{LLIN}  \PYG{l+m+mi}{1}     \PYG{n}{SHIF}      \PYG{l+m+mf}{0.500000}       \PYG{l+m+mf}{0.00000}       \PYG{l+m+mf}{300.000}
\PYG{n}{CGR}         \PYG{n}{DISA}  \PYG{n}{T1}    \PYG{n}{LGAP}  \PYG{l+m+mi}{1}     \PYG{n}{SHIF}      \PYG{l+m+mf}{0.500000}       \PYG{l+m+mf}{0.00000}       \PYG{l+m+mf}{300.000}
\PYG{n}{TENS}  \PYG{n}{TENS}  \PYG{n}{DISA}  \PYG{n}{T3}    \PYG{n}{LGAP}  \PYG{l+m+mi}{1}     \PYG{n}{SHIF}      \PYG{l+m+mf}{0.500000}       \PYG{l+m+mf}{14000.0}       \PYG{l+m+mf}{4000.00}
\PYG{n}{CAL}   \PYG{n}{CALI}  \PYG{n}{ALLO}  \PYG{n}{T1}    \PYG{n}{LSPO}  \PYG{l+m+mi}{1}     \PYG{n}{SHIF}      \PYG{l+m+mf}{0.500000}       \PYG{l+m+mf}{5.00000}       \PYG{l+m+mf}{15.0000}
\PYG{n}{BS}    \PYG{n}{BS}    \PYG{n}{DISA}  \PYG{n}{T1}    \PYG{n}{LGAP}  \PYG{l+m+mi}{1}     \PYG{n}{SHIF}      \PYG{l+m+mf}{0.500000}       \PYG{l+m+mf}{5.00000}       \PYG{l+m+mf}{15.0000}
\PYG{n}{FFLS}  \PYG{n}{FFLS}  \PYG{n}{DISA}  \PYG{n}{T1}    \PYG{n}{LLIN}  \PYG{l+m+mi}{2}     \PYG{n}{NB}        \PYG{l+m+mf}{0.500000}     \PYG{o}{\PYGZhy{}}\PYG{l+m+mf}{0.150000}      \PYG{l+m+mf}{0.150000}
\PYG{n}{FFSS}  \PYG{n}{FFSS}  \PYG{n}{DISA}  \PYG{n}{T1}    \PYG{n}{LDAS}  \PYG{l+m+mi}{2}     \PYG{n}{NB}        \PYG{l+m+mf}{0.500000}     \PYG{o}{\PYGZhy{}}\PYG{l+m+mf}{0.150000}      \PYG{l+m+mf}{0.150000}
\PYG{n}{LSHV}  \PYG{n}{LSHV}  \PYG{n}{DISA}  \PYG{n}{T3}    \PYG{n}{LLIN}  \PYG{l+m+mi}{2}     \PYG{n}{WRAP}      \PYG{l+m+mf}{0.500000}       \PYG{l+m+mf}{2150.00}       \PYG{l+m+mf}{2250.00}
\PYG{n}{SSHV}  \PYG{n}{SSHV}  \PYG{n}{DISA}  \PYG{n}{T3}    \PYG{n}{LDAS}  \PYG{l+m+mi}{2}     \PYG{n}{WRAP}      \PYG{l+m+mf}{0.500000}       \PYG{l+m+mf}{1950.00}       \PYG{l+m+mf}{2050.00}
\PYG{n}{FLS}   \PYG{n}{FLS}   \PYG{n}{DISA}  \PYG{n}{T2}    \PYG{n}{LLIN}  \PYG{l+m+mi}{2}     \PYG{n}{SHIF}      \PYG{l+m+mf}{0.500000}       \PYG{l+m+mf}{0.00000}       \PYG{l+m+mf}{150.000}
\PYG{n}{FSS}   \PYG{n}{FSS}   \PYG{n}{DISA}  \PYG{n}{T2}    \PYG{n}{LDAS}  \PYG{l+m+mi}{2}     \PYG{n}{SHIF}      \PYG{l+m+mf}{0.500000}       \PYG{l+m+mf}{0.00000}       \PYG{l+m+mf}{150.000}
\PYG{n}{RHOB}  \PYG{n}{RHOB}  \PYG{n}{ALLO}  \PYG{n}{T23}   \PYG{n}{LLIN}  \PYG{l+m+mi}{1}     \PYG{n}{SHIF}      \PYG{l+m+mf}{0.500000}       \PYG{l+m+mf}{1.95000}       \PYG{l+m+mf}{2.95000}
\PYG{n}{PHIX}  \PYG{n}{PHIX}  \PYG{n}{ALLO}  \PYG{n}{T1}    \PYG{n}{LLIN}  \PYG{l+m+mi}{1}     \PYG{n}{NB}        \PYG{l+m+mf}{0.500000}      \PYG{l+m+mf}{0.500000}       \PYG{l+m+mf}{0.00000}
\end{sphinxVerbatim}

TRAC nomenclature:

\sphinxcode{\textless{}LH \textbar{} RH\textgreater{} T n \textless{}m\textgreater{}}

Or as a regex: \sphinxcode{re.compile(r'\textasciicircum{}(LH\textbar{}RH)*T(\textbackslash{}d)(\textbackslash{}d)*\textbackslash{}s*\$')}

But this is handled by FILMCfg.

Example:

\begin{sphinxVerbatim}[commandchars=\\\{\}]
\PYGZdl{} python3 TableHistogram.py \PYGZhy{}k \PYGZhy{}\PYGZhy{}name=PRES \PYGZhy{}l40 \PYGZhy{}\PYGZhy{}col=TRAC ../../../pLogicTestData/LIS/
Cmd: TableHistogram.py \PYGZhy{}k \PYGZhy{}\PYGZhy{}name=PRES \PYGZhy{}l40 \PYGZhy{}\PYGZhy{}col=TRAC ../../../pLogicTestData/LIS/
2011\PYGZhy{}05\PYGZhy{}27 09:26:12,324 ERROR    Can not create Logical Record, error: can\PYGZsq{}t convert negative value to unsigned int
2011\PYGZhy{}05\PYGZhy{}27 09:26:12,335 ERROR    Can not create Logical Record, error: can\PYGZsq{}t convert negative value to unsigned int
2011\PYGZhy{}05\PYGZhy{}27 09:26:12,346 ERROR    Can not create Logical Record, error: can\PYGZsq{}t convert negative value to unsigned int
2011\PYGZhy{}05\PYGZhy{}27 09:26:13,086 ERROR    Can not read LIS file ../../../pLogicTestData/LIS/13576.S1 with error: Can not fit integer number of frames length 120 into LR length 824, modulo 104 [indirect size 0].
2011\PYGZhy{}05\PYGZhy{}27 09:26:13,907 ERROR    Can not read LIS file ../../../pLogicTestData/LIS/13610.S1 with error: Can not fit integer number of frames length 7176 into LR length 13354, modulo 6178 [indirect size 0].
======================== Count of all table entries =======================
\PYGZob{}\PYGZdq{}(34, b\PYGZsq{}PRES\PYGZsq{}, b\PYGZsq{}TRAC\PYGZsq{}, b\PYGZsq{}F1  \PYGZsq{})\PYGZdq{}: 4,
 \PYGZdq{}(34, b\PYGZsq{}PRES\PYGZsq{}, b\PYGZsq{}TRAC\PYGZsq{}, b\PYGZsq{}F2  \PYGZsq{})\PYGZdq{}: 4,
 \PYGZdq{}(34, b\PYGZsq{}PRES\PYGZsq{}, b\PYGZsq{}TRAC\PYGZsq{}, b\PYGZsq{}F3  \PYGZsq{})\PYGZdq{}: 4,
 \PYGZdq{}(34, b\PYGZsq{}PRES\PYGZsq{}, b\PYGZsq{}TRAC\PYGZsq{}, b\PYGZsq{}F4  \PYGZsq{})\PYGZdq{}: 3,
 \PYGZdq{}(34, b\PYGZsq{}PRES\PYGZsq{}, b\PYGZsq{}TRAC\PYGZsq{}, b\PYGZsq{}FD  \PYGZsq{})\PYGZdq{}: 4,
 \PYGZdq{}(34, b\PYGZsq{}PRES\PYGZsq{}, b\PYGZsq{}TRAC\PYGZsq{}, b\PYGZsq{}LHT1\PYGZsq{})\PYGZdq{}: 4,
 \PYGZdq{}(34, b\PYGZsq{}PRES\PYGZsq{}, b\PYGZsq{}TRAC\PYGZsq{}, b\PYGZsq{}LHT2\PYGZsq{})\PYGZdq{}: 8,
 \PYGZdq{}(34, b\PYGZsq{}PRES\PYGZsq{}, b\PYGZsq{}TRAC\PYGZsq{}, b\PYGZsq{}LHT3\PYGZsq{})\PYGZdq{}: 59,
 \PYGZdq{}(34, b\PYGZsq{}PRES\PYGZsq{}, b\PYGZsq{}TRAC\PYGZsq{}, b\PYGZsq{}RHT1\PYGZsq{})\PYGZdq{}: 4,
 \PYGZdq{}(34, b\PYGZsq{}PRES\PYGZsq{}, b\PYGZsq{}TRAC\PYGZsq{}, b\PYGZsq{}RHT2\PYGZsq{})\PYGZdq{}: 22,
 \PYGZdq{}(34, b\PYGZsq{}PRES\PYGZsq{}, b\PYGZsq{}TRAC\PYGZsq{}, b\PYGZsq{}RHT3\PYGZsq{})\PYGZdq{}: 63,
 \PYGZdq{}(34, b\PYGZsq{}PRES\PYGZsq{}, b\PYGZsq{}TRAC\PYGZsq{}, b\PYGZsq{}T1  \PYGZsq{})\PYGZdq{}: 2363,
 \PYGZdq{}(34, b\PYGZsq{}PRES\PYGZsq{}, b\PYGZsq{}TRAC\PYGZsq{}, b\PYGZsq{}T2  \PYGZsq{})\PYGZdq{}: 354,
 \PYGZdq{}(34, b\PYGZsq{}PRES\PYGZsq{}, b\PYGZsq{}TRAC\PYGZsq{}, b\PYGZsq{}T23 \PYGZsq{})\PYGZdq{}: 192,
 \PYGZdq{}(34, b\PYGZsq{}PRES\PYGZsq{}, b\PYGZsq{}TRAC\PYGZsq{}, b\PYGZsq{}T3  \PYGZsq{})\PYGZdq{}: 178,
 \PYGZdq{}(34, b\PYGZsq{}PRES\PYGZsq{}, b\PYGZsq{}TRAC\PYGZsq{}, b\PYGZsq{}TD  \PYGZsq{})\PYGZdq{}: 93,
 \PYGZdq{}(34, b\PYGZsq{}PRES\PYGZsq{}, b\PYGZsq{}TRAC\PYGZsq{}, b\PYGZsq{}XXXX\PYGZsq{})\PYGZdq{}: 18\PYGZcb{}
====================== Count of all table entries END =====================
\end{sphinxVerbatim}

DEST:

\begin{sphinxVerbatim}[commandchars=\\\{\}]
\PYG{o}{==}\PYG{o}{==}\PYG{o}{==}\PYG{o}{==}\PYG{o}{==}\PYG{o}{==}\PYG{o}{==}\PYG{o}{==}\PYG{o}{==}\PYG{o}{==}\PYG{o}{==}\PYG{o}{==} \PYG{n}{Count} \PYG{n}{of} \PYG{n+nb}{all} \PYG{n}{table} \PYG{n}{entries} \PYG{o}{==}\PYG{o}{==}\PYG{o}{==}\PYG{o}{==}\PYG{o}{==}\PYG{o}{==}\PYG{o}{==}\PYG{o}{==}\PYG{o}{==}\PYG{o}{==}\PYG{o}{==}\PYG{o}{=}
\PYG{p}{\PYGZob{}}\PYG{l+s+s2}{\PYGZdq{}}\PYG{l+s+s2}{(34, b}\PYG{l+s+s2}{\PYGZsq{}}\PYG{l+s+s2}{PRES}\PYG{l+s+s2}{\PYGZsq{}}\PYG{l+s+s2}{, b}\PYG{l+s+s2}{\PYGZsq{}}\PYG{l+s+s2}{DEST}\PYG{l+s+s2}{\PYGZsq{}}\PYG{l+s+s2}{, b}\PYG{l+s+s2}{\PYGZsq{}}\PYG{l+s+s2}{1   }\PYG{l+s+s2}{\PYGZsq{}}\PYG{l+s+s2}{)}\PYG{l+s+s2}{\PYGZdq{}}\PYG{p}{:} \PYG{l+m+mi}{457}\PYG{p}{,}
 \PYG{l+s+s2}{\PYGZdq{}}\PYG{l+s+s2}{(34, b}\PYG{l+s+s2}{\PYGZsq{}}\PYG{l+s+s2}{PRES}\PYG{l+s+s2}{\PYGZsq{}}\PYG{l+s+s2}{, b}\PYG{l+s+s2}{\PYGZsq{}}\PYG{l+s+s2}{DEST}\PYG{l+s+s2}{\PYGZsq{}}\PYG{l+s+s2}{, b}\PYG{l+s+s2}{\PYGZsq{}}\PYG{l+s+s2}{123 }\PYG{l+s+s2}{\PYGZsq{}}\PYG{l+s+s2}{)}\PYG{l+s+s2}{\PYGZdq{}}\PYG{p}{:} \PYG{l+m+mi}{2}\PYG{p}{,}
 \PYG{l+s+s2}{\PYGZdq{}}\PYG{l+s+s2}{(34, b}\PYG{l+s+s2}{\PYGZsq{}}\PYG{l+s+s2}{PRES}\PYG{l+s+s2}{\PYGZsq{}}\PYG{l+s+s2}{, b}\PYG{l+s+s2}{\PYGZsq{}}\PYG{l+s+s2}{DEST}\PYG{l+s+s2}{\PYGZsq{}}\PYG{l+s+s2}{, b}\PYG{l+s+s2}{\PYGZsq{}}\PYG{l+s+s2}{134 }\PYG{l+s+s2}{\PYGZsq{}}\PYG{l+s+s2}{)}\PYG{l+s+s2}{\PYGZdq{}}\PYG{p}{:} \PYG{l+m+mi}{20}\PYG{p}{,}
 \PYG{l+s+s2}{\PYGZdq{}}\PYG{l+s+s2}{(34, b}\PYG{l+s+s2}{\PYGZsq{}}\PYG{l+s+s2}{PRES}\PYG{l+s+s2}{\PYGZsq{}}\PYG{l+s+s2}{, b}\PYG{l+s+s2}{\PYGZsq{}}\PYG{l+s+s2}{DEST}\PYG{l+s+s2}{\PYGZsq{}}\PYG{l+s+s2}{, b}\PYG{l+s+s2}{\PYGZsq{}}\PYG{l+s+s2}{2   }\PYG{l+s+s2}{\PYGZsq{}}\PYG{l+s+s2}{)}\PYG{l+s+s2}{\PYGZdq{}}\PYG{p}{:} \PYG{l+m+mi}{139}\PYG{p}{,}
 \PYG{l+s+s2}{\PYGZdq{}}\PYG{l+s+s2}{(34, b}\PYG{l+s+s2}{\PYGZsq{}}\PYG{l+s+s2}{PRES}\PYG{l+s+s2}{\PYGZsq{}}\PYG{l+s+s2}{, b}\PYG{l+s+s2}{\PYGZsq{}}\PYG{l+s+s2}{DEST}\PYG{l+s+s2}{\PYGZsq{}}\PYG{l+s+s2}{, b}\PYG{l+s+s2}{\PYGZsq{}}\PYG{l+s+s2}{5   }\PYG{l+s+s2}{\PYGZsq{}}\PYG{l+s+s2}{)}\PYG{l+s+s2}{\PYGZdq{}}\PYG{p}{:} \PYG{l+m+mi}{26}\PYG{p}{,}
 \PYG{l+s+s2}{\PYGZdq{}}\PYG{l+s+s2}{(34, b}\PYG{l+s+s2}{\PYGZsq{}}\PYG{l+s+s2}{PRES}\PYG{l+s+s2}{\PYGZsq{}}\PYG{l+s+s2}{, b}\PYG{l+s+s2}{\PYGZsq{}}\PYG{l+s+s2}{DEST}\PYG{l+s+s2}{\PYGZsq{}}\PYG{l+s+s2}{, b}\PYG{l+s+s2}{\PYGZsq{}}\PYG{l+s+s2}{6   }\PYG{l+s+s2}{\PYGZsq{}}\PYG{l+s+s2}{)}\PYG{l+s+s2}{\PYGZdq{}}\PYG{p}{:} \PYG{l+m+mi}{122}\PYG{p}{,}
 \PYG{l+s+s2}{\PYGZdq{}}\PYG{l+s+s2}{(34, b}\PYG{l+s+s2}{\PYGZsq{}}\PYG{l+s+s2}{PRES}\PYG{l+s+s2}{\PYGZsq{}}\PYG{l+s+s2}{, b}\PYG{l+s+s2}{\PYGZsq{}}\PYG{l+s+s2}{DEST}\PYG{l+s+s2}{\PYGZsq{}}\PYG{l+s+s2}{, b}\PYG{l+s+s2}{\PYGZsq{}}\PYG{l+s+s2}{8   }\PYG{l+s+s2}{\PYGZsq{}}\PYG{l+s+s2}{)}\PYG{l+s+s2}{\PYGZdq{}}\PYG{p}{:} \PYG{l+m+mi}{16}\PYG{p}{,}
 \PYG{l+s+s2}{\PYGZdq{}}\PYG{l+s+s2}{(34, b}\PYG{l+s+s2}{\PYGZsq{}}\PYG{l+s+s2}{PRES}\PYG{l+s+s2}{\PYGZsq{}}\PYG{l+s+s2}{, b}\PYG{l+s+s2}{\PYGZsq{}}\PYG{l+s+s2}{DEST}\PYG{l+s+s2}{\PYGZsq{}}\PYG{l+s+s2}{, b}\PYG{l+s+s2}{\PYGZsq{}}\PYG{l+s+s2}{A   }\PYG{l+s+s2}{\PYGZsq{}}\PYG{l+s+s2}{)}\PYG{l+s+s2}{\PYGZdq{}}\PYG{p}{:} \PYG{l+m+mi}{19}\PYG{p}{,}
 \PYG{l+s+s2}{\PYGZdq{}}\PYG{l+s+s2}{(34, b}\PYG{l+s+s2}{\PYGZsq{}}\PYG{l+s+s2}{PRES}\PYG{l+s+s2}{\PYGZsq{}}\PYG{l+s+s2}{, b}\PYG{l+s+s2}{\PYGZsq{}}\PYG{l+s+s2}{DEST}\PYG{l+s+s2}{\PYGZsq{}}\PYG{l+s+s2}{, b}\PYG{l+s+s2}{\PYGZsq{}}\PYG{l+s+s2}{ALL }\PYG{l+s+s2}{\PYGZsq{}}\PYG{l+s+s2}{)}\PYG{l+s+s2}{\PYGZdq{}}\PYG{p}{:} \PYG{l+m+mi}{4}\PYG{p}{,}
 \PYG{l+s+s2}{\PYGZdq{}}\PYG{l+s+s2}{(34, b}\PYG{l+s+s2}{\PYGZsq{}}\PYG{l+s+s2}{PRES}\PYG{l+s+s2}{\PYGZsq{}}\PYG{l+s+s2}{, b}\PYG{l+s+s2}{\PYGZsq{}}\PYG{l+s+s2}{DEST}\PYG{l+s+s2}{\PYGZsq{}}\PYG{l+s+s2}{, b}\PYG{l+s+s2}{\PYGZsq{}}\PYG{l+s+s2}{BOTH}\PYG{l+s+s2}{\PYGZsq{}}\PYG{l+s+s2}{)}\PYG{l+s+s2}{\PYGZdq{}}\PYG{p}{:} \PYG{l+m+mi}{774}\PYG{p}{,}
 \PYG{l+s+s2}{\PYGZdq{}}\PYG{l+s+s2}{(34, b}\PYG{l+s+s2}{\PYGZsq{}}\PYG{l+s+s2}{PRES}\PYG{l+s+s2}{\PYGZsq{}}\PYG{l+s+s2}{, b}\PYG{l+s+s2}{\PYGZsq{}}\PYG{l+s+s2}{DEST}\PYG{l+s+s2}{\PYGZsq{}}\PYG{l+s+s2}{, b}\PYG{l+s+s2}{\PYGZsq{}}\PYG{l+s+s2}{D   }\PYG{l+s+s2}{\PYGZsq{}}\PYG{l+s+s2}{)}\PYG{l+s+s2}{\PYGZdq{}}\PYG{p}{:} \PYG{l+m+mi}{63}\PYG{p}{,}
 \PYG{l+s+s2}{\PYGZdq{}}\PYG{l+s+s2}{(34, b}\PYG{l+s+s2}{\PYGZsq{}}\PYG{l+s+s2}{PRES}\PYG{l+s+s2}{\PYGZsq{}}\PYG{l+s+s2}{, b}\PYG{l+s+s2}{\PYGZsq{}}\PYG{l+s+s2}{DEST}\PYG{l+s+s2}{\PYGZsq{}}\PYG{l+s+s2}{, b}\PYG{l+s+s2}{\PYGZsq{}}\PYG{l+s+s2}{E   }\PYG{l+s+s2}{\PYGZsq{}}\PYG{l+s+s2}{)}\PYG{l+s+s2}{\PYGZdq{}}\PYG{p}{:} \PYG{l+m+mi}{60}\PYG{p}{,}
 \PYG{l+s+s2}{\PYGZdq{}}\PYG{l+s+s2}{(34, b}\PYG{l+s+s2}{\PYGZsq{}}\PYG{l+s+s2}{PRES}\PYG{l+s+s2}{\PYGZsq{}}\PYG{l+s+s2}{, b}\PYG{l+s+s2}{\PYGZsq{}}\PYG{l+s+s2}{DEST}\PYG{l+s+s2}{\PYGZsq{}}\PYG{l+s+s2}{, b}\PYG{l+s+s2}{\PYGZsq{}}\PYG{l+s+s2}{J   }\PYG{l+s+s2}{\PYGZsq{}}\PYG{l+s+s2}{)}\PYG{l+s+s2}{\PYGZdq{}}\PYG{p}{:} \PYG{l+m+mi}{52}\PYG{p}{,}
 \PYG{l+s+s2}{\PYGZdq{}}\PYG{l+s+s2}{(34, b}\PYG{l+s+s2}{\PYGZsq{}}\PYG{l+s+s2}{PRES}\PYG{l+s+s2}{\PYGZsq{}}\PYG{l+s+s2}{, b}\PYG{l+s+s2}{\PYGZsq{}}\PYG{l+s+s2}{DEST}\PYG{l+s+s2}{\PYGZsq{}}\PYG{l+s+s2}{, b}\PYG{l+s+s2}{\PYGZsq{}}\PYG{l+s+s2}{K   }\PYG{l+s+s2}{\PYGZsq{}}\PYG{l+s+s2}{)}\PYG{l+s+s2}{\PYGZdq{}}\PYG{p}{:} \PYG{l+m+mi}{13}\PYG{p}{,}
 \PYG{l+s+s2}{\PYGZdq{}}\PYG{l+s+s2}{(34, b}\PYG{l+s+s2}{\PYGZsq{}}\PYG{l+s+s2}{PRES}\PYG{l+s+s2}{\PYGZsq{}}\PYG{l+s+s2}{, b}\PYG{l+s+s2}{\PYGZsq{}}\PYG{l+s+s2}{DEST}\PYG{l+s+s2}{\PYGZsq{}}\PYG{l+s+s2}{, b}\PYG{l+s+s2}{\PYGZsq{}}\PYG{l+s+s2}{NEIT}\PYG{l+s+s2}{\PYGZsq{}}\PYG{l+s+s2}{)}\PYG{l+s+s2}{\PYGZdq{}}\PYG{p}{:} \PYG{l+m+mi}{1610}\PYG{p}{\PYGZcb{}}
\PYG{o}{==}\PYG{o}{==}\PYG{o}{==}\PYG{o}{==}\PYG{o}{==}\PYG{o}{==}\PYG{o}{==}\PYG{o}{==}\PYG{o}{==}\PYG{o}{==}\PYG{o}{==} \PYG{n}{Count} \PYG{n}{of} \PYG{n+nb}{all} \PYG{n}{table} \PYG{n}{entries} \PYG{n}{END} \PYG{o}{==}\PYG{o}{==}\PYG{o}{==}\PYG{o}{==}\PYG{o}{==}\PYG{o}{==}\PYG{o}{==}\PYG{o}{==}\PYG{o}{==}\PYG{o}{==}\PYG{o}{=}
\end{sphinxVerbatim}

MODE:

\begin{sphinxVerbatim}[commandchars=\\\{\}]
\PYGZdl{} python3 TableHistogram.py \PYGZhy{}k \PYGZhy{}\PYGZhy{}name=PRES \PYGZhy{}l40 \PYGZhy{}\PYGZhy{}col=MODE ../../../pLogicTestData/LIS/
Cmd: TableHistogram.py \PYGZhy{}k \PYGZhy{}\PYGZhy{}name=PRES \PYGZhy{}l40 \PYGZhy{}\PYGZhy{}col=MODE ../../../pLogicTestData/LIS/
2011\PYGZhy{}06\PYGZhy{}02 08:22:54,839 ERROR    Can not read LIS file ../../../pLogicTestData/LIS/13576.S1 with error: Can not fit integer number of frames length 120 into LR length 824, modulo 104 [indirect size 0].
2011\PYGZhy{}06\PYGZhy{}02 08:22:55,671 ERROR    Can not read LIS file ../../../pLogicTestData/LIS/13610.S1 with error: Can not fit integer number of frames length 7176 into LR length 13354, modulo 6178 [indirect size 0].
======================== Count of all table entries =======================
\PYGZob{}\PYGZdq{}(34, b\PYGZsq{}PRES\PYGZsq{}, b\PYGZsq{}MODE\PYGZsq{}, b\PYGZsq{}GRAD\PYGZsq{})\PYGZdq{}: 245,
 \PYGZdq{}(34, b\PYGZsq{}PRES\PYGZsq{}, b\PYGZsq{}MODE\PYGZsq{}, b\PYGZsq{}NB  \PYGZsq{})\PYGZdq{}: 2329,
 \PYGZdq{}(34, b\PYGZsq{}PRES\PYGZsq{}, b\PYGZsq{}MODE\PYGZsq{}, b\PYGZsq{}SHIF\PYGZsq{})\PYGZdq{}: 597,
 \PYGZdq{}(34, b\PYGZsq{}PRES\PYGZsq{}, b\PYGZsq{}MODE\PYGZsq{}, b\PYGZsq{}SWF \PYGZsq{})\PYGZdq{}: 3,
 \PYGZdq{}(34, b\PYGZsq{}PRES\PYGZsq{}, b\PYGZsq{}MODE\PYGZsq{}, b\PYGZsq{}VDLN\PYGZsq{})\PYGZdq{}: 3,
 \PYGZdq{}(34, b\PYGZsq{}PRES\PYGZsq{}, b\PYGZsq{}MODE\PYGZsq{}, b\PYGZsq{}WRAP\PYGZsq{})\PYGZdq{}: 186,
 \PYGZdq{}(34, b\PYGZsq{}PRES\PYGZsq{}, b\PYGZsq{}MODE\PYGZsq{}, b\PYGZsq{}X10 \PYGZsq{})\PYGZdq{}: 10\PYGZcb{}
====================== Count of all table entries END =====================
\end{sphinxVerbatim}

COLO:

\begin{sphinxVerbatim}[commandchars=\\\{\}]
\PYG{o}{==}\PYG{o}{==}\PYG{o}{==}\PYG{o}{==}\PYG{o}{==}\PYG{o}{==}\PYG{o}{==}\PYG{o}{==}\PYG{o}{==}\PYG{o}{==}\PYG{o}{==}\PYG{o}{==} \PYG{n}{Count} \PYG{n}{of} \PYG{n+nb}{all} \PYG{n}{table} \PYG{n}{entries} \PYG{o}{==}\PYG{o}{==}\PYG{o}{==}\PYG{o}{==}\PYG{o}{==}\PYG{o}{==}\PYG{o}{==}\PYG{o}{==}\PYG{o}{==}\PYG{o}{==}\PYG{o}{==}\PYG{o}{=}
\PYG{p}{\PYGZob{}}\PYG{l+s+s2}{\PYGZdq{}}\PYG{l+s+s2}{(34, b}\PYG{l+s+s2}{\PYGZsq{}}\PYG{l+s+s2}{PRES}\PYG{l+s+s2}{\PYGZsq{}}\PYG{l+s+s2}{, b}\PYG{l+s+s2}{\PYGZsq{}}\PYG{l+s+s2}{COLO}\PYG{l+s+s2}{\PYGZsq{}}\PYG{l+s+s2}{, b}\PYG{l+s+s2}{\PYGZsq{}}\PYG{l+s+s2}{000 }\PYG{l+s+s2}{\PYGZsq{}}\PYG{l+s+s2}{)}\PYG{l+s+s2}{\PYGZdq{}}\PYG{p}{:} \PYG{l+m+mi}{173}\PYG{p}{,}
 \PYG{l+s+s2}{\PYGZdq{}}\PYG{l+s+s2}{(34, b}\PYG{l+s+s2}{\PYGZsq{}}\PYG{l+s+s2}{PRES}\PYG{l+s+s2}{\PYGZsq{}}\PYG{l+s+s2}{, b}\PYG{l+s+s2}{\PYGZsq{}}\PYG{l+s+s2}{COLO}\PYG{l+s+s2}{\PYGZsq{}}\PYG{l+s+s2}{, b}\PYG{l+s+s2}{\PYGZsq{}}\PYG{l+s+s2}{002 }\PYG{l+s+s2}{\PYGZsq{}}\PYG{l+s+s2}{)}\PYG{l+s+s2}{\PYGZdq{}}\PYG{p}{:} \PYG{l+m+mi}{3}\PYG{p}{,}
 \PYG{l+s+s2}{\PYGZdq{}}\PYG{l+s+s2}{(34, b}\PYG{l+s+s2}{\PYGZsq{}}\PYG{l+s+s2}{PRES}\PYG{l+s+s2}{\PYGZsq{}}\PYG{l+s+s2}{, b}\PYG{l+s+s2}{\PYGZsq{}}\PYG{l+s+s2}{COLO}\PYG{l+s+s2}{\PYGZsq{}}\PYG{l+s+s2}{, b}\PYG{l+s+s2}{\PYGZsq{}}\PYG{l+s+s2}{003 }\PYG{l+s+s2}{\PYGZsq{}}\PYG{l+s+s2}{)}\PYG{l+s+s2}{\PYGZdq{}}\PYG{p}{:} \PYG{l+m+mi}{6}\PYG{p}{,}
 \PYG{l+s+s2}{\PYGZdq{}}\PYG{l+s+s2}{(34, b}\PYG{l+s+s2}{\PYGZsq{}}\PYG{l+s+s2}{PRES}\PYG{l+s+s2}{\PYGZsq{}}\PYG{l+s+s2}{, b}\PYG{l+s+s2}{\PYGZsq{}}\PYG{l+s+s2}{COLO}\PYG{l+s+s2}{\PYGZsq{}}\PYG{l+s+s2}{, b}\PYG{l+s+s2}{\PYGZsq{}}\PYG{l+s+s2}{004 }\PYG{l+s+s2}{\PYGZsq{}}\PYG{l+s+s2}{)}\PYG{l+s+s2}{\PYGZdq{}}\PYG{p}{:} \PYG{l+m+mi}{33}\PYG{p}{,}
 \PYG{l+s+s2}{\PYGZdq{}}\PYG{l+s+s2}{(34, b}\PYG{l+s+s2}{\PYGZsq{}}\PYG{l+s+s2}{PRES}\PYG{l+s+s2}{\PYGZsq{}}\PYG{l+s+s2}{, b}\PYG{l+s+s2}{\PYGZsq{}}\PYG{l+s+s2}{COLO}\PYG{l+s+s2}{\PYGZsq{}}\PYG{l+s+s2}{, b}\PYG{l+s+s2}{\PYGZsq{}}\PYG{l+s+s2}{014 }\PYG{l+s+s2}{\PYGZsq{}}\PYG{l+s+s2}{)}\PYG{l+s+s2}{\PYGZdq{}}\PYG{p}{:} \PYG{l+m+mi}{6}\PYG{p}{,}
 \PYG{l+s+s2}{\PYGZdq{}}\PYG{l+s+s2}{(34, b}\PYG{l+s+s2}{\PYGZsq{}}\PYG{l+s+s2}{PRES}\PYG{l+s+s2}{\PYGZsq{}}\PYG{l+s+s2}{, b}\PYG{l+s+s2}{\PYGZsq{}}\PYG{l+s+s2}{COLO}\PYG{l+s+s2}{\PYGZsq{}}\PYG{l+s+s2}{, b}\PYG{l+s+s2}{\PYGZsq{}}\PYG{l+s+s2}{020 }\PYG{l+s+s2}{\PYGZsq{}}\PYG{l+s+s2}{)}\PYG{l+s+s2}{\PYGZdq{}}\PYG{p}{:} \PYG{l+m+mi}{9}\PYG{p}{,}
 \PYG{l+s+s2}{\PYGZdq{}}\PYG{l+s+s2}{(34, b}\PYG{l+s+s2}{\PYGZsq{}}\PYG{l+s+s2}{PRES}\PYG{l+s+s2}{\PYGZsq{}}\PYG{l+s+s2}{, b}\PYG{l+s+s2}{\PYGZsq{}}\PYG{l+s+s2}{COLO}\PYG{l+s+s2}{\PYGZsq{}}\PYG{l+s+s2}{, b}\PYG{l+s+s2}{\PYGZsq{}}\PYG{l+s+s2}{021 }\PYG{l+s+s2}{\PYGZsq{}}\PYG{l+s+s2}{)}\PYG{l+s+s2}{\PYGZdq{}}\PYG{p}{:} \PYG{l+m+mi}{8}\PYG{p}{,}
 \PYG{l+s+s2}{\PYGZdq{}}\PYG{l+s+s2}{(34, b}\PYG{l+s+s2}{\PYGZsq{}}\PYG{l+s+s2}{PRES}\PYG{l+s+s2}{\PYGZsq{}}\PYG{l+s+s2}{, b}\PYG{l+s+s2}{\PYGZsq{}}\PYG{l+s+s2}{COLO}\PYG{l+s+s2}{\PYGZsq{}}\PYG{l+s+s2}{, b}\PYG{l+s+s2}{\PYGZsq{}}\PYG{l+s+s2}{030 }\PYG{l+s+s2}{\PYGZsq{}}\PYG{l+s+s2}{)}\PYG{l+s+s2}{\PYGZdq{}}\PYG{p}{:} \PYG{l+m+mi}{34}\PYG{p}{,}
 \PYG{l+s+s2}{\PYGZdq{}}\PYG{l+s+s2}{(34, b}\PYG{l+s+s2}{\PYGZsq{}}\PYG{l+s+s2}{PRES}\PYG{l+s+s2}{\PYGZsq{}}\PYG{l+s+s2}{, b}\PYG{l+s+s2}{\PYGZsq{}}\PYG{l+s+s2}{COLO}\PYG{l+s+s2}{\PYGZsq{}}\PYG{l+s+s2}{, b}\PYG{l+s+s2}{\PYGZsq{}}\PYG{l+s+s2}{034 }\PYG{l+s+s2}{\PYGZsq{}}\PYG{l+s+s2}{)}\PYG{l+s+s2}{\PYGZdq{}}\PYG{p}{:} \PYG{l+m+mi}{2}\PYG{p}{,}
 \PYG{l+s+s2}{\PYGZdq{}}\PYG{l+s+s2}{(34, b}\PYG{l+s+s2}{\PYGZsq{}}\PYG{l+s+s2}{PRES}\PYG{l+s+s2}{\PYGZsq{}}\PYG{l+s+s2}{, b}\PYG{l+s+s2}{\PYGZsq{}}\PYG{l+s+s2}{COLO}\PYG{l+s+s2}{\PYGZsq{}}\PYG{l+s+s2}{, b}\PYG{l+s+s2}{\PYGZsq{}}\PYG{l+s+s2}{044 }\PYG{l+s+s2}{\PYGZsq{}}\PYG{l+s+s2}{)}\PYG{l+s+s2}{\PYGZdq{}}\PYG{p}{:} \PYG{l+m+mi}{6}\PYG{p}{,}
 \PYG{l+s+s2}{\PYGZdq{}}\PYG{l+s+s2}{(34, b}\PYG{l+s+s2}{\PYGZsq{}}\PYG{l+s+s2}{PRES}\PYG{l+s+s2}{\PYGZsq{}}\PYG{l+s+s2}{, b}\PYG{l+s+s2}{\PYGZsq{}}\PYG{l+s+s2}{COLO}\PYG{l+s+s2}{\PYGZsq{}}\PYG{l+s+s2}{, b}\PYG{l+s+s2}{\PYGZsq{}}\PYG{l+s+s2}{104 }\PYG{l+s+s2}{\PYGZsq{}}\PYG{l+s+s2}{)}\PYG{l+s+s2}{\PYGZdq{}}\PYG{p}{:} \PYG{l+m+mi}{3}\PYG{p}{,}
 \PYG{l+s+s2}{\PYGZdq{}}\PYG{l+s+s2}{(34, b}\PYG{l+s+s2}{\PYGZsq{}}\PYG{l+s+s2}{PRES}\PYG{l+s+s2}{\PYGZsq{}}\PYG{l+s+s2}{, b}\PYG{l+s+s2}{\PYGZsq{}}\PYG{l+s+s2}{COLO}\PYG{l+s+s2}{\PYGZsq{}}\PYG{l+s+s2}{, b}\PYG{l+s+s2}{\PYGZsq{}}\PYG{l+s+s2}{134 }\PYG{l+s+s2}{\PYGZsq{}}\PYG{l+s+s2}{)}\PYG{l+s+s2}{\PYGZdq{}}\PYG{p}{:} \PYG{l+m+mi}{14}\PYG{p}{,}
 \PYG{l+s+s2}{\PYGZdq{}}\PYG{l+s+s2}{(34, b}\PYG{l+s+s2}{\PYGZsq{}}\PYG{l+s+s2}{PRES}\PYG{l+s+s2}{\PYGZsq{}}\PYG{l+s+s2}{, b}\PYG{l+s+s2}{\PYGZsq{}}\PYG{l+s+s2}{COLO}\PYG{l+s+s2}{\PYGZsq{}}\PYG{l+s+s2}{, b}\PYG{l+s+s2}{\PYGZsq{}}\PYG{l+s+s2}{203 }\PYG{l+s+s2}{\PYGZsq{}}\PYG{l+s+s2}{)}\PYG{l+s+s2}{\PYGZdq{}}\PYG{p}{:} \PYG{l+m+mi}{3}\PYG{p}{,}
 \PYG{l+s+s2}{\PYGZdq{}}\PYG{l+s+s2}{(34, b}\PYG{l+s+s2}{\PYGZsq{}}\PYG{l+s+s2}{PRES}\PYG{l+s+s2}{\PYGZsq{}}\PYG{l+s+s2}{, b}\PYG{l+s+s2}{\PYGZsq{}}\PYG{l+s+s2}{COLO}\PYG{l+s+s2}{\PYGZsq{}}\PYG{l+s+s2}{, b}\PYG{l+s+s2}{\PYGZsq{}}\PYG{l+s+s2}{221 }\PYG{l+s+s2}{\PYGZsq{}}\PYG{l+s+s2}{)}\PYG{l+s+s2}{\PYGZdq{}}\PYG{p}{:} \PYG{l+m+mi}{3}\PYG{p}{,}
 \PYG{l+s+s2}{\PYGZdq{}}\PYG{l+s+s2}{(34, b}\PYG{l+s+s2}{\PYGZsq{}}\PYG{l+s+s2}{PRES}\PYG{l+s+s2}{\PYGZsq{}}\PYG{l+s+s2}{, b}\PYG{l+s+s2}{\PYGZsq{}}\PYG{l+s+s2}{COLO}\PYG{l+s+s2}{\PYGZsq{}}\PYG{l+s+s2}{, b}\PYG{l+s+s2}{\PYGZsq{}}\PYG{l+s+s2}{312 }\PYG{l+s+s2}{\PYGZsq{}}\PYG{l+s+s2}{)}\PYG{l+s+s2}{\PYGZdq{}}\PYG{p}{:} \PYG{l+m+mi}{10}\PYG{p}{,}
 \PYG{l+s+s2}{\PYGZdq{}}\PYG{l+s+s2}{(34, b}\PYG{l+s+s2}{\PYGZsq{}}\PYG{l+s+s2}{PRES}\PYG{l+s+s2}{\PYGZsq{}}\PYG{l+s+s2}{, b}\PYG{l+s+s2}{\PYGZsq{}}\PYG{l+s+s2}{COLO}\PYG{l+s+s2}{\PYGZsq{}}\PYG{l+s+s2}{, b}\PYG{l+s+s2}{\PYGZsq{}}\PYG{l+s+s2}{400 }\PYG{l+s+s2}{\PYGZsq{}}\PYG{l+s+s2}{)}\PYG{l+s+s2}{\PYGZdq{}}\PYG{p}{:} \PYG{l+m+mi}{44}\PYG{p}{,}
 \PYG{l+s+s2}{\PYGZdq{}}\PYG{l+s+s2}{(34, b}\PYG{l+s+s2}{\PYGZsq{}}\PYG{l+s+s2}{PRES}\PYG{l+s+s2}{\PYGZsq{}}\PYG{l+s+s2}{, b}\PYG{l+s+s2}{\PYGZsq{}}\PYG{l+s+s2}{COLO}\PYG{l+s+s2}{\PYGZsq{}}\PYG{l+s+s2}{, b}\PYG{l+s+s2}{\PYGZsq{}}\PYG{l+s+s2}{404 }\PYG{l+s+s2}{\PYGZsq{}}\PYG{l+s+s2}{)}\PYG{l+s+s2}{\PYGZdq{}}\PYG{p}{:} \PYG{l+m+mi}{6}\PYG{p}{,}
 \PYG{l+s+s2}{\PYGZdq{}}\PYG{l+s+s2}{(34, b}\PYG{l+s+s2}{\PYGZsq{}}\PYG{l+s+s2}{PRES}\PYG{l+s+s2}{\PYGZsq{}}\PYG{l+s+s2}{, b}\PYG{l+s+s2}{\PYGZsq{}}\PYG{l+s+s2}{COLO}\PYG{l+s+s2}{\PYGZsq{}}\PYG{l+s+s2}{, b}\PYG{l+s+s2}{\PYGZsq{}}\PYG{l+s+s2}{420 }\PYG{l+s+s2}{\PYGZsq{}}\PYG{l+s+s2}{)}\PYG{l+s+s2}{\PYGZdq{}}\PYG{p}{:} \PYG{l+m+mi}{9}\PYG{p}{,}
 \PYG{l+s+s2}{\PYGZdq{}}\PYG{l+s+s2}{(34, b}\PYG{l+s+s2}{\PYGZsq{}}\PYG{l+s+s2}{PRES}\PYG{l+s+s2}{\PYGZsq{}}\PYG{l+s+s2}{, b}\PYG{l+s+s2}{\PYGZsq{}}\PYG{l+s+s2}{COLO}\PYG{l+s+s2}{\PYGZsq{}}\PYG{l+s+s2}{, b}\PYG{l+s+s2}{\PYGZsq{}}\PYG{l+s+s2}{430 }\PYG{l+s+s2}{\PYGZsq{}}\PYG{l+s+s2}{)}\PYG{l+s+s2}{\PYGZdq{}}\PYG{p}{:} \PYG{l+m+mi}{17}\PYG{p}{,}
 \PYG{l+s+s2}{\PYGZdq{}}\PYG{l+s+s2}{(34, b}\PYG{l+s+s2}{\PYGZsq{}}\PYG{l+s+s2}{PRES}\PYG{l+s+s2}{\PYGZsq{}}\PYG{l+s+s2}{, b}\PYG{l+s+s2}{\PYGZsq{}}\PYG{l+s+s2}{COLO}\PYG{l+s+s2}{\PYGZsq{}}\PYG{l+s+s2}{, b}\PYG{l+s+s2}{\PYGZsq{}}\PYG{l+s+s2}{AQUA}\PYG{l+s+s2}{\PYGZsq{}}\PYG{l+s+s2}{)}\PYG{l+s+s2}{\PYGZdq{}}\PYG{p}{:} \PYG{l+m+mi}{3}\PYG{p}{,}
 \PYG{l+s+s2}{\PYGZdq{}}\PYG{l+s+s2}{(34, b}\PYG{l+s+s2}{\PYGZsq{}}\PYG{l+s+s2}{PRES}\PYG{l+s+s2}{\PYGZsq{}}\PYG{l+s+s2}{, b}\PYG{l+s+s2}{\PYGZsq{}}\PYG{l+s+s2}{COLO}\PYG{l+s+s2}{\PYGZsq{}}\PYG{l+s+s2}{, b}\PYG{l+s+s2}{\PYGZsq{}}\PYG{l+s+s2}{BLAC}\PYG{l+s+s2}{\PYGZsq{}}\PYG{l+s+s2}{)}\PYG{l+s+s2}{\PYGZdq{}}\PYG{p}{:} \PYG{l+m+mi}{1260}\PYG{p}{,}
 \PYG{l+s+s2}{\PYGZdq{}}\PYG{l+s+s2}{(34, b}\PYG{l+s+s2}{\PYGZsq{}}\PYG{l+s+s2}{PRES}\PYG{l+s+s2}{\PYGZsq{}}\PYG{l+s+s2}{, b}\PYG{l+s+s2}{\PYGZsq{}}\PYG{l+s+s2}{COLO}\PYG{l+s+s2}{\PYGZsq{}}\PYG{l+s+s2}{, b}\PYG{l+s+s2}{\PYGZsq{}}\PYG{l+s+s2}{BLUE}\PYG{l+s+s2}{\PYGZsq{}}\PYG{l+s+s2}{)}\PYG{l+s+s2}{\PYGZdq{}}\PYG{p}{:} \PYG{l+m+mi}{8}\PYG{p}{,}
 \PYG{l+s+s2}{\PYGZdq{}}\PYG{l+s+s2}{(34, b}\PYG{l+s+s2}{\PYGZsq{}}\PYG{l+s+s2}{PRES}\PYG{l+s+s2}{\PYGZsq{}}\PYG{l+s+s2}{, b}\PYG{l+s+s2}{\PYGZsq{}}\PYG{l+s+s2}{COLO}\PYG{l+s+s2}{\PYGZsq{}}\PYG{l+s+s2}{, b}\PYG{l+s+s2}{\PYGZsq{}}\PYG{l+s+s2}{GREE}\PYG{l+s+s2}{\PYGZsq{}}\PYG{l+s+s2}{)}\PYG{l+s+s2}{\PYGZdq{}}\PYG{p}{:} \PYG{l+m+mi}{37}\PYG{p}{,}
 \PYG{l+s+s2}{\PYGZdq{}}\PYG{l+s+s2}{(34, b}\PYG{l+s+s2}{\PYGZsq{}}\PYG{l+s+s2}{PRES}\PYG{l+s+s2}{\PYGZsq{}}\PYG{l+s+s2}{, b}\PYG{l+s+s2}{\PYGZsq{}}\PYG{l+s+s2}{COLO}\PYG{l+s+s2}{\PYGZsq{}}\PYG{l+s+s2}{, b}\PYG{l+s+s2}{\PYGZsq{}}\PYG{l+s+s2}{RED }\PYG{l+s+s2}{\PYGZsq{}}\PYG{l+s+s2}{)}\PYG{l+s+s2}{\PYGZdq{}}\PYG{p}{:} \PYG{l+m+mi}{12}\PYG{p}{\PYGZcb{}}
\PYG{o}{==}\PYG{o}{==}\PYG{o}{==}\PYG{o}{==}\PYG{o}{==}\PYG{o}{==}\PYG{o}{==}\PYG{o}{==}\PYG{o}{==}\PYG{o}{==}\PYG{o}{==} \PYG{n}{Count} \PYG{n}{of} \PYG{n+nb}{all} \PYG{n}{table} \PYG{n}{entries} \PYG{n}{END} \PYG{o}{==}\PYG{o}{==}\PYG{o}{==}\PYG{o}{==}\PYG{o}{==}\PYG{o}{==}\PYG{o}{==}\PYG{o}{==}\PYG{o}{==}\PYG{o}{==}\PYG{o}{=}
\end{sphinxVerbatim}

What are these numbers, RGB Base 5?

Track names F1 to F4 and FD.

This seems to be the nomenclature for four track plots. For example:

\begin{sphinxVerbatim}[commandchars=\\\{\}]
\PYG{n}{Table} \PYG{n}{record} \PYG{p}{(}\PYG{n+nb}{type} \PYG{l+m+mi}{34}\PYG{p}{)} \PYG{n+nb}{type}\PYG{p}{:} \PYG{n}{FILM}

\PYG{n}{MNEM}  \PYG{n}{GCOD}  \PYG{n}{GDEC}  \PYG{n}{DEST}  \PYG{n}{DSCA}
\PYG{o}{\PYGZhy{}}\PYG{o}{\PYGZhy{}}\PYG{o}{\PYGZhy{}}\PYG{o}{\PYGZhy{}}\PYG{o}{\PYGZhy{}}\PYG{o}{\PYGZhy{}}\PYG{o}{\PYGZhy{}}\PYG{o}{\PYGZhy{}}\PYG{o}{\PYGZhy{}}\PYG{o}{\PYGZhy{}}\PYG{o}{\PYGZhy{}}\PYG{o}{\PYGZhy{}}\PYG{o}{\PYGZhy{}}\PYG{o}{\PYGZhy{}}\PYG{o}{\PYGZhy{}}\PYG{o}{\PYGZhy{}}\PYG{o}{\PYGZhy{}}\PYG{o}{\PYGZhy{}}\PYG{o}{\PYGZhy{}}\PYG{o}{\PYGZhy{}}\PYG{o}{\PYGZhy{}}\PYG{o}{\PYGZhy{}}\PYG{o}{\PYGZhy{}}\PYG{o}{\PYGZhy{}}\PYG{o}{\PYGZhy{}}\PYG{o}{\PYGZhy{}}\PYG{o}{\PYGZhy{}}\PYG{o}{\PYGZhy{}}\PYG{o}{\PYGZhy{}}
                        
\PYG{l+m+mi}{8}     \PYG{n}{EB0}   \PYG{o}{\PYGZhy{}}\PYG{o}{\PYGZhy{}}\PYG{o}{\PYGZhy{}}   \PYG{n}{PF8}   \PYG{n}{D200}
\PYG{n}{A}     \PYG{n}{LLLL}  \PYG{l+m+mi}{1111}  \PYG{n}{PFA}   \PYG{n}{DM}  
\PYG{n}{E}     \PYG{n}{E4E}   \PYG{o}{\PYGZhy{}}\PYG{l+m+mi}{4}\PYG{o}{\PYGZhy{}}   \PYG{n}{PFE}   \PYG{n}{D200}
\PYG{n}{K}     \PYG{n}{E4E}   \PYG{o}{\PYGZhy{}}\PYG{l+m+mi}{4}\PYG{o}{\PYGZhy{}}   \PYG{n}{PFK}   \PYG{n}{D500}

\PYG{n}{Table} \PYG{n}{record} \PYG{p}{(}\PYG{n+nb}{type} \PYG{l+m+mi}{34}\PYG{p}{)} \PYG{n+nb}{type}\PYG{p}{:} \PYG{n}{PRES}

\PYG{n}{MNEM}  \PYG{n}{OUTP}  \PYG{n}{STAT}  \PYG{n}{TRAC}  \PYG{n}{CODI}  \PYG{n}{DEST}  \PYG{n}{MODE}      \PYG{n}{FILT}          \PYG{n}{LEDG}          \PYG{n}{REDG}      \PYG{n}{COLO}
\PYG{o}{\PYGZhy{}}\PYG{o}{\PYGZhy{}}\PYG{o}{\PYGZhy{}}\PYG{o}{\PYGZhy{}}\PYG{o}{\PYGZhy{}}\PYG{o}{\PYGZhy{}}\PYG{o}{\PYGZhy{}}\PYG{o}{\PYGZhy{}}\PYG{o}{\PYGZhy{}}\PYG{o}{\PYGZhy{}}\PYG{o}{\PYGZhy{}}\PYG{o}{\PYGZhy{}}\PYG{o}{\PYGZhy{}}\PYG{o}{\PYGZhy{}}\PYG{o}{\PYGZhy{}}\PYG{o}{\PYGZhy{}}\PYG{o}{\PYGZhy{}}\PYG{o}{\PYGZhy{}}\PYG{o}{\PYGZhy{}}\PYG{o}{\PYGZhy{}}\PYG{o}{\PYGZhy{}}\PYG{o}{\PYGZhy{}}\PYG{o}{\PYGZhy{}}\PYG{o}{\PYGZhy{}}\PYG{o}{\PYGZhy{}}\PYG{o}{\PYGZhy{}}\PYG{o}{\PYGZhy{}}\PYG{o}{\PYGZhy{}}\PYG{o}{\PYGZhy{}}\PYG{o}{\PYGZhy{}}\PYG{o}{\PYGZhy{}}\PYG{o}{\PYGZhy{}}\PYG{o}{\PYGZhy{}}\PYG{o}{\PYGZhy{}}\PYG{o}{\PYGZhy{}}\PYG{o}{\PYGZhy{}}\PYG{o}{\PYGZhy{}}\PYG{o}{\PYGZhy{}}\PYG{o}{\PYGZhy{}}\PYG{o}{\PYGZhy{}}\PYG{o}{\PYGZhy{}}\PYG{o}{\PYGZhy{}}\PYG{o}{\PYGZhy{}}\PYG{o}{\PYGZhy{}}\PYG{o}{\PYGZhy{}}\PYG{o}{\PYGZhy{}}\PYG{o}{\PYGZhy{}}\PYG{o}{\PYGZhy{}}\PYG{o}{\PYGZhy{}}\PYG{o}{\PYGZhy{}}\PYG{o}{\PYGZhy{}}\PYG{o}{\PYGZhy{}}\PYG{o}{\PYGZhy{}}\PYG{o}{\PYGZhy{}}\PYG{o}{\PYGZhy{}}\PYG{o}{\PYGZhy{}}\PYG{o}{\PYGZhy{}}\PYG{o}{\PYGZhy{}}\PYG{o}{\PYGZhy{}}\PYG{o}{\PYGZhy{}}\PYG{o}{\PYGZhy{}}\PYG{o}{\PYGZhy{}}\PYG{o}{\PYGZhy{}}\PYG{o}{\PYGZhy{}}\PYG{o}{\PYGZhy{}}\PYG{o}{\PYGZhy{}}\PYG{o}{\PYGZhy{}}\PYG{o}{\PYGZhy{}}\PYG{o}{\PYGZhy{}}\PYG{o}{\PYGZhy{}}\PYG{o}{\PYGZhy{}}\PYG{o}{\PYGZhy{}}\PYG{o}{\PYGZhy{}}\PYG{o}{\PYGZhy{}}\PYG{o}{\PYGZhy{}}\PYG{o}{\PYGZhy{}}\PYG{o}{\PYGZhy{}}\PYG{o}{\PYGZhy{}}\PYG{o}{\PYGZhy{}}\PYG{o}{\PYGZhy{}}\PYG{o}{\PYGZhy{}}\PYG{o}{\PYGZhy{}}\PYG{o}{\PYGZhy{}}\PYG{o}{\PYGZhy{}}\PYG{o}{\PYGZhy{}}\PYG{o}{\PYGZhy{}}\PYG{o}{\PYGZhy{}}\PYG{o}{\PYGZhy{}}\PYG{o}{\PYGZhy{}}                                                                                    
\PYG{o}{.}\PYG{o}{.}\PYG{o}{.}
\PYG{n}{AX1A}  \PYG{n}{AX}    \PYG{n}{ALLO}  \PYG{n}{F2}    \PYG{n}{LLIN}  \PYG{n}{A}     \PYG{n}{SHIF}      \PYG{l+m+mf}{0.500000}      \PYG{o}{\PYGZhy{}}\PYG{l+m+mf}{9.00000}       \PYG{l+m+mf}{9.00000}  \PYG{l+m+mi}{400} 
\PYG{n}{AY1A}  \PYG{n}{AY}    \PYG{n}{ALLO}  \PYG{n}{F2}    \PYG{n}{LSPO}  \PYG{n}{A}     \PYG{n}{SHIF}      \PYG{l+m+mf}{0.500000}      \PYG{o}{\PYGZhy{}}\PYG{l+m+mf}{9.00000}       \PYG{l+m+mf}{9.00000}  \PYG{l+m+mi}{400} 
\PYG{n}{AZ1A}  \PYG{n}{AZ}    \PYG{n}{ALLO}  \PYG{n}{F2}    \PYG{n}{LGAP}  \PYG{n}{A}     \PYG{n}{NB}        \PYG{l+m+mf}{0.500000}      \PYG{o}{\PYGZhy{}}\PYG{l+m+mf}{9.00000}       \PYG{l+m+mf}{9.00000}  \PYG{l+m+mi}{400} 
\PYG{n}{AN1A}  \PYG{n}{ANOR}  \PYG{n}{ALLO}  \PYG{n}{F4}    \PYG{n}{HDAS}  \PYG{n}{A}     \PYG{n}{NB}        \PYG{l+m+mf}{0.500000}       \PYG{l+m+mf}{9.00000}       \PYG{l+m+mf}{11.0000}  \PYG{l+m+mi}{420} 
\PYG{n}{CS1A}  \PYG{n}{CS}    \PYG{n}{DISA}  \PYG{n}{F1}    \PYG{n}{LDAS}  \PYG{n}{A}     \PYG{n}{NB}        \PYG{l+m+mf}{0.500000}       \PYG{l+m+mf}{0.00000}       \PYG{l+m+mf}{150000.}  \PYG{l+m+mi}{000} 
\PYG{n}{FX1A}  \PYG{n}{FX}    \PYG{n}{ALLO}  \PYG{n}{F3}    \PYG{n}{LLIN}  \PYG{n}{A}     \PYG{n}{NB}        \PYG{l+m+mf}{0.500000}     \PYG{o}{\PYGZhy{}}\PYG{l+m+mf}{0.700000}      \PYG{l+m+mf}{0.700000}  \PYG{l+m+mi}{003} 
\PYG{n}{FY1A}  \PYG{n}{FY}    \PYG{n}{ALLO}  \PYG{n}{F3}    \PYG{n}{LGAP}  \PYG{n}{A}     \PYG{n}{NB}        \PYG{l+m+mf}{0.500000}     \PYG{o}{\PYGZhy{}}\PYG{l+m+mf}{0.700000}      \PYG{l+m+mf}{0.700000}  \PYG{l+m+mi}{003} 
\PYG{n}{FZ1A}  \PYG{n}{FZ}    \PYG{n}{ALLO}  \PYG{n}{F3}    \PYG{n}{LSPO}  \PYG{n}{A}     \PYG{n}{NB}        \PYG{l+m+mf}{0.500000}     \PYG{o}{\PYGZhy{}}\PYG{l+m+mf}{0.700000}      \PYG{l+m+mf}{0.700000}  \PYG{l+m+mi}{003} 
\PYG{n}{FN1A}  \PYG{n}{FNOR}  \PYG{n}{ALLO}  \PYG{n}{F4}    \PYG{n}{LLIN}  \PYG{n}{A}     \PYG{n}{NB}        \PYG{l+m+mf}{0.500000}      \PYG{l+m+mf}{0.200000}      \PYG{l+m+mf}{0.700000}  \PYG{l+m+mi}{420} 
\PYG{n}{FI1A}  \PYG{n}{FINC}  \PYG{n}{ALLO}  \PYG{n}{F4}    \PYG{n}{LGAP}  \PYG{n}{A}     \PYG{n}{NB}         \PYG{l+m+mf}{1.00000}       \PYG{l+m+mf}{0.00000}       \PYG{l+m+mf}{90.0000}  \PYG{l+m+mi}{420} 
\PYG{n}{GA1A}  \PYG{n}{GADZ}  \PYG{n}{DISA}  \PYG{n}{F1}    \PYG{n}{LLIN}  \PYG{n}{A}     \PYG{n}{NB}         \PYG{l+m+mf}{1.00000}      \PYG{o}{\PYGZhy{}}\PYG{l+m+mf}{1.00000}       \PYG{l+m+mf}{1.00000}  \PYG{l+m+mi}{020} 
\PYG{n}{GP1A}  \PYG{n}{GPV}   \PYG{n}{DISA}  \PYG{n}{F1}    \PYG{n}{LGAP}  \PYG{n}{A}     \PYG{n}{NB}         \PYG{l+m+mf}{1.00000}       \PYG{l+m+mf}{14.0000}       \PYG{l+m+mf}{16.0000}  \PYG{l+m+mi}{020} 
\PYG{n}{GN1A}  \PYG{n}{GNV}   \PYG{n}{DISA}  \PYG{n}{F1}    \PYG{n}{LDAS}  \PYG{n}{A}     \PYG{n}{NB}         \PYG{l+m+mf}{1.00000}      \PYG{o}{\PYGZhy{}}\PYG{l+m+mf}{16.0000}      \PYG{o}{\PYGZhy{}}\PYG{l+m+mf}{14.0000}  \PYG{l+m+mi}{020} 
\PYG{n}{GM1A}  \PYG{n}{GMT}   \PYG{n}{DISA}  \PYG{n}{F3}    \PYG{n}{HDAS}  \PYG{n}{A}     \PYG{n}{SHIF}       \PYG{l+m+mf}{1.00000}       \PYG{l+m+mf}{80.0000}       \PYG{l+m+mf}{130.000}  \PYG{l+m+mi}{020} 
\PYG{n}{GA2A}  \PYG{n}{GAT}   \PYG{n}{DISA}  \PYG{n}{F2}    \PYG{n}{HDAS}  \PYG{n}{A}     \PYG{n}{SHIF}       \PYG{l+m+mf}{1.00000}       \PYG{l+m+mf}{80.0000}       \PYG{l+m+mf}{130.000}  \PYG{l+m+mi}{020} 
\PYG{n}{SI1A}  \PYG{n}{SILO}  \PYG{n}{DISA}  \PYG{n}{FD}    \PYG{n}{HLIN}  \PYG{n}{A}     \PYG{n}{NB}         \PYG{l+m+mf}{1.00000}       \PYG{l+m+mf}{0.00000}       \PYG{l+m+mf}{20.0000}  \PYG{l+m+mi}{000} 
\PYG{n}{ST1A}  \PYG{n}{STIT}  \PYG{n}{DISA}  \PYG{n}{FD}    \PYG{n}{LLIN}  \PYG{n}{A}     \PYG{n}{NB}         \PYG{l+m+mf}{1.00000}       \PYG{l+m+mf}{0.00000}       \PYG{l+m+mf}{20.0000}  \PYG{l+m+mi}{000} 
\PYG{n}{ST2A}  \PYG{n}{STIA}  \PYG{n}{ALLO}  \PYG{n}{FD}    \PYG{n}{LLIN}  \PYG{n}{A}     \PYG{n}{NB}         \PYG{l+m+mf}{1.00000}       \PYG{l+m+mf}{0.00000}       \PYG{l+m+mf}{20.0000}  \PYG{l+m+mi}{000} 
\PYG{n}{TE1A}  \PYG{n}{TENS}  \PYG{n}{DISA}  \PYG{n}{FD}    \PYG{n}{LDAS}  \PYG{n}{A}     \PYG{n}{WRAP}      \PYG{l+m+mf}{0.500000}       \PYG{l+m+mf}{2000.00}       \PYG{l+m+mf}{7000.00}  \PYG{l+m+mi}{000} 
\PYG{o}{.}\PYG{o}{.}\PYG{o}{.}
\end{sphinxVerbatim}
\index{ExceptionPRESCfg}

\begin{fulllineitems}
\phantomsection\label{\detokenize{ref/util/plot/PRESCfg:TotalDepth.util.plot.PRESCfg.ExceptionPRESCfg}}\pysigline{\sphinxbfcode{exception }\sphinxcode{TotalDepth.util.plot.PRESCfg.}\sphinxbfcode{ExceptionPRESCfg}}
Specialisation of exception for this module.

\end{fulllineitems}

\index{ExceptionLineTransBase}

\begin{fulllineitems}
\phantomsection\label{\detokenize{ref/util/plot/PRESCfg:TotalDepth.util.plot.PRESCfg.ExceptionLineTransBase}}\pysigline{\sphinxbfcode{exception }\sphinxcode{TotalDepth.util.plot.PRESCfg.}\sphinxbfcode{ExceptionLineTransBase}}
Specialisation of exception for LineTransBase and descendants.

\end{fulllineitems}

\index{ExceptionLineTransBaseMath}

\begin{fulllineitems}
\phantomsection\label{\detokenize{ref/util/plot/PRESCfg:TotalDepth.util.plot.PRESCfg.ExceptionLineTransBaseMath}}\pysigline{\sphinxbfcode{exception }\sphinxcode{TotalDepth.util.plot.PRESCfg.}\sphinxbfcode{ExceptionLineTransBaseMath}}
For LineTransBase and descendants where math errors occur.

\end{fulllineitems}

\index{ExceptionPRESCfgLISRead}

\begin{fulllineitems}
\phantomsection\label{\detokenize{ref/util/plot/PRESCfg:TotalDepth.util.plot.PRESCfg.ExceptionPRESCfgLISRead}}\pysigline{\sphinxbfcode{exception }\sphinxcode{TotalDepth.util.plot.PRESCfg.}\sphinxbfcode{ExceptionPRESCfgLISRead}}
Specialisation of exception for this module.

\end{fulllineitems}

\index{ExceptionPresCfg}

\begin{fulllineitems}
\phantomsection\label{\detokenize{ref/util/plot/PRESCfg:TotalDepth.util.plot.PRESCfg.ExceptionPresCfg}}\pysigline{\sphinxbfcode{exception }\sphinxcode{TotalDepth.util.plot.PRESCfg.}\sphinxbfcode{ExceptionPresCfg}}
Specialisation of exception for PresCfg.

\end{fulllineitems}

\index{ExceptionCurveCfg}

\begin{fulllineitems}
\phantomsection\label{\detokenize{ref/util/plot/PRESCfg:TotalDepth.util.plot.PRESCfg.ExceptionCurveCfg}}\pysigline{\sphinxbfcode{exception }\sphinxcode{TotalDepth.util.plot.PRESCfg.}\sphinxbfcode{ExceptionCurveCfg}}
Specialisation of exception for CurveCfg.

\end{fulllineitems}

\index{ExceptionCurveCfgCtor}

\begin{fulllineitems}
\phantomsection\label{\detokenize{ref/util/plot/PRESCfg:TotalDepth.util.plot.PRESCfg.ExceptionCurveCfgCtor}}\pysigline{\sphinxbfcode{exception }\sphinxcode{TotalDepth.util.plot.PRESCfg.}\sphinxbfcode{ExceptionCurveCfgCtor}}
Construction exception when making a CurveCfg object or descendant.

\end{fulllineitems}

\index{ExceptionCurveCfgLISRead}

\begin{fulllineitems}
\phantomsection\label{\detokenize{ref/util/plot/PRESCfg:TotalDepth.util.plot.PRESCfg.ExceptionCurveCfgLISRead}}\pysigline{\sphinxbfcode{exception }\sphinxcode{TotalDepth.util.plot.PRESCfg.}\sphinxbfcode{ExceptionCurveCfgLISRead}}
Specialisation of exception for CurveCfgLISRead and its travails.

\end{fulllineitems}

\index{DEFAULT\_LLINE\_WIDTH\_PX (in module TotalDepth.util.plot.PRESCfg)}

\begin{fulllineitems}
\phantomsection\label{\detokenize{ref/util/plot/PRESCfg:TotalDepth.util.plot.PRESCfg.DEFAULT_LLINE_WIDTH_PX}}\pysigline{\sphinxcode{TotalDepth.util.plot.PRESCfg.}\sphinxbfcode{DEFAULT\_LLINE\_WIDTH\_PX}\sphinxbfcode{ = 0.5}}
Deafult light line width in pixels

\end{fulllineitems}

\index{DEFAULT\_HLINE\_WIDTH\_PX (in module TotalDepth.util.plot.PRESCfg)}

\begin{fulllineitems}
\phantomsection\label{\detokenize{ref/util/plot/PRESCfg:TotalDepth.util.plot.PRESCfg.DEFAULT_HLINE_WIDTH_PX}}\pysigline{\sphinxcode{TotalDepth.util.plot.PRESCfg.}\sphinxbfcode{DEFAULT\_HLINE\_WIDTH\_PX}\sphinxbfcode{ = 1.5}}
Deafult heavy line width in pixels

\end{fulllineitems}

\index{LIS\_CODI\_MAP (in module TotalDepth.util.plot.PRESCfg)}

\begin{fulllineitems}
\phantomsection\label{\detokenize{ref/util/plot/PRESCfg:TotalDepth.util.plot.PRESCfg.LIS_CODI_MAP}}\pysigline{\sphinxcode{TotalDepth.util.plot.PRESCfg.}\sphinxbfcode{LIS\_CODI\_MAP}\sphinxbfcode{ = \{None: Stroke(width=0.5, colour='black', coding=None, opacity=1.0), b'LLIN': Stroke(width=0.5, colour='black', coding=None, opacity=1.0), b'LSPO': Stroke(width=0.5, colour='black', coding=(2, 2), opacity=1.0), b'LDAS': Stroke(width=0.5, colour='black', coding=(4, 4), opacity=1.0), b'LGAP': Stroke(width=0.5, colour='black', coding=(6, 2), opacity=1.0), b'HLIN': Stroke(width=1.5, colour='black', coding=None, opacity=1.0), b'HSPO': Stroke(width=1.5, colour='black', coding=(2, 2), opacity=1.0), b'HDAS': Stroke(width=1.5, colour='black', coding=(4, 4), opacity=1.0), b'HGAP': Stroke(width=1.5, colour='black', coding=(6, 2), opacity=1.0), b'SFLN': Stroke(width=0.0625, colour='black', coding=None, opacity=1.0), b'SFDA': Stroke(width=0.0625, colour='black', coding=(4, 4), opacity=1.0)\}}}
Maps LIS CODI mnemonics to a Stroke object:
If either value is None an SVG attribute is not needed i.e. default SVG behaviour

\end{fulllineitems}

\index{LIS\_COLO\_MAP (in module TotalDepth.util.plot.PRESCfg)}

\begin{fulllineitems}
\phantomsection\label{\detokenize{ref/util/plot/PRESCfg:TotalDepth.util.plot.PRESCfg.LIS_COLO_MAP}}\pysigline{\sphinxcode{TotalDepth.util.plot.PRESCfg.}\sphinxbfcode{LIS\_COLO\_MAP}\sphinxbfcode{ = \{Mnem(b'AQUA'): 'aqua', Mnem(b'BLAC'): 'black', Mnem(b'BLUE'): 'blue', Mnem(b'GREE'): 'green', Mnem(b'RED\textbackslash{}x00'): 'red', Mnem(b'YELL'): 'yellow', Mnem(b'CYAN'): 'cyan', Mnem(b'MAGE'): 'magenta', Mnem(b'GREY'): 'grey'\}}}
Maps LIS COLO mnemonics to CSS/SVG style colour specifications.

\end{fulllineitems}

\index{lisColo() (in module TotalDepth.util.plot.PRESCfg)}

\begin{fulllineitems}
\phantomsection\label{\detokenize{ref/util/plot/PRESCfg:TotalDepth.util.plot.PRESCfg.lisColo}}\pysiglinewithargsret{\sphinxcode{TotalDepth.util.plot.PRESCfg.}\sphinxbfcode{lisColo}}{\emph{theMnem}}{}
Returns a SVG colour as a string from a Mnem or None on failure.

\end{fulllineitems}

\index{coloStroke() (in module TotalDepth.util.plot.PRESCfg)}

\begin{fulllineitems}
\phantomsection\label{\detokenize{ref/util/plot/PRESCfg:TotalDepth.util.plot.PRESCfg.coloStroke}}\pysiglinewithargsret{\sphinxcode{TotalDepth.util.plot.PRESCfg.}\sphinxbfcode{coloStroke}}{\emph{theSt}, \emph{theCm}}{}
Takes a Stroke object and returns a new stroke object with the colour
replaced with the Color Mnem looked up, or determined from, the LIS\_COLO\_MAP.

\end{fulllineitems}

\index{BACKUP\_NONE (in module TotalDepth.util.plot.PRESCfg)}

\begin{fulllineitems}
\phantomsection\label{\detokenize{ref/util/plot/PRESCfg:TotalDepth.util.plot.PRESCfg.BACKUP_NONE}}\pysigline{\sphinxcode{TotalDepth.util.plot.PRESCfg.}\sphinxbfcode{BACKUP\_NONE}\sphinxbfcode{ = (1, -1)}}
No backup

\end{fulllineitems}

\index{BACKUP\_ALL (in module TotalDepth.util.plot.PRESCfg)}

\begin{fulllineitems}
\phantomsection\label{\detokenize{ref/util/plot/PRESCfg:TotalDepth.util.plot.PRESCfg.BACKUP_ALL}}\pysigline{\sphinxcode{TotalDepth.util.plot.PRESCfg.}\sphinxbfcode{BACKUP\_ALL}\sphinxbfcode{ = (0, 0)}}
Every backup i.e. ‘wrap’ Note: Plot.py has a way of limiting ludicrous
backup lines to a sensible number; say 8

\end{fulllineitems}

\index{BACKUP\_ONCE (in module TotalDepth.util.plot.PRESCfg)}

\begin{fulllineitems}
\phantomsection\label{\detokenize{ref/util/plot/PRESCfg:TotalDepth.util.plot.PRESCfg.BACKUP_ONCE}}\pysigline{\sphinxcode{TotalDepth.util.plot.PRESCfg.}\sphinxbfcode{BACKUP\_ONCE}\sphinxbfcode{ = (-1, 1)}}
Single backup to left or right

\end{fulllineitems}

\index{BACKUP\_TWICE (in module TotalDepth.util.plot.PRESCfg)}

\begin{fulllineitems}
\phantomsection\label{\detokenize{ref/util/plot/PRESCfg:TotalDepth.util.plot.PRESCfg.BACKUP_TWICE}}\pysigline{\sphinxcode{TotalDepth.util.plot.PRESCfg.}\sphinxbfcode{BACKUP\_TWICE}\sphinxbfcode{ = (-2, 2)}}
Two backups to left or right

\end{fulllineitems}

\index{BACKUP\_LEFT (in module TotalDepth.util.plot.PRESCfg)}

\begin{fulllineitems}
\phantomsection\label{\detokenize{ref/util/plot/PRESCfg:TotalDepth.util.plot.PRESCfg.BACKUP_LEFT}}\pysigline{\sphinxcode{TotalDepth.util.plot.PRESCfg.}\sphinxbfcode{BACKUP\_LEFT}\sphinxbfcode{ = (0, -1)}}
Single backup to left only

\end{fulllineitems}

\index{BACKUP\_RIGHT (in module TotalDepth.util.plot.PRESCfg)}

\begin{fulllineitems}
\phantomsection\label{\detokenize{ref/util/plot/PRESCfg:TotalDepth.util.plot.PRESCfg.BACKUP_RIGHT}}\pysigline{\sphinxcode{TotalDepth.util.plot.PRESCfg.}\sphinxbfcode{BACKUP\_RIGHT}\sphinxbfcode{ = (1, 0)}}
Single backup to right only

\end{fulllineitems}

\index{BACKUP\_FROM\_MODE\_MAP (in module TotalDepth.util.plot.PRESCfg)}

\begin{fulllineitems}
\phantomsection\label{\detokenize{ref/util/plot/PRESCfg:TotalDepth.util.plot.PRESCfg.BACKUP_FROM_MODE_MAP}}\pysigline{\sphinxcode{TotalDepth.util.plot.PRESCfg.}\sphinxbfcode{BACKUP\_FROM\_MODE\_MAP}\sphinxbfcode{ = \{None: (0, 0), b'SHIF': (-1, 1), b'GRAD': (1, -1), b'NB  ': (1, -1), b'WRAP': (0, 0)\}}}
Map of backup mode to internal representation

\end{fulllineitems}

\index{LineTransBase (class in TotalDepth.util.plot.PRESCfg)}

\begin{fulllineitems}
\phantomsection\label{\detokenize{ref/util/plot/PRESCfg:TotalDepth.util.plot.PRESCfg.LineTransBase}}\pysiglinewithargsret{\sphinxbfcode{class }\sphinxcode{TotalDepth.util.plot.PRESCfg.}\sphinxbfcode{LineTransBase}}{\emph{leftP}, \emph{rightP}, \emph{leftL}, \emph{rightL}, \emph{backup}}{}
Base class for line generators.
\index{\_\_init\_\_() (TotalDepth.util.plot.PRESCfg.LineTransBase method)}

\begin{fulllineitems}
\phantomsection\label{\detokenize{ref/util/plot/PRESCfg:TotalDepth.util.plot.PRESCfg.LineTransBase.__init__}}\pysiglinewithargsret{\sphinxbfcode{\_\_init\_\_}}{\emph{leftP}, \emph{rightP}, \emph{leftL}, \emph{rightL}, \emph{backup}}{}
Ctor with values; leftP, rightP are physical positions as numbers.
leftL, rightL are logical scales as numbers.
backup is a pair (left, right).
Will raise a ExceptionLineTransBase if leftP \textgreater{}= rightP.

\end{fulllineitems}

\index{leftL (TotalDepth.util.plot.PRESCfg.LineTransBase attribute)}

\begin{fulllineitems}
\phantomsection\label{\detokenize{ref/util/plot/PRESCfg:TotalDepth.util.plot.PRESCfg.LineTransBase.leftL}}\pysigline{\sphinxbfcode{leftL}}
The left value of the curve scale as a number.

\end{fulllineitems}

\index{rightL (TotalDepth.util.plot.PRESCfg.LineTransBase attribute)}

\begin{fulllineitems}
\phantomsection\label{\detokenize{ref/util/plot/PRESCfg:TotalDepth.util.plot.PRESCfg.LineTransBase.rightL}}\pysigline{\sphinxbfcode{rightL}}
The right value of the curve scale as a number.

\end{fulllineitems}

\index{L2P() (TotalDepth.util.plot.PRESCfg.LineTransBase method)}

\begin{fulllineitems}
\phantomsection\label{\detokenize{ref/util/plot/PRESCfg:TotalDepth.util.plot.PRESCfg.LineTransBase.L2P}}\pysiglinewithargsret{\sphinxbfcode{L2P}}{\emph{val}}{}
Scale a given value to a dimension.

\end{fulllineitems}

\index{wrapPos() (TotalDepth.util.plot.PRESCfg.LineTransBase method)}

\begin{fulllineitems}
\phantomsection\label{\detokenize{ref/util/plot/PRESCfg:TotalDepth.util.plot.PRESCfg.LineTransBase.wrapPos}}\pysiglinewithargsret{\sphinxbfcode{wrapPos}}{\emph{val}}{}
For a given value returns a pair (wrap, pos).
wrap is an integer number of times the curve is wrapped.
pos is a float that is the physical plot position of the value.
TODO: Benchmark this, it could be slow.

\end{fulllineitems}

\index{offScale() (TotalDepth.util.plot.PRESCfg.LineTransBase method)}

\begin{fulllineitems}
\phantomsection\label{\detokenize{ref/util/plot/PRESCfg:TotalDepth.util.plot.PRESCfg.LineTransBase.offScale}}\pysiglinewithargsret{\sphinxbfcode{offScale}}{\emph{w}}{}
Returns 0 if wrap integer is on scale depending on the backup setting.
Returns -1 if off scale low, +1 if off scale high.

\end{fulllineitems}

\index{isOffScaleLeft() (TotalDepth.util.plot.PRESCfg.LineTransBase method)}

\begin{fulllineitems}
\phantomsection\label{\detokenize{ref/util/plot/PRESCfg:TotalDepth.util.plot.PRESCfg.LineTransBase.isOffScaleLeft}}\pysiglinewithargsret{\sphinxbfcode{isOffScaleLeft}}{\emph{w}}{}
True is wrap integer is off-scale low according to the backup setting.

\end{fulllineitems}

\index{isOffScaleRight() (TotalDepth.util.plot.PRESCfg.LineTransBase method)}

\begin{fulllineitems}
\phantomsection\label{\detokenize{ref/util/plot/PRESCfg:TotalDepth.util.plot.PRESCfg.LineTransBase.isOffScaleRight}}\pysiglinewithargsret{\sphinxbfcode{isOffScaleRight}}{\emph{w}}{}
True is wrap integer is off-scale high according to the backup setting.

\end{fulllineitems}

\index{\_\_weakref\_\_ (TotalDepth.util.plot.PRESCfg.LineTransBase attribute)}

\begin{fulllineitems}
\phantomsection\label{\detokenize{ref/util/plot/PRESCfg:TotalDepth.util.plot.PRESCfg.LineTransBase.__weakref__}}\pysigline{\sphinxbfcode{\_\_weakref\_\_}}
list of weak references to the object (if defined)

\end{fulllineitems}


\end{fulllineitems}

\index{LineTransLin (class in TotalDepth.util.plot.PRESCfg)}

\begin{fulllineitems}
\phantomsection\label{\detokenize{ref/util/plot/PRESCfg:TotalDepth.util.plot.PRESCfg.LineTransLin}}\pysiglinewithargsret{\sphinxbfcode{class }\sphinxcode{TotalDepth.util.plot.PRESCfg.}\sphinxbfcode{LineTransLin}}{\emph{leftP}, \emph{rightP}, \emph{leftL}, \emph{rightL}, \emph{backup=(0}, \emph{0)}}{}
Linear grid.
\index{\_\_init\_\_() (TotalDepth.util.plot.PRESCfg.LineTransLin method)}

\begin{fulllineitems}
\phantomsection\label{\detokenize{ref/util/plot/PRESCfg:TotalDepth.util.plot.PRESCfg.LineTransLin.__init__}}\pysiglinewithargsret{\sphinxbfcode{\_\_init\_\_}}{\emph{leftP}, \emph{rightP}, \emph{leftL}, \emph{rightL}, \emph{backup=(0}, \emph{0)}}{}
Ctor with values; leftP, rightP are physical positions as numbers.
leftL, rightL are logical scales as numbers.
backup is a pair (left, right).

\end{fulllineitems}

\index{L2P() (TotalDepth.util.plot.PRESCfg.LineTransLin method)}

\begin{fulllineitems}
\phantomsection\label{\detokenize{ref/util/plot/PRESCfg:TotalDepth.util.plot.PRESCfg.LineTransLin.L2P}}\pysiglinewithargsret{\sphinxbfcode{L2P}}{\emph{val}}{}
Scale a given value to a dimension.

\end{fulllineitems}

\index{wrapPos() (TotalDepth.util.plot.PRESCfg.LineTransLin method)}

\begin{fulllineitems}
\phantomsection\label{\detokenize{ref/util/plot/PRESCfg:TotalDepth.util.plot.PRESCfg.LineTransLin.wrapPos}}\pysiglinewithargsret{\sphinxbfcode{wrapPos}}{\emph{val}}{}
For a given value returns a pair (wrap, pos).
wrap is an integer number of times the curve is wrapped.
pos is a float that is the physical plot position of the value.
TODO: Benchmark this, it could be slow.

\end{fulllineitems}


\end{fulllineitems}

\index{LineTransLog10 (class in TotalDepth.util.plot.PRESCfg)}

\begin{fulllineitems}
\phantomsection\label{\detokenize{ref/util/plot/PRESCfg:TotalDepth.util.plot.PRESCfg.LineTransLog10}}\pysiglinewithargsret{\sphinxbfcode{class }\sphinxcode{TotalDepth.util.plot.PRESCfg.}\sphinxbfcode{LineTransLog10}}{\emph{leftP}, \emph{rightP}, \emph{leftL}, \emph{rightR}, \emph{backup=(0}, \emph{0)}}{}
Logrithmic grid.
\index{\_\_init\_\_() (TotalDepth.util.plot.PRESCfg.LineTransLog10 method)}

\begin{fulllineitems}
\phantomsection\label{\detokenize{ref/util/plot/PRESCfg:TotalDepth.util.plot.PRESCfg.LineTransLog10.__init__}}\pysiglinewithargsret{\sphinxbfcode{\_\_init\_\_}}{\emph{leftP}, \emph{rightP}, \emph{leftL}, \emph{rightR}, \emph{backup=(0}, \emph{0)}}{}
Ctor with values; leftP, rightP are physical positions as numbers.
leftL, rightL are logical scales as numbers.
backup is a pair (left, right).

\end{fulllineitems}

\index{L2P() (TotalDepth.util.plot.PRESCfg.LineTransLog10 method)}

\begin{fulllineitems}
\phantomsection\label{\detokenize{ref/util/plot/PRESCfg:TotalDepth.util.plot.PRESCfg.LineTransLog10.L2P}}\pysiglinewithargsret{\sphinxbfcode{L2P}}{\emph{val}}{}
Scale a given value to a dimension.

\end{fulllineitems}

\index{wrapPos() (TotalDepth.util.plot.PRESCfg.LineTransLog10 method)}

\begin{fulllineitems}
\phantomsection\label{\detokenize{ref/util/plot/PRESCfg:TotalDepth.util.plot.PRESCfg.LineTransLog10.wrapPos}}\pysiglinewithargsret{\sphinxbfcode{wrapPos}}{\emph{val}}{}
For a given value returns a pair (wrap, pos).
wrap is an integer number of times the curve is wrapped.
pos is a float that is the physical plot position of the value.
TODO: Benchmark this, it could be slow.

\end{fulllineitems}


\end{fulllineitems}

\index{TrackWidthData (class in TotalDepth.util.plot.PRESCfg)}

\begin{fulllineitems}
\phantomsection\label{\detokenize{ref/util/plot/PRESCfg:TotalDepth.util.plot.PRESCfg.TrackWidthData}}\pysiglinewithargsret{\sphinxbfcode{class }\sphinxcode{TotalDepth.util.plot.PRESCfg.}\sphinxbfcode{TrackWidthData}}{\emph{leftP}, \emph{rightP}, \emph{halfTrackStart}, \emph{halfTracks}}{}
Used to record the physical width data of a track
leftP, rightP are Coord.Dim objects.
halfTrackStart, halfTracks are integers
\index{\_\_getnewargs\_\_() (TotalDepth.util.plot.PRESCfg.TrackWidthData method)}

\begin{fulllineitems}
\phantomsection\label{\detokenize{ref/util/plot/PRESCfg:TotalDepth.util.plot.PRESCfg.TrackWidthData.__getnewargs__}}\pysiglinewithargsret{\sphinxbfcode{\_\_getnewargs\_\_}}{}{}
Return self as a plain tuple.  Used by copy and pickle.

\end{fulllineitems}

\index{\_\_new\_\_() (TotalDepth.util.plot.PRESCfg.TrackWidthData static method)}

\begin{fulllineitems}
\phantomsection\label{\detokenize{ref/util/plot/PRESCfg:TotalDepth.util.plot.PRESCfg.TrackWidthData.__new__}}\pysiglinewithargsret{\sphinxbfcode{static }\sphinxbfcode{\_\_new\_\_}}{\emph{\_cls}, \emph{leftP}, \emph{rightP}, \emph{halfTrackStart}, \emph{halfTracks}}{}
Create new instance of TrackWidthData(leftP, rightP, halfTrackStart, halfTracks)

\end{fulllineitems}

\index{\_\_repr\_\_() (TotalDepth.util.plot.PRESCfg.TrackWidthData method)}

\begin{fulllineitems}
\phantomsection\label{\detokenize{ref/util/plot/PRESCfg:TotalDepth.util.plot.PRESCfg.TrackWidthData.__repr__}}\pysiglinewithargsret{\sphinxbfcode{\_\_repr\_\_}}{}{}
Return a nicely formatted representation string

\end{fulllineitems}

\index{halfTrackStart (TotalDepth.util.plot.PRESCfg.TrackWidthData attribute)}

\begin{fulllineitems}
\phantomsection\label{\detokenize{ref/util/plot/PRESCfg:TotalDepth.util.plot.PRESCfg.TrackWidthData.halfTrackStart}}\pysigline{\sphinxbfcode{halfTrackStart}}
Alias for field number 2

\end{fulllineitems}

\index{halfTracks (TotalDepth.util.plot.PRESCfg.TrackWidthData attribute)}

\begin{fulllineitems}
\phantomsection\label{\detokenize{ref/util/plot/PRESCfg:TotalDepth.util.plot.PRESCfg.TrackWidthData.halfTracks}}\pysigline{\sphinxbfcode{halfTracks}}
Alias for field number 3

\end{fulllineitems}

\index{leftP (TotalDepth.util.plot.PRESCfg.TrackWidthData attribute)}

\begin{fulllineitems}
\phantomsection\label{\detokenize{ref/util/plot/PRESCfg:TotalDepth.util.plot.PRESCfg.TrackWidthData.leftP}}\pysigline{\sphinxbfcode{leftP}}
Alias for field number 0

\end{fulllineitems}

\index{rightP (TotalDepth.util.plot.PRESCfg.TrackWidthData attribute)}

\begin{fulllineitems}
\phantomsection\label{\detokenize{ref/util/plot/PRESCfg:TotalDepth.util.plot.PRESCfg.TrackWidthData.rightP}}\pysigline{\sphinxbfcode{rightP}}
Alias for field number 1

\end{fulllineitems}


\end{fulllineitems}

\index{CurveCfg (class in TotalDepth.util.plot.PRESCfg)}

\begin{fulllineitems}
\phantomsection\label{\detokenize{ref/util/plot/PRESCfg:TotalDepth.util.plot.PRESCfg.CurveCfg}}\pysigline{\sphinxbfcode{class }\sphinxcode{TotalDepth.util.plot.PRESCfg.}\sphinxbfcode{CurveCfg}}
Contains the configuration of a single curve.
\index{\_\_init\_\_() (TotalDepth.util.plot.PRESCfg.CurveCfg method)}

\begin{fulllineitems}
\phantomsection\label{\detokenize{ref/util/plot/PRESCfg:TotalDepth.util.plot.PRESCfg.CurveCfg.__init__}}\pysiglinewithargsret{\sphinxbfcode{\_\_init\_\_}}{}{}
Populate attribute with reasonable default values.
Second stage is to set:
self.mnem, self.outp, self.trac, self.dest.
Optionally other properties as well.

\end{fulllineitems}

\index{longStr() (TotalDepth.util.plot.PRESCfg.CurveCfg method)}

\begin{fulllineitems}
\phantomsection\label{\detokenize{ref/util/plot/PRESCfg:TotalDepth.util.plot.PRESCfg.CurveCfg.longStr}}\pysiglinewithargsret{\sphinxbfcode{longStr}}{}{}
Returns a long descriptive string of the internal state.

\end{fulllineitems}

\index{tracWidthData() (TotalDepth.util.plot.PRESCfg.CurveCfg method)}

\begin{fulllineitems}
\phantomsection\label{\detokenize{ref/util/plot/PRESCfg:TotalDepth.util.plot.PRESCfg.CurveCfg.tracWidthData}}\pysiglinewithargsret{\sphinxbfcode{tracWidthData}}{\emph{theFilmID}}{}
Returns a TrackWidthData object for the film ID.

\end{fulllineitems}

\index{tracValueFunction() (TotalDepth.util.plot.PRESCfg.CurveCfg method)}

\begin{fulllineitems}
\phantomsection\label{\detokenize{ref/util/plot/PRESCfg:TotalDepth.util.plot.PRESCfg.CurveCfg.tracValueFunction}}\pysiglinewithargsret{\sphinxbfcode{tracValueFunction}}{\emph{theFilmID}}{}
Given a FILM ID (a Mnem() object) this returns a LineTransBase or
derivation that describes how this curve is plotted on that film.
In particular the return value will have a function wrapPos() for
generating track positions from a value.

\end{fulllineitems}

\index{\_\_weakref\_\_ (TotalDepth.util.plot.PRESCfg.CurveCfg attribute)}

\begin{fulllineitems}
\phantomsection\label{\detokenize{ref/util/plot/PRESCfg:TotalDepth.util.plot.PRESCfg.CurveCfg.__weakref__}}\pysigline{\sphinxbfcode{\_\_weakref\_\_}}
list of weak references to the object (if defined)

\end{fulllineitems}


\end{fulllineitems}

\index{PresCfg (class in TotalDepth.util.plot.PRESCfg)}

\begin{fulllineitems}
\phantomsection\label{\detokenize{ref/util/plot/PRESCfg:TotalDepth.util.plot.PRESCfg.PresCfg}}\pysigline{\sphinxbfcode{class }\sphinxcode{TotalDepth.util.plot.PRESCfg.}\sphinxbfcode{PresCfg}}
Contains the configuration equivalent to a complete PRES table.
\index{add() (TotalDepth.util.plot.PRESCfg.PresCfg method)}

\begin{fulllineitems}
\phantomsection\label{\detokenize{ref/util/plot/PRESCfg:TotalDepth.util.plot.PRESCfg.PresCfg.add}}\pysiglinewithargsret{\sphinxbfcode{add}}{\emph{theCurveCfg}, \emph{theFilmDestS}}{}
Adds to the IR. theCurveCfg is a CurvCfg object, theFilmDestS is a
list of film destinations expanded from the FILM table e.g. if the
destination is b’ALL’ then all FILM destination mnemonics should be in
the list.

\end{fulllineitems}

\index{keys() (TotalDepth.util.plot.PRESCfg.PresCfg method)}

\begin{fulllineitems}
\phantomsection\label{\detokenize{ref/util/plot/PRESCfg:TotalDepth.util.plot.PRESCfg.PresCfg.keys}}\pysiglinewithargsret{\sphinxbfcode{keys}}{}{}
Returns the curve mnemonics.

\end{fulllineitems}

\index{\_\_len\_\_() (TotalDepth.util.plot.PRESCfg.PresCfg method)}

\begin{fulllineitems}
\phantomsection\label{\detokenize{ref/util/plot/PRESCfg:TotalDepth.util.plot.PRESCfg.PresCfg.__len__}}\pysiglinewithargsret{\sphinxbfcode{\_\_len\_\_}}{}{}
Number of curves in this table.

\end{fulllineitems}

\index{\_\_getitem\_\_() (TotalDepth.util.plot.PRESCfg.PresCfg method)}

\begin{fulllineitems}
\phantomsection\label{\detokenize{ref/util/plot/PRESCfg:TotalDepth.util.plot.PRESCfg.PresCfg.__getitem__}}\pysiglinewithargsret{\sphinxbfcode{\_\_getitem\_\_}}{\emph{theCurvID}}{}
Returns the CurveCfg object corresponding to curve ID, a Mnem.Mnem object.

\end{fulllineitems}

\index{\_\_contains\_\_() (TotalDepth.util.plot.PRESCfg.PresCfg method)}

\begin{fulllineitems}
\phantomsection\label{\detokenize{ref/util/plot/PRESCfg:TotalDepth.util.plot.PRESCfg.PresCfg.__contains__}}\pysiglinewithargsret{\sphinxbfcode{\_\_contains\_\_}}{\emph{theCurvID}}{}
Returns True if I have an entry for the curve ID, a Mnem.Mnem object.

\end{fulllineitems}

\index{\_\_weakref\_\_ (TotalDepth.util.plot.PRESCfg.PresCfg attribute)}

\begin{fulllineitems}
\phantomsection\label{\detokenize{ref/util/plot/PRESCfg:TotalDepth.util.plot.PRESCfg.PresCfg.__weakref__}}\pysigline{\sphinxbfcode{\_\_weakref\_\_}}
list of weak references to the object (if defined)

\end{fulllineitems}

\index{hasCurvesForDest() (TotalDepth.util.plot.PRESCfg.PresCfg method)}

\begin{fulllineitems}
\phantomsection\label{\detokenize{ref/util/plot/PRESCfg:TotalDepth.util.plot.PRESCfg.PresCfg.hasCurvesForDest}}\pysiglinewithargsret{\sphinxbfcode{hasCurvesForDest}}{\emph{theDest}}{}
Returns True if there are curves that go to theDest i.e. FILM ID.

\end{fulllineitems}

\index{outpChIDs() (TotalDepth.util.plot.PRESCfg.PresCfg method)}

\begin{fulllineitems}
\phantomsection\label{\detokenize{ref/util/plot/PRESCfg:TotalDepth.util.plot.PRESCfg.PresCfg.outpChIDs}}\pysiglinewithargsret{\sphinxbfcode{outpChIDs}}{\emph{theDest}}{}
Returns a list of output channel IDs for a given film destination, a Mnem.Mnem object.

\end{fulllineitems}

\index{outpCurveIDs() (TotalDepth.util.plot.PRESCfg.PresCfg method)}

\begin{fulllineitems}
\phantomsection\label{\detokenize{ref/util/plot/PRESCfg:TotalDepth.util.plot.PRESCfg.PresCfg.outpCurveIDs}}\pysiglinewithargsret{\sphinxbfcode{outpCurveIDs}}{\emph{theDest}, \emph{theOutp}}{}
Returns a list of channel IDs for a given film destination and output
that feeds those curves. The curve data is accessible by \_\_getitem\_\_().
Arguments should be Mnem.Mnem objects

\end{fulllineitems}

\index{usesOutpChannel() (TotalDepth.util.plot.PRESCfg.PresCfg method)}

\begin{fulllineitems}
\phantomsection\label{\detokenize{ref/util/plot/PRESCfg:TotalDepth.util.plot.PRESCfg.PresCfg.usesOutpChannel}}\pysiglinewithargsret{\sphinxbfcode{usesOutpChannel}}{\emph{theDest}, \emph{theOutp}}{}
Returns True if this PRES table + FILM  destination uses theOutp ID.
Arguments should be , a Mnem.Mnem objects.

\end{fulllineitems}


\end{fulllineitems}

\index{PresCfgLISRead (class in TotalDepth.util.plot.PRESCfg)}

\begin{fulllineitems}
\phantomsection\label{\detokenize{ref/util/plot/PRESCfg:TotalDepth.util.plot.PRESCfg.PresCfgLISRead}}\pysiglinewithargsret{\sphinxbfcode{class }\sphinxcode{TotalDepth.util.plot.PRESCfg.}\sphinxbfcode{PresCfgLISRead}}{\emph{theLr}, \emph{theFILMCfg}}{}
Information from a complete LIS PRES table.
\index{\_\_init\_\_() (TotalDepth.util.plot.PRESCfg.PresCfgLISRead method)}

\begin{fulllineitems}
\phantomsection\label{\detokenize{ref/util/plot/PRESCfg:TotalDepth.util.plot.PRESCfg.PresCfgLISRead.__init__}}\pysiglinewithargsret{\sphinxbfcode{\_\_init\_\_}}{\emph{theLr}, \emph{theFILMCfg}}{}
Reads a LogiRec.Table object of type PRES and creates a
CurveCfgLISRead for each row. theFILMCfg is expected to be a
FILMCfgLISRead object.

Typical PRES table:

\begin{sphinxVerbatim}[commandchars=\\\{\}]
\PYG{n}{MNEM}  \PYG{n}{OUTP}  \PYG{n}{STAT}  \PYG{n}{TRAC}  \PYG{n}{CODI}  \PYG{n}{DEST}  \PYG{n}{MODE}      \PYG{n}{FILT}          \PYG{n}{LEDG}          \PYG{n}{REDG}
\PYG{o}{\PYGZhy{}}\PYG{o}{\PYGZhy{}}\PYG{o}{\PYGZhy{}}\PYG{o}{\PYGZhy{}}\PYG{o}{\PYGZhy{}}\PYG{o}{\PYGZhy{}}\PYG{o}{\PYGZhy{}}\PYG{o}{\PYGZhy{}}\PYG{o}{\PYGZhy{}}\PYG{o}{\PYGZhy{}}\PYG{o}{\PYGZhy{}}\PYG{o}{\PYGZhy{}}\PYG{o}{\PYGZhy{}}\PYG{o}{\PYGZhy{}}\PYG{o}{\PYGZhy{}}\PYG{o}{\PYGZhy{}}\PYG{o}{\PYGZhy{}}\PYG{o}{\PYGZhy{}}\PYG{o}{\PYGZhy{}}\PYG{o}{\PYGZhy{}}\PYG{o}{\PYGZhy{}}\PYG{o}{\PYGZhy{}}\PYG{o}{\PYGZhy{}}\PYG{o}{\PYGZhy{}}\PYG{o}{\PYGZhy{}}\PYG{o}{\PYGZhy{}}\PYG{o}{\PYGZhy{}}\PYG{o}{\PYGZhy{}}\PYG{o}{\PYGZhy{}}\PYG{o}{\PYGZhy{}}\PYG{o}{\PYGZhy{}}\PYG{o}{\PYGZhy{}}\PYG{o}{\PYGZhy{}}\PYG{o}{\PYGZhy{}}\PYG{o}{\PYGZhy{}}\PYG{o}{\PYGZhy{}}\PYG{o}{\PYGZhy{}}\PYG{o}{\PYGZhy{}}\PYG{o}{\PYGZhy{}}\PYG{o}{\PYGZhy{}}\PYG{o}{\PYGZhy{}}\PYG{o}{\PYGZhy{}}\PYG{o}{\PYGZhy{}}\PYG{o}{\PYGZhy{}}\PYG{o}{\PYGZhy{}}\PYG{o}{\PYGZhy{}}\PYG{o}{\PYGZhy{}}\PYG{o}{\PYGZhy{}}\PYG{o}{\PYGZhy{}}\PYG{o}{\PYGZhy{}}\PYG{o}{\PYGZhy{}}\PYG{o}{\PYGZhy{}}\PYG{o}{\PYGZhy{}}\PYG{o}{\PYGZhy{}}\PYG{o}{\PYGZhy{}}\PYG{o}{\PYGZhy{}}\PYG{o}{\PYGZhy{}}\PYG{o}{\PYGZhy{}}\PYG{o}{\PYGZhy{}}\PYG{o}{\PYGZhy{}}\PYG{o}{\PYGZhy{}}\PYG{o}{\PYGZhy{}}\PYG{o}{\PYGZhy{}}\PYG{o}{\PYGZhy{}}\PYG{o}{\PYGZhy{}}\PYG{o}{\PYGZhy{}}\PYG{o}{\PYGZhy{}}\PYG{o}{\PYGZhy{}}\PYG{o}{\PYGZhy{}}\PYG{o}{\PYGZhy{}}\PYG{o}{\PYGZhy{}}\PYG{o}{\PYGZhy{}}\PYG{o}{\PYGZhy{}}\PYG{o}{\PYGZhy{}}\PYG{o}{\PYGZhy{}}\PYG{o}{\PYGZhy{}}\PYG{o}{\PYGZhy{}}\PYG{o}{\PYGZhy{}}\PYG{o}{\PYGZhy{}}\PYG{o}{\PYGZhy{}}\PYG{o}{\PYGZhy{}}\PYG{o}{\PYGZhy{}}\PYG{o}{\PYGZhy{}}
\PYG{n}{NPHI}  \PYG{n}{NPHI}  \PYG{n}{ALLO}  \PYG{n}{T23}   \PYG{n}{LDAS}  \PYG{l+m+mi}{1}     \PYG{n}{SHIF}      \PYG{l+m+mf}{0.500000}      \PYG{l+m+mf}{0.450000}     \PYG{o}{\PYGZhy{}}\PYG{l+m+mf}{0.150000}
\PYG{n}{DRHO}  \PYG{n}{DRHO}  \PYG{n}{ALLO}  \PYG{n}{T3}    \PYG{n}{LSPO}  \PYG{l+m+mi}{1}     \PYG{n}{NB}        \PYG{l+m+mf}{0.500000}     \PYG{o}{\PYGZhy{}}\PYG{l+m+mf}{0.250000}      \PYG{l+m+mf}{0.250000}
\PYG{n}{PEF}   \PYG{n}{PEF}   \PYG{n}{ALLO}  \PYG{n}{T23}   \PYG{n}{LGAP}  \PYG{l+m+mi}{1}     \PYG{n}{SHIF}      \PYG{l+m+mf}{0.500000}       \PYG{l+m+mf}{0.00000}       \PYG{l+m+mf}{10.0000}
\end{sphinxVerbatim}

\end{fulllineitems}


\end{fulllineitems}



\paragraph{PRES Plotting Configuration from XML files}
\label{\detokenize{ref/util/plot/PRESCfgXML:module-TotalDepth.util.plot.PRESCfgXML}}\label{\detokenize{ref/util/plot/PRESCfgXML::doc}}\label{\detokenize{ref/util/plot/PRESCfgXML:pres-plotting-configuration-from-xml-files}}\index{TotalDepth.util.plot.PRESCfgXML (module)}
Creates PRES configurations from LgFormat XML files.

Created on Dec 16, 2011
\index{ExceptionPRESCfgXML}

\begin{fulllineitems}
\phantomsection\label{\detokenize{ref/util/plot/PRESCfgXML:TotalDepth.util.plot.PRESCfgXML.ExceptionPRESCfgXML}}\pysigline{\sphinxbfcode{exception }\sphinxcode{TotalDepth.util.plot.PRESCfgXML.}\sphinxbfcode{ExceptionPRESCfgXML}}
Specialisation of exception for PRESCfgXML module.

\end{fulllineitems}

\index{ExceptionCurveCfgXMLRead}

\begin{fulllineitems}
\phantomsection\label{\detokenize{ref/util/plot/PRESCfgXML:TotalDepth.util.plot.PRESCfgXML.ExceptionCurveCfgXMLRead}}\pysigline{\sphinxbfcode{exception }\sphinxcode{TotalDepth.util.plot.PRESCfgXML.}\sphinxbfcode{ExceptionCurveCfgXMLRead}}
Specialisation of exception for CurveCfgXMLRead module.

\end{fulllineitems}

\index{ExceptionPresCfgXMLRead}

\begin{fulllineitems}
\phantomsection\label{\detokenize{ref/util/plot/PRESCfgXML:TotalDepth.util.plot.PRESCfgXML.ExceptionPresCfgXMLRead}}\pysigline{\sphinxbfcode{exception }\sphinxcode{TotalDepth.util.plot.PRESCfgXML.}\sphinxbfcode{ExceptionPresCfgXMLRead}}
Specialisation of exception for PresCfgXMLRead module.

\end{fulllineitems}

\index{XML\_CODI\_MAP (in module TotalDepth.util.plot.PRESCfgXML)}

\begin{fulllineitems}
\phantomsection\label{\detokenize{ref/util/plot/PRESCfgXML:TotalDepth.util.plot.PRESCfgXML.XML_CODI_MAP}}\pysigline{\sphinxcode{TotalDepth.util.plot.PRESCfgXML.}\sphinxbfcode{XML\_CODI\_MAP}\sphinxbfcode{ = \{None: Stroke(width=0.5, colour='black', coding=None, opacity=1.0), 'LG\_SOLID\_LINE': Stroke(width=0.5, colour='black', coding=None, opacity=1.0), 'LG\_DOT\_LINE': Stroke(width=0.5, colour='black', coding=(2, 2), opacity=1.0), 'LG\_DASH\_LINE': Stroke(width=0.5, colour='black', coding=(4, 4), opacity=1.0), 'LG\_LONG\_DASH\_LINE': Stroke(width=0.5, colour='black', coding=(6, 2), opacity=1.0), 'LG\_SOLID\_HEAVY\_LINE': Stroke(width=1.5, colour='black', coding=None, opacity=1.0), 'LG\_SOLID\_HEAVY\_DOT': Stroke(width=1.5, colour='black', coding=(2, 2), opacity=1.0), 'LG\_SOLID\_HEAVY\_DASH': Stroke(width=1.5, colour='black', coding=(4, 4), opacity=1.0), 'LG\_SOLID\_HEAVY\_LONG\_DASH': Stroke(width=1.5, colour='black', coding=(6, 2), opacity=1.0)\}}}
Maps LgFormat mnemonics to a Stroke object:
If either value is None an SVG attribute is not needed i.e. default SVG behaviour

\end{fulllineitems}

\index{BACKUP\_FROM\_MODE\_MAP (in module TotalDepth.util.plot.PRESCfgXML)}

\begin{fulllineitems}
\phantomsection\label{\detokenize{ref/util/plot/PRESCfgXML:TotalDepth.util.plot.PRESCfgXML.BACKUP_FROM_MODE_MAP}}\pysigline{\sphinxcode{TotalDepth.util.plot.PRESCfgXML.}\sphinxbfcode{BACKUP\_FROM\_MODE\_MAP}\sphinxbfcode{ = \{None: (0, 0), 'LG\_LEFT\_WRAPPED': (0, -1), 'LG\_RIGHT\_WRAPPED': (1, 0), 'LG\_WRAPPED': (0, 0), 'LG\_X10': (0, 0), '1': (-1, 1), '2': (-2, 2)\}}}
Maps LgFormat backup specifications to an internal representation
Taken from the \sphinxcode{\textless{}WrapMode\textgreater{}} element.

\end{fulllineitems}

\index{BACKUP\_FROM\_COUNT\_MAP (in module TotalDepth.util.plot.PRESCfgXML)}

\begin{fulllineitems}
\phantomsection\label{\detokenize{ref/util/plot/PRESCfgXML:TotalDepth.util.plot.PRESCfgXML.BACKUP_FROM_COUNT_MAP}}\pysigline{\sphinxcode{TotalDepth.util.plot.PRESCfgXML.}\sphinxbfcode{BACKUP\_FROM\_COUNT\_MAP}\sphinxbfcode{ = \{None: (0, 0), '1': (-1, 1), '2': (-2, 2)\}}}
Fallback mapping LgFormat backup specifications to an internal representation
Taken from the \sphinxcode{\textless{}WrapCount\textgreater{}} element.

\end{fulllineitems}

\index{CurveCfgXMLRead (class in TotalDepth.util.plot.PRESCfgXML)}

\begin{fulllineitems}
\phantomsection\label{\detokenize{ref/util/plot/PRESCfgXML:TotalDepth.util.plot.PRESCfgXML.CurveCfgXMLRead}}\pysiglinewithargsret{\sphinxbfcode{class }\sphinxcode{TotalDepth.util.plot.PRESCfgXML.}\sphinxbfcode{CurveCfgXMLRead}}{\emph{e}, \emph{theTrac}, \emph{theFILMCfg}}{}
Represents a single curve from an XML file specification
\index{TRAC\_XML\_UNIQUEID\_TO\_PRES (TotalDepth.util.plot.PRESCfgXML.CurveCfgXMLRead attribute)}

\begin{fulllineitems}
\phantomsection\label{\detokenize{ref/util/plot/PRESCfgXML:TotalDepth.util.plot.PRESCfgXML.CurveCfgXMLRead.TRAC_XML_UNIQUEID_TO_PRES}}\pysigline{\sphinxbfcode{TRAC\_XML\_UNIQUEID\_TO\_PRES}\sphinxbfcode{ = \{'depthTrack': b'TD  ', 'DepthTrack': b'TD  ', 'track1': b'T1  ', 'track12': b'T12 ', 'track2': b'T2  ', 'Track2': b'T2  ', 'track23': b'T23 ', 'track3': b'T3  ', 'Track3': b'T3  ', 'track4': b'T4  ', 'TrackFC0': b'LHT2', 'TrackFC1': b'LHT2', 'TrackFC2': b'RHT2', 'TrackFC3': b'LHT3', 'TrackFC4': b'RHT2', 'trackLHT1': b'LHT1', 'trackRHT1': b'RHT1'\}}}
First column is observed tracks in the XML, note capitalisation inconsistencies.
Second column is LIS DEST equivalent

\end{fulllineitems}

\index{\_\_init\_\_() (TotalDepth.util.plot.PRESCfgXML.CurveCfgXMLRead method)}

\begin{fulllineitems}
\phantomsection\label{\detokenize{ref/util/plot/PRESCfgXML:TotalDepth.util.plot.PRESCfgXML.CurveCfgXMLRead.__init__}}\pysiglinewithargsret{\sphinxbfcode{\_\_init\_\_}}{\emph{e}, \emph{theTrac}, \emph{theFILMCfg}}{}
Creates a single CurveCfg object from an XML LgCurve element and populates a CurveCfg.
e is the root LgCurve element.
theTrac is the track name that this is being plotted on.
theFILMCfg is expected to be a FilmCfgXMLRead object.

Example:

\begin{sphinxVerbatim}[commandchars=\\\{\}]
\PYG{o}{\PYGZlt{}}\PYG{n}{LgCurve} \PYG{n}{UniqueId}\PYG{o}{=}\PYG{l+s+s2}{\PYGZdq{}}\PYG{l+s+s2}{ROP5}\PYG{l+s+s2}{\PYGZdq{}}\PYG{o}{\PYGZgt{}}
    \PYG{o}{\PYGZlt{}}\PYG{n}{ChannelName}\PYG{o}{\PYGZgt{}}\PYG{n}{ROP5}\PYG{o}{\PYGZlt{}}\PYG{o}{/}\PYG{n}{ChannelName}\PYG{o}{\PYGZgt{}}
    \PYG{o}{\PYGZlt{}}\PYG{n}{Color}\PYG{o}{\PYGZgt{}}\PYG{l+m+mi}{0000}\PYG{n}{FF}\PYG{o}{\PYGZlt{}}\PYG{o}{/}\PYG{n}{Color}\PYG{o}{\PYGZgt{}}
    \PYG{o}{\PYGZlt{}}\PYG{n}{LeftLimit}\PYG{o}{\PYGZgt{}}\PYG{l+m+mi}{500}\PYG{o}{\PYGZlt{}}\PYG{o}{/}\PYG{n}{LeftLimit}\PYG{o}{\PYGZgt{}}
    \PYG{o}{\PYGZlt{}}\PYG{n}{RightLimit}\PYG{o}{\PYGZgt{}}\PYG{l+m+mi}{0}\PYG{o}{\PYGZlt{}}\PYG{o}{/}\PYG{n}{RightLimit}\PYG{o}{\PYGZgt{}}
    \PYG{o}{\PYGZlt{}}\PYG{n}{LineStyle}\PYG{o}{\PYGZgt{}}\PYG{n}{LG\PYGZus{}DASH\PYGZus{}LINE}\PYG{o}{\PYGZlt{}}\PYG{o}{/}\PYG{n}{LineStyle}\PYG{o}{\PYGZgt{}}
    \PYG{o}{\PYGZlt{}}\PYG{n}{Thickness}\PYG{o}{\PYGZgt{}}\PYG{l+m+mf}{1.75}\PYG{o}{\PYGZlt{}}\PYG{o}{/}\PYG{n}{Thickness}\PYG{o}{\PYGZgt{}}
    \PYG{o}{\PYGZlt{}}\PYG{n}{WrapMode}\PYG{o}{\PYGZgt{}}\PYG{n}{LG\PYGZus{}LEFT\PYGZus{}WRAPPED}\PYG{o}{\PYGZlt{}}\PYG{o}{/}\PYG{n}{WrapMode}\PYG{o}{\PYGZgt{}}
\PYG{o}{\PYGZlt{}}\PYG{o}{/}\PYG{n}{LgCurve}\PYG{o}{\PYGZgt{}}
\PYG{n}{Or}\PYG{p}{:}
\PYG{o}{\PYGZlt{}}\PYG{n}{LgCurve} \PYG{n}{UniqueId}\PYG{o}{=}\PYG{l+s+s2}{\PYGZdq{}}\PYG{l+s+s2}{PSR}\PYG{l+s+s2}{\PYGZdq{}}\PYG{o}{\PYGZgt{}}
    \PYG{o}{\PYGZlt{}}\PYG{n}{ChannelName}\PYG{o}{\PYGZgt{}}\PYG{n}{PSR}\PYG{o}{\PYGZlt{}}\PYG{o}{/}\PYG{n}{ChannelName}\PYG{o}{\PYGZgt{}}
    \PYG{o}{\PYGZlt{}}\PYG{n}{Color}\PYG{o}{\PYGZgt{}}\PYG{l+m+mi}{00}\PYG{n}{C000}\PYG{o}{\PYGZlt{}}\PYG{o}{/}\PYG{n}{Color}\PYG{o}{\PYGZgt{}}
    \PYG{o}{\PYGZlt{}}\PYG{n}{LeftLimit}\PYG{o}{\PYGZgt{}}\PYG{l+m+mf}{0.2}\PYG{o}{\PYGZlt{}}\PYG{o}{/}\PYG{n}{LeftLimit}\PYG{o}{\PYGZgt{}}
    \PYG{o}{\PYGZlt{}}\PYG{n}{RightLimit}\PYG{o}{\PYGZgt{}}\PYG{l+m+mi}{2000}\PYG{o}{\PYGZlt{}}\PYG{o}{/}\PYG{n}{RightLimit}\PYG{o}{\PYGZgt{}}
    \PYG{o}{\PYGZlt{}}\PYG{n}{Thickness}\PYG{o}{\PYGZgt{}}\PYG{l+m+mi}{2}\PYG{o}{\PYGZlt{}}\PYG{o}{/}\PYG{n}{Thickness}\PYG{o}{\PYGZgt{}}
    \PYG{o}{\PYGZlt{}}\PYG{n}{Transform}\PYG{o}{\PYGZgt{}}\PYG{n}{LG\PYGZus{}LOGARITHMIC}\PYG{o}{\PYGZlt{}}\PYG{o}{/}\PYG{n}{Transform}\PYG{o}{\PYGZgt{}}
    \PYG{o}{\PYGZlt{}}\PYG{n}{WrapCount}\PYG{o}{\PYGZgt{}}\PYG{l+m+mi}{0}\PYG{o}{\PYGZlt{}}\PYG{o}{/}\PYG{n}{WrapCount}\PYG{o}{\PYGZgt{}}
\PYG{o}{\PYGZlt{}}\PYG{o}{/}\PYG{n}{LgCurve}\PYG{o}{\PYGZgt{}}
\end{sphinxVerbatim}

\end{fulllineitems}


\end{fulllineitems}

\index{PresCfgXMLRead (class in TotalDepth.util.plot.PRESCfgXML)}

\begin{fulllineitems}
\phantomsection\label{\detokenize{ref/util/plot/PRESCfgXML:TotalDepth.util.plot.PRESCfgXML.PresCfgXMLRead}}\pysiglinewithargsret{\sphinxbfcode{class }\sphinxcode{TotalDepth.util.plot.PRESCfgXML.}\sphinxbfcode{PresCfgXMLRead}}{\emph{theFILMCfg}, \emph{theUniqueId}}{}
Extracts all curve presentation information from a single XML file.
\index{\_\_init\_\_() (TotalDepth.util.plot.PRESCfgXML.PresCfgXMLRead method)}

\begin{fulllineitems}
\phantomsection\label{\detokenize{ref/util/plot/PRESCfgXML:TotalDepth.util.plot.PRESCfgXML.PresCfgXMLRead.__init__}}\pysiglinewithargsret{\sphinxbfcode{\_\_init\_\_}}{\emph{theFILMCfg}, \emph{theUniqueId}}{}
Reads a XML and creates a CurveCfgXMLRead for
each LgFormat/LgTrack/LgCurve element.

theFILMCfg is expected to be a FILMCfgXMLRead object.

\end{fulllineitems}


\end{fulllineitems}



\paragraph{Plotting Well Log API headers}
\label{\detokenize{ref/util/plot/LogHeader:plotting-well-log-api-headers}}\label{\detokenize{ref/util/plot/LogHeader:module-TotalDepth.util.plot.LogHeader}}\label{\detokenize{ref/util/plot/LogHeader::doc}}\index{TotalDepth.util.plot.LogHeader (module)}
Plots various headers as SVG. Of note is the API header

Created on Dec 30, 2011
\index{ExceptionLogHeader}

\begin{fulllineitems}
\phantomsection\label{\detokenize{ref/util/plot/LogHeader:TotalDepth.util.plot.LogHeader.ExceptionLogHeader}}\pysigline{\sphinxbfcode{exception }\sphinxcode{TotalDepth.util.plot.LogHeader.}\sphinxbfcode{ExceptionLogHeader}}
Exception for plotting Log Headers.

\end{fulllineitems}

\index{ExceptionLogHeaderLIS}

\begin{fulllineitems}
\phantomsection\label{\detokenize{ref/util/plot/LogHeader:TotalDepth.util.plot.LogHeader.ExceptionLogHeaderLIS}}\pysigline{\sphinxbfcode{exception }\sphinxcode{TotalDepth.util.plot.LogHeader.}\sphinxbfcode{ExceptionLogHeaderLIS}}
Exception for plotting Log Headers from LIS data.

\end{fulllineitems}

\index{ExceptionLogHeaderLAS}

\begin{fulllineitems}
\phantomsection\label{\detokenize{ref/util/plot/LogHeader:TotalDepth.util.plot.LogHeader.ExceptionLogHeaderLAS}}\pysigline{\sphinxbfcode{exception }\sphinxcode{TotalDepth.util.plot.LogHeader.}\sphinxbfcode{ExceptionLogHeaderLAS}}
Exception for plotting Log Headers from LAS data.

\end{fulllineitems}

\index{Static (class in TotalDepth.util.plot.LogHeader)}

\begin{fulllineitems}
\phantomsection\label{\detokenize{ref/util/plot/LogHeader:TotalDepth.util.plot.LogHeader.Static}}\pysiglinewithargsret{\sphinxbfcode{class }\sphinxcode{TotalDepth.util.plot.LogHeader.}\sphinxbfcode{Static}}{\emph{x}, \emph{y}, \emph{w}, \emph{d}, \emph{text}, \emph{font}, \emph{size}, \emph{tAttr}, \emph{rAttr}, \emph{mnem}, \emph{xMnem}}{}
Tuple to describe static data locations
If text is None then no text will be plotted
If rAttr is None then no box will be plotted
If mnem is non-None then a Mmem can be plotted at xMenm increment, a Coord.Dim()
\index{d (TotalDepth.util.plot.LogHeader.Static attribute)}

\begin{fulllineitems}
\phantomsection\label{\detokenize{ref/util/plot/LogHeader:TotalDepth.util.plot.LogHeader.Static.d}}\pysigline{\sphinxbfcode{d}}
Alias for field number 3

\end{fulllineitems}

\index{font (TotalDepth.util.plot.LogHeader.Static attribute)}

\begin{fulllineitems}
\phantomsection\label{\detokenize{ref/util/plot/LogHeader:TotalDepth.util.plot.LogHeader.Static.font}}\pysigline{\sphinxbfcode{font}}
Alias for field number 5

\end{fulllineitems}

\index{mnem (TotalDepth.util.plot.LogHeader.Static attribute)}

\begin{fulllineitems}
\phantomsection\label{\detokenize{ref/util/plot/LogHeader:TotalDepth.util.plot.LogHeader.Static.mnem}}\pysigline{\sphinxbfcode{mnem}}
Alias for field number 9

\end{fulllineitems}

\index{rAttr (TotalDepth.util.plot.LogHeader.Static attribute)}

\begin{fulllineitems}
\phantomsection\label{\detokenize{ref/util/plot/LogHeader:TotalDepth.util.plot.LogHeader.Static.rAttr}}\pysigline{\sphinxbfcode{rAttr}}
Alias for field number 8

\end{fulllineitems}

\index{size (TotalDepth.util.plot.LogHeader.Static attribute)}

\begin{fulllineitems}
\phantomsection\label{\detokenize{ref/util/plot/LogHeader:TotalDepth.util.plot.LogHeader.Static.size}}\pysigline{\sphinxbfcode{size}}
Alias for field number 6

\end{fulllineitems}

\index{tAttr (TotalDepth.util.plot.LogHeader.Static attribute)}

\begin{fulllineitems}
\phantomsection\label{\detokenize{ref/util/plot/LogHeader:TotalDepth.util.plot.LogHeader.Static.tAttr}}\pysigline{\sphinxbfcode{tAttr}}
Alias for field number 7

\end{fulllineitems}

\index{text (TotalDepth.util.plot.LogHeader.Static attribute)}

\begin{fulllineitems}
\phantomsection\label{\detokenize{ref/util/plot/LogHeader:TotalDepth.util.plot.LogHeader.Static.text}}\pysigline{\sphinxbfcode{text}}
Alias for field number 4

\end{fulllineitems}

\index{w (TotalDepth.util.plot.LogHeader.Static attribute)}

\begin{fulllineitems}
\phantomsection\label{\detokenize{ref/util/plot/LogHeader:TotalDepth.util.plot.LogHeader.Static.w}}\pysigline{\sphinxbfcode{w}}
Alias for field number 2

\end{fulllineitems}

\index{x (TotalDepth.util.plot.LogHeader.Static attribute)}

\begin{fulllineitems}
\phantomsection\label{\detokenize{ref/util/plot/LogHeader:TotalDepth.util.plot.LogHeader.Static.x}}\pysigline{\sphinxbfcode{x}}
Alias for field number 0

\end{fulllineitems}

\index{xMnem (TotalDepth.util.plot.LogHeader.Static attribute)}

\begin{fulllineitems}
\phantomsection\label{\detokenize{ref/util/plot/LogHeader:TotalDepth.util.plot.LogHeader.Static.xMnem}}\pysigline{\sphinxbfcode{xMnem}}
Alias for field number 10

\end{fulllineitems}

\index{y (TotalDepth.util.plot.LogHeader.Static attribute)}

\begin{fulllineitems}
\phantomsection\label{\detokenize{ref/util/plot/LogHeader:TotalDepth.util.plot.LogHeader.Static.y}}\pysigline{\sphinxbfcode{y}}
Alias for field number 1

\end{fulllineitems}


\end{fulllineitems}

\index{FONT\_PROP (in module TotalDepth.util.plot.LogHeader)}

\begin{fulllineitems}
\phantomsection\label{\detokenize{ref/util/plot/LogHeader:TotalDepth.util.plot.LogHeader.FONT_PROP}}\pysigline{\sphinxcode{TotalDepth.util.plot.LogHeader.}\sphinxbfcode{FONT\_PROP}\sphinxbfcode{ = 'Verdana'}}
Default font

\end{fulllineitems}

\index{RECT\_ATTRS\_FINE (in module TotalDepth.util.plot.LogHeader)}

\begin{fulllineitems}
\phantomsection\label{\detokenize{ref/util/plot/LogHeader:TotalDepth.util.plot.LogHeader.RECT_ATTRS_FINE}}\pysigline{\sphinxcode{TotalDepth.util.plot.LogHeader.}\sphinxbfcode{RECT\_ATTRS\_FINE}\sphinxbfcode{ = \{'fill': 'none', 'stroke': 'black', 'stroke-width': '.5'\}}}
SVG attributes for a fine lined rectangle

\end{fulllineitems}

\index{TEXT\_ATTRS\_FINE (in module TotalDepth.util.plot.LogHeader)}

\begin{fulllineitems}
\phantomsection\label{\detokenize{ref/util/plot/LogHeader:TotalDepth.util.plot.LogHeader.TEXT_ATTRS_FINE}}\pysigline{\sphinxcode{TotalDepth.util.plot.LogHeader.}\sphinxbfcode{TEXT\_ATTRS\_FINE}\sphinxbfcode{ = \{'text-anchor': 'start'\}}}
SVG attributes for a fine text

\end{fulllineitems}

\index{TEXT\_ATTRS\_LARGE (in module TotalDepth.util.plot.LogHeader)}

\begin{fulllineitems}
\phantomsection\label{\detokenize{ref/util/plot/LogHeader:TotalDepth.util.plot.LogHeader.TEXT_ATTRS_LARGE}}\pysigline{\sphinxcode{TotalDepth.util.plot.LogHeader.}\sphinxbfcode{TEXT\_ATTRS\_LARGE}\sphinxbfcode{ = \{'text-anchor': 'start', 'font-weight': 'bold'\}}}
SVG attributes for a bold text

\end{fulllineitems}

\index{TEXT\_ATTRS\_LARGE\_WOB (in module TotalDepth.util.plot.LogHeader)}

\begin{fulllineitems}
\phantomsection\label{\detokenize{ref/util/plot/LogHeader:TotalDepth.util.plot.LogHeader.TEXT_ATTRS_LARGE_WOB}}\pysigline{\sphinxcode{TotalDepth.util.plot.LogHeader.}\sphinxbfcode{TEXT\_ATTRS\_LARGE\_WOB}\sphinxbfcode{ = \{'text-anchor': 'start', 'font-weight': 'bold', 'fill': 'white'\}}}
SVG attributes for a large WOB text

\end{fulllineitems}

\index{HEADER\_PLOT\_UNITS (in module TotalDepth.util.plot.LogHeader)}

\begin{fulllineitems}
\phantomsection\label{\detokenize{ref/util/plot/LogHeader:TotalDepth.util.plot.LogHeader.HEADER_PLOT_UNITS}}\pysigline{\sphinxcode{TotalDepth.util.plot.LogHeader.}\sphinxbfcode{HEADER\_PLOT\_UNITS}\sphinxbfcode{ = 'in'}}
Standard plot units used in the layout definition

\end{fulllineitems}

\index{APIHeaderBase (class in TotalDepth.util.plot.LogHeader)}

\begin{fulllineitems}
\phantomsection\label{\detokenize{ref/util/plot/LogHeader:TotalDepth.util.plot.LogHeader.APIHeaderBase}}\pysiglinewithargsret{\sphinxbfcode{class }\sphinxcode{TotalDepth.util.plot.LogHeader.}\sphinxbfcode{APIHeaderBase}}{\emph{isTopOfLog=False}}{}
Base class to be used by APIHeaderLIS or APIHeaderLAS.

If isTopOfLog is True plot is rotated 90 deg as if to fit on top of a traditional log.
\index{missingFields() (TotalDepth.util.plot.LogHeader.APIHeaderBase method)}

\begin{fulllineitems}
\phantomsection\label{\detokenize{ref/util/plot/LogHeader:TotalDepth.util.plot.LogHeader.APIHeaderBase.missingFields}}\pysiglinewithargsret{\sphinxbfcode{missingFields}}{\emph{theWsd}}{}
Returns two sets:
A set of mnemonics that could be plotted but are not in the Logical Record(s).
A set of mnemonics that are in the Logical Record(s) but could not be plotted.

\end{fulllineitems}

\index{size() (TotalDepth.util.plot.LogHeader.APIHeaderBase method)}

\begin{fulllineitems}
\phantomsection\label{\detokenize{ref/util/plot/LogHeader:TotalDepth.util.plot.LogHeader.APIHeaderBase.size}}\pysiglinewithargsret{\sphinxbfcode{size}}{}{}
Returns a Coord.Box for my size, currently a single page on fan folded paper.

\end{fulllineitems}

\index{viewPort() (TotalDepth.util.plot.LogHeader.APIHeaderBase method)}

\begin{fulllineitems}
\phantomsection\label{\detokenize{ref/util/plot/LogHeader:TotalDepth.util.plot.LogHeader.APIHeaderBase.viewPort}}\pysiglinewithargsret{\sphinxbfcode{viewPort}}{\emph{theTl}}{}
The SVG viewport.

\end{fulllineitems}

\index{plot() (TotalDepth.util.plot.LogHeader.APIHeaderBase method)}

\begin{fulllineitems}
\phantomsection\label{\detokenize{ref/util/plot/LogHeader:TotalDepth.util.plot.LogHeader.APIHeaderBase.plot}}\pysiglinewithargsret{\sphinxbfcode{plot}}{\emph{xS}, \emph{theTl}, \emph{theWsdS=None}}{}
Write the header to the SVG stream at position offset top left.
theWsd is a list of records that contain well site data.
Will raise ExceptionLogHeader is wrong type of Logical Record.

\end{fulllineitems}


\end{fulllineitems}

\index{APIHeaderLIS (class in TotalDepth.util.plot.LogHeader)}

\begin{fulllineitems}
\phantomsection\label{\detokenize{ref/util/plot/LogHeader:TotalDepth.util.plot.LogHeader.APIHeaderLIS}}\pysiglinewithargsret{\sphinxbfcode{class }\sphinxcode{TotalDepth.util.plot.LogHeader.}\sphinxbfcode{APIHeaderLIS}}{\emph{isTopOfLog=False}}{}
Can lay out an API header from LIS information, specifically a type 34 CONS record.

If isTopOfLog is True plot is rotated 90 deg as if to fit on top of a traditional log.
\index{missingFields() (TotalDepth.util.plot.LogHeader.APIHeaderLIS method)}

\begin{fulllineitems}
\phantomsection\label{\detokenize{ref/util/plot/LogHeader:TotalDepth.util.plot.LogHeader.APIHeaderLIS.missingFields}}\pysiglinewithargsret{\sphinxbfcode{missingFields}}{\emph{theWsd}}{}
Well site data (theWsd) in LIS is a list of CONS Logical Records.

Returns two sets:

A set of mnemonics that could be plotted but are not in the Logical Record(s).

A set of mnemonics that are in the Logical Record(s) but could not be plotted.

\end{fulllineitems}

\index{lrDataCount() (TotalDepth.util.plot.LogHeader.APIHeaderLIS method)}

\begin{fulllineitems}
\phantomsection\label{\detokenize{ref/util/plot/LogHeader:TotalDepth.util.plot.LogHeader.APIHeaderLIS.lrDataCount}}\pysiglinewithargsret{\sphinxbfcode{lrDataCount}}{\emph{theLrS}}{}
Returns the number of Mnem’s in MNEM\_SET that could be plotted in a
header that are found in all the Logical Records.

\end{fulllineitems}


\end{fulllineitems}

\index{APIHeaderLAS (class in TotalDepth.util.plot.LogHeader)}

\begin{fulllineitems}
\phantomsection\label{\detokenize{ref/util/plot/LogHeader:TotalDepth.util.plot.LogHeader.APIHeaderLAS}}\pysiglinewithargsret{\sphinxbfcode{class }\sphinxcode{TotalDepth.util.plot.LogHeader.}\sphinxbfcode{APIHeaderLAS}}{\emph{isTopOfLog=False}}{}
Can lay out an API header from a representation of a LAS file.
\index{LIS\_MNEM\_TO\_LAS\_MNEM (TotalDepth.util.plot.LogHeader.APIHeaderLAS attribute)}

\begin{fulllineitems}
\phantomsection\label{\detokenize{ref/util/plot/LogHeader:TotalDepth.util.plot.LogHeader.APIHeaderLAS.LIS_MNEM_TO_LAS_MNEM}}\pysigline{\sphinxbfcode{LIS\_MNEM\_TO\_LAS\_MNEM}\sphinxbfcode{ = \{'BLI': 'STRT', 'TLI': 'STOP', 'CN': 'COMP', 'WN': 'WELL', 'FN': 'FLD', 'FL': 'LOC', 'FL1': 'PROV', 'FL2': 'UWI', 'LUL': 'SRVC', 'MRT': 'BHT', 'DFD': 'FD', 'LATI': 'LAT', 'LONG': 'LON', 'TDL': 'DL'\}}}
Some conversions from LIS standard to LAS standard
All as truncated ascii strings i.e. using pStr(strip=True)

\end{fulllineitems}

\index{missingFields() (TotalDepth.util.plot.LogHeader.APIHeaderLAS method)}

\begin{fulllineitems}
\phantomsection\label{\detokenize{ref/util/plot/LogHeader:TotalDepth.util.plot.LogHeader.APIHeaderLAS.missingFields}}\pysiglinewithargsret{\sphinxbfcode{missingFields}}{\emph{theLasFile}}{}
Returns two sets:

A set of mnemonics that could be plotted but are not in the Logical Record(s).

A set of mnemonics that are in the Logical Record(s) but could not be plotted.

\end{fulllineitems}


\end{fulllineitems}



\paragraph{Plotting Well Logs}
\label{\detokenize{ref/util/plot/Plot:module-TotalDepth.util.plot.Plot}}\label{\detokenize{ref/util/plot/Plot:plotting-well-logs}}\label{\detokenize{ref/util/plot/Plot::doc}}\index{TotalDepth.util.plot.Plot (module)}
Created on 28 Feb 2011

Plotting LIS data requires these components:
\begin{itemize}
\item {} 
A PlotConfig object

\item {} 
A data set (e.g. a LogPass with a FrameSet).

\item {} 
An output driver (e.g. screen, print PDF, web SVG).

\end{itemize}

User creates a PlotConfig (this reflects a PRES table). This is reusable.
\begin{description}
\item[{User specifies a data set (from, to, channels etc.).}] \leavevmode
e.g. Invoke LogPass.setFrameSet(File, theFrameSlice=theFrameSlice, theChList=theChList)

\item[{User says ‘plot this data set with this configuration to this output device’.}] \leavevmode
e.g. Plot(PlotConfig, LogPass, PlotDevice)
Plot uses LogPass.genChScValues(ch, sc) to plot individual curves.

\end{description}


\subparagraph{Lacunae}
\label{\detokenize{ref/util/plot/Plot:lacunae}}
Area plotting.
Caching (e.g. SVG fragments - is this worth it?)


\subparagraph{PlotConfig}
\label{\detokenize{ref/util/plot/Plot:plotconfig}}

\subparagraph{PlotTracks}
\label{\detokenize{ref/util/plot/Plot:plottracks}}
Typically a three track (+depth) have these dimensions in inches:


\begin{savenotes}\sphinxattablestart
\centering
\begin{tabulary}{\linewidth}[t]{|T|T|T|T|}
\hline
\sphinxstylethead{\sphinxstyletheadfamily 
Track
\unskip}\relax &\sphinxstylethead{\sphinxstyletheadfamily 
Left
\unskip}\relax &\sphinxstylethead{\sphinxstyletheadfamily 
Right
\unskip}\relax &\sphinxstylethead{\sphinxstyletheadfamily 
Width
\unskip}\relax \\
\hline
1
&
0
&
2.4
&
2.4
\\
\hline
Depth
&
2.4
&
3.2
&
0.8
\\
\hline
2
&
3.2
&
5.6
&
2.4
\\
\hline
3
&
5.6
&
8.0
&
2.4
\\
\hline
\end{tabulary}
\par
\sphinxattableend\end{savenotes}

Track names can be split (e.g. LHT1 is left hand track 1) or merged (T23 is
spread across tracks two and three).

Examples of PRES and FILM records:

\begin{sphinxVerbatim}[commandchars=\\\{\}]
 \PYG{n}{Table} \PYG{n}{record} \PYG{p}{(}\PYG{n+nb}{type} \PYG{l+m+mi}{34}\PYG{p}{)} \PYG{n+nb}{type}\PYG{p}{:} \PYG{n}{PRES}

\PYG{n}{MNEM}  \PYG{n}{OUTP}  \PYG{n}{STAT}  \PYG{n}{TRAC}  \PYG{n}{CODI}  \PYG{n}{DEST}  \PYG{n}{MODE}      \PYG{n}{FILT}          \PYG{n}{LEDG}          \PYG{n}{REDG}
\PYG{o}{\PYGZhy{}}\PYG{o}{\PYGZhy{}}\PYG{o}{\PYGZhy{}}\PYG{o}{\PYGZhy{}}\PYG{o}{\PYGZhy{}}\PYG{o}{\PYGZhy{}}\PYG{o}{\PYGZhy{}}\PYG{o}{\PYGZhy{}}\PYG{o}{\PYGZhy{}}\PYG{o}{\PYGZhy{}}\PYG{o}{\PYGZhy{}}\PYG{o}{\PYGZhy{}}\PYG{o}{\PYGZhy{}}\PYG{o}{\PYGZhy{}}\PYG{o}{\PYGZhy{}}\PYG{o}{\PYGZhy{}}\PYG{o}{\PYGZhy{}}\PYG{o}{\PYGZhy{}}\PYG{o}{\PYGZhy{}}\PYG{o}{\PYGZhy{}}\PYG{o}{\PYGZhy{}}\PYG{o}{\PYGZhy{}}\PYG{o}{\PYGZhy{}}\PYG{o}{\PYGZhy{}}\PYG{o}{\PYGZhy{}}\PYG{o}{\PYGZhy{}}\PYG{o}{\PYGZhy{}}\PYG{o}{\PYGZhy{}}\PYG{o}{\PYGZhy{}}\PYG{o}{\PYGZhy{}}\PYG{o}{\PYGZhy{}}\PYG{o}{\PYGZhy{}}\PYG{o}{\PYGZhy{}}\PYG{o}{\PYGZhy{}}\PYG{o}{\PYGZhy{}}\PYG{o}{\PYGZhy{}}\PYG{o}{\PYGZhy{}}\PYG{o}{\PYGZhy{}}\PYG{o}{\PYGZhy{}}\PYG{o}{\PYGZhy{}}\PYG{o}{\PYGZhy{}}\PYG{o}{\PYGZhy{}}\PYG{o}{\PYGZhy{}}\PYG{o}{\PYGZhy{}}\PYG{o}{\PYGZhy{}}\PYG{o}{\PYGZhy{}}\PYG{o}{\PYGZhy{}}\PYG{o}{\PYGZhy{}}\PYG{o}{\PYGZhy{}}\PYG{o}{\PYGZhy{}}\PYG{o}{\PYGZhy{}}\PYG{o}{\PYGZhy{}}\PYG{o}{\PYGZhy{}}\PYG{o}{\PYGZhy{}}\PYG{o}{\PYGZhy{}}\PYG{o}{\PYGZhy{}}\PYG{o}{\PYGZhy{}}\PYG{o}{\PYGZhy{}}\PYG{o}{\PYGZhy{}}\PYG{o}{\PYGZhy{}}\PYG{o}{\PYGZhy{}}\PYG{o}{\PYGZhy{}}\PYG{o}{\PYGZhy{}}\PYG{o}{\PYGZhy{}}\PYG{o}{\PYGZhy{}}\PYG{o}{\PYGZhy{}}\PYG{o}{\PYGZhy{}}\PYG{o}{\PYGZhy{}}\PYG{o}{\PYGZhy{}}\PYG{o}{\PYGZhy{}}\PYG{o}{\PYGZhy{}}\PYG{o}{\PYGZhy{}}\PYG{o}{\PYGZhy{}}\PYG{o}{\PYGZhy{}}\PYG{o}{\PYGZhy{}}\PYG{o}{\PYGZhy{}}\PYG{o}{\PYGZhy{}}\PYG{o}{\PYGZhy{}}\PYG{o}{\PYGZhy{}}\PYG{o}{\PYGZhy{}}\PYG{o}{\PYGZhy{}}\PYG{o}{\PYGZhy{}}\PYG{o}{\PYGZhy{}}

\PYG{n}{SP}    \PYG{n}{SP}    \PYG{n}{ALLO}  \PYG{n}{T1}    \PYG{n}{LLIN}  \PYG{l+m+mi}{1}     \PYG{n}{SHIF}      \PYG{l+m+mf}{0.500000}      \PYG{o}{\PYGZhy{}}\PYG{l+m+mf}{80.0000}       \PYG{l+m+mf}{20.0000}
\PYG{n}{CALI}  \PYG{n}{CALI}  \PYG{n}{ALLO}  \PYG{n}{T1}    \PYG{n}{LDAS}  \PYG{l+m+mi}{1}     \PYG{n}{SHIF}      \PYG{l+m+mf}{0.500000}       \PYG{l+m+mf}{5.00000}       \PYG{l+m+mf}{15.0000}
\PYG{n}{MINV}  \PYG{n}{MINV}  \PYG{n}{DISA}  \PYG{n}{T1}    \PYG{n}{LLIN}  \PYG{l+m+mi}{1}     \PYG{n}{SHIF}      \PYG{l+m+mf}{0.500000}       \PYG{l+m+mf}{30.0000}       \PYG{l+m+mf}{0.00000}
\PYG{n}{MNOR}  \PYG{n}{MNOR}  \PYG{n}{DISA}  \PYG{n}{T1}    \PYG{n}{LDAS}  \PYG{l+m+mi}{1}     \PYG{n}{SHIF}      \PYG{l+m+mf}{0.500000}       \PYG{l+m+mf}{30.0000}       \PYG{l+m+mf}{0.00000}
\PYG{n}{LLD}   \PYG{n}{LLD}   \PYG{n}{ALLO}  \PYG{n}{T23}   \PYG{n}{LDAS}  \PYG{l+m+mi}{1}     \PYG{n}{GRAD}      \PYG{l+m+mf}{0.500000}      \PYG{l+m+mf}{0.200000}       \PYG{l+m+mf}{2000.00}
\PYG{n}{LLDB}  \PYG{n}{LLD}   \PYG{n}{ALLO}  \PYG{n}{T2}    \PYG{n}{HDAS}  \PYG{l+m+mi}{1}     \PYG{n}{GRAD}      \PYG{l+m+mf}{0.500000}       \PYG{l+m+mf}{2000.00}       \PYG{l+m+mf}{200000.}
\PYG{n}{LLG}   \PYG{n}{LLG}   \PYG{n}{DISA}  \PYG{n}{T23}   \PYG{n}{LDAS}  \PYG{l+m+mi}{1}     \PYG{n}{GRAD}      \PYG{l+m+mf}{0.500000}      \PYG{l+m+mf}{0.200000}       \PYG{l+m+mf}{2000.00}
\PYG{n}{LLGB}  \PYG{n}{LLG}   \PYG{n}{DISA}  \PYG{n}{T2}    \PYG{n}{HDAS}  \PYG{l+m+mi}{1}     \PYG{n}{GRAD}      \PYG{l+m+mf}{0.500000}       \PYG{l+m+mf}{2000.00}       \PYG{l+m+mf}{200000.}
\PYG{n}{LLS}   \PYG{n}{LLS}   \PYG{n}{ALLO}  \PYG{n}{T23}   \PYG{n}{LSPO}  \PYG{l+m+mi}{1}     \PYG{n}{GRAD}      \PYG{l+m+mf}{0.500000}      \PYG{l+m+mf}{0.200000}       \PYG{l+m+mf}{2000.00}
\PYG{n}{LLSB}  \PYG{n}{LLS}   \PYG{n}{ALLO}  \PYG{n}{T2}    \PYG{n}{HSPO}  \PYG{l+m+mi}{1}     \PYG{n}{GRAD}      \PYG{l+m+mf}{0.500000}       \PYG{l+m+mf}{2000.00}       \PYG{l+m+mf}{200000.}
\PYG{n}{MSFL}  \PYG{n}{MSFL}  \PYG{n}{ALLO}  \PYG{n}{T23}   \PYG{n}{LLIN}  \PYG{l+m+mi}{1}     \PYG{n}{GRAD}      \PYG{l+m+mf}{0.500000}      \PYG{l+m+mf}{0.200000}       \PYG{l+m+mf}{2000.00}

\PYG{n}{Table} \PYG{n}{record} \PYG{p}{(}\PYG{n+nb}{type} \PYG{l+m+mi}{34}\PYG{p}{)} \PYG{n+nb}{type}\PYG{p}{:} \PYG{n}{FILM}

\PYG{n}{MNEM}  \PYG{n}{GCOD}  \PYG{n}{GDEC}  \PYG{n}{DEST}  \PYG{n}{DSCA}
\PYG{o}{\PYGZhy{}}\PYG{o}{\PYGZhy{}}\PYG{o}{\PYGZhy{}}\PYG{o}{\PYGZhy{}}\PYG{o}{\PYGZhy{}}\PYG{o}{\PYGZhy{}}\PYG{o}{\PYGZhy{}}\PYG{o}{\PYGZhy{}}\PYG{o}{\PYGZhy{}}\PYG{o}{\PYGZhy{}}\PYG{o}{\PYGZhy{}}\PYG{o}{\PYGZhy{}}\PYG{o}{\PYGZhy{}}\PYG{o}{\PYGZhy{}}\PYG{o}{\PYGZhy{}}\PYG{o}{\PYGZhy{}}\PYG{o}{\PYGZhy{}}\PYG{o}{\PYGZhy{}}\PYG{o}{\PYGZhy{}}\PYG{o}{\PYGZhy{}}\PYG{o}{\PYGZhy{}}\PYG{o}{\PYGZhy{}}\PYG{o}{\PYGZhy{}}\PYG{o}{\PYGZhy{}}\PYG{o}{\PYGZhy{}}\PYG{o}{\PYGZhy{}}\PYG{o}{\PYGZhy{}}\PYG{o}{\PYGZhy{}}\PYG{o}{\PYGZhy{}}

\PYG{l+m+mi}{1}     \PYG{n}{E20}   \PYG{o}{\PYGZhy{}}\PYG{l+m+mi}{4}\PYG{o}{\PYGZhy{}}\PYG{o}{\PYGZhy{}}  \PYG{n}{PF1}   \PYG{n}{D200}
\PYG{l+m+mi}{2}     \PYG{n}{EEE}   \PYG{o}{\PYGZhy{}}\PYG{o}{\PYGZhy{}}\PYG{o}{\PYGZhy{}}\PYG{o}{\PYGZhy{}}  \PYG{n}{PF2}   \PYG{n}{D200} 

\PYG{n}{Table} \PYG{n}{record} \PYG{p}{(}\PYG{n+nb}{type} \PYG{l+m+mi}{34}\PYG{p}{)} \PYG{n+nb}{type}\PYG{p}{:} \PYG{n}{PRES}

\PYG{n}{MNEM}  \PYG{n}{OUTP}  \PYG{n}{STAT}  \PYG{n}{TRAC}  \PYG{n}{CODI}  \PYG{n}{DEST}  \PYG{n}{MODE}      \PYG{n}{FILT}          \PYG{n}{LEDG}          \PYG{n}{REDG}
\PYG{o}{\PYGZhy{}}\PYG{o}{\PYGZhy{}}\PYG{o}{\PYGZhy{}}\PYG{o}{\PYGZhy{}}\PYG{o}{\PYGZhy{}}\PYG{o}{\PYGZhy{}}\PYG{o}{\PYGZhy{}}\PYG{o}{\PYGZhy{}}\PYG{o}{\PYGZhy{}}\PYG{o}{\PYGZhy{}}\PYG{o}{\PYGZhy{}}\PYG{o}{\PYGZhy{}}\PYG{o}{\PYGZhy{}}\PYG{o}{\PYGZhy{}}\PYG{o}{\PYGZhy{}}\PYG{o}{\PYGZhy{}}\PYG{o}{\PYGZhy{}}\PYG{o}{\PYGZhy{}}\PYG{o}{\PYGZhy{}}\PYG{o}{\PYGZhy{}}\PYG{o}{\PYGZhy{}}\PYG{o}{\PYGZhy{}}\PYG{o}{\PYGZhy{}}\PYG{o}{\PYGZhy{}}\PYG{o}{\PYGZhy{}}\PYG{o}{\PYGZhy{}}\PYG{o}{\PYGZhy{}}\PYG{o}{\PYGZhy{}}\PYG{o}{\PYGZhy{}}\PYG{o}{\PYGZhy{}}\PYG{o}{\PYGZhy{}}\PYG{o}{\PYGZhy{}}\PYG{o}{\PYGZhy{}}\PYG{o}{\PYGZhy{}}\PYG{o}{\PYGZhy{}}\PYG{o}{\PYGZhy{}}\PYG{o}{\PYGZhy{}}\PYG{o}{\PYGZhy{}}\PYG{o}{\PYGZhy{}}\PYG{o}{\PYGZhy{}}\PYG{o}{\PYGZhy{}}\PYG{o}{\PYGZhy{}}\PYG{o}{\PYGZhy{}}\PYG{o}{\PYGZhy{}}\PYG{o}{\PYGZhy{}}\PYG{o}{\PYGZhy{}}\PYG{o}{\PYGZhy{}}\PYG{o}{\PYGZhy{}}\PYG{o}{\PYGZhy{}}\PYG{o}{\PYGZhy{}}\PYG{o}{\PYGZhy{}}\PYG{o}{\PYGZhy{}}\PYG{o}{\PYGZhy{}}\PYG{o}{\PYGZhy{}}\PYG{o}{\PYGZhy{}}\PYG{o}{\PYGZhy{}}\PYG{o}{\PYGZhy{}}\PYG{o}{\PYGZhy{}}\PYG{o}{\PYGZhy{}}\PYG{o}{\PYGZhy{}}\PYG{o}{\PYGZhy{}}\PYG{o}{\PYGZhy{}}\PYG{o}{\PYGZhy{}}\PYG{o}{\PYGZhy{}}\PYG{o}{\PYGZhy{}}\PYG{o}{\PYGZhy{}}\PYG{o}{\PYGZhy{}}\PYG{o}{\PYGZhy{}}\PYG{o}{\PYGZhy{}}\PYG{o}{\PYGZhy{}}\PYG{o}{\PYGZhy{}}\PYG{o}{\PYGZhy{}}\PYG{o}{\PYGZhy{}}\PYG{o}{\PYGZhy{}}\PYG{o}{\PYGZhy{}}\PYG{o}{\PYGZhy{}}\PYG{o}{\PYGZhy{}}\PYG{o}{\PYGZhy{}}\PYG{o}{\PYGZhy{}}\PYG{o}{\PYGZhy{}}\PYG{o}{\PYGZhy{}}\PYG{o}{\PYGZhy{}}\PYG{o}{\PYGZhy{}}

\PYG{n}{NPHI}  \PYG{n}{NPHI}  \PYG{n}{ALLO}  \PYG{n}{T23}   \PYG{n}{LDAS}  \PYG{l+m+mi}{1}     \PYG{n}{SHIF}      \PYG{l+m+mf}{0.500000}      \PYG{l+m+mf}{0.450000}     \PYG{o}{\PYGZhy{}}\PYG{l+m+mf}{0.150000}
\PYG{n}{DRHO}  \PYG{n}{DRHO}  \PYG{n}{ALLO}  \PYG{n}{T3}    \PYG{n}{LSPO}  \PYG{l+m+mi}{1}     \PYG{n}{NB}        \PYG{l+m+mf}{0.500000}     \PYG{o}{\PYGZhy{}}\PYG{l+m+mf}{0.250000}      \PYG{l+m+mf}{0.250000}
\PYG{n}{PEF}   \PYG{n}{PEF}   \PYG{n}{ALLO}  \PYG{n}{T23}   \PYG{n}{LGAP}  \PYG{l+m+mi}{1}     \PYG{n}{SHIF}      \PYG{l+m+mf}{0.500000}       \PYG{l+m+mf}{0.00000}       \PYG{l+m+mf}{10.0000}
\PYG{n}{SGR}         \PYG{n}{DISA}  \PYG{n}{T1}    \PYG{n}{LLIN}  \PYG{l+m+mi}{1}     \PYG{n}{SHIF}      \PYG{l+m+mf}{0.500000}       \PYG{l+m+mf}{0.00000}       \PYG{l+m+mf}{300.000}
\PYG{n}{CGR}         \PYG{n}{DISA}  \PYG{n}{T1}    \PYG{n}{LGAP}  \PYG{l+m+mi}{1}     \PYG{n}{SHIF}      \PYG{l+m+mf}{0.500000}       \PYG{l+m+mf}{0.00000}       \PYG{l+m+mf}{300.000}
\PYG{n}{TENS}  \PYG{n}{TENS}  \PYG{n}{DISA}  \PYG{n}{T3}    \PYG{n}{LGAP}  \PYG{l+m+mi}{1}     \PYG{n}{SHIF}      \PYG{l+m+mf}{0.500000}       \PYG{l+m+mf}{14000.0}       \PYG{l+m+mf}{4000.00}
\PYG{n}{CAL}   \PYG{n}{CALI}  \PYG{n}{ALLO}  \PYG{n}{T1}    \PYG{n}{LSPO}  \PYG{l+m+mi}{1}     \PYG{n}{SHIF}      \PYG{l+m+mf}{0.500000}       \PYG{l+m+mf}{5.00000}       \PYG{l+m+mf}{15.0000}
\PYG{n}{BS}    \PYG{n}{BS}    \PYG{n}{DISA}  \PYG{n}{T1}    \PYG{n}{LGAP}  \PYG{l+m+mi}{1}     \PYG{n}{SHIF}      \PYG{l+m+mf}{0.500000}       \PYG{l+m+mf}{5.00000}       \PYG{l+m+mf}{15.0000}
\PYG{n}{FFLS}  \PYG{n}{FFLS}  \PYG{n}{DISA}  \PYG{n}{T1}    \PYG{n}{LLIN}  \PYG{l+m+mi}{2}     \PYG{n}{NB}        \PYG{l+m+mf}{0.500000}     \PYG{o}{\PYGZhy{}}\PYG{l+m+mf}{0.150000}      \PYG{l+m+mf}{0.150000}
\PYG{n}{FFSS}  \PYG{n}{FFSS}  \PYG{n}{DISA}  \PYG{n}{T1}    \PYG{n}{LDAS}  \PYG{l+m+mi}{2}     \PYG{n}{NB}        \PYG{l+m+mf}{0.500000}     \PYG{o}{\PYGZhy{}}\PYG{l+m+mf}{0.150000}      \PYG{l+m+mf}{0.150000}
\PYG{n}{LSHV}  \PYG{n}{LSHV}  \PYG{n}{DISA}  \PYG{n}{T3}    \PYG{n}{LLIN}  \PYG{l+m+mi}{2}     \PYG{n}{WRAP}      \PYG{l+m+mf}{0.500000}       \PYG{l+m+mf}{2150.00}       \PYG{l+m+mf}{2250.00}
\PYG{n}{SSHV}  \PYG{n}{SSHV}  \PYG{n}{DISA}  \PYG{n}{T3}    \PYG{n}{LDAS}  \PYG{l+m+mi}{2}     \PYG{n}{WRAP}      \PYG{l+m+mf}{0.500000}       \PYG{l+m+mf}{1950.00}       \PYG{l+m+mf}{2050.00}
\PYG{n}{FLS}   \PYG{n}{FLS}   \PYG{n}{DISA}  \PYG{n}{T2}    \PYG{n}{LLIN}  \PYG{l+m+mi}{2}     \PYG{n}{SHIF}      \PYG{l+m+mf}{0.500000}       \PYG{l+m+mf}{0.00000}       \PYG{l+m+mf}{150.000}
\PYG{n}{FSS}   \PYG{n}{FSS}   \PYG{n}{DISA}  \PYG{n}{T2}    \PYG{n}{LDAS}  \PYG{l+m+mi}{2}     \PYG{n}{SHIF}      \PYG{l+m+mf}{0.500000}       \PYG{l+m+mf}{0.00000}       \PYG{l+m+mf}{150.000}
\PYG{n}{RHOB}  \PYG{n}{RHOB}  \PYG{n}{ALLO}  \PYG{n}{T23}   \PYG{n}{LLIN}  \PYG{l+m+mi}{1}     \PYG{n}{SHIF}      \PYG{l+m+mf}{0.500000}       \PYG{l+m+mf}{1.95000}       \PYG{l+m+mf}{2.95000}
\PYG{n}{PHIX}  \PYG{n}{PHIX}  \PYG{n}{ALLO}  \PYG{n}{T1}    \PYG{n}{LLIN}  \PYG{l+m+mi}{1}     \PYG{n}{NB}        \PYG{l+m+mf}{0.500000}      \PYG{l+m+mf}{0.500000}       \PYG{l+m+mf}{0.00000}
\end{sphinxVerbatim}
\index{COMMENTS\_IN\_SVG\_TRACE (in module TotalDepth.util.plot.Plot)}

\begin{fulllineitems}
\phantomsection\label{\detokenize{ref/util/plot/Plot:TotalDepth.util.plot.Plot.COMMENTS_IN_SVG_TRACE}}\pysigline{\sphinxcode{TotalDepth.util.plot.Plot.}\sphinxbfcode{COMMENTS\_IN\_SVG\_TRACE}\sphinxbfcode{ = False}}
Allows detailed trace comments to appear in the SVG

\end{fulllineitems}

\index{COMMENTS\_IN\_SVG\_SECTION (in module TotalDepth.util.plot.Plot)}

\begin{fulllineitems}
\phantomsection\label{\detokenize{ref/util/plot/Plot:TotalDepth.util.plot.Plot.COMMENTS_IN_SVG_SECTION}}\pysigline{\sphinxcode{TotalDepth.util.plot.Plot.}\sphinxbfcode{COMMENTS\_IN\_SVG\_SECTION}\sphinxbfcode{ = True}}
A few comments in SVG to delinialte header, Grid, Xaxis, curves etc

\end{fulllineitems}

\index{COMMENTS\_IN\_SVG\_SECTION\_WIDTH (in module TotalDepth.util.plot.Plot)}

\begin{fulllineitems}
\phantomsection\label{\detokenize{ref/util/plot/Plot:TotalDepth.util.plot.Plot.COMMENTS_IN_SVG_SECTION_WIDTH}}\pysigline{\sphinxcode{TotalDepth.util.plot.Plot.}\sphinxbfcode{COMMENTS\_IN\_SVG\_SECTION\_WIDTH}\sphinxbfcode{ = 40}}
Width of comment for sections

\end{fulllineitems}

\index{COMMENTS\_IN\_SVG\_SECTION\_LEVEL\_TUPLE (in module TotalDepth.util.plot.Plot)}

\begin{fulllineitems}
\phantomsection\label{\detokenize{ref/util/plot/Plot:TotalDepth.util.plot.Plot.COMMENTS_IN_SVG_SECTION_LEVEL_TUPLE}}\pysigline{\sphinxcode{TotalDepth.util.plot.Plot.}\sphinxbfcode{COMMENTS\_IN\_SVG\_SECTION\_LEVEL\_TUPLE}\sphinxbfcode{ = ('=', '.', '\textasciicircum{}')}}
Map of section level to comment padding character

\end{fulllineitems}

\index{ExceptionTotalDepthLISPlot}

\begin{fulllineitems}
\phantomsection\label{\detokenize{ref/util/plot/Plot:TotalDepth.util.plot.Plot.ExceptionTotalDepthLISPlot}}\pysigline{\sphinxbfcode{exception }\sphinxcode{TotalDepth.util.plot.Plot.}\sphinxbfcode{ExceptionTotalDepthLISPlot}}
Exception for plotting.

\end{fulllineitems}

\index{ExceptionTotalDepthPlotRoll}

\begin{fulllineitems}
\phantomsection\label{\detokenize{ref/util/plot/Plot:TotalDepth.util.plot.Plot.ExceptionTotalDepthPlotRoll}}\pysigline{\sphinxbfcode{exception }\sphinxcode{TotalDepth.util.plot.Plot.}\sphinxbfcode{ExceptionTotalDepthPlotRoll}}
Exception for plotting.

\end{fulllineitems}

\index{CurvePlotScale (class in TotalDepth.util.plot.Plot)}

\begin{fulllineitems}
\phantomsection\label{\detokenize{ref/util/plot/Plot:TotalDepth.util.plot.Plot.CurvePlotScale}}\pysigline{\sphinxbfcode{class }\sphinxcode{TotalDepth.util.plot.Plot.}\sphinxbfcode{CurvePlotScale}}
Holds a minimal amount of curve plot scale and so on for the layout of
the scale pane at each end of the log.

halfTrackStart is the start of the curve for a standard three track log
T1=0, TD=2, T2=4, T3=6

halfTracks is the curve span as an integer so T23=4 LHT1=1 and so on.
\index{\_\_lt\_\_() (TotalDepth.util.plot.Plot.CurvePlotScale method)}

\begin{fulllineitems}
\phantomsection\label{\detokenize{ref/util/plot/Plot:TotalDepth.util.plot.Plot.CurvePlotScale.__lt__}}\pysiglinewithargsret{\sphinxbfcode{\_\_lt\_\_}}{\emph{other}}{}
Slightly weird sort order, larger halfTracks come first then names.

\end{fulllineitems}


\end{fulllineitems}

\index{ScaleSliceCurve (class in TotalDepth.util.plot.Plot)}

\begin{fulllineitems}
\phantomsection\label{\detokenize{ref/util/plot/Plot:TotalDepth.util.plot.Plot.ScaleSliceCurve}}\pysiglinewithargsret{\sphinxbfcode{class }\sphinxcode{TotalDepth.util.plot.Plot.}\sphinxbfcode{ScaleSliceCurve}}{\emph{slice}, \emph{curveName}, \emph{start}, \emph{span}}{}~\index{\_\_getnewargs\_\_() (TotalDepth.util.plot.Plot.ScaleSliceCurve method)}

\begin{fulllineitems}
\phantomsection\label{\detokenize{ref/util/plot/Plot:TotalDepth.util.plot.Plot.ScaleSliceCurve.__getnewargs__}}\pysiglinewithargsret{\sphinxbfcode{\_\_getnewargs\_\_}}{}{}
Return self as a plain tuple.  Used by copy and pickle.

\end{fulllineitems}

\index{\_\_new\_\_() (TotalDepth.util.plot.Plot.ScaleSliceCurve static method)}

\begin{fulllineitems}
\phantomsection\label{\detokenize{ref/util/plot/Plot:TotalDepth.util.plot.Plot.ScaleSliceCurve.__new__}}\pysiglinewithargsret{\sphinxbfcode{static }\sphinxbfcode{\_\_new\_\_}}{\emph{\_cls}, \emph{slice}, \emph{curveName}, \emph{start}, \emph{span}}{}
Create new instance of ScaleSliceCurve(slice, curveName, start, span)

\end{fulllineitems}

\index{\_\_repr\_\_() (TotalDepth.util.plot.Plot.ScaleSliceCurve method)}

\begin{fulllineitems}
\phantomsection\label{\detokenize{ref/util/plot/Plot:TotalDepth.util.plot.Plot.ScaleSliceCurve.__repr__}}\pysiglinewithargsret{\sphinxbfcode{\_\_repr\_\_}}{}{}
Return a nicely formatted representation string

\end{fulllineitems}

\index{curveName (TotalDepth.util.plot.Plot.ScaleSliceCurve attribute)}

\begin{fulllineitems}
\phantomsection\label{\detokenize{ref/util/plot/Plot:TotalDepth.util.plot.Plot.ScaleSliceCurve.curveName}}\pysigline{\sphinxbfcode{curveName}}
Alias for field number 1

\end{fulllineitems}

\index{slice (TotalDepth.util.plot.Plot.ScaleSliceCurve attribute)}

\begin{fulllineitems}
\phantomsection\label{\detokenize{ref/util/plot/Plot:TotalDepth.util.plot.Plot.ScaleSliceCurve.slice}}\pysigline{\sphinxbfcode{slice}}
Alias for field number 0

\end{fulllineitems}

\index{span (TotalDepth.util.plot.Plot.ScaleSliceCurve attribute)}

\begin{fulllineitems}
\phantomsection\label{\detokenize{ref/util/plot/Plot:TotalDepth.util.plot.Plot.ScaleSliceCurve.span}}\pysigline{\sphinxbfcode{span}}
Alias for field number 3

\end{fulllineitems}

\index{start (TotalDepth.util.plot.Plot.ScaleSliceCurve attribute)}

\begin{fulllineitems}
\phantomsection\label{\detokenize{ref/util/plot/Plot:TotalDepth.util.plot.Plot.ScaleSliceCurve.start}}\pysigline{\sphinxbfcode{start}}
Alias for field number 2

\end{fulllineitems}


\end{fulllineitems}

\index{CurvePlotScaleSlotMap (class in TotalDepth.util.plot.Plot)}

\begin{fulllineitems}
\phantomsection\label{\detokenize{ref/util/plot/Plot:TotalDepth.util.plot.Plot.CurvePlotScaleSlotMap}}\pysiglinewithargsret{\sphinxbfcode{class }\sphinxcode{TotalDepth.util.plot.Plot.}\sphinxbfcode{CurvePlotScaleSlotMap}}{\emph{theCpsS}}{}
Keeps track of which slots are available for putting the curve scales in
at the top and bottom of the log.
This scale are is divided into slices that span the plot from left to right.
These slices are subdivided into slots that correspond to a half-track in
that slice.
\index{\_\_init\_\_() (TotalDepth.util.plot.Plot.CurvePlotScaleSlotMap method)}

\begin{fulllineitems}
\phantomsection\label{\detokenize{ref/util/plot/Plot:TotalDepth.util.plot.Plot.CurvePlotScaleSlotMap.__init__}}\pysiglinewithargsret{\sphinxbfcode{\_\_init\_\_}}{\emph{theCpsS}}{}
Ctor with a list of CurvePlotScale objects (can be unsorted).

\end{fulllineitems}

\index{reset() (TotalDepth.util.plot.Plot.CurvePlotScaleSlotMap method)}

\begin{fulllineitems}
\phantomsection\label{\detokenize{ref/util/plot/Plot:TotalDepth.util.plot.Plot.CurvePlotScaleSlotMap.reset}}\pysiglinewithargsret{\sphinxbfcode{reset}}{}{}
Clears the slot map for the current slice.

\end{fulllineitems}

\index{canFit() (TotalDepth.util.plot.Plot.CurvePlotScaleSlotMap method)}

\begin{fulllineitems}
\phantomsection\label{\detokenize{ref/util/plot/Plot:TotalDepth.util.plot.Plot.CurvePlotScaleSlotMap.canFit}}\pysiglinewithargsret{\sphinxbfcode{canFit}}{\emph{theCps}}{}
Returns True if I can fit the CurvePlotScale object in the current slice.

\end{fulllineitems}

\index{fit() (TotalDepth.util.plot.Plot.CurvePlotScaleSlotMap method)}

\begin{fulllineitems}
\phantomsection\label{\detokenize{ref/util/plot/Plot:TotalDepth.util.plot.Plot.CurvePlotScaleSlotMap.fit}}\pysiglinewithargsret{\sphinxbfcode{fit}}{\emph{theCps}}{}
Populates slots from a CurvePlotScale, caller should call canFit() first.

\end{fulllineitems}

\index{genScaleSliceCurve() (TotalDepth.util.plot.Plot.CurvePlotScaleSlotMap method)}

\begin{fulllineitems}
\phantomsection\label{\detokenize{ref/util/plot/Plot:TotalDepth.util.plot.Plot.CurvePlotScaleSlotMap.genScaleSliceCurve}}\pysiglinewithargsret{\sphinxbfcode{genScaleSliceCurve}}{}{}
Generates a ordered list of ScaleSliceCurve objects laid out in a ‘nice’ fashion.

\end{fulllineitems}

\index{\_\_weakref\_\_ (TotalDepth.util.plot.Plot.CurvePlotScaleSlotMap attribute)}

\begin{fulllineitems}
\phantomsection\label{\detokenize{ref/util/plot/Plot:TotalDepth.util.plot.Plot.CurvePlotScaleSlotMap.__weakref__}}\pysigline{\sphinxbfcode{\_\_weakref\_\_}}
list of weak references to the object (if defined)

\end{fulllineitems}


\end{fulllineitems}

\index{PlotRoll (class in TotalDepth.util.plot.Plot)}

\begin{fulllineitems}
\phantomsection\label{\detokenize{ref/util/plot/Plot:TotalDepth.util.plot.Plot.PlotRoll}}\pysiglinewithargsret{\sphinxbfcode{class }\sphinxcode{TotalDepth.util.plot.Plot.}\sphinxbfcode{PlotRoll}}{\emph{theXStart}, \emph{theXStop}, \emph{theScale}, \emph{theLegendDepth}, \emph{theHeadDepth=Dim(value=0}, \emph{units='in')}, \emph{theTailDepth=Dim(value=0}, \emph{units='in')}, \emph{plotUp=True}, \emph{theWidth=Dim(value=8.5}, \emph{units='in')}, \emph{theMargin=Margin(left=Dim(value=0.25}, \emph{units='in')}, \emph{right=Dim(value=0.25}, \emph{units='in')}, \emph{top=Dim(value=0.25}, \emph{units='in')}, \emph{bottom=Dim(value=0.25}, \emph{units='in'))}}{}
Describes the plot canvas as if it were a roll of paper.
This can compute the various dimensions and positions of plot panes:

Legend:
\begin{itemize}
\item {} 
\sphinxcode{...} - The plot within margins.

\item {} 
\sphinxcode{***} - Optional headers and trailers.

\item {} 
\sphinxcode{+++} - The upper and lower headers for scales (legends).

\item {} 
\sphinxcode{\textasciicircum{}\textasciicircum{}\textasciicircum{}} - The main plotting area.

\end{itemize}

In this diagram adjacent lines overlay:

\begin{sphinxVerbatim}[commandchars=\\\{\}]
\PYG{o}{\textbar{}}\PYG{o}{\PYGZhy{}}\PYG{o}{\PYGZhy{}}\PYG{o}{\PYGZhy{}}\PYG{o}{\PYGZhy{}}\PYG{o}{\PYGZhy{}}\PYG{o}{\PYGZhy{}}\PYG{o}{\PYGZhy{}}\PYG{o}{\PYGZhy{}}\PYG{o}{\PYGZhy{}}\PYG{o}{\PYGZhy{}}\PYG{o}{\PYGZhy{}}\PYG{o}{\PYGZhy{}}\PYG{o}{\PYGZhy{}}\PYG{o}{\PYGZhy{}}\PYG{o}{\PYGZhy{}}\PYG{o}{\PYGZhy{}}\PYG{o}{\PYGZhy{}}\PYG{o}{\PYGZhy{}}\PYG{o}{\PYGZhy{}}\PYG{o}{\PYGZhy{}}\PYG{o}{\PYGZhy{}}\PYG{o}{\PYGZhy{}}\PYG{o}{\PYGZhy{}}\PYG{o}{\PYGZhy{}}\PYG{o}{\PYGZhy{}}\PYG{o}{\PYGZhy{}}\PYG{o}{\PYGZhy{}}\PYG{o}{\PYGZhy{}}\PYG{o}{\PYGZhy{}}\PYG{o}{\PYGZhy{}}\PYG{o}{\PYGZhy{}}\PYG{o}{\PYGZhy{}}\PYG{o}{\PYGZhy{}}\PYG{o}{\PYGZhy{}}\PYG{o}{\PYGZhy{}}\PYG{o}{\PYGZhy{}}\PYG{o}{\PYGZhy{}}\PYG{o}{\PYGZhy{}}\PYG{o}{\PYGZhy{}}\PYG{o}{\PYGZhy{}}\PYG{o}{\PYGZhy{}}\PYG{o}{\PYGZhy{}}\PYG{o}{\PYGZhy{}}\PYG{o}{\PYGZhy{}}\PYG{o}{\PYGZhy{}}\PYG{o}{\PYGZhy{}}\PYG{o}{\PYGZhy{}}\PYG{o}{\PYGZhy{}}\PYG{o}{\PYGZhy{}}\PYG{o}{\PYGZhy{}}\PYG{o}{\PYGZhy{}}\PYG{o}{\PYGZhy{}}\PYG{o}{\PYGZhy{}}\PYG{o}{\PYGZhy{}}\PYG{o}{\PYGZhy{}}\PYG{o}{\PYGZhy{}}\PYG{o}{\PYGZhy{}}\PYG{o}{\PYGZhy{}}\PYG{o}{\PYGZhy{}}\PYG{o}{\PYGZhy{}}\PYG{o}{\PYGZhy{}}\PYG{o}{\PYGZhy{}}\PYG{o}{\PYGZhy{}}\PYG{o}{\PYGZhy{}}\PYG{o}{\PYGZhy{}}\PYG{o}{\PYGZhy{}}\PYG{o}{\PYGZhy{}}\PYG{o}{\PYGZhy{}}\PYG{o}{\textbar{}}
\PYG{o}{\textbar{}}                           \PYG{n}{Plot} \PYG{n}{margin}                              \PYG{o}{\textbar{}}
\PYG{o}{\textbar{}}   \PYG{o}{.}\PYG{o}{.}\PYG{o}{.}\PYG{o}{.}\PYG{o}{.}\PYG{o}{.}\PYG{o}{.}\PYG{o}{.}\PYG{o}{.}\PYG{o}{.}\PYG{o}{.}\PYG{o}{.}\PYG{o}{.}\PYG{o}{.}\PYG{o}{.}\PYG{o}{.}\PYG{o}{.}\PYG{o}{.}\PYG{o}{.}\PYG{o}{.}\PYG{o}{.}\PYG{o}{.}\PYG{o}{.}\PYG{o}{.}\PYG{o}{.}\PYG{o}{.}\PYG{o}{.}\PYG{o}{.}\PYG{o}{.}\PYG{o}{.}\PYG{o}{.}\PYG{o}{.}\PYG{o}{.}\PYG{o}{.}\PYG{o}{.}\PYG{o}{.}\PYG{o}{.}\PYG{o}{.}\PYG{o}{.}\PYG{o}{.}\PYG{o}{.}\PYG{o}{.}\PYG{o}{.}\PYG{o}{.}\PYG{o}{.}\PYG{o}{.}\PYG{o}{.}\PYG{o}{.}\PYG{o}{.}\PYG{o}{.}\PYG{o}{.}\PYG{o}{.}\PYG{o}{.}\PYG{o}{.}\PYG{o}{.}\PYG{o}{.}\PYG{o}{.}\PYG{o}{.}\PYG{o}{.}\PYG{o}{.}\PYG{o}{.}\PYG{o}{.}   \PYG{o}{\textbar{}}
\PYG{o}{\textbar{}}   \PYG{o}{.}\PYG{o}{*}\PYG{o}{*}\PYG{o}{*}\PYG{o}{*}\PYG{o}{*}\PYG{o}{*}\PYG{o}{*}\PYG{o}{*}\PYG{o}{*}\PYG{o}{*}\PYG{o}{*}\PYG{o}{*}\PYG{o}{*}\PYG{o}{*}\PYG{o}{*}\PYG{o}{*}\PYG{o}{*}\PYG{o}{*}\PYG{o}{*}\PYG{o}{*}\PYG{o}{*}\PYG{o}{*}\PYG{o}{*}\PYG{o}{*}\PYG{o}{*}\PYG{o}{*}\PYG{o}{*}\PYG{o}{*}\PYG{o}{*}\PYG{o}{*}\PYG{o}{*}\PYG{o}{*}\PYG{o}{*}\PYG{o}{*}\PYG{o}{*}\PYG{o}{*}\PYG{o}{*}\PYG{o}{*}\PYG{o}{*}\PYG{o}{*}\PYG{o}{*}\PYG{o}{*}\PYG{o}{*}\PYG{o}{*}\PYG{o}{*}\PYG{o}{*}\PYG{o}{*}\PYG{o}{*}\PYG{o}{*}\PYG{o}{*}\PYG{o}{*}\PYG{o}{*}\PYG{o}{*}\PYG{o}{*}\PYG{o}{*}\PYG{o}{*}\PYG{o}{*}\PYG{o}{*}\PYG{o}{*}\PYG{o}{*}\PYG{o}{.}   \PYG{o}{\textbar{}}
\PYG{o}{\textbar{}}   \PYG{o}{.}\PYG{o}{*}                                                          \PYG{o}{*}\PYG{o}{.}   \PYG{o}{\textbar{}}
\PYG{o}{\textbar{}}   \PYG{o}{.}\PYG{o}{*}            \PYG{n}{Optional} \PYG{n}{header} \PYG{n}{e}\PYG{o}{.}\PYG{n}{g}\PYG{o}{.} \PYG{n}{API} \PYG{n}{header}               \PYG{o}{*}\PYG{o}{.}   \PYG{o}{\textbar{}}
\PYG{o}{\textbar{}}   \PYG{o}{.}\PYG{o}{*}                                                          \PYG{o}{*}\PYG{o}{.}   \PYG{o}{\textbar{}}
\PYG{o}{\textbar{}}   \PYG{o}{.}\PYG{o}{*}\PYG{o}{*}\PYG{o}{*}\PYG{o}{*}\PYG{o}{*}\PYG{o}{*}\PYG{o}{*}\PYG{o}{*}\PYG{o}{*}\PYG{o}{*}\PYG{o}{*}\PYG{o}{*}\PYG{o}{*}\PYG{o}{*}\PYG{o}{*}\PYG{o}{*}\PYG{o}{*}\PYG{o}{*}\PYG{o}{*}\PYG{o}{*}\PYG{o}{*}\PYG{o}{*}\PYG{o}{*}\PYG{o}{*}\PYG{o}{*}\PYG{o}{*}\PYG{o}{*}\PYG{o}{*}\PYG{o}{*}\PYG{o}{*}\PYG{o}{*}\PYG{o}{*}\PYG{o}{*}\PYG{o}{*}\PYG{o}{*}\PYG{o}{*}\PYG{o}{*}\PYG{o}{*}\PYG{o}{*}\PYG{o}{*}\PYG{o}{*}\PYG{o}{*}\PYG{o}{*}\PYG{o}{*}\PYG{o}{*}\PYG{o}{*}\PYG{o}{*}\PYG{o}{*}\PYG{o}{*}\PYG{o}{*}\PYG{o}{*}\PYG{o}{*}\PYG{o}{*}\PYG{o}{*}\PYG{o}{*}\PYG{o}{*}\PYG{o}{*}\PYG{o}{*}\PYG{o}{*}\PYG{o}{*}\PYG{o}{.}   \PYG{o}{\textbar{}}
\PYG{o}{\textbar{}}   \PYG{o}{.}\PYG{o}{+}\PYG{o}{+}\PYG{o}{+}\PYG{o}{+}\PYG{o}{+}\PYG{o}{+}\PYG{o}{+}\PYG{o}{+}\PYG{o}{+}\PYG{o}{+}\PYG{o}{+}\PYG{o}{+}\PYG{o}{+}\PYG{o}{+}\PYG{o}{+}\PYG{o}{+}\PYG{o}{+}\PYG{o}{+}\PYG{o}{+}\PYG{o}{+}\PYG{o}{+}\PYG{o}{+}\PYG{o}{+}\PYG{o}{+}\PYG{o}{+}\PYG{o}{+}\PYG{o}{+}\PYG{o}{+}\PYG{o}{+}\PYG{o}{+}\PYG{o}{+}\PYG{o}{+}\PYG{o}{+}\PYG{o}{+}\PYG{o}{+}\PYG{o}{+}\PYG{o}{+}\PYG{o}{+}\PYG{o}{+}\PYG{o}{+}\PYG{o}{+}\PYG{o}{+}\PYG{o}{+}\PYG{o}{+}\PYG{o}{+}\PYG{o}{+}\PYG{o}{+}\PYG{o}{+}\PYG{o}{+}\PYG{o}{+}\PYG{o}{+}\PYG{o}{+}\PYG{o}{+}\PYG{o}{+}\PYG{o}{+}\PYG{o}{+}\PYG{o}{+}\PYG{o}{+}\PYG{o}{+}\PYG{o}{+}\PYG{o}{.}   \PYG{o}{\textbar{}}
\PYG{o}{\textbar{}}   \PYG{o}{.}\PYG{o}{+}                                                          \PYG{o}{+}\PYG{o}{.}   \PYG{o}{\textbar{}}
\PYG{o}{\textbar{}}   \PYG{o}{.}\PYG{o}{+}       \PYG{n}{Scales} \PYG{p}{(}\PYG{n}{legends}\PYG{p}{)} \PYG{n}{at} \PYG{n}{the} \PYG{n}{top} \PYG{n}{of} \PYG{n}{the} \PYG{n}{plot}\PYG{o}{.}           \PYG{o}{+}\PYG{o}{.}   \PYG{o}{\textbar{}}
\PYG{o}{\textbar{}}   \PYG{o}{.}\PYG{o}{+}                                                          \PYG{o}{+}\PYG{o}{.}   \PYG{o}{\textbar{}}
\PYG{o}{\textbar{}}   \PYG{o}{.}\PYG{o}{+}\PYG{o}{+}\PYG{o}{+}\PYG{o}{+}\PYG{o}{+}\PYG{o}{+}\PYG{o}{+}\PYG{o}{+}\PYG{o}{+}\PYG{o}{+}\PYG{o}{+}\PYG{o}{+}\PYG{o}{+}\PYG{o}{+}\PYG{o}{+}\PYG{o}{+}\PYG{o}{+}\PYG{o}{+}\PYG{o}{+}\PYG{o}{+}\PYG{o}{+}\PYG{o}{+}\PYG{o}{+}\PYG{o}{+}\PYG{o}{+}\PYG{o}{+}\PYG{o}{+}\PYG{o}{+}\PYG{o}{+}\PYG{o}{+}\PYG{o}{+}\PYG{o}{+}\PYG{o}{+}\PYG{o}{+}\PYG{o}{+}\PYG{o}{+}\PYG{o}{+}\PYG{o}{+}\PYG{o}{+}\PYG{o}{+}\PYG{o}{+}\PYG{o}{+}\PYG{o}{+}\PYG{o}{+}\PYG{o}{+}\PYG{o}{+}\PYG{o}{+}\PYG{o}{+}\PYG{o}{+}\PYG{o}{+}\PYG{o}{+}\PYG{o}{+}\PYG{o}{+}\PYG{o}{+}\PYG{o}{+}\PYG{o}{+}\PYG{o}{+}\PYG{o}{+}\PYG{o}{+}\PYG{o}{+}\PYG{o}{.}   \PYG{o}{\textbar{}}
\PYG{o}{\textbar{}}   \PYG{o}{.}\PYG{o}{\PYGZca{}}\PYG{o}{\PYGZca{}}\PYG{o}{\PYGZca{}}\PYG{o}{\PYGZca{}}\PYG{o}{\PYGZca{}}\PYG{o}{\PYGZca{}}\PYG{o}{\PYGZca{}}\PYG{o}{\PYGZca{}}\PYG{o}{\PYGZca{}}\PYG{o}{\PYGZca{}}\PYG{o}{\PYGZca{}}\PYG{o}{\PYGZca{}}\PYG{o}{\PYGZca{}}\PYG{o}{\PYGZca{}}\PYG{o}{\PYGZca{}} \PYG{o}{\PYGZhy{}}\PYG{o}{\PYGZhy{}} \PYG{n}{X} \PYG{n}{Stop} \PYG{p}{(}\PYG{k}{if} \PYG{n}{up} \PYG{n}{log}\PYG{p}{)} \PYG{o}{\PYGZhy{}}\PYG{o}{\PYGZhy{}} \PYG{o}{\PYGZca{}}\PYG{o}{\PYGZca{}}\PYG{o}{\PYGZca{}}\PYG{o}{\PYGZca{}}\PYG{o}{\PYGZca{}}\PYG{o}{\PYGZca{}}\PYG{o}{\PYGZca{}}\PYG{o}{\PYGZca{}}\PYG{o}{\PYGZca{}}\PYG{o}{\PYGZca{}}\PYG{o}{\PYGZca{}}\PYG{o}{\PYGZca{}}\PYG{o}{\PYGZca{}}\PYG{o}{\PYGZca{}}\PYG{o}{\PYGZca{}}\PYG{o}{\PYGZca{}}\PYG{o}{\PYGZca{}}\PYG{o}{\PYGZca{}}\PYG{o}{\PYGZca{}}\PYG{o}{.}   \PYG{o}{\textbar{}}
\PYG{o}{\textbar{}}   \PYG{o}{.}\PYG{o}{\PYGZca{}}                                                          \PYG{o}{\PYGZca{}}\PYG{o}{.}   \PYG{o}{\textbar{}}
\PYG{o}{\textbar{}}   \PYG{o}{.}\PYG{o}{\PYGZca{}}                                                          \PYG{o}{\PYGZca{}}\PYG{o}{.}   \PYG{o}{\textbar{}}
\PYG{o}{\textbar{}}   \PYG{o}{.}\PYG{o}{\PYGZca{}}                                                          \PYG{o}{\PYGZca{}}\PYG{o}{.}   \PYG{o}{\textbar{}}
\PYG{o}{\textbar{}}   \PYG{o}{.}\PYG{o}{\PYGZca{}}                  \PYG{n}{Main} \PYG{n}{log} \PYG{n}{plotted} \PYG{n}{here}                   \PYG{o}{\PYGZca{}}\PYG{o}{.}   \PYG{o}{\textbar{}}
\PYG{o}{\textbar{}}   \PYG{o}{.}\PYG{o}{\PYGZca{}}                                                          \PYG{o}{\PYGZca{}}\PYG{o}{.}   \PYG{o}{\textbar{}}
\PYG{o}{\textbar{}}   \PYG{o}{.}\PYG{o}{\PYGZca{}}                                                          \PYG{o}{\PYGZca{}}\PYG{o}{.}   \PYG{o}{\textbar{}}
\PYG{o}{\textbar{}}   \PYG{o}{.}\PYG{o}{\PYGZca{}}                                                          \PYG{o}{\PYGZca{}}\PYG{o}{.}   \PYG{o}{\textbar{}}
\PYG{o}{\textbar{}}   \PYG{o}{.}\PYG{o}{\PYGZca{}}\PYG{o}{\PYGZca{}}\PYG{o}{\PYGZca{}}\PYG{o}{\PYGZca{}}\PYG{o}{\PYGZca{}}\PYG{o}{\PYGZca{}}\PYG{o}{\PYGZca{}}\PYG{o}{\PYGZca{}}\PYG{o}{\PYGZca{}}\PYG{o}{\PYGZca{}}\PYG{o}{\PYGZca{}}\PYG{o}{\PYGZca{}}\PYG{o}{\PYGZca{}}\PYG{o}{\PYGZca{}}\PYG{o}{\PYGZca{}}\PYG{o}{\PYGZca{}} \PYG{o}{\PYGZhy{}}\PYG{o}{\PYGZhy{}} \PYG{n}{X} \PYG{n}{Start} \PYG{p}{(}\PYG{k}{if} \PYG{n}{up} \PYG{n}{log}\PYG{p}{)} \PYG{o}{\PYGZhy{}}\PYG{o}{\PYGZhy{}} \PYG{o}{\PYGZca{}}\PYG{o}{\PYGZca{}}\PYG{o}{\PYGZca{}}\PYG{o}{\PYGZca{}}\PYG{o}{\PYGZca{}}\PYG{o}{\PYGZca{}}\PYG{o}{\PYGZca{}}\PYG{o}{\PYGZca{}}\PYG{o}{\PYGZca{}}\PYG{o}{\PYGZca{}}\PYG{o}{\PYGZca{}}\PYG{o}{\PYGZca{}}\PYG{o}{\PYGZca{}}\PYG{o}{\PYGZca{}}\PYG{o}{\PYGZca{}}\PYG{o}{\PYGZca{}}\PYG{o}{\PYGZca{}}\PYG{o}{.}   \PYG{o}{\textbar{}}
\PYG{o}{\textbar{}}   \PYG{o}{.}\PYG{o}{+}\PYG{o}{+}\PYG{o}{+}\PYG{o}{+}\PYG{o}{+}\PYG{o}{+}\PYG{o}{+}\PYG{o}{+}\PYG{o}{+}\PYG{o}{+}\PYG{o}{+}\PYG{o}{+}\PYG{o}{+}\PYG{o}{+}\PYG{o}{+}\PYG{o}{+}\PYG{o}{+}\PYG{o}{+}\PYG{o}{+}\PYG{o}{+}\PYG{o}{+}\PYG{o}{+}\PYG{o}{+}\PYG{o}{+}\PYG{o}{+}\PYG{o}{+}\PYG{o}{+}\PYG{o}{+}\PYG{o}{+}\PYG{o}{+}\PYG{o}{+}\PYG{o}{+}\PYG{o}{+}\PYG{o}{+}\PYG{o}{+}\PYG{o}{+}\PYG{o}{+}\PYG{o}{+}\PYG{o}{+}\PYG{o}{+}\PYG{o}{+}\PYG{o}{+}\PYG{o}{+}\PYG{o}{+}\PYG{o}{+}\PYG{o}{+}\PYG{o}{+}\PYG{o}{+}\PYG{o}{+}\PYG{o}{+}\PYG{o}{+}\PYG{o}{+}\PYG{o}{+}\PYG{o}{+}\PYG{o}{+}\PYG{o}{+}\PYG{o}{+}\PYG{o}{+}\PYG{o}{+}\PYG{o}{+}\PYG{o}{.}   \PYG{o}{\textbar{}}
\PYG{o}{\textbar{}}   \PYG{o}{.}\PYG{o}{+}                                                          \PYG{o}{+}\PYG{o}{.}   \PYG{o}{\textbar{}}
\PYG{o}{\textbar{}}   \PYG{o}{.}\PYG{o}{+}       \PYG{n}{Scales} \PYG{p}{(}\PYG{n}{legends}\PYG{p}{)} \PYG{n}{at} \PYG{n}{the} \PYG{n}{bottom} \PYG{n}{of} \PYG{n}{the} \PYG{n}{plot}\PYG{o}{.}        \PYG{o}{+}\PYG{o}{.}   \PYG{o}{\textbar{}}
\PYG{o}{\textbar{}}   \PYG{o}{.}\PYG{o}{+}                                                          \PYG{o}{+}\PYG{o}{.}   \PYG{o}{\textbar{}}
\PYG{o}{\textbar{}}   \PYG{o}{.}\PYG{o}{+}\PYG{o}{+}\PYG{o}{+}\PYG{o}{+}\PYG{o}{+}\PYG{o}{+}\PYG{o}{+}\PYG{o}{+}\PYG{o}{+}\PYG{o}{+}\PYG{o}{+}\PYG{o}{+}\PYG{o}{+}\PYG{o}{+}\PYG{o}{+}\PYG{o}{+}\PYG{o}{+}\PYG{o}{+}\PYG{o}{+}\PYG{o}{+}\PYG{o}{+}\PYG{o}{+}\PYG{o}{+}\PYG{o}{+}\PYG{o}{+}\PYG{o}{+}\PYG{o}{+}\PYG{o}{+}\PYG{o}{+}\PYG{o}{+}\PYG{o}{+}\PYG{o}{+}\PYG{o}{+}\PYG{o}{+}\PYG{o}{+}\PYG{o}{+}\PYG{o}{+}\PYG{o}{+}\PYG{o}{+}\PYG{o}{+}\PYG{o}{+}\PYG{o}{+}\PYG{o}{+}\PYG{o}{+}\PYG{o}{+}\PYG{o}{+}\PYG{o}{+}\PYG{o}{+}\PYG{o}{+}\PYG{o}{+}\PYG{o}{+}\PYG{o}{+}\PYG{o}{+}\PYG{o}{+}\PYG{o}{+}\PYG{o}{+}\PYG{o}{+}\PYG{o}{+}\PYG{o}{+}\PYG{o}{+}\PYG{o}{.}   \PYG{o}{\textbar{}}
\PYG{o}{\textbar{}}   \PYG{o}{.}\PYG{o}{*}\PYG{o}{*}\PYG{o}{*}\PYG{o}{*}\PYG{o}{*}\PYG{o}{*}\PYG{o}{*}\PYG{o}{*}\PYG{o}{*}\PYG{o}{*}\PYG{o}{*}\PYG{o}{*}\PYG{o}{*}\PYG{o}{*}\PYG{o}{*}\PYG{o}{*}\PYG{o}{*}\PYG{o}{*}\PYG{o}{*}\PYG{o}{*}\PYG{o}{*}\PYG{o}{*}\PYG{o}{*}\PYG{o}{*}\PYG{o}{*}\PYG{o}{*}\PYG{o}{*}\PYG{o}{*}\PYG{o}{*}\PYG{o}{*}\PYG{o}{*}\PYG{o}{*}\PYG{o}{*}\PYG{o}{*}\PYG{o}{*}\PYG{o}{*}\PYG{o}{*}\PYG{o}{*}\PYG{o}{*}\PYG{o}{*}\PYG{o}{*}\PYG{o}{*}\PYG{o}{*}\PYG{o}{*}\PYG{o}{*}\PYG{o}{*}\PYG{o}{*}\PYG{o}{*}\PYG{o}{*}\PYG{o}{*}\PYG{o}{*}\PYG{o}{*}\PYG{o}{*}\PYG{o}{*}\PYG{o}{*}\PYG{o}{*}\PYG{o}{*}\PYG{o}{*}\PYG{o}{*}\PYG{o}{*}\PYG{o}{.}   \PYG{o}{\textbar{}}
\PYG{o}{\textbar{}}   \PYG{o}{.}\PYG{o}{*}                                                          \PYG{o}{*}\PYG{o}{.}   \PYG{o}{\textbar{}}
\PYG{o}{\textbar{}}   \PYG{o}{.}\PYG{o}{*}         \PYG{n}{Optional} \PYG{n}{trailer} \PYG{n}{e}\PYG{o}{.}\PYG{n}{g}\PYG{o}{.} \PYG{n}{calibration} \PYG{n}{record}         \PYG{o}{*}\PYG{o}{.}   \PYG{o}{\textbar{}}
\PYG{o}{\textbar{}}   \PYG{o}{.}\PYG{o}{*}                                                          \PYG{o}{*}\PYG{o}{.}   \PYG{o}{\textbar{}}
\PYG{o}{\textbar{}}   \PYG{o}{.}\PYG{o}{*}\PYG{o}{*}\PYG{o}{*}\PYG{o}{*}\PYG{o}{*}\PYG{o}{*}\PYG{o}{*}\PYG{o}{*}\PYG{o}{*}\PYG{o}{*}\PYG{o}{*}\PYG{o}{*}\PYG{o}{*}\PYG{o}{*}\PYG{o}{*}\PYG{o}{*}\PYG{o}{*}\PYG{o}{*}\PYG{o}{*}\PYG{o}{*}\PYG{o}{*}\PYG{o}{*}\PYG{o}{*}\PYG{o}{*}\PYG{o}{*}\PYG{o}{*}\PYG{o}{*}\PYG{o}{*}\PYG{o}{*}\PYG{o}{*}\PYG{o}{*}\PYG{o}{*}\PYG{o}{*}\PYG{o}{*}\PYG{o}{*}\PYG{o}{*}\PYG{o}{*}\PYG{o}{*}\PYG{o}{*}\PYG{o}{*}\PYG{o}{*}\PYG{o}{*}\PYG{o}{*}\PYG{o}{*}\PYG{o}{*}\PYG{o}{*}\PYG{o}{*}\PYG{o}{*}\PYG{o}{*}\PYG{o}{*}\PYG{o}{*}\PYG{o}{*}\PYG{o}{*}\PYG{o}{*}\PYG{o}{*}\PYG{o}{*}\PYG{o}{*}\PYG{o}{*}\PYG{o}{*}\PYG{o}{*}\PYG{o}{.}   \PYG{o}{\textbar{}}
\PYG{o}{\textbar{}}   \PYG{o}{.}\PYG{o}{.}\PYG{o}{.}\PYG{o}{.}\PYG{o}{.}\PYG{o}{.}\PYG{o}{.}\PYG{o}{.}\PYG{o}{.}\PYG{o}{.}\PYG{o}{.}\PYG{o}{.}\PYG{o}{.}\PYG{o}{.}\PYG{o}{.}\PYG{o}{.}\PYG{o}{.}\PYG{o}{.}\PYG{o}{.}\PYG{o}{.}\PYG{o}{.}\PYG{o}{.}\PYG{o}{.}\PYG{o}{.}\PYG{o}{.}\PYG{o}{.}\PYG{o}{.}\PYG{o}{.}\PYG{o}{.}\PYG{o}{.}\PYG{o}{.}\PYG{o}{.}\PYG{o}{.}\PYG{o}{.}\PYG{o}{.}\PYG{o}{.}\PYG{o}{.}\PYG{o}{.}\PYG{o}{.}\PYG{o}{.}\PYG{o}{.}\PYG{o}{.}\PYG{o}{.}\PYG{o}{.}\PYG{o}{.}\PYG{o}{.}\PYG{o}{.}\PYG{o}{.}\PYG{o}{.}\PYG{o}{.}\PYG{o}{.}\PYG{o}{.}\PYG{o}{.}\PYG{o}{.}\PYG{o}{.}\PYG{o}{.}\PYG{o}{.}\PYG{o}{.}\PYG{o}{.}\PYG{o}{.}\PYG{o}{.}\PYG{o}{.}   \PYG{o}{\textbar{}}
\PYG{o}{\textbar{}}                           \PYG{n}{Plot} \PYG{n}{margin}                              \PYG{o}{\textbar{}}
\PYG{o}{\textbar{}}\PYG{o}{\PYGZhy{}}\PYG{o}{\PYGZhy{}}\PYG{o}{\PYGZhy{}}\PYG{o}{\PYGZhy{}}\PYG{o}{\PYGZhy{}}\PYG{o}{\PYGZhy{}}\PYG{o}{\PYGZhy{}}\PYG{o}{\PYGZhy{}}\PYG{o}{\PYGZhy{}}\PYG{o}{\PYGZhy{}}\PYG{o}{\PYGZhy{}}\PYG{o}{\PYGZhy{}}\PYG{o}{\PYGZhy{}}\PYG{o}{\PYGZhy{}}\PYG{o}{\PYGZhy{}}\PYG{o}{\PYGZhy{}}\PYG{o}{\PYGZhy{}}\PYG{o}{\PYGZhy{}}\PYG{o}{\PYGZhy{}}\PYG{o}{\PYGZhy{}}\PYG{o}{\PYGZhy{}}\PYG{o}{\PYGZhy{}}\PYG{o}{\PYGZhy{}}\PYG{o}{\PYGZhy{}}\PYG{o}{\PYGZhy{}}\PYG{o}{\PYGZhy{}}\PYG{o}{\PYGZhy{}}\PYG{o}{\PYGZhy{}}\PYG{o}{\PYGZhy{}}\PYG{o}{\PYGZhy{}}\PYG{o}{\PYGZhy{}}\PYG{o}{\PYGZhy{}}\PYG{o}{\PYGZhy{}}\PYG{o}{\PYGZhy{}}\PYG{o}{\PYGZhy{}}\PYG{o}{\PYGZhy{}}\PYG{o}{\PYGZhy{}}\PYG{o}{\PYGZhy{}}\PYG{o}{\PYGZhy{}}\PYG{o}{\PYGZhy{}}\PYG{o}{\PYGZhy{}}\PYG{o}{\PYGZhy{}}\PYG{o}{\PYGZhy{}}\PYG{o}{\PYGZhy{}}\PYG{o}{\PYGZhy{}}\PYG{o}{\PYGZhy{}}\PYG{o}{\PYGZhy{}}\PYG{o}{\PYGZhy{}}\PYG{o}{\PYGZhy{}}\PYG{o}{\PYGZhy{}}\PYG{o}{\PYGZhy{}}\PYG{o}{\PYGZhy{}}\PYG{o}{\PYGZhy{}}\PYG{o}{\PYGZhy{}}\PYG{o}{\PYGZhy{}}\PYG{o}{\PYGZhy{}}\PYG{o}{\PYGZhy{}}\PYG{o}{\PYGZhy{}}\PYG{o}{\PYGZhy{}}\PYG{o}{\PYGZhy{}}\PYG{o}{\PYGZhy{}}\PYG{o}{\PYGZhy{}}\PYG{o}{\PYGZhy{}}\PYG{o}{\PYGZhy{}}\PYG{o}{\PYGZhy{}}\PYG{o}{\PYGZhy{}}\PYG{o}{\PYGZhy{}}\PYG{o}{\PYGZhy{}}\PYG{o}{\textbar{}}
\end{sphinxVerbatim}
\index{\_\_init\_\_() (TotalDepth.util.plot.Plot.PlotRoll method)}

\begin{fulllineitems}
\phantomsection\label{\detokenize{ref/util/plot/Plot:TotalDepth.util.plot.Plot.PlotRoll.__init__}}\pysiglinewithargsret{\sphinxbfcode{\_\_init\_\_}}{\emph{theXStart}, \emph{theXStop}, \emph{theScale}, \emph{theLegendDepth}, \emph{theHeadDepth=Dim(value=0}, \emph{units='in')}, \emph{theTailDepth=Dim(value=0}, \emph{units='in')}, \emph{plotUp=True}, \emph{theWidth=Dim(value=8.5}, \emph{units='in')}, \emph{theMargin=Margin(left=Dim(value=0.25}, \emph{units='in')}, \emph{right=Dim(value=0.25}, \emph{units='in')}, \emph{top=Dim(value=0.25}, \emph{units='in')}, \emph{bottom=Dim(value=0.25}, \emph{units='in'))}}{}
Initialise with:
\begin{description}
\item[{\sphinxstyleemphasis{theXStart}}] \leavevmode
The X start position as an \sphinxcode{EngVal}.

\item[{\sphinxstyleemphasis{theXStop}}] \leavevmode
The X stop position as an \sphinxcode{EngVal}.

\item[{\sphinxstyleemphasis{theScale}}] \leavevmode
The plot scale as a number.

\item[{\sphinxstyleemphasis{theLegendDepth}}] \leavevmode
A \sphinxcode{Coord.Dim()} that is the depth of the scales used in
the header and footer.

\item[{\sphinxstyleemphasis{theHeadDepth}}] \leavevmode
A \sphinxcode{Coord.Dim()} that is the depth of any log header.

\item[{\sphinxstyleemphasis{theTailDepth}}] \leavevmode
A \sphinxcode{Coord.Dim()} that is the depth of any log trailer.

\item[{\sphinxstyleemphasis{plotUp, bool}}] \leavevmode
True if X start is at the bottom of the main pane.

\item[{\sphinxstyleemphasis{theWidth}}] \leavevmode
The absolute width of the plot as a \sphinxcode{Coord.Dim()}.

\item[{\sphinxstyleemphasis{theMargin}}] \leavevmode
A \sphinxcode{Coord.Margin()} that describes the untouchable edges of the plot.

\end{description}

\end{fulllineitems}

\index{viewBox (TotalDepth.util.plot.Plot.PlotRoll attribute)}

\begin{fulllineitems}
\phantomsection\label{\detokenize{ref/util/plot/Plot:TotalDepth.util.plot.Plot.PlotRoll.viewBox}}\pysigline{\sphinxbfcode{viewBox}}
The overall size of the plot.

\end{fulllineitems}

\index{width (TotalDepth.util.plot.Plot.PlotRoll attribute)}

\begin{fulllineitems}
\phantomsection\label{\detokenize{ref/util/plot/Plot:TotalDepth.util.plot.Plot.PlotRoll.width}}\pysigline{\sphinxbfcode{width}}
The overall width as a number in PlotConstants.DEFAULT\_PLOT\_UNITS.

\end{fulllineitems}

\index{depth (TotalDepth.util.plot.Plot.PlotRoll attribute)}

\begin{fulllineitems}
\phantomsection\label{\detokenize{ref/util/plot/Plot:TotalDepth.util.plot.Plot.PlotRoll.depth}}\pysigline{\sphinxbfcode{depth}}
The overall width as a number in PlotConstants.DEFAULT\_PLOT\_UNITS.

\end{fulllineitems}

\index{widthDim (TotalDepth.util.plot.Plot.PlotRoll attribute)}

\begin{fulllineitems}
\phantomsection\label{\detokenize{ref/util/plot/Plot:TotalDepth.util.plot.Plot.PlotRoll.widthDim}}\pysigline{\sphinxbfcode{widthDim}}
The overall width as a Coord.Dim().

\end{fulllineitems}

\index{depthDim (TotalDepth.util.plot.Plot.PlotRoll attribute)}

\begin{fulllineitems}
\phantomsection\label{\detokenize{ref/util/plot/Plot:TotalDepth.util.plot.Plot.PlotRoll.depthDim}}\pysigline{\sphinxbfcode{depthDim}}
The overall width as a Coord.Dim().

\end{fulllineitems}

\index{trackTopLeft (TotalDepth.util.plot.Plot.PlotRoll attribute)}

\begin{fulllineitems}
\phantomsection\label{\detokenize{ref/util/plot/Plot:TotalDepth.util.plot.Plot.PlotRoll.trackTopLeft}}\pysigline{\sphinxbfcode{trackTopLeft}}
“A Coord.Pt() that is the top left of the pane that tracks are
plotted within.

\end{fulllineitems}

\index{mainPanePlotDepth (TotalDepth.util.plot.Plot.PlotRoll attribute)}

\begin{fulllineitems}
\phantomsection\label{\detokenize{ref/util/plot/Plot:TotalDepth.util.plot.Plot.PlotRoll.mainPanePlotDepth}}\pysigline{\sphinxbfcode{mainPanePlotDepth}}
The depth of the plot of the main pane as a Coord.Dim().

\end{fulllineitems}

\index{availableWidth (TotalDepth.util.plot.Plot.PlotRoll attribute)}

\begin{fulllineitems}
\phantomsection\label{\detokenize{ref/util/plot/Plot:TotalDepth.util.plot.Plot.PlotRoll.availableWidth}}\pysigline{\sphinxbfcode{availableWidth}}
The available width inside the margins.

\end{fulllineitems}

\index{retHeadPane() (TotalDepth.util.plot.Plot.PlotRoll method)}

\begin{fulllineitems}
\phantomsection\label{\detokenize{ref/util/plot/Plot:TotalDepth.util.plot.Plot.PlotRoll.retHeadPane}}\pysiglinewithargsret{\sphinxbfcode{retHeadPane}}{}{}
Returns a pair of top-left Coord.Pt(), Coord.Box() for the top
header where the header goes.

\end{fulllineitems}

\index{retLegendPane() (TotalDepth.util.plot.Plot.PlotRoll method)}

\begin{fulllineitems}
\phantomsection\label{\detokenize{ref/util/plot/Plot:TotalDepth.util.plot.Plot.PlotRoll.retLegendPane}}\pysiglinewithargsret{\sphinxbfcode{retLegendPane}}{\emph{isTop}}{}
Returns a pair of top-left Coord.Pt(), Coord.Box() for the top or bottom
legend pane where the scales go.

\end{fulllineitems}

\index{retMainPane() (TotalDepth.util.plot.Plot.PlotRoll method)}

\begin{fulllineitems}
\phantomsection\label{\detokenize{ref/util/plot/Plot:TotalDepth.util.plot.Plot.PlotRoll.retMainPane}}\pysiglinewithargsret{\sphinxbfcode{retMainPane}}{}{}
Returns a pair of top-left Coord.Pt(), Coord.Box() for the pane
where the main log goes.

\end{fulllineitems}

\index{retTailPane() (TotalDepth.util.plot.Plot.PlotRoll method)}

\begin{fulllineitems}
\phantomsection\label{\detokenize{ref/util/plot/Plot:TotalDepth.util.plot.Plot.PlotRoll.retTailPane}}\pysiglinewithargsret{\sphinxbfcode{retTailPane}}{}{}
Returns a pair of top-left Coord.Pt(), Coord.Box() for the pane
where the trailer goes.

\end{fulllineitems}

\index{xDepth() (TotalDepth.util.plot.Plot.PlotRoll method)}

\begin{fulllineitems}
\phantomsection\label{\detokenize{ref/util/plot/Plot:TotalDepth.util.plot.Plot.PlotRoll.xDepth}}\pysiglinewithargsret{\sphinxbfcode{xDepth}}{\emph{theX}}{}
Returns a Coord.Dim() on the main pane that corresponds to theX as
a number or an EngVal. If this is a number it is expected to be in the
units of the xStart/xStop in the constructor.

\end{fulllineitems}

\index{polyLinePt() (TotalDepth.util.plot.Plot.PlotRoll method)}

\begin{fulllineitems}
\phantomsection\label{\detokenize{ref/util/plot/Plot:TotalDepth.util.plot.Plot.PlotRoll.polyLinePt}}\pysiglinewithargsret{\sphinxbfcode{polyLinePt}}{\emph{theX}, \emph{theTracPos}}{}
Returns a Coord.Pt from theX axis value (or EngVal) and theTracPos
that is a value in DEFAULT\_PLOT\_UNITS, for example given by a
tracValueFunction.
The Coord.Pt() will be scaled by VIEW\_BOX\_UNITS\_PER\_PLOT\_UNITS.

\end{fulllineitems}

\index{retMainPaneStart() (TotalDepth.util.plot.Plot.PlotRoll method)}

\begin{fulllineitems}
\phantomsection\label{\detokenize{ref/util/plot/Plot:TotalDepth.util.plot.Plot.PlotRoll.retMainPaneStart}}\pysiglinewithargsret{\sphinxbfcode{retMainPaneStart}}{}{}
Returns the start Coord.Pt() for the pane where the main log goes.
For and upPlot this will be pane-bottom-left, for a downPlot this will
be pane-top-left.

\end{fulllineitems}

\index{\_\_weakref\_\_ (TotalDepth.util.plot.Plot.PlotRoll attribute)}

\begin{fulllineitems}
\phantomsection\label{\detokenize{ref/util/plot/Plot:TotalDepth.util.plot.Plot.PlotRoll.__weakref__}}\pysigline{\sphinxbfcode{\_\_weakref\_\_}}
list of weak references to the object (if defined)

\end{fulllineitems}


\end{fulllineitems}

\index{Plot (class in TotalDepth.util.plot.Plot)}

\begin{fulllineitems}
\phantomsection\label{\detokenize{ref/util/plot/Plot:TotalDepth.util.plot.Plot.Plot}}\pysiglinewithargsret{\sphinxbfcode{class }\sphinxcode{TotalDepth.util.plot.Plot.}\sphinxbfcode{Plot}}{\emph{theFilmCfg}, \emph{thePresCfg}, \emph{theScale=0}}{}
Defines a plot configuration. The basic architecture follows the
data. The constructor takes all the static data, typically this can
be obtained from the PRES and FILE tables.
The dynamic (or user selected data) is passed in to plotLogPassLIS(). This is
intended as a single Plot object might be used multiple times (e.g. on
‘scroll up’).
\index{TITLE\_FONT\_FAMILY (TotalDepth.util.plot.Plot.Plot attribute)}

\begin{fulllineitems}
\phantomsection\label{\detokenize{ref/util/plot/Plot:TotalDepth.util.plot.Plot.Plot.TITLE_FONT_FAMILY}}\pysigline{\sphinxbfcode{TITLE\_FONT\_FAMILY}\sphinxbfcode{ = 'Verdana'}}
Title font

\end{fulllineitems}

\index{TITLE\_FONT\_SIZE (TotalDepth.util.plot.Plot.Plot attribute)}

\begin{fulllineitems}
\phantomsection\label{\detokenize{ref/util/plot/Plot:TotalDepth.util.plot.Plot.Plot.TITLE_FONT_SIZE}}\pysigline{\sphinxbfcode{TITLE\_FONT\_SIZE}\sphinxbfcode{ = 10}}
Title font size

\end{fulllineitems}

\index{LEGEND\_DEPTH\_PER\_CURVE (TotalDepth.util.plot.Plot.Plot attribute)}

\begin{fulllineitems}
\phantomsection\label{\detokenize{ref/util/plot/Plot:TotalDepth.util.plot.Plot.Plot.LEGEND_DEPTH_PER_CURVE}}\pysigline{\sphinxbfcode{LEGEND\_DEPTH\_PER\_CURVE}\sphinxbfcode{ = Dim(value=0.5, units='in')}}
How much depth to give each legend (curve scale) at the top and
bottom of the log

\end{fulllineitems}

\index{LEGEND\_DEPTH\_SPARE (TotalDepth.util.plot.Plot.Plot attribute)}

\begin{fulllineitems}
\phantomsection\label{\detokenize{ref/util/plot/Plot:TotalDepth.util.plot.Plot.Plot.LEGEND_DEPTH_SPARE}}\pysigline{\sphinxbfcode{LEGEND\_DEPTH\_SPARE}\sphinxbfcode{ = Dim(value=0.5, units='in')}}
How much spare depth to give over the legend (curve scale) sections at
the top and bottom of the log

\end{fulllineitems}

\index{LEGEND\_HORIZONTAL\_LINE\_DEPTH\_PROPORTION (TotalDepth.util.plot.Plot.Plot attribute)}

\begin{fulllineitems}
\phantomsection\label{\detokenize{ref/util/plot/Plot:TotalDepth.util.plot.Plot.Plot.LEGEND_HORIZONTAL_LINE_DEPTH_PROPORTION}}\pysigline{\sphinxbfcode{LEGEND\_HORIZONTAL\_LINE\_DEPTH\_PROPORTION}\sphinxbfcode{ = 0.625}}
Where the curve line in the legend section appears as a proportion of
LEGEND\_DEPTH\_PER\_CURVE

\end{fulllineitems}

\index{LEGEND\_ARROW\_DISPLAY (TotalDepth.util.plot.Plot.Plot attribute)}

\begin{fulllineitems}
\phantomsection\label{\detokenize{ref/util/plot/Plot:TotalDepth.util.plot.Plot.Plot.LEGEND_ARROW_DISPLAY}}\pysigline{\sphinxbfcode{LEGEND\_ARROW\_DISPLAY}\sphinxbfcode{ = True}}
Arrow heads on legend scales

\end{fulllineitems}

\index{LEGEND\_ARROW\_WIDTH (TotalDepth.util.plot.Plot.Plot attribute)}

\begin{fulllineitems}
\phantomsection\label{\detokenize{ref/util/plot/Plot:TotalDepth.util.plot.Plot.Plot.LEGEND_ARROW_WIDTH}}\pysigline{\sphinxbfcode{LEGEND\_ARROW\_WIDTH}\sphinxbfcode{ = Dim(value=0.08, units='in')}}
Arrow head width on legend scales

\end{fulllineitems}

\index{LEGEND\_ARROW\_DEPTH (TotalDepth.util.plot.Plot.Plot attribute)}

\begin{fulllineitems}
\phantomsection\label{\detokenize{ref/util/plot/Plot:TotalDepth.util.plot.Plot.Plot.LEGEND_ARROW_DEPTH}}\pysigline{\sphinxbfcode{LEGEND\_ARROW\_DEPTH}\sphinxbfcode{ = Dim(value=0.04, units='in')}}
Arrow head depth on legend scales

\end{fulllineitems}

\index{LEGEND\_ARROW\_WIDTH\_PX (TotalDepth.util.plot.Plot.Plot attribute)}

\begin{fulllineitems}
\phantomsection\label{\detokenize{ref/util/plot/Plot:TotalDepth.util.plot.Plot.Plot.LEGEND_ARROW_WIDTH_PX}}\pysigline{\sphinxbfcode{LEGEND\_ARROW\_WIDTH\_PX}\sphinxbfcode{ = 0.75}}
Arrow head line width on legend scales

\end{fulllineitems}

\index{CURVE\_LEGEND\_FONT\_FAMILY (TotalDepth.util.plot.Plot.Plot attribute)}

\begin{fulllineitems}
\phantomsection\label{\detokenize{ref/util/plot/Plot:TotalDepth.util.plot.Plot.Plot.CURVE_LEGEND_FONT_FAMILY}}\pysigline{\sphinxbfcode{CURVE\_LEGEND\_FONT\_FAMILY}\sphinxbfcode{ = 'Courier'}}
Legend font.

\end{fulllineitems}

\index{CURVE\_LEGEND\_FONT\_SIZE (TotalDepth.util.plot.Plot.Plot attribute)}

\begin{fulllineitems}
\phantomsection\label{\detokenize{ref/util/plot/Plot:TotalDepth.util.plot.Plot.Plot.CURVE_LEGEND_FONT_SIZE}}\pysigline{\sphinxbfcode{CURVE\_LEGEND\_FONT\_SIZE}\sphinxbfcode{ = 9}}
Legend font family.

\end{fulllineitems}

\index{MICRO\_MARGIN (TotalDepth.util.plot.Plot.Plot attribute)}

\begin{fulllineitems}
\phantomsection\label{\detokenize{ref/util/plot/Plot:TotalDepth.util.plot.Plot.Plot.MICRO_MARGIN}}\pysigline{\sphinxbfcode{MICRO\_MARGIN}\sphinxbfcode{ = Dim(value=0.04, units='in')}}
This is just a very small margin to provide a tiny bit of whitespace
between elements in certain places

\end{fulllineitems}

\index{MAX\_BACKUP\_TRACK\_CROSSING\_LINES (TotalDepth.util.plot.Plot.Plot attribute)}

\begin{fulllineitems}
\phantomsection\label{\detokenize{ref/util/plot/Plot:TotalDepth.util.plot.Plot.Plot.MAX_BACKUP_TRACK_CROSSING_LINES}}\pysigline{\sphinxbfcode{MAX\_BACKUP\_TRACK\_CROSSING\_LINES}\sphinxbfcode{ = 4}}
Maximum number of backup lines that can cross a single track in a single X step
See the source of \sphinxcode{\_filterCrossLineList()} for an explanation.

\end{fulllineitems}

\index{xScale() (TotalDepth.util.plot.Plot.Plot method)}

\begin{fulllineitems}
\phantomsection\label{\detokenize{ref/util/plot/Plot:TotalDepth.util.plot.Plot.Plot.xScale}}\pysiglinewithargsret{\sphinxbfcode{xScale}}{\emph{theFilmID}}{}
Returns the X axis scale as a number given the FILM ID.

\end{fulllineitems}

\index{filmIdS() (TotalDepth.util.plot.Plot.Plot method)}

\begin{fulllineitems}
\phantomsection\label{\detokenize{ref/util/plot/Plot:TotalDepth.util.plot.Plot.Plot.filmIdS}}\pysiglinewithargsret{\sphinxbfcode{filmIdS}}{}{}
Returns an unordered list of FILM IDs.

\end{fulllineitems}

\index{hasDataToPlotLIS() (TotalDepth.util.plot.Plot.Plot method)}

\begin{fulllineitems}
\phantomsection\label{\detokenize{ref/util/plot/Plot:TotalDepth.util.plot.Plot.Plot.hasDataToPlotLIS}}\pysiglinewithargsret{\sphinxbfcode{hasDataToPlotLIS}}{\emph{theLogPass}, \emph{theFilmId}}{}
Returns True if a call to plotLogPassLIS() is likely to lead to some
plot data being produced.

\end{fulllineitems}

\index{plotLogPassLIS() (TotalDepth.util.plot.Plot.Plot method)}

\begin{fulllineitems}
\phantomsection\label{\detokenize{ref/util/plot/Plot:TotalDepth.util.plot.Plot.Plot.plotLogPassLIS}}\pysiglinewithargsret{\sphinxbfcode{plotLogPassLIS}}{\emph{theLisFile}, \emph{theLogPass}, \emph{theXStart}, \emph{theXStop}, \emph{theFilmId}, \emph{theFpOut}, \emph{frameStep=1}, \emph{title=''}, \emph{lrCONS=None}, \emph{timerS=None}}{}
Plot a part of a LogPass and returns a list of Channel IDs plotted.
\begin{description}
\item[{\sphinxstyleemphasis{theLisFile}}] \leavevmode
The LIS File object.

\item[{\sphinxstyleemphasis{theLogPass}}] \leavevmode
A \sphinxcode{LogPass} object, the FrameSet will be populated here.

\item[{\sphinxstyleemphasis{theXStart}}] \leavevmode
The start X axis position as an \sphinxcode{EngVal}.

\item[{\sphinxstyleemphasis{theXStop}}] \leavevmode
The stop X axis position as an \sphinxcode{EngVal}.

\item[{\sphinxstyleemphasis{theFilmId}}] \leavevmode
The ID of the output device from the film table

\item[{\sphinxstyleemphasis{theFpOut}}] \leavevmode
A file path for the output SVG.

\item[{\sphinxstyleemphasis{frameStep}}] \leavevmode
Integer number of frame steps, 1 is all frames.

\item[{\sphinxstyleemphasis{title}}] \leavevmode
A string for the title that will appear in \sphinxcode{LEGEND\_DEPTH\_SPARE}

\item[{\sphinxstyleemphasis{lrCONS}}] \leavevmode
A \sphinxcode{CONS} Logical Record that will be used to plot an API header in SVG.

\item[{\sphinxstyleemphasis{timerS}}] \leavevmode
Optional \sphinxcode{ExecTimer.ExecTimerList} for performance measurement.

\end{description}

TODO: If title is empty do not use the space \sphinxcode{self.LEGEND\_DEPTH\_SPARE}

\end{fulllineitems}

\index{hasDataToPlotLAS() (TotalDepth.util.plot.Plot.Plot method)}

\begin{fulllineitems}
\phantomsection\label{\detokenize{ref/util/plot/Plot:TotalDepth.util.plot.Plot.Plot.hasDataToPlotLAS}}\pysiglinewithargsret{\sphinxbfcode{hasDataToPlotLAS}}{\emph{theLasFile}, \emph{theFilmId}}{}
Returns True if a call to plotLogPassLIS() is likely to lead to some
plot data being produced.

\end{fulllineitems}

\index{plotLogPassLAS() (TotalDepth.util.plot.Plot.Plot method)}

\begin{fulllineitems}
\phantomsection\label{\detokenize{ref/util/plot/Plot:TotalDepth.util.plot.Plot.Plot.plotLogPassLAS}}\pysiglinewithargsret{\sphinxbfcode{plotLogPassLAS}}{\emph{theLasFile}, \emph{theXStart}, \emph{theXStop}, \emph{theFilmId}, \emph{theFpOut}, \emph{frameStep=1}, \emph{title=''}, \emph{plotHeader=False}, \emph{timerS=None}}{}
Plot a part of a LogPass and returns a list of Channel IDs plotted.

theLisFile - The LIS File object.

theLogPass - A LogPass object, the FrameSet will be populated here.

theXStart - The start X axis position as an EngVal.

theXStop - The stop X axis position as an EngVal.

theFilmId - The ID of the output device from the film table

theFpOut - A file path for the output SVG.

frameStep - Integer number of frame steps, 1 is all frames.

title - A string for the title that will appear in LEGEND\_DEPTH\_SPARE

lrCONS - A CONS Logical Record that will be used to plot an API header in SVG.

timerS - Optional ExecTimer.ExecTimerList for performance measurement.

TODO: If title is empty do not use the space self.LEGEND\_DEPTH\_SPARE

\end{fulllineitems}

\index{\_\_weakref\_\_ (TotalDepth.util.plot.Plot.Plot attribute)}

\begin{fulllineitems}
\phantomsection\label{\detokenize{ref/util/plot/Plot:TotalDepth.util.plot.Plot.Plot.__weakref__}}\pysigline{\sphinxbfcode{\_\_weakref\_\_}}
list of weak references to the object (if defined)

\end{fulllineitems}


\end{fulllineitems}

\index{PlotReadLIS (class in TotalDepth.util.plot.Plot)}

\begin{fulllineitems}
\phantomsection\label{\detokenize{ref/util/plot/Plot:TotalDepth.util.plot.Plot.PlotReadLIS}}\pysiglinewithargsret{\sphinxbfcode{class }\sphinxcode{TotalDepth.util.plot.Plot.}\sphinxbfcode{PlotReadLIS}}{\emph{lrFILM}, \emph{lrPRES}, \emph{lrAREA=None}, \emph{lrPIP=None}, \emph{theScale=0}}{}
A subclass of Plot that is configured from FILM, PRES and (optionally) AREA, PIP Logical Records.

\end{fulllineitems}

\index{PlotReadXML (class in TotalDepth.util.plot.Plot)}

\begin{fulllineitems}
\phantomsection\label{\detokenize{ref/util/plot/Plot:TotalDepth.util.plot.Plot.PlotReadXML}}\pysiglinewithargsret{\sphinxbfcode{class }\sphinxcode{TotalDepth.util.plot.Plot.}\sphinxbfcode{PlotReadXML}}{\emph{uniqueId}, \emph{theScale=0}}{}
A subclass of Plot that is configured from XML file(s) using LgFormat.

\end{fulllineitems}



\section{A Bag Full of TODO’s}
\label{\detokenize{TODO::doc}}\label{\detokenize{TODO:a-bag-full-of-todo-s}}
This is just a gathering area for work items that should be done at some point.

TotalDepth is currently at \sphinxstylestrong{Alpha}, release version 0.2.0.


\subsection{Current Release}
\label{\detokenize{TODO:current-release}}
This lists known work (and area) that are known lacuna in the Alpha release.


\subsubsection{LIS Format Support}
\label{\detokenize{TODO:lis-format-support}}

\paragraph{General}
\label{\detokenize{TODO:general}}\begin{itemize}
\item {} 
XNAM direct support for LIS-A

\end{itemize}


\paragraph{LIS Index}
\label{\detokenize{TODO:lis-index}}\begin{itemize}
\item {} 
Various forms of persistence: XML, Pickle, JSON (no).

\item {} 
Insert or append binary representation of the index to the LIS file.

\end{itemize}


\paragraph{LIS FrameSet/LogPass}
\label{\detokenize{TODO:lis-frameset-logpass}}\begin{itemize}
\item {} 
Provide a numpy view on each sub-channel.

\item {} 
X Axis quality i.e. duplicate, missing frames.

\item {} 
Change up log to down log (needs to rearrange multi-sampled channels).

\item {} 
Clearer view on what ‘frame spacing’ actually is.

\end{itemize}


\paragraph{Representation Codes}
\label{\detokenize{TODO:representation-codes}}\begin{itemize}
\item {} 
Check overflow/underflow on write.

\end{itemize}


\subsubsection{LAS Format Support}
\label{\detokenize{TODO:las-format-support}}\begin{itemize}
\item {} 
Read directly from .zip files.

\item {} 
Version 3.0 support.

\item {} 
Merge \sphinxcode{\textasciitilde{}O} section int \sphinxcode{\textasciitilde{}P} if correct format.

\item {} 
Consistency checking of mutual data such as STRT/STOP/STEP

\end{itemize}


\subsubsection{Plotting}
\label{\detokenize{TODO:plotting}}\begin{itemize}
\item {} 
There is quite a lot of technical debt built up since we added LgFormat support, this area needs a review.

\item {} 
Header: Some mud parameters being dropped.

\item {} 
Benchmarks to characterise execution time and profiling.

\end{itemize}


\subsection{Future Releases}
\label{\detokenize{TODO:future-releases}}
This lists known work (and area) that are would be nice to have in future releases.


\subsubsection{Format Support}
\label{\detokenize{TODO:format-support}}

\paragraph{Wireline Formats}
\label{\detokenize{TODO:wireline-formats}}\begin{itemize}
\item {} 
RP66v1

\item {} 
RP66v2

\item {} 
WellLogML?

\item {} 
ATLAS BIT?

\end{itemize}


\paragraph{Seismic Formats}
\label{\detokenize{TODO:seismic-formats}}\begin{itemize}
\item {} 
SEG-Y

\item {} 
SEG-D?

\item {} 
Other SEG formats

\end{itemize}


\paragraph{Miscellaneous Formats}
\label{\detokenize{TODO:miscellaneous-formats}}\begin{itemize}
\item {} 
Position

\end{itemize}


\section{Glossary}
\label{\detokenize{glossary::doc}}\label{\detokenize{glossary:glossary}}\begin{description}
\item[{Backup Mode\index{Backup Mode|textbf}}] \leavevmode\phantomsection\label{\detokenize{glossary:term-backup-mode}}
A means of specifying what happens to plotted lines when they go off scale. Typical examples are None (all intermediate data is lost) and ‘wrap’ (all data is plotted with lines at modulo scale).

\item[{Data Format Specification Record\index{Data Format Specification Record|textbf}}] \leavevmode\phantomsection\label{\detokenize{glossary:term-data-format-specification-record}}
{[}{\hyperref[\detokenize{glossary:term-lis}]{\sphinxtermref{\DUrole{xref,std,std-term}{LIS}}}}{]} An EFLR that describes type 0 or type 1 binary IFLRs containing log data. A DFSR consists of a set of Entry Blocks followed by a list of Datum Specification Blocks.

NOTE: The LIS specification associates a particular DFSR type \sphinxcode{0 \textbar{} 1} with binary IFLRs of type \sphinxcode{0 \textbar{} 1}. These collections will be independent of each other and thus permits the simultaneous recording of entirely different data sets. In practice however this rarely happens and it is debatable whether any or all LIS processing implementations support this.

\item[{Datum Specification Block\index{Datum Specification Block|textbf}}] \leavevmode\phantomsection\label{\detokenize{glossary:term-datum-specification-block}}
{[}{\hyperref[\detokenize{glossary:term-lis}]{\sphinxtermref{\DUrole{xref,std,std-term}{LIS}}}}{]} A fixed format data block that defines the characteristics of a single, independent, data channel in a DFSR.

\item[{DFSR\index{DFSR|textbf}}] \leavevmode\phantomsection\label{\detokenize{glossary:term-dfsr}}
See {\hyperref[\detokenize{glossary:term-data-format-specification-record}]{\sphinxtermref{\DUrole{xref,std,std-term}{Data Format Specification Record}}}}

\item[{DLIS\index{DLIS|textbf}}] \leavevmode\phantomsection\label{\detokenize{glossary:term-dlis}}
This adds schema specific semantics to {\hyperref[\detokenize{glossary:term-rp66}]{\sphinxtermref{\DUrole{xref,std,std-term}{RP66}}}}.

\item[{DSB\index{DSB|textbf}}] \leavevmode\phantomsection\label{\detokenize{glossary:term-dsb}}
See {\hyperref[\detokenize{glossary:term-datum-specification-block}]{\sphinxtermref{\DUrole{xref,std,std-term}{Datum Specification Block}}}}.

\item[{EFLR\index{EFLR|textbf}}] \leavevmode\phantomsection\label{\detokenize{glossary:term-eflr}}
See {\hyperref[\detokenize{glossary:term-explicitly-formatted-logical-record}]{\sphinxtermref{\DUrole{xref,std,std-term}{Explicitly Formatted Logical Record}}}}.

\item[{Engineering Value\index{Engineering Value|textbf}}] \leavevmode\phantomsection\label{\detokenize{glossary:term-engineering-value}}
A numeric value together with its units of measure.

\item[{Entry Block\index{Entry Block|textbf}}] \leavevmode\phantomsection\label{\detokenize{glossary:term-entry-block}}
{[}{\hyperref[\detokenize{glossary:term-lis}]{\sphinxtermref{\DUrole{xref,std,std-term}{LIS}}}}{]} A variable length block of data that describes a static value in DFSR. This value is local to a Log Pass. For example and Entry Block might describe the NULL or absent value for any channel in a Log Pass.

\item[{Explicitly Formatted Logical Record\index{Explicitly Formatted Logical Record|textbf}}] \leavevmode\phantomsection\label{\detokenize{glossary:term-explicitly-formatted-logical-record}}
This is an internally complete, self-describing record.

\item[{Frame\index{Frame|textbf}}] \leavevmode\phantomsection\label{\detokenize{glossary:term-frame}}
An array of values for each channel at a particular depth (or time).

\item[{Frame Set\index{Frame Set|textbf}}] \leavevmode\phantomsection\label{\detokenize{glossary:term-frame-set}}
A set of frames representing multi-channel data that is typically depth or time series based.

\item[{IFLR\index{IFLR|textbf}}] \leavevmode\phantomsection\label{\detokenize{glossary:term-iflr}}
See {\hyperref[\detokenize{glossary:term-indirectly-formatted-logical-record}]{\sphinxtermref{\DUrole{xref,std,std-term}{Indirectly Formatted Logical Record}}}}.

\item[{Indirectly Formatted Logical Record\index{Indirectly Formatted Logical Record|textbf}}] \leavevmode\phantomsection\label{\detokenize{glossary:term-indirectly-formatted-logical-record}}
This is a Logical Record whose format is only completely described by another EFLR. The EFLR that describes an IFLR might be identified formally; for example by a specific reference to an EFLR (as in RP66) or informally; by some heuristic (as in LIS) such as “the immediately prior Logical Record that is type 64”.

\item[{LIS\index{LIS|textbf}}] \leavevmode\phantomsection\label{\detokenize{glossary:term-lis}}\item[{LIS-79\index{LIS-79|textbf}}] \leavevmode\phantomsection\label{\detokenize{glossary:term-lis-79}}
Log Information Standard. The original file format used by Schlumberger from 1979 onwards.
It is a \sphinxstyleemphasis{proprietary}, \sphinxstyleemphasis{de-facto} standard.

\item[{LIS-A\index{LIS-A|textbf}}] \leavevmode\phantomsection\label{\detokenize{glossary:term-lis-a}}
\sphinxstyleemphasis{Enhanced} Log Information Standard. This adds somewhat to the {[}{\hyperref[\detokenize{glossary:term-lis-79}]{\sphinxtermref{\DUrole{xref,std,std-term}{LIS-79}}}}{]} standard.

\item[{Log Pass\index{Log Pass|textbf}}] \leavevmode\phantomsection\label{\detokenize{glossary:term-log-pass}}
A TotalDepth term that describes a continuos body of logging data such as “Repeat Section” or “Main Log”.
In the {\hyperref[\detokenize{glossary:term-lis}]{\sphinxtermref{\DUrole{xref,std,std-term}{LIS}}}} format this is defined by a single Logical Record, the {\hyperref[\detokenize{glossary:term-dfsr}]{\sphinxtermref{\DUrole{xref,std,std-term}{DFSR}}}}, plus multiple type 0 or type 1 Logical Records that the DFSR describes.

\item[{Logical Record\index{Logical Record|textbf}}] \leavevmode\phantomsection\label{\detokenize{glossary:term-logical-record}}
{[}{\hyperref[\detokenize{glossary:term-lis}]{\sphinxtermref{\DUrole{xref,std,std-term}{LIS}}}}{]} A formal record from a LIS file. Logical Records consist of a header and optional payload. The Logical Records \sphinxstyleemphasis{type} is identified in the header. The interpretation of the payload of (some) Logical Records types is defined in the LIS standard. Logical Records consist of the payloads of one or more Physical Records. Logical Records are either EFLR or IFLR records.

\item[{LRH\index{LRH|textbf}}] \leavevmode\phantomsection\label{\detokenize{glossary:term-lrh}}\item[{Logical Record header\index{Logical Record header|textbf}}] \leavevmode\phantomsection\label{\detokenize{glossary:term-logical-record-header}}
The bytes that describe the type and attributes of a Logical Record.

\item[{Mnem\index{Mnem|textbf}}] \leavevmode\phantomsection\label{\detokenize{glossary:term-mnem}}
See {\hyperref[\detokenize{glossary:term-mnemonic}]{\sphinxtermref{\DUrole{xref,std,std-term}{Mnemonic}}}}.

\item[{Mnemonic\index{Mnemonic|textbf}}] \leavevmode\phantomsection\label{\detokenize{glossary:term-mnemonic}}
{[}{\hyperref[\detokenize{glossary:term-lis}]{\sphinxtermref{\DUrole{xref,std,std-term}{LIS}}}}{]} A four byte identifier that is human readable e.g. \sphinxcode{RHOB} for Bulk Density.

\item[{Physical Record\index{Physical Record|textbf}}] \leavevmode\phantomsection\label{\detokenize{glossary:term-physical-record}}
{[}{\hyperref[\detokenize{glossary:term-lis}]{\sphinxtermref{\DUrole{xref,std,std-term}{LIS}}}}{]} A formal record in a LIS file. Physical Records consist of a header, optional payload and optional trailer. Logical Records consist of the payloads of one or more Physical Records.

\item[{RepCode\index{RepCode|textbf}}] \leavevmode\phantomsection\label{\detokenize{glossary:term-repcode}}
See {\hyperref[\detokenize{glossary:term-representation-code}]{\sphinxtermref{\DUrole{xref,std,std-term}{Representation Code}}}}.

\item[{Representation Code\index{Representation Code|textbf}}] \leavevmode\phantomsection\label{\detokenize{glossary:term-representation-code}}
A code, usually an integer, that describes how a particular run of bytes should be interpreted as a
value (number, string etc.). For example in the LIS standard representation code 68 describes a 32 bit
floating point format.

\item[{RP66\index{RP66|textbf}}] \leavevmode\phantomsection\label{\detokenize{glossary:term-rp66}}
\sphinxstyleemphasis{Recommended Practice 66} is an API standard that describes a more advanced file format for, among other things, wireline logs. Comes in two flavours version 1 and version 2. Often (and incorrectly) referred to as {\hyperref[\detokenize{glossary:term-dlis}]{\sphinxtermref{\DUrole{xref,std,std-term}{DLIS}}}}.

\item[{UOM\index{UOM|textbf}}] \leavevmode\phantomsection\label{\detokenize{glossary:term-uom}}
\sphinxstyleemphasis{Unit of Measure} for engineering values. In {\hyperref[\detokenize{glossary:term-lis}]{\sphinxtermref{\DUrole{xref,std,std-term}{LIS}}}} these are a set of fixed terms organised into several categories, such as \sphinxstyleemphasis{Linear Length}. Values can only be converted between units of in the same category. In {\hyperref[\detokenize{glossary:term-rp66}]{\sphinxtermref{\DUrole{xref,std,std-term}{RP66}}}} these are composed by a BNF parseable string.

\item[{Xaxis\index{Xaxis|textbf}}] \leavevmode\phantomsection\label{\detokenize{glossary:term-xaxis}}\item[{X Axis\index{X Axis|textbf}}] \leavevmode\phantomsection\label{\detokenize{glossary:term-x-axis}}
The index channel in an array, for example an array of frames. Typically depth or time.

\end{description}


\section{TotalDepth Licence}
\label{\detokenize{licence:licence-text}}\label{\detokenize{licence::doc}}\label{\detokenize{licence:totaldepth-licence}}

\subsection{OpenSource Licence}
\label{\detokenize{licence:opensource-licence}}
TotalDepth is released under GPL 2.:

\begin{sphinxVerbatim}[commandchars=\\\{\}]
                    GNU GENERAL PUBLIC LICENSE
                       Version 2, June 1991

 Copyright (C) 1989, 1991 Free Software Foundation, Inc.,
 51 Franklin Street, Fifth Floor, Boston, MA 02110\PYGZhy{}1301 USA
 Everyone is permitted to copy and distribute verbatim copies
 of this license document, but changing it is not allowed.

                            Preamble

  The licenses for most software are designed to take away your
freedom to share and change it.  By contrast, the GNU General Public
License is intended to guarantee your freedom to share and change free
software\PYGZhy{}\PYGZhy{}to make sure the software is free for all its users.  This
General Public License applies to most of the Free Software
Foundation\PYGZsq{}s software and to any other program whose authors commit to
using it.  (Some other Free Software Foundation software is covered by
the GNU Lesser General Public License instead.)  You can apply it to
your programs, too.

  When we speak of free software, we are referring to freedom, not
price.  Our General Public Licenses are designed to make sure that you
have the freedom to distribute copies of free software (and charge for
this service if you wish), that you receive source code or can get it
if you want it, that you can change the software or use pieces of it
in new free programs; and that you know you can do these things.

  To protect your rights, we need to make restrictions that forbid
anyone to deny you these rights or to ask you to surrender the rights.
These restrictions translate to certain responsibilities for you if you
distribute copies of the software, or if you modify it.

  For example, if you distribute copies of such a program, whether
gratis or for a fee, you must give the recipients all the rights that
you have.  You must make sure that they, too, receive or can get the
source code.  And you must show them these terms so they know their
rights.

  We protect your rights with two steps: (1) copyright the software, and
(2) offer you this license which gives you legal permission to copy,
distribute and/or modify the software.

  Also, for each author\PYGZsq{}s protection and ours, we want to make certain
that everyone understands that there is no warranty for this free
software.  If the software is modified by someone else and passed on, we
want its recipients to know that what they have is not the original, so
that any problems introduced by others will not reflect on the original
authors\PYGZsq{} reputations.

  Finally, any free program is threatened constantly by software
patents.  We wish to avoid the danger that redistributors of a free
program will individually obtain patent licenses, in effect making the
program proprietary.  To prevent this, we have made it clear that any
patent must be licensed for everyone\PYGZsq{}s free use or not licensed at all.

  The precise terms and conditions for copying, distribution and
modification follow.

                    GNU GENERAL PUBLIC LICENSE
   TERMS AND CONDITIONS FOR COPYING, DISTRIBUTION AND MODIFICATION

  0. This License applies to any program or other work which contains
a notice placed by the copyright holder saying it may be distributed
under the terms of this General Public License.  The \PYGZdq{}Program\PYGZdq{}, below,
refers to any such program or work, and a \PYGZdq{}work based on the Program\PYGZdq{}
means either the Program or any derivative work under copyright law:
that is to say, a work containing the Program or a portion of it,
either verbatim or with modifications and/or translated into another
language.  (Hereinafter, translation is included without limitation in
the term \PYGZdq{}modification\PYGZdq{}.)  Each licensee is addressed as \PYGZdq{}you\PYGZdq{}.

Activities other than copying, distribution and modification are not
covered by this License; they are outside its scope.  The act of
running the Program is not restricted, and the output from the Program
is covered only if its contents constitute a work based on the
Program (independent of having been made by running the Program).
Whether that is true depends on what the Program does.

  1. You may copy and distribute verbatim copies of the Program\PYGZsq{}s
source code as you receive it, in any medium, provided that you
conspicuously and appropriately publish on each copy an appropriate
copyright notice and disclaimer of warranty; keep intact all the
notices that refer to this License and to the absence of any warranty;
and give any other recipients of the Program a copy of this License
along with the Program.

You may charge a fee for the physical act of transferring a copy, and
you may at your option offer warranty protection in exchange for a fee.

  2. You may modify your copy or copies of the Program or any portion
of it, thus forming a work based on the Program, and copy and
distribute such modifications or work under the terms of Section 1
above, provided that you also meet all of these conditions:

    a) You must cause the modified files to carry prominent notices
    stating that you changed the files and the date of any change.

    b) You must cause any work that you distribute or publish, that in
    whole or in part contains or is derived from the Program or any
    part thereof, to be licensed as a whole at no charge to all third
    parties under the terms of this License.

    c) If the modified program normally reads commands interactively
    when run, you must cause it, when started running for such
    interactive use in the most ordinary way, to print or display an
    announcement including an appropriate copyright notice and a
    notice that there is no warranty (or else, saying that you provide
    a warranty) and that users may redistribute the program under
    these conditions, and telling the user how to view a copy of this
    License.  (Exception: if the Program itself is interactive but
    does not normally print such an announcement, your work based on
    the Program is not required to print an announcement.)

These requirements apply to the modified work as a whole.  If
identifiable sections of that work are not derived from the Program,
and can be reasonably considered independent and separate works in
themselves, then this License, and its terms, do not apply to those
sections when you distribute them as separate works.  But when you
distribute the same sections as part of a whole which is a work based
on the Program, the distribution of the whole must be on the terms of
this License, whose permissions for other licensees extend to the
entire whole, and thus to each and every part regardless of who wrote it.

Thus, it is not the intent of this section to claim rights or contest
your rights to work written entirely by you; rather, the intent is to
exercise the right to control the distribution of derivative or
collective works based on the Program.

In addition, mere aggregation of another work not based on the Program
with the Program (or with a work based on the Program) on a volume of
a storage or distribution medium does not bring the other work under
the scope of this License.

  3. You may copy and distribute the Program (or a work based on it,
under Section 2) in object code or executable form under the terms of
Sections 1 and 2 above provided that you also do one of the following:

    a) Accompany it with the complete corresponding machine\PYGZhy{}readable
    source code, which must be distributed under the terms of Sections
    1 and 2 above on a medium customarily used for software interchange; or,

    b) Accompany it with a written offer, valid for at least three
    years, to give any third party, for a charge no more than your
    cost of physically performing source distribution, a complete
    machine\PYGZhy{}readable copy of the corresponding source code, to be
    distributed under the terms of Sections 1 and 2 above on a medium
    customarily used for software interchange; or,

    c) Accompany it with the information you received as to the offer
    to distribute corresponding source code.  (This alternative is
    allowed only for noncommercial distribution and only if you
    received the program in object code or executable form with such
    an offer, in accord with Subsection b above.)

The source code for a work means the preferred form of the work for
making modifications to it.  For an executable work, complete source
code means all the source code for all modules it contains, plus any
associated interface definition files, plus the scripts used to
control compilation and installation of the executable.  However, as a
special exception, the source code distributed need not include
anything that is normally distributed (in either source or binary
form) with the major components (compiler, kernel, and so on) of the
operating system on which the executable runs, unless that component
itself accompanies the executable.

If distribution of executable or object code is made by offering
access to copy from a designated place, then offering equivalent
access to copy the source code from the same place counts as
distribution of the source code, even though third parties are not
compelled to copy the source along with the object code.

  4. You may not copy, modify, sublicense, or distribute the Program
except as expressly provided under this License.  Any attempt
otherwise to copy, modify, sublicense or distribute the Program is
void, and will automatically terminate your rights under this License.
However, parties who have received copies, or rights, from you under
this License will not have their licenses terminated so long as such
parties remain in full compliance.

  5. You are not required to accept this License, since you have not
signed it.  However, nothing else grants you permission to modify or
distribute the Program or its derivative works.  These actions are
prohibited by law if you do not accept this License.  Therefore, by
modifying or distributing the Program (or any work based on the
Program), you indicate your acceptance of this License to do so, and
all its terms and conditions for copying, distributing or modifying
the Program or works based on it.

  6. Each time you redistribute the Program (or any work based on the
Program), the recipient automatically receives a license from the
original licensor to copy, distribute or modify the Program subject to
these terms and conditions.  You may not impose any further
restrictions on the recipients\PYGZsq{} exercise of the rights granted herein.
You are not responsible for enforcing compliance by third parties to
this License.

  7. If, as a consequence of a court judgment or allegation of patent
infringement or for any other reason (not limited to patent issues),
conditions are imposed on you (whether by court order, agreement or
otherwise) that contradict the conditions of this License, they do not
excuse you from the conditions of this License.  If you cannot
distribute so as to satisfy simultaneously your obligations under this
License and any other pertinent obligations, then as a consequence you
may not distribute the Program at all.  For example, if a patent
license would not permit royalty\PYGZhy{}free redistribution of the Program by
all those who receive copies directly or indirectly through you, then
the only way you could satisfy both it and this License would be to
refrain entirely from distribution of the Program.

If any portion of this section is held invalid or unenforceable under
any particular circumstance, the balance of the section is intended to
apply and the section as a whole is intended to apply in other
circumstances.

It is not the purpose of this section to induce you to infringe any
patents or other property right claims or to contest validity of any
such claims; this section has the sole purpose of protecting the
integrity of the free software distribution system, which is
implemented by public license practices.  Many people have made
generous contributions to the wide range of software distributed
through that system in reliance on consistent application of that
system; it is up to the author/donor to decide if he or she is willing
to distribute software through any other system and a licensee cannot
impose that choice.

This section is intended to make thoroughly clear what is believed to
be a consequence of the rest of this License.

  8. If the distribution and/or use of the Program is restricted in
certain countries either by patents or by copyrighted interfaces, the
original copyright holder who places the Program under this License
may add an explicit geographical distribution limitation excluding
those countries, so that distribution is permitted only in or among
countries not thus excluded.  In such case, this License incorporates
the limitation as if written in the body of this License.

  9. The Free Software Foundation may publish revised and/or new versions
of the General Public License from time to time.  Such new versions will
be similar in spirit to the present version, but may differ in detail to
address new problems or concerns.

Each version is given a distinguishing version number.  If the Program
specifies a version number of this License which applies to it and \PYGZdq{}any
later version\PYGZdq{}, you have the option of following the terms and conditions
either of that version or of any later version published by the Free
Software Foundation.  If the Program does not specify a version number of
this License, you may choose any version ever published by the Free Software
Foundation.

  10. If you wish to incorporate parts of the Program into other free
programs whose distribution conditions are different, write to the author
to ask for permission.  For software which is copyrighted by the Free
Software Foundation, write to the Free Software Foundation; we sometimes
make exceptions for this.  Our decision will be guided by the two goals
of preserving the free status of all derivatives of our free software and
of promoting the sharing and reuse of software generally.

                            NO WARRANTY

  11. BECAUSE THE PROGRAM IS LICENSED FREE OF CHARGE, THERE IS NO WARRANTY
FOR THE PROGRAM, TO THE EXTENT PERMITTED BY APPLICABLE LAW.  EXCEPT WHEN
OTHERWISE STATED IN WRITING THE COPYRIGHT HOLDERS AND/OR OTHER PARTIES
PROVIDE THE PROGRAM \PYGZdq{}AS IS\PYGZdq{} WITHOUT WARRANTY OF ANY KIND, EITHER EXPRESSED
OR IMPLIED, INCLUDING, BUT NOT LIMITED TO, THE IMPLIED WARRANTIES OF
MERCHANTABILITY AND FITNESS FOR A PARTICULAR PURPOSE.  THE ENTIRE RISK AS
TO THE QUALITY AND PERFORMANCE OF THE PROGRAM IS WITH YOU.  SHOULD THE
PROGRAM PROVE DEFECTIVE, YOU ASSUME THE COST OF ALL NECESSARY SERVICING,
REPAIR OR CORRECTION.

  12. IN NO EVENT UNLESS REQUIRED BY APPLICABLE LAW OR AGREED TO IN WRITING
WILL ANY COPYRIGHT HOLDER, OR ANY OTHER PARTY WHO MAY MODIFY AND/OR
REDISTRIBUTE THE PROGRAM AS PERMITTED ABOVE, BE LIABLE TO YOU FOR DAMAGES,
INCLUDING ANY GENERAL, SPECIAL, INCIDENTAL OR CONSEQUENTIAL DAMAGES ARISING
OUT OF THE USE OR INABILITY TO USE THE PROGRAM (INCLUDING BUT NOT LIMITED
TO LOSS OF DATA OR DATA BEING RENDERED INACCURATE OR LOSSES SUSTAINED BY
YOU OR THIRD PARTIES OR A FAILURE OF THE PROGRAM TO OPERATE WITH ANY OTHER
PROGRAMS), EVEN IF SUCH HOLDER OR OTHER PARTY HAS BEEN ADVISED OF THE
POSSIBILITY OF SUCH DAMAGES.

                     END OF TERMS AND CONDITIONS

            How to Apply These Terms to Your New Programs

  If you develop a new program, and you want it to be of the greatest
possible use to the public, the best way to achieve this is to make it
free software which everyone can redistribute and change under these terms.

  To do so, attach the following notices to the program.  It is safest
to attach them to the start of each source file to most effectively
convey the exclusion of warranty; and each file should have at least
the \PYGZdq{}copyright\PYGZdq{} line and a pointer to where the full notice is found.

    \PYGZlt{}one line to give the program\PYGZsq{}s name and a brief idea of what it does.\PYGZgt{}
    Copyright (C) \PYGZlt{}year\PYGZgt{}  \PYGZlt{}name of author\PYGZgt{}

    This program is free software; you can redistribute it and/or modify
    it under the terms of the GNU General Public License as published by
    the Free Software Foundation; either version 2 of the License, or
    (at your option) any later version.

    This program is distributed in the hope that it will be useful,
    but WITHOUT ANY WARRANTY; without even the implied warranty of
    MERCHANTABILITY or FITNESS FOR A PARTICULAR PURPOSE.  See the
    GNU General Public License for more details.

    You should have received a copy of the GNU General Public License along
    with this program; if not, write to the Free Software Foundation, Inc.,
    51 Franklin Street, Fifth Floor, Boston, MA 02110\PYGZhy{}1301 USA.

Also add information on how to contact you by electronic and paper mail.

If the program is interactive, make it output a short notice like this
when it starts in an interactive mode:

    Gnomovision version 69, Copyright (C) year name of author
    Gnomovision comes with ABSOLUTELY NO WARRANTY; for details type {}`show w\PYGZsq{}.
    This is free software, and you are welcome to redistribute it
    under certain conditions; type {}`show c\PYGZsq{} for details.

The hypothetical commands {}`show w\PYGZsq{} and {}`show c\PYGZsq{} should show the appropriate
parts of the General Public License.  Of course, the commands you use may
be called something other than {}`show w\PYGZsq{} and {}`show c\PYGZsq{}; they could even be
mouse\PYGZhy{}clicks or menu items\PYGZhy{}\PYGZhy{}whatever suits your program.

You should also get your employer (if you work as a programmer) or your
school, if any, to sign a \PYGZdq{}copyright disclaimer\PYGZdq{} for the program, if
necessary.  Here is a sample; alter the names:

  Yoyodyne, Inc., hereby disclaims all copyright interest in the program
  {}`Gnomovision\PYGZsq{} (which makes passes at compilers) written by James Hacker.

  \PYGZlt{}signature of Ty Coon\PYGZgt{}, 1 April 1989
  Ty Coon, President of Vice

This General Public License does not permit incorporating your program into
proprietary programs.  If your program is a subroutine library, you may
consider it more useful to permit linking proprietary applications with the
library.  If this is what you want to do, use the GNU Lesser General
Public License instead of this License.
\end{sphinxVerbatim}


\subsection{Other Licences}
\label{\detokenize{licence:other-licences}}
For other licences contact: \sphinxhref{mailto:apaulross@gmail.com}{apaulross@gmail.com}


\section{Credits}
\label{\detokenize{authors:credits}}\label{\detokenize{authors::doc}}

\subsection{Development Lead}
\label{\detokenize{authors:development-lead}}\begin{itemize}
\item {} 
Paul Ross \textless{}\sphinxhref{mailto:apaulross@gmail.com}{apaulross@gmail.com}\textgreater{}

\end{itemize}


\subsection{Contributors}
\label{\detokenize{authors:contributors}}
None yet. Why not be the first?


\section{Contributing}
\label{\detokenize{contributing:contributing}}\label{\detokenize{contributing::doc}}
Contributions are welcome, and they are greatly appreciated! Every
little bit helps, and credit will always be given.

You can contribute in many ways:


\subsection{Types of Contributions}
\label{\detokenize{contributing:types-of-contributions}}

\subsubsection{Report Bugs}
\label{\detokenize{contributing:report-bugs}}
Report bugs at \sphinxurl{https://github.com/paulross/TotalDepth/issues}.

If you are reporting a bug, please include:
\begin{itemize}
\item {} 
Your operating system name and version.

\item {} 
Any details about your local setup that might be helpful in troubleshooting.

\item {} 
Detailed steps to reproduce the bug.

\end{itemize}


\subsubsection{Fix Bugs}
\label{\detokenize{contributing:fix-bugs}}
Look through the GitHub issues for bugs. Anything tagged with “bug”
and “help wanted” is open to whoever wants to implement it.


\subsubsection{Implement Features}
\label{\detokenize{contributing:implement-features}}
Look through the GitHub issues for features. Anything tagged with “enhancement”
and “help wanted” is open to whoever wants to implement it.


\subsubsection{Write Documentation}
\label{\detokenize{contributing:write-documentation}}
TotalDepth could always use more documentation, whether as part of the
official TotalDepth docs, in docstrings, or even on the web in blog posts,
articles, and such.


\subsubsection{Submit Feedback}
\label{\detokenize{contributing:submit-feedback}}
The best way to send feedback is to file an issue at \sphinxurl{https://github.com/paulross/TotalDepth/issues}.

If you are proposing a feature:
\begin{itemize}
\item {} 
Explain in detail how it would work.

\item {} 
Keep the scope as narrow as possible, to make it easier to implement.

\item {} 
Remember that this is a volunteer-driven project, and that contributions
are welcome :)

\end{itemize}


\subsection{Get Started!}
\label{\detokenize{contributing:get-started}}
Ready to contribute? Here’s how to set up \sphinxtitleref{TotalDepth} for local development.
\begin{enumerate}
\item {} 
Fork the \sphinxtitleref{TotalDepth} repo on GitHub.

\item {} 
Clone your fork locally:

\begin{sphinxVerbatim}[commandchars=\\\{\}]
\PYGZdl{} git clone git@github.com:paulross/TotalDepth.git
\end{sphinxVerbatim}

\item {} 
Install your local copy into a virtualenv. Assuming you have virtualenvwrapper installed, this is how you set up your fork for local development:

\begin{sphinxVerbatim}[commandchars=\\\{\}]
\PYGZdl{} mkvirtualenv TotalDepth
\PYGZdl{} \PYG{n+nb}{cd} TotalDepth/
\PYGZdl{} python setup.py develop
\end{sphinxVerbatim}

\item {} 
Create a branch for local development:

\begin{sphinxVerbatim}[commandchars=\\\{\}]
\PYGZdl{} git checkout \PYGZhy{}b name\PYGZhy{}of\PYGZhy{}your\PYGZhy{}bugfix\PYGZhy{}or\PYGZhy{}feature
\end{sphinxVerbatim}

Now you can make your changes locally.

\item {} 
When you’re done making changes, check that your changes pass flake8 and the tests, including testing other Python versions with tox:

\begin{sphinxVerbatim}[commandchars=\\\{\}]
\PYGZdl{} flake8 TotalDepth tests
\PYGZdl{} python setup.py \PYG{n+nb}{test} or py.test
\PYGZdl{} tox
\end{sphinxVerbatim}

To get flake8 and tox, just pip install them into your virtualenv.

\item {} 
Commit your changes and push your branch to GitHub:

\begin{sphinxVerbatim}[commandchars=\\\{\}]
\PYGZdl{} git add .
\PYGZdl{} git commit \PYGZhy{}m \PYG{l+s+s2}{\PYGZdq{}Your detailed description of your changes.\PYGZdq{}}
\PYGZdl{} git push origin name\PYGZhy{}of\PYGZhy{}your\PYGZhy{}bugfix\PYGZhy{}or\PYGZhy{}feature
\end{sphinxVerbatim}

\item {} 
Submit a pull request through the GitHub website.

\end{enumerate}


\subsection{Pull Request Guidelines}
\label{\detokenize{contributing:pull-request-guidelines}}
Before you submit a pull request, check that it meets these guidelines:
\begin{enumerate}
\item {} 
The pull request should include tests.

\item {} 
If the pull request adds functionality, the docs should be updated. Put
your new functionality into a function with a docstring, and add the
feature to the list in README.rst.

\item {} 
The pull request should work for Python 3.3 and above (we don’t support Python 2.x), and for PyPy. Check
\sphinxurl{https://travis-ci.org/paulross/TotalDepth/pull\_requests}
and make sure that the tests pass for all supported Python versions.

\end{enumerate}


\subsection{Tips}
\label{\detokenize{contributing:tips}}
To run a subset of tests:

\begin{sphinxVerbatim}[commandchars=\\\{\}]
\PYGZdl{} pytest tests.test\PYGZus{}TotalDepth
\end{sphinxVerbatim}


\section{History}
\label{\detokenize{history::doc}}\label{\detokenize{history:history}}

\subsection{0.2.0 (2017-09-25)}
\label{\detokenize{history:id1}}\begin{itemize}
\item {} 
Moved to Github: \sphinxurl{https://github.com/paulross/TotalDepth}

\item {} 
First release on PyPI.

\end{itemize}


\subsection{0.1.0 (2012-03-03)}
\label{\detokenize{history:id2}}\begin{itemize}
\item {} 
First release on Sourceforge: \sphinxurl{https://sourceforge.net/projects/TotalDepth/} registered: 2011-10-02

\end{itemize}

Earlier versions (unreleased):
\begin{itemize}
\item {} 
OpenLis - 2010-11-11 to 2011-08-01

\item {} 
PyLis - 2009 to 2010

\end{itemize}


\section{TotalDepth}
\label{\detokenize{readme::doc}}\label{\detokenize{readme:totaldepth}}\sphinxhref{https://pypi.python.org/pypi/TotalDepth}{\sphinxincludegraphics{{/Users/paulross/Documents/workspace/git/src/paulross/TotalDepth/docs/_build/doctrees/images/https/img.shields.io/pypi/v/TotalDepth.svg/TotalDepth}.svg}}\sphinxhref{https://travis-ci.org/paulross/TotalDepth}{\sphinxincludegraphics{{/Users/paulross/Documents/workspace/git/src/paulross/TotalDepth/docs/_build/doctrees/images/https/img.shields.io/travis/paulross/TotalDepth.svg/TotalDepth}.svg}}\sphinxhref{https://TotalDepth.readthedocs.io/en/latest/?badge=latest}{\sphinxincludegraphics{{/Users/paulross/Documents/workspace/git/src/paulross/TotalDepth/docs/_build/doctrees/images/https/readthedocs.org/projects/TotalDepth/badge//version=latest/604aee7c87fb6496c0242fe82980accdf6788589}.svg}}\sphinxhref{https://pyup.io/repos/github/paulross/TotalDepth/}{}
Petrophysical software capable of processing wireline logs.
\begin{itemize}
\item {} 
Free software: GPL 2.0

\item {} 
Documentation: \sphinxurl{https://TotalDepth.readthedocs.io}

\end{itemize}


\subsection{Features}
\label{\detokenize{readme:features}}\begin{itemize}
\item {} 
Reads LIS, LAS (1.2, 2.0) file formats for analysis or conversion to other formats.

\item {} 
Plots log data as SVG viewable in any modern browser.

\item {} 
Plots can be made with a wide variety of plot formats.

\item {} 
TotalDepth can generate HTML summaries of log data.

\item {} 
TotalDepth is written in Python so it is fast to develop with.

\item {} 
Special indexing techniques are used to be able to randomly access sequential files.

\end{itemize}

Here is an example of a LAS file plotted with the Tripple Combo plot format as seen in a browser, this includes the API header:

\noindent\sphinxincludegraphics{{images/TrippleCombo}.png}

An example of a High Resolution Dipmeter plotted at 1:25 scale:

\noindent\sphinxincludegraphics{{images/HDT_25_no_hdr}.png}


\subsection{Credits}
\label{\detokenize{readme:credits}}
This package was created with \sphinxhref{https://github.com/audreyr/cookiecutter}{Cookiecutter} and the \sphinxhref{https://github.com/audreyr/cookiecutter-pypackage}{audreyr/cookiecutter-pypackage} project template.

For many year this project was hosted by \sphinxhref{https://sourceforge.net/projects/TotalDepth/}{Sourceforge}. Thank you Sourceforge!


\chapter{Indices and tables}
\label{\detokenize{index:indices-and-tables}}\label{\detokenize{index:audreyr-cookiecutter-pypackage}}\begin{itemize}
\item {} 
\DUrole{xref,std,std-ref}{genindex}

\item {} 
\DUrole{xref,std,std-ref}{modindex}

\item {} 
\DUrole{xref,std,std-ref}{search}

\end{itemize}


\renewcommand{\indexname}{Python Module Index}
\begin{sphinxtheindex}
\def\bigletter#1{{\Large\sffamily#1}\nopagebreak\vspace{1mm}}
\bigletter{t}
\item {\sphinxstyleindexentry{TotalDepth.LAS.core.LASRead}}\sphinxstyleindexpageref{ref/LAS/core/LASRead:\detokenize{module-TotalDepth.LAS.core.LASRead}}
\item {\sphinxstyleindexentry{TotalDepth.LIS.core.cRepCode}}\sphinxstyleindexpageref{ref/LIS/core/RepCode:\detokenize{module-TotalDepth.LIS.core.cRepCode}}
\item {\sphinxstyleindexentry{TotalDepth.LIS.core.EngVal}}\sphinxstyleindexpageref{ref/LIS/core/EngVal:\detokenize{module-TotalDepth.LIS.core.EngVal}}
\item {\sphinxstyleindexentry{TotalDepth.LIS.core.File}}\sphinxstyleindexpageref{ref/LIS/core/File:\detokenize{module-TotalDepth.LIS.core.File}}
\item {\sphinxstyleindexentry{TotalDepth.LIS.core.FileIndexer}}\sphinxstyleindexpageref{ref/LIS/core/FileIndexer:\detokenize{module-TotalDepth.LIS.core.FileIndexer}}
\item {\sphinxstyleindexentry{TotalDepth.LIS.core.FrameSet}}\sphinxstyleindexpageref{ref/LIS/core/FrameSet:\detokenize{module-TotalDepth.LIS.core.FrameSet}}
\item {\sphinxstyleindexentry{TotalDepth.LIS.core.LogiRec}}\sphinxstyleindexpageref{ref/LIS/core/LogiRec:\detokenize{module-TotalDepth.LIS.core.LogiRec}}
\item {\sphinxstyleindexentry{TotalDepth.LIS.core.LogPass}}\sphinxstyleindexpageref{ref/LIS/core/LogPass:\detokenize{module-TotalDepth.LIS.core.LogPass}}
\item {\sphinxstyleindexentry{TotalDepth.LIS.core.Mnem}}\sphinxstyleindexpageref{ref/LIS/core/Mnem:\detokenize{module-TotalDepth.LIS.core.Mnem}}
\item {\sphinxstyleindexentry{TotalDepth.LIS.core.PhysRec}}\sphinxstyleindexpageref{ref/LIS/core/PhysRec:\detokenize{module-TotalDepth.LIS.core.PhysRec}}
\item {\sphinxstyleindexentry{TotalDepth.LIS.core.pRepCode}}\sphinxstyleindexpageref{ref/LIS/core/RepCode:\detokenize{module-TotalDepth.LIS.core.pRepCode}}
\item {\sphinxstyleindexentry{TotalDepth.LIS.core.RawStream}}\sphinxstyleindexpageref{ref/LIS/core/RawStream:\detokenize{module-TotalDepth.LIS.core.RawStream}}
\item {\sphinxstyleindexentry{TotalDepth.LIS.core.RepCode}}\sphinxstyleindexpageref{ref/LIS/core/RepCode:\detokenize{module-TotalDepth.LIS.core.RepCode}}
\item {\sphinxstyleindexentry{TotalDepth.LIS.core.Rle}}\sphinxstyleindexpageref{ref/LIS/core/Rle:\detokenize{module-TotalDepth.LIS.core.Rle}}
\item {\sphinxstyleindexentry{TotalDepth.LIS.core.TifMarker}}\sphinxstyleindexpageref{ref/LIS/core/TifMarker:\detokenize{module-TotalDepth.LIS.core.TifMarker}}
\item {\sphinxstyleindexentry{TotalDepth.LIS.core.Type01Plan}}\sphinxstyleindexpageref{ref/LIS/core/Type01Plan:\detokenize{module-TotalDepth.LIS.core.Type01Plan}}
\item {\sphinxstyleindexentry{TotalDepth.LIS.core.Units}}\sphinxstyleindexpageref{ref/LIS/core/Units:\detokenize{module-TotalDepth.LIS.core.Units}}
\item {\sphinxstyleindexentry{TotalDepth.util.CmnCmdOpts}}\sphinxstyleindexpageref{ref/util/CmnCmdOpts:\detokenize{module-TotalDepth.util.CmnCmdOpts}}
\item {\sphinxstyleindexentry{TotalDepth.util.DictTree}}\sphinxstyleindexpageref{ref/util/DictTree:\detokenize{module-TotalDepth.util.DictTree}}
\item {\sphinxstyleindexentry{TotalDepth.util.DirWalk}}\sphinxstyleindexpageref{ref/util/DirWalk:\detokenize{module-TotalDepth.util.DirWalk}}
\item {\sphinxstyleindexentry{TotalDepth.util.ExecTimer}}\sphinxstyleindexpageref{ref/util/ExecTimer:\detokenize{module-TotalDepth.util.ExecTimer}}
\item {\sphinxstyleindexentry{TotalDepth.util.FileBuffer}}\sphinxstyleindexpageref{ref/util/FileBuffer:\detokenize{module-TotalDepth.util.FileBuffer}}
\item {\sphinxstyleindexentry{TotalDepth.util.Histogram}}\sphinxstyleindexpageref{ref/util/Histogram:\detokenize{module-TotalDepth.util.Histogram}}
\item {\sphinxstyleindexentry{TotalDepth.util.HtmlUtils}}\sphinxstyleindexpageref{ref/util/HtmlUtils:\detokenize{module-TotalDepth.util.HtmlUtils}}
\item {\sphinxstyleindexentry{TotalDepth.util.plot.AREACfg}}\sphinxstyleindexpageref{ref/util/plot/AREACfg:\detokenize{module-TotalDepth.util.plot.AREACfg}}
\item {\sphinxstyleindexentry{TotalDepth.util.plot.Coord}}\sphinxstyleindexpageref{ref/util/plot/Coord:\detokenize{module-TotalDepth.util.plot.Coord}}
\item {\sphinxstyleindexentry{TotalDepth.util.plot.FILMCfg}}\sphinxstyleindexpageref{ref/util/plot/FILMCfg:\detokenize{module-TotalDepth.util.plot.FILMCfg}}
\item {\sphinxstyleindexentry{TotalDepth.util.plot.FILMCfgXML}}\sphinxstyleindexpageref{ref/util/plot/FILMCfgXML:\detokenize{module-TotalDepth.util.plot.FILMCfgXML}}
\item {\sphinxstyleindexentry{TotalDepth.util.plot.LogHeader}}\sphinxstyleindexpageref{ref/util/plot/LogHeader:\detokenize{module-TotalDepth.util.plot.LogHeader}}
\item {\sphinxstyleindexentry{TotalDepth.util.plot.Plot}}\sphinxstyleindexpageref{ref/util/plot/Plot:\detokenize{module-TotalDepth.util.plot.Plot}}
\item {\sphinxstyleindexentry{TotalDepth.util.plot.PlotConstants}}\sphinxstyleindexpageref{ref/util/plot/PlotConstants:\detokenize{module-TotalDepth.util.plot.PlotConstants}}
\item {\sphinxstyleindexentry{TotalDepth.util.plot.PRESCfg}}\sphinxstyleindexpageref{ref/util/plot/PRESCfg:\detokenize{module-TotalDepth.util.plot.PRESCfg}}
\item {\sphinxstyleindexentry{TotalDepth.util.plot.PRESCfgXML}}\sphinxstyleindexpageref{ref/util/plot/PRESCfgXML:\detokenize{module-TotalDepth.util.plot.PRESCfgXML}}
\item {\sphinxstyleindexentry{TotalDepth.util.plot.Stroke}}\sphinxstyleindexpageref{ref/util/plot/Stroke:\detokenize{module-TotalDepth.util.plot.Stroke}}
\item {\sphinxstyleindexentry{TotalDepth.util.plot.SVGWriter}}\sphinxstyleindexpageref{ref/util/plot/SVGWriter:\detokenize{module-TotalDepth.util.plot.SVGWriter}}
\item {\sphinxstyleindexentry{TotalDepth.util.plot.Track}}\sphinxstyleindexpageref{ref/util/plot/Track:\detokenize{module-TotalDepth.util.plot.Track}}
\item {\sphinxstyleindexentry{TotalDepth.util.plot.XGrid}}\sphinxstyleindexpageref{ref/util/plot/XGrid:\detokenize{module-TotalDepth.util.plot.XGrid}}
\item {\sphinxstyleindexentry{TotalDepth.util.plot.XMLCfg}}\sphinxstyleindexpageref{ref/util/plot/XMLCfg:\detokenize{module-TotalDepth.util.plot.XMLCfg}}
\item {\sphinxstyleindexentry{TotalDepth.util.plot.XMLMatches}}\sphinxstyleindexpageref{ref/util/plot/XMLMatches:\detokenize{module-TotalDepth.util.plot.XMLMatches}}
\item {\sphinxstyleindexentry{TotalDepth.util.XmlWrite}}\sphinxstyleindexpageref{ref/util/XmlWrite:\detokenize{module-TotalDepth.util.XmlWrite}}
\end{sphinxtheindex}

\renewcommand{\indexname}{Index}
\printindex
\end{document}